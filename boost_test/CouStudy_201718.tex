%% ================================================================================
%% This LaTeX file was created by AbiWord.                                         
%% AbiWord is a free, Open Source word processor.                                  
%% More information about AbiWord is available at http://www.abisource.com/        
%% ================================================================================

\documentclass[a4paper,portrait,12pt]{article}
\usepackage[latin1]{inputenc}
\usepackage{calc}
\usepackage{setspace}
\usepackage{fixltx2e}
\usepackage{graphicx}
\usepackage{multicol}
\usepackage[normalem]{ulem}
%% Please revise the following command, if your babel
%% package does not support en-US
\usepackage[en]{babel}
\usepackage{color}
\usepackage{hyperref}
 
\begin{document}


\begin{flushleft}
	 COURSES OF STUDY
\end{flushleft}


	2017-2018





	





\begin{flushleft}
Undergraduate programmes
\end{flushleft}





	


	





\begin{flushleft}
Bachelor of Technology
\end{flushleft}


\begin{flushleft}
Dual Degree (B.Tech + M.Tech.)
\end{flushleft}





	





\begin{flushleft}
Postgraduate programmes
\end{flushleft}





	


	


	


	


	


	


	





\begin{flushleft}
Diploma of I.I.T. Delhi
\end{flushleft}


\begin{flushleft}
Master of Science
\end{flushleft}


\begin{flushleft}
Master of Business Administration
\end{flushleft}


\begin{flushleft}
Master of Design
\end{flushleft}


\begin{flushleft}
Master of Technology
\end{flushleft}


\begin{flushleft}
Master of Science (Research)
\end{flushleft}


\begin{flushleft}
Doctor of Philosophy
\end{flushleft}





\begin{flushleft}
INDIAN INSTITUTE OF TECHNOLOGY DELHI
\end{flushleft}





	


\begin{flushleft}
Hauz Khas, New Delhi 110 016, India.
\end{flushleft}


\begin{flushleft}
	http://www.iitd.ac.in
\end{flushleft}





\begin{flushleft}
\newpage
This book is available at the IIT Delhi website:
\end{flushleft}





\begin{flushleft}
http://www.iitd.ac.in
\end{flushleft}


\begin{flushleft}
Link:http://www.iitd.ac.in/content/curriculum-info
\end{flushleft}





\begin{flushleft}
In case of queries, please visit IIT Delhi website or contact:
\end{flushleft}


\begin{flushleft}
Undergraduate Programme	
\end{flushleft}


\begin{flushleft}
Postgraduate Programme
\end{flushleft}


\begin{flushleft}
Assistant Registrar 	
\end{flushleft}


\begin{flushleft}
Deputy Registrar
\end{flushleft}


\begin{flushleft}
Ph.	
\end{flushleft}


:	 +91 11 2659 1718	


\begin{flushleft}
Ph.	
\end{flushleft}


:	 +91 11 2659 1737


\begin{flushleft}
Fax	
\end{flushleft}


:	 +91 11 2659 7114	


\begin{flushleft}
Fax	
\end{flushleft}


:	 +91 11 2658 2032


\begin{flushleft}
E-mail	 :	arugs@iitd.ac.in	
\end{flushleft}


\begin{flushleft}
E-mail	 :	 drpgsr@iitd.ac.in
\end{flushleft}


\begin{flushleft}
		adcur@admin.iitd.ac.in			 adres@admin.iitd.ac.in	
\end{flushleft}


\begin{flushleft}
		deanacad@admin.iitd.ac.in			 deanacad@admin.iitd.ac.in
\end{flushleft}





\begin{flushleft}
Copyright : © IIT Delhi
\end{flushleft}





\begin{flushleft}
Produced by Publication Cell, IIT Delhi
\end{flushleft}





\begin{flushleft}
July 2017
\end{flushleft}





\begin{flushleft}
(ii)
\end{flushleft}





\begin{flushleft}
\newpage
CONTENTS
\end{flushleft}


1.	





\begin{flushleft}
INTRODUCTION...................................................................................................................1
\end{flushleft}





\begin{flushleft}
	1.1	Background..........................................................................................................................1
\end{flushleft}


	


1.2	


\begin{flushleft}
Departments, Centres and Schools.........................................................................................1
\end{flushleft}


	


1.3	


\begin{flushleft}
Programmes Offered.............................................................................................................2
\end{flushleft}


	


1.4	


\begin{flushleft}
Entry Number.......................................................................................................................4
\end{flushleft}


	


1.5	


\begin{flushleft}
Honour Code........................................................................................................................5
\end{flushleft}





2.	





\begin{flushleft}
COURSE STRUCTURE AND CREDIT SYSTEM...........................................................................5
\end{flushleft}





	


2.1	


\begin{flushleft}
Course Numbering Scheme....................................................................................................5
\end{flushleft}


	


2.2	


\begin{flushleft}
Credit System.......................................................................................................................6
\end{flushleft}


	


2.3	


\begin{flushleft}
Assignment of Credits to Courses...........................................................................................6
\end{flushleft}


	


2.4	


\begin{flushleft}
Earning Credits.....................................................................................................................7
\end{flushleft}


	


2.5	


\begin{flushleft}
Description of Course Content...............................................................................................7
\end{flushleft}


\begin{flushleft}
	2.6	Pre-requisites.......................................................................................................................8
\end{flushleft}


	


2.7	


\begin{flushleft}
Overlapping / Equivalent Courses...........................................................................................8
\end{flushleft}


	


2.8	


\begin{flushleft}
Course Coordinator...............................................................................................................8
\end{flushleft}


	


2.9	


\begin{flushleft}
Grading System....................................................................................................................8
\end{flushleft}


	


2.10	


\begin{flushleft}
Evaluation of Performance...................................................................................................11
\end{flushleft}





3.	





\begin{flushleft}
REGISTRATION AND ATTENDANCE...................................................................................... 12
\end{flushleft}





\begin{flushleft}
	3.1	Registration........................................................................................................................12
\end{flushleft}


	


3.2	


\begin{flushleft}
Registration and Student Status...........................................................................................13
\end{flushleft}


	


3.3	


\begin{flushleft}
Advice on Courses..............................................................................................................13
\end{flushleft}


	


3.4	


\begin{flushleft}
Validation of Registration.....................................................................................................13
\end{flushleft}


	


3.5	


\begin{flushleft}
Minimum Student Registration in a Course............................................................................13
\end{flushleft}


	


3.6	


\begin{flushleft}
Late Registration.................................................................................................................13
\end{flushleft}


	


3.7	


\begin{flushleft}
Add / Drop, Audit and Withdrawal of courses........................................................................13
\end{flushleft}


	


3.8	


\begin{flushleft}
Semester Withdrawal..........................................................................................................14
\end{flushleft}


	


3.9	


\begin{flushleft}
Registration in Special Module Courses.................................................................................14
\end{flushleft}


	


3.10	


\begin{flushleft}
Registration for Non-graded Units........................................................................................14
\end{flushleft}


	


3.11	


\begin{flushleft}
Pre-requisite Requirement(s) for Registration........................................................................14
\end{flushleft}


	


3.12	


\begin{flushleft}
Overlapping / Equivalent Courses.........................................................................................14
\end{flushleft}


	


3.13	


\begin{flushleft}
Limits on Registration..........................................................................................................15
\end{flushleft}


	


3.14	


\begin{flushleft}
Registration and Fee Payment..............................................................................................15
\end{flushleft}


	


3.15	


\begin{flushleft}
Continuous Absence and Registration Status.........................................................................15
\end{flushleft}


	


3.16	


\begin{flushleft}
Attendance Rule.................................................................................................................15
\end{flushleft}





4.	





\begin{flushleft}
UNDERGRADUATE DEGREE REQUIREMENTS, REGULATIONS AND PROCEDURES....................16
\end{flushleft}





	


	


	


	


	


	


	


	


	


	


	


	


	





4.1	


4.2	


4.3	


4.4	


4.5	


4.6	


4.7	


4.8	


4.9	


4.10	


4.11	


4.12	


4.13	





\begin{flushleft}
Overall Requirements..........................................................................................................16
\end{flushleft}


\begin{flushleft}
Breakup of Degree Requirements ........................................................................................16
\end{flushleft}


\begin{flushleft}
Non-graded Core Requirement ............................................................................................18
\end{flushleft}


\begin{flushleft}
Minimum and Maximum Durations for Completing Degree Requirements................................20
\end{flushleft}


\begin{flushleft}
Absence During the Semester..............................................................................................20
\end{flushleft}


\begin{flushleft}
Conditions for Termination of Registration, Probation and Warning.........................................21
\end{flushleft}


\begin{flushleft}
Scheme for Academic Advising of Undergraduate Students....................................................22
\end{flushleft}


\begin{flushleft}
Capability Linked Opportunities for Undergraduate Students..................................................24
\end{flushleft}


\begin{flushleft}
Change of Programme at the End of First Year......................................................................25
\end{flushleft}


\begin{flushleft}
Self-study Course................................................................................................................26
\end{flushleft}


\begin{flushleft}
Assistantship for Dual-degree Programmes...........................................................................26
\end{flushleft}


\begin{flushleft}
Admission of UG Students to PG Programmes.......................................................................26
\end{flushleft}


\begin{flushleft}
Measures for Helping SC / ST Students..................................................................................26
\end{flushleft}


\begin{flushleft}
(iii)
\end{flushleft}





\newpage
5.	





\begin{flushleft}
POSTGRADUATE DEGREE REQUIREMENTS, REGULATIONS and PROCEDURES........................27
\end{flushleft}





	


	


	


	


	


	


	


	


	


	


	


	


	





5.1	


5.2	


5.3	


5.4	


5.5	


5.6	


5.7	


5.8	


5.9	


5.10	


5.11	


5.12	


5.13	





6.	





\begin{flushleft}
UNDERGRADUATE PROGRAMME STRUCTURES.....................................................................36
\end{flushleft}





7.	





\begin{flushleft}
CAPABILITY-LINKED OPTIONS FOR UNDERGRADUATE STUDENTS.........................................70
\end{flushleft}





8.	





\begin{flushleft}
NON-GRADED CORE FOR UNDERGRADUATE STUDENTS.......................................................78
\end{flushleft}





9.	





\begin{flushleft}
POSTGRADUATE PROGRAMME STRUCTURES........................................................................92
\end{flushleft}





\begin{flushleft}
Degree Requirements..........................................................................................................27
\end{flushleft}


\begin{flushleft}
Continuation Requirements..................................................................................................27
\end{flushleft}


\begin{flushleft}
Minimum Student Registration for a Programme....................................................................27
\end{flushleft}


\begin{flushleft}
Lower and Upper Limits for Credits Registered......................................................................27
\end{flushleft}


\begin{flushleft}
Audit Courses for PG Students.............................................................................................27
\end{flushleft}


\begin{flushleft}
Award of D.I.I.T. to M.Tech. / M.B.A. Students.......................................................................28
\end{flushleft}


\begin{flushleft}
Regulations for Part-time Students.......................................................................................28
\end{flushleft}


\begin{flushleft}
Leave Rules for D.I.I.T., M.Des., M.Tech. and M.S. (Research)................................................28
\end{flushleft}


\begin{flushleft}
Assistantship Requirements.................................................................................................28
\end{flushleft}


\begin{flushleft}
Summer Registration...........................................................................................................28
\end{flushleft}


\begin{flushleft}
Master of Science (Research) Regulations.............................................................................29
\end{flushleft}


\begin{flushleft}
Migration from one PG Programme to another PG Programme of the Institute.........................29
\end{flushleft}


\begin{flushleft}
Doctor of Philosophy (Ph.D.) Regulations..............................................................................29
\end{flushleft}





\begin{flushleft}
10.	 COURSE DESCRIPTIONS................................................................................................... 143
\end{flushleft}


	


	


	


	


	


	


	


	


	


	


	


	


	


	


	


	


	


	


	


	


	


	


	


	


	


	





\begin{flushleft}
Department of Applied Mechanics.................................................................................................... 144
\end{flushleft}


\begin{flushleft}
Department of Biochemical Engineering and Biotechnology................................................................ 149
\end{flushleft}


\begin{flushleft}
Department of Chemical Engineering................................................................................................ 154
\end{flushleft}


\begin{flushleft}
Department of Chemistry................................................................................................................ 162
\end{flushleft}


\begin{flushleft}
Department of Civil Engineering ...................................................................................................... 166
\end{flushleft}


\begin{flushleft}
Department of Computer Science and Engineering ........................................................................... 184
\end{flushleft}


\begin{flushleft}
Department of Electrical Engineering................................................................................................ 193
\end{flushleft}


\begin{flushleft}
Department of Humanities and Social Sciences................................................................................. 213
\end{flushleft}


\begin{flushleft}
Department of Management Studies................................................................................................ 231
\end{flushleft}


\begin{flushleft}
Department of Mathematics............................................................................................................. 246
\end{flushleft}


\begin{flushleft}
Department of Mechanical Engineering............................................................................................. 256
\end{flushleft}


\begin{flushleft}
Department of Physics.................................................................................................................... 271
\end{flushleft}


\begin{flushleft}
Department of Textile Technology.................................................................................................... 285
\end{flushleft}


\begin{flushleft}
Centre for Applied Research in Electronics........................................................................................ 294
\end{flushleft}


\begin{flushleft}
Centre for Atmospheric Sciences...................................................................................................... 297
\end{flushleft}


\begin{flushleft}
Centre for Biomedical Engineering.................................................................................................... 302
\end{flushleft}


\begin{flushleft}
Centre for Energy Studies................................................................................................................ 305
\end{flushleft}


\begin{flushleft}
Centre for Industrial Tribology, Machine Dynamics and Maintenance Engineering ................................ 310
\end{flushleft}


\begin{flushleft}
Centre for Instrument Design and Development ............................................................................... 312
\end{flushleft}


\begin{flushleft}
Centre for Polymer Science and Technology...................................................................................... 316
\end{flushleft}


\begin{flushleft}
Centre for Rural Development and Technology.................................................................................. 318
\end{flushleft}


\begin{flushleft}
National Resource Centre for Value Education in Engineering............................................................. 320
\end{flushleft}


\begin{flushleft}
Amar Nath and Shashi Khosla School of Information Technology........................................................ 321
\end{flushleft}


\begin{flushleft}
Bharti School of Telecommunication Technology and Management..................................................... 321
\end{flushleft}


\begin{flushleft}
Kusuma School of Biological Sciences............................................................................................... 322
\end{flushleft}


\begin{flushleft}
Interdisciplinary M.Tech. Programmes.............................................................................................. 326
\end{flushleft}





\begin{flushleft}
ABBREVIATIONS....................................................................................................................... 327
\end{flushleft}


\begin{flushleft}
SLOT TIMINGS ......................................................................................................................... 328
\end{flushleft}


\begin{flushleft}
(iv)
\end{flushleft}





\newpage
1.	





\begin{flushleft}
INTRODUCTION
\end{flushleft}





\begin{flushleft}
1.1	Background
\end{flushleft}


\begin{flushleft}
I.I.T. Delhi provides science-based engineering education with a view to produce quality engineer-scientists. The
\end{flushleft}


\begin{flushleft}
curriculum provides broad based knowledge and simultaneously builds a temper for the life long process of learning
\end{flushleft}


\begin{flushleft}
and exploring. At the undergraduate level, a student needs to do compulsory foundation courses in the areas of
\end{flushleft}


\begin{flushleft}
basic sciences, humanities, social sciences and engineering sciences apart from departmental requirements in the
\end{flushleft}


\begin{flushleft}
core engineering discipline. Departmental courses (core and electives) constitute about half of the total curriculum.
\end{flushleft}


\begin{flushleft}
Further, students do open category electives to develop broad inter-disciplinary knowledge base or to specialize
\end{flushleft}


\begin{flushleft}
significantly in an area outside the parent discipline. Activities that enhance the quality of learning, but are not part
\end{flushleft}


\begin{flushleft}
of the foregoing, have been included in the undergraduate curriculum as non-graded core. At the postgraduate
\end{flushleft}


\begin{flushleft}
level, students are encouraged to look beyond their area of specialization to broaden their horizons through open
\end{flushleft}


\begin{flushleft}
electives and self-learning.
\end{flushleft}


\begin{flushleft}
The medium of instruction in the Institute is English.
\end{flushleft}


\begin{flushleft}
The Institute follows a semester system. An academic year runs from July through June next year and is essentially
\end{flushleft}


\begin{flushleft}
comprised of two semesters. Typically, the 1st semester starts in the last week of July and ends in the 1st week
\end{flushleft}


\begin{flushleft}
of December; the 2nd semester starts in the 1st week of January and ends in the 2nd week of May. Additionally,
\end{flushleft}


\begin{flushleft}
the summer semester which starts in the 3rd week of May and ends in the 2nd week of July, is utilized in some
\end{flushleft}


\begin{flushleft}
exceptional cases. Detailed schedule is given in the Semester Schedule that is made available before the start of
\end{flushleft}


\begin{flushleft}
each semester.
\end{flushleft}





\begin{flushleft}
1.2	 Departments, Centres and Schools
\end{flushleft}


\begin{flushleft}
Each course is offered by an Academic Unit which could be a Department, a Centre or a School. The names of
\end{flushleft}


\begin{flushleft}
Departments, Centres and Schools and their two-letter codes are given in Table 1. Some programmes are offered
\end{flushleft}


\begin{flushleft}
jointly by multiple academic units and are classified as interdisciplinary programmes; their codes are given in Table 2.
\end{flushleft}


\begin{flushleft}
Table 1. Academic Departments, Centres and Schools
\end{flushleft}


\begin{flushleft}
Name of Academic Unit (alphabetical order)
\end{flushleft}


\begin{flushleft}
Applied Mechanics, Department of
\end{flushleft}


\begin{flushleft}
Applied Research in Electronics, Centre for
\end{flushleft}


\begin{flushleft}
Atmospheric Sciences, Centre for
\end{flushleft}


\begin{flushleft}
Biochemical Engineering and Biotechnology, Department of
\end{flushleft}


\begin{flushleft}
Biological Sciences, Kusuma School of
\end{flushleft}


\begin{flushleft}
Biomedical Engineering, Centre for
\end{flushleft}


\begin{flushleft}
Chemical Engineering, Department of
\end{flushleft}


\begin{flushleft}
Chemistry, Department of
\end{flushleft}


\begin{flushleft}
Civil Engineering, Department of
\end{flushleft}


\begin{flushleft}
Computer Science and Engineering, Department of
\end{flushleft}


\begin{flushleft}
Electrical Engineering, Department of
\end{flushleft}


\begin{flushleft}
Energy Studies, Centre for
\end{flushleft}


\begin{flushleft}
Humanities and Social Sciences, Department of
\end{flushleft}


\begin{flushleft}
Industrial Tribology, Machine Dynamics and Maintenance Engineering Centre
\end{flushleft}


\begin{flushleft}
Information Technology, Amar Nath and Shashi Khosla School of
\end{flushleft}


\begin{flushleft}
Instrument Design and Development Centre
\end{flushleft}


\begin{flushleft}
Management Studies, Department of
\end{flushleft}


\begin{flushleft}
Mathematics, Department of
\end{flushleft}


\begin{flushleft}
Mechanical Engineering, Department of
\end{flushleft}


\begin{flushleft}
Physics, Department of
\end{flushleft}


\begin{flushleft}
Polymer Science and Technology, Centre for
\end{flushleft}


\begin{flushleft}
Rural Development and Technology, Centre for
\end{flushleft}


\begin{flushleft}
Telecommunication Technology and Management, Bharti School of
\end{flushleft}


\begin{flushleft}
Textile Technology, Department of
\end{flushleft}


\begin{flushleft}
Value Education in Engineering, National Resource Centre for
\end{flushleft}


1





\begin{flushleft}
Code of
\end{flushleft}


\begin{flushleft}
Academic Unit
\end{flushleft}





\begin{flushleft}
AM
\end{flushleft}


\begin{flushleft}
CR
\end{flushleft}


\begin{flushleft}
AS
\end{flushleft}


\begin{flushleft}
BE
\end{flushleft}


\begin{flushleft}
BL
\end{flushleft}


\begin{flushleft}
BM
\end{flushleft}


\begin{flushleft}
CH
\end{flushleft}


\begin{flushleft}
CY
\end{flushleft}


\begin{flushleft}
CE
\end{flushleft}


\begin{flushleft}
CS
\end{flushleft}


\begin{flushleft}
EE
\end{flushleft}


\begin{flushleft}
ES
\end{flushleft}


\begin{flushleft}
HU
\end{flushleft}


\begin{flushleft}
IT
\end{flushleft}


\begin{flushleft}
AN/SI
\end{flushleft}


\begin{flushleft}
ID
\end{flushleft}


\begin{flushleft}
SM
\end{flushleft}


\begin{flushleft}
MA
\end{flushleft}


\begin{flushleft}
ME
\end{flushleft}


\begin{flushleft}
PH
\end{flushleft}


\begin{flushleft}
PT
\end{flushleft}


\begin{flushleft}
RD
\end{flushleft}


\begin{flushleft}
BS
\end{flushleft}


\begin{flushleft}
TT
\end{flushleft}


\begin{flushleft}
VE
\end{flushleft}





\begin{flushleft}
Course
\end{flushleft}


\begin{flushleft}
Prefix
\end{flushleft}





\begin{flushleft}
AP
\end{flushleft}


\begin{flushleft}
CR
\end{flushleft}


\begin{flushleft}
AS
\end{flushleft}


\begin{flushleft}
BB
\end{flushleft}


\begin{flushleft}
SB
\end{flushleft}


\begin{flushleft}
BM
\end{flushleft}


\begin{flushleft}
CL
\end{flushleft}


\begin{flushleft}
CM
\end{flushleft}


\begin{flushleft}
CV
\end{flushleft}


\begin{flushleft}
CO
\end{flushleft}


\begin{flushleft}
EL
\end{flushleft}


\begin{flushleft}
ES
\end{flushleft}


\begin{flushleft}
HU
\end{flushleft}


\begin{flushleft}
IT
\end{flushleft}


\begin{flushleft}
SI
\end{flushleft}


\begin{flushleft}
DS
\end{flushleft}


\begin{flushleft}
MS
\end{flushleft}


\begin{flushleft}
MT
\end{flushleft}


\begin{flushleft}
MC
\end{flushleft}


\begin{flushleft}
PY
\end{flushleft}


\begin{flushleft}
PT
\end{flushleft}


\begin{flushleft}
RD
\end{flushleft}


\begin{flushleft}
BS
\end{flushleft}


\begin{flushleft}
TX
\end{flushleft}


\begin{flushleft}
VE
\end{flushleft}





\begin{flushleft}
\newpage
Courses of Study 2017-2018
\end{flushleft}





\begin{flushleft}
1.3	 Programmes Offered
\end{flushleft}


\begin{flushleft}
IIT Delhi offers a variety of academic programmes for students with a wide range of backgrounds. Admission to
\end{flushleft}


\begin{flushleft}
many of these programmes are based on performance in national level tests / entrance examinations. Details are
\end{flushleft}


\begin{flushleft}
given in the Prospectus.
\end{flushleft}


\begin{flushleft}
The programmes offered by IIT Delhi are presently classified as undergraduate (UG) and postgraduate (PG)
\end{flushleft}


\begin{flushleft}
programmes. This classification is based primarily on entry/admission qualification of students rather than the level
\end{flushleft}


\begin{flushleft}
of degree offered. For all undergraduate programmes, students are admitted after 10+2 years of schooling while
\end{flushleft}


\begin{flushleft}
for all postgraduate programmes, students are admitted after they have obtained at least a college level Bachelor's
\end{flushleft}


\begin{flushleft}
degree. Various programmes offered and their specializations are listed below.
\end{flushleft}


\begin{flushleft}
A.	
\end{flushleft}





\begin{flushleft}
Bachelor of Technology: (B.Tech.)
\end{flushleft}


\begin{flushleft}
Department
\end{flushleft}





\begin{flushleft}
Programme
\end{flushleft}





\begin{flushleft}
Biochemical Engg. and Biotechnology
\end{flushleft}





\begin{flushleft}
B.Tech. in Biochemical Engineering and Biotechnology
\end{flushleft}





\begin{flushleft}
Chemical Engineering
\end{flushleft}





\begin{flushleft}
B.Tech. in Chemical Engineering
\end{flushleft}





\begin{flushleft}
Computer Science and Engineering
\end{flushleft}





\begin{flushleft}
B.Tech. in Computer Science and Engineering
\end{flushleft}





\begin{flushleft}
Civil Engineering
\end{flushleft}





\begin{flushleft}
B.Tech. in Civil Engineering
\end{flushleft}





\begin{flushleft}
Electrical Engineering
\end{flushleft}


\begin{flushleft}
Mathematics
\end{flushleft}





\begin{flushleft}
B.Tech. in Electrical Engineering
\end{flushleft}


\begin{flushleft}
B.Tech. in Electrical Engineering (Power and Automation)
\end{flushleft}


\begin{flushleft}
B. Tech. in Mathematics \& Computing
\end{flushleft}





\begin{flushleft}
Mechanical Engineering
\end{flushleft}





\begin{flushleft}
B.Tech. in Mechanical Engineering
\end{flushleft}


\begin{flushleft}
B.Tech. in Production and Industrial Engineering
\end{flushleft}





\begin{flushleft}
Physics
\end{flushleft}





\begin{flushleft}
B.Tech. in Engineering Physics
\end{flushleft}





\begin{flushleft}
Textile Technology
\end{flushleft}





\begin{flushleft}
B.Tech. in Textile Engineering
\end{flushleft}





\begin{flushleft}
B.	
\end{flushleft}





\begin{flushleft}
BB1
\end{flushleft}


\begin{flushleft}
CH1
\end{flushleft}


\begin{flushleft}
CS1
\end{flushleft}


\begin{flushleft}
CE1
\end{flushleft}


\begin{flushleft}
EE1
\end{flushleft}


\begin{flushleft}
EE3
\end{flushleft}


\begin{flushleft}
MT1
\end{flushleft}


\begin{flushleft}
ME1
\end{flushleft}


\begin{flushleft}
ME2
\end{flushleft}


\begin{flushleft}
PH1
\end{flushleft}


\begin{flushleft}
TT1
\end{flushleft}





\begin{flushleft}
Dual-Degree : (B.Tech. and M.Tech.)
\end{flushleft}


\begin{flushleft}
Department
\end{flushleft}





\begin{flushleft}
Programme
\end{flushleft}





\begin{flushleft}
Biochemical Engg. \& Biotechnology
\end{flushleft}





\begin{flushleft}
B.Tech. and M.Tech. in Biochemical Engineering and
\end{flushleft}


\begin{flushleft}
Biotechnology
\end{flushleft}





\begin{flushleft}
Chemical Engineering
\end{flushleft}





\begin{flushleft}
B.Tech. and M.Tech. in Chemical Engineering
\end{flushleft}





\begin{flushleft}
Computer Science and Engineering
\end{flushleft}





\begin{flushleft}
B.Tech. and M.Tech. in Computer Science and Engineering
\end{flushleft}





\begin{flushleft}
Mathematics
\end{flushleft}





\begin{flushleft}
B.Tech. and M.Tech. in Mathematics \& Computing
\end{flushleft}





\begin{flushleft}
C.	
\end{flushleft}





\begin{flushleft}
Code
\end{flushleft}





\begin{flushleft}
Code
\end{flushleft}





\begin{flushleft}
BB5
\end{flushleft}


\begin{flushleft}
CH7
\end{flushleft}


\begin{flushleft}
CS5
\end{flushleft}


\begin{flushleft}
MT6
\end{flushleft}





\begin{flushleft}
Master of Technology: (M.Tech.)
\end{flushleft}


\begin{flushleft}
Department/Centre
\end{flushleft}





\begin{flushleft}
Programme
\end{flushleft}





\begin{flushleft}
Applied Mechanics
\end{flushleft}





\begin{flushleft}
M.Tech. in Engineering Analysis and Design
\end{flushleft}





\begin{flushleft}
Chemical Engineering
\end{flushleft}





\begin{flushleft}
M.Tech. in Chemical Engineering
\end{flushleft}





\begin{flushleft}
Chemistry
\end{flushleft}





\begin{flushleft}
M.Tech. in Molecular Engineering : Chemical Synthesis \&
\end{flushleft}


\begin{flushleft}
Analysis
\end{flushleft}


\begin{flushleft}
M.Tech. in Geotechnical and Geoenvironmental Engineering
\end{flushleft}


\begin{flushleft}
M.Tech. in Rock Engineering and Underground Structures
\end{flushleft}


\begin{flushleft}
M.Tech. in Structure Engineering
\end{flushleft}





\begin{flushleft}
Civil Engineering
\end{flushleft}





\begin{flushleft}
M.Tech. in Water Resources Engineering
\end{flushleft}


\begin{flushleft}
M.Tech. in Construction Engineering and Management
\end{flushleft}


\begin{flushleft}
M.Tech. in Construction Technology and Management (*)
\end{flushleft}


\begin{flushleft}
M.Tech. in Environmental Engineering and Management
\end{flushleft}


\begin{flushleft}
M.Tech. in Transportation Engineering
\end{flushleft}





\begin{flushleft}
Computer Science \& Engineering
\end{flushleft}





\begin{flushleft}
M.Tech. in Computer Science and Engineering
\end{flushleft}


2





\begin{flushleft}
Code
\end{flushleft}





\begin{flushleft}
AMA
\end{flushleft}


\begin{flushleft}
CHE
\end{flushleft}


\begin{flushleft}
CYM
\end{flushleft}





\begin{flushleft}
CEG
\end{flushleft}


\begin{flushleft}
CEU
\end{flushleft}


\begin{flushleft}
CES
\end{flushleft}


\begin{flushleft}
CEW
\end{flushleft}


\begin{flushleft}
CET
\end{flushleft}


\begin{flushleft}
CEC
\end{flushleft}


\begin{flushleft}
CEV
\end{flushleft}


\begin{flushleft}
CEP
\end{flushleft}


\begin{flushleft}
MCS
\end{flushleft}





\begin{flushleft}
\newpage
Courses of Study 2017-2018
\end{flushleft}





\begin{flushleft}
M.Tech. in Communications Engineering
\end{flushleft}


\begin{flushleft}
M.Tech. in Computer Technology
\end{flushleft}


\begin{flushleft}
M.Tech. in Control and Automation
\end{flushleft}


\begin{flushleft}
Electrical Engineering
\end{flushleft}





\begin{flushleft}
M.Tech. in Integrated Electronics and Circuits
\end{flushleft}


\begin{flushleft}
M.Tech. in Power Electronics, Electrical Machines and
\end{flushleft}


\begin{flushleft}
Drives
\end{flushleft}


\begin{flushleft}
M.Tech. in Power Systems
\end{flushleft}


\begin{flushleft}
M.Tech. in Mechanical Design
\end{flushleft}





\begin{flushleft}
Mechanical Engineering
\end{flushleft}





\begin{flushleft}
M.Tech. in Industrial Engineering
\end{flushleft}


\begin{flushleft}
M.Tech. in Production Engineering
\end{flushleft}


\begin{flushleft}
M.Tech. in Thermal Engineering
\end{flushleft}





\begin{flushleft}
Physics
\end{flushleft}





\begin{flushleft}
M.Tech. in Applied Optics
\end{flushleft}


\begin{flushleft}
M.Tech. in Solid State Materials
\end{flushleft}


\begin{flushleft}
M.Tech. in Fibre Science \& Technology
\end{flushleft}





\begin{flushleft}
Textile Technology
\end{flushleft}





\begin{flushleft}
M.Tech. in Textile Engineering
\end{flushleft}


\begin{flushleft}
M.Tech. in Textile Chemical Processing
\end{flushleft}





\begin{flushleft}
Applied Research in Electronics
\end{flushleft}





\begin{flushleft}
M.Tech. in Radio Frequency Design and Technology
\end{flushleft}





\begin{flushleft}
Atmospheric Sciences	
\end{flushleft}





\begin{flushleft}
M.Tech. in Atmospheric-Oceanic Science and Technology
\end{flushleft}





\begin{flushleft}
Biomedical Engineering
\end{flushleft}





\begin{flushleft}
M.Tech. in Biomedical Engineering
\end{flushleft}


\begin{flushleft}
M.Tech. in Energy Studies
\end{flushleft}


\begin{flushleft}
M.Tech. in Industrial Tribology and Maintenance Engineering
\end{flushleft}


\begin{flushleft}
M.Tech. in Instrument Technology
\end{flushleft}





\begin{flushleft}
Interdisciplinary Programme
\end{flushleft}





\begin{flushleft}
M.Tech. in Optoelectronics and Optical Communication
\end{flushleft}


\begin{flushleft}
M.Tech. in Polymer Science and Technology
\end{flushleft}


\begin{flushleft}
M.Tech. in Telecommunication Technology Management
\end{flushleft}


\begin{flushleft}
M.Tech. in VLSI Design Tools and Technology (*)
\end{flushleft}





\begin{flushleft}
EEE
\end{flushleft}


\begin{flushleft}
EET
\end{flushleft}


\begin{flushleft}
EEA
\end{flushleft}


\begin{flushleft}
EEN
\end{flushleft}


\begin{flushleft}
EEP
\end{flushleft}


\begin{flushleft}
EES
\end{flushleft}


\begin{flushleft}
MEM
\end{flushleft}


\begin{flushleft}
MEE
\end{flushleft}


\begin{flushleft}
MEP
\end{flushleft}


\begin{flushleft}
MET
\end{flushleft}


\begin{flushleft}
PHA
\end{flushleft}


\begin{flushleft}
PHM
\end{flushleft}


\begin{flushleft}
TTF
\end{flushleft}


\begin{flushleft}
TTE
\end{flushleft}


\begin{flushleft}
TTC
\end{flushleft}


\begin{flushleft}
CRF
\end{flushleft}


\begin{flushleft}
AST
\end{flushleft}


\begin{flushleft}
BMT
\end{flushleft}


\begin{flushleft}
JES
\end{flushleft}


\begin{flushleft}
JIT
\end{flushleft}


\begin{flushleft}
JID
\end{flushleft}


\begin{flushleft}
JOP
\end{flushleft}


\begin{flushleft}
JPT
\end{flushleft}


\begin{flushleft}
JTM
\end{flushleft}


\begin{flushleft}
JVL
\end{flushleft}





\begin{flushleft}
NOTE: (*) These are sponsored programmes.
\end{flushleft}





	


\begin{flushleft}
D.	
\end{flushleft}





\begin{flushleft}
Master of Science (Research): M.S.(R)
\end{flushleft}


\begin{flushleft}
Department/Schools
\end{flushleft}





\begin{flushleft}
Applied Mechanics
\end{flushleft}


\begin{flushleft}
Bharti School of Telecommunication Technology and Management
\end{flushleft}


\begin{flushleft}
Biochemical Engineering and Biotechnology
\end{flushleft}


\begin{flushleft}
Chemical Engineering	
\end{flushleft}


\begin{flushleft}
Civil Engineering
\end{flushleft}


\begin{flushleft}
Computer Science and Engineering
\end{flushleft}


\begin{flushleft}
Electrical Engineering	
\end{flushleft}


\begin{flushleft}
Mechanical Engineering
\end{flushleft}


\begin{flushleft}
Amar Nath and Shashi Khosla School of Information Technology
\end{flushleft}


\begin{flushleft}
Kusuma School of Biological Sciences
\end{flushleft}


3





\begin{flushleft}
Code
\end{flushleft}





\begin{flushleft}
AMY
\end{flushleft}


\begin{flushleft}
BSY
\end{flushleft}


\begin{flushleft}
BEY
\end{flushleft}


\begin{flushleft}
CHY
\end{flushleft}


\begin{flushleft}
CEY
\end{flushleft}


\begin{flushleft}
CSY
\end{flushleft}


\begin{flushleft}
EEY
\end{flushleft}


\begin{flushleft}
MEY
\end{flushleft}


\begin{flushleft}
SIY
\end{flushleft}


\begin{flushleft}
BLY
\end{flushleft}





\begin{flushleft}
\newpage
Courses of Study 2017-2018
\end{flushleft}





\begin{flushleft}
E.	
\end{flushleft}





\begin{flushleft}
Master of Design: (M.Des.)
\end{flushleft}


\begin{flushleft}
Department
\end{flushleft}





\begin{flushleft}
Interdisciplinary
\end{flushleft}





\begin{flushleft}
F.	
\end{flushleft}





\begin{flushleft}
Programme
\end{flushleft}


\begin{flushleft}
Master of Design in Industrial Design
\end{flushleft}





\begin{flushleft}
Programme
\end{flushleft}





\begin{flushleft}
Management Studies
\end{flushleft}





\begin{flushleft}
Code
\end{flushleft}





\begin{flushleft}
SMG
\end{flushleft}





\begin{flushleft}
M.B.A.
\end{flushleft}


\begin{flushleft}
M.B.A. (with focus on Telecommunication Systems Management)
\end{flushleft}


\begin{flushleft}
M.B.A. (with focus on Technology Management) (part-time and evening
\end{flushleft}


\begin{flushleft}
programme)
\end{flushleft}





\begin{flushleft}
SMT
\end{flushleft}


\begin{flushleft}
SMN
\end{flushleft}





\begin{flushleft}
Master of Science: (M.Sc.)
\end{flushleft}


\begin{flushleft}
Department
\end{flushleft}





\begin{flushleft}
Programme
\end{flushleft}





\begin{flushleft}
Chemistry
\end{flushleft}





\begin{flushleft}
M.Sc. in Chemistry
\end{flushleft}





\begin{flushleft}
Mathematics
\end{flushleft}





\begin{flushleft}
M.Sc. in Mathematics
\end{flushleft}





\begin{flushleft}
Physics
\end{flushleft}





\begin{flushleft}
M.Sc. in Physics
\end{flushleft}





\begin{flushleft}
H.	
\end{flushleft}





\begin{flushleft}
JDS
\end{flushleft}





\begin{flushleft}
Master of Business Administration: (M.B.A.)
\end{flushleft}


\begin{flushleft}
Department
\end{flushleft}





\begin{flushleft}
G.	
\end{flushleft}





\begin{flushleft}
Code
\end{flushleft}





\begin{flushleft}
Code
\end{flushleft}





\begin{flushleft}
CYS
\end{flushleft}


\begin{flushleft}
MAS
\end{flushleft}


\begin{flushleft}
PHS
\end{flushleft}





\begin{flushleft}
Postgraduate Diploma
\end{flushleft}


\begin{flushleft}
Department
\end{flushleft}





\begin{flushleft}
Applied Mechanics
\end{flushleft}





\begin{flushleft}
Programme
\end{flushleft}


\begin{flushleft}
D.I.I.T (Naval Construction)
\end{flushleft}


\begin{flushleft}
(for candidates sponsored by the Indian Navy)
\end{flushleft}





\begin{flushleft}
Code
\end{flushleft}





\begin{flushleft}
AMX
\end{flushleft}





\begin{flushleft}
The DIIT is also awarded under special circumstances in every Master of Technology programme listed in item C
\end{flushleft}


\begin{flushleft}
above. It is awarded only to those students who have not been able to complete the requirements of the corresponding
\end{flushleft}


\begin{flushleft}
M.Tech. degree. For details please see Section 5.6.
\end{flushleft}


\begin{flushleft}
I.	
\end{flushleft}





\begin{flushleft}
Doctor of Philosophy: (Ph.D.)
\end{flushleft}





\begin{flushleft}
All departments, centres and schools listed in Section 1.2 offer the Ph.D. programme. The two letter code of the
\end{flushleft}


\begin{flushleft}
academic unit followed by Z corresponds to the Ph.D. code of the respective academic unit. (e.g. MAZ is the Ph.D.
\end{flushleft}


\begin{flushleft}
code of the Mathematics Department).
\end{flushleft}





\begin{flushleft}
1.4	 Entry Number
\end{flushleft}


\begin{flushleft}
The entry number of a student consists of eleven alpha-numerals, as described below:
\end{flushleft}





4





\begin{flushleft}
\newpage
Courses of Study 2017-2018
\end{flushleft}





\begin{flushleft}
2	 0	1	 5	 A	 B	 C	 6	 7	8	 9
\end{flushleft}





\}





\}





\}





11	


10	


9	8	7	6	5	4	3	2	1





\begin{flushleft}
Unique identification
\end{flushleft}


\begin{flushleft}
number for each student
\end{flushleft}





\begin{flushleft}
Entry Year
\end{flushleft}


\begin{flushleft}
(academic year
\end{flushleft}


\begin{flushleft}
of joining)
\end{flushleft}





\begin{flushleft}
Fields 7 \& 6	
\end{flushleft}


\begin{flushleft}
Field 5	
\end{flushleft}


		


		


		


		





:	


:	


:	


:	


:	


:	





\begin{flushleft}
Programme code
\end{flushleft}





\begin{flushleft}
Academic Unit Code
\end{flushleft}


\begin{flushleft}
1-4 for B.Tech.
\end{flushleft}


\begin{flushleft}
5-9 for Dual Degree
\end{flushleft}


\begin{flushleft}
A-X for M.Sc., D.I.I.T., M.B.A., M.Des., M.Tech.
\end{flushleft}


\begin{flushleft}
Y for M.S. (Research)
\end{flushleft}


\begin{flushleft}
Z for Ph.D.
\end{flushleft}





\begin{flushleft}
In case of a programme change of a student, the programme code in his/her entry number (fields 5, 6 and 7) will be
\end{flushleft}


\begin{flushleft}
changed. However, his/her unique identification number will remain unchanged. Such students will have two entry
\end{flushleft}


\begin{flushleft}
numbers, one prior to programme change and one after the change. At any time, though, only one entry number,
\end{flushleft}


\begin{flushleft}
that corresponds to the student's present status will be valid and active.
\end{flushleft}





\begin{flushleft}
1.5	 Honour Code
\end{flushleft}


\begin{flushleft}
The Honour Code of IIT Delhi is given at the end of this document. Every student signs this Honour Code at the time
\end{flushleft}


\begin{flushleft}
of admission and is expected to adhere to the Honour Code throughout the period of his/her studies at the Institute.
\end{flushleft}





2.	





\begin{flushleft}
COURSE STRUCTURE AND CREDIT SYSTEM
\end{flushleft}





\begin{flushleft}
2.1	 Course Numbering Scheme
\end{flushleft}


\begin{flushleft}
Normally every course at IIT Delhi runs for the full length of the semester. Only exception is for V-type courses which
\end{flushleft}


\begin{flushleft}
may run for part of the semester. A student registers in advance for courses that he/she wants to study and at the
\end{flushleft}


\begin{flushleft}
end of the semester a grade is awarded. On obtaining a pass grade, the student earns all the credits associated
\end{flushleft}


\begin{flushleft}
with the course while a fail grade does not get any credit. Partial credits are not awarded.
\end{flushleft}


\begin{flushleft}
Each course is denoted by a unique code consisting of three alphabets followed by three numerals:
\end{flushleft}





\}





\}





\begin{flushleft}
E	 L	 L	 1	0	0
\end{flushleft}





\begin{flushleft}
Unique identification
\end{flushleft}


\begin{flushleft}
number for each course
\end{flushleft}





\begin{flushleft}
Course prefix for the academic
\end{flushleft}


\begin{flushleft}
unit offering the course. See
\end{flushleft}


\begin{flushleft}
Section 1 for all prefixes.
\end{flushleft}


\begin{flushleft}
Nature of
\end{flushleft}


\begin{flushleft}
the course. Please
\end{flushleft}


\begin{flushleft}
see details in Table 2.
\end{flushleft}





\begin{flushleft}
Level of the course as determined by
\end{flushleft}


\begin{flushleft}
pre-requisite courses or number of earned
\end{flushleft}


\begin{flushleft}
credits.
\end{flushleft}


5





\begin{flushleft}
\newpage
Courses of Study 2017-2018
\end{flushleft}





\begin{flushleft}
(a)	
\end{flushleft}





\begin{flushleft}
Codes for the nature of the course
\end{flushleft}


\begin{flushleft}
Table 2 : Codes for the nature of courses.
\end{flushleft}


\begin{flushleft}
Code
\end{flushleft}





\begin{flushleft}
Description
\end{flushleft}





\begin{flushleft}
D
\end{flushleft}





\begin{flushleft}
Project based courses (e.g. Major, Minor, Mini Projects)
\end{flushleft}





\begin{flushleft}
L
\end{flushleft}





\begin{flushleft}
Lecture courses
\end{flushleft}


\begin{flushleft}
(other than lecture hours, these courses can have Tutorial and Practical hours, e.g. L-T-P
\end{flushleft}


\begin{flushleft}
structures 3-0-0, 3-1-2, 3-0-2, 2-0-0, etc.)
\end{flushleft}





\begin{flushleft}
N
\end{flushleft}





\begin{flushleft}
Non-graded core component
\end{flushleft}





\begin{flushleft}
P
\end{flushleft}





\begin{flushleft}
Practical / Practice based courses
\end{flushleft}


\begin{flushleft}
(where performance is evaluated primarily on the basis of practice, practical or laboratory
\end{flushleft}


\begin{flushleft}
work with LTP structures such as 0-0-3, 0-0-4, 1-0-3, 0-1-3, etc.)
\end{flushleft}





\begin{flushleft}
Q
\end{flushleft}





\begin{flushleft}
Seminar Courses
\end{flushleft}





\begin{flushleft}
R
\end{flushleft}





\begin{flushleft}
Professional Practices
\end{flushleft}





\begin{flushleft}
S
\end{flushleft}





\begin{flushleft}
Independent Study
\end{flushleft}





\begin{flushleft}
T
\end{flushleft}





\begin{flushleft}
Practical Training
\end{flushleft}





\begin{flushleft}
V
\end{flushleft}





\begin{flushleft}
Lecture Courses on Special Topics (1 or 2 credits)
\end{flushleft}





\begin{flushleft}
(b)	 Level of the course
\end{flushleft}


\begin{flushleft}
The first digit of the numeric part of the course code indicates level of the course as determined by pre-requisite
\end{flushleft}


\begin{flushleft}
course(s) and/or by the maturity required for registering for the course. The latter requirement is enforced through
\end{flushleft}


\begin{flushleft}
a requirement of minimum number of earned credits. In general,
\end{flushleft}


\begin{flushleft}
100 -- 400 level courses	 :	 Core and elective courses for UG programmes.
\end{flushleft}


\begin{flushleft}
		 These courses are not open to any PG student.
\end{flushleft}


\begin{flushleft}
500 level courses	
\end{flushleft}





\begin{flushleft}
:	 Courses for M.Sc. programmes.
\end{flushleft}





		





\begin{flushleft}
These courses are not open to other students.
\end{flushleft}





\begin{flushleft}
600 level courses	
\end{flushleft}





\begin{flushleft}
:	 Preparatory/introductory courses for M.Tech. and advanced courses for M.Sc.
\end{flushleft}


\begin{flushleft}
programmes. 500 and 600 level courses are normally not open to UG students.
\end{flushleft}





\begin{flushleft}
700 - 800 level courses	
\end{flushleft}





\begin{flushleft}
:	 Core and elective courses for M.Tech., M.Des., M.B.A., M.S.(Research) and Ph.D.
\end{flushleft}


\begin{flushleft}
programmes. Usually 800 level courses are advanced courses for PG students.
\end{flushleft}





\begin{flushleft}
2.2	Credit System
\end{flushleft}


\begin{flushleft}
Education at the Institute is organized around the semester-based credit system of study. A student is allowed to
\end{flushleft}


\begin{flushleft}
attend classes in a course and earn credit for it, only if he/she has registered for that course. Prominent features
\end{flushleft}


\begin{flushleft}
of the credit system are a process of continuous evaluation of a student's performance/progress and flexibility to
\end{flushleft}


\begin{flushleft}
allow a student to progress at an optimum pace suited to his/her ability or convenience, subject to fulfilling minimum
\end{flushleft}


\begin{flushleft}
requirements for continuation and within the maximum allowable period for completion of a degree.
\end{flushleft}


\begin{flushleft}
A student's performance/progress is measured by the number of credits that he/she has earned, i.e. completed
\end{flushleft}


\begin{flushleft}
satisfactorily. Based on the course credits and grades obtained by the student, grade point average is calculated. A
\end{flushleft}


\begin{flushleft}
minimum grade point average is required to be maintained for satisfactory progress and continuation in the programme.
\end{flushleft}


\begin{flushleft}
Also, a minimum number of earned credits and a minimum grade point average should be acquired in order to qualify
\end{flushleft}


\begin{flushleft}
for the degree. All programmes are defined by the total credit requirement and a pattern of credit distribution over
\end{flushleft}


\begin{flushleft}
courses of different categories as defined in sections 4 and 5 for UG and PG programmes respectively.
\end{flushleft}





\begin{flushleft}
2.3	 Assignment of Credits to Courses
\end{flushleft}


\begin{flushleft}
Each course has a certain number of credit(s) or non-graded unit(s) assigned to it depending upon its lecture, tutorial
\end{flushleft}


\begin{flushleft}
and laboratory/practical contact hours in a week. This weightage is also indicative of the academic expectation that
\end{flushleft}


\begin{flushleft}
includes in-class contact and self-study outside class hours.
\end{flushleft}


6





\begin{flushleft}
\newpage
Courses of Study 2017-2018
\end{flushleft}





\begin{flushleft}
Lectures and Tutorials	
\end{flushleft}





\begin{flushleft}
:	 One lecture or tutorial hour per week over the period of one 14 week semester is
\end{flushleft}


\begin{flushleft}
assigned one credit.
\end{flushleft}





\begin{flushleft}
Practical/Practice	
\end{flushleft}





\begin{flushleft}
:	 One laboratory / practice hour per week over the period of one 14 week semester is
\end{flushleft}


\begin{flushleft}
assigned half credit.
\end{flushleft}





\begin{flushleft}
A few courses are without credit and are counted under non-graded (NG) courses.
\end{flushleft}


\begin{flushleft}
Example : Course ELL100 Fundamentals of Electrical Engineering; 4 credits (3-0-2)
\end{flushleft}


\begin{flushleft}
The credits indicated for this course are computed as follows:
\end{flushleft}


\begin{flushleft}
3 hours/week lectures		 =	 3 credits
\end{flushleft}


\begin{flushleft}
0 hours/week tutorial		 =	 0 credit			
\end{flushleft}





\begin{flushleft}
Total = 3 + 0 + 1 = 4 credits
\end{flushleft}





\begin{flushleft}
2 hours/week practicals		 =	 2 × 0.5 = 1 credit
\end{flushleft}


\begin{flushleft}
Total contact hours for the course	 =	 (3 h Lectures + 0 h Tutorial + 2 h Practical) per week
\end{flushleft}


\begin{flushleft}
			=	5 contact hours per week for 14 weeks.
\end{flushleft}


\begin{flushleft}
For each lecture or tutorial credit, the self study component is 1-2 hours / week (for 100-600 level courses) and 3
\end{flushleft}


\begin{flushleft}
hours / week (for 700-800 level courses). The self study component for practicals is 1 hour for every two hours of
\end{flushleft}


\begin{flushleft}
practicals per week. In the above example, the student is expected to devote a minimum of 3 + 1 = 4 hours per
\end{flushleft}


\begin{flushleft}
week on self study in addition to class contact of 5 hours per week.
\end{flushleft}





\begin{flushleft}
2.4	 Earning Credits
\end{flushleft}


\begin{flushleft}
At the end of every semester, a letter grade is awarded in each course for which a student had registered. On
\end{flushleft}


\begin{flushleft}
obtaining a pass grade, the student accumulates the course credits as earned credits. An undergraduate student
\end{flushleft}


\begin{flushleft}
has the option of auditing some courses within the credit requirements for graduation. Grades obtained in audit
\end{flushleft}


\begin{flushleft}
courses are not counted for computation of grade point average. However, a pass grade is essential for earning
\end{flushleft}


\begin{flushleft}
credits from an audit course. Section 2.9 defines the letter grades awarded at IIT Delhi and specifies the minimum
\end{flushleft}


\begin{flushleft}
grade for passing a course.
\end{flushleft}





\begin{flushleft}
2.5	 Description of Course Content
\end{flushleft}


\begin{flushleft}
Course content description consists of following components: (i) Course Number, (ii) Title of the Course, (iii) Credit
\end{flushleft}


\begin{flushleft}
and L-T-P, (iv) Pre-requisites and overlapping courses, if any and (v) List of broad topics covered in the course.
\end{flushleft}


\begin{flushleft}
Content descriptions for all courses are given in section 10 of this document. An example course content description
\end{flushleft}


\begin{flushleft}
of a 100 level course is as follows:
\end{flushleft}





\begin{flushleft}
MTL100 Calculus
\end{flushleft}





\begin{flushleft}
4 Credits (3-1-0)
\end{flushleft}


\begin{flushleft}
Review of Limit, Continuity and Differentiability, uniform continuity, Mean Value
\end{flushleft}


\begin{flushleft}
Theorems and applications, Taylor's Theorem, maxima and minima, Sequences and
\end{flushleft}


\begin{flushleft}
series, limsup, liminf, convergence of sequences and series of real numbers, absolute
\end{flushleft}


\begin{flushleft}
and conditional convergence.
\end{flushleft}


\begin{flushleft}
Reimann Integral, fundamental theorem of integral calculus, applications of definite
\end{flushleft}


\begin{flushleft}
integrals, improper integrals, beta and gamma functions.
\end{flushleft}


\begin{flushleft}
Functions of several variables, limit and continuity, partial derivatives and differentiability,
\end{flushleft}


\begin{flushleft}
gradient, directional derivatives, chain rule, Taylor's theorem, maxima and minima and
\end{flushleft}


\begin{flushleft}
method of Lagrange Multipliers.
\end{flushleft}


\begin{flushleft}
Double and triple integration, Jacobian and change of variables formula. Parametrization
\end{flushleft}


\begin{flushleft}
of curves and surfaces, vector fields, divergence and curl, Line integrals, Green's
\end{flushleft}


\begin{flushleft}
theorem, surface integral, Gauss and Stokes theorems with applications.
\end{flushleft}





7





\begin{flushleft}
\newpage
Courses of Study 2017-2018
\end{flushleft}





\begin{flushleft}
2.6	 Pre-requisites
\end{flushleft}


\begin{flushleft}
Each course, other than 100 level courses, may have specified pre-requisite(s) which may be other course(s) or
\end{flushleft}


\begin{flushleft}
a minimum number of earned credits or both. A student who has not obtained a pass grade in the pre-requisite(s)
\end{flushleft}


\begin{flushleft}
specified or has not earned requisite number of credits will not be eligible to register for that course. Example:
\end{flushleft}


\begin{flushleft}
TXL372 Speciality Yarns and Fabrics
\end{flushleft}


\begin{flushleft}
2 credits (2-0-0)
\end{flushleft}


\begin{flushleft}
Pre-requisites: TXL221 / TXL222 and TXL231 / TXL232 and EC50
\end{flushleft}


\begin{flushleft}
A student who has obtained a pass grade in TXL221 or TXL222, and in TXL231 or TXL232 and has also earned
\end{flushleft}


\begin{flushleft}
50 credits will be eligible to register for this course.
\end{flushleft}


\begin{flushleft}
For UG students the pre-requisites for some courses of special nature are given below.
\end{flushleft}


	





\begin{flushleft}
Independent Study	
\end{flushleft}





\begin{flushleft}
65 earned credits
\end{flushleft}





	





\begin{flushleft}
Mini Project	
\end{flushleft}





\begin{flushleft}
65 earned credits
\end{flushleft}





	





\begin{flushleft}
Minor Project (Dual Degree)	
\end{flushleft}





\begin{flushleft}
100 earned credits
\end{flushleft}





	





\begin{flushleft}
B.Tech. Project Part - I	
\end{flushleft}





\begin{flushleft}
100 earned credits
\end{flushleft}





	





\begin{flushleft}
B.Tech. Project Part - II	
\end{flushleft}





\begin{flushleft}
B Grade in B.Tech. Project Part - I
\end{flushleft}





	





\begin{flushleft}
M.Tech. Major Project Part-I (Dual Degree)	
\end{flushleft}





\begin{flushleft}
135 earned credits
\end{flushleft}





\begin{flushleft}
In addition to any pre-requisite specified for 700 and 800 level courses, a UG student needs to earn 75 and 100
\end{flushleft}


\begin{flushleft}
credits to register for 700 and 800 level courses, respectively.
\end{flushleft}





\begin{flushleft}
2.7	 Overlapping/Equivalent Courses
\end{flushleft}


\begin{flushleft}
Wherever applicable, overlapping and equivalent courses have been identified for each course. A student is not
\end{flushleft}


\begin{flushleft}
permitted to earn credits by registering for more than one course in a set of overlapping / equivalent courses.
\end{flushleft}


\begin{flushleft}
Departments / Centres / Schools may use these overlapping/equivalent courses for meeting degree / pre-requisite
\end{flushleft}


\begin{flushleft}
requirements in special circumstances. For example:
\end{flushleft}


\begin{flushleft}
CLL113 Numerical Methods in Chemical Engineering
\end{flushleft}


\begin{flushleft}
4 credits (3-0-2)
\end{flushleft}


\begin{flushleft}
Overlaps with: MTL107, MTP290, MTL445, CVL734, COL726
\end{flushleft}


\begin{flushleft}
A student who has earned a pass grade in CLL113 will not be eligible to register for MTL107, MTP290,
\end{flushleft}


\begin{flushleft}
MTL445, CVL734 or COL726. An overlapping course cannot serve as a substitute for a core course of his / her
\end{flushleft}


\begin{flushleft}
programme. In the above example, if MTL107 is a core course for a student, he / she is not allowed to register
\end{flushleft}


\begin{flushleft}
for CLL113 as a substitute for this core course.
\end{flushleft}





\begin{flushleft}
2.8	 Course Coordinator
\end{flushleft}


\begin{flushleft}
Every course is usually coordinated by a member of the teaching staff of a Department / Centre / School in a given
\end{flushleft}


\begin{flushleft}
semester. This faculty member is designated as the Course Coordinator. He / she has the full responsibility for
\end{flushleft}


\begin{flushleft}
conducting the course, coordinating the work of other members of the faculty and teaching assistants involved
\end{flushleft}


\begin{flushleft}
in that course, administering assignments, conducting the tests as well as moderating and awarding the grades.
\end{flushleft}


\begin{flushleft}
For any difficulty related to a course, the student is expected to approach the respective course coordinator
\end{flushleft}


\begin{flushleft}
for advice and clarification. The distribution of the weightage for tests, quizzes, assignments, laboratory work,
\end{flushleft}


\begin{flushleft}
workshop and drawing assignment, term paper, etc. that will be the basis for award of grade in a course will be
\end{flushleft}


\begin{flushleft}
decided by the course coordinator of that course, in consultation with other teachers involved, and announced
\end{flushleft}


\begin{flushleft}
at the beginning of the semester.
\end{flushleft}





\begin{flushleft}
2.9	 Grading System
\end{flushleft}


\begin{flushleft}
The grade obtained in a course reflects a student's performance in the course. While relative standing of the
\end{flushleft}


\begin{flushleft}
student is indicated by his/her grades, the process of awarding grades is not necessarily based upon fitting the
\end{flushleft}


\begin{flushleft}
marks scored by the students to some statistical distribution. The course coordinator and associated faculty for
\end{flushleft}


\begin{flushleft}
a course formulate appropriate procedure to award grades that are reflective of the student's performance vis\`{a}-vis the expected learning outcomes of the course.
\end{flushleft}


8





\begin{flushleft}
\newpage
Courses of Study 2017-2018
\end{flushleft}





2.9.1	





\begin{flushleft}
Grade points
\end{flushleft}





\begin{flushleft}
The grades and their equivalent numerical points (referred to as Grade Points) are listed in Table 3.
\end{flushleft}


\begin{flushleft}
Table 3 : Grades and their description.
\end{flushleft}





\begin{flushleft}
Grade
\end{flushleft}





2.9.2	





\begin{flushleft}
Grade points
\end{flushleft}





\begin{flushleft}
Description
\end{flushleft}





\begin{flushleft}
A
\end{flushleft}





10





\begin{flushleft}
Outstanding
\end{flushleft}





\begin{flushleft}
A (-)
\end{flushleft}





9





\begin{flushleft}
Excellent
\end{flushleft}





\begin{flushleft}
B
\end{flushleft}





8





\begin{flushleft}
Very good
\end{flushleft}





\begin{flushleft}
B (-)
\end{flushleft}





7





\begin{flushleft}
Good
\end{flushleft}





\begin{flushleft}
C
\end{flushleft}





6





\begin{flushleft}
Average
\end{flushleft}





\begin{flushleft}
C (-)
\end{flushleft}





5





\begin{flushleft}
Below average
\end{flushleft}





\begin{flushleft}
D
\end{flushleft}





4





\begin{flushleft}
Marginal
\end{flushleft}





\begin{flushleft}
E
\end{flushleft}





2





\begin{flushleft}
Poor
\end{flushleft}





\begin{flushleft}
F
\end{flushleft}





0





\begin{flushleft}
Very poor
\end{flushleft}





\begin{flushleft}
I
\end{flushleft}





-





\begin{flushleft}
Incomplete
\end{flushleft}





\begin{flushleft}
NP
\end{flushleft}





-





\begin{flushleft}
Audit pass
\end{flushleft}





\begin{flushleft}
NF
\end{flushleft}





-





\begin{flushleft}
Audit fail
\end{flushleft}





\begin{flushleft}
W
\end{flushleft}





-





\begin{flushleft}
Withdrawal
\end{flushleft}





\begin{flushleft}
X
\end{flushleft}





-





\begin{flushleft}
Project / Ph.D. Continuation
\end{flushleft}





\begin{flushleft}
S
\end{flushleft}





-





\begin{flushleft}
Satisfactory completion
\end{flushleft}





\begin{flushleft}
Z
\end{flushleft}





-





\begin{flushleft}
Course continuation
\end{flushleft}





\begin{flushleft}
U
\end{flushleft}





-





\begin{flushleft}
Unsatisfactory progress in Ph.D.
\end{flushleft}





\begin{flushleft}
Description of grades
\end{flushleft}





\begin{flushleft}
A grade
\end{flushleft}





\begin{flushleft}
An {`}A' grade stands for outstanding achievement. The minimum marks for award of an {`}A' grade is 80 \%. However,
\end{flushleft}


\begin{flushleft}
individual course coordinators may set a higher marks requirement for awarding an {`}A' grade.
\end{flushleft}





\begin{flushleft}
C grade
\end{flushleft}





\begin{flushleft}
The {`}C' grade stands for average performance. This is the minimum grade required to pass in the Major Project
\end{flushleft}


\begin{flushleft}
Part 1 and Part 2 of Dual degree and 2 year M.Tech. and M.S.(R) Programmes.
\end{flushleft}





\begin{flushleft}
D grade
\end{flushleft}





\begin{flushleft}
The {`}D' grade stands for marginal performance; i.e. it is the minimum passing grade in any course excluding the
\end{flushleft}


\begin{flushleft}
M.Tech. Major Project. The minimum marks for award of {`}D' grade is 30 \%. However, individual course coordinators
\end{flushleft}


\begin{flushleft}
may set a higher marks requirement.
\end{flushleft}





\begin{flushleft}
E and F grades
\end{flushleft}





\begin{flushleft}
A student who has scored at least 20\% aggregate marks in a subject can be awarded an {`}E' Grade. The Course
\end{flushleft}


\begin{flushleft}
Coordinators are, however, free to enhance this limit but should keep the percentage about 10\% less than the
\end{flushleft}


\begin{flushleft}
cut-off marks for {`}D' Grade. The Course Coordinators can also specify any additional requirements (to be specified
\end{flushleft}


\begin{flushleft}
at the beginning of the Semester) for awarding {`}E' Grade. Students who obtain an {`}E' Grade will be eligible to
\end{flushleft}


\begin{flushleft}
appear in a repeat major test (re-major test), an examination with weightage same as that of Major test, for only
\end{flushleft}


\begin{flushleft}
lecture courses ({`}L' Category Courses described in section 2.1). If they perform satisfactorily, they become eligible
\end{flushleft}


\begin{flushleft}
for getting the grade converted to a {`}D' Grade, otherwise they will continue to have {`}E' Grade. The student will
\end{flushleft}


\begin{flushleft}
have only one chance to appear for re-major for an {`}E' Grade. The re-major test will be conducted within the first
\end{flushleft}


\begin{flushleft}
week of the next semester. The date of re-major test of Institute core courses for undergraduate students will be
\end{flushleft}


9





\begin{flushleft}
\newpage
Courses of Study 2017-2018
\end{flushleft}





\begin{flushleft}
centrally notified, while for all other courses, the date would be announced by the respective course coordinators.
\end{flushleft}


\begin{flushleft}
A student can appear for a maximum of three such re-major tests in a given semester. If a student can not appear
\end{flushleft}


\begin{flushleft}
for the re-major test due to any reason(s), he / she will not get any additional chance.
\end{flushleft}


\begin{flushleft}
If a student with E grade in a course does not pass the course through a re-major test, or obtains an {`}F' grade in
\end{flushleft}


\begin{flushleft}
the course, he / she has to repeat the course if it is a core course. In case the course is an elective, the student
\end{flushleft}


\begin{flushleft}
may take the same course again or any other course from the same category. {`}E' and {`}F' Grades are not counted
\end{flushleft}


\begin{flushleft}
in the calculation of the CGPA; however, these are taken into account in the calculation of the SGPA. (see 2.10
\end{flushleft}


\begin{flushleft}
for definitions)
\end{flushleft}





\begin{flushleft}
I grade
\end{flushleft}





\begin{flushleft}
An {`}I' grade is temporarily awarded to a student on his / her request to denote incomplete performance in L (lecture), P
\end{flushleft}


\begin{flushleft}
(practical), V (special module) category courses. Requests for {`}I' grade should be made at the earliest but not later than
\end{flushleft}


\begin{flushleft}
the last day of major tests. An {`}I' grade is awarded in case of absence on medical grounds or other special circumstances,
\end{flushleft}


\begin{flushleft}
before or during the major examination period, provided the student has met the attendance criterion of the course.
\end{flushleft}


\begin{flushleft}
Attendance in the course for which {`}I' grade is being sought will be certified by the course coordinator of the course.
\end{flushleft}


\begin{flushleft}
The course coordinators can instruct all students awarded {`}I' grade as well as {`}E' grade to appear for a common
\end{flushleft}


\begin{flushleft}
re-major test. All evaluation requirements for such students in the corresponding course(s) should be completed
\end{flushleft}


\begin{flushleft}
before the end of the first week of the next semester. Upon completion of all course requirements, the {`}I' grade is
\end{flushleft}


\begin{flushleft}
converted to a regular grade (A to F, NP or NF).
\end{flushleft}





\begin{flushleft}
NP and NF grades
\end{flushleft}





\begin{flushleft}
These grades are awarded in a course that the student opts to audit. Only elective courses can be audited. Auditing
\end{flushleft}


\begin{flushleft}
a course is allowed until a date stipulated in the semester schedule. The audit pass (NP) grade is awarded if the
\end{flushleft}


\begin{flushleft}
student's attendance is above 75\% in the class and he / she has obtained at least {`}D' grade. The course coordinator
\end{flushleft}


\begin{flushleft}
can specify a higher criterion, at the beginning of the semester, for audit pass. If the stipulated requirements are
\end{flushleft}


\begin{flushleft}
not fulfilled, the audit fail (NF) grade is awarded. The grades obtained in an audit course are not considered in the
\end{flushleft}


\begin{flushleft}
calculation of SGPA, CGPA or DGPA. However, for undergraduate students, the credits will be counted in total
\end{flushleft}


\begin{flushleft}
earned credits in the respective category, subject to the maximum allowable limit for audit.
\end{flushleft}





\begin{flushleft}
W grade
\end{flushleft}





\begin{flushleft}
A {`}W' grade is awarded in a course from which the student has opted to withdraw. Withdrawal from a course is
\end{flushleft}


\begin{flushleft}
permitted until the date specified in the Semester Schedule. Withdrawal from PG major project part 2 is allowed
\end{flushleft}


\begin{flushleft}
only if he/she is given semester withdrawal. The W grade is mentioned on the grade card.
\end{flushleft}





\begin{flushleft}
X grade
\end{flushleft}





\begin{flushleft}
The {`}X' grade is awarded for incomplete work in Independent Study, Mini Project, Minor Project, or Major Project
\end{flushleft}


\begin{flushleft}
Part 1 and Part 2, based on the request of the student. On completion of the work, {`}X' grade can be converted to
\end{flushleft}


\begin{flushleft}
a regular grade within the first week of the next semester. Otherwise, the student will be awarded {`}X' grade on a
\end{flushleft}


\begin{flushleft}
permanent basis and it will appear in his / her grade card. Further, the student will be required to register for the
\end{flushleft}


\begin{flushleft}
course in the next semester. The credits of the course will be counted towards his / her total load for the semester.
\end{flushleft}


\begin{flushleft}
In case of Major Project Part 1, the student will not be permitted to register for Major Project Part 2 simultaneously
\end{flushleft}


\begin{flushleft}
as Major Project Part 1 is a pre-requisite for Major Project Part 2. A regular full-time student can be awarded {`}X'
\end{flushleft}


\begin{flushleft}
grade only once in a course, other than the summer semester. A part-time M.Tech. student is permitted a maximum
\end{flushleft}


\begin{flushleft}
of two X-grades in the major project part-2.
\end{flushleft}





\begin{flushleft}
S and Z grades
\end{flushleft}





\begin{flushleft}
The {`}S' grade denotes satisfactory performance and completion of a course. The {`}Z' grade is awarded for noncompletion of the course requirements, and if it is a core course, the student will have to register for the course
\end{flushleft}


\begin{flushleft}
until he/she obtains the {`}S' grade. The specific courses in which {`}S' or {`}Z' grades are awarded for undergraduate
\end{flushleft}


\begin{flushleft}
students are:
\end{flushleft}


\begin{flushleft}
(i)	
\end{flushleft}


\begin{flushleft}
Introduction to Engineering and Programme
\end{flushleft}


\begin{flushleft}
(ii)	 Language and writing skills
\end{flushleft}


\begin{flushleft}
(iii)	 NCC / NSO / NSS
\end{flushleft}


\begin{flushleft}
(iv)	 Professional Ethics and Social Responsibility
\end{flushleft}


\begin{flushleft}
(v)	 Communication Skills / Seminar
\end{flushleft}


\begin{flushleft}
(vi)	 Design / Practical Experience
\end{flushleft}


\begin{flushleft}
OOo
\end{flushleft}





\begin{flushleft}
Besides, summer / winter internships in some PG programmes are also awarded S / Z.
\end{flushleft}


10





\begin{flushleft}
\newpage
Courses of Study 2017-2018
\end{flushleft}





\begin{flushleft}
2.10	 Evaluation of Performance
\end{flushleft}


\begin{flushleft}
The performance of a student will be evaluated in terms of three indices, viz., the Semester Grade Point Average
\end{flushleft}


\begin{flushleft}
(SGPA) which is the Grade Point Average for a semester, Cumulative Grade Point Average (CGPA) which is the
\end{flushleft}


\begin{flushleft}
Grade Point Average for all the completed semesters at any point in time, and Degree Grade Point Average (DGPA).
\end{flushleft}


\begin{flushleft}
Degree Grade Point Average (DGPA) is calculated on the basis of the best valid credits in each category, after
\end{flushleft}


\begin{flushleft}
graduation requirements in all categories are met.
\end{flushleft}


\begin{flushleft}
The Earned Credits (E.C.) are defined as the sum of credits for courses in which a student has been awarded pass
\end{flushleft}


\begin{flushleft}
grades. Points secured in a semester =$\Sigma$ (Course credits × Grade point for all courses in which pass grade has
\end{flushleft}


\begin{flushleft}
been obtained). The SGPA is calculated on the basis of grades obtained in all courses the student registered for,
\end{flushleft}


\begin{flushleft}
in the particular semester, except audit courses.
\end{flushleft}


	


	


\begin{flushleft}
SGPA 	 =
\end{flushleft}


			





\begin{flushleft}
Points secured in the semester
\end{flushleft}


\begin{flushleft}
Credits registered in the semester, excluding audit and S / Z grade courses
\end{flushleft}





\begin{flushleft}
The CGPA is calculated on the basis of pass grades obtained in all completed semesters, except audit courses
\end{flushleft}


\begin{flushleft}
and courses in which S/Z grade is awarded.
\end{flushleft}


	


\begin{flushleft}
Cumulative points secured in courses with pass grades
\end{flushleft}


	


\begin{flushleft}
CGPA 	 =
\end{flushleft}


\begin{flushleft}
			 Cumulative earned credits, excluding audit and S / Z grade courses
\end{flushleft}


\begin{flushleft}
Examples of these calculations are given in Tables 4(a) and 4 (b).
\end{flushleft}


\begin{flushleft}
Table 4 : (a) Typical academic performance calculations - I semester
\end{flushleft}


\begin{flushleft}
Course no.
\end{flushleft}





\begin{flushleft}
Course credits
\end{flushleft}





\begin{flushleft}
Grade awarded
\end{flushleft}





\begin{flushleft}
Earned credits
\end{flushleft}





\begin{flushleft}
Grade points
\end{flushleft}





\begin{flushleft}
Points secured
\end{flushleft}





\begin{flushleft}
(column 1)
\end{flushleft}





\begin{flushleft}
(column 2)
\end{flushleft}





\begin{flushleft}
(column 3)
\end{flushleft}





\begin{flushleft}
(column 4)
\end{flushleft}





\begin{flushleft}
(column 5)
\end{flushleft}





\begin{flushleft}
(column 6)
\end{flushleft}





\begin{flushleft}
MTLXXX
\end{flushleft}





5





\begin{flushleft}
C
\end{flushleft}





5





6





30





\begin{flushleft}
COLXXX
\end{flushleft}





4





\begin{flushleft}
C (-)
\end{flushleft}





4





5





20





\begin{flushleft}
PYLXXX
\end{flushleft}





4





\begin{flushleft}
A
\end{flushleft}





4





10





40





\begin{flushleft}
PYPXXX
\end{flushleft}





2





\begin{flushleft}
B
\end{flushleft}





2





8





16





\begin{flushleft}
MCLXXX
\end{flushleft}





4





\begin{flushleft}
E
\end{flushleft}





0





2





08





\begin{flushleft}
TXNXXX
\end{flushleft}





2





\begin{flushleft}
S
\end{flushleft}





2





---





---





\begin{flushleft}
Credits registered in the semester (total of column 2)	
\end{flushleft}


\begin{flushleft}
Credits registered in the semester excluding audit and S/Z grade courses	
\end{flushleft}


\begin{flushleft}
Earned credits in the semester (total of column 4)	
\end{flushleft}


\begin{flushleft}
Earned credits in the semester excluding audit and S/Z grade courses	
\end{flushleft}


\begin{flushleft}
Points secured in the semester (total of column 6)	
\end{flushleft}


\begin{flushleft}
Points secured in the semester in all passed courses (total of column 6 and pass grade)	
\end{flushleft}





=	


21


=	


19


=	


17


=	


15


=	 114


=	106





	


\begin{flushleft}
Points secured in the semester	
\end{flushleft}


	 114	


\begin{flushleft}
	 SGPA 	=
\end{flushleft}


=


= 	6.000


\begin{flushleft}
			 Credits registered in the semester, excluding audit and S / Z grade courses		 19
\end{flushleft}





	


\begin{flushleft}
Cumulative points secured in courses with pass grades	
\end{flushleft}


	 106	


	


\begin{flushleft}
CGPA 	 =
\end{flushleft}


=


= 	7.067


\begin{flushleft}
			 Cumulative earned credits, excluding audit and S / Z grade courses		 15
\end{flushleft}


\begin{flushleft}
Semester performance:	 Earned credits (E.C.) = 17	
\end{flushleft}


\begin{flushleft}
Cumulative performance:	 Earned credits (E.C.) = 17	
\end{flushleft}





\begin{flushleft}
SGPA = 6.000
\end{flushleft}


\begin{flushleft}
CGPA = 7.067
\end{flushleft}


11





\begin{flushleft}
\newpage
Courses of Study 2017-2018
\end{flushleft}





\begin{flushleft}
Table 4 : (b) Typical academic performance calculations - II semester
\end{flushleft}


\begin{flushleft}
Course no.
\end{flushleft}





\begin{flushleft}
Course credits
\end{flushleft}





\begin{flushleft}
Grade awarded
\end{flushleft}





\begin{flushleft}
Earned credits
\end{flushleft}





\begin{flushleft}
Grade points
\end{flushleft}





\begin{flushleft}
Points secured
\end{flushleft}





\begin{flushleft}
(column 1)
\end{flushleft}





\begin{flushleft}
(column 2)
\end{flushleft}





\begin{flushleft}
(column 3)
\end{flushleft}





\begin{flushleft}
(column 4)
\end{flushleft}





\begin{flushleft}
(column 5)
\end{flushleft}





\begin{flushleft}
(column 6)
\end{flushleft}





\begin{flushleft}
MTLXXX
\end{flushleft}





5





\begin{flushleft}
B
\end{flushleft}





5





8





40





\begin{flushleft}
ELLXXX
\end{flushleft}





4





\begin{flushleft}
A (-)
\end{flushleft}





4





9





36





\begin{flushleft}
CMLXXX
\end{flushleft}





4





\begin{flushleft}
W
\end{flushleft}





---





---





---





\begin{flushleft}
CMPXXX
\end{flushleft}





2





\begin{flushleft}
B (-)
\end{flushleft}





2





7





14





\begin{flushleft}
MCLXXX
\end{flushleft}





4





\begin{flushleft}
C
\end{flushleft}





4





6





24





\begin{flushleft}
APLXXX
\end{flushleft}





4





\begin{flushleft}
A
\end{flushleft}





4





10





40





\begin{flushleft}
NLNXXX
\end{flushleft}





1





\begin{flushleft}
S
\end{flushleft}





1





---





---





\begin{flushleft}
Credits registered in the semester (total of column 2)	
\end{flushleft}





=	24





\begin{flushleft}
Credits registered in the semester excluding audit and S/Z grade courses	
\end{flushleft}





=	





\begin{flushleft}
Earned credits in the semester (total of column 4)	
\end{flushleft}





=	20





\begin{flushleft}
Earned credits in the semester excluding audit \& S/Z grade courses	
\end{flushleft}





=	





\begin{flushleft}
Points secured in this semester (total of column 6)	
\end{flushleft}





=	154





23


19





\begin{flushleft}
Points secured in this semester in all passed courses (total of column 6 \& A-D grade)	 =	154
\end{flushleft}


\begin{flushleft}
Cumulative points secured = 106 (I semester) + 154 (this sem.)	
\end{flushleft}





=	





260





\begin{flushleft}
Cumulative earned credits = 17 (I semester) + 20 (this sem.)	
\end{flushleft}





=	





37





	





	


\begin{flushleft}
Points secured in the semester	
\end{flushleft}


	 154	


\begin{flushleft}
SGPA 	=
\end{flushleft}


=


= 	8.105


\begin{flushleft}
			 Credits registered in the semester, excluding audit and S / Z grade courses		 19
\end{flushleft}


	


\begin{flushleft}
Cumulative points secured in courses with pass grades	
\end{flushleft}


	


\begin{flushleft}
	 CGPA 	 =
\end{flushleft}


=


\begin{flushleft}
			 Cumulative earned credits, excluding audit and S / Z grade courses		
\end{flushleft}


\begin{flushleft}
Semester performance:	
\end{flushleft}





\begin{flushleft}
Earned credits (E.C.) = 20	
\end{flushleft}





\begin{flushleft}
SGPA = 8.105
\end{flushleft}





\begin{flushleft}
Cumulative performance:	 Earned credits (E.C.) = 37	
\end{flushleft}





\begin{flushleft}
CGPA = 7.647
\end{flushleft}





106	+	154	


15	+	 19





=	7.647





\begin{flushleft}
On completing all the degree requirements, the degree grade point average, DGPA, will be calculated and this
\end{flushleft}


\begin{flushleft}
value will be indicated on the degree/diploma. The DGPA will be calculated on the basis of category-wise best
\end{flushleft}


\begin{flushleft}
valid credits required for graduation.
\end{flushleft}


\begin{flushleft}
A student who has earned the requisite credits but does not meet the graduation DGPA requirement may do
\end{flushleft}


\begin{flushleft}
additional courses in any elective category to meet the DGPA requirement within the maximum permissible time limit.
\end{flushleft}





3.	





\begin{flushleft}
REGISTRATION AND ATTENDANCE
\end{flushleft}





\begin{flushleft}
3.1	Registration
\end{flushleft}


\begin{flushleft}
Registration is a very important procedural part of the academic system. The registration procedure ensures that
\end{flushleft}


\begin{flushleft}
the student's name is on the roll list of each course that he / she wants to study. No credit is given if the student
\end{flushleft}


\begin{flushleft}
attends a course for which he/she has not registered. Registration for courses to be taken in a particular semester
\end{flushleft}


\begin{flushleft}
will be done according to a specified schedule before the end of the previous semester. Each student is required
\end{flushleft}


\begin{flushleft}
to complete the registration process on the web based system. The student must also take steps to pay his/her
\end{flushleft}


\begin{flushleft}
dues before the beginning of the semester. Students who do not make payments by a stipulated date can be
\end{flushleft}


\begin{flushleft}
de-registered for the particular semester.
\end{flushleft}


12





\begin{flushleft}
\newpage
Courses of Study 2017-2018
\end{flushleft}





\begin{flushleft}
In-absentia registration or registration after the specified date will be allowed only in rare cases at the discretion
\end{flushleft}


\begin{flushleft}
of Dean, Academics. In case of illness or absence during registration, the student should intimate the same to his/
\end{flushleft}


\begin{flushleft}
her Programme coordinator and Dean, Academics.
\end{flushleft}


\begin{flushleft}
Brief description of registration related activities is given in the following paragraphs. The relevant dates are included
\end{flushleft}


\begin{flushleft}
in the Semester Schedule that is made available before the start of the semester. There may be changes in the
\end{flushleft}


\begin{flushleft}
schedule and/ or procedure of registration from time to time. The students are intimated through e-mail about
\end{flushleft}


\begin{flushleft}
any such change to the e-mail address allocated to each student by the Institute at the time of admission. This
\end{flushleft}


\begin{flushleft}
e-mail address is the only channel through which the Institute would communicate with the student. For
\end{flushleft}


\begin{flushleft}
cyber security reasons, e-mail accounts / kerberos logins that are not used for a certain length of time are
\end{flushleft}


\begin{flushleft}
disabled and such accounts locked / deleted by the Institute. Students must therefore login into their e-mail
\end{flushleft}


\begin{flushleft}
accounts / kerberos logins regularly.
\end{flushleft}





\begin{flushleft}
3.2	 Registration and Student Status
\end{flushleft}


\begin{flushleft}
Failure to register before the last date for late registration will imply that the student has discontinued studies and
\end{flushleft}


\begin{flushleft}
his/her name will be struck-off the rolls.
\end{flushleft}


\begin{flushleft}
All registered students, except part-time postgraduate students and casual students, are considered as full-time
\end{flushleft}


\begin{flushleft}
students at the Institute. They are expected to be present at the Institute and devote full time to academics and
\end{flushleft}


\begin{flushleft}
co-curricular and extra curricular activities in the campus.
\end{flushleft}





\begin{flushleft}
3.3	 Advice on Courses
\end{flushleft}


\begin{flushleft}
At the time of registration, each student must finalize the academic programme, keeping in view factors such as,
\end{flushleft}


\begin{flushleft}
minimum/maximum numbers of total and lecture credits, past performance, backlog of courses, SGPA/CGPA,
\end{flushleft}


\begin{flushleft}
pre-requisite(s), work load and student's interests, amongst others. Special provisions exist for advising academically
\end{flushleft}


\begin{flushleft}
weak students. Details are given in section 4.7.
\end{flushleft}





\begin{flushleft}
3.4	 Validation of Registration
\end{flushleft}


\begin{flushleft}
Before the commencement of classes of each semester, on a date specified in the Semester Schedule, every student
\end{flushleft}


\begin{flushleft}
including part-time students, is required to be present on campus and validate his/her registration by logging into
\end{flushleft}


\begin{flushleft}
the website. Students who do not do registration validation will not be permitted to add/drop courses.
\end{flushleft}





\begin{flushleft}
3.5	 Minimum Student Registration in a Course
\end{flushleft}


\begin{flushleft}
Undergraduate courses (of 100, 200, 300, or 400 level) and M.Sc. courses (500 or 600 level) will run if a minimum
\end{flushleft}


\begin{flushleft}
of 12 students register for the course. Under special circumstances, a departmental elective course may be
\end{flushleft}


\begin{flushleft}
allowed to run with minimum registration of 8 students, with prior permission of Chairman, Senate. A 700 or 800
\end{flushleft}


\begin{flushleft}
level course can run with a minimum of 4 students. This requirement will be verified on the last date for Add/Drop.
\end{flushleft}


\begin{flushleft}
Courses without the minimum enrolment will be dropped. The students who had registered for these courses will
\end{flushleft}


\begin{flushleft}
be de-registered, and they will be given one more day for adding a course in lieu of the dropped course.
\end{flushleft}





\begin{flushleft}
3.6	 Late Registration
\end{flushleft}


\begin{flushleft}
For reasons beyond his/her control, if a student is not able to register or send an authorized representative
\end{flushleft}


\begin{flushleft}
with a medical certificate, he/she may apply to the Dean, Academics for late registration. Dean, Academics will
\end{flushleft}


\begin{flushleft}
consider and may approve late registration in genuine cases on payment of an extra fee called late registration
\end{flushleft}


\begin{flushleft}
fee. Late registration is permitted until a date specified in the Semester Schedule, typically one week after the
\end{flushleft}


\begin{flushleft}
beginning of the semester.
\end{flushleft}





\begin{flushleft}
3.7	 Add / Drop, Audit and Withdrawal of Courses
\end{flushleft}


\begin{flushleft}
a)	
\end{flushleft}





\begin{flushleft}
Add / Drop: A student has the option to add courses that he/she has not registered for, or drop courses for
\end{flushleft}


\begin{flushleft}
which he / she has already registered for. This facility is restricted to a period stipulated in the Semester
\end{flushleft}


\begin{flushleft}
Schedule, during the first week of the semester, subject to vacancy status of the courses concerned.
\end{flushleft}


13





\begin{flushleft}
\newpage
Courses of Study 2017-2018
\end{flushleft}





\begin{flushleft}
b)	
\end{flushleft}





\begin{flushleft}
Audit: A student may apply for changing a credit course to an audit course before a deadline specified in the
\end{flushleft}


\begin{flushleft}
Semester Schedule.
\end{flushleft}





\begin{flushleft}
c)	
\end{flushleft}





\begin{flushleft}
Withdrawal: A student who wants to withdraw from a course should apply before a deadline specified in the
\end{flushleft}


\begin{flushleft}
Semester Schedule. A withdrawal grade (W) will be awarded in such cases.
\end{flushleft}





\begin{flushleft}
Appropriate web-based applications are to be used for availing of the above-mentioned options.
\end{flushleft}





\begin{flushleft}
3.8	 Semester Withdrawal
\end{flushleft}


\begin{flushleft}
	Semester withdrawal and absence for a semester under different conditions, viz. (i) medical and personal
\end{flushleft}


\begin{flushleft}
grounds (ii) industrial internship (iii) exchange / deputation to another academic institution in India or abroad,
\end{flushleft}


\begin{flushleft}
and (iv) disciplinary condition can be granted on application. The condition as per the following should be clearly
\end{flushleft}


\begin{flushleft}
specified in the application.
\end{flushleft}


\begin{flushleft}
(a)	 Semester Withdrawal (SW) reflects the condition, in which a student is forced to withdraw from all courses in
\end{flushleft}


\begin{flushleft}
the semester for medical conditions, or for a part-time student when he / she is sent for an outstation assignment
\end{flushleft}


\begin{flushleft}
by his/her employer. A student can apply for semester withdrawal if he / she has missed 20 or more teaching
\end{flushleft}


\begin{flushleft}
days on these grounds. Under no circumstances will an application for semester withdrawal be accepted
\end{flushleft}


\begin{flushleft}
after the commencement of major tests. A student is not permitted to request for semester withdrawal with
\end{flushleft}


\begin{flushleft}
retrospective effect.
\end{flushleft}


\begin{flushleft}
(b)	
\end{flushleft}





\begin{flushleft}
Semester Leave (SL) indicates the situation in which a student is permitted to take one or more semesters off
\end{flushleft}


\begin{flushleft}
for industrial internship or any other assignment with prior approval and planning. The application is to be routed
\end{flushleft}


\begin{flushleft}
through his / her advisor / programme coordinator and Head of the concerned Department / Centre / School.
\end{flushleft}


\begin{flushleft}
Dean, Academics is the final approving authority for such requests. All such applications must be processed
\end{flushleft}


\begin{flushleft}
before the beginning of the semester in which the leave will be taken. At present, JEE-entry B.Tech. and dual
\end{flushleft}


\begin{flushleft}
degree students are allowed one extra semester for completion of the programme for every semester leave
\end{flushleft}


\begin{flushleft}
for industrial internship. Such students are permitted a maximum of two semesters of leave. The full-time 2
\end{flushleft}


\begin{flushleft}
year M.Tech. / M.S.(R) students would be permitted a maximum of one semester leave for industrial internship
\end{flushleft}


\begin{flushleft}
or other assignment as approved by the Dean. These semesters will not be counted towards the maximum
\end{flushleft}


\begin{flushleft}
permitted time period for completion of the degree similar to the provision for JEE entry students.
\end{flushleft}





\begin{flushleft}
(c)	 When a student registers at another academic institution in India or abroad with the expectation of credit
\end{flushleft}


\begin{flushleft}
transfer or research work through a pre-approved arrangement including an MoU, the student should be
\end{flushleft}


\begin{flushleft}
considered as being on a Semester Exchange (SE). The SE period will be counted towards the total period
\end{flushleft}


\begin{flushleft}
permitted for the degree.
\end{flushleft}


\begin{flushleft}
(d)	 When a student is suspended for one or more semesters on disciplinary grounds, the student status should
\end{flushleft}


\begin{flushleft}
be called Disciplinary Withdrawal period (DW).
\end{flushleft}





\begin{flushleft}
3.9	 Registration in Special Module Courses
\end{flushleft}


\begin{flushleft}
Special module courses, i.e. {`}V'-category courses, are 1 or 2 credit courses that can be offered at the beginning
\end{flushleft}


\begin{flushleft}
of the semester and the regular registration procedure will be followed. A {`}V'-category course may also be
\end{flushleft}


\begin{flushleft}
offered during the semester. In such a case, students will be allowed to add this course before classes for the
\end{flushleft}


\begin{flushleft}
course begin. These courses will usually cover specialized topics that are not generally available in the regular
\end{flushleft}


\begin{flushleft}
courses. Eligible students can register for these courses. The course coordinator will evaluate the students'
\end{flushleft}


\begin{flushleft}
performance and award a letter grade. The credits so earned will count towards the appropriate category for
\end{flushleft}


\begin{flushleft}
degree completion purposes.
\end{flushleft}





\begin{flushleft}
3.10  Registration for Non-Graded Units
\end{flushleft}


\begin{flushleft}
Details pertaining to registration and other modalities of earning non-graded units are given in section 8 of this booklet.
\end{flushleft}





\begin{flushleft}
3.11  Pre-requisite Requirement(s) for Registration
\end{flushleft}


\begin{flushleft}
A student can register for a course only if he / she fulfills the pre-requisite requirement(s). Request for relaxation
\end{flushleft}


\begin{flushleft}
of pre-requisite requirement(s) may be raised by students under special circumstances. Such a request needs
\end{flushleft}


\begin{flushleft}
approval of the Departmental Faculty Advisor and Chairman Grades and Registration.
\end{flushleft}





\begin{flushleft}
3.12  Overlapping/Equivalent Courses
\end{flushleft}


\begin{flushleft}
A student is not allowed to earn credits from two overlapping / equivalent courses. Overlapping / equivalent courses,
\end{flushleft}


\begin{flushleft}
wherever applicable, are specified in the Description of Course Contents.
\end{flushleft}


14





\begin{flushleft}
\newpage
Courses of Study 2017-2018
\end{flushleft}





\begin{flushleft}
3.13  Limits on Registration
\end{flushleft}


\begin{flushleft}
An undergraduate student (B.Tech. or Dual Degree) should register for a minimum of 12 credits in a semester. The
\end{flushleft}


\begin{flushleft}
maximum number of credits permitted for a UG student in a semester is 26, with a provision to register for up to
\end{flushleft}


\begin{flushleft}
28 credits in a maximum of two semesters during the entire period of their study. This number would be reduced to
\end{flushleft}


\begin{flushleft}
a maximum of 1.25 times the average credits earned by the student in the past two registered semesters, in case
\end{flushleft}


\begin{flushleft}
the student is placed on probation on the basis of academic performance.
\end{flushleft}





\begin{flushleft}
3.14  Registration and Fee Payment
\end{flushleft}


\begin{flushleft}
Every registered student must pay the stipulated fees in full before the specified deadlines. In the event that a
\end{flushleft}


\begin{flushleft}
student does not make these payments, he/she can be de-registered from all courses and his/her name can be
\end{flushleft}


\begin{flushleft}
struck off from the rolls.
\end{flushleft}





\begin{flushleft}
3.15  Continuous Absence and Registration Status
\end{flushleft}


\begin{flushleft}
If a student is absent from the Institute for more than four weeks without notifying the Head of Department/Centre/
\end{flushleft}


\begin{flushleft}
School or Dean, Academics his/her registration will be terminated and name will be removed from the Institute rolls.
\end{flushleft}





\begin{flushleft}
3.16  Attendance Rule
\end{flushleft}


\begin{flushleft}
It is mandatory for the students to attend all classes. Attendance Records of all students for each course will be
\end{flushleft}


\begin{flushleft}
maintained.
\end{flushleft}


\begin{flushleft}
The Course Coordinator will announce the class policy on attendance with respect to grading etc., at the beginning
\end{flushleft}


\begin{flushleft}
of the semester. This shall be done keeping in mind the importance of classroom learning in the teaching-learning
\end{flushleft}


\begin{flushleft}
process. Once the class attendance policy has been made clear to all the students registered for the course, the
\end{flushleft}


\begin{flushleft}
Course Coordinator will implement the same in totality.
\end{flushleft}


\begin{flushleft}
For the purpose of attendance calculation, every scheduled practical class will count as one unit irrespective of
\end{flushleft}


\begin{flushleft}
the number of contact hours.
\end{flushleft}


\begin{flushleft}
Attendance record will be maintained based upon roll calls (or any equivalent operation) in every scheduled lecture,
\end{flushleft}


\begin{flushleft}
tutorial and practical class. Students are required to strictly adhere to and comply with any method or device
\end{flushleft}


\begin{flushleft}
employed by the Course Coordinator/Instructor for purpose of Attendance Recording. Failure to do so may call for
\end{flushleft}


\begin{flushleft}
disciplinary action. The course coordinator will maintain and consolidate attendance record for the course (lectures,
\end{flushleft}


\begin{flushleft}
tutorials and practicals together, as applicable).
\end{flushleft}


\begin{flushleft}
A Course Coordinator may choose any one or more of the following as attendance policy.
\end{flushleft}


\begin{flushleft}
(a)	 The Course Coordinator can assign 10\% of the total marks to surprise quiz(zes). If attendance of
\end{flushleft}


\begin{flushleft}
the student is greater than 90\%, result of the best three quizzes will be considered, else average
\end{flushleft}


\begin{flushleft}
of all quizzes will be considered.
\end{flushleft}


\begin{flushleft}
(b)	 The Course Coordinator can allocate specific marks for participation in discussions in the class on
\end{flushleft}


\begin{flushleft}
a regular basis.
\end{flushleft}


\begin{flushleft}
(c)	 If a student's attendance is less than 75\%, the student will be awarded one grade less than the actual
\end{flushleft}


\begin{flushleft}
grade that he / she has earned. For example, a student who has got A grade but has attendance less
\end{flushleft}


\begin{flushleft}
than 75\% will be awarded A (-) grade.
\end{flushleft}


\begin{flushleft}
(d)	 A student cannot get NP for an audit course if his / her attendance is less than 75\%.
\end{flushleft}


\begin{flushleft}
The Course Coordinator can implement any other attendance policy provided the policy is approved by the Dean,
\end{flushleft}


\begin{flushleft}
Academics.
\end{flushleft}


\begin{flushleft}
Attendance statistics will also be used in the following way:
\end{flushleft}


\begin{flushleft}
(a)	 If a student's attendance is less than 75\% in more than two courses without any valid reason in a
\end{flushleft}


\begin{flushleft}
semester, he/she will be issued warning and put under probation. If this is repeated, he/she will not be
\end{flushleft}


\begin{flushleft}
allotted a hostel seat in the next semester.
\end{flushleft}


15





\begin{flushleft}
\newpage
Courses of Study 2017-2018
\end{flushleft}





\begin{flushleft}
(b)	 If a student's attendance is less than 75\% in any course or CGPA is less than 7.0, then he/she will not
\end{flushleft}


\begin{flushleft}
be eligible to hold any position of responsibility in the hostel/institute in the next semester.
\end{flushleft}





4.	





\begin{flushleft}
UNDERGRADUATE DEGREE REQUIREMENTS, REGULATIONS AND PROCEDURES
\end{flushleft}





\begin{flushleft}
4.1	 Overall Requirements
\end{flushleft}


\begin{flushleft}
4.1.1 B.Tech.
\end{flushleft}


\begin{flushleft}
The total credit requirement for the B.Tech. (4-year programme) is 145-155 credits (exact requirement is discipline
\end{flushleft}


\begin{flushleft}
specific). The minimum and maximum number of registered semesters for graduation requirements are listed in
\end{flushleft}


\begin{flushleft}
Table 7. For B. Tech. programmes, the total credits are distributed over following categories :
\end{flushleft}


\begin{flushleft}
(a)	
\end{flushleft}





\begin{flushleft}
Institute Core (IC) :
\end{flushleft}





	





$\bullet$	





\begin{flushleft}
Basic Sciences (BS) : Mathematics, Physics, Chemistry and Biology courses
\end{flushleft}





	





$\bullet$	





\begin{flushleft}
Engineering Arts and Science (EAS): Fundamental engineering courses
\end{flushleft}





	





$\bullet$	





\begin{flushleft}
Humanities and Social Sciences (HUSS): At least two courses to be taken in the 200 level 	
\end{flushleft}


\begin{flushleft}
and at least one course to be taken in the 300 level. Management Courses (MSL 3XX) are not
\end{flushleft}


\begin{flushleft}
counted under this category.
\end{flushleft}





\begin{flushleft}
(b)	
\end{flushleft}





\begin{flushleft}
Departmental Core (DC) : courses of relevant discipline.
\end{flushleft}





\begin{flushleft}
(c)	
\end{flushleft}





\begin{flushleft}
Departmental Electives (DE) : electives related to the parent discipline.
\end{flushleft}





\begin{flushleft}
(d)	
\end{flushleft}





\begin{flushleft}
Programme linked basic sciences / EAS (PL) : additional BS / EAS courses that are specified by the
\end{flushleft}


\begin{flushleft}
department.
\end{flushleft}





	


	





\begin{flushleft}
(e)	 Open Category (OC) : electives can be taken outside or within the discipline ; these credits can be
\end{flushleft}


\begin{flushleft}
used towards departmental specialization or minor area also (see Sec 4.6).
\end{flushleft}


\begin{flushleft}
(f)	
\end{flushleft}





\begin{flushleft}
Non-graded Core (NG) units : These are core requirements and can be earned through formal academic
\end{flushleft}


\begin{flushleft}
activity and informal co-curricular or extra-curricular activities.
\end{flushleft}





\begin{flushleft}
4.1.2  Dual degree programmes :
\end{flushleft}


\begin{flushleft}
The total credit requirements for a dual degree programme would depend upon the credit requirements of the
\end{flushleft}


\begin{flushleft}
B.Tech. and M.Tech. programmes that constitute the Dual Degree. The minimum credit requirement for the award
\end{flushleft}


\begin{flushleft}
of Dual Degree would typically be 10 less than the total credits of the constituent B.Tech. and M.Tech. programmes.
\end{flushleft}


\begin{flushleft}
The B.Tech. requirements for a dual degree are same as that given in Section 4.1.1. The M.Tech. part is divided
\end{flushleft}


\begin{flushleft}
into two categories -- Programme Core (PC) and Programme Elective (PE). The minimum and maximum number
\end{flushleft}


\begin{flushleft}
of registered semesters for graduation requirements are listed in Table 7.
\end{flushleft}





\begin{flushleft}
4.2	 Breakup of Degree Requirements
\end{flushleft}


\begin{flushleft}
4.2.1  Earned Credit Requirements
\end{flushleft}


\begin{flushleft}
The minimum earned credit/unit requirements for B.Tech. degree are given in Table 5.
\end{flushleft}


\begin{flushleft}
Table 5 : Degree requirements of B.Tech. programmes
\end{flushleft}


\begin{flushleft}
Category
\end{flushleft}





\begin{flushleft}
Symbol B. Tech. Requirements
\end{flushleft}





\begin{flushleft}
Remarks
\end{flushleft}





1





\begin{flushleft}
Institute Core
\end{flushleft}





\begin{flushleft}
IC
\end{flushleft}





\begin{flushleft}
55 Credits
\end{flushleft}





\begin{flushleft}
Common to all disciplines
\end{flushleft}





2





\begin{flushleft}
Programme Linked EAS/BS
\end{flushleft}





\begin{flushleft}
PL
\end{flushleft}





\begin{flushleft}
0-15 Credits
\end{flushleft}





\begin{flushleft}
Discipline specific as decided by the
\end{flushleft}


\begin{flushleft}
Department
\end{flushleft}





3





\begin{flushleft}
Departmental core
\end{flushleft}





\begin{flushleft}
DC
\end{flushleft}





4





\begin{flushleft}
Departmental Elective
\end{flushleft}





\begin{flushleft}
DE
\end{flushleft}





\begin{flushleft}
65-80 with min 10 as DE
\end{flushleft}





\begin{flushleft}
Discipline specific
\end{flushleft}





5





\begin{flushleft}
Open Category
\end{flushleft}





\begin{flushleft}
OC
\end{flushleft}





\begin{flushleft}
10 Credits
\end{flushleft}





\begin{flushleft}
Open to student's choice
\end{flushleft}





6





\begin{flushleft}
Non-graded Core
\end{flushleft}





\begin{flushleft}
NG
\end{flushleft}





\begin{flushleft}
15 units
\end{flushleft}





\begin{flushleft}
See Sec. 4.3
\end{flushleft}





\begin{flushleft}
Total
\end{flushleft}





\begin{flushleft}
145-155 Credits +15
\end{flushleft}


\begin{flushleft}
nongraded units
\end{flushleft}


16





\begin{flushleft}
\newpage
Courses of Study 2017-2018
\end{flushleft}





\begin{flushleft}
4.2.2  Degree Grade Point Average (DGPA) Requirement
\end{flushleft}


\begin{flushleft}
A student must obtain a minimum DGPA of 5.0 to be eligible for award of the B.Tech. degree. The minimum DGPA
\end{flushleft}


\begin{flushleft}
requirement for M.Tech. part of dual degree programme is 6.0. All exceptions to the above conditions will be dealt
\end{flushleft}


\begin{flushleft}
with as per following regulations:
\end{flushleft}


\begin{flushleft}
(a)	
\end{flushleft}





\begin{flushleft}
If a student completes required credits for B.Tech. with DGPA less than 5.0, then the student will be permitted
\end{flushleft}


\begin{flushleft}
to do additional elective courses under appropriate category to improve the DGPA within the maximum time
\end{flushleft}


\begin{flushleft}
limit for completion of B.Tech. degree. In case a DGPA of 5.0 or more is achieved within the stipulated period,
\end{flushleft}


\begin{flushleft}
a B.Tech. degree will be awarded and in case the same is not achieved no degree will be awarded and the
\end{flushleft}


\begin{flushleft}
student may apply for a diploma.
\end{flushleft}





\begin{flushleft}
(b)	
\end{flushleft}





\begin{flushleft}
If a student completes requisite credits for Dual Degree Programme:
\end{flushleft}


\begin{flushleft}
(i)	
\end{flushleft}





\begin{flushleft}
with B. Tech. DGPA less than 5.0 but M.Tech. DGPA more than 6.0
\end{flushleft}





	





\begin{flushleft}
The student will be permitted to do additional elective courses (under appropriate category) to
\end{flushleft}


\begin{flushleft}
improve the DGPA for completion of B.Tech. part within the maximum time limit. In case a DGPA of
\end{flushleft}


\begin{flushleft}
5.0 or more is achieved for B.Tech., the student will become eligible for award of the Dual Degree
\end{flushleft}


\begin{flushleft}
(B.Tech. \& M. Tech.) and in case the same is not achieved no degree will be awarded and the student
\end{flushleft}


\begin{flushleft}
may apply for a diploma.
\end{flushleft}





\begin{flushleft}
(ii)	
\end{flushleft}





\begin{flushleft}
with B.Tech. DGPA more than 5.0 but M. Tech. DGPA less than 6.0
\end{flushleft}





	





\begin{flushleft}
The student may opt to do additional elective courses (PE category only) to improve the DGPA within
\end{flushleft}


\begin{flushleft}
the maximum time limit. If no programme elective (PE) courses are available, other relevant 700 and
\end{flushleft}


\begin{flushleft}
800 level courses as approved by the department can be done for the purpose of improving the DGPA.
\end{flushleft}


\begin{flushleft}
In case DGPA of 6.0 or more is achieved for the M.Tech. part, the student will be eligible for award of
\end{flushleft}


\begin{flushleft}
the Dual Degree (B.Tech. \& M.Tech.). However, in case the same is not achieved at the end of the
\end{flushleft}


\begin{flushleft}
stipulated period, the student will be eligible for the award of only B.Tech. degree, provided a written
\end{flushleft}


\begin{flushleft}
request for the same is made to the Dean, Academics.
\end{flushleft}





\begin{flushleft}
(iii)	
\end{flushleft}





\begin{flushleft}
with B.Tech. DGPA less than 5.0 and M.Tech. DGPA less than 6.0
\end{flushleft}





	





\begin{flushleft}
The student will be permitted to do additional elective courses under appropriate categories to improve
\end{flushleft}


\begin{flushleft}
the DGPA for completion of B.Tech. and courses under PE category for completion of M.Tech. degree
\end{flushleft}


\begin{flushleft}
within the maximum time limit. If no programme elective courses are available, relevant 700 and 800
\end{flushleft}


\begin{flushleft}
level courses as approved by the department can be done for the purpose of improving the DGPA of
\end{flushleft}


\begin{flushleft}
the M.Tech. part. In case a DGPA of 5.0 or more for B.Tech. and 6.0 or more for M.Tech. are achieved,
\end{flushleft}


\begin{flushleft}
the student will be eligible for award of the Dual Degree (B.Tech. \& M.Tech.). However, in case a DGPA
\end{flushleft}


\begin{flushleft}
5.0 or more for B.Tech. is achieved but the DGPA of 6.0 or more for M.Tech. is not achieved at the end
\end{flushleft}


\begin{flushleft}
of stipulated period, the student will be eligible for award of only B.Tech. degree provided a written
\end{flushleft}


\begin{flushleft}
request for the same is made to the Dean, Academics.
\end{flushleft}





\begin{flushleft}
(c)	 A student may be permitted to do additional elective courses under appropriate elective categories for
\end{flushleft}


\begin{flushleft}
improving DGPA, even if he / she satisfies all graduation requirements. The student may be permitted to
\end{flushleft}


\begin{flushleft}
register for courses in the additional semesters, up to the maximum limit in terms of registered semesters for
\end{flushleft}


\begin{flushleft}
improving his / her DGPA provided a request for the same is made to the Dean, Academics within 15 days of
\end{flushleft}


\begin{flushleft}
the notification of grades in the final semester. During this period when the student is registered for improving
\end{flushleft}


\begin{flushleft}
DGPA, no hostel facilities or assistantship will be provided to the student.
\end{flushleft}


\begin{flushleft}
(d)	
\end{flushleft}





\begin{flushleft}
A student is eligible to apply for a Diploma provided he / she has earned 100 credits and has exhausted the
\end{flushleft}


\begin{flushleft}
maximum number of permitted registered semesters for completion of his / her degree. If the student has
\end{flushleft}


\begin{flushleft}
completed 50 credits (out of 100 credits) from his / her DC+DE+PC+PE categories then the student will be
\end{flushleft}


\begin{flushleft}
awarded {`}Undergraduate Diploma in the respective discipline' on completion. If the student has not completed
\end{flushleft}


\begin{flushleft}
50 credits from these categories but has completed 100 credits then he / she will be awarded {`}Undergraduate
\end{flushleft}


\begin{flushleft}
Diploma in Engineering'. The Diploma is not equivalent to an undergraduate degree.
\end{flushleft}





\begin{flushleft}
(e)	
\end{flushleft}





\begin{flushleft}
No self-study course will be permitted for the purpose of improvement of DGPA.
\end{flushleft}





\begin{flushleft}
4.2.3 Audit Courses
\end{flushleft}


\begin{flushleft}
Audit facility is open to all undergraduate students who have 85 Earned Credits. A student will be permitted to do
\end{flushleft}


\begin{flushleft}
any number of audit courses over and above the graduation requirements. The audit limits for graduation are:
\end{flushleft}


\begin{flushleft}
(a)	 B.Tech. (4-year) programme: A maximum of 8 credits from the elective courses in any category out of the
\end{flushleft}


\begin{flushleft}
total credits required for B.Tech. degree may be completed on audit basis.
\end{flushleft}


17





\begin{flushleft}
\newpage
Courses of Study 2017-2018
\end{flushleft}





\begin{flushleft}
(b)	
\end{flushleft}





\begin{flushleft}
Dual-degree programme: A maximum of 8 credits from the elective courses in any category may be completed
\end{flushleft}


\begin{flushleft}
on audit basis from the UG part of the programme.
\end{flushleft}





\begin{flushleft}
(c)	
\end{flushleft}





\begin{flushleft}
A student earns either an NP (audit pass) or an NF (audit fail) grade for an audit course. The audit pass (NP)
\end{flushleft}


\begin{flushleft}
grade may be awarded if the student satisfies the attendance criteria specified for the course and he/she
\end{flushleft}


\begin{flushleft}
has obtained at least a {`}D' grade. The course coordinator can specify a higher criterion for audit pass at the
\end{flushleft}


\begin{flushleft}
beginning of the semester. If either of these requirements is not fulfilled, the audit fail (NF) grade is awarded.
\end{flushleft}





\begin{flushleft}
(d)	
\end{flushleft}





\begin{flushleft}
Grades obtained in an audit course are not considered in the calculation of SGPA or CGPA.
\end{flushleft}





\begin{flushleft}
4.3  Non-graded Core Requirement
\end{flushleft}


\begin{flushleft}
As part of the curriculum, non-graded units have been prescribed as core requirements for the undergraduate
\end{flushleft}


\begin{flushleft}
degree. These units can be earned through a combination of formal academic activity and informal co-curricular
\end{flushleft}


\begin{flushleft}
or extra-curricular activities. The components of non-graded core requirement are listed in Table 6.
\end{flushleft}


\begin{flushleft}
Table 6 : Components of Non-Graded Core Requirement
\end{flushleft}


\begin{flushleft}
Components
\end{flushleft}





\begin{flushleft}
Units
\end{flushleft}





1





\begin{flushleft}
Introduction to the Engineering and Programme
\end{flushleft}





02





2





\begin{flushleft}
Language and Writing Skills
\end{flushleft}





02





3





\begin{flushleft}
NCC / NSO / NSS
\end{flushleft}





02





4





\begin{flushleft}
Professional Ethics and Social Responsibility
\end{flushleft}





02





5





\begin{flushleft}
Communication Skills / Seminar
\end{flushleft}





02





6





\begin{flushleft}
Design / Practical Experience :
\end{flushleft}





05





\begin{flushleft}
Total
\end{flushleft}





15





\begin{flushleft}
The 15 units listed in Table 6 will be core requirement for all undergraduate programmes. A student must earn
\end{flushleft}


\begin{flushleft}
these 15 units over the complete duration of the programme with special considerations and requirements for each
\end{flushleft}


\begin{flushleft}
component. A student must get S grades to earn these units. Incomplete performance in these components will be
\end{flushleft}


\begin{flushleft}
indicated by a Z grade. A brief description of the six components is given below.
\end{flushleft}


\begin{flushleft}
(a)	
\end{flushleft}


	





\begin{flushleft}
Introduction to Engineering (NIN100) (1 unit)
\end{flushleft}


\begin{flushleft}
All students will be required to undergo exercises in the first semester, for earning 1 unit. These may involve
\end{flushleft}


\begin{flushleft}
listening to lectures, developing project reports based upon self-study, visit to laboratories (in and outside
\end{flushleft}


\begin{flushleft}
the Institute) and industry, executing simple scientific or engineering projects.
\end{flushleft}





	





\begin{flushleft}
Introduction to Programme (XXN101) (1 unit)
\end{flushleft}





	





\begin{flushleft}
This would be discipline specific introduction to programme. This would be offered in the third semester. In
\end{flushleft}


\begin{flushleft}
the Course no, {`}XX' is the Course code prefix as shown in Table 1. An exception to this are the ME2 and EE3
\end{flushleft}


\begin{flushleft}
Programmes, for which the course number would be MCN111 and ELN111 respectively.
\end{flushleft}





\begin{flushleft}
(b)	 Language and Writing Skills (NLN100--101) (2 units)
\end{flushleft}


	





\begin{flushleft}
All students will be required to undergo exercises in the first year, spanning over two semesters. These
\end{flushleft}


\begin{flushleft}
exercises will be designed to impart language skills -- enhancing their ability of listening comprehension,
\end{flushleft}


\begin{flushleft}
reading and writing in English. Further, students will be exposed to principles of English Grammar and
\end{flushleft}


\begin{flushleft}
nuances of technical writing. These exercises will be tailored according to the background of the students.
\end{flushleft}


\begin{flushleft}
The background of the students will be assessed through a test to be conducted at the beginning of the first
\end{flushleft}


\begin{flushleft}
semester. These exercises can be organized either during normal academic hours or outside. A student can
\end{flushleft}


\begin{flushleft}
be prescribed self learning exercises or additional practice sessions during vacations as requirement for
\end{flushleft}


\begin{flushleft}
securing an S grade.
\end{flushleft}





\begin{flushleft}
(c)	
\end{flushleft}





\begin{flushleft}
NCC/NSO/NSS (NCN100/NSN100/NPN100) (2 units)
\end{flushleft}





	





\begin{flushleft}
NCC/NSO/NSS will form part of core requirement of the degree. Students will be required to earn 2 units
\end{flushleft}


\begin{flushleft}
from these activities. The faculty coordinator will devise a scheme for awarding these units.
\end{flushleft}


18





\begin{flushleft}
\newpage
Courses of Study 2017-2018
\end{flushleft}





\begin{flushleft}
(d)	 Professional Ethics and Social Responsibility (2 units)
\end{flushleft}


	





\begin{flushleft}
There is increasing consensus worldwide that professional ethics need to be incorporated into the
\end{flushleft}


\begin{flushleft}
engineering curriculum to provide students exposure to the kind of professional ethical dilemmas they
\end{flushleft}


\begin{flushleft}
might face on an individual basis as well as the larger ethical aspects of technology development.
\end{flushleft}


\begin{flushleft}
Workshops, discussions / debates will be organized to sensitize students about Professional Ethics
\end{flushleft}


\begin{flushleft}
and Social Responsibility. This course will be also associated with 2 units implying total involvement
\end{flushleft}


\begin{flushleft}
of about 100 hours. Involvement of students in these activities will be monitored by the coordinator for
\end{flushleft}


\begin{flushleft}
awarding the S grade.
\end{flushleft}





\begin{flushleft}
Part 1 : Regular Classroom Contact (NEN100-101) (1 unit)
\end{flushleft}


	





\begin{flushleft}
The first part of PESR involves regular sessions of 1.5-2 hours with a faculty mentor. Activities in the sessions
\end{flushleft}


\begin{flushleft}
would be decided by the faculty mentor, with a total of 14-15 hours in regular sessions in each of the first two
\end{flushleft}


\begin{flushleft}
semesters. NEN100 and NEN101 are compulsory for all students, and a student will earn one unit by getting
\end{flushleft}


\begin{flushleft}
S grade in both these courses.
\end{flushleft}





\begin{flushleft}
Part 2 : Case Studies and Practical / Field Activity (NEN300 and NEN201 / NEN202 / NEN203)
\end{flushleft}


	





\begin{flushleft}
The Second unit under PESR has two components. The first component, Professional Ethics Case Studies,
\end{flushleft}


\begin{flushleft}
is compulsory, and is offered under the course number NEN300. For the second component, the student can
\end{flushleft}


\begin{flushleft}
choose to participate in any one out of a large variety of activities relevant to the core themes of PESR. These
\end{flushleft}


\begin{flushleft}
activities have been divided into three broad categories, viz., (a) PESR internships (b) PESR workshops and
\end{flushleft}


\begin{flushleft}
(c) PESR projects with separate course numbers NEN201, NEN202 and NEN203 respectively. All requirements
\end{flushleft}


\begin{flushleft}
of PESR non-graded component should be completed before the beginning of 7th semester.
\end{flushleft}





\begin{flushleft}
(e)	
\end{flushleft}





\begin{flushleft}
Communication Skills (2 units)
\end{flushleft}





	





\begin{flushleft}
Communication skills is an essential requirement for a modern engineer. As a part of the degree requirements,
\end{flushleft}


\begin{flushleft}
undergraduate students will have to earn 2 units in communication skills.
\end{flushleft}





\begin{flushleft}
	(i)	
\end{flushleft}


\begin{flushleft}
Students need to register for at least one topic-specific seminar course in his / her parent department for
\end{flushleft}


\begin{flushleft}
earning one unit. These courses will be elective, offered in each semester. These seminar sessions will
\end{flushleft}


\begin{flushleft}
be held for two hours per week. Multiple such courses can run in parallel. These seminars will be open
\end{flushleft}


\begin{flushleft}
to all students and faculty of IIT Delhi. These seminars can be scheduled outside office hours as well.
\end{flushleft}


	





\begin{flushleft}
(ii)	 Further, students can earn the remaining one unit through any one of the following means:
\end{flushleft}





		





\begin{flushleft}
$\bullet$	 By successfully undergoing a Communication Skills course / workshop as an activity approved by
\end{flushleft}


\begin{flushleft}
Dean, Academics.
\end{flushleft}





		





\begin{flushleft}
$\bullet$	 By documentary evidence of excellence in debating and / or writing as certified by faculty in-charge
\end{flushleft}


\begin{flushleft}
of these activities.
\end{flushleft}





		





\begin{flushleft}
$\bullet$	 By participating in course seminars of regular courses the student is attending; for example regular
\end{flushleft}


\begin{flushleft}
L courses can have optional seminar component.
\end{flushleft}





		





\begin{flushleft}
$\bullet$	 Registering and completing a seminar course offered by any Department / Centre / School.
\end{flushleft}





	





\begin{flushleft}
A student will be required to earn these units during his/her 5th to 8th registered semester.
\end{flushleft}





\begin{flushleft}
(f)	
\end{flushleft}





\begin{flushleft}
Design and Practical Experience (5 units)
\end{flushleft}





	





\begin{flushleft}
The objective of this non-graded core requirement component is to give opportunities to students to acquire
\end{flushleft}


\begin{flushleft}
substantial design and practical experience both as a part of formal courses as well as in an informal
\end{flushleft}


\begin{flushleft}
setting. Second and even more important objective of this course is to inculcate design thinking among
\end{flushleft}


\begin{flushleft}
students and facilitate gaining some design immersion experience. Design and Practical Experience (DPE)
\end{flushleft}


\begin{flushleft}
component is introduced to promote learning by doing which does two important things: it allows students
\end{flushleft}


\begin{flushleft}
to immerse themselves in the environment in which work is to be done, so that they can understand the
\end{flushleft}


\begin{flushleft}
values and expectations of the target beneficiaries. Secondly, it enables a fresh look at problems, not only
\end{flushleft}


\begin{flushleft}
at the ways of defining them, but also at the ways to solve those including skill-sets that are required to
\end{flushleft}


\begin{flushleft}
address them. A shift from problem based learning (acquisition of knowledge) to project based learning
\end{flushleft}


\begin{flushleft}
(application of knowledge), where the projects are grounded in problems outside the classrooms and
\end{flushleft}


19





\begin{flushleft}
\newpage
Courses of Study 2017-2018
\end{flushleft}





\begin{flushleft}
labs in everyday scenarios, will involve students in reality, and reality in education. Design and Practical
\end{flushleft}


\begin{flushleft}
Experience bridges division between the curricular and the co-curricular, and encourages curiosity and
\end{flushleft}


\begin{flushleft}
involvement that arise out of total absorption in a subject of interest. Non-graded units in Design and
\end{flushleft}


\begin{flushleft}
Practical Experience can be earned through one or more the following:
\end{flushleft}





	





$\bullet$	





\begin{flushleft}
Specialized Elective Courses related to Design and Practical Experience (Maximum 2 Units)
\end{flushleft}





$\bullet$	





\begin{flushleft}
Regular Courses with optional Design and Practical Experience Component (Maximum 2 Units)
\end{flushleft}





$\bullet$	





\begin{flushleft}
Summer / winter / semester / SURA / DISA projects with Institute faculty, not evaluated for earning 	
\end{flushleft}


\begin{flushleft}
credits (Maximum 2 units)
\end{flushleft}





$\bullet$	





\begin{flushleft}
Co-curricular projects such as Robocon, SAE-mini-baja, etc. (Maximum 2 Units)
\end{flushleft}





$\bullet$	





\begin{flushleft}
Summer Internships with Industry (Maximum 2 Units)
\end{flushleft}





$\bullet$	





\begin{flushleft}
One Semester Internship (Maximum 5 Units)
\end{flushleft}





$\bullet$	





\begin{flushleft}
Workshop Module on Design and Practical Experience offered by Faculty / Visitors (1 Unit each)
\end{flushleft}





\begin{flushleft}
4.4	 Minimum and Maximum durations for completing degree requirements
\end{flushleft}


\begin{flushleft}
(a)	
\end{flushleft}





\begin{flushleft}
The minimum and maximum permitted duration of each academic programme will be determined in terms
\end{flushleft}


\begin{flushleft}
of number of registered regular semesters, hereinafter called registered semesters. Any semester in which
\end{flushleft}


\begin{flushleft}
a student has registered for a course will be called a registered semester subject to the following:
\end{flushleft}





	





\begin{flushleft}
(i)	
\end{flushleft}





	





\begin{flushleft}
(ii)	 A semester when a student has been granted semester withdrawal or granted semester leave will not
\end{flushleft}


\begin{flushleft}
be considered as a registered semester.
\end{flushleft}





	





\begin{flushleft}
(iii)	 The semester when a student is suspended from the Institute on disciplinary grounds will not be counted
\end{flushleft}


\begin{flushleft}
towards the number of registered semesters.
\end{flushleft}





	





\begin{flushleft}
(iv)	 A semester in which a student is allowed by the Institute to undergo semester - long internship will be
\end{flushleft}


\begin{flushleft}
counted as a registered semester.
\end{flushleft}





\begin{flushleft}
Only the First and Second semesters of an academic year can be registered semesters. The summer
\end{flushleft}


\begin{flushleft}
semester will not be counted as a registered semester.
\end{flushleft}





\begin{flushleft}
The summer semesters shall normally be available for earning credits. However, after the student has
\end{flushleft}


\begin{flushleft}
registered for the maximum permissible number of registered semesters, the subsequent summer semesters
\end{flushleft}


\begin{flushleft}
will not be available for earning credits.
\end{flushleft}


\begin{flushleft}
(b)	 The minimum and maximum permissible number of registered semesters for completing all degree
\end{flushleft}


\begin{flushleft}
requirements are defined in Table 7.
\end{flushleft}


\begin{flushleft}
Table 7 : Minimum and Maximum permissible duration for completing degree requirements.
\end{flushleft}


\begin{flushleft}
Programme Name
\end{flushleft}





\begin{flushleft}
Minimum Number of
\end{flushleft}


\begin{flushleft}
Registered Semesters
\end{flushleft}





\begin{flushleft}
Maximum Number of Registered Semesters
\end{flushleft}


\begin{flushleft}
Permitted for Completing Degree Requirements
\end{flushleft}





\begin{flushleft}
B.Tech.
\end{flushleft}





8





12*





\begin{flushleft}
Dual Degree
\end{flushleft}





12





14*





\begin{flushleft}
*If a student opts for the slow-paced programme, then the maximum permissible number of registered semesters shall be
\end{flushleft}


\begin{flushleft}
increased by two semesters.
\end{flushleft}





\begin{flushleft}
4.5	 Absence During the Semester
\end{flushleft}


\begin{flushleft}
(a)	
\end{flushleft}





\begin{flushleft}
A student must inform the Dean, Academics immediately of any instance of continuous absence from classes.
\end{flushleft}





\begin{flushleft}
(b)	 A student who is absent due to illness or any other emergency, up to a maximum of two weeks, should
\end{flushleft}


\begin{flushleft}
approach the course coordinator for make-up quizzes, assignments and laboratory work.
\end{flushleft}


\begin{flushleft}
(c)	
\end{flushleft}





\begin{flushleft}
A student who has been absent from a minor test due to illness should approach the course coordinator for
\end{flushleft}


\begin{flushleft}
a make-up test immediately on return to class. The request should be supported with a medical certificate
\end{flushleft}


\begin{flushleft}
from Institute's medical officer. A certificate from a registered medical practitioner will also be acceptable
\end{flushleft}


20





\begin{flushleft}
\newpage
Courses of Study 2017-2018
\end{flushleft}





\begin{flushleft}
for a student normally residing off-campus provided registration number of the medical practitioner appears
\end{flushleft}


\begin{flushleft}
explicitly on the certificate.
\end{flushleft}


\begin{flushleft}
(d)	 In case a student misses a minor test on the same day on which he / she has appeared in another test, a
\end{flushleft}


\begin{flushleft}
medical certificate from the institute's medical officer will only be acceptable.
\end{flushleft}


\begin{flushleft}
(e)	
\end{flushleft}





\begin{flushleft}
In case of absence on medical grounds or other special circumstances, before or during the major examination
\end{flushleft}


\begin{flushleft}
period, the student can apply for {`}I' grade. At least 75 \% attendance in a course is necessary for being eligible
\end{flushleft}


\begin{flushleft}
for request of I-grade in that course. An application requesting I-grade should be made at the earliest but not
\end{flushleft}


\begin{flushleft}
later than the last day of major tests. An online application should be made by the student. On submission of
\end{flushleft}


\begin{flushleft}
a medical certificate / Dean's permission, the UG section verifies the certificate and forwards the request to
\end{flushleft}


\begin{flushleft}
the concerned course coordinator. The course coordinator verifies the attendance requirement and forwards
\end{flushleft}


\begin{flushleft}
the application to the Head of the Department / Centre / School of the student's programme. Head's approval
\end{flushleft}


\begin{flushleft}
is contingent upon the satisfaction of attendance requirement. On approval, an {`}I' grade is awarded to the
\end{flushleft}


\begin{flushleft}
student. All evaluation requirements for students with {`}I' grade should be completed before the end of the
\end{flushleft}


\begin{flushleft}
first week of the next semester. Upon completion of all course requirements, the {`}I' grade is converted to a
\end{flushleft}


\begin{flushleft}
regular grade (A to F, NP or NF).
\end{flushleft}





\begin{flushleft}
(f)	
\end{flushleft}





\begin{flushleft}
In case the period of absence on medical grounds is more than 20 working days during the semester, a student
\end{flushleft}


\begin{flushleft}
may apply for withdrawal from the semester, i.e. withdrawal from all courses registered that semester. Such
\end{flushleft}


\begin{flushleft}
application must be made as early as possible and latest before the start of the major tests. No applications for
\end{flushleft}


\begin{flushleft}
semester withdrawal will be considered after the major tests have commenced. Dean, Academics, depending
\end{flushleft}


\begin{flushleft}
on the merit of the case, will approve such applications. Partial withdrawal from courses registered in a
\end{flushleft}


\begin{flushleft}
semester is not allowed.
\end{flushleft}





\begin{flushleft}
(g)	 If a student is continuously absent from the institute for more than four weeks without notifying the Dean
\end{flushleft}


\begin{flushleft}
Academics, his/her name will be removed from institute rolls.
\end{flushleft}





\begin{flushleft}
4.6	 Conditions for Continuation of Registration, Termination / Re-start, Probation
\end{flushleft}


\begin{flushleft}
and Warning
\end{flushleft}


\begin{flushleft}
During the first two registered semesters of an undergraduate programme, a student is registered for a total of
\end{flushleft}


\begin{flushleft}
34 credits, besides non-graded units. By the end of the first two registered semesters, not including summer, a
\end{flushleft}


\begin{flushleft}
student is expected to earn a minimum number of credits (excluding non-graded units) as specified in Table 8, in
\end{flushleft}


\begin{flushleft}
order to continue registration. If a student does not meet this criterion, his/her performance is classified as {``}Poor
\end{flushleft}


\begin{flushleft}
Performance'', and he/she may opt to start the programme afresh, or else his/her registration will be terminated.
\end{flushleft}


\begin{flushleft}
This option to re-start the programme is available to a student only once.
\end{flushleft}


\begin{flushleft}
Table 8 : Criteria for continuation at the end of second registered semester
\end{flushleft}





\begin{flushleft}
Description
\end{flushleft}


\begin{flushleft}
Minimum for Continuation
\end{flushleft}


\begin{flushleft}
Poor Performance
\end{flushleft}





\begin{flushleft}
Earned Credits
\end{flushleft}


\begin{flushleft}
(excluding non-graded units)
\end{flushleft}


\begin{flushleft}
GE / OBC
\end{flushleft}





\begin{flushleft}
SC / ST / PD
\end{flushleft}





23





19





$\leq$ 22





$\leq$ 18





\begin{flushleft}
Decision
\end{flushleft}


\begin{flushleft}
Continuation of registration
\end{flushleft}


\begin{flushleft}
Restart (Once only) / Termination of
\end{flushleft}


\begin{flushleft}
registration
\end{flushleft}





\begin{flushleft}
(a)	
\end{flushleft}





\begin{flushleft}
If a student chooses to restart after the first two registered semesters, then his / her credits earned and
\end{flushleft}


\begin{flushleft}
semesters registered will not be carried over. The re-start will be indicated on the transcript. The re-start will
\end{flushleft}


\begin{flushleft}
be permitted only once. If at the end of two registered semesters after re-start, the earned credits are less
\end{flushleft}


\begin{flushleft}
than or equal to 22 for GE / OBC or less than or equal to 18 for SC / ST / PD students, then the registration will
\end{flushleft}


\begin{flushleft}
be terminated.
\end{flushleft}





\begin{flushleft}
(b)	
\end{flushleft}





\begin{flushleft}
Each student is expected to earn at least 12 credits in each registered semester with an SGPA greater than
\end{flushleft}


\begin{flushleft}
or equal to 5.0. If the performance of a student at the end of any registered semester is below this minimum
\end{flushleft}


\begin{flushleft}
acceptable level, then he/she will be placed on probation, a warning shall be given to him/her and intimation
\end{flushleft}


\begin{flushleft}
sent to the parents.
\end{flushleft}


21





\begin{flushleft}
\newpage
Courses of Study 2017-2018
\end{flushleft}





\begin{flushleft}
(c)	
\end{flushleft}





\begin{flushleft}
A student placed on probation shall be monitored, including mandatory attendance in classes, special tutorials
\end{flushleft}


\begin{flushleft}
and mentoring. Mentoring would comprise structured guidance under a senior/postgraduate student.
\end{flushleft}





\begin{flushleft}
(d)	
\end{flushleft}





\begin{flushleft}
If the performance of a student on probation does not meet the criterion in item (b) in the following registered
\end{flushleft}


\begin{flushleft}
semester, then the student would face termination, and will be permitted to register by the Dean, Academics only
\end{flushleft}


\begin{flushleft}
if the department makes a favourable recommendation. The Head of the Department's recommendation shall
\end{flushleft}


\begin{flushleft}
be prepared after consultation with the student, and should include (i) feasibility of completing the programme
\end{flushleft}


\begin{flushleft}
requirements, and (ii) identification of remedial measures for the problems leading to poor performance.
\end{flushleft}





\begin{flushleft}
(e)	
\end{flushleft}





\begin{flushleft}
The registration of any student will be limited to 1.25 times the average earned credits of the previous two
\end{flushleft}


\begin{flushleft}
registered semesters, subject to a minimum of 12 credits and a maximum of 26 credits.
\end{flushleft}





\begin{flushleft}
Slow-paced programme
\end{flushleft}


\begin{flushleft}
(a)	
\end{flushleft}





\begin{flushleft}
If a student has earned the minimum credits specified in Table 8 for continuation but has less than 28 Earned
\end{flushleft}


\begin{flushleft}
Credits at the end of the first two registered semesters, he/ she will be eligible to opt for the slow-paced
\end{flushleft}


\begin{flushleft}
programme. A student opting for such a programme shall be permitted two additional registered semesters
\end{flushleft}


\begin{flushleft}
for completing degree requirements as indicated in Table 7.
\end{flushleft}





\begin{flushleft}
(b)	
\end{flushleft}





\begin{flushleft}
In the slow paced programme, the upper limit for credits registered in a semester will be 18. A student in this
\end{flushleft}


\begin{flushleft}
programme is expected to earn at least 9 credits with minimum SGPA of 5.0 in any semester, falling which
\end{flushleft}


\begin{flushleft}
he/ she will be issued a warning and placed on probation.
\end{flushleft}





	





\begin{flushleft}
A student placed on probation would be monitored, including mandatory attendance in special tutorials and
\end{flushleft}


\begin{flushleft}
mentoring.
\end{flushleft}





	





\begin{flushleft}
If the performance of a student on probation does not meet the above criterion in the following registered
\end{flushleft}


\begin{flushleft}
semester, then the student would face termination and will be permitted to register by the Dean Academics
\end{flushleft}


\begin{flushleft}
only if the department makes a favourable recommendation. The Head of the Department's recommendation
\end{flushleft}


\begin{flushleft}
shall be prepared after consultation with the student, and should include (i) feasibility of completing the
\end{flushleft}


\begin{flushleft}
programme, and (ii) identification of remedial measures for the problems leading to poor performance.
\end{flushleft}





\begin{flushleft}
(c)	
\end{flushleft}





\begin{flushleft}
The semester-wise schedule of the slow-paced programme shall be defined by the respective department
\end{flushleft}


\begin{flushleft}
for each student.
\end{flushleft}





\begin{flushleft}
4.7	 Scheme for Academic Advising of Undergraduate Students
\end{flushleft}


\begin{flushleft}
Advising Scheme for Regular Students
\end{flushleft}


\begin{flushleft}
(a)	 There is a class committee for each entry year of all programmes. The class committee is responsible for
\end{flushleft}


\begin{flushleft}
providing consistent and uniform academic advice to the entire batch of students.
\end{flushleft}


\begin{flushleft}
(b)	
\end{flushleft}





\begin{flushleft}
Class committee shall consist of a Chairman, at least two faculty members of the department (one of them will
\end{flushleft}


\begin{flushleft}
function as convenor of the class committee) and elected student representatives (as per CAIC constitution)
\end{flushleft}


\begin{flushleft}
including a student coordinator. All student coordinators of courses intended for the batch in a given semester
\end{flushleft}


\begin{flushleft}
and special advisors of academically weak students will be permanent invitees to the class committee. The
\end{flushleft}


\begin{flushleft}
faculty members in the class committee would be referred to as Faculty Mentors for the batch.
\end{flushleft}





\begin{flushleft}
(c)	
\end{flushleft}





\begin{flushleft}
A Chairperson appointed for each entry year of students by the Head of the Department shall be associated
\end{flushleft}


\begin{flushleft}
with the batch till it graduates and will provide basic guidance for formulating course plan and electives for
\end{flushleft}


\begin{flushleft}
the students of the batch.
\end{flushleft}





\begin{flushleft}
(d)	
\end{flushleft}





\begin{flushleft}
The Convenor of a class committee will be appointed in a year-specific fashion - for example, the convenor of
\end{flushleft}


\begin{flushleft}
the second year class committee would continue in the same position for 3 years, serving consecutive batches.
\end{flushleft}





\begin{flushleft}
(e)	
\end{flushleft}





\begin{flushleft}
Students can approach any class committee member for academic advice before registration. In other words,
\end{flushleft}


\begin{flushleft}
all the three members of the class committee will have the functional role of mentor and local guardian for all
\end{flushleft}


\begin{flushleft}
the students. In case of need for any exception and relaxation in rules or regulations pertaining to registration
\end{flushleft}


\begin{flushleft}
of courses, the class committee convenor will recommend and forward the request.
\end{flushleft}





\begin{flushleft}
(f)	
\end{flushleft}





\begin{flushleft}
The faculty members of the committee in consultation with the elected representatives of the students will
\end{flushleft}


\begin{flushleft}
provide academic advice applicable to all the students in general. The class committee is also expected to
\end{flushleft}


\begin{flushleft}
discharge following responsibilities:
\end{flushleft}


22





\begin{flushleft}
\newpage
Courses of Study 2017-2018
\end{flushleft}





\begin{flushleft}
(i)	
\end{flushleft}





\begin{flushleft}
Considering mid-semester feed-back about courses running in the current semester
\end{flushleft}





\begin{flushleft}
(ii)	
\end{flushleft}





\begin{flushleft}
Identifying electives for the subsequent semester
\end{flushleft}





\begin{flushleft}
(iii)	 Addressing issues related to scheduling and categorisation of courses
\end{flushleft}


\begin{flushleft}
(iv)	 Organising STIC events for the batch.
\end{flushleft}


\begin{flushleft}
(g)	 The class committee convenor with the support of student coordinator will organise at least one StudentTeacher Interaction Committee (STIC) event in each semester for interaction between class committee
\end{flushleft}


\begin{flushleft}
members and all the students of the batch.
\end{flushleft}


\begin{flushleft}
(h)	
\end{flushleft}





\begin{flushleft}
The Chairman, Convenor and the other faculty members of first year class committee would be identified by
\end{flushleft}


\begin{flushleft}
the department prior to the orientation of new students. During orientation, students and their parents will be
\end{flushleft}


\begin{flushleft}
introduced to these class committee members.
\end{flushleft}





\begin{flushleft}
Advising Scheme for Academically Weak Students
\end{flushleft}


\begin{flushleft}
(a)	
\end{flushleft}





\begin{flushleft}
The students on probation in each batch will be put under a special advisor, identified by the department,
\end{flushleft}


\begin{flushleft}
who is expected to monitor the students on probation in a personalised manner. Normally, not more than 5-8
\end{flushleft}


\begin{flushleft}
students would be assigned to a special advisor. Heads of Departments will appoint special advisors at the
\end{flushleft}


\begin{flushleft}
beginning of an academic session.
\end{flushleft}





\begin{flushleft}
(b)	
\end{flushleft}





\begin{flushleft}
A meeting of the special advisors with Dean, Academics would be held at the beginning of each semester
\end{flushleft}


\begin{flushleft}
for coordination of the advising process.
\end{flushleft}





\begin{flushleft}
(c)	
\end{flushleft}





\begin{flushleft}
A student on probation is expected to be in close contact with the advisor by meeting him/her at least once
\end{flushleft}


\begin{flushleft}
every 3 weeks for the entire period during which the student continues to remain in probation. Special advisors
\end{flushleft}


\begin{flushleft}
will be invitees to the class committee meetings.
\end{flushleft}





\begin{flushleft}
(d)	
\end{flushleft}





\begin{flushleft}
Special advisor in consultation with the parents and student counsellor, if required, will make a student-specific
\end{flushleft}


\begin{flushleft}
academic plan. The special advisor is expected to:
\end{flushleft}





	





$\bullet$	





\begin{flushleft}
Closely interact with the weak student and his/her parents
\end{flushleft}





	





$\bullet$	





\begin{flushleft}
Formulate individualised academic plan	
\end{flushleft}





	





$\bullet$	





\begin{flushleft}
Manage and track counselling process of the student, if any, in coordination with the Associate Dean,
\end{flushleft}


\begin{flushleft}
Student Welfare.
\end{flushleft}





	





$\bullet$	





\begin{flushleft}
Approve their registration
\end{flushleft}





	





$\bullet$	





\begin{flushleft}
Manage the recommendation/appeal for termination/continuation process in consultation with Head of
\end{flushleft}


\begin{flushleft}
the Department and Dean, Academics.
\end{flushleft}





\begin{flushleft}
(e)	
\end{flushleft}





\begin{flushleft}
At the time of registration for a semester, the student meets his / her advisor if possible with parents, to:
\end{flushleft}


\begin{flushleft}
$\bullet$	 	 Identify specific problems and ways to mitigate the same
\end{flushleft}


\begin{flushleft}
$\bullet$	 	 Formulate academic plan and target(s) for the semester
\end{flushleft}


\begin{flushleft}
$\bullet$	 	 Help Head of the Department in the processing of the student's appeal against termination, if applicable
\end{flushleft}


\begin{flushleft}
$\bullet$	 	 Approve the registration of the student online.
\end{flushleft}





\begin{flushleft}
(f)	
\end{flushleft}





\begin{flushleft}
The student being placed under probation for the first time may also meet the counsellor during this period,
\end{flushleft}


\begin{flushleft}
if needed. The counsellor can provide professional help in identifying to resolving problems. Counsellors'
\end{flushleft}


\begin{flushleft}
input will be available to the special advisor. During the add-drop period, the student, preferably along with
\end{flushleft}


\begin{flushleft}
his/her parents, should come and meet the Counsellor.
\end{flushleft}





\begin{flushleft}
(g)	 While considering any appeal from an academically weak student for continuation of his registration, the
\end{flushleft}


\begin{flushleft}
Dean, Academics would consider the following:
\end{flushleft}


	





\begin{flushleft}
(i)	
\end{flushleft}





	





\begin{flushleft}
(ii)	 whether he/she is regular in help sessions.
\end{flushleft}





		





\begin{flushleft}
whether he/she has met his/her Advisor and Counsellor at the scheduled times on a regular basis and
\end{flushleft}


\begin{flushleft}
Registration of a student under probation will not be approved for the next semester if he/she does not
\end{flushleft}


\begin{flushleft}
comply with the process of meeting the advisor and counsellor. He/ she will then be required to withdraw
\end{flushleft}


23





\begin{flushleft}
\newpage
Courses of Study 2017-2018
\end{flushleft}





\begin{flushleft}
from the semester.
\end{flushleft}


\begin{flushleft}
(h)	
\end{flushleft}





\begin{flushleft}
A student on probation will not be permitted to contest for any position of responsibility. However, he/ she will
\end{flushleft}


\begin{flushleft}
be permitted to participate in extra-curricular activities in a restricted fashion only on specific recommendation
\end{flushleft}


\begin{flushleft}
of his / her advisor.
\end{flushleft}





\begin{flushleft}
An Institute level committee known as the Welfare Committee would monitor the entire operation of academic
\end{flushleft}


\begin{flushleft}
advising for weak students. Functions of the Welfare committee include monitoring the performance of weak students
\end{flushleft}


\begin{flushleft}
and making the final recommendations regarding termination/ continuation, restarting first year and slow-paced
\end{flushleft}


\begin{flushleft}
programme requests. This committee would also evaluate the weak students based on the feed-back regarding
\end{flushleft}


	





\begin{flushleft}
(i)	
\end{flushleft}





\begin{flushleft}
regularity in meeting the advisor and /or counsellor
\end{flushleft}





	





\begin{flushleft}
(ii)	 student's attendance in help sessions and
\end{flushleft}





	





\begin{flushleft}
(iii)	 academic performance.
\end{flushleft}





\begin{flushleft}
A summary of the weak student's performance would be made available to the class committee members, Head of
\end{flushleft}


\begin{flushleft}
the student's Department as well as Course Coordinators of the courses in which the student is currently registered.
\end{flushleft}


\begin{flushleft}
Student Mentors
\end{flushleft}


\begin{flushleft}
(a)	
\end{flushleft}





\begin{flushleft}
Each student will be assigned a student mentor from the same hostel and preferably from the same discipline
\end{flushleft}


\begin{flushleft}
to mentor students on academic and extra-curricular activities and provide feed-back to the advisor and
\end{flushleft}


\begin{flushleft}
counselor in case of weak students.
\end{flushleft}





\begin{flushleft}
(b)	
\end{flushleft}





\begin{flushleft}
There are individual incentives for good student mentors. Also, hostels performing well on mentoring benefit
\end{flushleft}


\begin{flushleft}
in terms of points towards BSW trophy.
\end{flushleft}





\begin{flushleft}
4.8	 Capability Linked Opportunities for Undergraduate Students
\end{flushleft}


\begin{flushleft}
A student who clears all the first year credit requirements with CGPA 7.0 and above will be permitted to register
\end{flushleft}


\begin{flushleft}
for additional credits from third semester onwards. A student will be permitted to register for up to 26 credits per
\end{flushleft}


\begin{flushleft}
semester provided
\end{flushleft}


	





\begin{flushleft}
(a)	 The student has cleared all courses for which the student has registered till then and
\end{flushleft}





	





\begin{flushleft}
(b)	 his / her CGPA is 7 or above
\end{flushleft}





\begin{flushleft}
In case a student does not meet this requirement but has cleared 20×N credits, where N is the total number of
\end{flushleft}


\begin{flushleft}
semesters spent, then he/she can register up to a maximum of 24 credits.
\end{flushleft}


\begin{flushleft}
A student registering for 26 credits in each semester after the end of first year can complete a maximum of 190
\end{flushleft}


\begin{flushleft}
credits at the end of 4 years. Similarly, a student registering for 24 credits in each semester after first year can
\end{flushleft}


\begin{flushleft}
complete a maximum of 178 credits. Since the graduation requirement for 4-year B.Tech programmes varies
\end{flushleft}


\begin{flushleft}
between 145-155 Earned Credits, it will be feasible for capable students to add value to their degrees by registering
\end{flushleft}


\begin{flushleft}
for additional courses of their choice.
\end{flushleft}


\begin{flushleft}
Students can make use of these additional credits in two blocks of 20 credits to opt for:
\end{flushleft}


	





\begin{flushleft}
(a)	 Minor / Interdisciplinary Area Specialization
\end{flushleft}





	





\begin{flushleft}
(b)	 Departmental Specialization
\end{flushleft}





\begin{flushleft}
A student based on his / her performance and interest can choose either one on both. Successful completion of
\end{flushleft}


\begin{flushleft}
minor area credits and / or departmental Specialization will be indicated on the degree.
\end{flushleft}


\begin{flushleft}
When a student opts for a departmental specialization and / or a minor area, he / she can use 10 open category
\end{flushleft}


\begin{flushleft}
credits (mandatory degree requirement) towards departmental specialization and/or minor area requirements. For
\end{flushleft}


\begin{flushleft}
example, a student registered for B.Tech (Chemical engg.) and opting for minor area in Computer Science and
\end{flushleft}


\begin{flushleft}
Engg., can opt for courses prescribed for minor area in Computer Science and Engg., as part of mandatory 10
\end{flushleft}


\begin{flushleft}
credits requirements under OC. He / she will need to do additional 10 credits in the minor area to be eligible for
\end{flushleft}


\begin{flushleft}
Minor area specialization in the degree.
\end{flushleft}


\begin{flushleft}
A student may not opt for either of the two but can do additional credits through open choice of courses. In case
\end{flushleft}


\begin{flushleft}
a student cannot meet requirements of a minor area or departmental Specialization, additional credits earned by
\end{flushleft}


\begin{flushleft}
the student over and above his / her degree requirement will be used for DGPA calculation and will be indicated
\end{flushleft}


\begin{flushleft}
on his/her transcript.
\end{flushleft}


\begin{flushleft}
A set of pre-defined courses of total 20 credits in a focus area comprises a Departmental Specialization if the
\end{flushleft}


\begin{flushleft}
courses belong to the parent Department of an undergraduate programme, or a Minor/ Interdisciplinary Area
\end{flushleft}


24





\begin{flushleft}
\newpage
Courses of Study 2017-2018
\end{flushleft}





\begin{flushleft}
Specialization if the courses belong to a different Department / Centre / School. Additional conditions and details of
\end{flushleft}


\begin{flushleft}
individual specializations are given in Section 7.
\end{flushleft}


\begin{flushleft}
If any course of a Minor / Interdisciplinary area overlaps with any core course (DC or PC category courses) or
\end{flushleft}


\begin{flushleft}
elective course (DE or PE category courses) of the student's programme, then credits from this course will not count
\end{flushleft}


\begin{flushleft}
towards the minor area credit requirements, though this course may contribute towards satisfying the requirement
\end{flushleft}


\begin{flushleft}
of the Minor / Interdisciplinary area. In such a case, the requirement of 20 credits must be completed by taking
\end{flushleft}


\begin{flushleft}
other courses of the specialization.
\end{flushleft}





\begin{flushleft}
4.9	 Change of Programme at the End of the First Year
\end{flushleft}


\begin{flushleft}
(a)	
\end{flushleft}





\begin{flushleft}
An undergraduate student is eligible to apply for change of branch at the end of the first year only, provided
\end{flushleft}


\begin{flushleft}
he / she satisfies the following criteria:-
\end{flushleft}





	





\begin{flushleft}
(i)	
\end{flushleft}





	





\begin{flushleft}
(ii)	 CGPA for SC / ST and Person with Disability category students	 :	 $>$7.00
\end{flushleft}





	





\begin{flushleft}
(iii)	 Earned credits / non-graded units at the end	
\end{flushleft}


\begin{flushleft}
:	 All credits of core and non-graded
\end{flushleft}


\begin{flushleft}
of the second semester of the first year		 units of the first year
\end{flushleft}





\begin{flushleft}
CGPA for General and OBC category students	
\end{flushleft}





:	 $>$8.00





\begin{flushleft}
	(iv)	
\end{flushleft}


\begin{flushleft}
Optionally, one first year course would be identified by each programme, in which the grade of the
\end{flushleft}


\begin{flushleft}
applicant is equal to or above B. A list of such courses identified for various programmes is given in Table 9.
\end{flushleft}


\begin{flushleft}
(b)	
\end{flushleft}





\begin{flushleft}
The student should have no disciplinary action against him/her.
\end{flushleft}





\begin{flushleft}
(c)	
\end{flushleft}





\begin{flushleft}
Change of the branch will be permitted strictly in the order of merit, in each category, as determined by CGPA
\end{flushleft}


\begin{flushleft}
at the end of first year, subject to the limitation that the actual number of students in the third semester in the
\end{flushleft}


\begin{flushleft}
branch to which transfer is to be made should not exceed its sanctioned strength by more than 15\% and the
\end{flushleft}


\begin{flushleft}
strength of the branch from which transfer is being sought does not fall below 85\% of its sanctioned strength.
\end{flushleft}





\begin{flushleft}
(d)	
\end{flushleft}





\begin{flushleft}
In case more than one student applying for programme change have the same CGPA, the tie shall be resolved
\end{flushleft}


\begin{flushleft}
on the basis of JEE ranks of such applicants.
\end{flushleft}





\begin{flushleft}
(e)	
\end{flushleft}





\begin{flushleft}
The conditions mentioned in item (a) above will not be insisted upon for change to a branch in which a vacancy
\end{flushleft}


\begin{flushleft}
exists with reference to the sanctioned strengths, and the concerned student was eligible as per JEE Rank for
\end{flushleft}


\begin{flushleft}
admission to that branch at the time of entry to IIT Delhi. However, these conditions will continue to apply in
\end{flushleft}


\begin{flushleft}
case of students seeking change to a branch to which the concerned student was not eligible for admission
\end{flushleft}


\begin{flushleft}
at the time of entry to IIT Delhi.
\end{flushleft}


\begin{flushleft}
Table 9 : Qualifying criterion as per a(iv) for change of branch
\end{flushleft}





\begin{flushleft}
S. No.
\end{flushleft}





\begin{flushleft}
Programme Code and Name of the Programme
\end{flushleft}


\begin{flushleft}
to which Change is sought
\end{flushleft}





\begin{flushleft}
Specified Course in which a
\end{flushleft}


\begin{flushleft}
minimum of B grade is required
\end{flushleft}





1





\begin{flushleft}
BB1
\end{flushleft}





\begin{flushleft}
B.Tech. in Biochemical Engineering and
\end{flushleft}


\begin{flushleft}
Biotechnology
\end{flushleft}





\begin{flushleft}
CML100: Introduction to Chemistry
\end{flushleft}





2





\begin{flushleft}
BB5
\end{flushleft}





\begin{flushleft}
B.Tech. and M.Tech in Biochemical Engineering and
\end{flushleft}


\begin{flushleft}
Biotechnology
\end{flushleft}





\begin{flushleft}
CML100: Introduction to Chemistry
\end{flushleft}





3





\begin{flushleft}
CH1
\end{flushleft}





\begin{flushleft}
B.Tech in Chemical Engineering
\end{flushleft}





\begin{flushleft}
MTL101: Linear Algebra and Differential Equations
\end{flushleft}





4





\begin{flushleft}
CH7
\end{flushleft}





\begin{flushleft}
B.Tech. and M.Tech in Chemical Engineering
\end{flushleft}





\begin{flushleft}
MTL101: Linear Algebra and Differential Equations
\end{flushleft}





5





\begin{flushleft}
CE1
\end{flushleft}





\begin{flushleft}
B.Tech in Civil Engineering
\end{flushleft}





\begin{flushleft}
APL100: Engineering Mechanics
\end{flushleft}





6





\begin{flushleft}
CS1
\end{flushleft}





\begin{flushleft}
B.Tech. in Computer Science and Engineering
\end{flushleft}





\begin{flushleft}
COL100: Introduction to Computer
\end{flushleft}


\begin{flushleft}
Science
\end{flushleft}





25





\begin{flushleft}
\newpage
Courses of Study 2017-2018
\end{flushleft}





7





\begin{flushleft}
CS5
\end{flushleft}





\begin{flushleft}
B.Tech. and M.Tech in Computer Science and
\end{flushleft}


\begin{flushleft}
Engineering
\end{flushleft}





\begin{flushleft}
COL100: Introduction to Computer
\end{flushleft}


\begin{flushleft}
Science
\end{flushleft}





8





\begin{flushleft}
EE1
\end{flushleft}





\begin{flushleft}
B.Tech. in Electrical Engineering
\end{flushleft}





\begin{flushleft}
None
\end{flushleft}





9





\begin{flushleft}
EE3
\end{flushleft}





\begin{flushleft}
B.Tech. in Electrical Engineering (Power and
\end{flushleft}


\begin{flushleft}
Automation)
\end{flushleft}





\begin{flushleft}
None
\end{flushleft}





10





\begin{flushleft}
MT1
\end{flushleft}





\begin{flushleft}
B.Tech. in Mathematics and Computing
\end{flushleft}





\begin{flushleft}
MTL100: Calculus
\end{flushleft}





11





\begin{flushleft}
MT6
\end{flushleft}





\begin{flushleft}
B.Tech. and M.Tech. in Mathematics and Computing MTL100: Calculus
\end{flushleft}





12





\begin{flushleft}
ME1
\end{flushleft}





\begin{flushleft}
B.Tech. in Mechanical Engineering
\end{flushleft}





\begin{flushleft}
None
\end{flushleft}





13





\begin{flushleft}
ME2
\end{flushleft}





\begin{flushleft}
B.Tech. in Production and Industrial Engineering
\end{flushleft}





\begin{flushleft}
None
\end{flushleft}





14





\begin{flushleft}
PH1
\end{flushleft}





\begin{flushleft}
B.Tech. in Engineering Physics
\end{flushleft}





\begin{flushleft}
PYL100: Electromagnetic Waves and
\end{flushleft}


\begin{flushleft}
Quantum Mechanics
\end{flushleft}





15





\begin{flushleft}
TT1
\end{flushleft}





\begin{flushleft}
B.Tech. in Textile Technology
\end{flushleft}





\begin{flushleft}
APL100: Engineering Mechanics
\end{flushleft}





\begin{flushleft}
4.10  Self-study Course
\end{flushleft}


\begin{flushleft}
A self-study course will be from the regular UG courses listed in this document (Section 10). The main features of
\end{flushleft}


\begin{flushleft}
a self-study course are as follows:
\end{flushleft}


\begin{flushleft}
(a)	
\end{flushleft}





\begin{flushleft}
A student may be given a self-study course not exceeding 5 credits in the final semester if he / she is short
\end{flushleft}


\begin{flushleft}
by a maximum of 5 earned credits required for graduation provided that the course is not running in that
\end{flushleft}


\begin{flushleft}
semester as a regular course. Students in the Dual-degree programmes are allowed to avail of this provision
\end{flushleft}


\begin{flushleft}
during their last semester. However, they would be permitted to take only a UG course as a possible selfstudy course. A student can make use of this provision only once during the programme.
\end{flushleft}





\begin{flushleft}
(b)	
\end{flushleft}





\begin{flushleft}
A student may also be permitted to do a U.G. core course not exceeding 5 credits in self-study mode at most
\end{flushleft}


\begin{flushleft}
once during the program, provided he / she has failed in it earlier and the course is not being offered as a
\end{flushleft}


\begin{flushleft}
regular course during that semester.
\end{flushleft}





\begin{flushleft}
(c)	
\end{flushleft}





\begin{flushleft}
Students should apply for a self-study course with appropriate recommendation of a Course Coordinator
\end{flushleft}


\begin{flushleft}
and the Head of the Department of the student's programme. The final sanction of a self-study course to a
\end{flushleft}


\begin{flushleft}
student is made by the Dean, Academics.
\end{flushleft}





\begin{flushleft}
(d)	 Normally, no formal lectures will be held for a self-study course but laboratory, design and computation
\end{flushleft}


\begin{flushleft}
exercises will be conducted if they form an integral part of the course.
\end{flushleft}


\begin{flushleft}
(e)	 The Course Coordinator will hold minor and major tests besides other tests/quizzes for giving his/her
\end{flushleft}


\begin{flushleft}
assessment at the end of the semester. In summer semester, there will be at least one mid semester test
\end{flushleft}


\begin{flushleft}
and a major test.
\end{flushleft}


\begin{flushleft}
(f)	
\end{flushleft}





\begin{flushleft}
The self-study course will run during the total duration of the semester (including summer semester).
\end{flushleft}





\begin{flushleft}
4.11  Assistantship for Dual-degree Programmes
\end{flushleft}


\begin{flushleft}
The students of dual-degree programmes will be considered for award of institute research/ teaching assistantship
\end{flushleft}


\begin{flushleft}
if they have earned 135 credits. Only those students who have either qualified GATE or have a CGPA more than
\end{flushleft}


\begin{flushleft}
8.0 will be eligible for this assistantship. The assistantship will be provided for a maximum period of 14 months
\end{flushleft}


\begin{flushleft}
beginning from the summer semester following eighth semester, provided the student is registered for M.Tech
\end{flushleft}


\begin{flushleft}
Major Project in that semester. A student availing assistantship will be required to provide 8 hours of assistance
\end{flushleft}


\begin{flushleft}
per week besides his/ her normal academic work. For continuation of assistantship a student will need to secure
\end{flushleft}


\begin{flushleft}
SGPA of 7.0. A student will be eligible to receive assistantship from sources other than institute fund or MHRD if
\end{flushleft}


\begin{flushleft}
he/she has a CGPA of 7.0 and has earned 135 credits.
\end{flushleft}


\begin{flushleft}
A student receiving assistantship will be eligible for total of 30 days leave during the 14-month period. He/she will
\end{flushleft}


\begin{flushleft}
not be entitled to mid-semester breaks, winter and summer vacations.
\end{flushleft}





26





\begin{flushleft}
\newpage
Courses of Study 2017-2018
\end{flushleft}





5.	





\begin{flushleft}
POSTGRADUATE DEGREE REQUIREMENTS, REGULATIONS AND PROCEDURES
\end{flushleft}





\begin{flushleft}
5.1	 Degree requirements
\end{flushleft}


\begin{flushleft}
The detailed degree requirements for M.Sc., M.B.A., M.Des. M.Tech., M.S. (Research) and Ph.D. degrees and
\end{flushleft}


\begin{flushleft}
D.I.I.T. are listed in Table 10.
\end{flushleft}





\begin{flushleft}
5.2	 Continuation Requirements
\end{flushleft}


\begin{flushleft}
The detailed requirements for continuation as a student in the respective programme for M.Sc., D.I.I.T., M.B.A.,
\end{flushleft}


\begin{flushleft}
M.Des. M.Tech., M.S. (Research) and Ph.D. degrees and D.I.I.T. are listed in Table 10. Failure to maintain the
\end{flushleft}


\begin{flushleft}
specified academic standing will result in termination of registration and the student's name will be struck-off the rolls.
\end{flushleft}


\begin{flushleft}
The maximum permitted duration of each programme will be determined in terms of number of registered semesters.
\end{flushleft}


\begin{flushleft}
Any semester in which a student has registered for a course will be called a registered semester subject to the
\end{flushleft}


\begin{flushleft}
following:
\end{flushleft}


\begin{flushleft}
(a)	 Only the 1st and 2nd semesters of an academic year can be registered semesters. The summer semester
\end{flushleft}


\begin{flushleft}
will not be considered as a registered semester.
\end{flushleft}


\begin{flushleft}
(b)	 A semester when a student has been granted semester withdrawal or granted leave will not be considered
\end{flushleft}


\begin{flushleft}
as a registered semester.
\end{flushleft}


\begin{flushleft}
(c)	 The semester when a student is suspended from the Institute on disciplinary grounds will not be counted
\end{flushleft}


\begin{flushleft}
towards the number of registered semesters.
\end{flushleft}


\begin{flushleft}
The summer semesters falling in between the permitted registered semesters shall be available for earning credits.
\end{flushleft}


\begin{flushleft}
After the student has registered for the maximum permissible number of registered semesters, the subsequent
\end{flushleft}


\begin{flushleft}
summer semesters will not be available for earning credits.
\end{flushleft}





\begin{flushleft}
5.3	 Minimum Student Registration for a Programme
\end{flushleft}


\begin{flushleft}
A M.Sc., M.B.A., M.Des. or M.Tech. programme will not be run unless the number of students registered for
\end{flushleft}


\begin{flushleft}
that programme is six or more. If the number of students left in a programme at the end of the 2nd semester is
\end{flushleft}


\begin{flushleft}
less than four, the same programme may be looked into for temporary suspension by the Board of Educational
\end{flushleft}


\begin{flushleft}
Research and Planning.
\end{flushleft}





\begin{flushleft}
5.4	 Lower and Upper Limits for Credits Registered
\end{flushleft}


\begin{flushleft}
For students pursuing M.Sc., M.B.A., M.Tech. and M.S.(Research), the minimum registration requirement in a
\end{flushleft}


\begin{flushleft}
semester are specified in Table 10. These minimum credit requirements are not applicable for graduating students
\end{flushleft}


\begin{flushleft}
who require lower than the proposed minimum to graduate.
\end{flushleft}





\begin{flushleft}
5.5	 Audit Courses for PG Students
\end{flushleft}


\begin{flushleft}
(a)	M.Tech. / M.S.(R) / M.Sc. /  Ph.D. students are eligible for auditing a course at any time before completion of
\end{flushleft}


\begin{flushleft}
the programme.
\end{flushleft}


\begin{flushleft}
(b)	 A student can request for an audit grade in any course provided he / she is eligible to earn audit credits,
\end{flushleft}


\begin{flushleft}
he / she is already registered for that course and it is not a core requirement of the student's programme.
\end{flushleft}


\begin{flushleft}
The request for auditing a course should be made on or before the last date for audit requests as defined
\end{flushleft}


\begin{flushleft}
in the semester schedule.
\end{flushleft}


\begin{flushleft}
(c)	 A student earns either an NP (audit pass) or an NF (audit fail) grade for an audit course. The audit pass (NP)
\end{flushleft}


\begin{flushleft}
grade may be awarded if the student satisfies the attendance criteria specified for the course and he / she
\end{flushleft}


\begin{flushleft}
has obtained at least a {`}D' grade. The course coordinator can specify a higher criterion for audit pass at the
\end{flushleft}


\begin{flushleft}
beginning of the semester. If either of these requirements is not fulfilled, the audit fail (NF) grade is awarded.
\end{flushleft}


\begin{flushleft}
(d)	 Grades obtained in an audit course are not considered in the calculation of SGPA or CGPA.
\end{flushleft}


\begin{flushleft}
(e)	 M.Tech., M.Sc., M.S.(R) and Ph.D. students can audit a course over and above their credit requirements, as
\end{flushleft}


\begin{flushleft}
specified by the supervisor and SRC. Audited credits do not count for graduation requirements of PG students.
\end{flushleft}


27





\begin{flushleft}
\newpage
Courses of Study 2017-2018
\end{flushleft}





\begin{flushleft}
(f)	 Non-credit core courses or core courses not considered for calculation of SGPA or CGPA for PG programmes
\end{flushleft}


\begin{flushleft}
like Ph.D., MBA, M.Tech., M.S. (R) should not be referred to as audit courses. These courses should be treated
\end{flushleft}


\begin{flushleft}
like similar core requirements for UG programmes such as Introduction to Programme. A student can earn
\end{flushleft}


\begin{flushleft}
either a S or Z grade in such courses. The grade S indicates successful completion. A student has to earn a
\end{flushleft}


\begin{flushleft}
S grade in such a course to meet the core requirements of a programme.
\end{flushleft}





\begin{flushleft}
5.6	 Award of D.I.I.T. to M.Tech./MBA Students
\end{flushleft}


\begin{flushleft}
If a student after completing the maximum period available for the M.Tech. programme is not able to get the required
\end{flushleft}


\begin{flushleft}
minimum DGPA of 6.0 with the minimum required credits for the respective programme, then he / she can apply
\end{flushleft}


\begin{flushleft}
for a D.I.I.T. irrespective of whether the department/centre runs a Diploma programme or not. For the award of
\end{flushleft}


\begin{flushleft}
D.I.I.T., the student must have earned a minimum of 36 valid credits with a minimum CGPA of 5.5. The request for
\end{flushleft}


\begin{flushleft}
the award of DIIT must be made within 5 years of the date of joining the programme.
\end{flushleft}


\begin{flushleft}
In case of M.B.A., DIIT shall be considered if at least 36 credits (9 courses from core and 3 courses from focus
\end{flushleft}


\begin{flushleft}
module) +4 compulsory audit courses, have been completed satisfactorily with a minimum CGPA of 5.5.
\end{flushleft}





\begin{flushleft}
5.7	 Regulations for Part-time Students
\end{flushleft}


\begin{flushleft}
Normally, part-time M.Tech. and M.S. (Research) students are expected to complete the degree requirements in six
\end{flushleft}


\begin{flushleft}
semesters. In case of special circumstances, including extension of project work, the student can be allowed to continue
\end{flushleft}


\begin{flushleft}
beyond six semesters but in any case he/she cannot extend registration beyond ten semesters excluding summer
\end{flushleft}


\begin{flushleft}
semesters. In case of full-time students converting to part-time registration, the limit of six semesters will continue to apply.
\end{flushleft}





\begin{flushleft}
5.8	 Leave rules for D.I.I.T., M.Des., M.Tech. and M.S. (Research)
\end{flushleft}


\begin{flushleft}
A full-time D.I.I.T., M.Des., M.Tech. or M.S. (Research) student during his / her stay at the Institute will be entitled to
\end{flushleft}


\begin{flushleft}
leave for 30 days (including leave on medical grounds), per academic year. Even during mid-semester breaks, and
\end{flushleft}


\begin{flushleft}
summer and winter vacations, he/she will have to explicitly apply for leave. He / she, however, may be permitted to
\end{flushleft}


\begin{flushleft}
avail of leave only up to 15 days during winter vacation at the end of the first semester.
\end{flushleft}


\begin{flushleft}
The leave will be subject to approval of the Head of Department / Centre / Programme / School Coordinator concerned
\end{flushleft}


\begin{flushleft}
and a proper leave account of each student shall be maintained by the Department / Centre / Programme / School
\end{flushleft}


\begin{flushleft}
Coordinator concerned.
\end{flushleft}





\begin{flushleft}
5.9	 Assistantship requirements
\end{flushleft}


\begin{flushleft}
A D.I.I.T., M.Des., M.Tech. or M.S. (Research) student irrespective of the source of assistantship, must attend at
\end{flushleft}


\begin{flushleft}
least 75 \% of classes in each course in which he / she is registered. In case his/her attendance falls below 75 \% in
\end{flushleft}


\begin{flushleft}
any course during a month, he/she will not be paid assistantship for that month. Further, if his / her attendance again
\end{flushleft}


\begin{flushleft}
falls short of 75 \% in any course in any subsequent month in that semester, his/her studentship and assistantship
\end{flushleft}


\begin{flushleft}
will be terminated. For the above purpose, if 75 \% works out to be a number which is not a whole number, the
\end{flushleft}


\begin{flushleft}
immediate lower whole number will be treated as the required 75 \% attendance.
\end{flushleft}


\begin{flushleft}
The students are expected to put in 8 hours per week towards the work assigned by the Institute. Continuation
\end{flushleft}


\begin{flushleft}
of assistantship in a subsequent semester would be conditional to satisfactory performance of the assigned
\end{flushleft}


\begin{flushleft}
work and a SGPA of 7.0 or more (relaxed to 6.75 for SC / ST and PH students registered in M.Des., M.Tech. and
\end{flushleft}


\begin{flushleft}
M.S. (Research) programmes).
\end{flushleft}





\begin{flushleft}
5.10 Summer registration
\end{flushleft}


\begin{flushleft}
Summer semester registration for PG students is admissible. M.Tech. / M.S. (R) / M.Des. students will be
\end{flushleft}


\begin{flushleft}
allowed to register for maximum of one course (upto 4 Credits) and M.B.A. / M.Sc. students upto 2 courses in
\end{flushleft}


\begin{flushleft}
the summer. Summer semester registration for PG students is permitted only when a student would graduate
\end{flushleft}


\begin{flushleft}
on completion of the courses registered in summer, and it is recommended by DRC / CRC. For projects, in
\end{flushleft}


\begin{flushleft}
case X grade is awarded in the second semester, the student would be expected to register during summer for
\end{flushleft}


\begin{flushleft}
completion of the project. Normally regular courses would not be offered during summer semester. Courses
\end{flushleft}


\begin{flushleft}
can, however, be offered by departments/centres/Schools for taking care of special situations subject to the
\end{flushleft}


\begin{flushleft}
availability and consent of faculty.
\end{flushleft}


28





\begin{flushleft}
\newpage
Courses of Study 2017-2018
\end{flushleft}





\begin{flushleft}
5.11 Master of Science (Research) Regulations
\end{flushleft}


\begin{flushleft}
The M.S. (Research) programme will comprise of 15 credits of the course work and 36 credits of the research
\end{flushleft}


\begin{flushleft}
work. The 15 credits of course work should not include any component of minor project. In the first semester
\end{flushleft}


\begin{flushleft}
the student has to register for a minimum of 09 and a maximum of 15 credits. In the first semester, the part-time
\end{flushleft}


\begin{flushleft}
students can only register for course work with minimum and maximum limits of 3 and 12 credits, respectively.
\end{flushleft}


\begin{flushleft}
The course work must be completed by the end of third semester; otherwise the registration of the student will
\end{flushleft}


\begin{flushleft}
stand cancelled.
\end{flushleft}


\begin{flushleft}
The larger project component gives the student an opportunity to conduct in-depth investigation on a topic of
\end{flushleft}


\begin{flushleft}
his / her interest. The project will be monitored by the Student Research Committee (SRC) and the students
\end{flushleft}


\begin{flushleft}
will have to register for thesis (project course no. xxD895, {`}xx' is department / school code) for 36 credits. An {`}X'
\end{flushleft}


\begin{flushleft}
grade is awarded at the end of each semester until the project work gets completed and the thesis is written.
\end{flushleft}


\begin{flushleft}
Nominally the M.S.(R) programme is expected to take 4 semesters (excluding summer). Upon completion of
\end{flushleft}


\begin{flushleft}
project work, a thesis is written that is evaluated by one internal and one external examiner. Upon satisfactory
\end{flushleft}


\begin{flushleft}
recommendations from the examiners, the thesis defence can be conducted before a committee. Conversion
\end{flushleft}


\begin{flushleft}
to Ph.D. is also possible. For further details, see the {``}Rules and Regulations for Master of Science (Research)
\end{flushleft}


\begin{flushleft}
Programme'' booklet.
\end{flushleft}





\begin{flushleft}
5.12 Migration from one PG programme to another PG Programme of the Institute
\end{flushleft}


\begin{flushleft}
Provision exists for the PG students of the Institute to move from (i) M.Tech. / M.S.(R) to Ph.D., (ii) M.Tech. to
\end{flushleft}


\begin{flushleft}
M.S.(R), and (iii) M.S.(R) to M.Tech. as per details given in the table below:
\end{flushleft}


\begin{flushleft}
M.Tech./M.S.(R) to Ph.D.
\end{flushleft}





\begin{flushleft}
M.Tech. to M.S.(R)
\end{flushleft}





\begin{flushleft}
M.S.(R) to M.Tech.
\end{flushleft}





\begin{flushleft}
$>$ 1st Sem. \& $\leq$ 3rd Sem.
\end{flushleft}





\begin{flushleft}
$>$ 1st Sem. \& $\leq$ 3rd Sem.
\end{flushleft}





\begin{flushleft}
Timing
\end{flushleft}





\begin{flushleft}
$>$ 1st Sem.
\end{flushleft}





\begin{flushleft}
Eligibility
\end{flushleft}





\begin{flushleft}
$\geq$ 8.0 SGPA / CGPA \& $\geq$ 12 credits $\geq$ 12 credits
\end{flushleft}





\begin{flushleft}
$\geq$ 12 credits
\end{flushleft}





\begin{flushleft}
Admission
\end{flushleft}





\begin{flushleft}
DRC / CRC (Evaluation)
\end{flushleft}





\begin{flushleft}
DRC / CRC (Evaluation)
\end{flushleft}





\begin{flushleft}
Credits
\end{flushleft}





\begin{flushleft}
Credits transfer as recommended Credits transfer as
\end{flushleft}


\begin{flushleft}
by DRC / CRC
\end{flushleft}


\begin{flushleft}
recommended by DRC / CRC
\end{flushleft}





\begin{flushleft}
Credits transfer as
\end{flushleft}


\begin{flushleft}
recommended by DRC / CRC
\end{flushleft}





\begin{flushleft}
Duration
\end{flushleft}





\begin{flushleft}
Max. 7 years from date of joining
\end{flushleft}


\begin{flushleft}
M.Tech. / M.S.(R)
\end{flushleft}





\begin{flushleft}
Max. 5 years from date of
\end{flushleft}


\begin{flushleft}
joining M.S. (R)
\end{flushleft}





\begin{flushleft}
DRC / CRC (Evaluation)
\end{flushleft}





\begin{flushleft}
Max. 5 years from date of
\end{flushleft}


\begin{flushleft}
joining M.Tech.
\end{flushleft}





\begin{flushleft}
Full-time M.Tech. and M.S.(R) students of IIT Delhi interested in joining the Ph.D. programme within two years of
\end{flushleft}


\begin{flushleft}
completion of their M.Tech. / M.S.(R) will be granted waiver of residency period. The course work requirements be
\end{flushleft}


\begin{flushleft}
made up by either additional credits (6 credits as per present norms) taken during their M.Tech. / M.S.(R) period
\end{flushleft}


\begin{flushleft}
(over and above their minimum Degree requirements) or in the summer semester (first or second) by identifying
\end{flushleft}


\begin{flushleft}
courses. In all cases, the request for such credit transfer be recommended by the concerned DRC / CRC / SRC as
\end{flushleft}


\begin{flushleft}
relevant to their respective Ph.D. programmes.
\end{flushleft}





\begin{flushleft}
5.13 Doctor of Philosophy (Ph.D.) Regulations
\end{flushleft}


\begin{flushleft}
The award of Ph.D. degree is in recognition of high achievements, independent research and application of
\end{flushleft}


\begin{flushleft}
scientific knowledge to the solution of technical and scientific problems. Creative and productive enquiry is the
\end{flushleft}


\begin{flushleft}
basic concept underlying the research work. In order to overcome any deficiency in the breadth of fundamental
\end{flushleft}


\begin{flushleft}
training or proper foundation for advanced work, special preliminary or pre-Ph.D. courses are given by each
\end{flushleft}


\begin{flushleft}
Department / Centre / School. These courses are given either by faculty members or by guest-speakers and
\end{flushleft}


\begin{flushleft}
specialists in the field of research.
\end{flushleft}


\begin{flushleft}
5.13.1 Course requirements
\end{flushleft}


\begin{flushleft}
Candidates admitted to non-engineering departments and having a B.Tech. / M.Sc. / M.A. or equivalent degree are
\end{flushleft}


\begin{flushleft}
required to complete a minimum of 12 credits. Relaxation up to 6 credits in the course work can be considered
\end{flushleft}


\begin{flushleft}
for those with M.Phil. degree. The requirement of pre-Ph.D. Course Credits / work for Ph.D. student admitted to
\end{flushleft}


\begin{flushleft}
engineering department and having a B.Tech. and M.Sc. Degree is 20 credits. Individual Academic Unit may
\end{flushleft}


\begin{flushleft}
recommend minimum course work requirements beyond the minimum requirements specified by the Institute, with
\end{flushleft}


\begin{flushleft}
details as described below.
\end{flushleft}


29





\begin{flushleft}
\newpage
Minimum 12 credits
\end{flushleft}


\begin{flushleft}
Maximum 26 credits
\end{flushleft}





\begin{flushleft}
M.Sc., Chemistry
\end{flushleft}





30





\begin{flushleft}
Minimum 09 credits
\end{flushleft}


\begin{flushleft}
Maximum 15 credits
\end{flushleft}





\begin{flushleft}
Minimum 3 credits
\end{flushleft}


\begin{flushleft}
Maximum 15 credits
\end{flushleft}





\begin{flushleft}
Minimum 09 credits
\end{flushleft}


\begin{flushleft}
Maximum 15 credits
\end{flushleft}





\begin{flushleft}
Same as M.Tech. full time
\end{flushleft}





\begin{flushleft}
Same as M.Tech. part time
\end{flushleft}





\begin{flushleft}
M.Tech., Full Time
\end{flushleft}





\begin{flushleft}
M.Tech., Part Time
\end{flushleft}





\begin{flushleft}
M. Des.
\end{flushleft}





\begin{flushleft}
M.B.A., Full Time
\end{flushleft}





\begin{flushleft}
M.B.A., Part Time
\end{flushleft}





\begin{flushleft}
M.Sc., Physics
\end{flushleft}





\begin{flushleft}
M.Sc., Mathematics
\end{flushleft}





\begin{flushleft}
Minimum 12 credits
\end{flushleft}


\begin{flushleft}
Maximum 20 credits
\end{flushleft}





\begin{flushleft}
Registration limits
\end{flushleft}


\begin{flushleft}
(Per semester)
\end{flushleft}





\begin{flushleft}
D.I.I.T.
\end{flushleft}


\begin{flushleft}
(Naval Construction)
\end{flushleft}





\begin{flushleft}
Degree
\end{flushleft}





\begin{flushleft}
(iv)	 The registration of any student shall be limited to 1.25 times the average earned credits
\end{flushleft}


\begin{flushleft}
of the previous two registered semesters, subject to a minimum of 09 credits and a
\end{flushleft}


\begin{flushleft}
maximum of 15 credits for full time students.
\end{flushleft}





\begin{flushleft}
(iii)	 If a student is on probation and his/her academic performance is below the minimum
\end{flushleft}


\begin{flushleft}
acceptable level in the following registered semester then his / her registration will be
\end{flushleft}


\begin{flushleft}
terminated.
\end{flushleft}





\begin{flushleft}
(ii)	 If at the end of any registered semester the SGPA is less than 6.0, then the student will
\end{flushleft}


\begin{flushleft}
be issued a warning letter and placed on probation; a copy of the warning letter will be
\end{flushleft}


\begin{flushleft}
sent to Chairperson DRC / CRC. The Chairperson DRC / CRC shall assess the feasibility
\end{flushleft}


\begin{flushleft}
of completing degree requirements and identify remedial measures for problems leading
\end{flushleft}


\begin{flushleft}
to poor performance.
\end{flushleft}





\begin{flushleft}
(i)	 The minimum acceptable performance level in any registered semester is SGPA of
\end{flushleft}


\begin{flushleft}
6.0. However, at the end of the 1st registered semester, a student with SGPA of 5.0 or
\end{flushleft}


\begin{flushleft}
more will be permitted to continue. If the SGPA is less than 5.0 then registration will be
\end{flushleft}


\begin{flushleft}
terminated.
\end{flushleft}





\begin{flushleft}
(iv)	 The registration of any student will be limited to 1.25 times the average earned credits
\end{flushleft}


\begin{flushleft}
of the previous two registered semesters, subject to a minimum of 15 credits and a
\end{flushleft}


\begin{flushleft}
maximum of 26 credits.
\end{flushleft}





\begin{flushleft}
(iii)	 If a student is on probation and his/her academic performance is below the minimum
\end{flushleft}


\begin{flushleft}
acceptable level in the following registered semester then his/her registration will be
\end{flushleft}


\begin{flushleft}
terminated.
\end{flushleft}





\begin{flushleft}
(ii)	 If at the end of any registered semester, the SGPA is less than 5.0 then the student
\end{flushleft}


\begin{flushleft}
will be issued a warning letter and placed on probation; a copy of the warning letter
\end{flushleft}


\begin{flushleft}
will be sent to the parents. The Chairperson DRC / CRC shall assess the feasibility of
\end{flushleft}


\begin{flushleft}
completing degree requirements and identify remedial measures for problems leading
\end{flushleft}


\begin{flushleft}
to poor performance.
\end{flushleft}





\begin{flushleft}
(i)	 The minimum acceptable performance level in any registered semester is SGPA of
\end{flushleft}


\begin{flushleft}
5.0. However, at the end of the 1st registered semester, a student with SGPA of 4.0 or
\end{flushleft}


\begin{flushleft}
more will be permitted to continue. If the SGPA is less than 4.0 then registration will be
\end{flushleft}


\begin{flushleft}
terminated.
\end{flushleft}





\begin{flushleft}
CGPA $>$ 5.0 at the end of each semester.
\end{flushleft}





\begin{flushleft}
Criteria for continuation of registration
\end{flushleft}





72 (+ 6


\begin{flushleft}
compulsory
\end{flushleft}


\begin{flushleft}
audit courses)
\end{flushleft}





54





\begin{flushleft}
48-54 credits
\end{flushleft}





75-81





49





\begin{flushleft}
Valid Credits
\end{flushleft}


(\$)





6.0





6.0





6.0





5.0





6.0





\begin{flushleft}
Minimum
\end{flushleft}


\begin{flushleft}
DGPA
\end{flushleft}





\begin{flushleft}
10 sem. @
\end{flushleft}





\begin{flushleft}
6 sem.
\end{flushleft}





\begin{flushleft}
6 sem.
\end{flushleft}





\begin{flushleft}
10 sem. @
\end{flushleft}





\begin{flushleft}
6 sem.
\end{flushleft}





\begin{flushleft}
6 sem.
\end{flushleft}





\begin{flushleft}
6 sem. \#
\end{flushleft}





\begin{flushleft}
Max. Period
\end{flushleft}


\begin{flushleft}
of stay
\end{flushleft}





\begin{flushleft}
Graduation requirements
\end{flushleft}





\begin{flushleft}
Table 10. Continuation of Registration and Graduation Requirements for Postgraduate Programmes
\end{flushleft}





\begin{flushleft}
Courses of Study 2017-2018
\end{flushleft}





\begin{flushleft}
\newpage
See note ++
\end{flushleft}





\begin{flushleft}
For details please refer to
\end{flushleft}


\begin{flushleft}
Ph.D.
\end{flushleft}


\begin{flushleft}
Ordinances and
\end{flushleft}


\begin{flushleft}
Regulations
\end{flushleft}





\begin{flushleft}
M.S. (Res.) Part Time
\end{flushleft}





\begin{flushleft}
Ph.D.
\end{flushleft}





31





\begin{flushleft}
In the first semester the student has to register for a minimum of 9 and a maximum of 15 credits of course work only. In the subsequent 3-semesters the student shall complete the research work and
\end{flushleft}


\begin{flushleft}
the course work remaining, if any.
\end{flushleft}





\begin{flushleft}
In the first two semesters the part-time student shall register only for the course work with the minimum and maximum limits of 3-15 credits. The research work and the remaining course work, if any,
\end{flushleft}


\begin{flushleft}
shall be completed in the remaining 4 semesters. However, the course work must be completed within the first 4-semesters of registration.
\end{flushleft}





\begin{flushleft}
The 10 Semester rule for part-time M.S. (Research) students will be applicable only to those who have joined initially as part-time students. For students converting from full-time to part-time the
\end{flushleft}


\begin{flushleft}
maximum stay limit of 6 semesters will be applicable, subject to recommendations of DRC / CRC / SRC and approval by Dean, Academics.
\end{flushleft}





\begin{flushleft}
The 10 Semester rule for part-time M.Tech. students will be applicable only to those who have joined initially as part-time students. For students converting from full-time to part-time, the 	
\end{flushleft}


\begin{flushleft}
maximum stay limit of 6 semester will be applicable.
\end{flushleft}





\begin{flushleft}
The summer semester will not be considered as a registered semester.
\end{flushleft}





+	





++	





+++	





@	





\#	





\begin{flushleft}
14 sem.
\end{flushleft}





\begin{flushleft}
10 sem. +++
\end{flushleft}





\begin{flushleft}
6 sem.
\end{flushleft}





\begin{flushleft}
If a student at the end of the M.Tech. programme fails to complete required valid credits with a CGPA of 6.00 or above, he / she still can get a DIIT even though the Department / Interdisciplinary
\end{flushleft}


\begin{flushleft}
Programme does not have a regular Diploma programme provided: (i) he / she has a minimum of 45 valid credits; and (ii) he / she has secured a minimum CGPA of 5.50. The request for the award of
\end{flushleft}


\begin{flushleft}
D.I.I.T. must be made within 5 years of the date of joining the programme.
\end{flushleft}





7.5





7.0





£	





\begin{flushleft}
12 for B.Tech. / M.Sc.,
\end{flushleft}


\begin{flushleft}
6 for M.Tech. or
\end{flushleft}


\begin{flushleft}
equivalent;
\end{flushleft}


\begin{flushleft}
A Deptt. / Centre /
\end{flushleft}


\begin{flushleft}
School may prescribe
\end{flushleft}


\begin{flushleft}
additional credits
\end{flushleft}


+


\begin{flushleft}
Thesis
\end{flushleft}





\begin{flushleft}
51 including Thesis.
\end{flushleft}





\begin{flushleft}
Detailed break-up of core, elective and open category courses are given in the latter pages of this document.
\end{flushleft}





\begin{flushleft}
(iii)	 In the subsequent semesters, the student must maintain a CGPA of more than 7.0 to
\end{flushleft}


\begin{flushleft}
continue registration.
\end{flushleft}





\begin{flushleft}
(ii)	 Registration of a Ph.D. student will be terminated at the end of Ist Semester on account
\end{flushleft}


\begin{flushleft}
of performance in the course work if the SGPA is less than 6.0. In case the SGPA is
\end{flushleft}


\begin{flushleft}
equal to or more than 6.0, the student will be allowed to continue the course work even
\end{flushleft}


\begin{flushleft}
if the credit requirements as recommended by the SRC have been completed in the first
\end{flushleft}


\begin{flushleft}
semester itself.
\end{flushleft}





\begin{flushleft}
(i)	 A student will be evaluated on completion of pre-Ph.D. course work in terms of Degree
\end{flushleft}


\begin{flushleft}
Grade Point Average (DGPA) which is calculated on the basis of the best valid credits
\end{flushleft}


\begin{flushleft}
as prescribed by the Department/Centre/School. The requirement for completion of
\end{flushleft}


\begin{flushleft}
pre-Ph.D. course work is DGPA of 7.5 or more. within the maximum permissible period
\end{flushleft}


\begin{flushleft}
i.e 18 and 24 months respectively for full-time and part time students.
\end{flushleft}





\begin{flushleft}
(iv)	 During the research work period, each unsatisfactory performance grade would entail a
\end{flushleft}


\begin{flushleft}
warning and two consecutive warnings would result in termination of registration.
\end{flushleft}





\begin{flushleft}
(iii)	 If a student is on probation and his/her academic performance is below the minimum
\end{flushleft}


\begin{flushleft}
acceptable level in the following registered semester then his/her registration will be
\end{flushleft}


\begin{flushleft}
terminated.
\end{flushleft}





\begin{flushleft}
(ii)	 If at the end of any registered semester, the SGPA is less than 7.0, then the student
\end{flushleft}


\begin{flushleft}
should be issued a warning letter and placed on probation; a copy of the warning letter
\end{flushleft}


\begin{flushleft}
should be sent to the Chairperson DRC/CRC. The Chairperson DRC/CRC shall assess
\end{flushleft}


\begin{flushleft}
the feasibility of completing degree requirements and identify remedial measures for
\end{flushleft}


\begin{flushleft}
problems leading to poor performance.
\end{flushleft}





\begin{flushleft}
(i)	 The minimum acceptable performance level in any registered semester is SGPA of 7.0
\end{flushleft}


\begin{flushleft}
or more. Howevere, at the end of the 1st registered semester, a student with SGPA of
\end{flushleft}


\begin{flushleft}
6.0 or more will be permitted to continue. If the SGPA is less than 6.0 then registration
\end{flushleft}


\begin{flushleft}
will be terminated.
\end{flushleft}





\$	





\begin{flushleft}
NOTE:
\end{flushleft}





\begin{flushleft}
See note +
\end{flushleft}





\begin{flushleft}
M.S. (Res.) Full Time
\end{flushleft}





\begin{flushleft}
Courses of Study 2017-2018
\end{flushleft}





\begin{flushleft}
\newpage
Courses of Study 2017-2018
\end{flushleft}





\begin{flushleft}
M.Tech. or equivalent degree holders admitted to Ph.D. are required to complete a minimum of 6 credits. The
\end{flushleft}


\begin{flushleft}
Departments  / Centres / Schools may stipulate a larger number of credits in general or in specific cases. The course
\end{flushleft}


\begin{flushleft}
requirement will be determined by the Department / Centre / School Research Committee (DRC / CRC / SRC) on
\end{flushleft}


\begin{flushleft}
the recommendations of the supervisor after due consideration of the background of the student in relation to
\end{flushleft}


\begin{flushleft}
the proposed topic of research. These courses can be prescribed from existing M.Tech. courses and / or from
\end{flushleft}


\begin{flushleft}
special pre-Ph.D. courses including laboratory, seminar, foreign language, etc. Normally, no independent study
\end{flushleft}


\begin{flushleft}
course will be allowed for Ph.D. students.
\end{flushleft}


\begin{flushleft}
Further, in case a Ph.D. student having completed 20 credits is unable to complete the research at the Ph.D.
\end{flushleft}


\begin{flushleft}
level for any reason whatsoever, he / she may be allowed to complete M.S. (Research) degree requirement as
\end{flushleft}


\begin{flushleft}
per Institute rules.
\end{flushleft}


\begin{flushleft}
A student shall be formally registered / admitted to the candidacy of Ph.D. degree only after he / she has cleared
\end{flushleft}


\begin{flushleft}
the comprehensive examination. Students would be permitted to take the comprehensive examination only after
\end{flushleft}


\begin{flushleft}
they have submitted a research plan and have completed the course work (including compulsory audit course
\end{flushleft}


\begin{flushleft}
- HUL 810: Communication Skills). Full-time and part-time students must clear the comprehensive examination
\end{flushleft}


\begin{flushleft}
within a period of 18 months and 24 months, respectively, from the date of joining. A maximum of two chances
\end{flushleft}


\begin{flushleft}
will be given to any student to clear the comprehensive examination. Every student, after having completed the
\end{flushleft}


\begin{flushleft}
comprehensive examination must formally register for the candidacy on a form obtainable from the PG Section.
\end{flushleft}


\begin{flushleft}
5.13.2 Time limit
\end{flushleft}


\begin{flushleft}
In addition to the information in Table 10, the time limits shown in Table 11 apply for Ph.D. work.
\end{flushleft}


\begin{flushleft}
Table 11 : Time limits for students registered under Ph.D. Programme
\end{flushleft}


\begin{flushleft}
Candidate's qualification
\end{flushleft}





\begin{flushleft}
S. No.
\end{flushleft}





\begin{flushleft}
M.Tech. or equivalent
\end{flushleft}


1





\begin{flushleft}
B.Tech. / M.Sc. or equivalent
\end{flushleft}





\begin{flushleft}
Limits for Registration
\end{flushleft}





1.1





\begin{flushleft}
Minimum period of registration
\end{flushleft}





\begin{flushleft}
2 years
\end{flushleft}





\begin{flushleft}
3 years (can be reduced to
\end{flushleft}


\begin{flushleft}
2 years with the approval of
\end{flushleft}


\begin{flushleft}
Senate)
\end{flushleft}





1.2





\begin{flushleft}
Normal maximum period of registration
\end{flushleft}





\begin{flushleft}
10 Semesters
\end{flushleft}





\begin{flushleft}
10 Semesters
\end{flushleft}





1.3





\begin{flushleft}
Extended maximum period of registration
\end{flushleft}





\begin{flushleft}
14 Semesters
\end{flushleft}





\begin{flushleft}
14 Semesters
\end{flushleft}





\begin{flushleft}
Conversion from Full-time to Part-time
\end{flushleft}


\begin{flushleft}
Registration
\end{flushleft}





\begin{flushleft}
Comprehensive examination with the approval of dean
\end{flushleft}


\begin{flushleft}
academics
\end{flushleft}





2





\begin{flushleft}
5.13.3	 Leave regulations
\end{flushleft}


\begin{flushleft}
(a)	
\end{flushleft}





\begin{flushleft}
Leave during course work
\end{flushleft}





\begin{flushleft}
A full-time Ph.D. student, during his / her stay at the Institute will be entitled to leave for 30 days, including leave
\end{flushleft}


\begin{flushleft}
on medical grounds, per academic year. Even during mid-semester breaks, and summer and winter vacations,
\end{flushleft}


\begin{flushleft}
he / she will have to explicitly apply for leave. He / she, however, may be permitted to avail of leave only up to 15
\end{flushleft}


\begin{flushleft}
days during winter vacation at the end of the first semester.
\end{flushleft}


\begin{flushleft}
Leave beyond 30 days in an academic year may be granted to a research scholar in exceptional cases subject to
\end{flushleft}


\begin{flushleft}
the following conditions:
\end{flushleft}


\begin{flushleft}
(i)	
\end{flushleft}





\begin{flushleft}
the leave beyond 30 days will be without Assistantship/Scholarship, and
\end{flushleft}





\begin{flushleft}
(ii)	 such an extension of up to additional 30 days will be granted only once during the programme of the scholar.
\end{flushleft}


\begin{flushleft}
In addition, a Ph.D. student who has completed his/her course work may be granted leave on medical grounds
\end{flushleft}


\begin{flushleft}
up to 10 days per academic year.
\end{flushleft}


\begin{flushleft}
Women research scholars will be eligible for Maternity Leave with assistantship for a period not exceeding 135
\end{flushleft}


\begin{flushleft}
days once during the tenure of their Ph.D. programme.
\end{flushleft}


\begin{flushleft}
The leave may be subject to the approval of the Head of Department / Centre / School / Programme Coordinator
\end{flushleft}


\begin{flushleft}
concerned on the recommendation of the Supervisor; and a proper leave account of each research scholar shall
\end{flushleft}


\begin{flushleft}
be maintained by the Department / Centre / School / Programme Coordinator concerned.
\end{flushleft}


32





\begin{flushleft}
\newpage
Courses of Study 2017-2018
\end{flushleft}





\begin{flushleft}
5.13.4 Attendance requirements for assistantship
\end{flushleft}


\begin{flushleft}
A Ph.D. student irrespective of the source of research assistantship while pursuing course work, must attend at least
\end{flushleft}


\begin{flushleft}
75 \% of classes in each course in which he / she is registered. In case his / her attendance falls below 75 \% in any
\end{flushleft}


\begin{flushleft}
course during a month, he/she will not be paid Assistantship for that month. Further, if his / her attendance again
\end{flushleft}


\begin{flushleft}
falls short of 75 \% in any course in any subsequent month in that semester, his / her studentship and Assistantship
\end{flushleft}


\begin{flushleft}
will be terminated. A research scholar after having completed the course work must attend to his / her research work
\end{flushleft}


\begin{flushleft}
on all the working days and mark attendance except when he / she is on duly sanctioned leave. The requirement of
\end{flushleft}


\begin{flushleft}
75 \% attendance will apply as above, on daily attendance except in the cases where longer leave has been duly
\end{flushleft}


\begin{flushleft}
sanctioned within the leave entitlement of the student. For the above purpose, if 75 \% works out to be a number
\end{flushleft}


\begin{flushleft}
which is not a whole number, the immediate lower whole number will be treated as the required 75 \% attendance.
\end{flushleft}


\begin{flushleft}
All scholars who are offered assistantship are expected to put in 8 hours per week towards the work assigned by
\end{flushleft}


\begin{flushleft}
the Institute. Continuation of assistantship in the subsequent semester would be conditional, subject to satisfactory
\end{flushleft}


\begin{flushleft}
performance in the work assigned.
\end{flushleft}


\begin{flushleft}
5.13.5 Further regulations governing Ph.D. students
\end{flushleft}


\begin{flushleft}
The Ph.D. degree of the Institute may be conferred on a candidate who fulfills all the requirements detailed in the
\end{flushleft}


\begin{flushleft}
Ordinances and other rules, approved by the Senate. Some of the important regulations are given below:
\end{flushleft}


\begin{flushleft}
(i)	
\end{flushleft}





\begin{flushleft}
Applications for Ph.D. registration, i.e., for entry to a course of study and research leading to Ph.D.
\end{flushleft}


\begin{flushleft}
degree must be made to the Board of Academic Programmes (BAP) on the approved form. The date
\end{flushleft}


\begin{flushleft}
of registration is normally the date of joining the programme. However, in exceptional cases the date of
\end{flushleft}


\begin{flushleft}
registration may be preponed by a maximum of 6 months by the BAP if it is convinced that the candidate
\end{flushleft}


\begin{flushleft}
has spent adequate amount of time on research earlier.
\end{flushleft}





\begin{flushleft}
(ii)	 The academic programme of all the Ph.D. candidates in a Department / Centre / School will be coordinated
\end{flushleft}


\begin{flushleft}
by the DRC / CRC / SRC appointed by the BAP.
\end{flushleft}


\begin{flushleft}
(iii)	 The supervisor shall be a full-time member of the academic staff of the Institute. The supervisor(s) shall
\end{flushleft}


\begin{flushleft}
be appointed within three months of joining the programme. For this, Ph.D. candidates must fill up the
\end{flushleft}


\begin{flushleft}
required portion of the prescribed form, following which supervisor(s) must fill up the required portion,
\end{flushleft}


\begin{flushleft}
and the Student Research Committee (SRC) must be finalized by the respective DRC / CRC / SRC, of
\end{flushleft}


\begin{flushleft}
the Academic Unit. This process must be completed within three months of the Ph.D. candidates date
\end{flushleft}


\begin{flushleft}
of first registration. If necessary, the Board of Academic Programme on the recommendations of the
\end{flushleft}


\begin{flushleft}
Supervisor through the DRC / CRC / SRC, may appoint Joint Supervisor(s) not exceeding two from inside
\end{flushleft}


\begin{flushleft}
or outside the Institute. Normally, there should not be more than two supervisors for a candidate from
\end{flushleft}


\begin{flushleft}
within the Institute. Appointment of any Joint Supervisor would not be permitted after a lapse of eighteen
\end{flushleft}


\begin{flushleft}
months from the date of registration of the candidate, except in case when none of the supervisors is in
\end{flushleft}


\begin{flushleft}
the Institute for a year or more at a stretch.
\end{flushleft}


\begin{flushleft}
(iv)	 The DRC / CRC / SRC shall meet from time to time and review the progress of each candidate in the course
\end{flushleft}


\begin{flushleft}
work, as well as research, by any means, including oral examination of the candidate, if necessary, and
\end{flushleft}


\begin{flushleft}
recommend, after due consultation with the supervisor(s), such steps to the candidate as are necessary
\end{flushleft}


\begin{flushleft}
to improve his / her performance.
\end{flushleft}


\begin{flushleft}
(v)	 The progress of each candidate will be monitored by the DRC / CRC / SRC. For this purpose the following
\end{flushleft}


\begin{flushleft}
procedures will be followed:
\end{flushleft}


\begin{flushleft}
(a)	
\end{flushleft}





\begin{flushleft}
Ph.D. research work will be compulsorily given a course number, DTD 899 (Doctoral Thesis) for
\end{flushleft}


\begin{flushleft}
all candidates across the Institute.
\end{flushleft}





\begin{flushleft}
(b)	 The DRC / CRC / SRC Secretary / Ph.D. Coordinator will be Coordinating collection of progress
\end{flushleft}


\begin{flushleft}
reports written and signed by the scholars and forwarded by the supervisors every semester.
\end{flushleft}


\begin{flushleft}
(c)	
\end{flushleft}





\begin{flushleft}
The supervisor(s) / SRC / DRC / CRC will evaluate the progress of the student every semester.
\end{flushleft}





\begin{flushleft}
(d)	
\end{flushleft}





\begin{flushleft}
X' grade will be awarded if the progress is {`}satisfactory' in that semester.
\end{flushleft}





\begin{flushleft}
(e)	
\end{flushleft}





\begin{flushleft}
If the progress is {`}unsatisfactory', {`}U' grades will be awarded. For the first appearance of {`}U' grade,
\end{flushleft}


\begin{flushleft}
a warning would be issued to the candidate by Dean, Academics. If his / her performance does not
\end{flushleft}


\begin{flushleft}
improve after warning, the assistantship may be withheld.
\end{flushleft}





\begin{flushleft}
(f)	
\end{flushleft}





\begin{flushleft}
If there are two consecutive {`}U' grade (in consecutive semesters), the registration will stand
\end{flushleft}


\begin{flushleft}
terminated.
\end{flushleft}


33





\begin{flushleft}
\newpage
Courses of Study 2017-2018
\end{flushleft}





\begin{flushleft}
(g)	
\end{flushleft}





\begin{flushleft}
Submission of progress report should continue till submission of thesis.
\end{flushleft}





\begin{flushleft}
(h)	
\end{flushleft}





\begin{flushleft}
Like all other courses, the grades for DTD 899 will be discussed in the Department/Centre/School
\end{flushleft}


\begin{flushleft}
as per the semester schedule.
\end{flushleft}





		





\begin{flushleft}
The above process will continue till the thesis is submitted.
\end{flushleft}





\begin{flushleft}
(vi)	 The candidate may submit the thesis at any time provided that :
\end{flushleft}


\begin{flushleft}
(a)	
\end{flushleft}





\begin{flushleft}
He / she has completed the minimum period of registration including any extension prescribed by
\end{flushleft}


\begin{flushleft}
the Board of Academic Programmes (BAP).
\end{flushleft}





\begin{flushleft}
(b)	
\end{flushleft}





\begin{flushleft}
He / she has completed the course work requirement as prescribed by the DRC / CRC / SRC with
\end{flushleft}


\begin{flushleft}
DGPA not below 7.50 and has also cleared the comprehensive examination.
\end{flushleft}





\begin{flushleft}
(c)	
\end{flushleft}





\begin{flushleft}
He / she has submitted at least two months in advance, the title and a synopsis of the thesis. The
\end{flushleft}


\begin{flushleft}
Synopsis along with the list of examiners suggested by the supervisor needs to be approved by
\end{flushleft}


\begin{flushleft}
the DRC / CRC / SRC and then forwarded to Dean, Academics.
\end{flushleft}





\begin{flushleft}
(vii)	 The thesis shall normally be written in English in the specific format and shall contain a critical account
\end{flushleft}


\begin{flushleft}
of the candidate's research. It should be characterized by discovery of facts, of fresh approach towards
\end{flushleft}


\begin{flushleft}
interpretation of facts and theories or significant contribution to knowledge of design or development, or
\end{flushleft}


\begin{flushleft}
a combination of them. It should bear evidence of the candidate's capacity for analysis and judgement
\end{flushleft}


\begin{flushleft}
and also his / her ability to carry out independent investigation, design or development. A thesis should
\end{flushleft}


\begin{flushleft}
normally be supplemented by published work. No part of the thesis or supplementary published work,
\end{flushleft}


\begin{flushleft}
shall have been submitted for the award of any other Degree / Diploma. Normally, three copies of thesis
\end{flushleft}


\begin{flushleft}
in soft cover have to be submitted in the format prescribed by the Institute. In case of joint supervision,
\end{flushleft}


\begin{flushleft}
four copies of the thesis are required to be submitted.
\end{flushleft}


\begin{flushleft}
(viii)	 On receipt of the title and synopsis of a thesis, the Dean, Academics will appoint a Board of Examiners
\end{flushleft}


\begin{flushleft}
for each candidate. The Board will consist of one (or two) internal examiner(s), normally the supervisor(s),
\end{flushleft}


\begin{flushleft}
and two external examiners, one from within India and one from abroad who shall be expert in the
\end{flushleft}


\begin{flushleft}
subject of thesis. These external examiners shall be chosen from a list of eight, to be recommended by
\end{flushleft}


\begin{flushleft}
the supervisor(s) through the DRC / CRC / SRC while forwarding the title and synopsis of the thesis. The
\end{flushleft}


\begin{flushleft}
candidate will be required to submit a fresh synopsis if more than 9 months elapse from the synopsis
\end{flushleft}


\begin{flushleft}
submission date to the thesis submission date.
\end{flushleft}


\begin{flushleft}
(ix)	 Each Examiner will submit a detailed assessment report recommending to the BAP, one of the following
\end{flushleft}


\begin{flushleft}
courses of action:
\end{flushleft}


\begin{flushleft}
(a)	
\end{flushleft}





\begin{flushleft}
that the thesis be deemed satisfactory and that the candidate may defend his / her thesis orally
\end{flushleft}


\begin{flushleft}
before a committee constituted for the purpose and any members of the faculty and research
\end{flushleft}


\begin{flushleft}
students who wish to be present.
\end{flushleft}





\begin{flushleft}
(b)	
\end{flushleft}





\begin{flushleft}
that the candidate may submit a revised thesis before the expiry of a specific period. In the normal
\end{flushleft}


\begin{flushleft}
circumstances, he / she may submit the revised thesis within a period of one year from the date of
\end{flushleft}


\begin{flushleft}
communication in this regard from the Dean, Academics. However, in exceptional circumstances,
\end{flushleft}


\begin{flushleft}
this period may be extended by the BAP by another one year : the total revision time irrespective
\end{flushleft}


\begin{flushleft}
of the number of revisions allowed will not exceed a period of two years.
\end{flushleft}





\begin{flushleft}
(c)	
\end{flushleft}





\begin{flushleft}
that the thesis be rejected outright.
\end{flushleft}





		





\begin{flushleft}
In the event of disagreement between the external examiners, the BAP may, as a special case,
\end{flushleft}


\begin{flushleft}
appoint another external examiner, if the merit of the case so demands. The examiner will report
\end{flushleft}


\begin{flushleft}
independently to the BAP.
\end{flushleft}





\begin{flushleft}
(x)	 The oral defence of the thesis shall be conducted by a committee consisting of the internal examiner(s)
\end{flushleft}


\begin{flushleft}
and one external examiner. If none of the external examiners, is available for the conduct of the oral
\end{flushleft}


\begin{flushleft}
defence, an alternative external examiner shall be appointed by the BAP for this purpose only.
\end{flushleft}


\begin{flushleft}
(xi)	 On the completion of all stages of the examination, the Oral Defence Committee shall recommend to
\end{flushleft}


\begin{flushleft}
the BAP one of the following courses of action:
\end{flushleft}


\begin{flushleft}
(a)	
\end{flushleft}





\begin{flushleft}
that the degree be awarded.
\end{flushleft}





\begin{flushleft}
(b)	
\end{flushleft}





\begin{flushleft}
that the candidate should be examined on a further occasion in a manner they shall prescribe.
\end{flushleft}


34





\begin{flushleft}
\newpage
Courses of Study 2017-2018
\end{flushleft}





\begin{flushleft}
(c)	
\end{flushleft}





\begin{flushleft}
that the degree shall not be awarded.
\end{flushleft}





		





\begin{flushleft}
In the case of (a) above, the Oral Defence Committee shall also provide to the candidate a list of
\end{flushleft}


\begin{flushleft}
all corrections and modifications, if any, suggested by the examiners.
\end{flushleft}





\begin{flushleft}
(xii)	 The degree shall be awarded by the Senate, provided that:
\end{flushleft}


\begin{flushleft}
(a)	
\end{flushleft}





\begin{flushleft}
the Oral Defence Committee, through the BAP so recommends.
\end{flushleft}





\begin{flushleft}
(b)	
\end{flushleft}





\begin{flushleft}
the candidate produces a {`}no dues certificate' from all concerned in the prescribed form and gets
\end{flushleft}


\begin{flushleft}
it forwarded along with the report of the Oral Defence Committee; and
\end{flushleft}





\begin{flushleft}
(c)	
\end{flushleft}





\begin{flushleft}
the candidate has submitted two hard cover copies of the thesis, after incorporating all necessary
\end{flushleft}


\begin{flushleft}
corrections and modifications including appropriate IPR notice. The hard bound copies of the
\end{flushleft}


\begin{flushleft}
Ph.D. thesis, submitted after the viva-voce examination, must contain the appropriate copyright
\end{flushleft}


\begin{flushleft}
certificate in the beginning of the thesis, on a separate page on the left side. One of these copies
\end{flushleft}


\begin{flushleft}
is for the Department / Centre / School Library and the other is for the Central Library. A softcopy
\end{flushleft}


\begin{flushleft}
of the thesis has been submitted to the Central Library.
\end{flushleft}





\begin{flushleft}
(d)	
\end{flushleft}





\begin{flushleft}
A Hindi translation of the thesis abstract is to be submitted as part of final submission (after examiner
\end{flushleft}


\begin{flushleft}
reports are received). The students can seek assistance from Hindi Cell in this regard.
\end{flushleft}





\begin{flushleft}
(xiii)	 The relevant IPR notice to be incorporated in the soft/hard bound thesis, reports etc. shall be chosen
\end{flushleft}


\begin{flushleft}
from the following:
\end{flushleft}


\begin{flushleft}
(a)	
\end{flushleft}





\begin{flushleft}
the thesis / report etc. for which formal copyright application has NOT been filed should carry the
\end{flushleft}


\begin{flushleft}
copyright notice as:
\end{flushleft}





		





\begin{flushleft}
© Indian Institute of Technology Delhi (IITD), New Delhi, 200 ... [the year of submission of the
\end{flushleft}


\begin{flushleft}
thesis / report].
\end{flushleft}





\begin{flushleft}
(b)	
\end{flushleft}





\begin{flushleft}
and for which formal copyright application has been filed with the copyright office. Should carry
\end{flushleft}


\begin{flushleft}
the copyright notice as:
\end{flushleft}





		





\begin{flushleft}
© Indian Institute of Technology Delhi (IITD), New Delhi, 200 ...[the year of submission of the thesis/
\end{flushleft}


\begin{flushleft}
report]. All right reserved. Copyright Registration Pending.
\end{flushleft}





\begin{flushleft}
(c)	
\end{flushleft}





\begin{flushleft}
and for which, in-addition to a formal copyright application with the Copyright Office, patent/design
\end{flushleft}


\begin{flushleft}
application has also been filed with the patent office, should carry the {``}IPR Notice'' as:
\end{flushleft}


\begin{flushleft}
Intellectual Property Rights
\end{flushleft}


\begin{flushleft}
(IPR) notice
\end{flushleft}





		





\begin{flushleft}
Part of this thesis may be protected by one or more of Indian Copyright Registrations (Pending)
\end{flushleft}


\begin{flushleft}
and / or Indian Patent / Design (Pending) by Dean, Industrial Research \& Development (IRD) Unit
\end{flushleft}


\begin{flushleft}
Indian Institute of Technology Delhi (IITD) New Delhi-110016, India. IITD restricts the use, in any
\end{flushleft}


\begin{flushleft}
form, of the information, in part or full, contained in this thesis only on written permission of the
\end{flushleft}


\begin{flushleft}
Competent Authority: Dean, IRD, IIT Delhi or MD, FITT, IIT Delhi.
\end{flushleft}





		





\begin{flushleft}
The notices at {`}b' and {`}c' should only be, inserted after the formal application(s) has (have) been filed
\end{flushleft}


\begin{flushleft}
with the appropriate office(s) as the case may be and the same has been confirmed by FITT office.
\end{flushleft}





\begin{flushleft}
(xiv)	If a member of the academic staff, who is registered for the degree, leaves the Institute before the
\end{flushleft}


\begin{flushleft}
minimum period of registration is completed, he/she will be permitted to submit his thesis in due course,
\end{flushleft}


\begin{flushleft}
provided that:
\end{flushleft}


\begin{flushleft}
(a)	
\end{flushleft}





\begin{flushleft}
a substantial part of the research has been completed at the Institute; and
\end{flushleft}





\begin{flushleft}
(b)	
\end{flushleft}





\begin{flushleft}
any additional work required can be adequately supervised.
\end{flushleft}





\begin{flushleft}
(xv)	 A member of the academic staff who has commenced his research before joining the Institute may, at
\end{flushleft}


\begin{flushleft}
the discretion of the BAP and on the recommendation of the Supervisor through the DRC/CRC/SRC
\end{flushleft}


\begin{flushleft}
concerned, be permitted to include in his period of registration, part or all of the time spent on research
\end{flushleft}


\begin{flushleft}
before joining the Institute, up to a maximum of one year.
\end{flushleft}


\begin{flushleft}
(xvi)	 A member of the non-academic staff of the Institute who satisfies eligibility qualifications may be considered
\end{flushleft}


\begin{flushleft}
for admission to the degree as a part-time candidate provided his/her application is duly approved by
\end{flushleft}


\begin{flushleft}
the Director of the Institute.
\end{flushleft}


35





\begin{flushleft}
\newpage
6. UNDERGRADUATE PROGRAMME STRUCTURES
\end{flushleft}





\begin{flushleft}
\newpage
Programme Code: BB1
\end{flushleft}





\begin{flushleft}
Bachelor of Technology in Biochemical Engineering and Biotechnology
\end{flushleft}


\begin{flushleft}
Department of Biochemical Engineering and Biotechnology
\end{flushleft}


\begin{flushleft}
The overall Credit Structure
\end{flushleft}





\begin{flushleft}
BBP332	 Bioprocess Engineering Laboratory	
\end{flushleft}


0	0	3	1.5


\begin{flushleft}
BBL431	 Bioprocess Technology	
\end{flushleft}


2	0	0	2


\begin{flushleft}
BBL432	 Fluid Solid Systems	
\end{flushleft}


2	0	0	2


\begin{flushleft}
BBL433	 Enzyme Science and Engineering	
\end{flushleft}


3	 0	 2	 4


\begin{flushleft}
BBL434	 Bioinformatics	
\end{flushleft}


2	0	2	3


\begin{flushleft}
BBD451	 Major Project Part-I (BB1)	
\end{flushleft}


0	 0	 6	 3


\begin{flushleft}
BBL731	 Bioseparation Engineering	
\end{flushleft}


3	0	3	4.5


\begin{flushleft}
BBL732	 Bioprocess Plant Design	
\end{flushleft}


3	0	2	4


\begin{flushleft}
BBL733	 Recombinant DNA Technology	
\end{flushleft}


2	0	3	3.5


\begin{flushleft}
CLL122	 Chemical Reaction Engineering-I	
\end{flushleft}


3	1	0	4


\begin{flushleft}
CLL231	 Fluid Mechanics for Chemical Engineers	
\end{flushleft}


3	 1	 0	 4


\begin{flushleft}
CLL251	 Heat Transfer for Chemical Engineers 	
\end{flushleft}


3	 1	 0	 4


\begin{flushleft}
CLL252	 Mass Transfer-I	
\end{flushleft}


3	0	0	3


\begin{flushleft}
CLL261	 Process Dynamics and Control	
\end{flushleft}


3	 1	 0	 4


\begin{flushleft}
CLP301	 Chemical Engineering Laboratory-I	
\end{flushleft}


0	0	3	1.5


\begin{flushleft}
CLP302	 Chemical Engineering Laboratory-II	
\end{flushleft}


0	0	3	1.5


	


\begin{flushleft}
Total Credits				69
\end{flushleft}





\begin{flushleft}
Course Category	
\end{flushleft}


\begin{flushleft}
Credits
\end{flushleft}


\begin{flushleft}
Institute Core Courses
\end{flushleft}


\begin{flushleft}
Basic Sciences (BS)		 22
\end{flushleft}


\begin{flushleft}
Engineering Arts and Science (EAS)		 18
\end{flushleft}


\begin{flushleft}
Humanities and Social Sciences (HuSS)		 15
\end{flushleft}


\begin{flushleft}
Programme-linked Courses		11
\end{flushleft}


\begin{flushleft}
Departmental Courses
\end{flushleft}


\begin{flushleft}
Departmental Core 		 69
\end{flushleft}


\begin{flushleft}
Departmental Electives		 10
\end{flushleft}


\begin{flushleft}
Open Category Courses		 10
\end{flushleft}


\begin{flushleft}
Total Graded Credit requirement		 155
\end{flushleft}


\begin{flushleft}
Non Graded Units		 15
\end{flushleft}


\begin{flushleft}
Institute Core : Basic Sciences
\end{flushleft}


\begin{flushleft}
CML100	 General Chemistry	
\end{flushleft}


3	 0	 0	 3


\begin{flushleft}
CMP100	Chemistry Laboratory	
\end{flushleft}


0	0	4	2


\begin{flushleft}
MTL100	 Calculus	
\end{flushleft}


3	1	0	4


\begin{flushleft}
MTL101	 Linear Algebra and Differential Equations	
\end{flushleft}


3	 1	 0	 4


\begin{flushleft}
PYL100	 Electromagnetic Waves and Quantum 	
\end{flushleft}


3	 0	 0	 3


\begin{flushleft}
	Mechanics	
\end{flushleft}


\begin{flushleft}
PYP100	 Physics Laboratory	
\end{flushleft}


0	0	4	2


\begin{flushleft}
SBL100	 Introductory Biology for Engineers	
\end{flushleft}


3	 0	 2	 4


	


\begin{flushleft}
Total Credits				22
\end{flushleft}





\begin{flushleft}
Departmental Electives
\end{flushleft}


\begin{flushleft}
BBL341	 Environmental Biotechnology	
\end{flushleft}


\begin{flushleft}
BBL342	 Physical and Chemical Properties of	
\end{flushleft}


\begin{flushleft}
	Biomolecules
\end{flushleft}


\begin{flushleft}
BBL343	 Carbohydrates and Lipids in Biotechnology	
\end{flushleft}


\begin{flushleft}
BBV350	 Special Module in Biochemical Engineering	
\end{flushleft}


	


\begin{flushleft}
and Biotechnology
\end{flushleft}


\begin{flushleft}
BBD351	Mini Project (BB)	
\end{flushleft}


\begin{flushleft}
BBL441	 Food Science and Engineering	
\end{flushleft}


\begin{flushleft}
BBL442	 Immunology	
\end{flushleft}


\begin{flushleft}
BBL443	 Modeling and Simulation of Bioprocesses	
\end{flushleft}


\begin{flushleft}
BBL444	 Advanced Bioprocess Control	
\end{flushleft}


\begin{flushleft}
BBL445	 Membrane Applications in Bioprocessing	
\end{flushleft}


\begin{flushleft}
BBL446	 Biophysics	
\end{flushleft}


\begin{flushleft}
BBL447	 Enzyme Catalyzed Organic Synthesis	
\end{flushleft}


\begin{flushleft}
BBD452	 Major Project Part-II (BB1)	
\end{flushleft}


\begin{flushleft}
CLL477	 Materials of Construction	
\end{flushleft}


\begin{flushleft}
BBL734	 Metabolic Regulation and Engineering	
\end{flushleft}


\begin{flushleft}
BBL735	 Genomics and Proteomics	
\end{flushleft}


\begin{flushleft}
BBL736	 Dynamics of Microbial Systems	
\end{flushleft}


\begin{flushleft}
BBL737	 Instrumentation and Analytical Methods in	
\end{flushleft}


\begin{flushleft}
	Bioengineering
\end{flushleft}


\begin{flushleft}
BBL740	 Plant Cell Technology	
\end{flushleft}


\begin{flushleft}
BBL741	 Protein Science and Engineering	
\end{flushleft}


\begin{flushleft}
BBL742	 Biological Waste Treatment	
\end{flushleft}


\begin{flushleft}
BBL743	 High Resolution Methods in Biotechnology	
\end{flushleft}


\begin{flushleft}
BBL744	 Animal Cell Technology	
\end{flushleft}


\begin{flushleft}
BBL745	 Combinatorial Biotechnology	
\end{flushleft}


\begin{flushleft}
BBL746	 Current Topics in Biochemical Engineering	
\end{flushleft}


	


\begin{flushleft}
and Biotechnology
\end{flushleft}


\begin{flushleft}
BBL747	 Bionanotechnology	
\end{flushleft}


\begin{flushleft}
BBL748	 Data Analysis for DNA Microarrays	
\end{flushleft}


\begin{flushleft}
BBL749	 Cancer Cell Biology	
\end{flushleft}


\begin{flushleft}
BBL750	 Genome Engineering	
\end{flushleft}


\begin{flushleft}
CLL728	 Biomass Conversion and Utilization	
\end{flushleft}





\begin{flushleft}
Institute Core: Engineering Arts and Sciences
\end{flushleft}


\begin{flushleft}
APL100	 Engineering Mechanics	
\end{flushleft}


3	1	0	4


\begin{flushleft}
COL100	 Introduction to Computer Science	
\end{flushleft}


3	 0	 2	 4


\begin{flushleft}
CVL100	 Environmental Science	
\end{flushleft}


2	0	0	2


\begin{flushleft}
ELL100	 Introduction to Electrical Engineering	
\end{flushleft}


3	 0	 2	 4


\begin{flushleft}
MCP100	Engineering Visualization	
\end{flushleft}


0	0	4	2


\begin{flushleft}
MCP101	 Product Realization through Manufacturing	 0	 0	 4	 2
\end{flushleft}


	


\begin{flushleft}
Total Credits				18
\end{flushleft}


\begin{flushleft}
Programme-Linked Basic / Engineering Arts / Sciences Core
\end{flushleft}


\begin{flushleft}
APL102	 Introduction to Materials Science	
\end{flushleft}


3	 0	 2	 4


	


\begin{flushleft}
and Engineering
\end{flushleft}


\begin{flushleft}
CLL110	 Transport Phenomena	
\end{flushleft}


3	1	0	4


\begin{flushleft}
MTL102	 Differential Equations	
\end{flushleft}


3	0	0	3


	


\begin{flushleft}
Total Credits				11
\end{flushleft}


\begin{flushleft}
Humanities and Social Sciences
\end{flushleft}


\begin{flushleft}
Courses from Humanities, Social Sciences and Management 	
\end{flushleft}


\begin{flushleft}
offered under this category				
\end{flushleft}


15


\begin{flushleft}
Departmental Core
\end{flushleft}


\begin{flushleft}
BBL131	 Principles of Biochemistry	
\end{flushleft}


\begin{flushleft}
BBL132	 General Microbiology	
\end{flushleft}


\begin{flushleft}
BBL133	 Mass and Energy Balances in Biochemical	
\end{flushleft}


\begin{flushleft}
	Engineering
\end{flushleft}


\begin{flushleft}
BBL231	 Molecular Biology and Genetics	
\end{flushleft}


\begin{flushleft}
BBL331	 Bioprocess Engineering	
\end{flushleft}





3	0	3	4.5


3	 0	 3	 4.5


3	 0	 0	 3


3	 0	 3	 4.5


3	0	0	3





37





3	0	0	3


2	 1	 0	 3


2	 1	 0	 3


1	 0	 0	 1


0	0	6	3


3	 0	 0	 3


3	0	2	4


3	 0	 2	 4


3	0	0	3


3	0	0	3


3	0	0	3


2	 0	 2	 3


0	 0	 16	8


3	0	0	3


3	 0	 0	 3


2	 0	 2	 3


3	 0	 0	 3


2	 0	 2	 3


2	0	2	3


3	 0	 0	 3


3	0	2	4


2	 0	 2	 3


3	0	2	4


3	0	0	3


3	 0	 0	 3


3	0	0	3


3	 0	 2	 4


3	0	3	4.5


2	 0	 2	 3


3	 0	 0	 3





\newpage
38





\begin{flushleft}
Semester
\end{flushleft}





\begin{flushleft}
VIII
\end{flushleft}





\begin{flushleft}
VII
\end{flushleft}





\begin{flushleft}
VI
\end{flushleft}





\begin{flushleft}
V
\end{flushleft}





\begin{flushleft}
IV
\end{flushleft}





\begin{flushleft}
III
\end{flushleft}





\begin{flushleft}
II
\end{flushleft}





\begin{flushleft}
I
\end{flushleft}





0


2


\begin{flushleft}
CLL251
\end{flushleft}





4





1


0


\begin{flushleft}
BBL231
\end{flushleft}





4





0


3


\begin{flushleft}
CLP302
\end{flushleft}





4.5





3





3





0





0





2





0


0


\begin{flushleft}
DE1 (4)
\end{flushleft}





\begin{flushleft}
Humanities Elective-4
\end{flushleft}





0


3


\begin{flushleft}
HUL3XX
\end{flushleft}





4





3





1.5





\begin{flushleft}
Chemical Engineering Laboratory
\end{flushleft}


\begin{flushleft}
II
\end{flushleft}





3





\begin{flushleft}
Molecular Biology and Genetics
\end{flushleft}





3





\begin{flushleft}
Heat Transfer for Chemical
\end{flushleft}


\begin{flushleft}
Engineers
\end{flushleft}





3





2





4





3





3





0





0





3





3





3





3





3





3





1


0


\begin{flushleft}
MTL101
\end{flushleft}





\begin{flushleft}
Calculus
\end{flushleft}





4





3





1





0





4





\begin{flushleft}
Linear Algebra and Differential
\end{flushleft}


\begin{flushleft}
Equations
\end{flushleft}





3





\begin{flushleft}
Course-4
\end{flushleft}





\begin{flushleft}
MTL100
\end{flushleft}





0





0





\begin{flushleft}
Course-5
\end{flushleft}





0





4





\begin{flushleft}
Chemistry Laboratory
\end{flushleft}





0


4


\begin{flushleft}
CMP100
\end{flushleft}





\begin{flushleft}
Physics Laboratory
\end{flushleft}





\begin{flushleft}
PYP100
\end{flushleft}





2





2





\begin{flushleft}
Course-6
\end{flushleft}





0





0





4





2





\begin{flushleft}
Product Realization through
\end{flushleft}


\begin{flushleft}
Manufacturing
\end{flushleft}





\begin{flushleft}
MCP101
\end{flushleft}





\begin{flushleft}
Course-7
\end{flushleft}





0





0





2





1





\begin{flushleft}
Introduction to Engineering
\end{flushleft}


\begin{flushleft}
(Non-graded)
\end{flushleft}





\begin{flushleft}
NIN100
\end{flushleft}





\begin{flushleft}
Course-8
\end{flushleft}





0


1


\begin{flushleft}
NEN100
\end{flushleft}





0.5





0





0





1





0.5





\begin{flushleft}
Professional Ethics and Social
\end{flushleft}


\begin{flushleft}
Responsibility-2 (Non-graded)
\end{flushleft}





0





\begin{flushleft}
Professional Ethics and Social
\end{flushleft}


\begin{flushleft}
Responsibility-1 (Non-graded)
\end{flushleft}





\begin{flushleft}
NEN100
\end{flushleft}





0





0





1


0


\begin{flushleft}
CLL122
\end{flushleft}





4





3





0


2


\begin{flushleft}
CLL231
\end{flushleft}





4





\begin{flushleft}
Introductory Biology for
\end{flushleft}


\begin{flushleft}
Engineers
\end{flushleft}





\begin{flushleft}
SBL100
\end{flushleft}





0





0





0


2


\begin{flushleft}
DE 2 (3)
\end{flushleft}





1


0


\begin{flushleft}
OC1 (4)
\end{flushleft}





\begin{flushleft}
Humanities Elective-3
\end{flushleft}





0


0


\begin{flushleft}
HUL2XX
\end{flushleft}





\begin{flushleft}
Mass Transfer I
\end{flushleft}





1


0


\begin{flushleft}
CLL252
\end{flushleft}





3





4





4





3





4





1


0


\begin{flushleft}
CLL261
\end{flushleft}





4





3





0





2





3





0





0





0


6


\begin{flushleft}
DE 3 (3)
\end{flushleft}





\begin{flushleft}
B.Tech. Project
\end{flushleft}





0


0


\begin{flushleft}
BED451
\end{flushleft}





\begin{flushleft}
Fluid Solid Systems
\end{flushleft}





1


0


\begin{flushleft}
BBL432
\end{flushleft}





3





3





2





4





\begin{flushleft}
Process Dynamics and Control
\end{flushleft}





3





0


3


\begin{flushleft}
CVL100
\end{flushleft}





0


2


\begin{flushleft}
BBL731
\end{flushleft}





\begin{flushleft}
Bioinformatics
\end{flushleft}





0


3


\begin{flushleft}
BBL434
\end{flushleft}





3





1.5





\begin{flushleft}
Chemical Engineering
\end{flushleft}


\begin{flushleft}
Laboratory I
\end{flushleft}





0


0


\begin{flushleft}
CLP301
\end{flushleft}





3





3





0





0





0


3


\begin{flushleft}
OC 2 (3)
\end{flushleft}


3





4.5





\begin{flushleft}
Bioseparation Engineering
\end{flushleft}





2





0





2





2





4.5





\begin{flushleft}
Environmental Science
\end{flushleft}





3





\begin{flushleft}
Principles of Biochemistry
\end{flushleft}





\begin{flushleft}
BBL131
\end{flushleft}





0


0


\begin{flushleft}
BBP332
\end{flushleft}





\begin{flushleft}
Differential Equations
\end{flushleft}





0


3


\begin{flushleft}
MTL102
\end{flushleft}





0


2


\begin{flushleft}
BBL732
\end{flushleft}





\begin{flushleft}
Enzyme Science and
\end{flushleft}


\begin{flushleft}
Engineering
\end{flushleft}





0


3


\begin{flushleft}
BBL433
\end{flushleft}





4





1.5





3





3





0





0





0


2


\begin{flushleft}
OC 3 (3)
\end{flushleft}





3





4





\begin{flushleft}
Bioprocess Plant Design
\end{flushleft}





3





0





3





4.5





\begin{flushleft}
Bioprocess Engineering
\end{flushleft}


\begin{flushleft}
Laboratory
\end{flushleft}





3





3





\begin{flushleft}
General Microbiology
\end{flushleft}





\begin{flushleft}
BBL132
\end{flushleft}





\begin{flushleft}
BBN101
\end{flushleft}





0 0


\begin{flushleft}
HUL2XX
\end{flushleft}


1 0


\begin{flushleft}
BBL331
\end{flushleft}





4





3





0 0


\begin{flushleft}
BBL431
\end{flushleft}





3





2





2





0





3





\begin{flushleft}
Recombinant DNA
\end{flushleft}


\begin{flushleft}
Technology
\end{flushleft}





0 0


\begin{flushleft}
BBL733
\end{flushleft}





3.5





2





\begin{flushleft}
Bioprocess Technology
\end{flushleft}





3





\begin{flushleft}
Bioprocess Engineering
\end{flushleft}





3





\begin{flushleft}
Humanities Elective-1
\end{flushleft}





3





3





0





2





1





0





\begin{flushleft}
Humanities Elective-2
\end{flushleft}





\begin{flushleft}
HUL2XX
\end{flushleft}





0





4





1





\begin{flushleft}
Mass and Energy Balances in Introduction to Biochem. Engg.
\end{flushleft}


\begin{flushleft}
Biochemical Engg.
\end{flushleft}


\begin{flushleft}
And Biotech. (Non-graded)
\end{flushleft}





\begin{flushleft}
BBL133
\end{flushleft}





\begin{flushleft}
L
\end{flushleft}





\begin{flushleft}
T
\end{flushleft}





0





2





\begin{flushleft}
Language and
\end{flushleft}


\begin{flushleft}
Writing Skills-2
\end{flushleft}


\begin{flushleft}
(Non-Graded)
\end{flushleft}





15 0





14 0





12 1





15 2





17 4





18 1





1 12 2





0 2 1 9.5 1


\begin{flushleft}
NLN100
\end{flushleft}





\begin{flushleft}
Language and
\end{flushleft}


\begin{flushleft}
Writing Skills-1
\end{flushleft}


\begin{flushleft}
(Non-Graded)
\end{flushleft}





\begin{flushleft}
NLN100
\end{flushleft}





\begin{flushleft}
Course-9
\end{flushleft}





\begin{flushleft}
Note: Courses 1-6 above are attended in the given order by half of all first year students. The other half of First year students attend the Courses 1-6 of II semester first.
\end{flushleft}





0





0


0


\begin{flushleft}
CML100
\end{flushleft}





\begin{flushleft}
Introduction to Chemistry
\end{flushleft}





3





\begin{flushleft}
Electromagnetic Waves and
\end{flushleft}


\begin{flushleft}
Quantum Mechanics
\end{flushleft}





\begin{flushleft}
PYL100
\end{flushleft}





\begin{flushleft}
Course-3
\end{flushleft}





\begin{flushleft}
Chemical Reaction Engineering Fluid Mechanics for Chemical
\end{flushleft}


\begin{flushleft}
I
\end{flushleft}


\begin{flushleft}
Engineers
\end{flushleft}





3





3





\begin{flushleft}
Transport Phenomena
\end{flushleft}





4





\begin{flushleft}
Introduction to Materials Science
\end{flushleft}


\begin{flushleft}
and Engineering
\end{flushleft}





0





\begin{flushleft}
CLL110
\end{flushleft}





1





\begin{flushleft}
Introduction to Computer
\end{flushleft}


\begin{flushleft}
Science
\end{flushleft}





\begin{flushleft}
APL102
\end{flushleft}





3





\begin{flushleft}
Engineering Mechanics
\end{flushleft}





2





0.5





3





0


3


\begin{flushleft}
COL100
\end{flushleft}





\begin{flushleft}
Introduction to Engineering
\end{flushleft}


\begin{flushleft}
Visualization
\end{flushleft}





\begin{flushleft}
Introduction to Electrical
\end{flushleft}


\begin{flushleft}
Engineering
\end{flushleft}





4





\begin{flushleft}
MCP100
\end{flushleft}





\begin{flushleft}
Course-1
\end{flushleft}





0


2


\begin{flushleft}
APL100
\end{flushleft}





\begin{flushleft}
Course-2
\end{flushleft}





\begin{flushleft}
ELL100
\end{flushleft}





\begin{flushleft}
Credits
\end{flushleft}





1





17.0





30.0





20.0





26.0





21.0





31.0





\#\#





\begin{flushleft}
TOTAL=155.0
\end{flushleft}





2 16.0





16 22.0





7 16.5





9 21.5





0 21.0





10 24.0





6 17.0 1.5 23.0





13 17.0 2.5 28.5





\begin{flushleft}
P
\end{flushleft}





\begin{flushleft}
Non-Graded Units
\end{flushleft}





\begin{flushleft}
B.Tech. in Biochemical Engineering and Biotechnology	BB1
\end{flushleft}


\begin{flushleft}
Contact Hours
\end{flushleft}





\begin{flushleft}
\newpage
Programme Code: BB5
\end{flushleft}





\begin{flushleft}
Dual Degree Programme : Bachelor of Technology and Master of
\end{flushleft}


\begin{flushleft}
Technology in Biochemical Engineering and Biotechnology
\end{flushleft}


\begin{flushleft}
Department of Biochemical Engineering and Biotechnology
\end{flushleft}


\begin{flushleft}
The overall Credit Structure
\end{flushleft}





\begin{flushleft}
BBP332	 Bioprocess Engineering Laboratory	
\end{flushleft}


0	0	3	1.5


\begin{flushleft}
BBL431	 Bioprocess Technology	
\end{flushleft}


2	0	0	2


\begin{flushleft}
BBL432	 Fluid Solid Systems	
\end{flushleft}


2	0	0	2


\begin{flushleft}
BBL433	 Enzyme Science and Engineering	
\end{flushleft}


3	 0	 2	 4


\begin{flushleft}
BBL434	 Bioinformatics	
\end{flushleft}


2	0	2	3


\begin{flushleft}
BBL731	 Bioseparation Engineering	
\end{flushleft}


3	0	3	4.5


\begin{flushleft}
BBL732	 Bioprocess Plant Design	
\end{flushleft}


3	0	2	4


\begin{flushleft}
BBL733	 Recombinant DNA Technology	
\end{flushleft}


2	0	3	3.5


\begin{flushleft}
CLL122	 Chemical Reaction Engineering-I	
\end{flushleft}


3	1	0	4


\begin{flushleft}
CLL231	 Fluid Mechanics for Chemical Engineers	
\end{flushleft}


3	 1	 0	 4


\begin{flushleft}
CLL251	 Heat Transfer for Chemical Engineers 	
\end{flushleft}


3	 1	 0	 4


\begin{flushleft}
CLL252	 Mass Transfer-I	
\end{flushleft}


3	0	0	3


\begin{flushleft}
CLL261	 Process Dynamics and Control	
\end{flushleft}


3	 1	 0	 4


\begin{flushleft}
CLP301	 Chemical Engineering Laboratory-I	
\end{flushleft}


0	0	3	1.5


\begin{flushleft}
CLP302	 Chemical Engineering Laboratory-II	
\end{flushleft}


0	0	3	1.5


	


\begin{flushleft}
Total Credits				66
\end{flushleft}





\begin{flushleft}
Course Category	
\end{flushleft}


\begin{flushleft}
Credits
\end{flushleft}


\begin{flushleft}
B.Tech Part
\end{flushleft}


\begin{flushleft}
Institute Core Courses
\end{flushleft}


\begin{flushleft}
Basic Sciences (BS)		 22
\end{flushleft}


\begin{flushleft}
Engineering Arts and Science (EAS)		 18
\end{flushleft}


\begin{flushleft}
Humanities and Social Sciences (HuSS)		 15
\end{flushleft}


\begin{flushleft}
Programme-linked Courses		11
\end{flushleft}


\begin{flushleft}
Departmental Courses
\end{flushleft}


\begin{flushleft}
Departmental Core 		 66*
\end{flushleft}


\begin{flushleft}
Departmental Electives		 6
\end{flushleft}


\begin{flushleft}
Open Category Courses		 4
\end{flushleft}


\begin{flushleft}
Total B.Tech. Credit Requirement		142*
\end{flushleft}


\begin{flushleft}
Non Graded Units		 15
\end{flushleft}


\begin{flushleft}
M.Tech. Part
\end{flushleft}


\begin{flushleft}
Programme Core Courses		
\end{flushleft}


\begin{flushleft}
Programme Elective Courses		
\end{flushleft}


\begin{flushleft}
Total M.Tech. Credit Requirement		
\end{flushleft}


\begin{flushleft}
Grand Total Credit Requirement		
\end{flushleft}





32


16


48


190





\begin{flushleft}
Departmental Electives
\end{flushleft}





\begin{flushleft}
BBL341	 Environmental Biotechnology	
\end{flushleft}


\begin{flushleft}
BBL342	 Physical and Chemical 	
\end{flushleft}


	


\begin{flushleft}
Properties of Biomolecules
\end{flushleft}


\begin{flushleft}
*Those students who join the dual degree program from JEE or
\end{flushleft}


\begin{flushleft}
BBL343	 Carbohydrates and Lipids in Biotechnology	
\end{flushleft}


\begin{flushleft}
those who choose to pursue a M.Tech degree along with their B.Tech
\end{flushleft}


\begin{flushleft}
BBV350	 Special Module in Biochemical 	
\end{flushleft}


\begin{flushleft}
program, will not be required to do the 3-credit B.Tech Project as part
\end{flushleft}


	


\begin{flushleft}
Engineering and Biotechnology
\end{flushleft}


\begin{flushleft}
of the Departmental core requirement.
\end{flushleft}


\begin{flushleft}
BBD351	Mini Project (BB)	
\end{flushleft}


\begin{flushleft}
Institute Core: Basic Sciences
\end{flushleft}


\begin{flushleft}
BBL441	 Food Science and Engineering	
\end{flushleft}


\begin{flushleft}
CML100	 General Chemistry	
\end{flushleft}


3	 0	 0	 3


\begin{flushleft}
BBL442	 Immunology	
\end{flushleft}


\begin{flushleft}
CMP100	Chemistry Laboratory	
\end{flushleft}


0	0	4	2


\begin{flushleft}
BBL443	 Modeling and Simulation of Bioprocesses	
\end{flushleft}


\begin{flushleft}
MTL100	 Calculus	
\end{flushleft}


3	1	0	4


\begin{flushleft}
BBL444	 Advanced Bioprocess Control	
\end{flushleft}


\begin{flushleft}
MTL101	 Linear Algebra and Differential Equations	
\end{flushleft}


3	 1	 0	 4


\begin{flushleft}
BBL445	 Membrane Applications in Bioprocessing	
\end{flushleft}


\begin{flushleft}
PYL100	 Electromagnetic Waves and Quantum 	
\end{flushleft}


3	 0	 0	 3


\begin{flushleft}
BBL446	 Biophysics	
\end{flushleft}


\begin{flushleft}
	Mechanics	
\end{flushleft}


\begin{flushleft}
BBL447	 Enzyme Catalyzed Organic Synthesis	
\end{flushleft}


\begin{flushleft}
PYP100	 Physics Laboratory	
\end{flushleft}


0	0	4	2


\begin{flushleft}
BBL740	 Plant Cell Technology	
\end{flushleft}


\begin{flushleft}
SBL100	 Introductory Biology for Engineers	
\end{flushleft}


3	 0	 2	 4


\begin{flushleft}
BBL741	 Protein Science and Engineering	
\end{flushleft}


	


\begin{flushleft}
Total Credits				22	 CLL728	 Biomass Conversion and Utilization	
\end{flushleft}


\begin{flushleft}
CLL477	 Materials of Construction	
\end{flushleft}


\begin{flushleft}
Institute Core: Engineering Arts and Sciences
\end{flushleft}


\begin{flushleft}
APL100	 Engineering Mechanics	
\end{flushleft}


3	1	0	4


\begin{flushleft}
COL100	 Introduction to Computer Science	
\end{flushleft}


3	 0	 2	 4


\begin{flushleft}
CVL100	 Environmental Science	
\end{flushleft}


2	0	0	2


\begin{flushleft}
ELL100	 Introduction to Electrical Engineering	
\end{flushleft}


3	 0	 2	 4


\begin{flushleft}
MCP100	Engineering Visualization	
\end{flushleft}


0	0	4	2


\begin{flushleft}
MCP101	 Product Realization through Manufacturing	 0	 0	 4	 2
\end{flushleft}


	


\begin{flushleft}
Total Credits				18	
\end{flushleft}


\begin{flushleft}
Programme-Linked Basic / Engineering Arts / Sciences Core
\end{flushleft}





3	0	0	3


2	 1	 0	 3


2	 1	 0	 3


1	 0	 0	 1


0	0	6	3


3	 0	 0	 3


3	0	2	4


3	 0	 2	 4


3	0	0	3


3	0	0	3


3	0	0	3


2	 0	 2	 3


2	0	2	3


3	 0	 0	 3


3	 0	 0	 3


3	0	0	3





\begin{flushleft}
Program Core
\end{flushleft}


\begin{flushleft}
BBL734	 Metabolic Regulation and Engineering	
\end{flushleft}


3	 0	 0	 3


\begin{flushleft}
BBL735	 Genomics and Proteomics	
\end{flushleft}


2	 0	 2	 3


\begin{flushleft}
BBL736	 Dynamics of Microbial Systems	
\end{flushleft}


3	 0	 0	 3


\begin{flushleft}
BBL737	 Instrumentation and Analytical	
\end{flushleft}


2	0	2	3


	


\begin{flushleft}
Methods in Bioengineering
\end{flushleft}


\begin{flushleft}
BBD851*	 Major Project Part-I (BB5)	
\end{flushleft}


0	 0	 12	6


\begin{flushleft}
BBD852*	 Major Project Part-II (BB5)	
\end{flushleft}


0	 0	 28	14


\begin{flushleft}
BBD853	 Major Project Part-I (BB5)	
\end{flushleft}


0	 0	 8	 4


\begin{flushleft}
BBD854	 Major Project Part-II (BB5)	
\end{flushleft}


0	 0	 32	16


	


\begin{flushleft}
Total Credits				32	
\end{flushleft}





\begin{flushleft}
APL102	 Introduction to Materials Science	
\end{flushleft}


3	 0	 2	 4


	


\begin{flushleft}
and Engineering
\end{flushleft}


\begin{flushleft}
CLL110	 Transport Phenomena	
\end{flushleft}


3	1	0	4


\begin{flushleft}
*BBD851 and BBD852 together are alternatives to BBD853 and
\end{flushleft}


\begin{flushleft}
MTL102	 Differential Equations	
\end{flushleft}


3	0	0	3


\begin{flushleft}
BBD854
\end{flushleft}


	


\begin{flushleft}
Total Credits				11	
\end{flushleft}


\begin{flushleft}
Program Electives
\end{flushleft}


\begin{flushleft}
Humanities and Social Sciences
\end{flushleft}


\begin{flushleft}
BBL742	 Biological Waste Treatment	
\end{flushleft}


3	0	2	4


\begin{flushleft}
Courses from Humanities, Social Sciences and Management 	
\end{flushleft}


\begin{flushleft}
BBL743	 High Resolution Methods in Biotechnology	
\end{flushleft}


2	 0	 2	 3


\begin{flushleft}
offered under this category				
\end{flushleft}


15


\begin{flushleft}
BBL744	 Animal Cell Technology	
\end{flushleft}


3	0	2	4


\begin{flushleft}
BBL745	 Combinatorial Biotechnology	
\end{flushleft}


3	0	0	3


\begin{flushleft}
Departmental Core
\end{flushleft}


\begin{flushleft}
BBL746	 Current Topics in Biochemical Engineering	
\end{flushleft}


3	 0	 0	 3


\begin{flushleft}
BBL131	 Principles of Biochemistry	
\end{flushleft}


3	0	3	4.5


	


\begin{flushleft}
and Biotechnology
\end{flushleft}


\begin{flushleft}
BBL132	 General Microbiology	
\end{flushleft}


3	 0	 3	 4.5


\begin{flushleft}
BBL747	 Bionanotechnology	
\end{flushleft}


3	0	0	3


\begin{flushleft}
BBL133	 Mass and Energy Balances in 	
\end{flushleft}


3	 0	 0	 3


\begin{flushleft}
BBL748	 Data Analysis for DNA Microarrays	
\end{flushleft}


3	 0	 2	 4


	


\begin{flushleft}
Biochemical Engineering
\end{flushleft}


\begin{flushleft}
BBL749	 Cancer Cell Biology	
\end{flushleft}


3	0	3	4.5


\begin{flushleft}
BBL231	 Molecular Biology and Genetics	
\end{flushleft}


3	 0	 3	 4.5


\begin{flushleft}
BBL750	 Genome Engineering	
\end{flushleft}


2	 0	 2	 3


\begin{flushleft}
BBL331	 Bioprocess Engineering	
\end{flushleft}


3	0	0	3


\begin{flushleft}
BBV750	 Bioreaction Engineering	
\end{flushleft}


1	0	0	1





39





\newpage
40





\begin{flushleft}
Semester
\end{flushleft}





\begin{flushleft}
X
\end{flushleft}





\begin{flushleft}
IX
\end{flushleft}





\begin{flushleft}
Summer
\end{flushleft}





\begin{flushleft}
VIII
\end{flushleft}





\begin{flushleft}
VII
\end{flushleft}





\begin{flushleft}
VI
\end{flushleft}





\begin{flushleft}
V
\end{flushleft}





\begin{flushleft}
IV
\end{flushleft}





\begin{flushleft}
III
\end{flushleft}





\begin{flushleft}
II
\end{flushleft}





\begin{flushleft}
I
\end{flushleft}





4





1


0


\begin{flushleft}
BBL231
\end{flushleft}





4





4.5





0





\begin{flushleft}
BBL735
\end{flushleft}





0





0


0


\begin{flushleft}
DE2 (3)
\end{flushleft}





3





3





1.5





\begin{flushleft}
Humanities Elective-4
\end{flushleft}





0


3


\begin{flushleft}
HUL3XX
\end{flushleft}





\begin{flushleft}
Chemical Engineering
\end{flushleft}


\begin{flushleft}
Laboratory II
\end{flushleft}





0


3


\begin{flushleft}
CLP302
\end{flushleft}





0





2





2





0





28





\begin{flushleft}
M.Tech. Project II
\end{flushleft}





0





14





3





\begin{flushleft}
Genomics and Protenomics
\end{flushleft}





3





3





0





3





\begin{flushleft}
Molecular Biology and Genetics
\end{flushleft}





3





\begin{flushleft}
Heat Transfer for Chemical
\end{flushleft}


\begin{flushleft}
Engineers
\end{flushleft}





0


2


\begin{flushleft}
CLL251
\end{flushleft}





4





0


0


\begin{flushleft}
CML100
\end{flushleft}





3





3





0





0





3





\begin{flushleft}
Introduction to Chemistry
\end{flushleft}





3





\begin{flushleft}
Electromagnetic Waves and
\end{flushleft}


\begin{flushleft}
Quantum Mechanics
\end{flushleft}





4





\begin{flushleft}
SBL100
\end{flushleft}





3





0


2


\begin{flushleft}
CLL231
\end{flushleft}





4





\begin{flushleft}
Introductory Biology for
\end{flushleft}


\begin{flushleft}
Engineers
\end{flushleft}





1


0


\begin{flushleft}
BBL731
\end{flushleft}





\begin{flushleft}
Humanities Elective-3
\end{flushleft}





0


0


\begin{flushleft}
HUL2XX
\end{flushleft}





\begin{flushleft}
Mass Transfer I
\end{flushleft}





1


0


\begin{flushleft}
CLL252
\end{flushleft}





4





3





4





0





3





3





0





2





0


2


\begin{flushleft}
PE 1 (4)
\end{flushleft}





2





0





2





3





0





12





\begin{flushleft}
M.Tech. Project I
\end{flushleft}





4





0


0


\begin{flushleft}
BBL732
\end{flushleft}





\begin{flushleft}
Fluid Solid Systems
\end{flushleft}





1


0


\begin{flushleft}
BBL432
\end{flushleft}





\begin{flushleft}
BED853
\end{flushleft}





2





4.5





4





2





4





6





4





4





\begin{flushleft}
Bioprocess Plant Design
\end{flushleft}





2





3





\begin{flushleft}
BBL737
\end{flushleft}





0





0


3


\begin{flushleft}
OC1 (4)
\end{flushleft}





1


0


\begin{flushleft}
CLL261
\end{flushleft}





\begin{flushleft}
Process Dynamics and Control
\end{flushleft}





3





\begin{flushleft}
Instrumentation and Analytical
\end{flushleft}


\begin{flushleft}
Methods in Bioengineering
\end{flushleft}





3





3





\begin{flushleft}
Bioseparation Engineering
\end{flushleft}





3





3





3





\begin{flushleft}
Chemical Reaction Engineering Fluid Mechanics for Chemical
\end{flushleft}


\begin{flushleft}
I
\end{flushleft}


\begin{flushleft}
Engineers
\end{flushleft}





1


0


\begin{flushleft}
CLL122
\end{flushleft}





\begin{flushleft}
Transport Phenomena
\end{flushleft}





3





\begin{flushleft}
CLL110
\end{flushleft}





\begin{flushleft}
APL102
\end{flushleft}





3





2





\begin{flushleft}
Course-3
\end{flushleft}





\begin{flushleft}
PYL100
\end{flushleft}





1


0


\begin{flushleft}
MTL101
\end{flushleft}





\begin{flushleft}
Calculus
\end{flushleft}





4





3





1





0





4





\begin{flushleft}
Linear Algebra and Differential
\end{flushleft}


\begin{flushleft}
Equations
\end{flushleft}





3





\begin{flushleft}
Course-4
\end{flushleft}





\begin{flushleft}
MTL100
\end{flushleft}





0





0





\begin{flushleft}
Course-5
\end{flushleft}





0





4





\begin{flushleft}
Chemistry Laboratory
\end{flushleft}





0


4


\begin{flushleft}
CMP100
\end{flushleft}





\begin{flushleft}
Physics Laboratory
\end{flushleft}





\begin{flushleft}
PYP100
\end{flushleft}





2





2





\begin{flushleft}
Course-6
\end{flushleft}





0





0





4





2





\begin{flushleft}
Product Realization through
\end{flushleft}


\begin{flushleft}
Manufacturing
\end{flushleft}





\begin{flushleft}
MCP101
\end{flushleft}





\begin{flushleft}
Course-7
\end{flushleft}





0





0





2





1





\begin{flushleft}
Introduction to Engineering
\end{flushleft}


\begin{flushleft}
(Non-graded)
\end{flushleft}





\begin{flushleft}
NIN100
\end{flushleft}





0





0





1





0.5





\begin{flushleft}
Professional Ethics and
\end{flushleft}


\begin{flushleft}
Social Responsibility-2
\end{flushleft}


\begin{flushleft}
(Non-graded)
\end{flushleft}





0


1 0.5


\begin{flushleft}
NEN100
\end{flushleft}





0


2


\begin{flushleft}
NLN100
\end{flushleft}





0


3


\begin{flushleft}
CVL100
\end{flushleft}





0


2


\begin{flushleft}
BBL733
\end{flushleft}





\begin{flushleft}
Bioinformatics
\end{flushleft}





0


3


\begin{flushleft}
BBL434
\end{flushleft}





3





1.5





\begin{flushleft}
Chemical Engineering
\end{flushleft}


\begin{flushleft}
Laboratory I
\end{flushleft}





0


0


\begin{flushleft}
CLP301
\end{flushleft}





3





3





2





2





0





2





\begin{flushleft}
PE 4 (4)
\end{flushleft}





0





0


3


\begin{flushleft}
PE 2 (4)
\end{flushleft}





4





4





3.5





\begin{flushleft}
Recombinant DNA Technology
\end{flushleft}





2





0





2





2





4.5





\begin{flushleft}
Environmental Science
\end{flushleft}





3





\begin{flushleft}
Principles of Biochemistry
\end{flushleft}





\begin{flushleft}
BBL131
\end{flushleft}





0


0


\begin{flushleft}
BBP332
\end{flushleft}





3





4.5





\begin{flushleft}
Differential Equations
\end{flushleft}





0


3


\begin{flushleft}
MTL102
\end{flushleft}





0


2


\begin{flushleft}
BBL734
\end{flushleft}





4





1.5





\begin{flushleft}
Enzyme Science and
\end{flushleft}


\begin{flushleft}
Engineering
\end{flushleft}





0


3


\begin{flushleft}
BBL433
\end{flushleft}





3





3





0





2





0


0


\begin{flushleft}
PE 3 (4)
\end{flushleft}


4





3





\begin{flushleft}
Metabolic Regulation and
\end{flushleft}


\begin{flushleft}
Engineering
\end{flushleft}





3





0





\begin{flushleft}
Bioprocess Engineering
\end{flushleft}


\begin{flushleft}
Laboratory
\end{flushleft}





3





3





\begin{flushleft}
General Microbiology
\end{flushleft}





\begin{flushleft}
BBL132
\end{flushleft}





\begin{flushleft}
BBL133
\end{flushleft}





1


0


\begin{flushleft}
BBL331
\end{flushleft}





\begin{flushleft}
Humanities Elective-1
\end{flushleft}





0


0


\begin{flushleft}
HUL2XX
\end{flushleft}





4





3





0


0


\begin{flushleft}
BBL431
\end{flushleft}





3





3





2





0





0





0





\begin{flushleft}
Dynamics of Microbial
\end{flushleft}


\begin{flushleft}
Systems
\end{flushleft}





\begin{flushleft}
BBL736
\end{flushleft}





0





3





2





\begin{flushleft}
Bioprocess Technology
\end{flushleft}





3





\begin{flushleft}
Bioprocess Engineering
\end{flushleft}





3





3





\begin{flushleft}
Mass and Energy Balances in
\end{flushleft}


\begin{flushleft}
Biochemical Engg.
\end{flushleft}





\begin{flushleft}
BBN101
\end{flushleft}





3





3





0





2





0





0





1


0


\begin{flushleft}
DE 1 (3)
\end{flushleft}





\begin{flushleft}
Humanities Elective-2
\end{flushleft}





\begin{flushleft}
HUL2XX
\end{flushleft}





0





3





4





1





\begin{flushleft}
Introduction to Biochem. Engg.
\end{flushleft}


\begin{flushleft}
and Biotech. (Non-graded)
\end{flushleft}





1





0





0





2





1





\begin{flushleft}
Language and Writing Skills-2
\end{flushleft}


\begin{flushleft}
(Non-Graded)
\end{flushleft}





0





\begin{flushleft}
Language and Writing Skills-1
\end{flushleft}


\begin{flushleft}
(Non-Graded)
\end{flushleft}





\begin{flushleft}
Professional Ethics and
\end{flushleft}


\begin{flushleft}
Social Responsibility-1
\end{flushleft}


\begin{flushleft}
(Non-graded)
\end{flushleft}





0





\begin{flushleft}
NLN100
\end{flushleft}





\begin{flushleft}
Course-8
\end{flushleft}





\begin{flushleft}
NEN100
\end{flushleft}





\begin{flushleft}
Course-9
\end{flushleft}





\begin{flushleft}
Note: Courses 1-6 above are attended in the given order by half of all first year students. The other half of First year students attend the Courses 1-6 of II semester first.
\end{flushleft}





0





\begin{flushleft}
Introduction to Materials
\end{flushleft}


\begin{flushleft}
Science and Engineering
\end{flushleft}





4





3





0





3





1





\begin{flushleft}
Introduction to Computer
\end{flushleft}


\begin{flushleft}
Science
\end{flushleft}





\begin{flushleft}
Engineering Mechanics
\end{flushleft}





2





0.5





3





0


3


\begin{flushleft}
COL100
\end{flushleft}





\begin{flushleft}
Introduction to Engineering
\end{flushleft}


\begin{flushleft}
Visualization
\end{flushleft}





\begin{flushleft}
Introduction to Electrical
\end{flushleft}


\begin{flushleft}
Engineering
\end{flushleft}





4





\begin{flushleft}
MCP100
\end{flushleft}





\begin{flushleft}
Course-1
\end{flushleft}





0


2


\begin{flushleft}
APL100
\end{flushleft}





\begin{flushleft}
Course-2
\end{flushleft}





\begin{flushleft}
ELL100
\end{flushleft}





0





7





18





14





15





15





17





18





12





9.5





\begin{flushleft}
L
\end{flushleft}





0





0





0





0





1





2





4





1





2





1





\begin{flushleft}
T
\end{flushleft}





\begin{flushleft}
Credits
\end{flushleft}





17.0





22.0





18.0





19.5





21.5





21.0





\#





28.0





25.0





26.0





22.0





23.0





26.0





21.0





1 31.0





2 23.0





3 28.5





\begin{flushleft}
TOTAL=190.0
\end{flushleft}





28 14.0





18 16.0





8





8





7





9





0





10 24.0





6





13 17.0





\begin{flushleft}
P
\end{flushleft}





\begin{flushleft}
Non-Graded Units
\end{flushleft}





\begin{flushleft}
Dual Degree Programme : B.Tech. and M.Tech. in Biochemical Engineering and Biotechnology	BB5
\end{flushleft}


\begin{flushleft}
Contact Hours
\end{flushleft}





\begin{flushleft}
\newpage
Bachelor of Technology in Chemical Engineering
\end{flushleft}





\begin{flushleft}
Programme Code: CH1
\end{flushleft}





\begin{flushleft}
Department of Chemical Engineering
\end{flushleft}


\begin{flushleft}
The overall Credit Structure
\end{flushleft}





\begin{flushleft}
CLD412	
\end{flushleft}


\begin{flushleft}
CLD413	
\end{flushleft}


\begin{flushleft}
CLD414	
\end{flushleft}


	


\begin{flushleft}
CLD415	
\end{flushleft}


	


\begin{flushleft}
CLL475	
\end{flushleft}





\begin{flushleft}
Course Category	
\end{flushleft}


\begin{flushleft}
Credits
\end{flushleft}


\begin{flushleft}
Institute Core Courses
\end{flushleft}


\begin{flushleft}
Basic Sciences (BS)		 22
\end{flushleft}


\begin{flushleft}
Engineering Arts and Science (EAS)		 18
\end{flushleft}


\begin{flushleft}
Humanities and Social Sciences (HuSS)		 15
\end{flushleft}


\begin{flushleft}
Programme-linked Courses		7
\end{flushleft}


\begin{flushleft}
Departmental Courses
\end{flushleft}


\begin{flushleft}
Departmental Core 		 67
\end{flushleft}


\begin{flushleft}
Departmental Electives		 12
\end{flushleft}


\begin{flushleft}
Open Category Courses		 10
\end{flushleft}


\begin{flushleft}
Total Graded Credit requirement		 151
\end{flushleft}


\begin{flushleft}
Non Graded Units		 15
\end{flushleft}





\begin{flushleft}
Major Project in Energy and Environment	
\end{flushleft}


\begin{flushleft}
Major Project in Complex Fluids	
\end{flushleft}


\begin{flushleft}
Major Project in Process Engineering, 	
\end{flushleft}


\begin{flushleft}
Modeling and Optimization
\end{flushleft}


\begin{flushleft}
Major Project in Biopharmaceuticals	
\end{flushleft}


\begin{flushleft}
and Fine Chemicals
\end{flushleft}


\begin{flushleft}
Safety and Hazards in Process Industries	
\end{flushleft}





0	 0	 10	5


0	 0	 10	5


0	 0	 10	5


0	 0	 10	5


3	 0	 0	 3





\begin{flushleft}
Departmental Electives
\end{flushleft}





\begin{flushleft}
CLL477	 Materials of Construction	
\end{flushleft}


\begin{flushleft}
CLL705	 Petroleum Reservoir Engineering	
\end{flushleft}


\begin{flushleft}
CLL706	 Petroleum Production Engineering	
\end{flushleft}


\begin{flushleft}
CLL707	 Population Balance Modeling	
\end{flushleft}


\begin{flushleft}
CLL720	 Principles of Electrochemical Engineering	
\end{flushleft}


\begin{flushleft}
Institute Core: Basic Sciences
\end{flushleft}


\begin{flushleft}
CLL721	 Electrochemical Methods	
\end{flushleft}


\begin{flushleft}
CLL722	 Electrochemical Conversion and Storage	
\end{flushleft}


\begin{flushleft}
CML100	 General Chemistry	
\end{flushleft}


3	 0	 0	 3


\begin{flushleft}
CMP100	Chemistry Laboratory	
\end{flushleft}


0	0	4	2


\begin{flushleft}
	Devices
\end{flushleft}


\begin{flushleft}
MTL100	 Calculus	
\end{flushleft}


3	1	0	4


\begin{flushleft}
CLL723	 Hydrogen Energy and Fuel Cell Technology	
\end{flushleft}


\begin{flushleft}
MTL101	 Linear Algebra and Differential Equations	
\end{flushleft}


3	 1	 0	 4


\begin{flushleft}
CLL724	 Environmental Engineering and Waste Mgmt	
\end{flushleft}


\begin{flushleft}
PYL100	 Electromagnetic Waves and Quantum	
\end{flushleft}


3	 0	 0	 3


\begin{flushleft}
CLL725	 Air Pollution Control Engineering	
\end{flushleft}


\begin{flushleft}
	Mechanics
\end{flushleft}


\begin{flushleft}
CLL726	 Molecular Modeling of Catalytic Reactions	
\end{flushleft}


\begin{flushleft}
PYP100	 Physics Laboratory	
\end{flushleft}


0	0	4	2


\begin{flushleft}
CLL727	 Heterogeneous Catalysis and Catalytic	
\end{flushleft}


\begin{flushleft}
SBL100	 Introductory Biology for Engineers	
\end{flushleft}


3	 0	 2	 4


\begin{flushleft}
	Reactors
\end{flushleft}


\begin{flushleft}
CLL728	 Biomass Conversion and Utilization	
\end{flushleft}


	


\begin{flushleft}
Total Credits				22
\end{flushleft}


\begin{flushleft}
CLL730	 Structure, Transport and Reactions	
\end{flushleft}


\begin{flushleft}
Institute Core: Engineering Arts and Sciences
\end{flushleft}


	


\begin{flushleft}
in BioNano Systems
\end{flushleft}


\begin{flushleft}
CLL731	 Advanced Transport Phenomena	
\end{flushleft}


\begin{flushleft}
APL100	 Engineering Mechanics	
\end{flushleft}


3	1	0	4


\begin{flushleft}
CLL732	 Advanced Chemical Engineering	
\end{flushleft}


\begin{flushleft}
COL100	 Introduction to Computer Science	
\end{flushleft}


3	 0	 2	 4


\begin{flushleft}
	Thermodynamics
\end{flushleft}


\begin{flushleft}
CVL100	 Environmental Science	
\end{flushleft}


2	0	0	2


\begin{flushleft}
ELL100	 Introduction to Electrical Engineering	
\end{flushleft}


3	 0	 2	 4


\begin{flushleft}
CLL733	 Industrial Multiphase Reactors	
\end{flushleft}


\begin{flushleft}
MCP100	Engineering Visualization	
\end{flushleft}


0	0	4	2


\begin{flushleft}
CLL734	 Process Intensification and Novel Reactors	
\end{flushleft}


\begin{flushleft}
MCP101	 Product Realization through Manufacturing	 0	 0	 4	 2
\end{flushleft}


\begin{flushleft}
CLL735	 Design of Multicomponent Separation	
\end{flushleft}


\begin{flushleft}
	Processes
\end{flushleft}


	


\begin{flushleft}
Total Credits				18
\end{flushleft}


\begin{flushleft}
CLL736	 Experimental Characterization of Multiphase	
\end{flushleft}


\begin{flushleft}
Programme-Linked Basic / Engineering Arts / Sciences Core
\end{flushleft}


\begin{flushleft}
	Reactors
\end{flushleft}


\begin{flushleft}
CLL742	 Experimental Characterization	
\end{flushleft}


\begin{flushleft}
APL102	 Introduction to Materials Science	
\end{flushleft}


3	 0	 2	 4


	


\begin{flushleft}
of BioMacromolecules
\end{flushleft}


	


\begin{flushleft}
and Engineering
\end{flushleft}


\begin{flushleft}
CLL743	 Petrochemicals Technology	
\end{flushleft}


\begin{flushleft}
CML103	 Applied Chemistry - Chemistry at Interfaces	 3	 0	 0	 3
\end{flushleft}


\begin{flushleft}
CLL761	 Chemical Engineering Mathematics	
\end{flushleft}


	


\begin{flushleft}
Total Credits				7
\end{flushleft}


\begin{flushleft}
CLL762	 Advanced Computational Techniques	
\end{flushleft}


\begin{flushleft}
Humanities and Social Sciences
\end{flushleft}


	


\begin{flushleft}
in Chemical Engineering
\end{flushleft}


\begin{flushleft}
CLL766	 Interfacial Engineering	
\end{flushleft}


\begin{flushleft}
Courses from Humanities, Social Sciences and Management 	
\end{flushleft}


\begin{flushleft}
CLL767	 Structures and Properties of Polymers 	
\end{flushleft}


\begin{flushleft}
offered under this category				
\end{flushleft}


15


\begin{flushleft}
CLL768	 Fundamentals of Computational Fluid Dynamics	
\end{flushleft}


\begin{flushleft}
Departmental Core
\end{flushleft}


\begin{flushleft}
CLL769	 Applications of Computational Fluid Dynamics	
\end{flushleft}


\begin{flushleft}
CLL771	 Introduction to Complex Fluids	
\end{flushleft}


\begin{flushleft}
CLL110	 Transport Phenomena	
\end{flushleft}


3	1	0	4


\begin{flushleft}
CLL772	 Transport Phenomena in Complex Fluids	
\end{flushleft}


\begin{flushleft}
CLL111	 Material and Energy Balances	
\end{flushleft}


2	 2	 0	 4


\begin{flushleft}
CLL773	 Thermodynamics of Complex Fluids	
\end{flushleft}


\begin{flushleft}
CLL113	 Numerical Methods in Chemical Engineering	 3	 0	 2	 4
\end{flushleft}


\begin{flushleft}
CLL774	 Simulation Techniques for Complex Fluids	
\end{flushleft}


\begin{flushleft}
CLL121	 Chemical Engineering Thermodynamics	
\end{flushleft}


3	1	0	4


\begin{flushleft}
CLL775	 Polymerization Process Modeling	
\end{flushleft}


\begin{flushleft}
CLL122	 Chemical Reaction Engineering-I	
\end{flushleft}


3	1	0	4


\begin{flushleft}
CLL776	 Granular Materials	
\end{flushleft}


\begin{flushleft}
CLL141	 Intro. to Materials for Chemical Engineers	
\end{flushleft}


3	 0	 0	 3


\begin{flushleft}
CLL777	 Complex Fluids Technology	
\end{flushleft}


\begin{flushleft}
CLL222	 Chemical Reaction Engineering-II	
\end{flushleft}


3	0	0	3


\begin{flushleft}
CLL778	 Interfacial Behaviour and Transport	
\end{flushleft}


\begin{flushleft}
CLL231	 Fluid Mechanics for Chemical Engineers	
\end{flushleft}


3	 1	 0	 4


	


\begin{flushleft}
of Biomolecules
\end{flushleft}


\begin{flushleft}
CLL251	 Heat Transfer for Chemical Engineers 	
\end{flushleft}


3	 1	 0	 4


\begin{flushleft}
CLL779	 Molecular Biotechnology and in-vitro	
\end{flushleft}


\begin{flushleft}
CLL252	 Mass Transfer-I	
\end{flushleft}


3	0	0	3


\begin{flushleft}
	Diagnostics
\end{flushleft}


\begin{flushleft}
CLL261	 Process Dynamics and Control	
\end{flushleft}


3	 1	 0	 4


\begin{flushleft}
CLL271	 Introduction to Industrial Biotechnology	
\end{flushleft}


3	 0	 0	 3


\begin{flushleft}
CLL780	 Bioprocessing and Bioseparations	
\end{flushleft}


\begin{flushleft}
CLP301	 Chemical Engineering Laboratory-I	
\end{flushleft}


0	0	3	1.5


\begin{flushleft}
CLL781	 Process Operations Scheduling	
\end{flushleft}


\begin{flushleft}
CLP302	 Chemical Engineering Laboratory-II	
\end{flushleft}


0	0	3	1.5


\begin{flushleft}
CLL782	 Process Optimization	
\end{flushleft}


\begin{flushleft}
CLP303	 Chemical Engineering Laboratory-III	
\end{flushleft}


0	0	3	1.5


\begin{flushleft}
CLL783	 Advanced Process Control	
\end{flushleft}


\begin{flushleft}
CLL331	 Fluid-Particle Mechanics	
\end{flushleft}


3	1	0	4


\begin{flushleft}
CLL784	 Process Modeling and Simulation	
\end{flushleft}


\begin{flushleft}
CLL352	 Mass Transfer-II	
\end{flushleft}


3	1	0	4


\begin{flushleft}
CLL785	 Evolutionary Optimization	
\end{flushleft}


\begin{flushleft}
CLL361	 Instrumentation and Automation	
\end{flushleft}


1	0	3	2.5


\begin{flushleft}
CLL786	 Fine Chemicals Technology	
\end{flushleft}


\begin{flushleft}
CLL371	 Chemical Process Technology and Economics	3	 1	 0	 4
\end{flushleft}


\begin{flushleft}
CLL791	 Chemical Product and Process Integration	
\end{flushleft}


\begin{flushleft}
CLD411	 B. Tech. project	
\end{flushleft}


0	0	8	4


\begin{flushleft}
CLL792	 Chemical Product Development	
\end{flushleft}


\begin{flushleft}
and Commercialization
\end{flushleft}


	


\begin{flushleft}
Total Credits				67	 	
\end{flushleft}


\begin{flushleft}
CLL793	 Membrane Science and Engineering	
\end{flushleft}


\begin{flushleft}
Departmental Electives
\end{flushleft}


\begin{flushleft}
CLL794	 Petroleum Refinery Engineering	
\end{flushleft}


\begin{flushleft}
CLL133	 Powder Processing and Technology	
\end{flushleft}


3	0	0	3


\begin{flushleft}
CLV796	 Current Topics in Chemical Engineering	
\end{flushleft}


\begin{flushleft}
CLL296	 Nano-engineering of Soft Materials	
\end{flushleft}


3	 0	 0	 3


\begin{flushleft}
CLV797	 Recent Advances in Chemical Engineering	
\end{flushleft}


\begin{flushleft}
CLL390	 Process Utilities and Pipeline Design	
\end{flushleft}


3	 0	 0	 3


\begin{flushleft}
CLL798	 Selected Topics in Chemical Engineering-I	
\end{flushleft}


\begin{flushleft}
CLL402	 Process Plant Design	
\end{flushleft}


3	0	0	3


\begin{flushleft}
CLL799	 Selected Topics in Chemical Engineering-II	
\end{flushleft}





41





3	0	0	3


3	0	0	3


3	0	0	3


3	0	0	3


3	 0	 0	 3


3	0	0	3


3	 0	 0	 3


3	


3	


3	


3	


3	





0	


0	


0	


0	


0	





0	


0	


0	


0	


0	





3


3


3


3


3





3	 0	 0	 3


3	0	0	3


3	0	0	3


3	0	0	3


3	0	0	3


3	 0	 0	 3


3	 0	 0	 3


3	 0	 0	 3


3	0	0	3


3	0	0	3


3	0	0	3


2	0	2	3


3	0	0	3


3	 0	 0	 3


2	0	2	3


2	0	2	3


3	 0	 0	 3


3	 0	 0	 3


3	 0	 0	 3


3	 0	 0	 3


3	0	0	3


3	 0	 0	 3


3	0	0	3


3	0	0	3


3	 0	 0	 3


3	0	0	3


3	0	0	3


3	0	0	3


3	0	0	3


3	 0	 0	 3


3	0	0	3


3	0	0	3


3	 0	 0	 3


3	0	0	3


3	


3	


1	


2	


3	


3	





0	


0	


0	


0	


0	


0	





0	


0	


0	


0	


0	


0	





3


3


1


2


3


3





\newpage
42





\begin{flushleft}
Semester
\end{flushleft}





\begin{flushleft}
VIII
\end{flushleft}





\begin{flushleft}
VII
\end{flushleft}





\begin{flushleft}
VI
\end{flushleft}





\begin{flushleft}
V
\end{flushleft}





\begin{flushleft}
IV
\end{flushleft}





\begin{flushleft}
III
\end{flushleft}





\begin{flushleft}
II
\end{flushleft}





\begin{flushleft}
I
\end{flushleft}





0





2





4





3





3





0





0





3





3





3





3





3





4





3





0





0





3





0


0


3


\begin{flushleft}
OC1 / DE 3
\end{flushleft}





1


0


\begin{flushleft}
DE 2
\end{flushleft}





\begin{flushleft}
Mass Transfer II
\end{flushleft}





0


0


\begin{flushleft}
CLL352
\end{flushleft}





\begin{flushleft}
Mass Transfer I
\end{flushleft}





3





3





3





3





3





0





0





0 0


\begin{flushleft}
DE 4
\end{flushleft}





3





3





0 0 3


\begin{flushleft}
DE 3 / OC 1
\end{flushleft}





0 0


\begin{flushleft}
DE 1
\end{flushleft}





\begin{flushleft}
Chemical Reaction
\end{flushleft}


\begin{flushleft}
Engineering II
\end{flushleft}





1 0 4


\begin{flushleft}
CLL222
\end{flushleft}





3





4





3





1


0


\begin{flushleft}
CLL252
\end{flushleft}





\begin{flushleft}
Chemical Reaction
\end{flushleft}


\begin{flushleft}
Engineering I
\end{flushleft}





\begin{flushleft}
Chemical Engineering
\end{flushleft}


\begin{flushleft}
Thermodynamics
\end{flushleft}





0 2


\begin{flushleft}
CLL231
\end{flushleft}





4





1 0


\begin{flushleft}
CLL331
\end{flushleft}





4





4





3





1





0





4





1 0


\begin{flushleft}
CLL271
\end{flushleft}





4





\begin{flushleft}
CML103
\end{flushleft}





0


0


\begin{flushleft}
CLL251
\end{flushleft}





3





1


0


\begin{flushleft}
CLL141
\end{flushleft}





4





3





0


0


\begin{flushleft}
CLL371
\end{flushleft}





3





\begin{flushleft}
Introduction to Materials for
\end{flushleft}


\begin{flushleft}
Chemical Engineers
\end{flushleft}





3





\begin{flushleft}
Heat Transfer for Chemical
\end{flushleft}


\begin{flushleft}
Engineers
\end{flushleft}





3





\begin{flushleft}
Applied Chemistry: Chemistry
\end{flushleft}


\begin{flushleft}
at Interfaces
\end{flushleft}





3





3





3





0





0





1 0


\begin{flushleft}
OC 3
\end{flushleft}





0 0


\begin{flushleft}
OC 2
\end{flushleft}





3





4





3





3





0





3





0





0





0


3


\begin{flushleft}
HUL3XX
\end{flushleft}





4





3





1.5





\begin{flushleft}
Chemical Engineering
\end{flushleft}


\begin{flushleft}
Laboratory III
\end{flushleft}





1


0


\begin{flushleft}
CLP303
\end{flushleft}





\begin{flushleft}
Introduction to Industrial Chemical Process Technology
\end{flushleft}


\begin{flushleft}
Biotechnology
\end{flushleft}


\begin{flushleft}
and Economics
\end{flushleft}





3





\begin{flushleft}
Fluid-Particle Mechanics
\end{flushleft}





3





\begin{flushleft}
Fluid Mechanics for
\end{flushleft}


\begin{flushleft}
Chemical Engineers
\end{flushleft}





3





3





2 0 4


\begin{flushleft}
CLL122
\end{flushleft}





2





\begin{flushleft}
Transport Phenomena
\end{flushleft}





4





\begin{flushleft}
CLL113
\end{flushleft}





\begin{flushleft}
Numerical Methods in
\end{flushleft}


\begin{flushleft}
Chemical Engineering
\end{flushleft}





\begin{flushleft}
CLL111
\end{flushleft}





1


0


\begin{flushleft}
CLL121
\end{flushleft}





1


0


\begin{flushleft}
MTL101
\end{flushleft}





\begin{flushleft}
Calculus
\end{flushleft}





\begin{flushleft}
Linear Algebra and Differential
\end{flushleft}


\begin{flushleft}
Equations
\end{flushleft}





3





\begin{flushleft}
Course-4
\end{flushleft}





\begin{flushleft}
MTL100
\end{flushleft}





\begin{flushleft}
Course-5
\end{flushleft}





0 4 2


\begin{flushleft}
CMP100
\end{flushleft}





0





0





4





2





\begin{flushleft}
Chemistry Laboratory
\end{flushleft}





0





\begin{flushleft}
Physics Laboratory
\end{flushleft}





\begin{flushleft}
PYP100
\end{flushleft}





\begin{flushleft}
Course-6
\end{flushleft}





0





0





4





2





\begin{flushleft}
Product Realization
\end{flushleft}


\begin{flushleft}
through Manufacturing
\end{flushleft}





\begin{flushleft}
MCP101
\end{flushleft}





\begin{flushleft}
Course-7
\end{flushleft}





0





0





2





1





\begin{flushleft}
Introduction to Engineering
\end{flushleft}


\begin{flushleft}
(Non-graded)
\end{flushleft}





\begin{flushleft}
NIN100
\end{flushleft}





\begin{flushleft}
Course-8
\end{flushleft}





0 1 0.5


\begin{flushleft}
NEN100
\end{flushleft}





0





0





1





0.5





\begin{flushleft}
Professional Ethics and
\end{flushleft}


\begin{flushleft}
Social Responsibility-2
\end{flushleft}


\begin{flushleft}
(Non-graded)
\end{flushleft}





0





\begin{flushleft}
Professional Ethics and
\end{flushleft}


\begin{flushleft}
Social Responsibility-1
\end{flushleft}


\begin{flushleft}
(Non-graded)
\end{flushleft}





\begin{flushleft}
NEN100
\end{flushleft}





0





0





\begin{flushleft}
L
\end{flushleft}





1 0 4


\begin{flushleft}
SBL100
\end{flushleft}


\begin{flushleft}
APL102
\end{flushleft}





0 2 4


\begin{flushleft}
CLL261
\end{flushleft}





1 0 4


\begin{flushleft}
CLL361
\end{flushleft}





0





1





0





8





\begin{flushleft}
B. Tech. Project
\end{flushleft}





4





0 3 2.5


\begin{flushleft}
CLD411
\end{flushleft}





\begin{flushleft}
Instrumentation and
\end{flushleft}


\begin{flushleft}
Automation
\end{flushleft}





3





\begin{flushleft}
Process Dynamics and
\end{flushleft}


\begin{flushleft}
Control
\end{flushleft}





3





0 2


\begin{flushleft}
CVL100
\end{flushleft}





4





3





3





2





1





0





1 0


\begin{flushleft}
HUL2XX
\end{flushleft}





0 0


\begin{flushleft}
HUL2XX
\end{flushleft}





4





4





2





\begin{flushleft}
Environmental Science
\end{flushleft}





3





\begin{flushleft}
Introductory Biology for Introduction to Materials
\end{flushleft}


\begin{flushleft}
Engineers
\end{flushleft}


\begin{flushleft}
Science and Engineering
\end{flushleft}





3





\begin{flushleft}
HUL2XX
\end{flushleft}





\begin{flushleft}
CLN101
\end{flushleft}





2





\begin{flushleft}
CLP301
\end{flushleft}





0





0


3


\begin{flushleft}
CLP302
\end{flushleft}





0





0





3





1





1.5





1.5





\begin{flushleft}
Chemical Engineering
\end{flushleft}


\begin{flushleft}
Laboratory II
\end{flushleft}





0





\begin{flushleft}
Chemical Engineering
\end{flushleft}


\begin{flushleft}
Laboratory I
\end{flushleft}





0





\begin{flushleft}
Introduction to Chemical
\end{flushleft}


\begin{flushleft}
Engineering (Non-graded)
\end{flushleft}





\begin{flushleft}
T P
\end{flushleft}





\begin{flushleft}
Credits
\end{flushleft}





0





2





\begin{flushleft}
Language and
\end{flushleft}


\begin{flushleft}
Writing Skills-2
\end{flushleft}


\begin{flushleft}
(Non-Graded)
\end{flushleft}





1





\begin{flushleft}
TOTAL=151.0
\end{flushleft}





12.0 151.0





0 25.0





0 25.0





0 22.0





0 26.0





12 0 0 12.0 0





12 2 11 19.5





16 3 6 22.0





17 2 3 20.5





18 4 4 24.0





14 4 2 19.0





1 22.0





12 2 6 17.0 1.5 23.0





0 2 1 9.5 1 13 17.0 2.5 28.5


\begin{flushleft}
NLN100
\end{flushleft}





\begin{flushleft}
Language and
\end{flushleft}


\begin{flushleft}
Writing Skills-1
\end{flushleft}


\begin{flushleft}
(Non-Graded)
\end{flushleft}





\begin{flushleft}
NLN100
\end{flushleft}





\begin{flushleft}
Course-9
\end{flushleft}





\begin{flushleft}
Note: Courses 1-6 above are attended in the given order by half of all first year students. The other half of First year students attend the Courses 1-6 of II semester first.
\end{flushleft}





4





0 0


\begin{flushleft}
CML100
\end{flushleft}





\begin{flushleft}
Introduction to Chemistry
\end{flushleft}





3





\begin{flushleft}
Course-3
\end{flushleft}





\begin{flushleft}
Material and Energy
\end{flushleft}


\begin{flushleft}
Balances
\end{flushleft}





\begin{flushleft}
CLL110
\end{flushleft}





0





3





1





3





0 3 2


\begin{flushleft}
COL100
\end{flushleft}





\begin{flushleft}
Introduction to
\end{flushleft}


\begin{flushleft}
Computer Science
\end{flushleft}





0.5





\begin{flushleft}
Engineering Mechanics
\end{flushleft}





4





\begin{flushleft}
Introduction to
\end{flushleft}


\begin{flushleft}
Engineering
\end{flushleft}


\begin{flushleft}
Visualization
\end{flushleft}





\begin{flushleft}
Introduction to Electrical
\end{flushleft}


\begin{flushleft}
Engineering
\end{flushleft}





0


2


\begin{flushleft}
APL100
\end{flushleft}





\begin{flushleft}
Electromagnetic Waves
\end{flushleft}


\begin{flushleft}
and Quantum Mechanics
\end{flushleft}





\begin{flushleft}
MCP100
\end{flushleft}





\begin{flushleft}
Course-1
\end{flushleft}





3





\begin{flushleft}
PYL100
\end{flushleft}





\begin{flushleft}
Course-2
\end{flushleft}





\begin{flushleft}
ELL100
\end{flushleft}





\begin{flushleft}
Non-Graded Units
\end{flushleft}





\begin{flushleft}
B.Tech. in Chemical Engineering	CH1
\end{flushleft}


\begin{flushleft}
Contact Hours
\end{flushleft}





\begin{flushleft}
\newpage
Programme Code: CH7
\end{flushleft}





\begin{flushleft}
Dual Degree Programme : Bachelor of Technology and Master of Technology
\end{flushleft}


\begin{flushleft}
in Chemical Engineering
\end{flushleft}


\begin{flushleft}
Department of Chemical Engineering
\end{flushleft}


\begin{flushleft}
The overall Credit Structure
\end{flushleft}





\begin{flushleft}
Departmental Electives
\end{flushleft}





\begin{flushleft}
Course Category	
\end{flushleft}


\begin{flushleft}
Credits
\end{flushleft}


\begin{flushleft}
B.Tech Part
\end{flushleft}


\begin{flushleft}
Institute Core Courses
\end{flushleft}


\begin{flushleft}
Basic Sciences (BS)		 22
\end{flushleft}


\begin{flushleft}
Engineering Arts and Science (EAS)		 18
\end{flushleft}


\begin{flushleft}
Humanities and Social Sciences (HuSS)		 15
\end{flushleft}


\begin{flushleft}
Programme-linked Courses		7
\end{flushleft}


\begin{flushleft}
Departmental Courses
\end{flushleft}


\begin{flushleft}
Departmental Core 		 63
\end{flushleft}


\begin{flushleft}
Departmental Electives		 09
\end{flushleft}


\begin{flushleft}
Open Category Courses		 3
\end{flushleft}


\begin{flushleft}
Total B.Tech. Credit Requirement		 137
\end{flushleft}


\begin{flushleft}
Non Graded Units		 15
\end{flushleft}


\begin{flushleft}
M.Tech. Part
\end{flushleft}


\begin{flushleft}
Programme Core Courses		 33
\end{flushleft}


\begin{flushleft}
Programme Elective Courses		 12
\end{flushleft}


\begin{flushleft}
Open Elective		 3	
\end{flushleft}


\begin{flushleft}
Total M.Tech. Credit Requirement		 48		
\end{flushleft}


\begin{flushleft}
Grand Total Credit Requirement		 185
\end{flushleft}





\begin{flushleft}
CLL133	 Powder Processing and Technology	
\end{flushleft}


\begin{flushleft}
CLL296	 Nano-engineering of Soft Materials	
\end{flushleft}


\begin{flushleft}
CLL390	 Process Utilities and Pipeline Design	
\end{flushleft}


\begin{flushleft}
CLL402	 Process Plant Design	
\end{flushleft}


\begin{flushleft}
CLD412	 Major Project in Energy and Environment	
\end{flushleft}


\begin{flushleft}
CLD413	 Major Project in Complex Fluids	
\end{flushleft}


\begin{flushleft}
CLD414	 Major Project in Process Engineering, 	
\end{flushleft}


	


\begin{flushleft}
Modeling and Optimization
\end{flushleft}


\begin{flushleft}
CLD415	 Major Project in Biopharmaceuticals and	
\end{flushleft}


	


\begin{flushleft}
Fine Chemicals
\end{flushleft}


\begin{flushleft}
CLL475	 Safety and Hazards in Process Industries	
\end{flushleft}


\begin{flushleft}
CLL477	 Materials of Construction	
\end{flushleft}


\begin{flushleft}
CLL705	 Petroleum Reservoir Engineering	
\end{flushleft}


\begin{flushleft}
CLL706	 Petroleum Production Engineering	
\end{flushleft}


\begin{flushleft}
CLL707	 Population Balance Modeling	
\end{flushleft}


\begin{flushleft}
CLL720	 Principles of Electrochemical Engineering	
\end{flushleft}


\begin{flushleft}
CLL721	 Electrochemical Methods	
\end{flushleft}


\begin{flushleft}
CLL722	 Electrochemical Conversion and Storage	
\end{flushleft}


\begin{flushleft}
	Devices
\end{flushleft}


\begin{flushleft}
CLL723	 Hydrogen Energy and Fuel Cell Technology	
\end{flushleft}


\begin{flushleft}
CLL724	 Environmental Engineering and Waste	
\end{flushleft}


\begin{flushleft}
	Management
\end{flushleft}


\begin{flushleft}
CLL725	 Air Pollution Control Engineering	
\end{flushleft}


\begin{flushleft}
CLL726	 Molecular Modeling of Catalytic Reactions	
\end{flushleft}


\begin{flushleft}
CLL727	 Heterogeneous Catalysis and Catalytic	
\end{flushleft}


\begin{flushleft}
	Reactors
\end{flushleft}


\begin{flushleft}
CLL728	 Biomass Conversion and Utilization	
\end{flushleft}


\begin{flushleft}
CLL730	 Structure, Transport and Reactions	
\end{flushleft}


	


\begin{flushleft}
in BioNano Systems
\end{flushleft}


\begin{flushleft}
CLL732	 Advanced Chemical Engineering	
\end{flushleft}


\begin{flushleft}
	Thermodynamics
\end{flushleft}


\begin{flushleft}
CLL734	 Process Intensification and Novel Reactors	
\end{flushleft}


\begin{flushleft}
CLL735	 Design of Multicomponent Separation	
\end{flushleft}


\begin{flushleft}
	Processes
\end{flushleft}


\begin{flushleft}
CLL736	 Experimental Characterization	
\end{flushleft}


	


\begin{flushleft}
of Multiphase Reactors
\end{flushleft}





\begin{flushleft}
Institute Core: Basic Sciences
\end{flushleft}


\begin{flushleft}
CML100	 General Chemistry	
\end{flushleft}


3	 0	 0	 3


\begin{flushleft}
CMP100	Chemistry Laboratory	
\end{flushleft}


0	0	4	2


\begin{flushleft}
MTL100	 Calculus	
\end{flushleft}


3	1	0	4


\begin{flushleft}
MTL101	 Linear Algebra and Differential Equations	
\end{flushleft}


3	 1	 0	 4


\begin{flushleft}
PYL100	 Electromagnetic Waves and Quantum Mechanics	3	0	0	3
\end{flushleft}


\begin{flushleft}
PYP100	 Physics Laboratory	
\end{flushleft}


0	0	4	2


\begin{flushleft}
SBL100	 Introductory Biology for Engineers	
\end{flushleft}


3	 0	 2	 4


	


\begin{flushleft}
Total Credits				22
\end{flushleft}


\begin{flushleft}
Institute Core: Engineering Arts and Sciences
\end{flushleft}





\begin{flushleft}
APL100	 Engineering Mechanics	
\end{flushleft}


3	1	0	4


\begin{flushleft}
COL100	 Introduction to Computer Science	
\end{flushleft}


3	 0	 2	 4


\begin{flushleft}
CVL100	 Environmental Science	
\end{flushleft}


2	0	0	2


\begin{flushleft}
ELL100	 Introduction to Electrical Engineering	
\end{flushleft}


3	 0	 2	 4


\begin{flushleft}
MCP100	Engineering Visualization	
\end{flushleft}


0	0	4	2


\begin{flushleft}
CLL742	 Experimental Characterization	
\end{flushleft}


\begin{flushleft}
MCP101	 Product Realization through Manufacturing	 0	 0	 4	 2
\end{flushleft}


\begin{flushleft}
of BioMacromolecules
\end{flushleft}


	


\begin{flushleft}
Total Credits				18	 	
\end{flushleft}


\begin{flushleft}
CLL743	 Petrochemicals Technology	
\end{flushleft}


\begin{flushleft}
Programme-Linked Basic / Engineering Arts / Sciences Core
\end{flushleft}


\begin{flushleft}
CLL761	 Chemical Engineering Mathematics	
\end{flushleft}


\begin{flushleft}
CLL762	 Advanced Computational Techniques	
\end{flushleft}


\begin{flushleft}
APL102	 Introduction to Materials Science and Engineering	3	0	2	4
\end{flushleft}


	


\begin{flushleft}
in Chemical Engineering
\end{flushleft}


\begin{flushleft}
CML103	 Applied Chemistry - Chemistry at Interfaces	 3	 0	 0	 3
\end{flushleft}


\begin{flushleft}
CLL766	 Interfacial Engineering	
\end{flushleft}


	


\begin{flushleft}
Total Credits				7
\end{flushleft}


\begin{flushleft}
CLL767	 Structures and Properties of Polymers 	
\end{flushleft}


\begin{flushleft}
Humanities and Social Sciences
\end{flushleft}


\begin{flushleft}
CLL768	 Fundamentals of Computational Fluid	
\end{flushleft}


\begin{flushleft}
	Dynamics
\end{flushleft}


\begin{flushleft}
Courses from Humanities, Social Sciences and Management 	
\end{flushleft}


\begin{flushleft}
CLL769	 Applications of Computational Fluid 	
\end{flushleft}


\begin{flushleft}
offered under this category				
\end{flushleft}


15


\begin{flushleft}
	Dynamics
\end{flushleft}


\begin{flushleft}
Departmental Core
\end{flushleft}


\begin{flushleft}
CLL771	 Introduction to Complex Fluids	
\end{flushleft}


\begin{flushleft}
CLL772	 Transport Phenomena in Complex Fluids	
\end{flushleft}


\begin{flushleft}
CLL110	 Transport Phenomena	
\end{flushleft}


3	1	0	4


\begin{flushleft}
CLL773	 Thermodynamics of Complex Fluids	
\end{flushleft}


\begin{flushleft}
CLL111	 Material and Energy Balances	
\end{flushleft}


2	 2	 0	 4


\begin{flushleft}
CLL774	 Simulation Techniques for Complex Fluids	
\end{flushleft}


\begin{flushleft}
CLL113	 Numerical Methods in Chemical Engineering	 3	 0	 2	 4
\end{flushleft}


\begin{flushleft}
CLL775	 Polymerization Process Modeling	
\end{flushleft}


\begin{flushleft}
CLL121	 Chemical Engineering Thermodynamics	
\end{flushleft}


3	1	0	4


\begin{flushleft}
CLL776	 Granular Materials	
\end{flushleft}


\begin{flushleft}
CLL122	 Chemical Reaction Engineering-I	
\end{flushleft}


3	1	0	4


\begin{flushleft}
CLL777	 Complex Fluids Technology	
\end{flushleft}


\begin{flushleft}
CLL141	 Introduction to Materials for Chemical	
\end{flushleft}


3	 0	 0	 3


\begin{flushleft}
	Engineers
\end{flushleft}


\begin{flushleft}
CLL778	 Interfacial Behaviour and Transport	
\end{flushleft}


\begin{flushleft}
CLL222	 Chemical Reaction Engineering-II	
\end{flushleft}


3	0	0	3


	


\begin{flushleft}
of Biomolecules
\end{flushleft}


\begin{flushleft}
CLL231	 Fluid Mechanics for Chemical Engineers	
\end{flushleft}


3	 1	 0	 4


\begin{flushleft}
CLL779	 Molecular Biotechnology and in-vitro	
\end{flushleft}


\begin{flushleft}
CLL251	 Heat Transfer for Chemical Engineers 	
\end{flushleft}


3	 1	 0	 4


\begin{flushleft}
	Diagnostics
\end{flushleft}


\begin{flushleft}
CLL252	 Mass Transfer-I	
\end{flushleft}


3	0	0	3


\begin{flushleft}
CLL780	 Bioprocessing and Bioseparations	
\end{flushleft}


\begin{flushleft}
CLL261	 Process Dynamics and Control	
\end{flushleft}


3	 1	 0	 4


\begin{flushleft}
CLL781	 Process Operations Scheduling	
\end{flushleft}


\begin{flushleft}
CLL271	 Introduction to Industrial Biotechnology	
\end{flushleft}


3	 0	 0	 3


\begin{flushleft}
CLL782	 Process Optimization	
\end{flushleft}


\begin{flushleft}
CLP301	 Chemical Engineering Laboratory-I	
\end{flushleft}


0	0	3	1.5


\begin{flushleft}
CLL783	 Advanced Process Control	
\end{flushleft}


\begin{flushleft}
CLP302	 Chemical Engineering Laboratory-II	
\end{flushleft}


0	0	3	1.5


\begin{flushleft}
CLL784	 Process Modeling and Simulation	
\end{flushleft}


\begin{flushleft}
CLP303	 Chemical Engineering Laboratory-III	
\end{flushleft}


0	0	3	1.5


\begin{flushleft}
CLL785	 Evolutionary Optimization	
\end{flushleft}


\begin{flushleft}
CLL331	 Fluid-Particle Mechanics	
\end{flushleft}


3	1	0	4


\begin{flushleft}
CLL786	 Fine Chemicals Technology	
\end{flushleft}


\begin{flushleft}
CLL352	 Mass Transfer-II	
\end{flushleft}


3	1	0	4


\begin{flushleft}
CLL791	 Chemical Product and Process Integration	
\end{flushleft}


\begin{flushleft}
CLL361	 Instrumentation and Automation	
\end{flushleft}


1	0	3	2.5


\begin{flushleft}
CLL792	 Chemical Product Development	
\end{flushleft}


\begin{flushleft}
CLL371	 Chemical Process Technology and Economics	3	1	0	4
\end{flushleft}


	


\begin{flushleft}
and Commercialization
\end{flushleft}


\begin{flushleft}
CLL793	 Membrane Science and Engineering	
\end{flushleft}


	


\begin{flushleft}
Total Credits				63
\end{flushleft}





43





3	0	0	3


3	 0	 0	 3


3	 0	 0	 3


3	0	0	3


0	 0	 10	5


0	 0	 10	5


0	 0	 10	5


0	 0	 10	5


3	 0	 0	 3


3	0	0	3


3	0	0	3


3	0	0	3


3	0	0	3


3	 0	 0	 3


3	0	0	3


3	 0	 0	 3


3	 0	 0	 3


3	 0	 0	 3


3	 0	 0	 3


3	 0	 0	 3


3	 0	 0	 3


3	 0	 0	 3


3	0	0	3


3	0	0	3


3	 0	 0	 3


3	 0	 0	 3


3	0	0	3


3	0	0	3


3	0	0	3


3	0	0	3


2	0	2	3


3	0	0	3


3	 0	 0	 3


2	 0	 2	 3


2	 0	 2	 3


3	 0	 0	 3


3	 0	 0	 3


3	 0	 0	 3


3	 0	 0	 3


3	0	0	3


3	 0	 0	 3


3	0	0	3


3	0	0	3


3	 0	 0	 3


3	0	0	3


3	0	0	3


3	0	0	3


3	0	0	3


3	 0	 0	 3


3	0	0	3


3	0	0	3


3	 0	 0	 3


3	0	0	3


3	 0	 0	 3





\begin{flushleft}
\newpage
CLL794	
\end{flushleft}


\begin{flushleft}
CLV796	
\end{flushleft}


\begin{flushleft}
CLV797	
\end{flushleft}


\begin{flushleft}
CLL798	
\end{flushleft}


\begin{flushleft}
CLL799	
\end{flushleft}





\begin{flushleft}
Petroleum Refinery Engineering	
\end{flushleft}


\begin{flushleft}
Current Topics in Chemical Engineering	
\end{flushleft}


\begin{flushleft}
Recent Advances in Chemical Engineering	
\end{flushleft}


\begin{flushleft}
Selected Topics in Chemical Engineering-I	
\end{flushleft}


\begin{flushleft}
Selected Topics in Chemical Engineering-II	
\end{flushleft}





3	


1	


2	


3	


3	





0	


0	


0	


0	


0	





0	


0	


0	


0	


0	





3


1


2


3


3





\begin{flushleft}
CLL743	Petrochemicals Technology	
\end{flushleft}


\begin{flushleft}
CLL761	 Chemical Engineering Mathematics	
\end{flushleft}


\begin{flushleft}
CLL762	 Advanced Computational Techniques	
\end{flushleft}


	


\begin{flushleft}
in Chemical Engineering
\end{flushleft}


\begin{flushleft}
CLL766	 Interfacial Engineering	
\end{flushleft}


\begin{flushleft}
CLL767	 Structures and Properties of Polymers 	
\end{flushleft}


\begin{flushleft}
CLL768	 Fundamentals of Computational Fluid	
\end{flushleft}


\begin{flushleft}
	Dynamics
\end{flushleft}


\begin{flushleft}
CLL769	 Applications of Computational Fluid 	
\end{flushleft}


\begin{flushleft}
	Dynamics
\end{flushleft}


\begin{flushleft}
CLL771	 Introduction to Complex Fluids	
\end{flushleft}


\begin{flushleft}
CLL772	 Transport Phenomena in Complex Fluids	
\end{flushleft}


\begin{flushleft}
CLL773	 Thermodynamics of Complex Fluids	
\end{flushleft}


\begin{flushleft}
CLL774	 Simulation Techniques for Complex Fluids	
\end{flushleft}


\begin{flushleft}
CLL775	 Polymerization Process Modeling	
\end{flushleft}


\begin{flushleft}
CLL776	 Granular Materials	
\end{flushleft}


\begin{flushleft}
CLL777	 Complex Fluids Technology	
\end{flushleft}


\begin{flushleft}
CLL778	 Interfacial Behaviour and Transport	
\end{flushleft}


	


\begin{flushleft}
of Biomolecules
\end{flushleft}


\begin{flushleft}
CLL779	 Molecular Biotechnology and in-vitro	
\end{flushleft}


\begin{flushleft}
	Diagnostics
\end{flushleft}


\begin{flushleft}
CLL780	 Bioprocessing and Bioseparations	
\end{flushleft}


\begin{flushleft}
CLL781	 Process Operations Scheduling	
\end{flushleft}


\begin{flushleft}
CLL782	 Process Optimization	
\end{flushleft}


\begin{flushleft}
CLL783	 Advanced Process Control	
\end{flushleft}


\begin{flushleft}
CLL784	 Process Modeling and Simulation	
\end{flushleft}


\begin{flushleft}
CLL785	 Evolutionary Optimization	
\end{flushleft}


\begin{flushleft}
CLL786	 Fine Chemicals Technology	
\end{flushleft}


\begin{flushleft}
CLL791	 Chemical Product and Process Integration	
\end{flushleft}


\begin{flushleft}
CLL792	 Chemical Product Development	
\end{flushleft}


	


\begin{flushleft}
and Commercialization
\end{flushleft}


\begin{flushleft}
CLL793	 Membrane Science and Engineering	
\end{flushleft}


\begin{flushleft}
CLL794	 Petroleum Refinery Engineering	
\end{flushleft}


\begin{flushleft}
CLV796	 Current Topics in Chemical Engineering	
\end{flushleft}


\begin{flushleft}
CLV797	 Recent Advances in Chemical Engineering	
\end{flushleft}


\begin{flushleft}
CLL798	 Selected Topics in Chemical Engineering-I	
\end{flushleft}


\begin{flushleft}
CLL799	 Selected Topics in Chemical Engineering-II	
\end{flushleft}





\begin{flushleft}
Program Core
\end{flushleft}


\begin{flushleft}
CLL703	 Process Engineering	
\end{flushleft}


3	0	0	3


\begin{flushleft}
CLL731	 Advanced Transport Phenomena	
\end{flushleft}


3	0	0	3


\begin{flushleft}
CLL733	 Industrial Multiphase Reactors	
\end{flushleft}


3	0	0	3


\begin{flushleft}
CLD880	 Minor Project	
\end{flushleft}


0	0	8	4


\begin{flushleft}
CLD881	 Major Project Part-I	
\end{flushleft}


0	 0	 16	8


\begin{flushleft}
CLD882	 Major Project Part-II	
\end{flushleft}


0	 0	 24	12


	


\begin{flushleft}
Total Credits				32
\end{flushleft}


\begin{flushleft}
Program Electives
\end{flushleft}


\begin{flushleft}
CLL705	 Petroleum Reservoir Engineering	
\end{flushleft}


\begin{flushleft}
CLL706	 Petroleum Production Engineering	
\end{flushleft}


\begin{flushleft}
CLL707	 Population Balance Modeling	
\end{flushleft}


\begin{flushleft}
CLL720	 Principles of Electrochemical Engineering	
\end{flushleft}


\begin{flushleft}
CLL721	 Electrochemical Methods	
\end{flushleft}


\begin{flushleft}
CLL722	 Electrochemical Conversion and Storage	
\end{flushleft}


\begin{flushleft}
	Devices
\end{flushleft}


\begin{flushleft}
CLL723	 Hydrogen Energy and Fuel Cell Technology	
\end{flushleft}


\begin{flushleft}
CLL724	 Environmental Engineering and Waste	
\end{flushleft}


\begin{flushleft}
	Management
\end{flushleft}


\begin{flushleft}
CLL725	 Air Pollution Control Engineering	
\end{flushleft}


\begin{flushleft}
CLL726	 Molecular Modeling of Catalytic Reactions	
\end{flushleft}


\begin{flushleft}
CLL727	 Heterogeneous Catalysis and Catalytic	
\end{flushleft}


\begin{flushleft}
	Reactors
\end{flushleft}


\begin{flushleft}
CLL728	 Biomass Conversion and Utilization	
\end{flushleft}


\begin{flushleft}
CLL730	 Structure, Transport and Reactions	
\end{flushleft}


	


\begin{flushleft}
in BioNano Systems
\end{flushleft}


\begin{flushleft}
CLL732	 Advanced Chemical Engineering	
\end{flushleft}


\begin{flushleft}
	Thermodynamics
\end{flushleft}


\begin{flushleft}
CLL734	 Process Intensification and Novel Reactors	
\end{flushleft}


\begin{flushleft}
CLL735	 Design of Multicomponent Separation	
\end{flushleft}


\begin{flushleft}
	Processes
\end{flushleft}


\begin{flushleft}
CLL736	 Experimental Characterization	
\end{flushleft}


	


\begin{flushleft}
of Multiphase Reactors
\end{flushleft}


\begin{flushleft}
CLL742	 Experimental Characterization	
\end{flushleft}


	


\begin{flushleft}
of BioMacromolecules
\end{flushleft}





3	0	0	3


3	0	0	3


3	0	0	3


3	 0	 0	 3


3	0	0	3


3	 0	 0	 3


3	 0	 0	 3


3	 0	 0	 3


3	 0	 0	 3


3	 0	 0	 3


3	 0	 0	 3


3	 0	 0	 3


3	0	0	3


3	0	0	3


3	 0	 0	 3


3	 0	 0	 3


3	0	0	3


3	0	0	3





44





3	0	0	3


3	0	0	3


2	0	2	3


3	0	0	3


3	 0	 0	 3


2	 0	 2	 3


2	 0	 2	 3


3	 0	 0	 3


3	 0	 0	 3


3	 0	 0	 3


3	 0	 0	 3


3	0	0	3


3	 0	 0	 3


3	0	0	3


3	0	0	3


3	 0	 0	 3


3	0	0	3


3	0	0	3


3	0	0	3


3	0	0	3


3	 0	 0	 3


3	0	0	3


3	0	0	3


3	 0	 0	 3


3	0	0	3


3	


3	


1	


2	


3	


3	





0	


0	


0	


0	


0	


0	





0	


0	


0	


0	


0	


0	





3


3


1


2


3


3





\newpage
45





\begin{flushleft}
Semester
\end{flushleft}





\begin{flushleft}
X
\end{flushleft}





\begin{flushleft}
IX
\end{flushleft}





\begin{flushleft}
Summer
\end{flushleft}





\begin{flushleft}
VIII
\end{flushleft}





\begin{flushleft}
VII
\end{flushleft}





\begin{flushleft}
VI
\end{flushleft}





\begin{flushleft}
V
\end{flushleft}





\begin{flushleft}
IV
\end{flushleft}





\begin{flushleft}
III
\end{flushleft}





\begin{flushleft}
II
\end{flushleft}





\begin{flushleft}
I
\end{flushleft}





0





4





1


0


\begin{flushleft}
CLL121
\end{flushleft}





4





0





0





3





3





3





3





3





0





0





24





\begin{flushleft}
Major Project II
\end{flushleft}





0 16


\begin{flushleft}
CLD882
\end{flushleft}





\begin{flushleft}
Major Project I
\end{flushleft}





\begin{flushleft}
CLD881
\end{flushleft}





0





0


0


\begin{flushleft}
PE 3
\end{flushleft}





1


0


\begin{flushleft}
DE 2
\end{flushleft}





\begin{flushleft}
Mass Transfer II
\end{flushleft}





0


0


\begin{flushleft}
CLL352
\end{flushleft}





\begin{flushleft}
Mass Transfer I
\end{flushleft}





1


0


\begin{flushleft}
CLL252
\end{flushleft}





12





8





3





3





4





3





4





\begin{flushleft}
Chemical Engineering
\end{flushleft}


\begin{flushleft}
Thermodynamics
\end{flushleft}





3





\begin{flushleft}
Transport Phenomena
\end{flushleft}





\begin{flushleft}
CLL110
\end{flushleft}





1





3





3





3





3





3





3





2





0





0





0





\begin{flushleft}
PE 4
\end{flushleft}





0





0


0


\begin{flushleft}
DE 3
\end{flushleft}





0


0


\begin{flushleft}
PE 1
\end{flushleft}





0


0


\begin{flushleft}
DE 1
\end{flushleft}





\begin{flushleft}
Chemical Reaction
\end{flushleft}


\begin{flushleft}
Engineering II
\end{flushleft}





1


0


\begin{flushleft}
CLL222
\end{flushleft}





\begin{flushleft}
Chemical Reaction
\end{flushleft}


\begin{flushleft}
Engineering I
\end{flushleft}





2


0


\begin{flushleft}
CLL122
\end{flushleft}





3





3





3





3





3





4





4





0





0





3





0


2


\begin{flushleft}
CLL231
\end{flushleft}





4





1


0


\begin{flushleft}
CLL331
\end{flushleft}





4





1


0


\begin{flushleft}
CLL271
\end{flushleft}





4





3





0





3





3





8





0





0





\begin{flushleft}
OE 1
\end{flushleft}





0





\begin{flushleft}
Minor Project
\end{flushleft}





0


0


\begin{flushleft}
CLD880
\end{flushleft}





0


0


\begin{flushleft}
PE 2
\end{flushleft}





3





4





3





3





\begin{flushleft}
Introduction to Industrial
\end{flushleft}


\begin{flushleft}
Biotechnology
\end{flushleft}





3





\begin{flushleft}
Fluid-Particle Mechanics
\end{flushleft}





3





\begin{flushleft}
Fluid Mechanics for
\end{flushleft}


\begin{flushleft}
Chemical Engineers
\end{flushleft}





3





\begin{flushleft}
CLL113
\end{flushleft}





3





\begin{flushleft}
Numerical Methods in
\end{flushleft}


\begin{flushleft}
Chemical Engineering
\end{flushleft}





4





\begin{flushleft}
CLL111
\end{flushleft}





2





3





\begin{flushleft}
Material and Energy
\end{flushleft}


\begin{flushleft}
Balances
\end{flushleft}





0





0


0


\begin{flushleft}
CML100
\end{flushleft}





\begin{flushleft}
Introduction to Chemistry
\end{flushleft}





3





1


0


\begin{flushleft}
MTL101
\end{flushleft}





\begin{flushleft}
Calculus
\end{flushleft}





4





0





\begin{flushleft}
CML103
\end{flushleft}





1





4





0


0


\begin{flushleft}
CLL251
\end{flushleft}





3





1


0


\begin{flushleft}
CLL141
\end{flushleft}





4





0


0


\begin{flushleft}
CLL371
\end{flushleft}





3





3





0





3





0





0





0


3


\begin{flushleft}
HUL3XX
\end{flushleft}





\begin{flushleft}
Chemical Engineering
\end{flushleft}


\begin{flushleft}
Laboratory III
\end{flushleft}





1


0


\begin{flushleft}
CLP303
\end{flushleft}





3





1.5





4





\begin{flushleft}
Chemical Process Technology
\end{flushleft}


\begin{flushleft}
and Economics
\end{flushleft}





3





\begin{flushleft}
Introduction to Materials for
\end{flushleft}


\begin{flushleft}
Chemical Engineers
\end{flushleft}





3





\begin{flushleft}
Heat Transfer for Chemical
\end{flushleft}


\begin{flushleft}
Engineers
\end{flushleft}





3





\begin{flushleft}
Applied Chemistry: Chemistry
\end{flushleft}


\begin{flushleft}
at Interfaces
\end{flushleft}





3





\begin{flushleft}
Linear Algebra and Differential
\end{flushleft}


\begin{flushleft}
Equations
\end{flushleft}





3





\begin{flushleft}
Course-4
\end{flushleft}





\begin{flushleft}
MTL100
\end{flushleft}





\begin{flushleft}
Course-5
\end{flushleft}





0 4


\begin{flushleft}
CMP100
\end{flushleft}





2





4





1 0


\begin{flushleft}
SBL100
\end{flushleft}





\begin{flushleft}
HUL2XX
\end{flushleft}





0





4





2





0 2


\begin{flushleft}
CLL261
\end{flushleft}





4





1 0


\begin{flushleft}
CLL361
\end{flushleft}





4





3





3





1





0





0





\begin{flushleft}
Adv Trans Pheno
\end{flushleft}





0 0


\begin{flushleft}
CLL731
\end{flushleft}





\begin{flushleft}
Proc. Engg
\end{flushleft}





3





3





0 3 2.5


\begin{flushleft}
CLL703
\end{flushleft}





\begin{flushleft}
Instrumentation and
\end{flushleft}


\begin{flushleft}
Automation
\end{flushleft}





3





\begin{flushleft}
Process Dynamics and
\end{flushleft}


\begin{flushleft}
Control
\end{flushleft}





3





\begin{flushleft}
Introductory Biology for
\end{flushleft}


\begin{flushleft}
Engineers
\end{flushleft}





3





0





\begin{flushleft}
Chemistry Laboratory
\end{flushleft}





0





\begin{flushleft}
Physics Laboratory
\end{flushleft}





\begin{flushleft}
PYP100
\end{flushleft}





\begin{flushleft}
Course-6
\end{flushleft}





0





0





4





2





\begin{flushleft}
Product Realization through
\end{flushleft}


\begin{flushleft}
Manufacturing
\end{flushleft}





\begin{flushleft}
MCP101
\end{flushleft}





\begin{flushleft}
Course-7
\end{flushleft}





0





0





2





1





\begin{flushleft}
Introduction to Engineering
\end{flushleft}


\begin{flushleft}
(Non-graded)
\end{flushleft}





\begin{flushleft}
NIN100
\end{flushleft}





0


2


\begin{flushleft}
CVL100
\end{flushleft}





4





3





3





3





2





0





0





\begin{flushleft}
Ind Multiph Reac
\end{flushleft}





1


0


\begin{flushleft}
CLL733
\end{flushleft}





1


0


\begin{flushleft}
HUL2XX
\end{flushleft}





0


0


\begin{flushleft}
HUL2XX
\end{flushleft}





3





4





4





2





\begin{flushleft}
Environmental Science
\end{flushleft}





3





\begin{flushleft}
Introduction to Materials
\end{flushleft}


\begin{flushleft}
Science and Engineering
\end{flushleft}





\begin{flushleft}
APL102
\end{flushleft}





\begin{flushleft}
CLN101
\end{flushleft}





2





\begin{flushleft}
CLP301
\end{flushleft}





0





1





0


3


\begin{flushleft}
CLP302
\end{flushleft}





1.5





3





0





0





0





0


3


\begin{flushleft}
OC 1
\end{flushleft}





3





1.5





\begin{flushleft}
Chemical Engineering
\end{flushleft}


\begin{flushleft}
Laboratory II
\end{flushleft}





0





\begin{flushleft}
Chemical Engineering
\end{flushleft}


\begin{flushleft}
Laboratory I
\end{flushleft}





0





\begin{flushleft}
Introduction to Chemical
\end{flushleft}


\begin{flushleft}
Engineering (Non-graded)
\end{flushleft}





0





0





1





0.5





\begin{flushleft}
Professional Ethics and
\end{flushleft}


\begin{flushleft}
Social Responsibility-2
\end{flushleft}


\begin{flushleft}
(Non-graded)
\end{flushleft}





0 1 0.5


\begin{flushleft}
NEN100
\end{flushleft}





0





0





0





2





\begin{flushleft}
Language and
\end{flushleft}


\begin{flushleft}
Writing Skills-2
\end{flushleft}


\begin{flushleft}
(Non-Graded)
\end{flushleft}





1





0 2 1


\begin{flushleft}
NLN100
\end{flushleft}





\begin{flushleft}
Language and
\end{flushleft}


\begin{flushleft}
Writing Skills-1
\end{flushleft}


\begin{flushleft}
(Non-Graded)
\end{flushleft}





\begin{flushleft}
Professional Ethics and
\end{flushleft}


\begin{flushleft}
Social Responsibility-1
\end{flushleft}


\begin{flushleft}
(Non-graded)
\end{flushleft}





0





\begin{flushleft}
NLN100
\end{flushleft}





\begin{flushleft}
Course-8
\end{flushleft}





\begin{flushleft}
NEN100
\end{flushleft}





\begin{flushleft}
Course-9
\end{flushleft}





\begin{flushleft}
L
\end{flushleft}





0.0





6.0





15





18





16





17





18





14





12





9.5





\begin{flushleft}
Note: Courses 1-6 above are attended in the given order by half of all first year students. The other half of First year students attend the Courses 1-6 of II semester first.
\end{flushleft}





3





2





3





0


3


\begin{flushleft}
COL100
\end{flushleft}





\begin{flushleft}
Introduction to Computer
\end{flushleft}


\begin{flushleft}
Science
\end{flushleft}





4





\begin{flushleft}
Engineering Mechanics
\end{flushleft}





0


2


\begin{flushleft}
APL100
\end{flushleft}





0.5





\begin{flushleft}
PYL100
\end{flushleft}





3





\begin{flushleft}
Course-1
\end{flushleft}





\begin{flushleft}
Introduction to Engineering Electromagnetic Waves
\end{flushleft}


\begin{flushleft}
Visualization
\end{flushleft}


\begin{flushleft}
and Quantum Mechanics
\end{flushleft}





\begin{flushleft}
MCP100
\end{flushleft}





\begin{flushleft}
Course-3
\end{flushleft}





\begin{flushleft}
Introduction to Electrical
\end{flushleft}


\begin{flushleft}
Engineering
\end{flushleft}





\begin{flushleft}
Course-2
\end{flushleft}





\begin{flushleft}
ELL100
\end{flushleft}





0





0





0





1





3





2





4





4





2





1





\begin{flushleft}
T
\end{flushleft}





\begin{flushleft}
Credits
\end{flushleft}





19.0





20.5





22.0





20.5





24.0





19.0





0





24.0 185.0





22.0





23.0





22.0





25.0





22.0





26.0





22.0





\begin{flushleft}
TOTAL=185.0
\end{flushleft}





24 12.0





0





0





0





0





0





0





1





17.0 1.5 23.0





16 14.0





8





3





6





3





4





2





6





13 17.0 2.5 28.5





\begin{flushleft}
P
\end{flushleft}





\begin{flushleft}
Non-Graded
\end{flushleft}


\begin{flushleft}
Units
\end{flushleft}





\begin{flushleft}
Dual Degree Programme : B.Tech. and M.Tech. in Chemical Engineering	CH7
\end{flushleft}


\begin{flushleft}
Contact Hours
\end{flushleft}





\begin{flushleft}
\newpage
Bachelor of Technology in Civil Engineering
\end{flushleft}





\begin{flushleft}
Programme Code: CE1
\end{flushleft}





\begin{flushleft}
Department of Civil Engineering
\end{flushleft}


\begin{flushleft}
The overall Credit Structure
\end{flushleft}





\begin{flushleft}
Departmental Electives
\end{flushleft}





\begin{flushleft}
Course Category	
\end{flushleft}


\begin{flushleft}
Credits
\end{flushleft}


\begin{flushleft}
Institute Core Courses
\end{flushleft}


\begin{flushleft}
Basic Sciences (BS)		 22
\end{flushleft}


\begin{flushleft}
Engineering Arts and Science (EAS)		 18
\end{flushleft}


\begin{flushleft}
Humanities and Social Sciences (HuSS)		 15
\end{flushleft}


\begin{flushleft}
Programme-linked Courses		10
\end{flushleft}


\begin{flushleft}
Departmental Courses
\end{flushleft}


\begin{flushleft}
Departmental Core 		 66
\end{flushleft}


\begin{flushleft}
Departmental Electives		 14
\end{flushleft}


\begin{flushleft}
Open Category Courses		 10
\end{flushleft}


\begin{flushleft}
Total Graded Credit requirement		 155
\end{flushleft}


\begin{flushleft}
Non Graded Units		 15
\end{flushleft}





\begin{flushleft}
CVL284	 Fundamentals of Geographic 	
\end{flushleft}


	


\begin{flushleft}
Information Systems
\end{flushleft}


\begin{flushleft}
CVL311	 Industrial Waste Management	
\end{flushleft}


\begin{flushleft}
CVL312	 Environmental Assessment Methodologies	
\end{flushleft}


\begin{flushleft}
CVL313	 Air and Noise Pollution	
\end{flushleft}


\begin{flushleft}
CVL344	 Construction Project Management	
\end{flushleft}


\begin{flushleft}
CVL361	 Introduction to Railway Engineering	
\end{flushleft}


\begin{flushleft}
CVL382	 Groundwater	
\end{flushleft}


\begin{flushleft}
CVL383	 Water Resources Systems	
\end{flushleft}


\begin{flushleft}
CVL384	 Urban Hydrology	
\end{flushleft}


\begin{flushleft}
CVL385	 Frequency Analysis in Hydrology	
\end{flushleft}


\begin{flushleft}
CVL386	 Fundamentals of Remote Sensing	
\end{flushleft}


\begin{flushleft}
Institute Core: Basic Sciences
\end{flushleft}


\begin{flushleft}
CVD412	 B.Tech. Project Part-II	
\end{flushleft}


\begin{flushleft}
CVL421	 Ground Engineering	
\end{flushleft}


\begin{flushleft}
CML100	 General Chemistry	
\end{flushleft}


3	 0	 0	 3


\begin{flushleft}
CVL422	 Rock Engineering	
\end{flushleft}


\begin{flushleft}
CMP100	Chemistry Laboratory	
\end{flushleft}


0	0	4	2


\begin{flushleft}
CVL423	 Soil Dynamics	
\end{flushleft}


\begin{flushleft}
MTL100	 Calculus	
\end{flushleft}


3	1	0	4


\begin{flushleft}
MTL101	 Linear Algebra and Differential Equations	
\end{flushleft}


3	 1	 0	 4


\begin{flushleft}
CVL424	 Environmental Geotechniques \& Geosyntheses	
\end{flushleft}


\begin{flushleft}
PYL100	 Electromagnetic Waves and Quantum 	
\end{flushleft}


3	 0	 0	 3


\begin{flushleft}
CVL431	 Design of Foundations \& Retaining	
\end{flushleft}


\begin{flushleft}
	Mechanics	
\end{flushleft}


\begin{flushleft}
	Structures
\end{flushleft}


\begin{flushleft}
PYP100	 Physics Laboratory	
\end{flushleft}


0	0	4	2


\begin{flushleft}
CVL432	 Stability of Slopes	
\end{flushleft}


\begin{flushleft}
SBL100	 Introductory Biology for Engineers	
\end{flushleft}


3	 0	 2	 4


\begin{flushleft}
CVL433	 FEM in Geotechnical Engineering	
\end{flushleft}


	


\begin{flushleft}
Total Credits				22	 CVP434	 Geotechnical Design Studio	
\end{flushleft}


\begin{flushleft}
CVL435	 Underground Structures	
\end{flushleft}


\begin{flushleft}
Institute Core: Engineering Arts and Sciences
\end{flushleft}


\begin{flushleft}
CVL441	 Structural Design	
\end{flushleft}


\begin{flushleft}
APL100	 Engineering Mechanics	
\end{flushleft}


3	1	0	4


\begin{flushleft}
CVL442	 Structural Analysis-III	
\end{flushleft}


\begin{flushleft}
COL100	 Introduction to Computer Science	
\end{flushleft}


3	 0	 2	 4


\begin{flushleft}
CVL443	 Prestressed Concrete \& Industrial Structures	
\end{flushleft}


\begin{flushleft}
CVL100	 Environmental Science	
\end{flushleft}


2	0	0	2


\begin{flushleft}
CVL461	 Logistics and Freight Transport	
\end{flushleft}


\begin{flushleft}
ELL100	 Introduction to Electrical Engineering	
\end{flushleft}


3	 0	 2	 4


\begin{flushleft}
CVL462	 Introduction to Intelligent Transportation Systems	
\end{flushleft}


\begin{flushleft}
MCP100	Engineering Visualization	
\end{flushleft}


0	0	4	2


\begin{flushleft}
CVL481	 Water Resources Management	
\end{flushleft}


\begin{flushleft}
MCP101	 Product Realization through Manufacturing	 0	 0	 4	 2
\end{flushleft}


\begin{flushleft}
CVL482	 Water Power Engineering	
\end{flushleft}


	


\begin{flushleft}
Total Credits				18
\end{flushleft}


\begin{flushleft}
CVL483	 Groundwater \& Surface-water Pollution	
\end{flushleft}


\begin{flushleft}
CVP484	 Computational Aspects in Water Resources	
\end{flushleft}


\begin{flushleft}
Programme-Linked Basic / Engineering Arts / Sciences Core
\end{flushleft}


\begin{flushleft}
CVL485	 River Mechanics	
\end{flushleft}


\begin{flushleft}
APL 107	 Mechanics of Fluids	
\end{flushleft}


3	 1	 2	 5


\begin{flushleft}
CVL486	 Geo-informatics	
\end{flushleft}


\begin{flushleft}
APL 108	 Mechanics of Solids	
\end{flushleft}


3	 1	 2	 5


\begin{flushleft}
CVL721	 Solid Waste Engineering	
\end{flushleft}


	


\begin{flushleft}
Total Credits				10
\end{flushleft}


\begin{flushleft}
CVL724	 Environmental systems analysis	
\end{flushleft}


\begin{flushleft}
CVL727	 Environmental risk assessment
\end{flushleft}


	


\begin{flushleft}
Humanities and Social Sciences
\end{flushleft}


\begin{flushleft}
CVL728	 Environmental Quality Modeling	
\end{flushleft}


\begin{flushleft}
Courses from Humanities, Social Sciences and
\end{flushleft}


\begin{flushleft}
CVL740	 Pavement Materials and Design of	
\end{flushleft}


\begin{flushleft}
Management offered under this category		
\end{flushleft}


15


\begin{flushleft}
	Pavements
\end{flushleft}


\begin{flushleft}
CVL741	 Urban and Regional Transportation Planning	
\end{flushleft}


\begin{flushleft}
Departmental Core
\end{flushleft}


\begin{flushleft}
CVL742	 Traffic Engineering	
\end{flushleft}


\begin{flushleft}
CVL111	 Elements of Surveying	
\end{flushleft}


3	0	2	4


\begin{flushleft}
CVL743	 Airport Planning and Design	
\end{flushleft}


\begin{flushleft}
CVL121	 Engineering Geology	
\end{flushleft}


3	 0	 0	 3


\begin{flushleft}
CVL744	 Transportation Infrastructure Design	
\end{flushleft}


\begin{flushleft}
CVP121	 Engineering Geology Lab	
\end{flushleft}


0	 0	 2	 1


\begin{flushleft}
CVL746	 Public Transportation Systems	
\end{flushleft}


\begin{flushleft}
CVL141	 Civil Engineering Materials	
\end{flushleft}


3	0	0	3


\begin{flushleft}
CVL763	 Analytical and Numerical Methods for	
\end{flushleft}


\begin{flushleft}
CVL212	 Environmental Engineering	
\end{flushleft}


3	0	2	4


	


\begin{flushleft}
Structural Engineering
\end{flushleft}


\begin{flushleft}
CVL222	 Soil Mechanics	
\end{flushleft}


3	0	0	3


\begin{flushleft}
CVL765	 Concrete Mechanics	
\end{flushleft}


\begin{flushleft}
CVP222	Soil Mechanics Lab	
\end{flushleft}


0	0	2	1


\begin{flushleft}
CVL242	 Structural Analysis-I	
\end{flushleft}


3	0	0	3


\begin{flushleft}
CVL766	 Design of Bridge Structures	
\end{flushleft}


\begin{flushleft}
CVP242	Structural Analysis Lab	
\end{flushleft}


0	0	2	1


\begin{flushleft}
CVL768	 Design of Masonry Structures	
\end{flushleft}


\begin{flushleft}
CVL243	 RC Design	
\end{flushleft}


3	0	0	3


\begin{flushleft}
CVL769	 Design of Tall Buildings	
\end{flushleft}


\begin{flushleft}
CVP243	 Structures \& Material (Concrete) Lab	
\end{flushleft}


0	 0	 3	 1.5


\begin{flushleft}
CVL770	 Prestressed and Composite Structures	
\end{flushleft}


\begin{flushleft}
CVL244	 Construction Practices	
\end{flushleft}


2	0	0	2


\begin{flushleft}
CVL771	 Advanced Concrete Technology 	
\end{flushleft}


\begin{flushleft}
CVL245	 Construction Management	
\end{flushleft}


2	0	0	2


\begin{flushleft}
CVL820	 Environmental impact assessment
\end{flushleft}


	


\begin{flushleft}
CVL261	 Introduction to Transportation Engineering	 3	0	0	3
\end{flushleft}


\begin{flushleft}
CVL822	 Emerging Technologies for Environmental	
\end{flushleft}


\begin{flushleft}
CVP261	Transportation Engineering Lab	
\end{flushleft}


0	0	2	1


\begin{flushleft}
	Management
\end{flushleft}


\begin{flushleft}
CVL281	 Hydraulics	
\end{flushleft}


3	1	0	4


\begin{flushleft}
CVL823	 Thermal Techniques for Waste Management	
\end{flushleft}


\begin{flushleft}
CVP281	Hydraulics Lab	
\end{flushleft}


0	0	2	1


\begin{flushleft}
CVL824	 Life Cycle Analysis and Design for Environment	
\end{flushleft}


\begin{flushleft}
CVL282	 Engineering Hydrology	
\end{flushleft}


3	0	2	4


\begin{flushleft}
CVL837	 Mechanics of Sediment Transport	
\end{flushleft}


\begin{flushleft}
CVL321	 Geotechnical Engineering	
\end{flushleft}


3	 1	 0	 4


\begin{flushleft}
CVL841	 Advanced Transportation Modelling	
\end{flushleft}


\begin{flushleft}
CVP321	 Geotechnical Engineering Lab	
\end{flushleft}


0	 0	 2	 1


\begin{flushleft}
CVL842	 Geometric Design of Roads	
\end{flushleft}


\begin{flushleft}
CVL341	 Structural Analysis-II	
\end{flushleft}


3	0	0	3


\begin{flushleft}
CVL342	 Steel Design	
\end{flushleft}


3	0	0	3


\begin{flushleft}
CVL847	 Transportation Economics	
\end{flushleft}


\begin{flushleft}
CVP342	 Structures \& Material (Steel) Lab	
\end{flushleft}


0	 0	 2	 1


\begin{flushleft}
CVL857	 Structural Safety and Reliability	
\end{flushleft}


\begin{flushleft}
CVL381	 Design of Hydraulic Structures	
\end{flushleft}


3	 0	 2	 4


\begin{flushleft}
CVL858	 Theory of Plates and Shells	
\end{flushleft}


\begin{flushleft}
CVD411	 B.Tech. Project Part-I	
\end{flushleft}


0	0	8	4


\begin{flushleft}
CVL859	 Theory of Structural Stability	
\end{flushleft}


\begin{flushleft}
CVP441	 Structural Design \& Detailing	
\end{flushleft}


0	 0	 3	 1.5


\begin{flushleft}
CVL862	 Design of Offshore Structures	
\end{flushleft}


	


\begin{flushleft}
Total Credits				66	 CVL866	 Wind Resistant Design of Structures	
\end{flushleft}





46





2	 0	 2	 3


3	0	0	3


3	0	0	3


3	 0	 0	 3


3	0	0	3


3	 0	 0	 3


2	 0	 0	 2


2	0	0	2


2	0	0	2


2	0	0	2


2	 0	 2	 3


0	 0	 12	6


3	 0	 0	 3


3	0	0	3


3	0	0	3


3	0	0	3


3	 0	 0	 3


2	0	0	2


3	 0	 0	 3


0	 0	 4	 2


2	0	0	2


3	0	0	3


3	0	0	3


3	 0	 0	 3


3	0	0	3


3	0	0	3


3	0	0	3


2	0	2	3


2	 0	 0	 2


1	 0	 4	 3


2	0	2	3


2	 0	 2	 3


3	0	0	3


3	0	0	3


3	 0	 0	 3


3	0	0	3


3	 0	 2	 4


3	 0	 2	 4


3	 0	 2	 4


3	 0	 0	 3


2	0	2	3


3	0	0	3


3	0	0	3


3	0	0	3


3	 0	 0	 3


3	 0	 0	 3


3	0	0	3


3	 0	 0	 3


3	0	0	3


3	 0	 0	 3


3	0	0	3


3	 0	 0	 3


3	0	0	3


2	0	2	3


2	0	2	3


2	 0	 2	 3


3	0	0	3


3	 0	 0	 3


3	 0	 0	 3


3	 0	 0	 3


3	 0	 0	 3


3	 0	 0	 3





\newpage
47





\begin{flushleft}
Semester
\end{flushleft}





\begin{flushleft}
VIII
\end{flushleft}





\begin{flushleft}
VII
\end{flushleft}





\begin{flushleft}
VI
\end{flushleft}





\begin{flushleft}
V
\end{flushleft}





\begin{flushleft}
IV
\end{flushleft}





\begin{flushleft}
III
\end{flushleft}





\begin{flushleft}
II
\end{flushleft}





\begin{flushleft}
I
\end{flushleft}





2





4





3





3





0





0





3





1 0


\begin{flushleft}
MTL101
\end{flushleft}





4





3





1





0





4





\begin{flushleft}
Linear Algebra and
\end{flushleft}


\begin{flushleft}
Differential Equations
\end{flushleft}





3





\begin{flushleft}
Calculus
\end{flushleft}





\begin{flushleft}
CVL141
\end{flushleft}





\begin{flushleft}
Construction
\end{flushleft}


\begin{flushleft}
Management
\end{flushleft}





0 2 4


\begin{flushleft}
CVL245
\end{flushleft}





0 3 4.5 2 0 0 2


\begin{flushleft}
CVL212
\end{flushleft}


\begin{flushleft}
CVL244
\end{flushleft}





3





2





2





3





0





2





0 2


\begin{flushleft}
DE 4
\end{flushleft}





0 2


\begin{flushleft}
DE 1
\end{flushleft}





3





3





4





3





3





2





0





0





0 0


\begin{flushleft}
DE 5
\end{flushleft}





0 0


\begin{flushleft}
DE 2
\end{flushleft}





3





3





2





\begin{flushleft}
Environmental Engineering Construction Practices
\end{flushleft}





3





\begin{flushleft}
RC Design + Lab
\end{flushleft}





3 0 2 4


\begin{flushleft}
CVL243+CVP243
\end{flushleft}





0 2


\begin{flushleft}
CVL282
\end{flushleft}





4





3





3





3





0





0





3





0 0 3


\begin{flushleft}
OC 2/ DE 3
\end{flushleft}





0 2 4


\begin{flushleft}
DE 3 / OC 2
\end{flushleft}





\begin{flushleft}
Steel Design + Lab
\end{flushleft}





3 0 2 4


\begin{flushleft}
CVL342+CVP342
\end{flushleft}





\begin{flushleft}
Engineering Hydrology
\end{flushleft}





3





\begin{flushleft}
Intro. to Transportation
\end{flushleft}


\begin{flushleft}
Engg + Lab
\end{flushleft}





1 2


\begin{flushleft}
CVL381
\end{flushleft}





5





3





3





2





1





0





\begin{flushleft}
OC 3
\end{flushleft}





0





4





4





\begin{flushleft}
Design of Hydraulic
\end{flushleft}


\begin{flushleft}
Structures
\end{flushleft}





3





\begin{flushleft}
Geotechnical Engineering
\end{flushleft}


\begin{flushleft}
+ Lab
\end{flushleft}





3 1 2 5


\begin{flushleft}
CVL321+CVP321
\end{flushleft}





\begin{flushleft}
Hydraulics + Lab
\end{flushleft}





\begin{flushleft}
Soil Mechanics + Lab
\end{flushleft}





\begin{flushleft}
Structural Analysis I +
\end{flushleft}


\begin{flushleft}
Lab
\end{flushleft}





\begin{flushleft}
Mechanics of Fluids
\end{flushleft}





\begin{flushleft}
APL107
\end{flushleft}





3 1 2 5


\begin{flushleft}
CVL281+CVP281
\end{flushleft}





\begin{flushleft}
Engineering Geology Civil Engineering Materials
\end{flushleft}





\begin{flushleft}
CVL121
\end{flushleft}





3 0 2 4 3 0 0 3 3 0 0 3


\begin{flushleft}
CVL222+ CVP222 CVL242+CVP242 CVL261+CVP261
\end{flushleft}





\begin{flushleft}
Elements of Surveying
\end{flushleft}





\begin{flushleft}
CVL111
\end{flushleft}





0





0 0


\begin{flushleft}
CML100
\end{flushleft}





\begin{flushleft}
Course-3
\end{flushleft}





\begin{flushleft}
Introduction to Chemistry
\end{flushleft}





3





\begin{flushleft}
Course-4
\end{flushleft}





\begin{flushleft}
MTL100
\end{flushleft}





\begin{flushleft}
Course-5
\end{flushleft}





0 4 2


\begin{flushleft}
CMP100
\end{flushleft}





0





0





4





2





\begin{flushleft}
Chemistry Laboratory
\end{flushleft}





0





\begin{flushleft}
Physics Laboratory
\end{flushleft}





\begin{flushleft}
PYP100
\end{flushleft}





\begin{flushleft}
Course-6
\end{flushleft}





0





0





4





2





\begin{flushleft}
Product Realization
\end{flushleft}


\begin{flushleft}
through Manufacturing
\end{flushleft}





\begin{flushleft}
MCP101
\end{flushleft}





0





\begin{flushleft}
NIN100
\end{flushleft}





\begin{flushleft}
Course-7
\end{flushleft}





0





2





\begin{flushleft}
Introduction to
\end{flushleft}


\begin{flushleft}
Engineering
\end{flushleft}


\begin{flushleft}
(Non-graded)
\end{flushleft}





1





\begin{flushleft}
NEN100
\end{flushleft}





\begin{flushleft}
Course-8
\end{flushleft}





0 1 0.5


\begin{flushleft}
NEN100
\end{flushleft}





0





0





1





0.5





\begin{flushleft}
Professional Ethics and
\end{flushleft}


\begin{flushleft}
Social Responsibility-2
\end{flushleft}


\begin{flushleft}
(Non-graded)
\end{flushleft}





0





\begin{flushleft}
Professional Ethics and
\end{flushleft}


\begin{flushleft}
Social Responsibility-1
\end{flushleft}


\begin{flushleft}
(Non-graded)
\end{flushleft}





0





0





0





2





\begin{flushleft}
Language and
\end{flushleft}


\begin{flushleft}
Writing Skills-2
\end{flushleft}


\begin{flushleft}
(Non-Graded)
\end{flushleft}





0 2


\begin{flushleft}
NLN100
\end{flushleft}





\begin{flushleft}
Language and
\end{flushleft}


\begin{flushleft}
Writing Skills-1
\end{flushleft}


\begin{flushleft}
(Non-Graded)
\end{flushleft}





\begin{flushleft}
NLN100
\end{flushleft}





\begin{flushleft}
Course-9
\end{flushleft}





1





12





2





1 9.5 1





\begin{flushleft}
T
\end{flushleft}





1 2 5


\begin{flushleft}
CVL100
\end{flushleft}


0 0 2


\begin{flushleft}
CVL341
\end{flushleft}


3





0 0 3


\begin{flushleft}
CVP441
\end{flushleft}





0 0


\begin{flushleft}
OC 1
\end{flushleft}





0





0





1 0 4


\begin{flushleft}
SBL100
\end{flushleft}





1 0 4


\begin{flushleft}
HUL2XX
\end{flushleft}





\begin{flushleft}
HUL2XX
\end{flushleft}





3





3





0





0





3





1 0 4


\begin{flushleft}
HUL3XX
\end{flushleft}





0 2 4


\begin{flushleft}
HUL2XX
\end{flushleft}





\begin{flushleft}
Introductory Biology
\end{flushleft}


\begin{flushleft}
for Engineers
\end{flushleft}





3





3





3 1.5 3





\begin{flushleft}
Structural Design and
\end{flushleft}


\begin{flushleft}
Detailing
\end{flushleft}





3





3





\begin{flushleft}
Structural Analysis II
\end{flushleft}





2





\begin{flushleft}
Environmental Science
\end{flushleft}





3





\begin{flushleft}
Mechanics of Solids
\end{flushleft}





\begin{flushleft}
APL108
\end{flushleft}





\begin{flushleft}
CVN121
\end{flushleft}





0





0





2





0





8





\begin{flushleft}
B.Tech. Project
\end{flushleft}





\begin{flushleft}
CVD411
\end{flushleft}





0





4





1





\begin{flushleft}
Introduction to
\end{flushleft}


\begin{flushleft}
Civil Engineering
\end{flushleft}


\begin{flushleft}
(Non-graded)
\end{flushleft}





11





11





17





17





17





18





1





0





1





1





2





3





\begin{flushleft}
P
\end{flushleft}





\begin{flushleft}
Credits
\end{flushleft}





0





0





0





0





1





24.0





24.0





27.0





27.0





29.0





2 13.0 0 14.0 155.0


\begin{flushleft}
TOTAL=155.0
\end{flushleft}





13 17.5





6 21.0





9 22.5





8 23.0





6 24.0





6 17.0 1.5 23.0





13 17.0 2.5 28.5





\begin{flushleft}
Note: Courses 1-6 above are attended in the given order by half of all first year students. The other half of First year students attend the Courses 1-6 of II semester first.
\end{flushleft}





4





3





0





3





1





\begin{flushleft}
Introduction to
\end{flushleft}


\begin{flushleft}
Computer Science
\end{flushleft}





\begin{flushleft}
Engineering Mechanics
\end{flushleft}





4 0.5 0 3 2


\begin{flushleft}
COL100
\end{flushleft}





\begin{flushleft}
Introduction to Electrical
\end{flushleft}


\begin{flushleft}
Engineering
\end{flushleft}





0 2


\begin{flushleft}
APL100
\end{flushleft}





\begin{flushleft}
Electromagnetic Waves and
\end{flushleft}


\begin{flushleft}
Quantum Mechanics
\end{flushleft}





\begin{flushleft}
MCP100
\end{flushleft}





\begin{flushleft}
Introduction to
\end{flushleft}


\begin{flushleft}
Engineering
\end{flushleft}


\begin{flushleft}
Visualization
\end{flushleft}





\begin{flushleft}
Course-1
\end{flushleft}





3





\begin{flushleft}
PYL100
\end{flushleft}





\begin{flushleft}
Course-2
\end{flushleft}





\begin{flushleft}
ELL100
\end{flushleft}





\begin{flushleft}
L
\end{flushleft}





\begin{flushleft}
Non-Graded
\end{flushleft}


\begin{flushleft}
Units
\end{flushleft}





\begin{flushleft}
B.Tech. in Civil Engineering	CE1
\end{flushleft}


\begin{flushleft}
Contact Hours
\end{flushleft}





\begin{flushleft}
\newpage
Programme Code: CS1
\end{flushleft}





\begin{flushleft}
Bachelor of Technology in Computer Science and Engineering
\end{flushleft}


\begin{flushleft}
Department of Computer Science and Engineering
\end{flushleft}


\begin{flushleft}
The overall Credit Structure
\end{flushleft}





\begin{flushleft}
COL703***	Logic for Computer Science	
\end{flushleft}


3	 0	 2	 4


\begin{flushleft}
COL718	 Architecture of High Performance Computers	 3	 0	 2	 4
\end{flushleft}


\begin{flushleft}
COL719	 Synthesis of Digital Systems	
\end{flushleft}


3	 0	 2	 4


\begin{flushleft}
COL722	 Introduction to Compressed Sensing	
\end{flushleft}


3	 0	 0	 3


\begin{flushleft}
COL724	Advanced Computer Networks	
\end{flushleft}


3	0	2	4


\begin{flushleft}
COL726	Numerical Algorithms	
\end{flushleft}


3	0	2	4


\begin{flushleft}
COL728	Compiler Design	
\end{flushleft}


3	0	3	4.5


\begin{flushleft}
COL729	Compiler Optimization	
\end{flushleft}


3	0	3	4.5


\begin{flushleft}
COL730	Parallel Programming	
\end{flushleft}


3	0	2	4


\begin{flushleft}
COL732	 Virtualization and Cloud Computing	
\end{flushleft}


3	 0	 2	 4


\begin{flushleft}
COL733	Cloud Computing Technology Fundamentals	3	0	2	4
\end{flushleft}


\begin{flushleft}
COL740	Software Engineering	
\end{flushleft}


3	0	2	4


\begin{flushleft}
COL750	 Foundations of Automatic Verification	
\end{flushleft}


3	 0	 2	 4


\begin{flushleft}
COL751	 Algorithmic Graph Theory	
\end{flushleft}


3	 0	 0	 3


\begin{flushleft}
COL752	 Geometric Algorithms	
\end{flushleft}


3	 0	 0	 3


\begin{flushleft}
COL753	Complexity Theory	
\end{flushleft}


3	0	0	3


\begin{flushleft}
COL754	Approximation Algorithms	
\end{flushleft}


3	0	0	3


\begin{flushleft}
COL756	Mathematical Programming	
\end{flushleft}


3	0	0	3


\begin{flushleft}
COL757	Model Centric Algorithm Design	
\end{flushleft}


3	0	2	4


\begin{flushleft}
COL758	Advanced Algorithms	
\end{flushleft}


3	0	2	4


\begin{flushleft}
COL759	 Cryptography \& Computer Security	
\end{flushleft}


3	 0	 0	 3


\begin{flushleft}
COL760	Advanced Data Management	
\end{flushleft}


3	0	2	4


\begin{flushleft}
COL762	Database Implementation	
\end{flushleft}


3	0	2	4


\begin{flushleft}
COL765	 Logic and Functional Programming	
\end{flushleft}


3	 0	 2	 4


\begin{flushleft}
COL768	Wireless Networks	
\end{flushleft}


3	0	2	4


\begin{flushleft}
COL770	 Advanced Artificial Intelligence	
\end{flushleft}


3	 0	 2	 4


\begin{flushleft}
COL772	Natural Language Processing	
\end{flushleft}


3	0	2	4


\begin{flushleft}
COL774	Machine Learning	
\end{flushleft}


3	0	2	4


\begin{flushleft}
COL776	 Learning Probabilistic Graphical Models	
\end{flushleft}


3	 0	 2	 4


\begin{flushleft}
COL780	Computer Vision	
\end{flushleft}


3	0	2	4


\begin{flushleft}
COL781	 Computer Graphics	
\end{flushleft}


3	 0	 3	 4.5


\begin{flushleft}
COL783	Digital Image Analysis	
\end{flushleft}


3	0	3	4.5


\begin{flushleft}
COL786	 Advanced Functional Brain Imaging	
\end{flushleft}


3	 0	 2	 4


\begin{flushleft}
COL788	 Advanced Topics in Embedded Computing	
\end{flushleft}


3	 0	 0	 3


\begin{flushleft}
COL860	 Special Topics in Parallel Computation	
\end{flushleft}


3	 0	 0	 3


\begin{flushleft}
COL861	 Special Topics in Hardware Systems	
\end{flushleft}


3	 0	 0	 3


\begin{flushleft}
COL862	 Special Topics in Software Systems	
\end{flushleft}


3	 0	 0	 3


\begin{flushleft}
COL863	 Special Topics in Theoretical Computer Science	 3	 0	 0	 3
\end{flushleft}


\begin{flushleft}
COL864	 Special Topics in Artificial Intelligence	
\end{flushleft}


3	 0	 0	 3


\begin{flushleft}
COL865	Special Topics in Computer Applications	
\end{flushleft}


3	0	0	3


\begin{flushleft}
COL866	Special Topics in Algorithms	
\end{flushleft}


3	0	0	3


\begin{flushleft}
COL867	 Special Topics in High Speed Networks	
\end{flushleft}


3	 0	 0	 3


\begin{flushleft}
COL868	 Special Topics in Database Systems	
\end{flushleft}


3	 0	 0	 3


\begin{flushleft}
COL869	Special Topics in Concurrency	
\end{flushleft}


3	0	0	3


\begin{flushleft}
COL870	 Special Topics in Machine Learning	
\end{flushleft}


3	 0	 0	 3


\begin{flushleft}
COL871	 Special Topics in programming 	
\end{flushleft}


3	 0	 0	 3


	


\begin{flushleft}
languages \& Compilers
\end{flushleft}


\begin{flushleft}
COL872	Special Topics in Cryptography	
\end{flushleft}


3	0	0	3


\begin{flushleft}
COV877	 Special Module on Visual Computing	
\end{flushleft}


1	 0	 0	 1


\begin{flushleft}
COV878	 Special Module in Machine Learning	
\end{flushleft}


1	 0	 0	 1


\begin{flushleft}
COV879	 Special Module in Financial Algorithms	
\end{flushleft}


2	 0	 0	 2


\begin{flushleft}
COV880	 Special Module in Parallel Computation	
\end{flushleft}


1	 0	 0	 1


\begin{flushleft}
COV881	 Special Module in Hardware Systems	
\end{flushleft}


1	 0	 0	 1


\begin{flushleft}
COV882	 Special Module in Software Systems	
\end{flushleft}


1	 0	 0	 1


\begin{flushleft}
COV883	 Special Module in Theoretical Computer Science	 1	 0	 0	 1
\end{flushleft}


\begin{flushleft}
COV884	 Special Module in Artificial Intelligence	
\end{flushleft}


1	 0	 0	 1


\begin{flushleft}
COV885	 Special Module in Computer Applications	
\end{flushleft}


1	 0	 0	 1


\begin{flushleft}
COV886	Special Module in Algorithms	
\end{flushleft}


1	0	0	1


\begin{flushleft}
COV887	 Special Module in High Speed Networks	
\end{flushleft}


1	 0	 0	 1


\begin{flushleft}
COV888	 Special Module in Database Systems	
\end{flushleft}


1	 0	 0	 1


\begin{flushleft}
COV889	 Special Module in Concurrency	
\end{flushleft}


1	 0	 0	 1


\begin{flushleft}
SIL765	 Networks \& System Security	
\end{flushleft}


3	 0	 2	 4


\begin{flushleft}
SIL769	 Internet Traffic -Measurement, Modeling \& Analysis	3	0	2	4
\end{flushleft}


\begin{flushleft}
SIL801	 Special Topics in Multimedia System	
\end{flushleft}


3	 0	 0	 3


\begin{flushleft}
SIL802	 Special Topics in Web Based Computing	
\end{flushleft}


3	 0	 0	 3


\begin{flushleft}
SIV813	 Applications of Computer in Medicines	
\end{flushleft}


1	 0	 0	 1


\begin{flushleft}
SIV861	 Information and Comm Technologies for Development	1	0	0	1
\end{flushleft}


\begin{flushleft}
SIV864	 Special Module on Media Processing \& Communication	1	0	0	1
\end{flushleft}


\begin{flushleft}
SIV895	 Special Module on Intelligent Information Processing	1	0	0	1
\end{flushleft}


\begin{flushleft}
*One of COL333 or COL362 will be considered as DC and other will
\end{flushleft}


\begin{flushleft}
be considered as DE
\end{flushleft}


\begin{flushleft}
**DC for CS1 students with specialization, DE for other CS1 students
\end{flushleft}


\begin{flushleft}
but with at most 4 credits counted towards DE.
\end{flushleft}


\begin{flushleft}
***DC for CS1 students with specialization.
\end{flushleft}





\begin{flushleft}
Course Category	
\end{flushleft}


\begin{flushleft}
Credits
\end{flushleft}


\begin{flushleft}
Institute Core Courses
\end{flushleft}


\begin{flushleft}
Basic Sciences (BS)		 22
\end{flushleft}


\begin{flushleft}
Engineering Arts and Science (EAS)		 18
\end{flushleft}


\begin{flushleft}
Humanities and Social Sciences (HuSS)		 15
\end{flushleft}


\begin{flushleft}
Programme-linked Courses		14
\end{flushleft}


\begin{flushleft}
Departmental Courses
\end{flushleft}


\begin{flushleft}
Departmental Core 		 55
\end{flushleft}


\begin{flushleft}
Departmental Electives		 11
\end{flushleft}


\begin{flushleft}
Open Category Courses		 10
\end{flushleft}


\begin{flushleft}
Total Graded Credit requirement		 145
\end{flushleft}


\begin{flushleft}
Non Graded Units		 15
\end{flushleft}


\begin{flushleft}
Institute Core: Basic Sciences
\end{flushleft}


\begin{flushleft}
CML100	 General Chemistry	
\end{flushleft}


3	 0	 0	 3


\begin{flushleft}
CMP100	Chemistry Laboratory	
\end{flushleft}


0	0	4	2


\begin{flushleft}
MTL100	 Calculus	
\end{flushleft}


3	1	0	4


\begin{flushleft}
MTL101	 Linear Algebra and Differential Equations	
\end{flushleft}


3	 1	 0	 4


\begin{flushleft}
PYL100	 Electromagnetic Waves and Quantum Mechanics	3	0	0	3
\end{flushleft}


\begin{flushleft}
PYP100	 Physics Laboratory	
\end{flushleft}


0	0	4	2


\begin{flushleft}
SBL100	 Introductory Biology for Engineers	
\end{flushleft}


3	 0	 2	 4


	


\begin{flushleft}
Total Credits				22
\end{flushleft}


\begin{flushleft}
Institute Core: Engineering Arts and Sciences
\end{flushleft}


\begin{flushleft}
APL100	 Engineering Mechanics	
\end{flushleft}


3	1	0	4


\begin{flushleft}
CVL100	 Environmental Science	
\end{flushleft}


2	0	0	2


\begin{flushleft}
COL100	 Introduction to Computer Science	
\end{flushleft}


3	 0	 2	 4


\begin{flushleft}
ELL100	 Introduction to Electrical Engineering	
\end{flushleft}


3	 0	 2	 4


\begin{flushleft}
MCP100	Engineering Visualization	
\end{flushleft}


0	0	4	2


\begin{flushleft}
MCP101	 Product Realization through Manufacturing	 0	 0	 4	 2
\end{flushleft}


	


\begin{flushleft}
Total Credits				18
\end{flushleft}


\begin{flushleft}
Programme-Linked Basic / Engineering Arts / Sciences Core
\end{flushleft}


\begin{flushleft}
ELL205	 Signals and Systems	
\end{flushleft}


3	1	0	4


\begin{flushleft}
MTL103	 Optimization Methods and Applications	
\end{flushleft}


3	0	0	3


\begin{flushleft}
MTL104	 Linear Algebra and Applications	
\end{flushleft}


3	0	0	3


\begin{flushleft}
MTL105	 Algebra	
\end{flushleft}


3	0	0	3


\begin{flushleft}
MTL106	 Probability and Stochastic Processes	
\end{flushleft}


3	 1	 0	 4


\begin{flushleft}
PYL102	 Principles of Electronic Materials	
\end{flushleft}


3	 0	 0	 3


\begin{flushleft}
PYL103	 Physics of Nanomaterials	
\end{flushleft}


3	0	0	3


	


\begin{flushleft}
Total Credits				14
\end{flushleft}


\begin{flushleft}
Humanities and Social Sciences
\end{flushleft}


\begin{flushleft}
Courses from Humanities, Social Sciences and Management 	
\end{flushleft}


\begin{flushleft}
offered under this category				
\end{flushleft}


15


\begin{flushleft}
Departmental Core
\end{flushleft}


\begin{flushleft}
COL106	Data Structures and Algorithms	
\end{flushleft}


3	0	4	5


\begin{flushleft}
COL202	 Discrete Mathematical Structures 	
\end{flushleft}


3	 1	 0	 4


\begin{flushleft}
COL215	 Digital Logic and System Design	
\end{flushleft}


3	 0	 4	 5


\begin{flushleft}
COL216	Computer Architecture	
\end{flushleft}


3	0	2	4


\begin{flushleft}
COL226	Programming Languages	
\end{flushleft}


3	0	4	5


\begin{flushleft}
COP290	Design Practices	
\end{flushleft}


0	0	6	3


\begin{flushleft}
COL331	Operating Systems	
\end{flushleft}


3	0	4	5


\begin{flushleft}
COL333	 Principles of Artificial Intelligence*	
\end{flushleft}


3	 0	 2	 4


\begin{flushleft}
COL334	Computer Networks	
\end{flushleft}


3	0	2	4


\begin{flushleft}
COL351	 Analysis and Design of Algorithms	
\end{flushleft}


3	 1	 0	 4


\begin{flushleft}
COL352	 Introduction to Automata and Theory of Computation	3	0	0	3
\end{flushleft}


\begin{flushleft}
COL362	 Introduction to Database Management Systems*	
\end{flushleft}


3	0	2	4


\begin{flushleft}
COL380	 Introduction to Parallel and Distributed Programming	2	0	2	3
\end{flushleft}


\begin{flushleft}
COD490	 B.Tech. Project 	
\end{flushleft}


0	 0	 12	6


\begin{flushleft}
COD492	 B.Tech. Project Part-I	
\end{flushleft}


0	 0	 12	6


	


\begin{flushleft}
Total Credits				55
\end{flushleft}


\begin{flushleft}
Departmental Electives
\end{flushleft}


\begin{flushleft}
COD300	Design Project	
\end{flushleft}


0	0	4	2


\begin{flushleft}
COD310	Mini Project	
\end{flushleft}


0	0	6	3


\begin{flushleft}
COL333	 Principles of Artificial Intelligence*	
\end{flushleft}


3	 0	 2	 4


\begin{flushleft}
COL341	Machine Learning	
\end{flushleft}


3	0	2	4


\begin{flushleft}
COL362	 Introduction to Database Management Systems*	3	0	2	4
\end{flushleft}


\begin{flushleft}
COP315	 Embedded System Design Project	
\end{flushleft}


0	 1	 6	 4


\begin{flushleft}
COD494**	B.Tech. Project Part-II	
\end{flushleft}


0	 0	 16	8


\begin{flushleft}
COR310	Professional Practices (CS)	
\end{flushleft}


1	0	2	2


\begin{flushleft}
COS310	Independent Study (CS)	
\end{flushleft}


0	3	0	3





48





\newpage
49





\begin{flushleft}
Semester
\end{flushleft}





\begin{flushleft}
VIII
\end{flushleft}





\begin{flushleft}
VII
\end{flushleft}





\begin{flushleft}
VI
\end{flushleft}





\begin{flushleft}
V
\end{flushleft}





\begin{flushleft}
IV
\end{flushleft}





\begin{flushleft}
III
\end{flushleft}





\begin{flushleft}
II
\end{flushleft}





\begin{flushleft}
I
\end{flushleft}





\begin{flushleft}
MCP100
\end{flushleft}





\begin{flushleft}
PYL100
\end{flushleft}





\begin{flushleft}
Course-3
\end{flushleft}





\begin{flushleft}
Course-1
\end{flushleft}





0 2


\begin{flushleft}
APL100
\end{flushleft}





4





4





3





0





2





4





4





3





3





3





4





1





0





4





0 0 3


\begin{flushleft}
OC 2 (4)
\end{flushleft}





0 2


\begin{flushleft}
DE 2 (3)
\end{flushleft}





\begin{flushleft}
Introduction to
\end{flushleft}


\begin{flushleft}
Database Management
\end{flushleft}


\begin{flushleft}
Systems*
\end{flushleft}





3 0 2 4


\begin{flushleft}
COL362 / DE1
\end{flushleft}





\begin{flushleft}
Principles of Artificial
\end{flushleft}


\begin{flushleft}
Intelligence*
\end{flushleft}





3 0 4 5


\begin{flushleft}
COL333 / DE 1
\end{flushleft}





\begin{flushleft}
Programming
\end{flushleft}


\begin{flushleft}
Languages
\end{flushleft}





1 0


\begin{flushleft}
COL226
\end{flushleft}





0


4


\begin{flushleft}
COL216
\end{flushleft}





5





0


2


\begin{flushleft}
COL334
\end{flushleft}





4





3





3





3





3





0





0





0


0


\begin{flushleft}
OC 3 (3)
\end{flushleft}





0


4


\begin{flushleft}
OC 1 (3)
\end{flushleft}





\begin{flushleft}
Operating Systems
\end{flushleft}





0


2


\begin{flushleft}
COL331
\end{flushleft}





3





3





5





4





\begin{flushleft}
Computer Networks
\end{flushleft}





3





\begin{flushleft}
Computer Architecture
\end{flushleft}





3





\begin{flushleft}
Digital Logic \& System
\end{flushleft}


\begin{flushleft}
Design
\end{flushleft}





3





0


0


\begin{flushleft}
CML100
\end{flushleft}





3





3





0





0





3





\begin{flushleft}
Introduction to Chemistry
\end{flushleft}





3





1 0


\begin{flushleft}
MTL101
\end{flushleft}





\begin{flushleft}
Calculus
\end{flushleft}





4





3





1





0





4





\begin{flushleft}
Linear Algebra and
\end{flushleft}


\begin{flushleft}
Differential Equations
\end{flushleft}





3





\begin{flushleft}
Course-4
\end{flushleft}





\begin{flushleft}
MTL100
\end{flushleft}





\begin{flushleft}
Course-5
\end{flushleft}





0 4 2


\begin{flushleft}
CMP100
\end{flushleft}





0





0





4





2





\begin{flushleft}
Chemistry Laboratory
\end{flushleft}





0





\begin{flushleft}
Physics Laboratory
\end{flushleft}





\begin{flushleft}
PYP100
\end{flushleft}





\begin{flushleft}
Course-6
\end{flushleft}





0





0





4





2





\begin{flushleft}
Product Realization
\end{flushleft}


\begin{flushleft}
through Manufacturing
\end{flushleft}





\begin{flushleft}
MCP101
\end{flushleft}





0





0





2





1





0


1 0.5


\begin{flushleft}
NEN100
\end{flushleft}





0





0





1





0.5





\begin{flushleft}
Professional Ethics and
\end{flushleft}


\begin{flushleft}
Social Responsibility-2
\end{flushleft}


\begin{flushleft}
(Non-graded)
\end{flushleft}





0





\begin{flushleft}
Professional Ethics and
\end{flushleft}


\begin{flushleft}
Social Responsibility-1
\end{flushleft}


\begin{flushleft}
(Non-graded)
\end{flushleft}





\begin{flushleft}
Introduction to
\end{flushleft}


\begin{flushleft}
Engineering
\end{flushleft}


\begin{flushleft}
(Non-graded)
\end{flushleft}





\begin{flushleft}
Course-7
\end{flushleft}





\begin{flushleft}
NEN100
\end{flushleft}





\begin{flushleft}
Course-8
\end{flushleft}





\begin{flushleft}
NIN100
\end{flushleft}





0





0





0





2





\begin{flushleft}
Language and
\end{flushleft}


\begin{flushleft}
Writing Skills-2
\end{flushleft}


\begin{flushleft}
(Non-Graded)
\end{flushleft}





1





0 2 1


\begin{flushleft}
NLN100
\end{flushleft}





\begin{flushleft}
Language and
\end{flushleft}


\begin{flushleft}
Writing Skills-1
\end{flushleft}


\begin{flushleft}
(Non-Graded)
\end{flushleft}





\begin{flushleft}
NLN100
\end{flushleft}





\begin{flushleft}
Course-9
\end{flushleft}





12





9.5





\begin{flushleft}
L
\end{flushleft}





2





1





\begin{flushleft}
T
\end{flushleft}





0


4


\begin{flushleft}
ELL205
\end{flushleft}





5





1


0


\begin{flushleft}
COL351
\end{flushleft}





4





1


0


\begin{flushleft}
COL352
\end{flushleft}





4





0


0


3


\begin{flushleft}
COD490/492
\end{flushleft}





3





0





0





2





0 12


\begin{flushleft}
DE 3 (4)
\end{flushleft}


4





6





\begin{flushleft}
B.Tech Project part - I
\end{flushleft}





3





\begin{flushleft}
Intro to Automata \&
\end{flushleft}


\begin{flushleft}
Theory of Computation
\end{flushleft}





3





\begin{flushleft}
Analysis and Design of
\end{flushleft}


\begin{flushleft}
Algorithms
\end{flushleft}





3





\begin{flushleft}
Signals and Systems
\end{flushleft}





3





\begin{flushleft}
Data Structures \&
\end{flushleft}


\begin{flushleft}
Algorithms
\end{flushleft}





\begin{flushleft}
COL106
\end{flushleft}





0 0


\begin{flushleft}
CVL100
\end{flushleft}





3





0 0


\begin{flushleft}
SBL100
\end{flushleft}





2





0 2


\begin{flushleft}
MTLXXX
\end{flushleft}





4





3





3





0





0





0





\begin{flushleft}
HUL3XX
\end{flushleft}





0





3





3





\begin{flushleft}
Programme-Linked
\end{flushleft}


\begin{flushleft}
course in Mathematics
\end{flushleft}





3





\begin{flushleft}
Introductory Biology for
\end{flushleft}


\begin{flushleft}
Engineers
\end{flushleft}





2





\begin{flushleft}
Environmental Science
\end{flushleft}





3





\begin{flushleft}
Principles of Electronic
\end{flushleft}


\begin{flushleft}
Materials
\end{flushleft}





\begin{flushleft}
PYL102
\end{flushleft}





3





3





3





3





1





0





4





1 0 4


\begin{flushleft}
HUL2XX
\end{flushleft}





1 0 4


\begin{flushleft}
HUL2XX
\end{flushleft}





1 0 4


\begin{flushleft}
HUL2XX
\end{flushleft}





\begin{flushleft}
Probability and
\end{flushleft}


\begin{flushleft}
Stochastic Processes
\end{flushleft}





\begin{flushleft}
MTL106
\end{flushleft}





0


6


\begin{flushleft}
COD3XX
\end{flushleft}





3





0


4


\begin{flushleft}
COL380
\end{flushleft}





0





2





0





2





3





\begin{flushleft}
Intro to Parallel \&
\end{flushleft}


\begin{flushleft}
Distributed Programming
\end{flushleft}





0





\begin{flushleft}
Non-Graded Design
\end{flushleft}


\begin{flushleft}
Project
\end{flushleft}





0





\begin{flushleft}
Design Practices
\end{flushleft}





\begin{flushleft}
COP290
\end{flushleft}





0





0





2





1





\begin{flushleft}
Intro. to Comp. Sc.
\end{flushleft}


\begin{flushleft}
and Engg.
\end{flushleft}


\begin{flushleft}
(Non-graded)
\end{flushleft}





\begin{flushleft}
CON101
\end{flushleft}





12





6





17





15





14





15





1





0





1





2





2





2





\begin{flushleft}
Note: Courses 1-6 above are attended in the given order by half of all first year students. The other half of First year students attend the Courses 1-6 of II semester first.
\end{flushleft}





\begin{flushleft}
Discrete Mathematical
\end{flushleft}


\begin{flushleft}
Structures
\end{flushleft}





0





2





\begin{flushleft}
COL215
\end{flushleft}





1





0


3


\begin{flushleft}
COL100
\end{flushleft}





\begin{flushleft}
Introduction to Computer
\end{flushleft}


\begin{flushleft}
Science
\end{flushleft}





0.5





\begin{flushleft}
COL202
\end{flushleft}





3





\begin{flushleft}
Engineering Mechanics
\end{flushleft}





3





\begin{flushleft}
Introduction to Electrical Introduction to Engineering Electromagnetic Waves
\end{flushleft}


\begin{flushleft}
Engineering
\end{flushleft}


\begin{flushleft}
Visualization
\end{flushleft}


\begin{flushleft}
and Quantum Mechanics
\end{flushleft}





\begin{flushleft}
Course-2
\end{flushleft}





\begin{flushleft}
ELL100
\end{flushleft}





\begin{flushleft}
Credits
\end{flushleft}





15.0 145.0





18.0





26.0





27.0





28.0





27.0





\begin{flushleft}
TOTAL=145.0
\end{flushleft}





2 14.0 0





12 12.0 0





8 22.0 0





10 20.0 0





12 22.0 0





8 21.0 1





6 17.0 1.5 23.0





13 17.0 2.5 28.5





\begin{flushleft}
P
\end{flushleft}





\begin{flushleft}
Non-Graded Units
\end{flushleft}





\begin{flushleft}
B.Tech. in Computer Science and Engineering	CS1
\end{flushleft}


\begin{flushleft}
Contact Hours
\end{flushleft}





\begin{flushleft}
\newpage
Programme Code: CS5
\end{flushleft}





\begin{flushleft}
Dual Degree Programme: Bachelor of Technology and Master of Technology
\end{flushleft}


\begin{flushleft}
in Computer Science and Engineering
\end{flushleft}


\begin{flushleft}
Department of Computer Science and Engineering
\end{flushleft}


\begin{flushleft}
The overall Credit Structure
\end{flushleft}





\begin{flushleft}
Departmental Electives
\end{flushleft}





\begin{flushleft}
COD300	Design Project	
\end{flushleft}


0	0	4	2


\begin{flushleft}
Course Category	
\end{flushleft}


\begin{flushleft}
Credits
\end{flushleft}


\begin{flushleft}
COD310	Mini Project	
\end{flushleft}


0	0	6	3


\begin{flushleft}
Institute Core Courses
\end{flushleft}


\begin{flushleft}
Basic Sciences (BS)		 22
\end{flushleft}


\begin{flushleft}
COL333	 Principles of Artificial Intelligence*	
\end{flushleft}


3	 0	 2	 4


\begin{flushleft}
Engineering Arts and Science (EAS)		 18
\end{flushleft}


\begin{flushleft}
COL341	Machine Learning	
\end{flushleft}


3	0	2	4


\begin{flushleft}
Humanities and Social Sciences (HuSS)		 15
\end{flushleft}


\begin{flushleft}
COL362	 Introduction to Database Management	
\end{flushleft}


3	 0	 2	 4


\begin{flushleft}
Programme-linked Courses		14
\end{flushleft}


\begin{flushleft}
	Systems*
\end{flushleft}


\begin{flushleft}
Departmental Courses
\end{flushleft}


\begin{flushleft}
COP315	 Embedded System Design Project	
\end{flushleft}


0	 1	 6	 4


\begin{flushleft}
Departmental Core 		 49
\end{flushleft}


\begin{flushleft}
COR310	Professional Practices (CS)	
\end{flushleft}


1	0	2	2


\begin{flushleft}
Departmental Electives		 11
\end{flushleft}


\begin{flushleft}
COS310	Independent Study (CS)	
\end{flushleft}


0	3	0	3


\begin{flushleft}
Open Category Courses		 10
\end{flushleft}


\begin{flushleft}
COL718	 Architecture of High Performance Computers	 3	 0	 2	 4
\end{flushleft}


\begin{flushleft}
Total B.Tech. Credit Requirement		 139
\end{flushleft}


\begin{flushleft}
COL719 	 Synthesis of Digital Systems	
\end{flushleft}


3	 0	 2	 4


\begin{flushleft}
Non Graded Units		 15
\end{flushleft}


\begin{flushleft}
COL722	 Introduction to Compressed Sensing	
\end{flushleft}


3	 0	 0	 3


\begin{flushleft}
M. Tech. Part
\end{flushleft}


\begin{flushleft}
COL724	Advanced Computer Networks	
\end{flushleft}


3	0	2	4


\begin{flushleft}
Programme Core Courses		 32
\end{flushleft}


\begin{flushleft}
COL728	Compiler Design	
\end{flushleft}


3	0	3	4.5


\begin{flushleft}
Programme Elective Courses		 14
\end{flushleft}


\begin{flushleft}
COL729	Compiler Optimization	
\end{flushleft}


3	0	3	4.5


\begin{flushleft}
Total M.Tech. Credit Requirement		
\end{flushleft}


46


\begin{flushleft}
COL730	Parallel Programming	
\end{flushleft}


3	0	2	4


\begin{flushleft}
Grand Total Credit Requirement		
\end{flushleft}


185


\begin{flushleft}
COL732	 Virtualization and Cloud Computing	
\end{flushleft}


3	 0	 2	 4


\begin{flushleft}
Institute Core: Basic Sciences
\end{flushleft}


\begin{flushleft}
COL733	Cloud Computing Technology Fundamentals	3	0	2	4
\end{flushleft}


\begin{flushleft}
COL740	Software Engineering	
\end{flushleft}


3	0	2	4


\begin{flushleft}
CML100	 General Chemistry	
\end{flushleft}


3	 0	 0	 3


\begin{flushleft}
COL750	 Foundations of Automatic Verification	
\end{flushleft}


3	 0	 2	 4


\begin{flushleft}
CMP100	Chemistry Laboratory	
\end{flushleft}


0	0	4	2


\begin{flushleft}
COL751	 Algorithmic Graph Theory	
\end{flushleft}


3	 0	 0	 3


\begin{flushleft}
MTL100	 Calculus	
\end{flushleft}


3	1	0	4


\begin{flushleft}
COL752	 Geometric Algorithms	
\end{flushleft}


3	 0	 0	 3


\begin{flushleft}
MTL101	 Linear Algebra and Differential Equations	
\end{flushleft}


3	 1	 0	 4


\begin{flushleft}
COL753	Complexity Theory	
\end{flushleft}


3	0	0	3


\begin{flushleft}
PYL100	 Electromagnetic Waves and Quantum	
\end{flushleft}


3	 0	 0	 3


\begin{flushleft}
COL754	Approximation Algorithms	
\end{flushleft}


3	0	0	3


\begin{flushleft}
	Mechanics
\end{flushleft}


\begin{flushleft}
PYP100	 Physics Laboratory	
\end{flushleft}


0	0	4	2


\begin{flushleft}
COL756	Mathematical Programming	
\end{flushleft}


3	0	0	3


\begin{flushleft}
SBL100	 Introductory Biology for Engineers	
\end{flushleft}


3	 0	 2	 4


\begin{flushleft}
COL757	Model Centric Algorithm Design	
\end{flushleft}


3	0	2	4


3	0	2	4


	


\begin{flushleft}
Total Credits				22	 COL758	Advanced Algorithms	
\end{flushleft}


\begin{flushleft}
COL759	 Cryptography \& Computer Security	
\end{flushleft}


3	 0	 0	 3


\begin{flushleft}
Institute Core: Engineering Arts and Sciences
\end{flushleft}


\begin{flushleft}
COL760	Advanced Data Management	
\end{flushleft}


3	0	2	4


\begin{flushleft}
APL100	 Engineering Mechanics	
\end{flushleft}


3	1	0	4


\begin{flushleft}
COL762	Database Implementation	
\end{flushleft}


3	0	2	4


\begin{flushleft}
COL100	 Introduction to Computer Science	
\end{flushleft}


3	 0	 2	 4


\begin{flushleft}
COL765	 Logic and Functional Programming	
\end{flushleft}


3	 0	 2	 4


\begin{flushleft}
CVL100	 Environmental Science	
\end{flushleft}


2	0	0	2


\begin{flushleft}
COL768	Wireless Networks	
\end{flushleft}


3	0	2	4


\begin{flushleft}
ELL100	 Introduction to Electrical Engineering	
\end{flushleft}


3	 0	 2	 4


\begin{flushleft}
COL770	 Advanced Artificial Intelligence	
\end{flushleft}


3	 0	 2	 4


\begin{flushleft}
MCP100	Engineering Visualization	
\end{flushleft}


0	0	4	2


\begin{flushleft}
COL772	Natural Language Processing	
\end{flushleft}


3	0	2	4


\begin{flushleft}
MCP101	 Product Realization through Manufacturing	 0	 0	 4	 2
\end{flushleft}


\begin{flushleft}
COL774	Machine Learning	
\end{flushleft}


3	0	2	4


	


\begin{flushleft}
Total Credits				18
\end{flushleft}


\begin{flushleft}
COL776	 Learning Probabilistic Graphical Models	
\end{flushleft}


3	 0	 2	 4


\begin{flushleft}
COL780	Computer Vision	
\end{flushleft}


3	0	2	4


\begin{flushleft}
Programme-Linked Basic / Engineering Arts / Sciences Core
\end{flushleft}


\begin{flushleft}
COL781	 Computer Graphics	
\end{flushleft}


3	 0	 3	 4.5


\begin{flushleft}
ELL205	 Signals and Systems	
\end{flushleft}


3	1	0	4


\begin{flushleft}
COL783	Digital Image Analysis	
\end{flushleft}


3	0	3	4.5


\begin{flushleft}
MTL103*	Optimization Methods and Applications	
\end{flushleft}


3	0	0	3


\begin{flushleft}
COL786	 Advanced Functional Brain Imaging	
\end{flushleft}


3	 0	 2	 4


\begin{flushleft}
MTL104*	Linear Algebra and Applications	
\end{flushleft}


3	0	0	3


\begin{flushleft}
COL788	 Advanced Topics in Embedded Computing	
\end{flushleft}


3	 0	 0	 3


\begin{flushleft}
MTL105*	Algebra	
\end{flushleft}


3	0	0	3


\begin{flushleft}
COL829	
\end{flushleft}


\begin{flushleft}
Advanced
\end{flushleft}


\begin{flushleft}
Computer
\end{flushleft}


\begin{flushleft}
Graphics	
\end{flushleft}


3	 0	 2	 4


\begin{flushleft}
MTL106	 Probability and Stochastic Processes	
\end{flushleft}


3	 1	 0	 4


\begin{flushleft}
COL851	 Special Topics in Operating Systems	
\end{flushleft}


3	 0	 0	 3


\begin{flushleft}
PYL102\#	 Principles of Electronic Materials	
\end{flushleft}


3	 0	 0	 3


\begin{flushleft}
COL852	 Special Topics in Compiler Design	
\end{flushleft}


3	 0	 0	 4


\begin{flushleft}
PYL103\#	Physics of Nanomaterials	
\end{flushleft}


3	0	0	3


\begin{flushleft}
COL860	 Special Topics in Parallel Computation	
\end{flushleft}


3	 0	 0	 3


	


\begin{flushleft}
Total Credits				14
\end{flushleft}


\begin{flushleft}
COL861	 Special Topics in Hardware Systems	
\end{flushleft}


3	 0	 0	 3


\begin{flushleft}
*One of these three courses
\end{flushleft}


\begin{flushleft}
COL862	 Special Topics in Software Systems	
\end{flushleft}


3	 0	 0	 3


\begin{flushleft}
\# one of these two courses
\end{flushleft}


\begin{flushleft}
COL863	 Special Topics in Theoretical Computer Science	 3	0	0	3
\end{flushleft}


\begin{flushleft}
COL864	 Special Topics in Artificial Intelligence	
\end{flushleft}


3	 0	 0	 3


\begin{flushleft}
Humanities and Social Sciences
\end{flushleft}


\begin{flushleft}
COL865	Special
\end{flushleft}


\begin{flushleft}
Topics
\end{flushleft}


\begin{flushleft}
in
\end{flushleft}


\begin{flushleft}
Computer
\end{flushleft}


\begin{flushleft}
Applications	
\end{flushleft}


3	0	0	3


\begin{flushleft}
Courses from Humanities, Social Sciences and Management 	
\end{flushleft}


\begin{flushleft}
COL866	Special Topics in Algorithms	
\end{flushleft}


3	0	0	3


\begin{flushleft}
offered under this category				
\end{flushleft}


15


\begin{flushleft}
COL867	 Special Topics in High Speed Networks	
\end{flushleft}


3	 0	 0	 3


\begin{flushleft}
COL868	 Special Topics in Database Systems	
\end{flushleft}


3	 0	 0	 3


\begin{flushleft}
Departmental Core
\end{flushleft}


\begin{flushleft}
COL869	Special
\end{flushleft}


\begin{flushleft}
Topics
\end{flushleft}


\begin{flushleft}
in
\end{flushleft}


\begin{flushleft}
Concurrency	
\end{flushleft}


3	0	0	3


\begin{flushleft}
COL106	Data Structures and Algorithms	
\end{flushleft}


3	0	4	5


\begin{flushleft}
COL870	 Special Topics in Machine Learning	
\end{flushleft}


3	 0	 0	 3


\begin{flushleft}
COL202	 Discrete Mathematical Structures 	
\end{flushleft}


3	 1	 0	 4


\begin{flushleft}
COL871	
\end{flushleft}


\begin{flushleft}
Special
\end{flushleft}


\begin{flushleft}
Topics
\end{flushleft}


\begin{flushleft}
in
\end{flushleft}


\begin{flushleft}
programming
\end{flushleft}


\begin{flushleft}
languages
\end{flushleft}


	


3	 0	 0	 3


\begin{flushleft}
COL215	 Digital Logic and System Design	
\end{flushleft}


3	 0	 4	 5


	


\begin{flushleft}
\& Compilers
\end{flushleft}


\begin{flushleft}
COL216	Computer Architecture	
\end{flushleft}


3	0	2	4


\begin{flushleft}
COL872	Special Topics in Cryptography	
\end{flushleft}


3	0	0	3


\begin{flushleft}
COL226	Programming Languages	
\end{flushleft}


3	0	4	5


\begin{flushleft}
COV877	 Special Module on Visual Computing	
\end{flushleft}


1	 0	 0	 1


\begin{flushleft}
COP290	Design Practices	
\end{flushleft}


0	0	6	3


\begin{flushleft}
COV878	 Special Module in Machine Learning	
\end{flushleft}


1	 0	 0	 1


\begin{flushleft}
COL331	Operating Systems	
\end{flushleft}


3	0	4	5


\begin{flushleft}
COV879	 Special Module in Financial Algorithms	
\end{flushleft}


2	 0	 0	 2


\begin{flushleft}
COL333	 Principles of Artificial Intelligence*	
\end{flushleft}


3	 0	 2	 4


\begin{flushleft}
COV880	 Special Module in Parallel Computation	
\end{flushleft}


1	 0	 0	 1


\begin{flushleft}
COL334	Computer Networks	
\end{flushleft}


3	0	2	4


\begin{flushleft}
COV881	 Special Module in Hardware Systems	
\end{flushleft}


1	 0	 0	 1


\begin{flushleft}
COL351	 Analysis and Design of Algorithms	
\end{flushleft}


3	 1	 0	 4


\begin{flushleft}
COL352	 Introduction to Automata and 	
\end{flushleft}


3	 0	 0	 3


\begin{flushleft}
COV882	 Special Module in Software Systems	
\end{flushleft}


1	 0	 0	 1


	


\begin{flushleft}
Theory of Computation
\end{flushleft}


\begin{flushleft}
COV883	 Special Module in Theoretical Computer Science	 1	0	0	1
\end{flushleft}


\begin{flushleft}
COL380	 Introduction to Parallel and 	
\end{flushleft}


2	 0	 2	 3


\begin{flushleft}
COV884	 Special Module in Artificial Intelligence	
\end{flushleft}


1	 0	 0	 1


	


\begin{flushleft}
Distributed Programming
\end{flushleft}


\begin{flushleft}
COV885	 Special Module in Computer Applications	
\end{flushleft}


1	 0	 0	 1


	


\begin{flushleft}
Total Credits				49
\end{flushleft}


\begin{flushleft}
COV886	Special Module in Algorithms	
\end{flushleft}


1	0	0	1





50





\begin{flushleft}
\newpage
COV887	 Special Module in High Speed Networks	
\end{flushleft}


1	 0	 0	 1


\begin{flushleft}
COV888	 Special Module in Database Systems	
\end{flushleft}


1	 0	 0	 1


\begin{flushleft}
COV889	 Special Module in Concurrency	
\end{flushleft}


1	 0	 0	 1


\begin{flushleft}
SIL765	 Networks \& System Security	
\end{flushleft}


3	 0	 2	 4


\begin{flushleft}
SIL769	 Internet Traffic -Measurement, Modeling \& Analysis	 3	0	2	4
\end{flushleft}


\begin{flushleft}
SIL801	 Special Topics in Multimedia System	
\end{flushleft}


3	 0	 0	 3


\begin{flushleft}
SIL802	 Special Topics in Web Based Computing	
\end{flushleft}


3	 0	 0	 3


\begin{flushleft}
SIV813	 Applications of Computer in Medicines	
\end{flushleft}


1	 0	 0	 1


\begin{flushleft}
SIV861	 Information and Comm Technologies for	
\end{flushleft}


1	 0	 0	 1


\begin{flushleft}
	Development
\end{flushleft}


\begin{flushleft}
SIV864	 Special Module on Media Processing 	
\end{flushleft}


1	 0	 0	 1


	


\begin{flushleft}
\& Communication
\end{flushleft}


\begin{flushleft}
SIV895	 Special Module on Intelligent Information Processing	1	0	0	1
\end{flushleft}





\begin{flushleft}
COL788	 Advanced Topics in Embedded Computing	
\end{flushleft}


\begin{flushleft}
COS799	Independent Study	
\end{flushleft}


\begin{flushleft}
COL812	 System Level Design and Modelling	
\end{flushleft}


\begin{flushleft}
COL818	 Principles of Multiprocessor Systems	
\end{flushleft}


\begin{flushleft}
COL819	Advanced Distributed Systems	
\end{flushleft}


\begin{flushleft}
COL821	 Reconfigurable Computing	
\end{flushleft}


\begin{flushleft}
COL830	Distributed Computing	
\end{flushleft}


\begin{flushleft}
COL831	 Semantics of Programming Languages	
\end{flushleft}


\begin{flushleft}
COL832	Proofs and Types	
\end{flushleft}


\begin{flushleft}
COL859	 Advanced Computer Graphics	
\end{flushleft}


\begin{flushleft}
COL860	 Special Topics in Parallel Computation	
\end{flushleft}


\begin{flushleft}
COL861	 Special Topics in Hardware Systems	
\end{flushleft}


\begin{flushleft}
COL862	 Special Topics in Software Systems	
\end{flushleft}


\begin{flushleft}
COL863	Special Topics in Theoretical 	
\end{flushleft}


	


\begin{flushleft}
Computer Science
\end{flushleft}


\begin{flushleft}
COL864	 Special Topics in Artificial Intelligence	
\end{flushleft}


\begin{flushleft}
COL865	Special Topics in Computer Applications	
\end{flushleft}


\begin{flushleft}
COL866	Special Topics in Algorithms	
\end{flushleft}


\begin{flushleft}
COL867	 Special Topics in High Speed Networks	
\end{flushleft}


\begin{flushleft}
COL868	 Special Topics in Database Systems	
\end{flushleft}


\begin{flushleft}
COL869	Special Topics in Concurrency	
\end{flushleft}


\begin{flushleft}
COL870	 Special Topics in Machine Learning	
\end{flushleft}


\begin{flushleft}
COL871	 Special Topics in programming languages 	
\end{flushleft}


	


\begin{flushleft}
\& Compilers
\end{flushleft}


\begin{flushleft}
COL872	Special Topics in Cryptography	
\end{flushleft}


\begin{flushleft}
COV877	 Special Module on Visual Computing	
\end{flushleft}


\begin{flushleft}
COV878	 Special Module in Machine Learning	
\end{flushleft}


\begin{flushleft}
COV879	 Special Module in Financial Algorithms	
\end{flushleft}


\begin{flushleft}
COV880	 Special Module in Parallel Computation	
\end{flushleft}


\begin{flushleft}
COV881	 Special Module in Hardware Systems	
\end{flushleft}


\begin{flushleft}
COV882	 Special Module in Software Systems	
\end{flushleft}


\begin{flushleft}
COV883	 Special Module in Theoretical 	
\end{flushleft}


	


\begin{flushleft}
Computer Science
\end{flushleft}


\begin{flushleft}
COV884	 Special Module in Artificial Intelligence	
\end{flushleft}


\begin{flushleft}
COV885	 Special Module in Computer Applications	
\end{flushleft}


\begin{flushleft}
COV886	Special Module in Algorithms	
\end{flushleft}


\begin{flushleft}
COV887	 Special Module in High Speed Networks	
\end{flushleft}


\begin{flushleft}
COV888	 Special Module in Database Systems	
\end{flushleft}


\begin{flushleft}
COV889	 Special Module in Concurrency	
\end{flushleft}


\begin{flushleft}
SIL765	 Networks \& System Security	
\end{flushleft}


\begin{flushleft}
SIL769	 Internet Traffic -Measurement,	
\end{flushleft}


	


\begin{flushleft}
Modeling \& Analysis
\end{flushleft}


\begin{flushleft}
SIL801	 Special Topics in Multimedia System	
\end{flushleft}


\begin{flushleft}
SIL802	 Special Topics in Web Based Computing	
\end{flushleft}


\begin{flushleft}
SIV813	 Applications of Computer in Medicines	
\end{flushleft}


\begin{flushleft}
SIV861	 Information and Comm Technologies	
\end{flushleft}


	


\begin{flushleft}
for Development
\end{flushleft}


\begin{flushleft}
SIV864	 Special Module on Media Processing 	
\end{flushleft}


	


\begin{flushleft}
\& Communication
\end{flushleft}


\begin{flushleft}
SIV871	 Special Module in Computational Neuroscience	
\end{flushleft}


\begin{flushleft}
SIV895	 Special Module on Intelligent Information	
\end{flushleft}


\begin{flushleft}
	Processing
\end{flushleft}





\begin{flushleft}
Program Core
\end{flushleft}


\begin{flushleft}
COL703	 Logic for Computer Science	
\end{flushleft}


3	 0	 2	 4


\begin{flushleft}
COL726	Numerical Algorithms	
\end{flushleft}


3	0	2	4


\begin{flushleft}
COD891	 Minor Project 	
\end{flushleft}


0	 0	 6	 3


\begin{flushleft}
COD892	 M.Tech. Project Part--I 	
\end{flushleft}


0	 0	 14	7


\begin{flushleft}
COD893	 M.Tech. Project Part--II 	
\end{flushleft}


0	 0	 28	14


	


\begin{flushleft}
Total Credits				32
\end{flushleft}


\begin{flushleft}
Program Electives
\end{flushleft}


\begin{flushleft}
COD745	Minor Project	
\end{flushleft}


\begin{flushleft}
COL718	 Architecture of High Performance Computers	
\end{flushleft}


\begin{flushleft}
COL719	 Synthesis of Digital Systems	
\end{flushleft}


\begin{flushleft}
COL724	Advanced Computer Networks	
\end{flushleft}


\begin{flushleft}
COL728	Compiler Design	
\end{flushleft}


\begin{flushleft}
COL729	Compiler Optimization	
\end{flushleft}


\begin{flushleft}
COL730	Parallel Programming	
\end{flushleft}


\begin{flushleft}
COL732	 Virtualization and Cloud Computing	
\end{flushleft}


\begin{flushleft}
COL740	Software Engineering	
\end{flushleft}


\begin{flushleft}
COL750	 Foundations of Automatic Verification	
\end{flushleft}


\begin{flushleft}
COL751	 Algorithmic Graph Theory	
\end{flushleft}


\begin{flushleft}
COL752	 Geometric Algorithms	
\end{flushleft}


\begin{flushleft}
COL753	Complexity Theory	
\end{flushleft}


\begin{flushleft}
COL754	Approximation Algorithms	
\end{flushleft}


\begin{flushleft}
COL756	Mathematical Programming	
\end{flushleft}


\begin{flushleft}
COL757	Model Centric Algorithm Design	
\end{flushleft}


\begin{flushleft}
COL758	Advanced Algorithms	
\end{flushleft}


\begin{flushleft}
COL759	 Cryptography \& Computer Security	
\end{flushleft}


\begin{flushleft}
COL760	Advanced Data Management	
\end{flushleft}


\begin{flushleft}
COL762	Database Implementation	
\end{flushleft}


\begin{flushleft}
COL768	Wireless Networks	
\end{flushleft}


\begin{flushleft}
COL770	 Advanced Artificial Intelligence	
\end{flushleft}


\begin{flushleft}
COL772	Natural Language Processing	
\end{flushleft}


\begin{flushleft}
COL774	Machine Learning	
\end{flushleft}


\begin{flushleft}
COL776	 Learning Probabilistic Graphical Models	
\end{flushleft}


\begin{flushleft}
COL780	Computer Vision	
\end{flushleft}


\begin{flushleft}
COL781	 Computer Graphics	
\end{flushleft}


\begin{flushleft}
COL783	Digital Image Analysis	
\end{flushleft}





0	0	6	3


3	 0	 2	 4


3	 0	 2	 4


3	0	2	4


3	0	3	4.5


3	0	3	4.5


3	0	2	4


3	 0	 2	 4


3	0	2	4


3	 0	 2	 4


3	 0	 0	 3


3	 0	 0	 3


3	0	0	3


3	0	0	3


3	0	0	3


3	0	2	4


3	0	2	4


3	 0	 0	 3


3	0	2	4


3	0	2	4


3	0	2	4


3	 0	 2	 4


3	0	2	4


3	0	2	4


3	 0	 2	 4


3	0	2	4


3	 0	 3	 4.5


3	0	3	4.5





51





3	 0	 0	 3


0	3	0	3


3	 0	 0	 3


3	 0	 2	 4


3	0	2	4


3	 0	 0	 3


3	0	0	3


3	 0	 0	 3


3	0	0	3


3	 0	 2	 4


3	 0	 0	 3


3	 0	 0	 3


3	 0	 0	 3


3	0	0	3


3	 0	 0	 3


3	0	0	3


3	0	0	3


3	 0	 0	 3


3	 0	 0	 3


3	0	0	3


3	 0	 0	 3


3	 0	 0	 3


3	0	0	3


1	 0	 0	 1


1	 0	 0	 1


2	 0	 0	 2


1	 0	 0	 1


1	 0	 0	 1


1	 0	 0	 1


1	 0	 0	 1


1	 0	 0	 1


1	 0	 0	 1


1	0	0	1


1	 0	 0	 1


1	 0	 0	 1


1	 0	 0	 1


3	 0	 2	 4


3	 0	 2	 4


3	 0	 0	 3


3	 0	 0	 3


1	 0	 0	 1


1	0	0	1


1	 0	 0	 1


1	0	0	1


1	 0	 0	 1





\newpage
52





\begin{flushleft}
Semester
\end{flushleft}





\begin{flushleft}
X
\end{flushleft}





\begin{flushleft}
IX
\end{flushleft}





\begin{flushleft}
VIII
\end{flushleft}





\begin{flushleft}
VII
\end{flushleft}





\begin{flushleft}
VI
\end{flushleft}





\begin{flushleft}
V
\end{flushleft}





\begin{flushleft}
IV
\end{flushleft}





\begin{flushleft}
III
\end{flushleft}





\begin{flushleft}
II
\end{flushleft}





\begin{flushleft}
I
\end{flushleft}





4





4





0


2


4


\begin{flushleft}
COL362 / DE1
\end{flushleft}





\begin{flushleft}
Principles of Artificial
\end{flushleft}


\begin{flushleft}
Intelligence*
\end{flushleft}





0


4


5


\begin{flushleft}
COL333 / DE 1
\end{flushleft}





3





0





3





3





6





0





0





\begin{flushleft}
PE 4 (3)
\end{flushleft}





0





\begin{flushleft}
Minor Project
\end{flushleft}





0


0


\begin{flushleft}
COD891
\end{flushleft}





0


2


\begin{flushleft}
DE 2 (3)
\end{flushleft}





3





3





3





4





\begin{flushleft}
Introduction to Database
\end{flushleft}


\begin{flushleft}
Management Systems*
\end{flushleft}





3





3





3





0





2





4





0


4


\begin{flushleft}
COL216
\end{flushleft}





5





0


2


\begin{flushleft}
COL334
\end{flushleft}





4





0


2


\begin{flushleft}
COL331
\end{flushleft}





0


2


\begin{flushleft}
COL726
\end{flushleft}





0


4


\begin{flushleft}
DE 3(4)
\end{flushleft}





4





5





4





3





3





2





0





0





\begin{flushleft}
PE 5 (3)
\end{flushleft}





0





3





4





\begin{flushleft}
Numerical Algorithms
\end{flushleft}





3





3





\begin{flushleft}
Operating Systems
\end{flushleft}





3





\begin{flushleft}
Computer Networks
\end{flushleft}





3





\begin{flushleft}
Computer Architecture
\end{flushleft}





3





\begin{flushleft}
COL215
\end{flushleft}





1


0


\begin{flushleft}
COL226
\end{flushleft}





3





0


0


\begin{flushleft}
CML100
\end{flushleft}





3





3





0





0





3





1


0


\begin{flushleft}
MTL101
\end{flushleft}





\begin{flushleft}
Calculus
\end{flushleft}





4





3





1





0





4





\begin{flushleft}
Linear Algebra and
\end{flushleft}


\begin{flushleft}
Differential Equations
\end{flushleft}





3





\begin{flushleft}
Course-4
\end{flushleft}


\begin{flushleft}
MTL100
\end{flushleft}





\begin{flushleft}
Course-5
\end{flushleft}


0


4


\begin{flushleft}
CMP100
\end{flushleft}





2





0





0





4





2





\begin{flushleft}
Chemistry Laboratory
\end{flushleft}





0





\begin{flushleft}
Physics Laboratory
\end{flushleft}





\begin{flushleft}
PYP100
\end{flushleft}





\begin{flushleft}
Course-6
\end{flushleft}


0





0





4





2





\begin{flushleft}
Product Realization
\end{flushleft}


\begin{flushleft}
through Manufacturing
\end{flushleft}





\begin{flushleft}
MCP101
\end{flushleft}





0





0





2





1





0


1


\begin{flushleft}
NEN100
\end{flushleft}





0.5





0





0





1





0.5





\begin{flushleft}
Professional Ethics and
\end{flushleft}


\begin{flushleft}
Social Responsibility-2 (Nongraded)
\end{flushleft}





0





\begin{flushleft}
Professional Ethics and
\end{flushleft}


\begin{flushleft}
Social Responsibility-1 (Nongraded)
\end{flushleft}





\begin{flushleft}
Introduction to
\end{flushleft}


\begin{flushleft}
Engineering
\end{flushleft}


\begin{flushleft}
(Non-graded)
\end{flushleft}





\begin{flushleft}
Course-7
\end{flushleft}





\begin{flushleft}
NEN100
\end{flushleft}





\begin{flushleft}
Course-8
\end{flushleft}





\begin{flushleft}
NIN100
\end{flushleft}





0





0





0





2





\begin{flushleft}
Language and
\end{flushleft}


\begin{flushleft}
Writing Skills-2
\end{flushleft}


\begin{flushleft}
(Non-Graded)
\end{flushleft}





0 2


\begin{flushleft}
NLN100
\end{flushleft}





\begin{flushleft}
Language and
\end{flushleft}


\begin{flushleft}
Writing Skills-1
\end{flushleft}


\begin{flushleft}
(Non-Graded)
\end{flushleft}





\begin{flushleft}
NLN100
\end{flushleft}





\begin{flushleft}
Course-9
\end{flushleft}


1





1





12





9.5





\begin{flushleft}
L
\end{flushleft}





0


4


\begin{flushleft}
ELL205
\end{flushleft}





5





1


0


\begin{flushleft}
COL351
\end{flushleft}





4





1


0


\begin{flushleft}
COL352
\end{flushleft}





4





0


0


\begin{flushleft}
COL703
\end{flushleft}





3





0





\begin{flushleft}
COD892
\end{flushleft}





0





0


2


\begin{flushleft}
HUL3XX
\end{flushleft}


3





4





0 14


\begin{flushleft}
COD893
\end{flushleft}





7





0





0





28





14





\begin{flushleft}
M.Tech Project Part II
\end{flushleft}





0





\begin{flushleft}
M.Tech Project Part I
\end{flushleft}





3





3





\begin{flushleft}
Logic for Computer
\end{flushleft}


\begin{flushleft}
Science
\end{flushleft}





3





\begin{flushleft}
Intro to Automata \&
\end{flushleft}


\begin{flushleft}
Theory of Computation
\end{flushleft}





3





\begin{flushleft}
Analysis and Design of
\end{flushleft}


\begin{flushleft}
Algorithms
\end{flushleft}





3





\begin{flushleft}
Signals and Systems
\end{flushleft}





3





\begin{flushleft}
Data Structures \&
\end{flushleft}


\begin{flushleft}
Algorithms
\end{flushleft}





\begin{flushleft}
COL106
\end{flushleft}





0


0


\begin{flushleft}
CVL100
\end{flushleft}





3





0


0


\begin{flushleft}
SBL100
\end{flushleft}





2





0


2


\begin{flushleft}
MTLXXX
\end{flushleft}





4





3





3





3





3





0





0





0





\begin{flushleft}
OC (3)
\end{flushleft}





0





0


0


\begin{flushleft}
PE 2 (3)
\end{flushleft}





0


0


\begin{flushleft}
OC 1 (3)
\end{flushleft}





3





3





3





3





\begin{flushleft}
Programme-Linked
\end{flushleft}


\begin{flushleft}
course in Mathematics
\end{flushleft}





3





\begin{flushleft}
Introductory Biology for
\end{flushleft}


\begin{flushleft}
Engineers
\end{flushleft}





2





\begin{flushleft}
Environmental Science
\end{flushleft}





3





\begin{flushleft}
Programme-linked
\end{flushleft}


\begin{flushleft}
courses in Physics
\end{flushleft}





\begin{flushleft}
PYLXXX
\end{flushleft}





3





3





3





3





3





0





0





0





\begin{flushleft}
PE 3 (3)
\end{flushleft}





1





1


0


\begin{flushleft}
HUL2XX
\end{flushleft}





1


0


\begin{flushleft}
HUL2XX
\end{flushleft}





1


0


\begin{flushleft}
HUL2XX
\end{flushleft}





3





4





4





4





4





\begin{flushleft}
Probability and
\end{flushleft}


\begin{flushleft}
Stochastic Processes
\end{flushleft}





\begin{flushleft}
MTL106
\end{flushleft}





6





\begin{flushleft}
COL380
\end{flushleft}





0





3





3





3





2





0





0





0


0


\begin{flushleft}
OC (2)
\end{flushleft}





0


2


\begin{flushleft}
PE 1 (3)
\end{flushleft}





3





3





3





\begin{flushleft}
Intro to Parallel \&
\end{flushleft}


\begin{flushleft}
Distributed Programming
\end{flushleft}





0





\begin{flushleft}
Design Practices
\end{flushleft}





\begin{flushleft}
COP290
\end{flushleft}





0





0





2





1





\begin{flushleft}
Intro. to Comp. Science
\end{flushleft}


\begin{flushleft}
and Engineering
\end{flushleft}


\begin{flushleft}
(Nongraded)
\end{flushleft}





\begin{flushleft}
C0N101
\end{flushleft}





0





9





15





15





17





15





14





15





\begin{flushleft}
Note: Courses 1-6 above are attended in the given order by half of all first year students. The other half of First year students attend the Courses 1-6 of II semester first.
\end{flushleft}





\begin{flushleft}
Digital Logic \& System
\end{flushleft}


\begin{flushleft}
Design
\end{flushleft}





4





\begin{flushleft}
COL202
\end{flushleft}





0





2





\begin{flushleft}
Introduction to Computer
\end{flushleft}


\begin{flushleft}
Introduction to Chemistry
\end{flushleft}


\begin{flushleft}
Science
\end{flushleft}





0


3


\begin{flushleft}
COL100
\end{flushleft}





\begin{flushleft}
Discrete Mathematical
\end{flushleft}


\begin{flushleft}
Structures
\end{flushleft}





1





\begin{flushleft}
Programming Languages
\end{flushleft}





3





3





\begin{flushleft}
Engineering Mechanics
\end{flushleft}





0


2


\begin{flushleft}
APL100
\end{flushleft}





0.5





\begin{flushleft}
PYL100
\end{flushleft}





3





\begin{flushleft}
Course-1
\end{flushleft}





\begin{flushleft}
Introduction to Engineering Electromagnetic Waves
\end{flushleft}


\begin{flushleft}
Visualization
\end{flushleft}


\begin{flushleft}
and Quantum Mechanics
\end{flushleft}





\begin{flushleft}
MCP100
\end{flushleft}





\begin{flushleft}
Course-3
\end{flushleft}





\begin{flushleft}
Introduction to Electrical
\end{flushleft}


\begin{flushleft}
Engineering
\end{flushleft}





\begin{flushleft}
Course-2
\end{flushleft}





\begin{flushleft}
ELL100
\end{flushleft}





0





0





0





0





1





2





2





2





2





1





\begin{flushleft}
T
\end{flushleft}





19.0





17.0





22.0





20.0





22.0





21.0





0





28.0





23.0





23.0





19.0





26.0





23.0





28.0





27.0





\begin{flushleft}
TOTAL=185.0
\end{flushleft}





28 14.0





0





0





0





0





0





0





1





17.0 1.5 23.0





17.0 2.5 28.5





\begin{flushleft}
Credits
\end{flushleft}





14 16.0





8





4





8





6





12





8





6





13





\begin{flushleft}
P
\end{flushleft}





\begin{flushleft}
Non-Graded Units
\end{flushleft}





\begin{flushleft}
Dual Degree Programme: B.Tech. and M.Tech. in Computer Science and Engineering	CS5
\end{flushleft}


\begin{flushleft}
Contact Hours
\end{flushleft}


185.0





\begin{flushleft}
\newpage
Bachelor of Technology in Electrical Engineering
\end{flushleft}





\begin{flushleft}
Programme Code: EE1
\end{flushleft}





\begin{flushleft}
Department of Electrical Engineering
\end{flushleft}


\begin{flushleft}
The overall Credit Structure
\end{flushleft}





\begin{flushleft}
ELL205	 Signals and Systems	
\end{flushleft}


\begin{flushleft}
ELL211	 Physical Electronics	
\end{flushleft}


\begin{flushleft}
ELL212	 Engineering Electromagnetics	
\end{flushleft}


\begin{flushleft}
ELP212	 Electromagnetics Laboratory	
\end{flushleft}


\begin{flushleft}
ELL225	 Control Engineering-I	
\end{flushleft}


\begin{flushleft}
ELP225	 Control Engineering Laboratory	
\end{flushleft}


\begin{flushleft}
ELL302	 Power Electronics	
\end{flushleft}


\begin{flushleft}
ELP302	 Power Electronics Laboratory	
\end{flushleft}


\begin{flushleft}
ELL303	 Power Engineering-I	
\end{flushleft}


\begin{flushleft}
ELP303	 Power Engineering Laboratory	
\end{flushleft}


\begin{flushleft}
ELL304	 Analog Electronic Circuits	
\end{flushleft}


\begin{flushleft}
ELL305	 Computer Architecture	
\end{flushleft}


\begin{flushleft}
ELP305	 Design and System Laboratory	
\end{flushleft}


\begin{flushleft}
ELL311 	 Communication Engineering	
\end{flushleft}


\begin{flushleft}
ELP311	 Communication Engineering Laboratory	
\end{flushleft}


\begin{flushleft}
ELD411	 B.Tech. Project-I	
\end{flushleft}





\begin{flushleft}
Course Category	
\end{flushleft}


\begin{flushleft}
Credits
\end{flushleft}


\begin{flushleft}
Institute Core Courses
\end{flushleft}


\begin{flushleft}
Basic Sciences (BS)		 22
\end{flushleft}


\begin{flushleft}
Engineering Arts and Science (EAS)		 18
\end{flushleft}


\begin{flushleft}
Humanities and Social Sciences (HuSS)		 15
\end{flushleft}


\begin{flushleft}
Programme-linked Courses		15
\end{flushleft}


\begin{flushleft}
Departmental Courses
\end{flushleft}


\begin{flushleft}
Departmental Core 		 60
\end{flushleft}


\begin{flushleft}
Departmental Electives		 10
\end{flushleft}


\begin{flushleft}
Open Category Courses		 10
\end{flushleft}


\begin{flushleft}
Total Graded Credit requirement		 150
\end{flushleft}


\begin{flushleft}
Non Graded Units		 15
\end{flushleft}


\begin{flushleft}
Institute Core: Basic Sciences
\end{flushleft}


\begin{flushleft}
CML100	 General Chemistry	
\end{flushleft}


\begin{flushleft}
CMP100	Chemistry Laboratory	
\end{flushleft}


\begin{flushleft}
MTL100	 Calculus	
\end{flushleft}


\begin{flushleft}
MTL101	 Linear Algebra and Differential Equations	
\end{flushleft}


\begin{flushleft}
PYL100	 Electromagnetic Waves and 	
\end{flushleft}


	


\begin{flushleft}
Quantum Mechanics
\end{flushleft}


\begin{flushleft}
PYP100	 Physics Laboratory	
\end{flushleft}


\begin{flushleft}
SBL100	 Introductory Biology for Engineers	
\end{flushleft}


	





3	 0	 0	 3


0	0	4	2


3	1	0	4


3	 1	 0	 4


3	 0	 0	 3





	





\begin{flushleft}
ELL301 	 Electrical and Electronics Instrumentation 	
\end{flushleft}


3	 0	 0	 3


\begin{flushleft}
ELL312 	 Semiconductor process technology 	
\end{flushleft}


3	 0	 0	 3


\begin{flushleft}
ELL313 	 Antennas and Propagation 	
\end{flushleft}


3	 0	 0	 3


\begin{flushleft}
ELL315	 Introduction to Analog Integrated Circuits	
\end{flushleft}


3	 0	 0	 3


\begin{flushleft}
ELL316	 Introduction to VLSI Design	
\end{flushleft}


3	 0	 0	 3


\begin{flushleft}
ELL318	 Digital Hardware Design 	
\end{flushleft}


3	 0	 0	 3


\begin{flushleft}
ELL319	 Digital Signal Processing 	
\end{flushleft}


3	 0	 2	 4


\begin{flushleft}
ELL332 	Electric Drives	
\end{flushleft}


3	0	0	3


\begin{flushleft}
ELL333 	 Multivariable Control 	
\end{flushleft}


3	 0	 0	 3


\begin{flushleft}
ELL365	 Embedded Systems	
\end{flushleft}


3	0	0	3


\begin{flushleft}
ELL400 	 Power Systems Protection 	
\end{flushleft}


3	 0	 0	 3


\begin{flushleft}
ELL401 	 Advanced Electromechanics 	
\end{flushleft}


3	 0	 0	 3


\begin{flushleft}
ELL402 	Computer Communication	
\end{flushleft}


3	0	0	3


\begin{flushleft}
ELL405	 Operating Systems 	
\end{flushleft}


3	0	0	3


\begin{flushleft}
ELL406	 Robotics and Automation 	
\end{flushleft}


3	0	0	3


\begin{flushleft}
ELL407	 Power Quality 	
\end{flushleft}


3	0	2	4


\begin{flushleft}
ELL409	 Machine Intelligence and Learning	
\end{flushleft}


3	 0	 2	 4


\begin{flushleft}
ELL410 	 Multicore Systems 	
\end{flushleft}


3	 0	 0	 3


\begin{flushleft}
ELL411 	 Digital Communications 	
\end{flushleft}


3	 0	 2	 4


\begin{flushleft}
ELL703 	 Optimal Control Theory 	
\end{flushleft}


3	 0	 0	 3


\begin{flushleft}
ELL710	 Coding Theory	
\end{flushleft}


3	0	0	3


\begin{flushleft}
ELL715	 Digital Image Processing		3	0	2	4
\end{flushleft}


\begin{flushleft}
ELL716 	 Telecommunication Switching and Transmission	
\end{flushleft}


3	0	0	3


\begin{flushleft}
ELL725 	Wireless Communications		3	0	0	3
\end{flushleft}


\begin{flushleft}
ELL730	 I.C. Technology		3	0	0	3
\end{flushleft}


\begin{flushleft}
ELL738 	 Micro and Nano Photonics		 3	 0	 0	 3
\end{flushleft}


\begin{flushleft}
ELL740	 Compact Modeling of Semiconductor Devices		 3	 0	 2	 4
\end{flushleft}


\begin{flushleft}
ELL758 	Power Quality		3	0	0	3
\end{flushleft}


\begin{flushleft}
ELL765 	 Smart Grid Technology		 3	 0	 0	 3
\end{flushleft}


\begin{flushleft}
ELS310	 Independent Study (EL)	
\end{flushleft}


0	3	0	3





0	0	4	2


3	 0	 2	 4





\begin{flushleft}
Institute Core: Engineering Arts and Sciences
\end{flushleft}





	





3	1	0	4


3	 0	 2	 4


2	0	0	2


3	 0	 2	 4


0	0	4	2


0	 0	 4	 2





\begin{flushleft}
Total Credits				18
\end{flushleft}





\begin{flushleft}
Programme-Linked Basic / Engineering Arts / Sciences Core
\end{flushleft}


\begin{flushleft}
COL106	Data Structures and Algorithms	
\end{flushleft}


\begin{flushleft}
MTL106	 Probability and Stochastic Processes	
\end{flushleft}


\begin{flushleft}
MCL142	 Thermal Science for Electrical Engineers	
\end{flushleft}


\begin{flushleft}
PYL102	 Principles of Electronic Materials	
\end{flushleft}


	





3	0	4	5


3	 1	 0	 4


3	 0	 0	 3


3	 0	 0	 3





\begin{flushleft}
Total Credits				15
\end{flushleft}





\begin{flushleft}
Departmental Core
\end{flushleft}


\begin{flushleft}
ELL201	 Digital Electronics	
\end{flushleft}


\begin{flushleft}
ELL202	 Circuit Theory	
\end{flushleft}


\begin{flushleft}
ELL203	 Electromechanics	
\end{flushleft}


\begin{flushleft}
ELP203	 Electromechanics Laboratory	
\end{flushleft}





\begin{flushleft}
Total Credits				60	
\end{flushleft}





\begin{flushleft}
Departmental Electives
\end{flushleft}





\begin{flushleft}
Total Credits				22
\end{flushleft}





\begin{flushleft}
APL100	 Engineering Mechanics	
\end{flushleft}


\begin{flushleft}
COL100	 Introduction to Computer Science	
\end{flushleft}


\begin{flushleft}
CVL100	 Environmental Science	
\end{flushleft}


\begin{flushleft}
ELL100	 Introduction to Electrical Engineering	
\end{flushleft}


\begin{flushleft}
MCP100	Engineering Visualization	
\end{flushleft}


\begin{flushleft}
MCP101	 Introduction to Product Realization 	
\end{flushleft}


	


\begin{flushleft}
through Manufacturing
\end{flushleft}





3	1	0	4


3	0	0	3


3	1	0	4


0	0	3	1.5


3	1	0	4


0	0	3	1.5


3	0	0	3


0	0	3	1.5


3	1	0	4


0	0	3	1.5


3	1	3	5.5


3	0	0	3


0	 0	 3	 1.5


3	1	0	4


0	0	2	1


0	0	6	3





3	0	3	4.5


3	1	0	4


3	1	0	4


0	0	3	1.5





53





\newpage
54





\begin{flushleft}
Semester
\end{flushleft}





\begin{flushleft}
VIII
\end{flushleft}





\begin{flushleft}
VII
\end{flushleft}





\begin{flushleft}
VI
\end{flushleft}





\begin{flushleft}
V
\end{flushleft}





\begin{flushleft}
IV
\end{flushleft}





\begin{flushleft}
III
\end{flushleft}





\begin{flushleft}
II
\end{flushleft}





\begin{flushleft}
I
\end{flushleft}





\begin{flushleft}
MCP100
\end{flushleft}





\begin{flushleft}
PYL100
\end{flushleft}





\begin{flushleft}
Course-3
\end{flushleft}





\begin{flushleft}
Course-1
\end{flushleft}





0 3


\begin{flushleft}
COL100
\end{flushleft}





2





4





0 3 4.5


\begin{flushleft}
ELL304
\end{flushleft}





\begin{flushleft}
Digital Electronics
\end{flushleft}





1 0


\begin{flushleft}
ELL201
\end{flushleft}





\begin{flushleft}
Circuit Theory
\end{flushleft}





4





4





3





0





0





3





3





0





0





\begin{flushleft}
DE 3
\end{flushleft}





0





3





3





3





3





0





0





1 0


\begin{flushleft}
OC2
\end{flushleft}





1 0


\begin{flushleft}
HUL2XX
\end{flushleft}





3





4





4





4





4





5





1 0


\begin{flushleft}
SBL100
\end{flushleft}





4





0 2


\begin{flushleft}
CVL100
\end{flushleft}





4





2





0 0


\begin{flushleft}
PYL102
\end{flushleft}





2





\begin{flushleft}
Environmental Science
\end{flushleft}





3





\begin{flushleft}
Introductory Biology for
\end{flushleft}


\begin{flushleft}
Engineers
\end{flushleft}





3





3





3





0





0


\begin{flushleft}
OC3
\end{flushleft}


0





0





0 0


\begin{flushleft}
DE 2
\end{flushleft}





3





3





3





3





0





3





1 0


\begin{flushleft}
HUL2XX
\end{flushleft}





\begin{flushleft}
Communication
\end{flushleft}


\begin{flushleft}
Engineering
\end{flushleft}





1 0


\begin{flushleft}
ELL311
\end{flushleft}





\begin{flushleft}
Engineering
\end{flushleft}


\begin{flushleft}
Electromagnetics
\end{flushleft}





0 4


\begin{flushleft}
ELL212
\end{flushleft}





\begin{flushleft}
Electromechanics
\end{flushleft}





3





3





3





3





\begin{flushleft}
ELL203
\end{flushleft}





\begin{flushleft}
COL106
\end{flushleft}





\begin{flushleft}
Data Structures \&
\end{flushleft}


\begin{flushleft}
Algorithms
\end{flushleft}





\begin{flushleft}
Principles of Electronic
\end{flushleft}


\begin{flushleft}
Materials
\end{flushleft}





1 3 5.5


\begin{flushleft}
MCL142
\end{flushleft}





2





3





1 0


\begin{flushleft}
MTL101
\end{flushleft}





\begin{flushleft}
Calculus
\end{flushleft}





4





3





1





0





4





\begin{flushleft}
Linear Algebra and
\end{flushleft}


\begin{flushleft}
Differential Equations
\end{flushleft}





3





\begin{flushleft}
Course-4
\end{flushleft}





\begin{flushleft}
MTL100
\end{flushleft}





\begin{flushleft}
Course-5
\end{flushleft}





0 4 2


\begin{flushleft}
CMP100
\end{flushleft}





0





0





4





2





\begin{flushleft}
Chemistry Laboratory
\end{flushleft}





0





\begin{flushleft}
Physics Laboratory
\end{flushleft}





\begin{flushleft}
PYP100
\end{flushleft}





\begin{flushleft}
Course-6
\end{flushleft}





0





0





4





2





\begin{flushleft}
Product Realization
\end{flushleft}


\begin{flushleft}
through Manufacturing
\end{flushleft}





\begin{flushleft}
MCP101
\end{flushleft}





\begin{flushleft}
Course-7
\end{flushleft}





0





0





2





1





\begin{flushleft}
Introduction to Engineering
\end{flushleft}


\begin{flushleft}
(Non-graded)
\end{flushleft}





\begin{flushleft}
NIN100
\end{flushleft}





0





0





0 2


\begin{flushleft}
NLN100
\end{flushleft}





1





0





2





1





\begin{flushleft}
Language and Writing
\end{flushleft}


\begin{flushleft}
Skills-2 (Non-Graded)
\end{flushleft}





1 0.5 0





\begin{flushleft}
Professional Ethics and
\end{flushleft}


\begin{flushleft}
Social Responsibility-2
\end{flushleft}


\begin{flushleft}
(Non-graded)
\end{flushleft}





0 1 0.5 0


\begin{flushleft}
NEN100
\end{flushleft}





\begin{flushleft}
Language and Writing
\end{flushleft}


\begin{flushleft}
Skills-1 (Non-Graded)
\end{flushleft}





\begin{flushleft}
Professional Ethics and
\end{flushleft}


\begin{flushleft}
Social Responsibility-1
\end{flushleft}


\begin{flushleft}
(Non-graded)
\end{flushleft}





0





\begin{flushleft}
NLN100
\end{flushleft}





\begin{flushleft}
Course-8
\end{flushleft}





\begin{flushleft}
NEN100
\end{flushleft}





\begin{flushleft}
Course-9
\end{flushleft}





12





9.5





\begin{flushleft}
L
\end{flushleft}





0 0


\begin{flushleft}
MTL106
\end{flushleft}





3





0 0


\begin{flushleft}
ELL303
\end{flushleft}





\begin{flushleft}
Power Electronics
\end{flushleft}





1 0


\begin{flushleft}
ELL302
\end{flushleft}





3





3





3


\begin{flushleft}
OC1
\end{flushleft}





0





0





0





0 2


\begin{flushleft}
HUL3XX
\end{flushleft}





1





\begin{flushleft}
Power Engineering-I
\end{flushleft}





3





3





3





4





4





3





4





\begin{flushleft}
Probability and Stochastic
\end{flushleft}


\begin{flushleft}
Processes
\end{flushleft}





3





\begin{flushleft}
Physical Electronics
\end{flushleft}





\begin{flushleft}
ELL211
\end{flushleft}





1 0


\begin{flushleft}
ELL225
\end{flushleft}





4





1 0


\begin{flushleft}
ELL305
\end{flushleft}





4





0





3





3





0





6





\begin{flushleft}
B.Tech. Project
\end{flushleft}





0 2


\begin{flushleft}
ELD411
\end{flushleft}





0 0


\begin{flushleft}
DE 1
\end{flushleft}





3





4





3





\begin{flushleft}
Computer Architecture
\end{flushleft}





3





\begin{flushleft}
Control Engineering-I
\end{flushleft}





3





\begin{flushleft}
Signals and Systems
\end{flushleft}





\begin{flushleft}
ELL205
\end{flushleft}





4





0 3 1.5


\begin{flushleft}
ELP311
\end{flushleft}





\begin{flushleft}
Electromagnetics
\end{flushleft}


\begin{flushleft}
Laboratory
\end{flushleft}





0 3 1.5


\begin{flushleft}
ELP212
\end{flushleft}





\begin{flushleft}
Electromechanics
\end{flushleft}


\begin{flushleft}
Laboratory
\end{flushleft}





1 0


\begin{flushleft}
ELP203
\end{flushleft}





\begin{flushleft}
ELN101
\end{flushleft}





2





\begin{flushleft}
ELP225
\end{flushleft}





0





1





0





0


3


\begin{flushleft}
ELP305
\end{flushleft}





1.5





\begin{flushleft}
Control Engineering Lab
\end{flushleft}





0





\begin{flushleft}
Introduction to Electrical
\end{flushleft}


\begin{flushleft}
Engineering
\end{flushleft}


\begin{flushleft}
(Non-graded)
\end{flushleft}





0 2


\begin{flushleft}
ELP303
\end{flushleft}


0





0





1





3 1.5





\begin{flushleft}
Power Engineering
\end{flushleft}


\begin{flushleft}
Laboratory
\end{flushleft}





0





0





0





3





1.5





\begin{flushleft}
Communication
\end{flushleft}


\begin{flushleft}
Design and System Laboratory
\end{flushleft}


\begin{flushleft}
Engineering Laboratory
\end{flushleft}





0





0





3





\begin{flushleft}
HUL 2XX
\end{flushleft}





0





0





3 1.5





\begin{flushleft}
Power Electronics
\end{flushleft}


\begin{flushleft}
Laboratory
\end{flushleft}





\begin{flushleft}
ELP302
\end{flushleft}





\begin{flushleft}
P
\end{flushleft}





\begin{flushleft}
Credits
\end{flushleft}





2





9 20.5 1.5 25.0





8 22.0 0 26.0





\begin{flushleft}
TOTAL=150.0
\end{flushleft}





0 12.0 0 12.0 150.0





1 11 15.5 0 21.0





12.0 0





9.0





15.0 2 10 22.0 3 27.0





14.0 2





15.0 3





4 24.0 1 28.0





6 17.0 1.5 23.0





1 13 17.0 2.5 28.5





\begin{flushleft}
T
\end{flushleft}





18.0 4





\begin{flushleft}
Note: Courses 1-6 above are attended in the given order by half of all first year students. The other half of First year students attend the Courses 1-6 of II semester first.
\end{flushleft}





0





0 0


\begin{flushleft}
CML100
\end{flushleft}





\begin{flushleft}
Introduction to Chemistry
\end{flushleft}





3





\begin{flushleft}
Thermal Science for
\end{flushleft}


\begin{flushleft}
Electrical Engineers
\end{flushleft}





3





\begin{flushleft}
Analog Electronic Circuits
\end{flushleft}





3





3





0





\begin{flushleft}
ELL202
\end{flushleft}





1





3





0.5





3





4





\begin{flushleft}
Introduction to Computer
\end{flushleft}


\begin{flushleft}
Science
\end{flushleft}





0 2


\begin{flushleft}
APL100
\end{flushleft}





\begin{flushleft}
Engineering Mechanics
\end{flushleft}





3





\begin{flushleft}
Introduction to Electrical Introduction to Engineering Electromagnetic Waves
\end{flushleft}


\begin{flushleft}
Engineering
\end{flushleft}


\begin{flushleft}
Visualization
\end{flushleft}


\begin{flushleft}
and Quantum Mechanics
\end{flushleft}





\begin{flushleft}
Course-2
\end{flushleft}





\begin{flushleft}
ELL100
\end{flushleft}





\begin{flushleft}
Non-Graded
\end{flushleft}


\begin{flushleft}
Units
\end{flushleft}





\begin{flushleft}
B.Tech. in Electrical Engineering	EE1
\end{flushleft}


\begin{flushleft}
Contact Hours
\end{flushleft}





\begin{flushleft}
\newpage
Programme Code: EE3
\end{flushleft}





\begin{flushleft}
Bachelor of Technology in Electrical Engineering Power and Automation
\end{flushleft}


\begin{flushleft}
Department of Electrical Engineering
\end{flushleft}


\begin{flushleft}
The overall Credit Structure
\end{flushleft}





\begin{flushleft}
ELP203	 Electromechanics Laboratory	
\end{flushleft}


\begin{flushleft}
ELL205	 Signals and Systems	
\end{flushleft}


\begin{flushleft}
ELL225	 Control Engineering-I	
\end{flushleft}


\begin{flushleft}
ELP225	 Control Engineering Laboratory	
\end{flushleft}


\begin{flushleft}
ELL231	 Power Electronics and Energy Devices	
\end{flushleft}


\begin{flushleft}
ELL302	 Power Electronics	
\end{flushleft}


\begin{flushleft}
ELP302	 Power Electronics Laboratory	
\end{flushleft}


\begin{flushleft}
ELL303	 Power Engineering-I	
\end{flushleft}


\begin{flushleft}
ELP303	 Power Engineering Laboratory	
\end{flushleft}


\begin{flushleft}
ELL304	 Analog Electronic Circuits	
\end{flushleft}


\begin{flushleft}
ELL305	 Computer Architecture	
\end{flushleft}


\begin{flushleft}
ELP305	 Design and System Laboratory	
\end{flushleft}


\begin{flushleft}
ELL332 	Electric Drives	
\end{flushleft}


\begin{flushleft}
ELP332	 Electric Drives Laboratory	
\end{flushleft}


\begin{flushleft}
ELL363	 Power Engineering-II	
\end{flushleft}


\begin{flushleft}
ELL365	 Embedded Systems	
\end{flushleft}


\begin{flushleft}
ELD431	 B.Tech. Project-I	
\end{flushleft}





\begin{flushleft}
Course Category	
\end{flushleft}


\begin{flushleft}
Credits
\end{flushleft}


\begin{flushleft}
Institute Core Courses
\end{flushleft}


\begin{flushleft}
Basic Sciences (BS)		 22
\end{flushleft}


\begin{flushleft}
Engineering Arts and Science (EAS)		 18
\end{flushleft}


\begin{flushleft}
Humanities and Social Sciences (HuSS)		 15
\end{flushleft}


\begin{flushleft}
Programme-linked Courses		14
\end{flushleft}


\begin{flushleft}
Departmental Courses
\end{flushleft}


\begin{flushleft}
Departmental Core 		 60
\end{flushleft}


\begin{flushleft}
Departmental Electives		 10
\end{flushleft}


\begin{flushleft}
Open Category Courses		 10
\end{flushleft}


\begin{flushleft}
Total Graded Credit requirement		 150
\end{flushleft}


\begin{flushleft}
Non Graded Units		 15
\end{flushleft}


\begin{flushleft}
Institute Core : Basic Sciences
\end{flushleft}


\begin{flushleft}
CML100	 General Chemistry	
\end{flushleft}


3	 0	 0	 3


\begin{flushleft}
CMP100	Chemistry Laboratory	
\end{flushleft}


0	0	4	2


\begin{flushleft}
MTL100	 Calculus	
\end{flushleft}


3	1	0	4


\begin{flushleft}
MTL101	 Linear Algebra and Differential Equations	
\end{flushleft}


3	 1	 0	 4


\begin{flushleft}
PYL100	 Electromagnetic Waves and	
\end{flushleft}


3	0	0	3


	


\begin{flushleft}
Quantum Mechanics
\end{flushleft}


\begin{flushleft}
PYP100	 Physics Laboratory	
\end{flushleft}


0	0	4	2


\begin{flushleft}
SBL100	 Introductory Biology for Engineers	
\end{flushleft}


3	 0	 2	 4


	


\begin{flushleft}
Total Credits				22
\end{flushleft}





	





0	0	3	1.5


3	1	0	4


3	1	0	4


0	0	3	1.5


3	 0	 0	 3


3	0	0	3


0	0	3	1.5


3	1	0	4


0	0	3	1.5


3	1	3	5.5


3	0	0	3


0	 0	 3	 1.5


3	0	0	3


0	0	3	1.5


3	0	0	3


3	0	0	3


0	0	6	3





\begin{flushleft}
Total Credits				60	
\end{flushleft}





\begin{flushleft}
Departmental Electives
\end{flushleft}





\begin{flushleft}
ELL301 	 Electrical and Electronics Instrumentation 	
\end{flushleft}


3	 0	 0	 3


\begin{flushleft}
ELL311 	 Communication Engineering	
\end{flushleft}


3	1	0	4


\begin{flushleft}
ELL319	 Digital Signal Processing 	
\end{flushleft}


3	 0	 2	 4


\begin{flushleft}
ELL333 	 Multivariable Control 	
\end{flushleft}


3	 0	 0	 3


\begin{flushleft}
Institute Core: Engineering Arts and Sciences
\end{flushleft}


\begin{flushleft}
ELL334	 Multivariable Control 	
\end{flushleft}


3	0	2	4


\begin{flushleft}
APL100	 Engineering Mechanics	
\end{flushleft}


3	1	0	4


\begin{flushleft}
ELL400 	 Power Systems Protection 	
\end{flushleft}


3	 0	 0	 3


\begin{flushleft}
COL100	 Introduction to Computer Science	
\end{flushleft}


3	 0	 2	 4


\begin{flushleft}
ELL401
\end{flushleft}


	


\begin{flushleft}
Advanced
\end{flushleft}


\begin{flushleft}
Electromechanics
\end{flushleft}


	


3	 0	 0	 3


\begin{flushleft}
CVL100	 Environmental Science	
\end{flushleft}


2	0	0	2


\begin{flushleft}
ELL405 	 Operating Systems 	
\end{flushleft}


3	 0	 0	 3


\begin{flushleft}
ELL100	 Introduction to Electrical Engineering	
\end{flushleft}


3	 0	 2	 4


\begin{flushleft}
ELL406
\end{flushleft}


	


\begin{flushleft}
Robotics
\end{flushleft}


\begin{flushleft}
and
\end{flushleft}


\begin{flushleft}
Automation
\end{flushleft}


	


3	 0	 0	 3


\begin{flushleft}
MCP100	Engineering Visualization	
\end{flushleft}


0	0	4	2


\begin{flushleft}
ELL407	 Power Quality 	
\end{flushleft}


3	0	2	4


\begin{flushleft}
MCP101	 Product Realization through Manufacturing	 0	 0	 4	 2
\end{flushleft}


\begin{flushleft}
ELL409	 Machine Intelligence and Learning	
\end{flushleft}


3	 0	 2	 4


	


\begin{flushleft}
Total Credits				18	
\end{flushleft}


\begin{flushleft}
ELL410 	 Multicore Systems 	
\end{flushleft}


3	 0	 0	 3


\begin{flushleft}
Programme-Linked Basic / Engineering Arts / Sciences Core
\end{flushleft}


\begin{flushleft}
ELL417	 Renewable Energy System	
\end{flushleft}


3	0	0	3


\begin{flushleft}
ELL431	 Power System Optimization	
\end{flushleft}


3	0	0	3


\begin{flushleft}
COL106	Data Structures and Algorithms	
\end{flushleft}


3	0	4	5


\begin{flushleft}
ELL436	 Digital Control	
\end{flushleft}


3	0	0	3


\begin{flushleft}
MTL106	 Probability and Stochastic Processes	
\end{flushleft}


3	 1	 0	 4


\begin{flushleft}
ELL437	 Switch Mode Power Conversion	
\end{flushleft}


3	 0	 0	 3


\begin{flushleft}
MCL142	 Thermal Science for Electrical Engineers	
\end{flushleft}


3	 0	 0	 3


\begin{flushleft}
ELL453	 Power System Dynamics and Control	
\end{flushleft}


3	 0	 0	 3


\begin{flushleft}
PYL102	 Principles of Electronic Materials	
\end{flushleft}


3	 0	 0	 3


\begin{flushleft}
ELL703 	 Optimal Control Theory 	
\end{flushleft}


3	 0	 0	 3


	


\begin{flushleft}
Total Credits				15
\end{flushleft}


\begin{flushleft}
ELL715	 Digital Image Processing		3	0	2	4
\end{flushleft}


\begin{flushleft}
Departmental Core
\end{flushleft}


\begin{flushleft}
ELL730	 I.C. Technology		3	0	0	3
\end{flushleft}


\begin{flushleft}
ELL758 	Power Quality		3	0	0	3
\end{flushleft}


\begin{flushleft}
ELL201	 Digital Electronics	
\end{flushleft}


3	0	3	4.5


\begin{flushleft}
ELL765 	 Smart Grid Technology		 3	 0	 0	 3
\end{flushleft}


\begin{flushleft}
ELL202	 Circuit Theory	
\end{flushleft}


3	1	0	4


\begin{flushleft}
ELS330 	 Independent Study (EP) 	
\end{flushleft}


0	 3	 0	 3


\begin{flushleft}
ELL203	 Electromechanics	
\end{flushleft}


3	1	0	4





55





\newpage
56





\begin{flushleft}
Semester
\end{flushleft}





\begin{flushleft}
VIII
\end{flushleft}





\begin{flushleft}
VII
\end{flushleft}





\begin{flushleft}
VI
\end{flushleft}





\begin{flushleft}
V
\end{flushleft}





\begin{flushleft}
IV
\end{flushleft}





\begin{flushleft}
III
\end{flushleft}





\begin{flushleft}
II
\end{flushleft}





\begin{flushleft}
I
\end{flushleft}





0


3


\begin{flushleft}
ELL304
\end{flushleft}





\begin{flushleft}
Digital Electronics
\end{flushleft}





1


0


\begin{flushleft}
ELL201
\end{flushleft}





4.5





4





3





3





3





0





\begin{flushleft}
DE 2
\end{flushleft}





0





0





0





\begin{flushleft}
Thermal Science for
\end{flushleft}


\begin{flushleft}
Electrical Engineers
\end{flushleft}





3





3





1


3 5.5


\begin{flushleft}
MCL142
\end{flushleft}





\begin{flushleft}
Analog Electronic Circuits
\end{flushleft}





3





3





\begin{flushleft}
Circuit Theory
\end{flushleft}





\begin{flushleft}
ELL202
\end{flushleft}





0





2





4





\begin{flushleft}
Course-3
\end{flushleft}





0 0


\begin{flushleft}
CML100
\end{flushleft}





3





3





0





0





3





\begin{flushleft}
Introduction to Chemistry
\end{flushleft}





3





\begin{flushleft}
Electromagnetic Waves
\end{flushleft}


\begin{flushleft}
and Quantum Mechanics
\end{flushleft}





\begin{flushleft}
PYL100
\end{flushleft}





1 0


\begin{flushleft}
MTL101
\end{flushleft}





\begin{flushleft}
Calculus
\end{flushleft}





4





3





1





0





4





\begin{flushleft}
Linear Algebra and
\end{flushleft}


\begin{flushleft}
Differential Equations
\end{flushleft}





3





\begin{flushleft}
Course-4
\end{flushleft}





\begin{flushleft}
MTL100
\end{flushleft}





\begin{flushleft}
Course-5
\end{flushleft}





0 4 2


\begin{flushleft}
CMP100
\end{flushleft}





0





0





4





2





\begin{flushleft}
Chemistry Laboratory
\end{flushleft}





0





\begin{flushleft}
Physics Laboratory
\end{flushleft}





\begin{flushleft}
PYP100
\end{flushleft}





\begin{flushleft}
Course-6
\end{flushleft}





0





0





4





2





\begin{flushleft}
Product Realization
\end{flushleft}


\begin{flushleft}
through Manufacturing
\end{flushleft}





\begin{flushleft}
MCP101
\end{flushleft}





0





0





2





1





0 1 0.5


\begin{flushleft}
NEN100
\end{flushleft}





0





0





1





0.5





\begin{flushleft}
Professional Ethics and
\end{flushleft}


\begin{flushleft}
Social Responsibility-2
\end{flushleft}


\begin{flushleft}
(Non-graded)
\end{flushleft}





0





\begin{flushleft}
Professional Ethics and
\end{flushleft}


\begin{flushleft}
Social Responsibility-1
\end{flushleft}


\begin{flushleft}
(Non-graded)
\end{flushleft}





\begin{flushleft}
Introduction to
\end{flushleft}


\begin{flushleft}
Engineering
\end{flushleft}


\begin{flushleft}
(Non-graded)
\end{flushleft}





\begin{flushleft}
Course-7
\end{flushleft}





\begin{flushleft}
NEN100
\end{flushleft}





\begin{flushleft}
Course-8
\end{flushleft}





\begin{flushleft}
NIN100
\end{flushleft}





0





0





0





2





\begin{flushleft}
Language and
\end{flushleft}


\begin{flushleft}
Writing Skills-2
\end{flushleft}


\begin{flushleft}
(Non-Graded)
\end{flushleft}





0 2


\begin{flushleft}
NLN100
\end{flushleft}





\begin{flushleft}
Language and
\end{flushleft}


\begin{flushleft}
Writing Skills-1
\end{flushleft}


\begin{flushleft}
(Non-Graded)
\end{flushleft}





\begin{flushleft}
NLN100
\end{flushleft}





\begin{flushleft}
Course-9
\end{flushleft}





0


4


\begin{flushleft}
ELL231
\end{flushleft}





5





3





3





3





3





3





0





0





1


0


\begin{flushleft}
DE 3
\end{flushleft}





0


0


\begin{flushleft}
HUL2XX
\end{flushleft}





\begin{flushleft}
Embedded Systems
\end{flushleft}





0


2


\begin{flushleft}
ELL365
\end{flushleft}





0


0


\begin{flushleft}
DE 1
\end{flushleft}





3





4





3





4





3





1 0


\begin{flushleft}
CVL100
\end{flushleft}





4





0 0


\begin{flushleft}
PYL102
\end{flushleft}





\begin{flushleft}
Power Electronics
\end{flushleft}





0 0


\begin{flushleft}
ELL302
\end{flushleft}


3





2





0 0


\begin{flushleft}
ELL363
\end{flushleft}





3





3





3


0





0





0 0


\begin{flushleft}
OC2
\end{flushleft}


3





3





\begin{flushleft}
Power Engineering-II
\end{flushleft}





3





\begin{flushleft}
Principles of Electronic
\end{flushleft}


\begin{flushleft}
Materials
\end{flushleft}





3





2





\begin{flushleft}
Environmental Science
\end{flushleft}





3





\begin{flushleft}
Electromechanics
\end{flushleft}





\begin{flushleft}
Data Structures \&
\end{flushleft}


\begin{flushleft}
Algorithms
\end{flushleft}





\begin{flushleft}
Power Electronics and
\end{flushleft}


\begin{flushleft}
Energy Devices
\end{flushleft}





3





\begin{flushleft}
ELL203
\end{flushleft}





\begin{flushleft}
COL106
\end{flushleft}





0 2


\begin{flushleft}
MTL106
\end{flushleft}





4





1 0


\begin{flushleft}
ELL303
\end{flushleft}





1 0


\begin{flushleft}
HUL2XX
\end{flushleft}


4





4





3





3





3





0





0





0 2


\begin{flushleft}
OC3
\end{flushleft}





1 0


\begin{flushleft}
OC1
\end{flushleft}





3





4





4





\begin{flushleft}
Power Engineering-I
\end{flushleft}





3





3





\begin{flushleft}
Probability and
\end{flushleft}


\begin{flushleft}
Stochastic Processes
\end{flushleft}





3





\begin{flushleft}
Introductory Biology for
\end{flushleft}


\begin{flushleft}
Engineers
\end{flushleft}





\begin{flushleft}
SBL100
\end{flushleft}





1 0


\begin{flushleft}
ELL225
\end{flushleft}





4





1 0


\begin{flushleft}
ELL305
\end{flushleft}





4





3





0





3





3





0





0





0 6


\begin{flushleft}
HUL3XX
\end{flushleft}





\begin{flushleft}
B.Tech. Project
\end{flushleft}





0 0


\begin{flushleft}
ELD431
\end{flushleft}





\begin{flushleft}
Electric Drives
\end{flushleft}





0 0


\begin{flushleft}
ELL332
\end{flushleft}





3





3





3





3





\begin{flushleft}
Computer Architecture
\end{flushleft}





3





\begin{flushleft}
Control Engineering-I
\end{flushleft}





3





\begin{flushleft}
Signals and Systems
\end{flushleft}





\begin{flushleft}
ELL205
\end{flushleft}





0 3 1.5


\begin{flushleft}
ELP225
\end{flushleft}


0 3 1.5


\begin{flushleft}
ELP305
\end{flushleft}





0





0





0





3





1.5





\begin{flushleft}
Power Engineering
\end{flushleft}


\begin{flushleft}
Laboratory
\end{flushleft}





0 3 1.5


\begin{flushleft}
ELP303
\end{flushleft}





\begin{flushleft}
Design and System
\end{flushleft}


\begin{flushleft}
Laboratory
\end{flushleft}





0





\begin{flushleft}
Control Engineering-I
\end{flushleft}





0





\begin{flushleft}
Electromechanics
\end{flushleft}


\begin{flushleft}
Laboratory
\end{flushleft}





\begin{flushleft}
ELP203
\end{flushleft}





\begin{flushleft}
ELN111
\end{flushleft}





0





\begin{flushleft}
ELP302
\end{flushleft}





1





0 2


\begin{flushleft}
HUL2XX
\end{flushleft}





0





0





4





1





0





3 1.5





\begin{flushleft}
Electric Drives
\end{flushleft}


\begin{flushleft}
Laboratory
\end{flushleft}





0 3 1.5


\begin{flushleft}
ELP332
\end{flushleft}





\begin{flushleft}
Power Electronics
\end{flushleft}


\begin{flushleft}
Laboratory
\end{flushleft}





3





0





\begin{flushleft}
Intro. to Elec. Engg
\end{flushleft}


\begin{flushleft}
Power \& Automation
\end{flushleft}


\begin{flushleft}
(Non-graded)
\end{flushleft}





\begin{flushleft}
T
\end{flushleft}





\begin{flushleft}
P
\end{flushleft}





\begin{flushleft}
Credits
\end{flushleft}





1 12 2





6 19.0





8 21.0





6 23.0





15.0 150.0





\begin{flushleft}
TOTAL=150.0
\end{flushleft}





0 15.0





24.0





22.0





25.0





26.0





6 21.0 1 26.0





1 14 17.0


15 0





9





15 1





15 2





17 3





15 3





6 17.0 1.5 23.0





1 9.5 1 13 17.0 2.5 28.5





\begin{flushleft}
L
\end{flushleft}





\begin{flushleft}
Note: Courses 1-6 above are attended in the given order by half of all first year students. The other half of First year students attend the Courses 1-6 of II semester first.
\end{flushleft}





4





3





0





3





1





\begin{flushleft}
Introduction to Computer
\end{flushleft}


\begin{flushleft}
Science
\end{flushleft}





\begin{flushleft}
Engineering Mechanics
\end{flushleft}





2





0.5





3





0


3


\begin{flushleft}
COL100
\end{flushleft}





\begin{flushleft}
Introduction to Engineering
\end{flushleft}


\begin{flushleft}
Visualization
\end{flushleft}





\begin{flushleft}
Introduction to Electrical
\end{flushleft}


\begin{flushleft}
Engineering
\end{flushleft}





4





\begin{flushleft}
MCP100
\end{flushleft}





\begin{flushleft}
Course-1
\end{flushleft}





0


2


\begin{flushleft}
APL100
\end{flushleft}





\begin{flushleft}
Course-2
\end{flushleft}





\begin{flushleft}
ELL100
\end{flushleft}





\begin{flushleft}
Non-Graded Units
\end{flushleft}





\begin{flushleft}
B.Tech. in Electrical Engineering Power and Automation	EE3
\end{flushleft}


\begin{flushleft}
Contact Hours
\end{flushleft}





\begin{flushleft}
\newpage
Bachelor of Technology in Mechanical Engineering
\end{flushleft}





\begin{flushleft}
Programme Code: ME1
\end{flushleft}





\begin{flushleft}
Department of Mechanical Engineering
\end{flushleft}


\begin{flushleft}
The overall Credit Structure
\end{flushleft}





\begin{flushleft}
MCL261	 Introduction to Operations Research 	
\end{flushleft}


3	 0	 0	 3


\begin{flushleft}
MCP301	 Mechanical Engineering Laboratory-I 	
\end{flushleft}


0	 0	 3	 1.5


\begin{flushleft}
MCL311	 CAD and Finite Element Analysis 	
\end{flushleft}


3	 0	 2	 4


\begin{flushleft}
MCP331	 Manufacturing Laboratory-II 	
\end{flushleft}


0	 0	 2	 1


\begin{flushleft}
MCL361	 Manufacturing System Design 	
\end{flushleft}


3	 0	 0	 3


\begin{flushleft}
MCP401	 Mechanical Engineering Laboratory-II	
\end{flushleft}


0	 0	 4	 2


\begin{flushleft}
MCD411	B.Tech.Project	
\end{flushleft}


0	0	 8	4


\begin{flushleft}
MCL431	 CAM and Automation 	
\end{flushleft}


2	 0	 2	 3


	


\begin{flushleft}
Total Credits				64	
\end{flushleft}





\begin{flushleft}
Course Category	
\end{flushleft}


\begin{flushleft}
Credits
\end{flushleft}


\begin{flushleft}
Institute Core Courses
\end{flushleft}


\begin{flushleft}
Basic Sciences (BS)		 22
\end{flushleft}


\begin{flushleft}
Engineering Arts and Science (EAS)		 18
\end{flushleft}


\begin{flushleft}
Humanities and Social Sciences (HuSS)		 15
\end{flushleft}


\begin{flushleft}
Programme-linked Courses		11
\end{flushleft}


\begin{flushleft}
Departmental Courses
\end{flushleft}


\begin{flushleft}
Departmental Core 		 64
\end{flushleft}


\begin{flushleft}
Departmental Electives		 12
\end{flushleft}


\begin{flushleft}
Open Category Courses		 10
\end{flushleft}


\begin{flushleft}
Total Graded Credit requirement		 152
\end{flushleft}


\begin{flushleft}
Non Graded Units		 15
\end{flushleft}





\begin{flushleft}
Departmental Electives
\end{flushleft}





\begin{flushleft}
MCD310	Mini Project	
\end{flushleft}


0	0	


\begin{flushleft}
MCL314	 Acoustics and Noise Control	
\end{flushleft}


3	 0	


\begin{flushleft}
MCL321	Automotive Systems	
\end{flushleft}


3	0	


\begin{flushleft}
Institute Core : Basic Sciences
\end{flushleft}


\begin{flushleft}
MCL322	Power Train Design	
\end{flushleft}


3	0	


\begin{flushleft}
CML100	 General Chemistry	
\end{flushleft}


3	 0	 0	 3


\begin{flushleft}
MCL330	 Special Topics Production Engineering	
\end{flushleft}


3	 0	


\begin{flushleft}
CMP100	Chemistry Laboratory	
\end{flushleft}


0	0	 4	2


\begin{flushleft}
MCL334	Industrial Automation	
\end{flushleft}


3	0	


\begin{flushleft}
MTL100	 Calculus	
\end{flushleft}


3	1	 0	4


\begin{flushleft}
MCL336	 Advances in Welding	
\end{flushleft}


3	 0	


\begin{flushleft}
MTL101	 Linear Algebra and Differential Equations	
\end{flushleft}


3	 1	 0	 4


\begin{flushleft}
MCL337	 Advanced Machining Processes	
\end{flushleft}


3	 0	


\begin{flushleft}
PYL100	 Electromagnetic Waves and	
\end{flushleft}


3	 0	 0	 3


\begin{flushleft}
MCL338	 Mechatronic Applications in Manufacturing	 3	 0	
\end{flushleft}


	


\begin{flushleft}
Quantum Mechanics
\end{flushleft}


\begin{flushleft}
MCL341	 Gas Dynamics and Propulsion	
\end{flushleft}


3	 0	


\begin{flushleft}
PYP100	 Physics Laboratory	
\end{flushleft}


0	0	 4	2


\begin{flushleft}
MCL343	 Introduction to Combustion	
\end{flushleft}


3	 0	


\begin{flushleft}
SBL100	 Introductory Biology for Engineers	
\end{flushleft}


3	 0	 2	 4


\begin{flushleft}
MCL344	Refrigeration and Air-conditioning	
\end{flushleft}


3	0	


	


\begin{flushleft}
Total Credits				22	 MCL345	 Reciprocating Internal Combustion Engines	 3	 0	
\end{flushleft}


\begin{flushleft}
MCL347	Intermediate Heat Transfer	
\end{flushleft}


3	0	


\begin{flushleft}
Institute Core: Engineering Arts and Sciences
\end{flushleft}


\begin{flushleft}
MCL348	 Thermal Management of Electronics	
\end{flushleft}


3	 0	


\begin{flushleft}
APL100	 Engineering Mechanics	
\end{flushleft}


3	1	 0	4


\begin{flushleft}
MCL350	 Mechanical Engineering Product Synthesis	 1	 0	
\end{flushleft}


\begin{flushleft}
COL100	 Introduction to Computer Science	
\end{flushleft}


3	 0	 2	 4


\begin{flushleft}
MCL363	Investment Planning	
\end{flushleft}


3	0	


\begin{flushleft}
CVL100	 Environmental Science	
\end{flushleft}


2	0	 0	2


\begin{flushleft}
MCL364	Value Engineering	
\end{flushleft}


3	0	


\begin{flushleft}
ELL100	 Introduction to Electrical Engineering	
\end{flushleft}


3	 0	 2	 4


\begin{flushleft}
MCL366	 OR Methods in Policy Governance	
\end{flushleft}


3	 0	


\begin{flushleft}
MCP100	Engineering Visualization	
\end{flushleft}


0	0	 4	2


\begin{flushleft}
MCL368	 Quality and Reliability Engineering	
\end{flushleft}


3	 0	


\begin{flushleft}
MCP101	 Product Realization through Manufacturing	 0	 0	 4	 2
\end{flushleft}


\begin{flushleft}
MCL370	 Special Topics in Industrial Engineering	
\end{flushleft}


3	 0	


	


\begin{flushleft}
Total Credits				18	
\end{flushleft}


\begin{flushleft}
MCL380	 Special Topics in Mechanical Engineering	
\end{flushleft}


3	 0	


\begin{flushleft}
MCV390	 Special module in Mechanical Engineering	 1	 0	
\end{flushleft}


\begin{flushleft}
Programme-Linked Basic / Engineering Arts / Sciences Core
\end{flushleft}


\begin{flushleft}
MCD412	 B.Tech. Project-II	
\end{flushleft}


0	 0	


\begin{flushleft}
APL102	 Introduction to Materials Science	
\end{flushleft}


3	 0	 2	 4


\begin{flushleft}
MCL421	 Automotive Structural Design	
\end{flushleft}


2	 0	


	


\begin{flushleft}
and Engineering
\end{flushleft}


\begin{flushleft}
MCL422	 Design of Brake Systems	
\end{flushleft}


2	 0	


\begin{flushleft}
MTL107	 Numerical Methods and Computations	
\end{flushleft}


3	 0	 0	 3


\begin{flushleft}
MCL441	 Modelling and Experiments in Heat Transfer	 2	 0	
\end{flushleft}


\begin{flushleft}
MTL108	 Introduction to Statistics	
\end{flushleft}


3	 1	 0	 4


\begin{flushleft}
MCL442	 ThermoFluid Analysis of Biosystems	
\end{flushleft}


3	 0	


	


\begin{flushleft}
Total Credits				11
\end{flushleft}


\begin{flushleft}
MCL443	 Electrochemical Energy Systems	
\end{flushleft}


3	 0	


\begin{flushleft}
MCL721	 Automotive Prime Movers	
\end{flushleft}


3	 0	


\begin{flushleft}
Humanities and Social Sciences
\end{flushleft}


\begin{flushleft}
MCL722	 Mechanical Design of Prime Mover Elements	3	0	
\end{flushleft}


\begin{flushleft}
MCL723	Vehicle Dynamics	
\end{flushleft}


3	0	


\begin{flushleft}
Courses from Humanities, Social Sciences and Management
\end{flushleft}


\begin{flushleft}
MCL724	 Biomechanics of Trauma in Automotive Design	 3	0	
\end{flushleft}


\begin{flushleft}
offered under this category			
\end{flushleft}


	 15


\begin{flushleft}
MCL725	 Design Electronic Assist Systems in Automobiles	 3	0	
\end{flushleft}


\begin{flushleft}
Departmental Core
\end{flushleft}


\begin{flushleft}
MCL726	 Design of Steering Systems	
\end{flushleft}


3	 0	


\begin{flushleft}
MCL729	Nanomechanics	
\end{flushleft}


2	0	


\begin{flushleft}
APL 104	 Solid Mechanics	
\end{flushleft}


3	 1	 0	 4


\begin{flushleft}
MCL747	 Design of Precision Machines	
\end{flushleft}


2	 0	


\begin{flushleft}
APL 106	 Fluid Mechanics	
\end{flushleft}


3	 1	 0	 4


\begin{flushleft}
MCL749	 Mechatronics Product Design	
\end{flushleft}


3	 0	


\begin{flushleft}
MCL111	 Kinematics and Dynamics of Machines 	
\end{flushleft}


3	 0	 2	 4


\begin{flushleft}
MCL750	 Product Design and Manufacturing	
\end{flushleft}


1	 0	


\begin{flushleft}
MCL131	 Manufacturing Processes-I 	
\end{flushleft}


3	 0	 0	 3


\begin{flushleft}
MCL753	 Manufacturing Informatics 	
\end{flushleft}


3	 0	


\begin{flushleft}
MCL140	Engineering Thermodynamics 	
\end{flushleft}


3	1	 0	4


\begin{flushleft}
MCL755	 Service System Design 	
\end{flushleft}


3	 0	


\begin{flushleft}
MCL201	 Mechanical Engineering Drawing 	
\end{flushleft}


2	 0	 3	 3.5


\begin{flushleft}
MCL756	 Supply Chain Management 	
\end{flushleft}


3	 0	


\begin{flushleft}
MCL211	 Design of Machines 	
\end{flushleft}


3	 0	 2	 4


\begin{flushleft}
MCL759	Entrepreneurship 	
\end{flushleft}


3	0	


\begin{flushleft}
MCL212	 Control Theory and Applications 	
\end{flushleft}


3	 0	 2	 4


\begin{flushleft}
MCL760	 Project Management 	
\end{flushleft}


3	 0	


\begin{flushleft}
MCL231	 Manufacturing Processes-II 	
\end{flushleft}


3	 0	 0	 3


\begin{flushleft}
MCL776	 Advances in Metal Forming	
\end{flushleft}


3	 0	


\begin{flushleft}
MCP231	 Manufacturing Laboratory-I 	
\end{flushleft}


0	 0	 2	 1


\begin{flushleft}
MCL777	Machine Tool Design	
\end{flushleft}


3	0	


\begin{flushleft}
MCL241	 Energy systems and Technologies 	
\end{flushleft}


3	 0.5	1	 4


\begin{flushleft}
MCL242	 Heat and Mass Transfer 	
\end{flushleft}


3	 1	 0	 4


\begin{flushleft}
MCL788	Surface Engineering	
\end{flushleft}


3	0	





57





6	3


2	 4


2	4


0	3


0	 3


2	4


2	 4


0	 3


2	 4


2	 4


0	 3


2	4


2	 4


0	3


0	 3


2	 2


0	3


2	4


0	 3


0	 3


0	 3


0	 3


0	 1


14	7


2	 3


2	 3


4	 4


0	 3


0	 3


0	 3


0	3


0	3


0	3


0	3


0	 3


2	3


2	 3


2	 4


4	 3


2	 4


0	 3


0	 3


0	3


0	 3


0	 3


2	4


2	4





\newpage
58





\begin{flushleft}
Semester
\end{flushleft}





\begin{flushleft}
VIII
\end{flushleft}





\begin{flushleft}
VII
\end{flushleft}





\begin{flushleft}
VI
\end{flushleft}





\begin{flushleft}
V
\end{flushleft}





\begin{flushleft}
IV
\end{flushleft}





\begin{flushleft}
III
\end{flushleft}





\begin{flushleft}
II
\end{flushleft}





\begin{flushleft}
I
\end{flushleft}





4





0


2


\begin{flushleft}
SBL100
\end{flushleft}





4





0


2


\begin{flushleft}
MCL261
\end{flushleft}





4





0


0


\begin{flushleft}
MCL361
\end{flushleft}





3





0





3





0





0


2


\begin{flushleft}
DE 3 (3)
\end{flushleft}





\begin{flushleft}
CAM and Automation
\end{flushleft}





0


0


\begin{flushleft}
MCL431
\end{flushleft}





2





3





3





3





3





\begin{flushleft}
Manufacturing System Design
\end{flushleft}





3





\begin{flushleft}
Introduction to Operations
\end{flushleft}


\begin{flushleft}
Research
\end{flushleft}





3





\begin{flushleft}
Introductory Biology for
\end{flushleft}


\begin{flushleft}
Engineers
\end{flushleft}





3





0


0


\begin{flushleft}
CML100
\end{flushleft}





3





\begin{flushleft}
Introduction to Chemistry
\end{flushleft}





3





1


0


\begin{flushleft}
MTL101
\end{flushleft}





\begin{flushleft}
Calculus
\end{flushleft}





4





\begin{flushleft}
Linear Algebra and Differential
\end{flushleft}


\begin{flushleft}
Equations
\end{flushleft}





3





\begin{flushleft}
Course-4
\end{flushleft}





\begin{flushleft}
MTL100
\end{flushleft}





\begin{flushleft}
Course-5
\end{flushleft}





0 4


\begin{flushleft}
CMP100
\end{flushleft}





2





\begin{flushleft}
Chemistry Laboratory
\end{flushleft}





0





\begin{flushleft}
Physics Laboratory
\end{flushleft}





\begin{flushleft}
PYP100
\end{flushleft}





\begin{flushleft}
Course-6
\end{flushleft}





0





0





4





2





\begin{flushleft}
Product Realization
\end{flushleft}


\begin{flushleft}
through Manufacturing
\end{flushleft}





\begin{flushleft}
MCP101
\end{flushleft}





0





0





2





1





\begin{flushleft}
Language and
\end{flushleft}


\begin{flushleft}
Writing Skills-1
\end{flushleft}


\begin{flushleft}
(Non-Graded)
\end{flushleft}





\begin{flushleft}
NLN100
\end{flushleft}





\begin{flushleft}
Language and
\end{flushleft}


\begin{flushleft}
Writing Skills-2
\end{flushleft}


\begin{flushleft}
(Non-Graded)
\end{flushleft}





0 1 0.5 0 0 2 1


\begin{flushleft}
NEN100
\end{flushleft}


\begin{flushleft}
NLN100
\end{flushleft}





\begin{flushleft}
Professional Ethics and
\end{flushleft}


\begin{flushleft}
Social Responsibility-2
\end{flushleft}


\begin{flushleft}
(Non-graded)
\end{flushleft}





0





\begin{flushleft}
Professional Ethics and
\end{flushleft}


\begin{flushleft}
Social Responsibility-1
\end{flushleft}


\begin{flushleft}
(Non-graded)
\end{flushleft}





\begin{flushleft}
Introduction to
\end{flushleft}


\begin{flushleft}
Engineering
\end{flushleft}


\begin{flushleft}
(Non-graded)
\end{flushleft}





\begin{flushleft}
Course-7
\end{flushleft}





\begin{flushleft}
NEN100
\end{flushleft}





\begin{flushleft}
Course-8
\end{flushleft}





\begin{flushleft}
NIN100
\end{flushleft}





\begin{flushleft}
Course-9
\end{flushleft}





9.5





\begin{flushleft}
L
\end{flushleft}





1 0


\begin{flushleft}
MCL131
\end{flushleft}





4





0 0


\begin{flushleft}
MCL231
\end{flushleft}





3





3





0





3





3





0





0





0 8


\begin{flushleft}
OC 2 (3)
\end{flushleft}





\begin{flushleft}
B.Tech.Project
\end{flushleft}





0 2


\begin{flushleft}
MCD411
\end{flushleft}





\begin{flushleft}
Control theory and
\end{flushleft}


\begin{flushleft}
applications
\end{flushleft}





0 0


\begin{flushleft}
MCL212
\end{flushleft}





3





4





4





3





\begin{flushleft}
Manufacturing Processes-II
\end{flushleft}





3





\begin{flushleft}
Manufacturing Processes-I
\end{flushleft}





3





0.5 1


\begin{flushleft}
MCL242
\end{flushleft}





\begin{flushleft}
Energy systems and
\end{flushleft}


\begin{flushleft}
Technologies
\end{flushleft}





1


0


\begin{flushleft}
MCL241
\end{flushleft}


4





4





1


0


\begin{flushleft}
MCL311
\end{flushleft}





4





3





3





3





1





0





0


0


\begin{flushleft}
OC 3 (4)
\end{flushleft}





0


2


\begin{flushleft}
DE 1 (3)
\end{flushleft}





4





3





4





\begin{flushleft}
CAD and Finite Element
\end{flushleft}


\begin{flushleft}
Analysis
\end{flushleft}





3





\begin{flushleft}
Heat and Mass Transfer
\end{flushleft}





3





3





\begin{flushleft}
Solid Mechanics
\end{flushleft}





\begin{flushleft}
APL 104
\end{flushleft}





1


0


\begin{flushleft}
MTL108
\end{flushleft}





4





1


0


\begin{flushleft}
MTL107
\end{flushleft}





4





0


0


\begin{flushleft}
MCP301
\end{flushleft}





4





3





3





0





0





0





0


0


\begin{flushleft}
DE 4 (3)
\end{flushleft}





0


3


\begin{flushleft}
OC 1 (3)
\end{flushleft}





3





3





1.5





\begin{flushleft}
Mechanical Engineering Lab I
\end{flushleft}





3





\begin{flushleft}
Numerical Methods and
\end{flushleft}


\begin{flushleft}
Computations
\end{flushleft}





3





\begin{flushleft}
Introduction to Statistics
\end{flushleft}





3





\begin{flushleft}
Engineering Thermodynamics
\end{flushleft}





\begin{flushleft}
MCL140
\end{flushleft}





0 2


\begin{flushleft}
MCL201
\end{flushleft}





4





0 3 3.5


\begin{flushleft}
MCL211
\end{flushleft}


0 2


\begin{flushleft}
MCP331
\end{flushleft}





4





1 0


\begin{flushleft}
MCP231
\end{flushleft}





4





0





0 2


\begin{flushleft}
CVL100
\end{flushleft}





1





\begin{flushleft}
Manufacturing LaboratoryI
\end{flushleft}





3





\begin{flushleft}
HUL2XX
\end{flushleft}





0 2


\begin{flushleft}
MCP401
\end{flushleft}





1





3





0





0





0





0 4


\begin{flushleft}
HUL3XX
\end{flushleft}





3





2





\begin{flushleft}
Mechanical Engineering
\end{flushleft}


\begin{flushleft}
Lab-II
\end{flushleft}





0





3





2





0





0





0 0


\begin{flushleft}
DE 2 (3)
\end{flushleft}





3





2





\begin{flushleft}
Manufacturing LaboratoryEnvironmental Science
\end{flushleft}


\begin{flushleft}
II
\end{flushleft}





3





\begin{flushleft}
Design of Machines
\end{flushleft}





2





\begin{flushleft}
Mechanical Engineering
\end{flushleft}


\begin{flushleft}
Drawing
\end{flushleft}





3





\begin{flushleft}
Kinematics and Dynamics
\end{flushleft}


\begin{flushleft}
of Machines
\end{flushleft}





\begin{flushleft}
MCL111
\end{flushleft}





3





3





0





2





1





1





0





4





1 0 4


\begin{flushleft}
HUL2XX
\end{flushleft}





\begin{flushleft}
HUL2XX
\end{flushleft}





0





\begin{flushleft}
Intro. to Mechanical
\end{flushleft}


\begin{flushleft}
Engg.
\end{flushleft}


\begin{flushleft}
(Non-graded)
\end{flushleft}





\begin{flushleft}
MCN101
\end{flushleft}





\begin{flushleft}
P
\end{flushleft}





\begin{flushleft}
Credits
\end{flushleft}





2





9 19.5 4 24.0





4 22.0 4 24.0





6 22.5 0 25.5





15.0 1





0





0





16.0 152.0


\begin{flushleft}
TOTAL=152.0
\end{flushleft}


16.0





11.0 0 14 18.0 0 25.0





14.0 1





18.0 2





17.0 3





4 20.0 1 24.0





6 17.0 1.5 23.0





1 13 17.0 2.5 28.5





\begin{flushleft}
T
\end{flushleft}





15.0 3





3 0 2 4 3 0 0 3 3 1 0 4 0 0 4 2


0 0 1 0.5 0 0 2 1 12


\begin{flushleft}
Note: Courses 1-6 above are attended in the given order by half of all first year students. The other half of First year students attend the Courses 1-6 of II semester first.
\end{flushleft}





\begin{flushleft}
Fluid Mechanics
\end{flushleft}





4





\begin{flushleft}
Introduction to Materials
\end{flushleft}


\begin{flushleft}
Science and Engineering
\end{flushleft}





0





\begin{flushleft}
APL 106
\end{flushleft}





1





2





\begin{flushleft}
Introduction to Computer
\end{flushleft}


\begin{flushleft}
Science
\end{flushleft}





0 3


\begin{flushleft}
COL100
\end{flushleft}





\begin{flushleft}
APL102
\end{flushleft}





3





\begin{flushleft}
Engineering Mechanics
\end{flushleft}





0


2


\begin{flushleft}
APL100
\end{flushleft}





0.5





\begin{flushleft}
PYL100
\end{flushleft}





3





\begin{flushleft}
Course-1
\end{flushleft}





\begin{flushleft}
Introduction to Engineering Electromagnetic Waves and
\end{flushleft}


\begin{flushleft}
Visualization
\end{flushleft}


\begin{flushleft}
Quantum Mechanics
\end{flushleft}





\begin{flushleft}
MCP100
\end{flushleft}





\begin{flushleft}
Course-3
\end{flushleft}





\begin{flushleft}
Introduction to Electrical
\end{flushleft}


\begin{flushleft}
Engineering
\end{flushleft}





\begin{flushleft}
Course-2
\end{flushleft}





\begin{flushleft}
ELL100
\end{flushleft}





\begin{flushleft}
Non-Graded Units
\end{flushleft}





\begin{flushleft}
B. Tech. in Mechanical Engineering	ME1
\end{flushleft}


\begin{flushleft}
Contact Hours
\end{flushleft}





\begin{flushleft}
\newpage
Programme Code: ME2
\end{flushleft}





\begin{flushleft}
Bachelor of Technology in Production and Industrial Engineering
\end{flushleft}


\begin{flushleft}
Department of Mechanical Engineering
\end{flushleft}


\begin{flushleft}
The overall Credit Structure
\end{flushleft}





\begin{flushleft}
MCL262	 Stochastic Modelling and Simulation	
\end{flushleft}


3	 0	 0	 3


\begin{flushleft}
MCL311	 CAD and Finite Element Analysis 	
\end{flushleft}


3	 0	 2	 4


\begin{flushleft}
MCL331	 Micro and Nano Manufacturing 	
\end{flushleft}


3	 0	 0	 3


\begin{flushleft}
MCP332	 Production Engineering Laboratory-II 	
\end{flushleft}


0	 0	 2	 1


\begin{flushleft}
MCL361	 Manufacturing System Design 	
\end{flushleft}


3	 0	 0	 3


\begin{flushleft}
MCP361	 Industrial Engineering Laboratory-II	
\end{flushleft}


0	 0	 2	 1


\begin{flushleft}
MCD411	B.Tech. Project	
\end{flushleft}


0	0	 8	4


\begin{flushleft}
MCL431	 CAM and Automation 	
\end{flushleft}


2	 0	 2	 3


	


\begin{flushleft}
Total Credits				66
\end{flushleft}





\begin{flushleft}
Course Category	
\end{flushleft}


\begin{flushleft}
Credits
\end{flushleft}


\begin{flushleft}
Institute Core Courses
\end{flushleft}


\begin{flushleft}
Basic Sciences (BS)		 22
\end{flushleft}


\begin{flushleft}
Engineering Arts and Science (EAS)		 18
\end{flushleft}


\begin{flushleft}
Humanities and Social Sciences (HuSS)		 15
\end{flushleft}


\begin{flushleft}
Programme-linked Courses		11
\end{flushleft}


\begin{flushleft}
Departmental Courses
\end{flushleft}


\begin{flushleft}
Departmental Core 		 66
\end{flushleft}


\begin{flushleft}
Departmental Electives		 12
\end{flushleft}


\begin{flushleft}
Open Category Courses		 10
\end{flushleft}


\begin{flushleft}
Total Graded Credit requirement		 154
\end{flushleft}


\begin{flushleft}
Non Graded Units		 15
\end{flushleft}





\begin{flushleft}
Departmental Electives
\end{flushleft}


\begin{flushleft}
MCD310	Mini Project	
\end{flushleft}


0	0	


\begin{flushleft}
MCL314	 Acoustics and Noise Control	
\end{flushleft}


3	 0	


\begin{flushleft}
MCL321	Automotive Systems	
\end{flushleft}


3	0	


\begin{flushleft}
MCL322	Power Train Design	
\end{flushleft}


3	0	


\begin{flushleft}
MCL330	 Special Topics Production Engg	
\end{flushleft}


3	 0	


\begin{flushleft}
MCL334	Industrial Automation	
\end{flushleft}


3	0	


\begin{flushleft}
MCL336	 Advances in Welding	
\end{flushleft}


3	 0	


\begin{flushleft}
MCL337	 Advanced Machining Processes	
\end{flushleft}


3	 0	


\begin{flushleft}
MCL338	 Mechatronic Applications in Manufacturing	 3	 0	
\end{flushleft}


\begin{flushleft}
MCL341	 Gas Dynamics and Propulsion	
\end{flushleft}


3	 0	


\begin{flushleft}
MCL343	 Introduction to Combustion	
\end{flushleft}


3	 0	


\begin{flushleft}
MCL344	Refrigeration and Air-conditioning	
\end{flushleft}


3	0	


\begin{flushleft}
MCL345	 Reciprocating Internal Combustion Engines	 3	 0	
\end{flushleft}


\begin{flushleft}
MCL347	Intermediate Heat Transfer	
\end{flushleft}


3	0	


\begin{flushleft}
MCL348	 Thermal Management of Electronics	
\end{flushleft}


3	 0	


\begin{flushleft}
MCL350	 Mechanical Engineering Product Synthesis	 1	 0	
\end{flushleft}


\begin{flushleft}
MCL363	Investment Planning	
\end{flushleft}


3	0	


\begin{flushleft}
MCL364	Value Engineering	
\end{flushleft}


3	0	


\begin{flushleft}
MCL366	 OR Methods in Policy Governance	
\end{flushleft}


3	 0	


\begin{flushleft}
MCL368	 Quality and Reliability Engineering	
\end{flushleft}


3	 0	


\begin{flushleft}
MCL370	 Special Topics in Industrial Engg	
\end{flushleft}


3	 0	


\begin{flushleft}
MCL380	 Special Topics in Mechanical Engineering	
\end{flushleft}


3	 0	


\begin{flushleft}
MCV390	 Special module in Mechanical Engineering	 1	 0	
\end{flushleft}


\begin{flushleft}
MCD412	 B.Tech. Project-II	
\end{flushleft}


0	 0	


\begin{flushleft}
MCL421	 Automotive Structural Design	
\end{flushleft}


2	 0	


\begin{flushleft}
MCL422	 Design of Brake Systems	
\end{flushleft}


2	 0	


\begin{flushleft}
MCL441	 Modelling and Experiments in Heat Transfer	 2	 0	
\end{flushleft}


\begin{flushleft}
MCL442	 ThermoFluid Analysis of Biosystems	
\end{flushleft}


3	 0	


\begin{flushleft}
MCL443	 Electrochemical Energy Systems	
\end{flushleft}


3	 0	


\begin{flushleft}
MCL721	 Automotive Prime Movers	
\end{flushleft}


3	 0	


\begin{flushleft}
MCL722	 Mechanical Design of Prime Mover Elements	 3	0	
\end{flushleft}


\begin{flushleft}
MCL723	Vehicle Dynamics	
\end{flushleft}


3	0	


\begin{flushleft}
MCL724	 Biomechanics of Trauma in Automotive Design	 3	0	
\end{flushleft}


\begin{flushleft}
MCL725	 Design Electronic Assist Systems in Automobiles	 3	0	
\end{flushleft}


\begin{flushleft}
MCL726	 Design of Steering Systems	
\end{flushleft}


3	 0	


\begin{flushleft}
MCL729	Nanomechanics	
\end{flushleft}


2	0	


\begin{flushleft}
MCL747	 Design of Precision Machines	
\end{flushleft}


2	 0	


\begin{flushleft}
MCL749	 Mechatronics Product Design	
\end{flushleft}


3	 0	


\begin{flushleft}
MCL750	 Product Design and Manufacturing	
\end{flushleft}


1	 0	


\begin{flushleft}
MCL753	 Manufacturing Informatics 	
\end{flushleft}


3	 0	


\begin{flushleft}
MCL755	 Service System Design 	
\end{flushleft}


3	 0	


\begin{flushleft}
MCL756	 Supply Chain Management 	
\end{flushleft}


3	 0	


\begin{flushleft}
MCL759	Entrepreneurship 	
\end{flushleft}


3	0	


\begin{flushleft}
MCL760	 Project Management 	
\end{flushleft}


3	 0	


\begin{flushleft}
MCL776	 Advances in Metal Forming	
\end{flushleft}


3	 0	


\begin{flushleft}
MCL777	Machine Tool Design	
\end{flushleft}


3	0	


\begin{flushleft}
MCL788	Surface Engineering	
\end{flushleft}


3	0	





\begin{flushleft}
Institute Core : Basic Sciences
\end{flushleft}


\begin{flushleft}
CML100	 General Chemistry	
\end{flushleft}


3	 0	 0	 3


\begin{flushleft}
CMP100	Chemistry Laboratory	
\end{flushleft}


0	0	 4	2


\begin{flushleft}
MTL100	 Calculus	
\end{flushleft}


3	1	 0	4


\begin{flushleft}
MTL101	 Linear Algebra and Differential Equations	
\end{flushleft}


3	 1	 0	 4


\begin{flushleft}
PYL100	 Electromagnetic Waves and	
\end{flushleft}


3	 0	 0	 3


	


\begin{flushleft}
Quantum Mechanics
\end{flushleft}


\begin{flushleft}
PYP100	 Physics Laboratory	
\end{flushleft}


0	0	 4	2


\begin{flushleft}
SBL100	 Introductory Biology for Engineers	
\end{flushleft}


3	 0	 2	 4


	


\begin{flushleft}
Total Credits				22
\end{flushleft}


\begin{flushleft}
Institute Core: Engineering Arts and Sciences
\end{flushleft}


\begin{flushleft}
APL100	 Engineering Mechanics	
\end{flushleft}


3	1	 0	4


\begin{flushleft}
COL100	 Introduction to Computer Science	
\end{flushleft}


3	 0	 2	 4


\begin{flushleft}
CVL100	 Environmental Science	
\end{flushleft}


2	0	 0	2


\begin{flushleft}
ELL100	 Introduction to Electrical Engineering	
\end{flushleft}


3	 0	 2	 4


\begin{flushleft}
MCP100	 Introduction to Engineering Visualization	
\end{flushleft}


0	 0	 4	 2


\begin{flushleft}
MCP101	 Product Realization through Manufacturing	 0	 0	 4	 2
\end{flushleft}


	


\begin{flushleft}
Total Credits				18
\end{flushleft}


\begin{flushleft}
Programme-Linked Basic / Engineering Arts / Sciences Core
\end{flushleft}


\begin{flushleft}
APL102	
\end{flushleft}


\begin{flushleft}
MTL107	
\end{flushleft}


\begin{flushleft}
MTL108	
\end{flushleft}


	





\begin{flushleft}
Introduction to Materials Science and Engineering	3	0	 2	4
\end{flushleft}


\begin{flushleft}
Numerical Methods and Computations	
\end{flushleft}


3	 0	 0	 3


\begin{flushleft}
Introduction to Statistics	
\end{flushleft}


3	 1	 0	 4


\begin{flushleft}
Total Credits				11
\end{flushleft}





\begin{flushleft}
Humanities and Social Sciences
\end{flushleft}


\begin{flushleft}
Courses from Humanities, Social Sciences and Management
\end{flushleft}


\begin{flushleft}
offered under this category			
\end{flushleft}


	 15


\begin{flushleft}
Departmental Core
\end{flushleft}


\begin{flushleft}
APL 104	
\end{flushleft}


\begin{flushleft}
MCL111	
\end{flushleft}


\begin{flushleft}
MCL132	
\end{flushleft}


\begin{flushleft}
MCL133	
\end{flushleft}


\begin{flushleft}
MCL134	
\end{flushleft}


\begin{flushleft}
MCL135	
\end{flushleft}


\begin{flushleft}
MCL136	
\end{flushleft}


\begin{flushleft}
MCL141	
\end{flushleft}


\begin{flushleft}
MCL201	
\end{flushleft}


\begin{flushleft}
MCL211	
\end{flushleft}


\begin{flushleft}
MCL212	
\end{flushleft}


\begin{flushleft}
MCP232	
\end{flushleft}


\begin{flushleft}
MCL261	
\end{flushleft}


\begin{flushleft}
MCP261	
\end{flushleft}





\begin{flushleft}
Solid Mechanics	
\end{flushleft}


\begin{flushleft}
Kinematics and Dynamics of Machines 	
\end{flushleft}


\begin{flushleft}
Metal Forming and Press Tools	
\end{flushleft}


\begin{flushleft}
Near Net Shape Manufacturing	
\end{flushleft}


\begin{flushleft}
Metrology and Quality Assurance	
\end{flushleft}


\begin{flushleft}
Welding and Allied Processes	
\end{flushleft}


\begin{flushleft}
Material Removal Processes	
\end{flushleft}


\begin{flushleft}
Thermal Science for Manufacturing 	
\end{flushleft}


\begin{flushleft}
Mechanical Engineering Drawing 	
\end{flushleft}


\begin{flushleft}
Design of Machines 	
\end{flushleft}


\begin{flushleft}
Control Theory and Applications 	
\end{flushleft}


\begin{flushleft}
Production Engineering Laboratory-I	
\end{flushleft}


\begin{flushleft}
Introduction to Operations Research 	
\end{flushleft}


\begin{flushleft}
Industrial Engineering Laboratory-I	
\end{flushleft}





3	


3	


3	


3	


3	


3	


3	


3	


2	


3	


3	


0	


3	


0	





1	


0	


0	


0	


0	


0	


0	


1	


0	


0	


0	


0	


0	


0	





0	


2	


0	


0	


1	


0	


0	


0	


3	


2	


2	


2	


0	


2	





4


4


3


3


3.5


3


3


4


3.5


4


4


1


3


1





59





6	3


2	 4


2	4


0	3


0	 3


2	4


2	 4


0	 3


2	 4


2	 4


0	 3


2	4


2	 4


0	3


0	 3


2	 2


0	3


2	4


0	 3


0	 3


0	 3


0	 3


0	 1


14	7


2	 3


2	 3


4	 4


0	 3


0	 3


0	 3


0	3


0	3


0	3


0	3


0	 3


2	3


2	 3


2	 4


4	 3


2	 4


0	 3


0	 3


0	3


0	 3


0	 3


2	4


2	4





\newpage
60





\begin{flushleft}
Semester
\end{flushleft}





\begin{flushleft}
VIII
\end{flushleft}





\begin{flushleft}
VII
\end{flushleft}





\begin{flushleft}
VI
\end{flushleft}





\begin{flushleft}
V
\end{flushleft}





\begin{flushleft}
IV
\end{flushleft}





\begin{flushleft}
III
\end{flushleft}





\begin{flushleft}
II
\end{flushleft}





\begin{flushleft}
I
\end{flushleft}





2





0





0





3





4





3





1


0


\begin{flushleft}
MCL133
\end{flushleft}





4





3





0 0


\begin{flushleft}
MCL431
\end{flushleft}





3





3





2





0





0





0 2


\begin{flushleft}
DE 2 (3)
\end{flushleft}





3





3





\begin{flushleft}
CAM and Automation
\end{flushleft}





3





\begin{flushleft}
Manufacturing System
\end{flushleft}


\begin{flushleft}
Design
\end{flushleft}





0 0


\begin{flushleft}
MCL361
\end{flushleft}





3





0





3





0





0





0 8


\begin{flushleft}
OC 2 (3)
\end{flushleft}





\begin{flushleft}
B.Tech.Project
\end{flushleft}





0 2


\begin{flushleft}
MCD411
\end{flushleft}





\begin{flushleft}
Control theory and
\end{flushleft}


\begin{flushleft}
applications
\end{flushleft}





3





4





4





0 1 3.5


\begin{flushleft}
MCL212
\end{flushleft}





3





3





3





0 0


\begin{flushleft}
MCL134
\end{flushleft}





\begin{flushleft}
Metrology and Quality
\end{flushleft}


\begin{flushleft}
Assurance
\end{flushleft}





4





\begin{flushleft}
Introduction to Operations
\end{flushleft}


\begin{flushleft}
Research
\end{flushleft}





0 2


\begin{flushleft}
MCL261
\end{flushleft}





0


0


\begin{flushleft}
MCL135
\end{flushleft}





3





0


0


\begin{flushleft}
MCL311
\end{flushleft}





3





3





0





3





1





0





0


2


\begin{flushleft}
OC 3 (4)
\end{flushleft}





\begin{flushleft}
Industrial Engineering
\end{flushleft}


\begin{flushleft}
Laboratory-II
\end{flushleft}





0


2


\begin{flushleft}
MCP361
\end{flushleft}





4





1





4





\begin{flushleft}
CAD and Finite Element
\end{flushleft}


\begin{flushleft}
Analysis
\end{flushleft}





3





\begin{flushleft}
Welding and Allied Processes
\end{flushleft}





3





3





1 0


\begin{flushleft}
MCL132
\end{flushleft}





\begin{flushleft}
MCL141
\end{flushleft}





\begin{flushleft}
Thermal Science for
\end{flushleft}


\begin{flushleft}
Manufacturing
\end{flushleft}





3





4





3





\begin{flushleft}
Metal Forming and Press
\end{flushleft}


\begin{flushleft}
Near Net Shape Manufacturing
\end{flushleft}


\begin{flushleft}
Tools
\end{flushleft}





0 2


\begin{flushleft}
SBL100
\end{flushleft}





4





3





\begin{flushleft}
Introductory Biology for
\end{flushleft}


\begin{flushleft}
Engineers
\end{flushleft}





3





\begin{flushleft}
Solid Mechanics
\end{flushleft}





3





0





0


0


\begin{flushleft}
CML100
\end{flushleft}





\begin{flushleft}
Introduction to Chemistry
\end{flushleft}





3





3





3





\begin{flushleft}
Course-4
\end{flushleft}





1





0





\begin{flushleft}
Linear Algebra and
\end{flushleft}


\begin{flushleft}
Differential Equations
\end{flushleft}





1


0


\begin{flushleft}
MTL101
\end{flushleft}





\begin{flushleft}
Calculus
\end{flushleft}





\begin{flushleft}
MTL100
\end{flushleft}





4





4





0 4


\begin{flushleft}
CMP100
\end{flushleft}





\begin{flushleft}
Physics Laboratory
\end{flushleft}





2





0





0





4





2





\begin{flushleft}
Chemistry Laboratory
\end{flushleft}





0





\begin{flushleft}
Course-5
\end{flushleft}





\begin{flushleft}
PYP100
\end{flushleft}





\begin{flushleft}
Course-6
\end{flushleft}





0





0





4





2





\begin{flushleft}
Product Realization through
\end{flushleft}


\begin{flushleft}
Manufacturing
\end{flushleft}





\begin{flushleft}
MCP101
\end{flushleft}





\begin{flushleft}
Course-7
\end{flushleft}





0





0





2





1





\begin{flushleft}
Introduction to Engineering
\end{flushleft}


\begin{flushleft}
(Non-graded)
\end{flushleft}





\begin{flushleft}
NIN100
\end{flushleft}





0





0





1





0.5





\begin{flushleft}
Professional Ethics and
\end{flushleft}


\begin{flushleft}
Social Responsibility-2
\end{flushleft}


\begin{flushleft}
(Non-graded)
\end{flushleft}





0 1 0.5


\begin{flushleft}
NEN100
\end{flushleft}





0





0





0





2





\begin{flushleft}
Language and
\end{flushleft}


\begin{flushleft}
Writing Skills-2
\end{flushleft}


\begin{flushleft}
(Non-Graded)
\end{flushleft}





0 2


\begin{flushleft}
NLN100
\end{flushleft}





\begin{flushleft}
Language and
\end{flushleft}


\begin{flushleft}
Writing Skills-1
\end{flushleft}


\begin{flushleft}
(Non-Graded)
\end{flushleft}





\begin{flushleft}
Professional Ethics and
\end{flushleft}


\begin{flushleft}
Social Responsibility-1
\end{flushleft}


\begin{flushleft}
(Non-graded)
\end{flushleft}





0





\begin{flushleft}
NLN100
\end{flushleft}





\begin{flushleft}
Course-8
\end{flushleft}





\begin{flushleft}
NEN100
\end{flushleft}





\begin{flushleft}
Course-9
\end{flushleft}





1





1





0


2


\begin{flushleft}
MTL108
\end{flushleft}





4





1


0


\begin{flushleft}
MTL107
\end{flushleft}





4





0


0


\begin{flushleft}
MCP261
\end{flushleft}





3





3





0





0





0





0


0


\begin{flushleft}
DE 3 (3)
\end{flushleft}





0


2


\begin{flushleft}
OC 1 (3)
\end{flushleft}





\begin{flushleft}
Industrial Engineering
\end{flushleft}


\begin{flushleft}
Laboratory-I
\end{flushleft}





3





3





3





1





3





\begin{flushleft}
Numerical Methods and
\end{flushleft}


\begin{flushleft}
Computations
\end{flushleft}





3





\begin{flushleft}
Introduction to Statistics
\end{flushleft}





3





\begin{flushleft}
Kinematics and Dynamics of
\end{flushleft}


\begin{flushleft}
Machines
\end{flushleft}





\begin{flushleft}
MCL111
\end{flushleft}





1 0


\begin{flushleft}
MCL201
\end{flushleft}





4





0 2


\begin{flushleft}
MCP332
\end{flushleft}





\begin{flushleft}
Design of Machines
\end{flushleft}





4





0 3 3.5


\begin{flushleft}
MCL211
\end{flushleft}





3





3





0





0





0





1 0


\begin{flushleft}
HUL3XX
\end{flushleft}





0 2


\begin{flushleft}
HUL2XX
\end{flushleft}





3





4





1





\begin{flushleft}
Production Engineering
\end{flushleft}


\begin{flushleft}
Laboratory-II
\end{flushleft}





3





2





\begin{flushleft}
Mechanical Engineering
\end{flushleft}


\begin{flushleft}
Drawing
\end{flushleft}





3





\begin{flushleft}
HUL2XX
\end{flushleft}





1


0


\begin{flushleft}
MCP232
\end{flushleft}





0


2


\begin{flushleft}
MCL136
\end{flushleft}





1





4





3





3





3





0





0





0


0


\begin{flushleft}
DE 4 (3)
\end{flushleft}





0


0


\begin{flushleft}
DE 1 (3)
\end{flushleft}





3





3





3





\begin{flushleft}
Material Removal Processes
\end{flushleft}





0





\begin{flushleft}
Production Engineering
\end{flushleft}


\begin{flushleft}
Laboratory-I
\end{flushleft}





3





\begin{flushleft}
HUL2XX
\end{flushleft}





2





\begin{flushleft}
MCL262
\end{flushleft}





0





1





3





3





0





0





\begin{flushleft}
Micro and Nano
\end{flushleft}


\begin{flushleft}
Manufacturing
\end{flushleft}





0


0


\begin{flushleft}
MCL331
\end{flushleft}





3





3





\begin{flushleft}
Stochastic Modelling and
\end{flushleft}


\begin{flushleft}
Simulation
\end{flushleft}





0





\begin{flushleft}
Intro. to Prod. and Industrial
\end{flushleft}


\begin{flushleft}
Engg.
\end{flushleft}


\begin{flushleft}
(Nongraded)
\end{flushleft}





\begin{flushleft}
MCN111
\end{flushleft}





2





0





0





2





\begin{flushleft}
Environmental Science
\end{flushleft}





\begin{flushleft}
CVL100
\end{flushleft}





\begin{flushleft}
Note: Courses 1-6 above are attended in the given order by half of all first year students. The other half of First year students attend the Courses 1-6 of II semester first.
\end{flushleft}





\begin{flushleft}
APL 104
\end{flushleft}





4





\begin{flushleft}
APL102
\end{flushleft}





0





\begin{flushleft}
Introduction to Materials
\end{flushleft}


\begin{flushleft}
Science and Engineering
\end{flushleft}





1





3





2





3





0 3


\begin{flushleft}
COL100
\end{flushleft}





\begin{flushleft}
Introduction to Computer
\end{flushleft}


\begin{flushleft}
Science
\end{flushleft}





4





\begin{flushleft}
Engineering Mechanics
\end{flushleft}





0 2


\begin{flushleft}
APL100
\end{flushleft}





0.5





\begin{flushleft}
PYL100
\end{flushleft}





3





\begin{flushleft}
Course-1
\end{flushleft}





\begin{flushleft}
Introduction to Engineering Electromagnetic Waves and
\end{flushleft}


\begin{flushleft}
Visualization
\end{flushleft}


\begin{flushleft}
Quantum Mechanics
\end{flushleft}





\begin{flushleft}
MCP100
\end{flushleft}





\begin{flushleft}
Course-3
\end{flushleft}





\begin{flushleft}
Introduction to Electrical
\end{flushleft}


\begin{flushleft}
Engineering
\end{flushleft}





\begin{flushleft}
Course-2
\end{flushleft}





\begin{flushleft}
ELL100
\end{flushleft}





\begin{flushleft}
T
\end{flushleft}





\begin{flushleft}
P
\end{flushleft}





\begin{flushleft}
Credits
\end{flushleft}





2





8 21.0 5 25.0





5 20.5 3 23.0





5 21.5 0 24.0





4 20.0 1 24.0





6 17.0 2 23.0





18.0 1





\begin{flushleft}
TOTAL=154.0
\end{flushleft}





0 19.0 0 19.0 154.0





11.0 1 12 18.0 0 24.0





17.0 0





18.0 0





17.0 2





15.0 3





12





9.5 1 13 17.0 3 28.5





\begin{flushleft}
L
\end{flushleft}





\begin{flushleft}
Non-Graded Units
\end{flushleft}





\begin{flushleft}
B. Tech. in Production and Industrial Engineering	ME2
\end{flushleft}


\begin{flushleft}
Contact Hours
\end{flushleft}





\begin{flushleft}
\newpage
Programme Code: MT1
\end{flushleft}





\begin{flushleft}
Bachelor of Technology in Mathematics and Computing
\end{flushleft}


\begin{flushleft}
Department of Mathematics
\end{flushleft}


\begin{flushleft}
The overall Credit Structure
\end{flushleft}


\begin{flushleft}
Course Category	
\end{flushleft}


\begin{flushleft}
Credits
\end{flushleft}


\begin{flushleft}
Institute Core Courses
\end{flushleft}


\begin{flushleft}
Basic Sciences (BS)		 22
\end{flushleft}


\begin{flushleft}
Engineering Arts and Science (EAS)		 18
\end{flushleft}


\begin{flushleft}
Humanities and Social Sciences (HuSS)		 15
\end{flushleft}


\begin{flushleft}
Programme-linked Courses		12.5
\end{flushleft}


\begin{flushleft}
Departmental Courses
\end{flushleft}


\begin{flushleft}
Departmental Core 		63.5
\end{flushleft}


\begin{flushleft}
Departmental Electives		 12
\end{flushleft}


\begin{flushleft}
Open Category Courses		 10
\end{flushleft}


\begin{flushleft}
Total Graded Credit requirement		 153
\end{flushleft}


\begin{flushleft}
Non Graded Units		 15
\end{flushleft}


\begin{flushleft}
Institute Core : Basic Sciences
\end{flushleft}


\begin{flushleft}
CML100	
\end{flushleft}


\begin{flushleft}
CMP100	
\end{flushleft}


\begin{flushleft}
MTL100	
\end{flushleft}


\begin{flushleft}
MTL101	
\end{flushleft}


\begin{flushleft}
PYL100	
\end{flushleft}


	


\begin{flushleft}
PYP100	
\end{flushleft}


\begin{flushleft}
SBL100	
\end{flushleft}


	





\begin{flushleft}
General Chemistry	
\end{flushleft}


3	 0	 0	 3


\begin{flushleft}
Chemistry Laboratory	
\end{flushleft}


0	 0	 4	 2


\begin{flushleft}
Calculus	
\end{flushleft}


3	1	 0	 4


\begin{flushleft}
Linear Algebra and Differential Equations	
\end{flushleft}


3	 1	 0	 4


\begin{flushleft}
Electromagnetic Waves and 	
\end{flushleft}


3	 0	 0	 3


\begin{flushleft}
Quantum Mechanics
\end{flushleft}


\begin{flushleft}
Physics Laboratory	
\end{flushleft}


0	 0	 4	 2


\begin{flushleft}
Introductory Biology for Engineers	
\end{flushleft}


3	 0	 2	 4


\begin{flushleft}
Total Credits				 22	
\end{flushleft}





\begin{flushleft}
Institute Core: Engineering Arts and Sciences
\end{flushleft}


\begin{flushleft}
APL100	
\end{flushleft}


\begin{flushleft}
COL100	
\end{flushleft}


\begin{flushleft}
CVL100	
\end{flushleft}


\begin{flushleft}
ELL100	
\end{flushleft}


\begin{flushleft}
MCP100	
\end{flushleft}


\begin{flushleft}
MCP101	
\end{flushleft}


	





\begin{flushleft}
Engineering Mechanics	
\end{flushleft}


3	 1	 0	


\begin{flushleft}
Introduction to Computer Science	
\end{flushleft}


3	 0	 2	


\begin{flushleft}
Environmental Science	
\end{flushleft}


2	 0	 0	


\begin{flushleft}
Introduction to Electrical Engineering	
\end{flushleft}


3	 0	 2	


\begin{flushleft}
Introduction to Engineering Visualization	
\end{flushleft}


0	 0	 4	


\begin{flushleft}
Product Realization through Manufacturing	
\end{flushleft}


0	 0	 4	


\begin{flushleft}
Total Credits				
\end{flushleft}





4


4


2


4


2


2


18	





\begin{flushleft}
Programme-Linked Basic / Engineering Arts / Sciences Core
\end{flushleft}


\begin{flushleft}
COL106	
\end{flushleft}


\begin{flushleft}
ELL201	
\end{flushleft}


\begin{flushleft}
PYL102	
\end{flushleft}


	





\begin{flushleft}
Data Structures and Algorithms	
\end{flushleft}


3	 0	 4	


\begin{flushleft}
Digital Electronics	
\end{flushleft}


3	 0	 3	


\begin{flushleft}
Principles of Electronic Materials	
\end{flushleft}


3	 0	 0	


\begin{flushleft}
Total Credits				
\end{flushleft}





5


4.5


3


12.5





\begin{flushleft}
Departmental Core
\end{flushleft}


\begin{flushleft}
ELL305	
\end{flushleft}


\begin{flushleft}
ELP305	
\end{flushleft}


\begin{flushleft}
MTL102	
\end{flushleft}


\begin{flushleft}
MTL103	
\end{flushleft}


\begin{flushleft}
MTL104	
\end{flushleft}


\begin{flushleft}
MTL105	
\end{flushleft}


\begin{flushleft}
MTL106	
\end{flushleft}


\begin{flushleft}
MTL107	
\end{flushleft}


\begin{flushleft}
MTL122	
\end{flushleft}


\begin{flushleft}
MTL180	
\end{flushleft}


\begin{flushleft}
MTP290	
\end{flushleft}


\begin{flushleft}
MTL342	
\end{flushleft}


\begin{flushleft}
MTL390	
\end{flushleft}


\begin{flushleft}
MTD411	
\end{flushleft}


\begin{flushleft}
MTL421	
\end{flushleft}


\begin{flushleft}
MTL445	
\end{flushleft}


	


\begin{flushleft}
MTL458	
\end{flushleft}


	





\begin{flushleft}
Computer Architecture	
\end{flushleft}


3	 0	 0	 3


\begin{flushleft}
Design and System Laboratory	
\end{flushleft}


0	 0	 3	 1.5


\begin{flushleft}
Differential Equations	
\end{flushleft}


3	 0	 0	 3


\begin{flushleft}
Optimization Methods and Applications	
\end{flushleft}


3	 0	 0	 3


\begin{flushleft}
Linear Algebra and Applications	
\end{flushleft}


3	 0	 0	 3


\begin{flushleft}
Algebra	
\end{flushleft}


3	0	 0	 3


\begin{flushleft}
Probability and Stochastic Processes	
\end{flushleft}


3	 1	 0	 4


\begin{flushleft}
Numerical Methods and Computations	
\end{flushleft}


3	 0	 0	 3


\begin{flushleft}
Real and Complex Analysis	
\end{flushleft}


3	 1	 0	 4


\begin{flushleft}
Discrete Mathematical Structures	
\end{flushleft}


3	 1	 0	 4


\begin{flushleft}
Computing Laboratory	
\end{flushleft}


0	 0	 4	 2


\begin{flushleft}
Analysis and Design of Algorithms	
\end{flushleft}


3	 1	 0	 4


\begin{flushleft}
Statistical Methods	
\end{flushleft}


3	 1	 0	 4


\begin{flushleft}
B.Tech. Project	
\end{flushleft}


0	 0	 8	 4


\begin{flushleft}
Functional Analysis	
\end{flushleft}


3	 0	 0	 3


\begin{flushleft}
Computational Methods for 	
\end{flushleft}


3	 0	 2	 4


\begin{flushleft}
Differential Equations
\end{flushleft}


\begin{flushleft}
Operating Systems	
\end{flushleft}


3	 0	 2	 4


\begin{flushleft}
Total Credits			63.5	
\end{flushleft}





\begin{flushleft}
Departmental Electives
\end{flushleft}


\begin{flushleft}
COL334	
\end{flushleft}


\begin{flushleft}
COL728	
\end{flushleft}


\begin{flushleft}
ELL365	
\end{flushleft}


\begin{flushleft}
ELL715	
\end{flushleft}


\begin{flushleft}
ELL786	
\end{flushleft}


\begin{flushleft}
ELL789	
\end{flushleft}





\begin{flushleft}
Computer Networks	
\end{flushleft}


\begin{flushleft}
Compiler Design	
\end{flushleft}


\begin{flushleft}
Embedded Systems	
\end{flushleft}


\begin{flushleft}
Digital Image Processing	
\end{flushleft}


\begin{flushleft}
Multimedia Systems	
\end{flushleft}


\begin{flushleft}
Intelligent Systems	
\end{flushleft}





3	 0	 2	


3	 0	 3	


3	 0	 0	


3	 0	 2	


3	 0	 0	


3	 0	 0	





4


4.5


3


4


3


3





61





\begin{flushleft}
ELL792	 Computer Graphics	
\end{flushleft}


3	 0	 0	 3


\begin{flushleft}
ELL793	 Computer Vision	
\end{flushleft}


3	 0	 0	 3


\begin{flushleft}
ELL884	 Information Retrieval	
\end{flushleft}


3	 0	 0	 3


\begin{flushleft}
MTL145	 Number Theory	
\end{flushleft}


3	 0	 0	 3


\begin{flushleft}
MTL146	 Combinatorics	
\end{flushleft}


3	0	 0	 3


\begin{flushleft}
MTL260	 Boundary Value Problems	
\end{flushleft}


3	 0	 0	 3


\begin{flushleft}
MTL265	 Mathematical Programming Techniques	
\end{flushleft}


3	 0	 0	 3


\begin{flushleft}
MTL270	 Measure Integral and Probability	
\end{flushleft}


3	 0	 0	 3


\begin{flushleft}
MTD350	 Mini Project	
\end{flushleft}


0	 0	 6	 3


\begin{flushleft}
MTL415	 Parallel Algorithms	
\end{flushleft}


3	 0	 0	 3


\begin{flushleft}
MTL704	 Numerical Optimization	
\end{flushleft}


3	 0	 0	 3


\begin{flushleft}
MTL717	 Fuzzy Sets and Applications	
\end{flushleft}


3	 0	 0	 3


\begin{flushleft}
MTL720	 Neurocomputing and Applications	
\end{flushleft}


3	 0	 0	 3


\begin{flushleft}
MTL725	 Stochastic Processes and its Applications 	
\end{flushleft}


3	 0	 0	 3


\begin{flushleft}
MTL728	 Category Theory	
\end{flushleft}


3	 0	 0	 3


\begin{flushleft}
MTL729	 Computational Algebra and its Applications	
\end{flushleft}


3	 0	 0	 3


\begin{flushleft}
MTL730	 Cryptography	
\end{flushleft}


3	0	 0	 3


\begin{flushleft}
MTL731	 Introduction to Chaotic Dynamical Systems	 3	 0	 0	 3
\end{flushleft}


\begin{flushleft}
MTL732	 Financial Mathematics	
\end{flushleft}


3	 0	 0	 3


\begin{flushleft}
MTL733	 Stochastic of Finance	
\end{flushleft}


3	 0	 0	 3


\begin{flushleft}
MTL735	 Advanced Number Theory	
\end{flushleft}


3	 0	 0	 3


\begin{flushleft}
MTL738	 Commutative Algebra	
\end{flushleft}


3	 0	 0	 3


\begin{flushleft}
MTL739	 Representation of Finite Groups	
\end{flushleft}


3	 0	 0	 3


\begin{flushleft}
MTL741	 Fractal Geometry	
\end{flushleft}


3	 0	 0	 3


\begin{flushleft}
MTL742	 Operator Theory	
\end{flushleft}


3	 0	 0	 3


\begin{flushleft}
MTL743	 Fourier Analysis	
\end{flushleft}


3	 0	 0	 3


\begin{flushleft}
MTL744	 Mathematical Theory of Coding	
\end{flushleft}


3	 0	 0	 3


\begin{flushleft}
MTL745	 Advanced Matrix Theory 	
\end{flushleft}


3	 0	 0	 3


\begin{flushleft}
MTL747	 Mathematical Logic	
\end{flushleft}


3	 0	 0	 3


\begin{flushleft}
MTL751	 Symbolic Dynamics	
\end{flushleft}


3	 0	 0	 3


\begin{flushleft}
MTL754	 Principles of Computer Graphics	
\end{flushleft}


3	 0	 0	 3


\begin{flushleft}
MTL755	 Algebraic Geometry	
\end{flushleft}


3	 0	 0	 3


\begin{flushleft}
MTL756	 Lie Algebras and Lie Groups	
\end{flushleft}


3	 0	 0	 3


\begin{flushleft}
MTL757	 Introduction to Algebraic Topology	
\end{flushleft}


3	 0	 0	 3


\begin{flushleft}
MTL760	 Advanced Algorithms	
\end{flushleft}


3	 0	 0	 3


\begin{flushleft}
MTL761	 Basic Ergodic Theory	
\end{flushleft}


3	 0	 0	 3


\begin{flushleft}
MTL762	 Probability Theory	
\end{flushleft}


3	 0	 0	 3


\begin{flushleft}
MTL763	 Introduction to Game Theory	
\end{flushleft}


3	 0	 0	 3


\begin{flushleft}
MTL766	 Multivariate Statistical Methods	
\end{flushleft}


3	 0	 0	 3


\begin{flushleft}
MTL768	 Graph Theory	
\end{flushleft}


3	 0	 0	 3


\begin{flushleft}
MTL773	 Wavelets and Applications	
\end{flushleft}


3	 0	 0	 3


\begin{flushleft}
MTL781	 Finite Element Theory and Applications	
\end{flushleft}


3	 0	 0	 3


\begin{flushleft}
MTL785	 Natural Language Processing	
\end{flushleft}


3	 0	 0	 3


\begin{flushleft}
MTL792	 Modern Methods in Partial 	
\end{flushleft}


3	 0	 0	 3


	


\begin{flushleft}
Differential equations
\end{flushleft}


\begin{flushleft}
MTL793	 Numerical Methods for Hyperbolic PDEs	
\end{flushleft}


3	 0	 0	 3


\begin{flushleft}
MTL794	 Advanced Probability Theory	
\end{flushleft}


3	 0	 0	 3


\begin{flushleft}
MTL795	 Numerical Method for Partial 	
\end{flushleft}


3	 1	 0	 4


	


\begin{flushleft}
Differential Equations
\end{flushleft}


\begin{flushleft}
MTV791	 Special Module in Dynamical System	
\end{flushleft}


1	 0	 0	 1


\begin{flushleft}
MTL811	 Mathematical Foundation of Artificial Intelligence	 3	0	 0	 3
\end{flushleft}


\begin{flushleft}
MTL843	 Mathematical Modeling of Credit Risk	
\end{flushleft}


3	 0	 0	 3


\begin{flushleft}
MTL851	 Applied Numerical Analysis 	
\end{flushleft}


3	 0	 0	 3


\begin{flushleft}
MTL854	 Interpolation and Approximation	
\end{flushleft}


3	 0	 0	 3


\begin{flushleft}
MTL855	 Multiple Decision Procedures in Ranking	
\end{flushleft}


3	 0	 0	 3


	


\begin{flushleft}
and Selection
\end{flushleft}


\begin{flushleft}
MTL860	 Linear Algebra	
\end{flushleft}


3	 0	 0	 3


\begin{flushleft}
MTL863	 Algebraic Number Theory	
\end{flushleft}


3	 0	 0	 3


\begin{flushleft}
MTV874	 Analysis	
\end{flushleft}


3	0	 0	 3


\begin{flushleft}
MTL882	 Applied Analysis	
\end{flushleft}


3	 0	 0	 3


\begin{flushleft}
MTL883	 Physical Fluid Mechanics	
\end{flushleft}


3	 0	 0	 3


\begin{flushleft}
MTL888	 Boundary Elements Methods with Computer 	 3	 0	 0	 3
\end{flushleft}


\begin{flushleft}
	Implementation
\end{flushleft}





\newpage
62





\begin{flushleft}
Semester
\end{flushleft}





\begin{flushleft}
VIII
\end{flushleft}





\begin{flushleft}
VII
\end{flushleft}





\begin{flushleft}
VI
\end{flushleft}





\begin{flushleft}
V
\end{flushleft}





\begin{flushleft}
IV
\end{flushleft}





\begin{flushleft}
III
\end{flushleft}





\begin{flushleft}
II
\end{flushleft}





\begin{flushleft}
I
\end{flushleft}





0





4





0


4


\begin{flushleft}
MTL122
\end{flushleft}





5





4





3





3





4





0 3 4.5


\begin{flushleft}
ELL305
\end{flushleft}





\begin{flushleft}
Digital Electronics
\end{flushleft}





1 0


\begin{flushleft}
ELL201
\end{flushleft}





0 2


\begin{flushleft}
MTL783
\end{flushleft}





4





3





3





1





0





0


2


\begin{flushleft}
OC 2
\end{flushleft}





4





4





3





0





0





0 0


\begin{flushleft}
OC 3
\end{flushleft}





3





3





3





3





\begin{flushleft}
Data Mining
\end{flushleft}





0 0


\begin{flushleft}
MTL782
\end{flushleft}





3





3





0





3





0


0


\begin{flushleft}
MTL103
\end{flushleft}





3





3





3





3





3





3





0





0





0


0


\begin{flushleft}
DE 3
\end{flushleft}





1


0


\begin{flushleft}
DE 2
\end{flushleft}





\begin{flushleft}
Statistical Methods
\end{flushleft}





0


0


\begin{flushleft}
MTL390
\end{flushleft}





\begin{flushleft}
Algebra
\end{flushleft}





0


0


\begin{flushleft}
MTL105
\end{flushleft}





3





3





4





3





3





\begin{flushleft}
Optimization Methods and
\end{flushleft}


\begin{flushleft}
Applications
\end{flushleft}





3





\begin{flushleft}
Principles of Electronic
\end{flushleft}


\begin{flushleft}
Materials
\end{flushleft}





\begin{flushleft}
Theory of Computation
\end{flushleft}





0


0


\begin{flushleft}
MTL712
\end{flushleft}





0





\begin{flushleft}
Discrete Mathematical
\end{flushleft}


\begin{flushleft}
Structures
\end{flushleft}





\begin{flushleft}
Computational Methods for
\end{flushleft}


\begin{flushleft}
Differential Equations
\end{flushleft}





3





\begin{flushleft}
Differential Equations
\end{flushleft}





4





3


\begin{flushleft}
PYL102
\end{flushleft}





3





1


0


\begin{flushleft}
MTL102
\end{flushleft}





3





\begin{flushleft}
MTL180
\end{flushleft}





3





4





2





\begin{flushleft}
Computer Architecture
\end{flushleft}





1


0


\begin{flushleft}
MTL106
\end{flushleft}





0





0


0


\begin{flushleft}
CML100
\end{flushleft}





\begin{flushleft}
Introduction to Chemistry
\end{flushleft}





3





1 0


\begin{flushleft}
MTL101
\end{flushleft}





\begin{flushleft}
Calculus
\end{flushleft}





4





3





1





0





4





\begin{flushleft}
Linear Algebra and
\end{flushleft}


\begin{flushleft}
Differential Equations
\end{flushleft}





3





\begin{flushleft}
Course-4
\end{flushleft}





\begin{flushleft}
MTL100
\end{flushleft}





0 4


\begin{flushleft}
CMP100
\end{flushleft}





\begin{flushleft}
Physics Laboratory
\end{flushleft}





2





0





0





4





2





\begin{flushleft}
Chemistry Laboratory
\end{flushleft}





0





\begin{flushleft}
Course-5
\end{flushleft}





\begin{flushleft}
PYP100
\end{flushleft}





\begin{flushleft}
Course-6
\end{flushleft}





0





0





4





2





\begin{flushleft}
Product Realization through
\end{flushleft}


\begin{flushleft}
Manufacturing
\end{flushleft}





\begin{flushleft}
MCP101
\end{flushleft}





\begin{flushleft}
Course-7
\end{flushleft}





0





0





2





1





\begin{flushleft}
Introduction to Engineering
\end{flushleft}


\begin{flushleft}
(Non-graded)
\end{flushleft}





\begin{flushleft}
NIN100
\end{flushleft}





0





1





0.5 0





\begin{flushleft}
Professional Ethics and
\end{flushleft}


\begin{flushleft}
Social Responsibility-2
\end{flushleft}


\begin{flushleft}
(Non-graded)
\end{flushleft}





0





\begin{flushleft}
L
\end{flushleft}





0 0


\begin{flushleft}
SBL100
\end{flushleft}





2





0 2


\begin{flushleft}
MTL107
\end{flushleft}





4





3





3





3





3





0





0





0 2


\begin{flushleft}
DE 4
\end{flushleft}





\begin{flushleft}
Operating Systems
\end{flushleft}





0 0


\begin{flushleft}
MTL458
\end{flushleft}





\begin{flushleft}
Functional Analysis
\end{flushleft}





0 0


\begin{flushleft}
MTL411
\end{flushleft}





3





4





3





3





\begin{flushleft}
Numerical Methods and
\end{flushleft}


\begin{flushleft}
Computation
\end{flushleft}





3





\begin{flushleft}
Introduction to Biology for
\end{flushleft}


\begin{flushleft}
Engineers
\end{flushleft}





2





\begin{flushleft}
Environmental Science
\end{flushleft}





\begin{flushleft}
CVL100
\end{flushleft}





0 0


\begin{flushleft}
MTP290
\end{flushleft}





3





0 4


\begin{flushleft}
MTL342
\end{flushleft}





2





0





3





3





3





0





8





\begin{flushleft}
B.Tech. Project
\end{flushleft}





0 0


\begin{flushleft}
MTD421
\end{flushleft}





0 0


\begin{flushleft}
OC 1
\end{flushleft}





1 0


\begin{flushleft}
DE 1
\end{flushleft}





4





3





3





4





\begin{flushleft}
Analysis and Design of
\end{flushleft}


\begin{flushleft}
Algorithms
\end{flushleft}





0





\begin{flushleft}
Computing Laboratory
\end{flushleft}





3





\begin{flushleft}
Linear Algebra and
\end{flushleft}


\begin{flushleft}
Applications
\end{flushleft}





\begin{flushleft}
MTL104
\end{flushleft}





3





0





3





3





3





4





4





4





0





0





3





0 3 1.5


\begin{flushleft}
HUL3XX
\end{flushleft}





\begin{flushleft}
Design and System
\end{flushleft}


\begin{flushleft}
Laboratory
\end{flushleft}





1 0


\begin{flushleft}
ELP305
\end{flushleft}





1 0


\begin{flushleft}
HUL2XX
\end{flushleft}





1 0


\begin{flushleft}
HUL2XX
\end{flushleft}





\begin{flushleft}
HUL2XX
\end{flushleft}





0





0





2





1





\begin{flushleft}
Intro. to Mathematics \&
\end{flushleft}


\begin{flushleft}
Computing (Non-graded)
\end{flushleft}





\begin{flushleft}
MTN101
\end{flushleft}





\begin{flushleft}
T
\end{flushleft}





0





2





\begin{flushleft}
Language and
\end{flushleft}


\begin{flushleft}
Writing Skills-2
\end{flushleft}


\begin{flushleft}
(Non-Graded)
\end{flushleft}





1





12





12





18





15





18





15





17





1





0





1





3





2





2





2





0 1 0.5 0 0 2 1 9.5 1


\begin{flushleft}
NEN100
\end{flushleft}


\begin{flushleft}
NLN100
\end{flushleft}





\begin{flushleft}
Language and
\end{flushleft}


\begin{flushleft}
Writing Skills-1
\end{flushleft}


\begin{flushleft}
(Non-Graded)
\end{flushleft}





\begin{flushleft}
Professional Ethics and
\end{flushleft}


\begin{flushleft}
Social Responsibility-1
\end{flushleft}


\begin{flushleft}
(Non-graded)
\end{flushleft}





0





\begin{flushleft}
NLN100
\end{flushleft}





\begin{flushleft}
Course-8
\end{flushleft}





\begin{flushleft}
NEN100
\end{flushleft}





\begin{flushleft}
Course-9
\end{flushleft}





\begin{flushleft}
Note: Courses 1-6 above are attended in the given order by half of all first year students. The other half of First year students attend the Courses 1-6 of II semester first.
\end{flushleft}





\begin{flushleft}
Probability and Stochastic
\end{flushleft}


\begin{flushleft}
Processes
\end{flushleft}





3





\begin{flushleft}
Real and Complex Analysis
\end{flushleft}





3





\begin{flushleft}
Data Structures \& Algorithms
\end{flushleft}





\begin{flushleft}
COL106
\end{flushleft}





1





3





2





3





0 3


\begin{flushleft}
COL100
\end{flushleft}





\begin{flushleft}
Introduction to Computer
\end{flushleft}


\begin{flushleft}
Science
\end{flushleft}





4





\begin{flushleft}
Engineering Mechanics
\end{flushleft}





0


2


\begin{flushleft}
APL100
\end{flushleft}





0.5





\begin{flushleft}
PYL100
\end{flushleft}





3





\begin{flushleft}
Course-1
\end{flushleft}





\begin{flushleft}
Introduction to Engineering Electromagnetic Waves and
\end{flushleft}


\begin{flushleft}
Visualization
\end{flushleft}


\begin{flushleft}
Quantum Mechanics
\end{flushleft}





\begin{flushleft}
MCP100
\end{flushleft}





\begin{flushleft}
Course-3
\end{flushleft}





\begin{flushleft}
Introduction to Electrical
\end{flushleft}


\begin{flushleft}
Engineering
\end{flushleft}





\begin{flushleft}
Course-2
\end{flushleft}





\begin{flushleft}
ELL100
\end{flushleft}





\begin{flushleft}
Credits
\end{flushleft}





0





0





0





0





0





1





21.0 153.0





22.0





21.0





21.0





26.0





25.0





\begin{flushleft}
TOTAL=153.0
\end{flushleft}





8 17.0





4 20.0





5 18.5





0 21.0





9 21.5





4 21.0





6 17.0 1.5 23.0





13 17.0 2.5 28.5





\begin{flushleft}
P
\end{flushleft}





\begin{flushleft}
Non-Graded Units
\end{flushleft}





\begin{flushleft}
B. Tech. in Mathematics and Computing	MT1
\end{flushleft}


\begin{flushleft}
Contact Hours
\end{flushleft}





\begin{flushleft}
\newpage
Programme Code: MT6
\end{flushleft}





\begin{flushleft}
Dual Degree Programme: Bachelor of Technology and Master of Technology
\end{flushleft}


\begin{flushleft}
in Mathematics and Computing
\end{flushleft}


\begin{flushleft}
Department of Mathematics
\end{flushleft}


\begin{flushleft}
The overall Credit Structure
\end{flushleft}





\begin{flushleft}
Departmental Electives
\end{flushleft}





\begin{flushleft}
Course Category	
\end{flushleft}


\begin{flushleft}
Credits
\end{flushleft}


\begin{flushleft}
Institute Core Courses
\end{flushleft}


\begin{flushleft}
Basic Sciences (BS)		 22
\end{flushleft}


\begin{flushleft}
Engineering Arts and Science (EAS)		 18
\end{flushleft}


\begin{flushleft}
Humanities and Social Sciences (HuSS)		 15
\end{flushleft}


\begin{flushleft}
Programme-linked Courses		12.5
\end{flushleft}


\begin{flushleft}
Departmental Courses
\end{flushleft}


\begin{flushleft}
Departmental Core 		59.5
\end{flushleft}


\begin{flushleft}
Departmental Electives		 6
\end{flushleft}


\begin{flushleft}
Open Category Courses		 12
\end{flushleft}


\begin{flushleft}
Total B.Tech. Credit requirement		 145
\end{flushleft}


\begin{flushleft}
Non Graded Units		 15
\end{flushleft}


\begin{flushleft}
M.Tech. Part
\end{flushleft}


\begin{flushleft}
Programme Core Courses 		 24
\end{flushleft}


\begin{flushleft}
Programme Electives Courses 		 18
\end{flushleft}


\begin{flushleft}
Total M.Tech. Requirement		 42
\end{flushleft}


\begin{flushleft}
Total Graded Requirement		 187
\end{flushleft}





\begin{flushleft}
COL334	Computer Networks	
\end{flushleft}


\begin{flushleft}
ELL365	 Embedded Systems	
\end{flushleft}


\begin{flushleft}
MTL145	 Number Theory	
\end{flushleft}


\begin{flushleft}
MTL146	 Combinatorics	
\end{flushleft}


\begin{flushleft}
MTL260	 Boundary Value Problems	
\end{flushleft}


\begin{flushleft}
MTL265	 Mathematical Programming Techniques	
\end{flushleft}


\begin{flushleft}
MTL270	 Measure Integral and Probability	
\end{flushleft}


\begin{flushleft}
MTD350	Mini Project	
\end{flushleft}


\begin{flushleft}
MTL415	 Parallel Algorithms	
\end{flushleft}


\begin{flushleft}
MTL768	 Graph Theory	
\end{flushleft}


\begin{flushleft}
MTL773	 Wavelets and Applications	
\end{flushleft}


\begin{flushleft}
Program Core
\end{flushleft}





\begin{flushleft}
MTL766	 Multivariate Statistical Methods	
\end{flushleft}


3	 0	 0	 3


\begin{flushleft}
MTL781	 Finite Elements and Applications	
\end{flushleft}


3	 0	 0	 3


\begin{flushleft}
MTD853*	Major Project Part-I	
\end{flushleft}


0	 0	 8	 4


\begin{flushleft}
MTD854*	Major Project Part-II	
\end{flushleft}


0	 0	 28	14


	


\begin{flushleft}
Total Credits				 22
\end{flushleft}





\begin{flushleft}
Institute Core : Basic Sciences
\end{flushleft}


\begin{flushleft}
CML100	 General Chemistry	
\end{flushleft}


3	 0	 0	 3


\begin{flushleft}
CMP100	Chemistry Laboratory	
\end{flushleft}


0	0	4	 2


\begin{flushleft}
MTL100	 Calculus	
\end{flushleft}


3	1	0	 4


\begin{flushleft}
MTL101	 Linear Algebra and Differential Equations	 3	 1	 0	 4
\end{flushleft}


\begin{flushleft}
PYL100	 Electromagnetic Waves and 	
\end{flushleft}


3	 0	 0	 3


	


\begin{flushleft}
Quantum Mechanics
\end{flushleft}


\begin{flushleft}
PYP100	 Physics Laboratory	
\end{flushleft}


0	0	4	 2


\begin{flushleft}
SBL100	 Introductory Biology for Engineers	
\end{flushleft}


3	 0	 2	 4


	


\begin{flushleft}
Total Credits				 22
\end{flushleft}





\begin{flushleft}
*MTD853 and MTD854 together are alternatives to MTD851 and
\end{flushleft}


\begin{flushleft}
MTD852
\end{flushleft}


\begin{flushleft}
Program Electives
\end{flushleft}


\begin{flushleft}
COL728	Compiler Design	
\end{flushleft}


3	0	3	 4.5


\begin{flushleft}
ELL715	 Digital Image Processing	
\end{flushleft}


3	 0	 2	 4


\begin{flushleft}
ELL786	 Multimedia Systems	
\end{flushleft}


3	0	0	 3


\begin{flushleft}
ELL789	 Intelligent Systems	
\end{flushleft}


3	0	0	 3


\begin{flushleft}
ELL792	 Computer Graphics	
\end{flushleft}


3	 0	 0	 3


\begin{flushleft}
ELL793	 Computer Vision	
\end{flushleft}


3	0	0	 3


\begin{flushleft}
ELL884	 Information Retrieval	
\end{flushleft}


3	0	0	 3


\begin{flushleft}
MTL704	 Numerical Optimization	
\end{flushleft}


3	0	0	 3


\begin{flushleft}
MTL717	 Fuzzy Sets and Applications	
\end{flushleft}


3	 0	 0	 3


\begin{flushleft}
MTL720	 Neurocomputing and Applications	
\end{flushleft}


3	0	0	 3


\begin{flushleft}
MTL725	 Stochastic Processes and its Applications 	 3	 0	 0	 3
\end{flushleft}


\begin{flushleft}
MTL728	 Category Theory	
\end{flushleft}


3	0	0	 3


\begin{flushleft}
MTL729	 Computational Algebra and its Applications	 3	 0	 0	 3
\end{flushleft}


\begin{flushleft}
MTL730	 Cryptography	
\end{flushleft}


3	0	0	 3


\begin{flushleft}
MTL731	 Introduction to Chaotic Dynamical Systems	 3	 0	 0	 3
\end{flushleft}


\begin{flushleft}
MTL732	 Financial Mathematics	
\end{flushleft}


3	0	0	 3


\begin{flushleft}
MTL733	 Stochastic of Finance	
\end{flushleft}


3	 0	 0	 3


\begin{flushleft}
MTL735	 Advanced Number Theory	
\end{flushleft}


3	0	0	 3


\begin{flushleft}
MTL738	 Commutative Algebra	
\end{flushleft}


3	0	0	 3


\begin{flushleft}
MTL739	 Representation of Finite Groups	
\end{flushleft}


3	 0	 0	 3


\begin{flushleft}
MTL741	 Fractal Geometry	
\end{flushleft}


3	 0	 0	 3


\begin{flushleft}
MTL742	 Operator Theory	
\end{flushleft}


3	0	0	 3


\begin{flushleft}
MTL743	 Fourier Analysis	
\end{flushleft}


3	0	0	 3


\begin{flushleft}
MTL744	 Mathematical Theory of Coding	
\end{flushleft}


3	 0	 0	 3


\begin{flushleft}
MTL745	 Advanced Matrix Theory 	
\end{flushleft}


3	 0	 0	 3


\begin{flushleft}
MTL747	 Mathematical Logic	
\end{flushleft}


3	0	0	 3


\begin{flushleft}
MTL751	 Symbolic Dynamics	
\end{flushleft}


3	0	0	 3


\begin{flushleft}
MTL754	 Principles of Computer Graphics	
\end{flushleft}


3	 0	 0	 3


\begin{flushleft}
MTL755	 Algebraic Geometry	
\end{flushleft}


3	 0	 0	 3


\begin{flushleft}
MTL756	 Lie Algebras and Lie Groups	
\end{flushleft}


3	 0	 0	 3


\begin{flushleft}
MTL757	 Introduction to Algebraic Topology	
\end{flushleft}


3	0	0	 3


\begin{flushleft}
MTL760	 Advanced Algorithms	
\end{flushleft}


3	0	0	 3


\begin{flushleft}
MTL761	 Basic Ergodic Theory	
\end{flushleft}


3	0	0	 3


\begin{flushleft}
MTL762	 Probability Theory	
\end{flushleft}


3	0	0	 3


\begin{flushleft}
MTL763	 Introduction to Game Theory	
\end{flushleft}


3	 0	 0	 3


\begin{flushleft}
MTL785	 Natural Language Processing	
\end{flushleft}


3	 0	 0	 3


\begin{flushleft}
MTL792	 Modern Methods in Partial 	
\end{flushleft}


3	 0	 0	 3


	


\begin{flushleft}
Differential equations
\end{flushleft}


\begin{flushleft}
MTL793	 Numerical Methods for Hyperbolic PDEs	
\end{flushleft}


3	 0	 0	 3


\begin{flushleft}
MTL794	 Advanced Probability Theory	
\end{flushleft}


3	0	0	 3


\begin{flushleft}
MTL795	 Numerical Method for Partial 	
\end{flushleft}


3	 1	 0	 4


	


\begin{flushleft}
Differential Equations
\end{flushleft}


\begin{flushleft}
MTV791	 Special Module in Dynamical System	
\end{flushleft}


1	 0	 0	 1


\begin{flushleft}
MTL811	 Mathematical Foundation of Artificial Intelligence	
\end{flushleft}


3	0	0	 3





\begin{flushleft}
Institute Core: Engineering Arts and Sciences
\end{flushleft}


\begin{flushleft}
APL100	 Engineering Mechanics	
\end{flushleft}


3	1	0	 4


\begin{flushleft}
COL100	 Introduction to Computer Science	
\end{flushleft}


3	 0	 2	 4


\begin{flushleft}
CVL100	 Environmental Science	
\end{flushleft}


2	0	0	 2


\begin{flushleft}
ELL100	 Introduction to Electrical Engineering	
\end{flushleft}


3	 0	 2	 4


\begin{flushleft}
MCP100	Engineering Visualization	
\end{flushleft}


0	0	4	 2


\begin{flushleft}
MCP101	 Product Realization through Manufacturing	 0	 0	 4	 2
\end{flushleft}


	


\begin{flushleft}
Total Credits				 18
\end{flushleft}


\begin{flushleft}
Programme-Linked Basic / Engineering Arts / Sciences Core
\end{flushleft}


\begin{flushleft}
COL106	
\end{flushleft}


\begin{flushleft}
ELL201	
\end{flushleft}


\begin{flushleft}
PYL102	
\end{flushleft}


	





\begin{flushleft}
Data Structures and Algorithms	
\end{flushleft}


3	 0	 4	 5


\begin{flushleft}
Digital Electronics	
\end{flushleft}


3	0	3	 4.5


\begin{flushleft}
Principles of Electronic Materials	
\end{flushleft}


3	 0	 0	 3


\begin{flushleft}
Total Credits				 12.5
\end{flushleft}





\begin{flushleft}
Humanities and Social Sciences
\end{flushleft}


\begin{flushleft}
Courses from Humanities, Social Sciences and
\end{flushleft}


\begin{flushleft}
Management offered under this category		
\end{flushleft}





3	0	2	 4


3	0	 0	3


3	0	0	 3


3	0	0	 3


3	 0	 0	 3


3	0	0	 3


3	 0	 0	 3


0	0	6	 3


3	0	0	 3


3	 0	 0	 3


3	0	0	 3





15





\begin{flushleft}
Departmental Core
\end{flushleft}


\begin{flushleft}
ELL305	 Computer Architecture	
\end{flushleft}


3	0	0	 3


\begin{flushleft}
ELP305	 Design and System Laboratory	
\end{flushleft}


0	 0	 3	 1.5


\begin{flushleft}
MTL102	 Differential Equations	
\end{flushleft}


3	0	0	 3


\begin{flushleft}
MTL103	 Optimization Methods and Applications	
\end{flushleft}


3	 0	 0	 3


\begin{flushleft}
MTL104	 Linear Algebra and Applications	
\end{flushleft}


3	 0	 0	 3


\begin{flushleft}
MTL105	 Algebra	
\end{flushleft}


3	0	0	 3


\begin{flushleft}
MTL106	 Probability and Stochastic Processes	
\end{flushleft}


3	 1	 0	 4


\begin{flushleft}
MTL107	 Numerical Methods and Computations	
\end{flushleft}


3	 0	 0	 3


\begin{flushleft}
MTL122	 Real and Complex Analysis	
\end{flushleft}


3	 1	 0	 4


\begin{flushleft}
MTL180	 Discrete Mathematical Structures	
\end{flushleft}


3	 1	 0	 4


\begin{flushleft}
MTP290	Computing Laboratory	
\end{flushleft}


0	0	4	 2


\begin{flushleft}
MTL342	 Analysis and Design of Algorithms	
\end{flushleft}


3	 1	 0	 4


\begin{flushleft}
MTL383	 Theory of Computation	
\end{flushleft}


3	 0	 0	 3


\begin{flushleft}
MTL390	 Statistical Methods	
\end{flushleft}


3	1	0	 4


\begin{flushleft}
MTL411	 Functional Analysis	
\end{flushleft}


3	0	0	 3


\begin{flushleft}
MTL445	 Computational Methods for 	
\end{flushleft}


3	 0	 2	 4


	


\begin{flushleft}
Differential Equations
\end{flushleft}


\begin{flushleft}
MTL458	 Operating Systems	
\end{flushleft}


3	0	2	 4


	


\begin{flushleft}
Total Credits			59.5
\end{flushleft}





63





\begin{flushleft}
\newpage
MTL843	
\end{flushleft}


\begin{flushleft}
MTL851	
\end{flushleft}


\begin{flushleft}
MTL854	
\end{flushleft}


\begin{flushleft}
MTL855	
\end{flushleft}


	


\begin{flushleft}
MTL860	
\end{flushleft}





\begin{flushleft}
Mathematical Modeling of Credit Risk	
\end{flushleft}


\begin{flushleft}
Applied Numerical Analysis 	
\end{flushleft}


\begin{flushleft}
Interpolation and Approximation	
\end{flushleft}


\begin{flushleft}
Multiple Decision Procedures in Ranking	
\end{flushleft}


\begin{flushleft}
and Selection
\end{flushleft}


\begin{flushleft}
Linear Algebra	
\end{flushleft}





3	0	0	 3


3	0	0	 3


3	0	0	 3


3	0	0	 3





\begin{flushleft}
MTL863	 Algebraic Number Theory	
\end{flushleft}


3	0	0	 3


\begin{flushleft}
MTV874	 Analysis	
\end{flushleft}


3	0	0	 3


\begin{flushleft}
MTL882	 Applied Analysis	
\end{flushleft}


3	0	0	 3


\begin{flushleft}
MTL883	 Physical Fluid Mechanics	
\end{flushleft}


3	0	0	 3


\begin{flushleft}
MTL888	 Boundary Elements Methods with Computer 	3	0	0	 3
\end{flushleft}


	


\begin{flushleft}
Implementation
\end{flushleft}





3	0	0	 3





64





\newpage
65





\begin{flushleft}
Semester
\end{flushleft}





3





0





\begin{flushleft}
COL106
\end{flushleft}





1





\begin{flushleft}
X
\end{flushleft}





\begin{flushleft}
IX
\end{flushleft}





\begin{flushleft}
VIII
\end{flushleft}





\begin{flushleft}
VII
\end{flushleft}





\begin{flushleft}
VI
\end{flushleft}





\begin{flushleft}
V
\end{flushleft}





\begin{flushleft}
IV
\end{flushleft}





4





1 0


\begin{flushleft}
ELL201
\end{flushleft}





4





0 2


\begin{flushleft}
MTL783
\end{flushleft}





4





0





0





3





\begin{flushleft}
OC 2
\end{flushleft}





2





0





24





0


12


\begin{flushleft}
MTD852
\end{flushleft}





0


0


\begin{flushleft}
MTD851
\end{flushleft}





0





12





6





3





4





0 0


\begin{flushleft}
MTL781
\end{flushleft}





3





3





3





0





0





3





\begin{flushleft}
Finite Elements Theory
\end{flushleft}


\begin{flushleft}
and Applications
\end{flushleft}





3





0 0


\begin{flushleft}
OE 1
\end{flushleft}





3





3





\begin{flushleft}
Theory of Computation
\end{flushleft}





3





3





3





0


0


\begin{flushleft}
MTL712
\end{flushleft}





\begin{flushleft}
Data Mining
\end{flushleft}





0 0


\begin{flushleft}
MTL782
\end{flushleft}





\begin{flushleft}
Computational Methods for
\end{flushleft}


\begin{flushleft}
Differential Equations
\end{flushleft}





3





\begin{flushleft}
Differential Equations
\end{flushleft}





4





3





1


0


\begin{flushleft}
MTL102
\end{flushleft}





3





0 3 4.5


\begin{flushleft}
ELL305
\end{flushleft}





\begin{flushleft}
Computer Architecture
\end{flushleft}





4





\begin{flushleft}
Probability and Stochastic
\end{flushleft}


\begin{flushleft}
Processes
\end{flushleft}





1


0


\begin{flushleft}
MTL106
\end{flushleft}





3





3





3





5





\begin{flushleft}
MTL180
\end{flushleft}





\begin{flushleft}
Discrete Mathematical
\end{flushleft}


\begin{flushleft}
Structures
\end{flushleft}





\begin{flushleft}
Digital Electronics
\end{flushleft}





0


4


\begin{flushleft}
MTL122
\end{flushleft}





2





\begin{flushleft}
Real and Complex Analysis
\end{flushleft}





3





0 3


\begin{flushleft}
COL100
\end{flushleft}





\begin{flushleft}
Introduction to
\end{flushleft}


\begin{flushleft}
Computer Science
\end{flushleft}





0.5





3





\begin{flushleft}
Course-3
\end{flushleft}





\begin{flushleft}
Introduction to
\end{flushleft}


\begin{flushleft}
Chemistry
\end{flushleft}





0 0 3


\begin{flushleft}
CML100
\end{flushleft}





1 0 4


\begin{flushleft}
MTL101
\end{flushleft}





\begin{flushleft}
Calculus
\end{flushleft}





\begin{flushleft}
Linear Algebra and
\end{flushleft}


\begin{flushleft}
Differential Equations
\end{flushleft}





3





\begin{flushleft}
Course-4
\end{flushleft}





\begin{flushleft}
MTL100
\end{flushleft}





\begin{flushleft}
Course-5
\end{flushleft}





0 4


\begin{flushleft}
CMP100
\end{flushleft}





2





\begin{flushleft}
Chemistry Laboratory
\end{flushleft}





0





\begin{flushleft}
Physics Laboratory
\end{flushleft}





\begin{flushleft}
PYP100
\end{flushleft}





\begin{flushleft}
Course-6
\end{flushleft}





0





0





4





2





\begin{flushleft}
Product Realization
\end{flushleft}


\begin{flushleft}
through Manufacturing
\end{flushleft}





\begin{flushleft}
MCP101
\end{flushleft}





0





\begin{flushleft}
Course-7
\end{flushleft}





0





2





\begin{flushleft}
Introduction to Engineering
\end{flushleft}


\begin{flushleft}
(Non-graded)
\end{flushleft}





\begin{flushleft}
NIN100
\end{flushleft}





1





\begin{flushleft}
Course-8
\end{flushleft}





0


1 0.5


\begin{flushleft}
NEN100
\end{flushleft}





\begin{flushleft}
Professional Ethics and
\end{flushleft}


\begin{flushleft}
Social Responsibility-2
\end{flushleft}


\begin{flushleft}
(Non-graded)
\end{flushleft}





0





\begin{flushleft}
Professional Ethics and
\end{flushleft}


\begin{flushleft}
Social Responsibility-1
\end{flushleft}


\begin{flushleft}
(Non-graded)
\end{flushleft}





\begin{flushleft}
NEN100
\end{flushleft}





0 2


\begin{flushleft}
NLN100
\end{flushleft}





1





\begin{flushleft}
Language and Writing
\end{flushleft}


\begin{flushleft}
Skills-2 (Non-Graded)
\end{flushleft}





0





\begin{flushleft}
Language and Writing
\end{flushleft}


\begin{flushleft}
Skills-1 (Non-Graded)
\end{flushleft}





\begin{flushleft}
NLN100
\end{flushleft}





\begin{flushleft}
Course-9
\end{flushleft}





\begin{flushleft}
T
\end{flushleft}





9.5 1





\begin{flushleft}
L
\end{flushleft}





\begin{flushleft}
CVL100
\end{flushleft}





0 0


\begin{flushleft}
MTL103
\end{flushleft}





3





3





0 0 2


\begin{flushleft}
SBL100
\end{flushleft}





0 0


\begin{flushleft}
MTL390
\end{flushleft}





\begin{flushleft}
Algebra
\end{flushleft}





0 0


\begin{flushleft}
MTL105
\end{flushleft}


3





3





0 0


\begin{flushleft}
MTL766
\end{flushleft}





0 0


\begin{flushleft}
PE 1
\end{flushleft}





1 0


\begin{flushleft}
DE 2
\end{flushleft}





3





3





4





3





0





0





3





\begin{flushleft}
Multivariate Statistical
\end{flushleft}


\begin{flushleft}
Methods
\end{flushleft}





3





3





3





\begin{flushleft}
Statistical Methods
\end{flushleft}





3





3





0 2 4


\begin{flushleft}
MTL107
\end{flushleft}


0 0 3


\begin{flushleft}
MTL411
\end{flushleft}


0 0 3


\begin{flushleft}
MTL358
\end{flushleft}





3





3





3





0





0





0 0


\begin{flushleft}
PE 5
\end{flushleft}





0 2


\begin{flushleft}
PE 2
\end{flushleft}





3





3





4





\begin{flushleft}
Operating Systems
\end{flushleft}





3





\begin{flushleft}
Functional Analysis
\end{flushleft}





3





\begin{flushleft}
Numerical Methods
\end{flushleft}


\begin{flushleft}
and Computation
\end{flushleft}





3





\begin{flushleft}
Optimization Methods Introductory to Biology
\end{flushleft}


\begin{flushleft}
and Applications
\end{flushleft}


\begin{flushleft}
for Engineers
\end{flushleft}





3





\begin{flushleft}
Principles of Electronic
\end{flushleft}


\begin{flushleft}
Environmental Science
\end{flushleft}


\begin{flushleft}
Materials
\end{flushleft}





\begin{flushleft}
PYL102
\end{flushleft}


0 0


\begin{flushleft}
MTP290
\end{flushleft}





3





0 4


\begin{flushleft}
MTL342
\end{flushleft}





2





3





3





3





3





3





0





0





0 0


\begin{flushleft}
PE 6
\end{flushleft}





0 0


\begin{flushleft}
PE 3
\end{flushleft}





0 0


\begin{flushleft}
OC 1
\end{flushleft}





1 0


\begin{flushleft}
DE 1
\end{flushleft}





3





3





3





3





4





\begin{flushleft}
Analysis and Design of
\end{flushleft}


\begin{flushleft}
Algorithms
\end{flushleft}





0





\begin{flushleft}
Computing Laboratory
\end{flushleft}





2





\begin{flushleft}
Linear Algebra and
\end{flushleft}


\begin{flushleft}
Applications
\end{flushleft}





\begin{flushleft}
MTL104
\end{flushleft}





1 0 4


\begin{flushleft}
ELP305
\end{flushleft}





1 0 4


\begin{flushleft}
HUL2XX
\end{flushleft}





1 0 4


\begin{flushleft}
HUL2XX
\end{flushleft}





3





3





3





0





0





0





0 0


\begin{flushleft}
OE 2
\end{flushleft}





0 0


\begin{flushleft}
PE 4
\end{flushleft}





3





3





3





0 3 1.5


\begin{flushleft}
HUL3XX
\end{flushleft}





\begin{flushleft}
Design and System
\end{flushleft}


\begin{flushleft}
Laboratory
\end{flushleft}





3





3





3





\begin{flushleft}
HUL2XX
\end{flushleft}


0





0





2





1





\begin{flushleft}
Intro. to Mathematics \& Computing
\end{flushleft}


\begin{flushleft}
(Non-graded)
\end{flushleft}





\begin{flushleft}
MTN101
\end{flushleft}





0





15





18





18





15





18





15





17





0





0





0





0





1





3





2





2





3 0 2 4 3 0 0 3 3 1 0 4 0 0 4 2


0


0


1 0.5 0 0 2 1 12 2


\begin{flushleft}
Note: Courses 1-6 above are attended in the given order by half of all first year students. The other half of First year students attend the Courses 1-6 of II semester first.
\end{flushleft}





4





\begin{flushleft}
Engineering Mechanics
\end{flushleft}





0


2


\begin{flushleft}
APL100
\end{flushleft}





\begin{flushleft}
Introduction to
\end{flushleft}


\begin{flushleft}
Engineering
\end{flushleft}


\begin{flushleft}
Visualization
\end{flushleft}





\begin{flushleft}
Introduction to Electrical
\end{flushleft}


\begin{flushleft}
Engineering
\end{flushleft}





3





\begin{flushleft}
Electromagnetic Waves
\end{flushleft}


\begin{flushleft}
and Quantum
\end{flushleft}


\begin{flushleft}
Mechanics
\end{flushleft}





\begin{flushleft}
MCP100
\end{flushleft}





\begin{flushleft}
Course-1
\end{flushleft}





\begin{flushleft}
III Data Structures \& Algorithms
\end{flushleft}





\begin{flushleft}
II
\end{flushleft}





\begin{flushleft}
I
\end{flushleft}





\begin{flushleft}
PYL100
\end{flushleft}





\begin{flushleft}
Course-2
\end{flushleft}





\begin{flushleft}
ELL100
\end{flushleft}





\begin{flushleft}
Credits
\end{flushleft}





21





1





21





0





18





20





0





0





0





0





24 187.0





27





18





22





21





21





\begin{flushleft}
TOTAL=187.0
\end{flushleft}





24 12





12 21





0





4





5 18.5 0





0





26





25





17 1.5 23





9 21.5 0





4





6





13 17 2.5 29





\begin{flushleft}
P
\end{flushleft}





\begin{flushleft}
Non-Graded Units
\end{flushleft}





\begin{flushleft}
Dual Degree Programme: B. Tech. and M. Tech. in Mathematics and Computing	MT6
\end{flushleft}


\begin{flushleft}
Contact Hours
\end{flushleft}





\begin{flushleft}
\newpage
Bachelor of Technology in Engineering Physics
\end{flushleft}





\begin{flushleft}
Programme Code: PH1
\end{flushleft}





\begin{flushleft}
Department of Physics
\end{flushleft}


\begin{flushleft}
The overall Credit Structure
\end{flushleft}





\begin{flushleft}
PYL113	 Mathematical Physics	
\end{flushleft}


3	 1	0	4


\begin{flushleft}
PYL114	 Solid State Physics	
\end{flushleft}


3	 1	 0	 4


\begin{flushleft}
PYL115	 Applied Optics	
\end{flushleft}


3	 1	0	4


\begin{flushleft}
PYL116	 Elements of Materials Processing	
\end{flushleft}


3	 1	 0	 4


\begin{flushleft}
PYL202	 Statistical Physics	
\end{flushleft}


3	 1	0	4


\begin{flushleft}
PYL203	 Classical Mechanics \& Relativity	
\end{flushleft}


3	 1	 0	 4


\begin{flushleft}
PYL204	 Computational Physics	
\end{flushleft}


3	 1	0	4


\begin{flushleft}
PYP212	 Engineering Physics Laboratory-II	
\end{flushleft}


0	 0	 6	 3


\begin{flushleft}
PYP221	 Engineering Physics Laboratory-III	
\end{flushleft}


0	 0	 8	 4


\begin{flushleft}
PYP222	 Engineering Physics Laboratory-IV	
\end{flushleft}


0	 0	 8	 4


\begin{flushleft}
PYD411	 Project-I	
\end{flushleft}


0	 0	8	4


	


\begin{flushleft}
Total Credits				58	
\end{flushleft}





\begin{flushleft}
Course Category	
\end{flushleft}


\begin{flushleft}
Credits
\end{flushleft}


\begin{flushleft}
Institute Core Courses
\end{flushleft}


\begin{flushleft}
Basic Sciences (BS)		 22
\end{flushleft}


\begin{flushleft}
Engineering Arts and Science (EAS)		 18
\end{flushleft}


\begin{flushleft}
Humanities and Social Sciences (HuSS)		 15
\end{flushleft}


\begin{flushleft}
Programme-linked Courses		14.5
\end{flushleft}


\begin{flushleft}
Departmental Courses
\end{flushleft}


\begin{flushleft}
Departmental Core 		 58
\end{flushleft}


\begin{flushleft}
Departmental Electives		 12
\end{flushleft}


\begin{flushleft}
Open Category Courses		 10
\end{flushleft}


\begin{flushleft}
Total Graded Credit requirement		149.5
\end{flushleft}


\begin{flushleft}
Non Graded Units		 15
\end{flushleft}





\begin{flushleft}
Departmental Electives
\end{flushleft}





\begin{flushleft}
Institute Core : Basic Sciences
\end{flushleft}





\begin{flushleft}
PYS300	 Independent Study	
\end{flushleft}


\begin{flushleft}
PYL301	 Vacuum Technology \& Surface Science	
\end{flushleft}


\begin{flushleft}
PYL302	 Nuclear Science and Engineering	
\end{flushleft}


\begin{flushleft}
PYL303	 Materials Science and Engineering	
\end{flushleft}


\begin{flushleft}
PYL304	 Superconductivity and Applications	
\end{flushleft}


\begin{flushleft}
PYL305	 Engineering Applications of Plasmas	
\end{flushleft}


\begin{flushleft}
PYL306	 Microelectronic Devices	
\end{flushleft}


\begin{flushleft}
PYL311	 Lasers	
\end{flushleft}


\begin{flushleft}
PYL312	 Semiconductor Optoelectronics	
\end{flushleft}


\begin{flushleft}
PYL313	 Fourier Optics and Holography	
\end{flushleft}


\begin{flushleft}
PYL321	 Low Dimensional Physics	
\end{flushleft}


\begin{flushleft}
PYL322	 Nanoscale Fabrication	
\end{flushleft}


\begin{flushleft}
PYL323	 Nanoscale Microscopy	
\end{flushleft}


\begin{flushleft}
PYL324	 Spectroscopy of Nanomaterials	
\end{flushleft}


\begin{flushleft}
PYL331	 Applied Quantum Mechanics	
\end{flushleft}


\begin{flushleft}
PYL332	 General Theory of Relativity \& Cosmology	
\end{flushleft}


\begin{flushleft}
PYL411	 Quantum Electronics	
\end{flushleft}


\begin{flushleft}
PYD412	 Project-II	
\end{flushleft}


\begin{flushleft}
PYL412	 Ultrafast Laser Systems and Applications 	
\end{flushleft}


\begin{flushleft}
PYL413	 Fiber and Integrated Optics	
\end{flushleft}


\begin{flushleft}
PYD414	Project-III	
\end{flushleft}


\begin{flushleft}
PYL414	 Engineering Optics	
\end{flushleft}


\begin{flushleft}
PYV418	 Selected Topics in Photonics	
\end{flushleft}


\begin{flushleft}
PYV419	 Special Topics in Photonics	
\end{flushleft}


\begin{flushleft}
PYL421	 Functional Nanostructures	
\end{flushleft}


\begin{flushleft}
PYL422	 Spintronics	
\end{flushleft}


\begin{flushleft}
PYL423	 Nanoscale Energy Materials \& Devices	
\end{flushleft}


\begin{flushleft}
PYV428	 Selected Topics in Nanotechnology	
\end{flushleft}


\begin{flushleft}
PYV429	 Special Topics in Nanotechnology	
\end{flushleft}


\begin{flushleft}
PYL431	 Relativistic Quantum Mechanics	
\end{flushleft}


\begin{flushleft}
PYL432	 Quantum Electrodynamics	
\end{flushleft}


\begin{flushleft}
PYL433	 Introduction to Gauge Field Theories	
\end{flushleft}


\begin{flushleft}
PYL434	 Particle Accelerators	
\end{flushleft}


\begin{flushleft}
PYV438	 Selected Topics in Theoretical Physics	
\end{flushleft}


\begin{flushleft}
PYV439	 Special Topics in Theoretical Physics	
\end{flushleft}





\begin{flushleft}
CML100	 General Chemistry	
\end{flushleft}


3	 0	 0	 3


\begin{flushleft}
CMP100	Chemistry Laboratory	
\end{flushleft}


0	 0	4	2


\begin{flushleft}
MTL100	 Calculus	
\end{flushleft}


3	 1	0	4


\begin{flushleft}
MTL101	 Linear Algebra and Differential Equations	 3	 1	 0	 4
\end{flushleft}


\begin{flushleft}
PYL100	 Electromagnetic Waves and 	
\end{flushleft}


3	 0	 0	 3


	


\begin{flushleft}
Quantum Mechanics
\end{flushleft}


\begin{flushleft}
PYP100	 Physics Laboratory	
\end{flushleft}


0	 0	4	2


\begin{flushleft}
SBL100	 Introductory Biology for Engineers	
\end{flushleft}


3	 0	 2	 4


	


\begin{flushleft}
Total Credits				22
\end{flushleft}


\begin{flushleft}
Institute Core: Engineering Arts and Sciences
\end{flushleft}


\begin{flushleft}
APL100	 Engineering Mechanics	
\end{flushleft}


3	 1	0	4


\begin{flushleft}
COL100	 Introduction to Computer Science	
\end{flushleft}


3	 0	 2	 4


\begin{flushleft}
CVL100	 Environmental Science	
\end{flushleft}


2	 0	0	2


\begin{flushleft}
ELL100	 Introduction to Electrical Engineering	
\end{flushleft}


3	 0	 2	 4


\begin{flushleft}
MCP100	Engineering Visualization	
\end{flushleft}


0	 0	4	2


\begin{flushleft}
MCP101	 Product Realization through Manufacturing	 0	 0	 4	 2
\end{flushleft}


	


\begin{flushleft}
Total Credits				18
\end{flushleft}


\begin{flushleft}
Programme-Linked Basic / Engineering Arts / Sciences Core
\end{flushleft}


\begin{flushleft}
CML102	
\end{flushleft}


\begin{flushleft}
ELL201	
\end{flushleft}


\begin{flushleft}
ELL205	
\end{flushleft}


\begin{flushleft}
ESL350	
\end{flushleft}


	





\begin{flushleft}
Chemical Synthesis of Functional Materials	 3	 0	 0	 3
\end{flushleft}


\begin{flushleft}
Digital Electronics	
\end{flushleft}


3	 0	3	4.5


\begin{flushleft}
Signals and Systems	
\end{flushleft}


3	 1	 0	 4


\begin{flushleft}
Energy Conservation and Management	
\end{flushleft}


3	 0	 0	 3


\begin{flushleft}
Total Credits				14.5
\end{flushleft}





\begin{flushleft}
Humanities and Social Sciences
\end{flushleft}


\begin{flushleft}
Courses from Humanities, Social Sciences and Management 	
\end{flushleft}


\begin{flushleft}
offered under this category				
\end{flushleft}


15


\begin{flushleft}
Departmental Core
\end{flushleft}


\begin{flushleft}
PYL111	 Electrodynamics	
\end{flushleft}


\begin{flushleft}
PYP111	 Engineering Physics Laboratory-I	
\end{flushleft}


\begin{flushleft}
PYL112	 Quantum Mechanics	
\end{flushleft}





3	 1	0	4


0	 0	 6	 3


3	 1	0	4





66





0	 3	0	3


3	 0	 0	 3


3	 0	 0	 3


3	 0	 0	 3


3	 0	0	3


3	 0	 0	 3


3	 0	0	3


3	 0	0	3


3	 0	0	3


3	 0	 0	 3


3	 0	 0	 3


3	 0	0	3


2	 0	0	2


2	 0	 0	 2


3	 0	 0	 3


3	 0	 0	 3


3	 0	0	3


0	 0	 16	8


3	 0	 0	 3


3	 0	 0	 3


0	 0	8	4


3	 0	0	3


2	 0	 0	 2


1	 0	 0	 1


3	 0	0	3


3	 0	0	3


3	 0	 0	 3


2	 0	 0	 2


1	 0	 0	 1


2	 0	 0	 2


3	 0	0	3


2	 0	 0	 2


2	 0	0	2


2	 0	 0	 2


1	 0	 0	 1





\newpage
67





\begin{flushleft}
Semester
\end{flushleft}





\begin{flushleft}
VIII
\end{flushleft}





\begin{flushleft}
VII
\end{flushleft}





\begin{flushleft}
VI
\end{flushleft}





\begin{flushleft}
V
\end{flushleft}





\begin{flushleft}
IV
\end{flushleft}





\begin{flushleft}
III
\end{flushleft}





\begin{flushleft}
II
\end{flushleft}





\begin{flushleft}
I
\end{flushleft}





\begin{flushleft}
MCP100
\end{flushleft}





\begin{flushleft}
PYL100
\end{flushleft}





\begin{flushleft}
Course-3
\end{flushleft}





\begin{flushleft}
Course-1
\end{flushleft}





0 2


\begin{flushleft}
APL100
\end{flushleft}





4





0





1 0


\begin{flushleft}
PYL112
\end{flushleft}





4





3





3





3





3





3





0





0





0 0


\begin{flushleft}
DE 3
\end{flushleft}





1 0


\begin{flushleft}
DE 1
\end{flushleft}





\begin{flushleft}
Statistical Physics
\end{flushleft}





1 0


\begin{flushleft}
PYL202
\end{flushleft}





\begin{flushleft}
Fundamentals of
\end{flushleft}


\begin{flushleft}
Dielectrics \&
\end{flushleft}


\begin{flushleft}
Semiconductors
\end{flushleft}





1 0


\begin{flushleft}
PYL201
\end{flushleft}





0 3


\begin{flushleft}
COL100
\end{flushleft}





2





3





0





2





4





\begin{flushleft}
Introduction to Computer
\end{flushleft}


\begin{flushleft}
Science
\end{flushleft}





0.5





0


0


\begin{flushleft}
CML100
\end{flushleft}





3





3





0





0





3





\begin{flushleft}
Introduction to Chemistry
\end{flushleft}





3





1 0 4


\begin{flushleft}
MTL101
\end{flushleft}





\begin{flushleft}
Calculus
\end{flushleft}





3





1





0





4





\begin{flushleft}
Linear Algebra and
\end{flushleft}


\begin{flushleft}
Differential Equations
\end{flushleft}





3





\begin{flushleft}
Course-4
\end{flushleft}





\begin{flushleft}
MTL100
\end{flushleft}





\begin{flushleft}
Course-5
\end{flushleft}





0 4


\begin{flushleft}
CMP100
\end{flushleft}





2





0





0





4





2





\begin{flushleft}
Chemistry Laboratory
\end{flushleft}





0





\begin{flushleft}
Physics Laboratory
\end{flushleft}





\begin{flushleft}
PYP100
\end{flushleft}





\begin{flushleft}
Course-6
\end{flushleft}





0





0





4





2





\begin{flushleft}
Product Realization
\end{flushleft}


\begin{flushleft}
through Manufacturing
\end{flushleft}





\begin{flushleft}
MCP101
\end{flushleft}





0





0





2





1





0 1 0.5


\begin{flushleft}
NEN100
\end{flushleft}





0





0





1





0.5





\begin{flushleft}
Professional Ethics and
\end{flushleft}


\begin{flushleft}
Social Responsibility-2
\end{flushleft}


\begin{flushleft}
(Non-graded)
\end{flushleft}





0





\begin{flushleft}
Professional Ethics and
\end{flushleft}


\begin{flushleft}
Social Responsibility-1
\end{flushleft}


\begin{flushleft}
(Non-graded)
\end{flushleft}





\begin{flushleft}
Introduction to
\end{flushleft}


\begin{flushleft}
Engineering
\end{flushleft}


\begin{flushleft}
(Non-graded)
\end{flushleft}





\begin{flushleft}
Course-7
\end{flushleft}





\begin{flushleft}
NEN100
\end{flushleft}





\begin{flushleft}
Course-8
\end{flushleft}





\begin{flushleft}
NIN100
\end{flushleft}





0





0





\begin{flushleft}
L
\end{flushleft}





\begin{flushleft}
T
\end{flushleft}





4





3





3





4





4





4





1 0


\begin{flushleft}
PYL114
\end{flushleft}





1 0


\begin{flushleft}
PYL203
\end{flushleft}





4





4





1 0


\begin{flushleft}
PYL204
\end{flushleft}





4





3





3





3





0





0





0 0


\begin{flushleft}
DE 4
\end{flushleft}





1 0


\begin{flushleft}
DE 2
\end{flushleft}





3





3





4





\begin{flushleft}
Computational Physics
\end{flushleft}





3





\begin{flushleft}
Classical Mechanics \&
\end{flushleft}


\begin{flushleft}
Relativity
\end{flushleft}





3





\begin{flushleft}
Solid State Physics
\end{flushleft}





3





\begin{flushleft}
Mathematical Physics
\end{flushleft}





\begin{flushleft}
PYL113
\end{flushleft}





1


0


\begin{flushleft}
PYL116
\end{flushleft}





4





1


0


\begin{flushleft}
ELL205
\end{flushleft}





4





1


0


\begin{flushleft}
ESL350
\end{flushleft}





4





3





3





3





1





0





0


0


\begin{flushleft}
OC 2
\end{flushleft}





0


0


\begin{flushleft}
OC 1
\end{flushleft}





4





3





3





\begin{flushleft}
Energy Conservation and
\end{flushleft}


\begin{flushleft}
Management
\end{flushleft}





3





\begin{flushleft}
Signals and Systems
\end{flushleft}





3





\begin{flushleft}
Elements of Materials
\end{flushleft}


\begin{flushleft}
Processing
\end{flushleft}





3





\begin{flushleft}
Applied Optics
\end{flushleft}





\begin{flushleft}
PYL115
\end{flushleft}





1 0


\begin{flushleft}
ELL201
\end{flushleft}





4





3





3





3





3





3





0





0





0 0


\begin{flushleft}
OC 3
\end{flushleft}





1 0


\begin{flushleft}
HUL3XX
\end{flushleft}





1 0


\begin{flushleft}
HUL2XX
\end{flushleft}





3





3





4





4





0 3 4.5


\begin{flushleft}
HUL2XX
\end{flushleft}





\begin{flushleft}
Digital Electronics
\end{flushleft}





3





\begin{flushleft}
HUL2XX
\end{flushleft}





0 0


\begin{flushleft}
PYP212
\end{flushleft}





2





0 6


\begin{flushleft}
CML102
\end{flushleft}





3





0 0


\begin{flushleft}
SBL100
\end{flushleft}





3





0





3





0





8





\begin{flushleft}
B.Tech. Project
\end{flushleft}





0 2


\begin{flushleft}
PYD411
\end{flushleft}





4





4





\begin{flushleft}
Introductory Biology for
\end{flushleft}


\begin{flushleft}
Engineers
\end{flushleft}





3





\begin{flushleft}
Chemical Synthesis of
\end{flushleft}


\begin{flushleft}
Functional Materials
\end{flushleft}





0





\begin{flushleft}
Engineering Physics
\end{flushleft}


\begin{flushleft}
Laboratory-II
\end{flushleft}





2





\begin{flushleft}
Environmental Science
\end{flushleft}





\begin{flushleft}
CVL100
\end{flushleft}





6





3





0 8


\begin{flushleft}
PYP222
\end{flushleft}





4





0





0





8





4





\begin{flushleft}
Engineering Physics
\end{flushleft}


\begin{flushleft}
Laboratory-IV
\end{flushleft}





0





\begin{flushleft}
Engineering Physics
\end{flushleft}


\begin{flushleft}
Laboratory-III
\end{flushleft}





\begin{flushleft}
PYP221
\end{flushleft}





0





0





0





2





1





\begin{flushleft}
Engineering Physics
\end{flushleft}


\begin{flushleft}
Laboratory-I
\end{flushleft}





0





\begin{flushleft}
PYN101
\end{flushleft}


\begin{flushleft}
Introduction To
\end{flushleft}


\begin{flushleft}
Engineering Physics
\end{flushleft}


\begin{flushleft}
(Non-graded)
\end{flushleft}





\begin{flushleft}
PYP111
\end{flushleft}





\begin{flushleft}
P
\end{flushleft}





\begin{flushleft}
Credits
\end{flushleft}





0





2





\begin{flushleft}
Language and
\end{flushleft}


\begin{flushleft}
Writing Skills-2
\end{flushleft}


\begin{flushleft}
(Non-Graded)
\end{flushleft}





1 12 2





8 23.0 0 27.0





9 19.5 0 24.0





6 21.0 1 26.0





12 1





12 0





\begin{flushleft}
TOTAL=149.5
\end{flushleft}





0 13.0 0 13.0 149.5





8 16.0 0 20.0





15 3 10 23.0 0 28.0





15 4





12 3





14 4





6 17.0 1.5 23.0





0 2 1 9.5 1 13 17.0 2.5 28.5


\begin{flushleft}
NLN100
\end{flushleft}





\begin{flushleft}
Language and
\end{flushleft}


\begin{flushleft}
Writing Skills-1
\end{flushleft}


\begin{flushleft}
(Non-Graded)
\end{flushleft}





\begin{flushleft}
NLN100
\end{flushleft}





\begin{flushleft}
Course-9
\end{flushleft}





\begin{flushleft}
Note: Courses 1-6 above are attended in the given order by half of all first year students. The other half of First year students attend the Courses 1-6 of II semester first.
\end{flushleft}





\begin{flushleft}
Electrodynamics
\end{flushleft}





\begin{flushleft}
PYL111
\end{flushleft}





1





\begin{flushleft}
Quantum Mechanics
\end{flushleft}





3





3





\begin{flushleft}
Engineering Mechanics
\end{flushleft}





3





\begin{flushleft}
Introduction to Electrical Introduction to Engineering Electromagnetic Waves
\end{flushleft}


\begin{flushleft}
Engineering
\end{flushleft}


\begin{flushleft}
Visualization
\end{flushleft}


\begin{flushleft}
and Quantum Mechanics
\end{flushleft}





\begin{flushleft}
Course-2
\end{flushleft}





\begin{flushleft}
ELL100
\end{flushleft}





\begin{flushleft}
Non-Graded Units
\end{flushleft}





\begin{flushleft}
B. Tech. in Engineering Physics	PH1
\end{flushleft}


\begin{flushleft}
Contact Hours
\end{flushleft}





\begin{flushleft}
\newpage
Bachelor of Technology in Textile Technology
\end{flushleft}





\begin{flushleft}
Programme Code: TT1
\end{flushleft}





\begin{flushleft}
Department of Textile
\end{flushleft}


\begin{flushleft}
The overall Credit Structure
\end{flushleft}





\begin{flushleft}
TXL231	 Fabric Manufacture-I	
\end{flushleft}


3	 0	0	3


\begin{flushleft}
TXP231	 Fabric Manufacture Laboratory-I	
\end{flushleft}


0	 0	 2	 1


\begin{flushleft}
TXL232	 Fabric Manufacture-II	
\end{flushleft}


3	 0	0	3


\begin{flushleft}
TXP232	 Fabric Manufacture Laboratory-II	
\end{flushleft}


0	 0	 2	 1


\begin{flushleft}
TXL241	 Technology of Textile 	
\end{flushleft}


3	 0	 0	 3


	


\begin{flushleft}
Preparation \& Finishing
\end{flushleft}


\begin{flushleft}
TXP241	 Technology of Textile Preparation \& 	
\end{flushleft}


0	 0	 3	 1.5


	


\begin{flushleft}
Finishing Lab
\end{flushleft}


\begin{flushleft}
TXL242	 Technology of Textile Coloration	
\end{flushleft}


3	 0	 0	 3


\begin{flushleft}
TXP242	 Technology of Textile Coloration Lab	
\end{flushleft}


0	 0	 3	 1.5


\begin{flushleft}
TXL361	 Evaluation of Textile Materials	
\end{flushleft}


3	 0	 0	 3


\begin{flushleft}
TXP361	 Evaluation of Textiles Lab	
\end{flushleft}


0	 0	 2	 1


\begin{flushleft}
TXL371	 Theory of Textile Structures	
\end{flushleft}


3	 1	 0	 4


\begin{flushleft}
TXL372	 Speciality Yarns and Fabrics	
\end{flushleft}


2	 0	 0	 2


\begin{flushleft}
TXD401	 Major Project Part-I	
\end{flushleft}


0	 0	 8	 4


	


\begin{flushleft}
Total Credits				52
\end{flushleft}





\begin{flushleft}
Course Category	
\end{flushleft}


\begin{flushleft}
Credits
\end{flushleft}


\begin{flushleft}
Institute Core Courses
\end{flushleft}


\begin{flushleft}
Basic Sciences (BS)		 22
\end{flushleft}


\begin{flushleft}
Engineering Arts and Science (EAS)		 18
\end{flushleft}


\begin{flushleft}
Humanities and Social Sciences (HuSS)		 15
\end{flushleft}


\begin{flushleft}
Programme-linked Courses		12
\end{flushleft}


\begin{flushleft}
Departmental Courses
\end{flushleft}


\begin{flushleft}
Departmental Core 		 52
\end{flushleft}


\begin{flushleft}
Departmental Electives		 16
\end{flushleft}


\begin{flushleft}
Open Category Courses		 10
\end{flushleft}


\begin{flushleft}
Total Graded Credit requirement		 145
\end{flushleft}


\begin{flushleft}
Non Graded Units		 15
\end{flushleft}


\begin{flushleft}
Institute Core : Basic Sciences
\end{flushleft}


\begin{flushleft}
CML100	 General Chemistry	
\end{flushleft}


3	 0	 0	 3


\begin{flushleft}
CMP100	Chemistry Laboratory	
\end{flushleft}


0	 0	4	2


\begin{flushleft}
MTL100	Calculus	
\end{flushleft}


3	 1	0	4


\begin{flushleft}
MTL101	 Linear Algebra and Differential Equations	 3	 1	 0	 4
\end{flushleft}


\begin{flushleft}
PYL100	 Electromagnetic Waves and 	
\end{flushleft}


3	 0	 0	 3


	


\begin{flushleft}
Quantum Mechanics
\end{flushleft}


\begin{flushleft}
PYP100	Physics Laboratory	
\end{flushleft}


0	 0	4	2


\begin{flushleft}
SBL100	 Introductory Biology for Engineers	
\end{flushleft}


3	 0	 2	 4


	


\begin{flushleft}
Total Credits				22	
\end{flushleft}





\begin{flushleft}
Departmental Electives
\end{flushleft}


\begin{flushleft}
TXD301	Mini Project	
\end{flushleft}


0	 0	6	3


\begin{flushleft}
TXR301	Professional Practices	
\end{flushleft}


0	 1	2	2


\begin{flushleft}
TXS301	Independent Studies	
\end{flushleft}


0	 3	0	3


\begin{flushleft}
TXL321	 Multi and Long Fibre Spinning	
\end{flushleft}


3	 0	 0	 3


\begin{flushleft}
TXL331	 Woven Textile Design	
\end{flushleft}


3	 0	0	3


\begin{flushleft}
TXL341	 Colour Science	
\end{flushleft}


2	 0	0	2


\begin{flushleft}
TXL381	 Costing and its Application in Textiles	
\end{flushleft}


3	 1	 0	 4


\begin{flushleft}
TXD402	 Major Project Part-II	
\end{flushleft}


0	 0	 16	8


\begin{flushleft}
TXL700	 Modelling and Simulation in Fibrous Assemblies	2	 0	2	3
\end{flushleft}


\begin{flushleft}
TXL710	 High Performance \& Specialty. Fiber	
\end{flushleft}


3	 0	 0	 3


\begin{flushleft}
TXL719	 Functional \& Smart Textiles	
\end{flushleft}


3	 0	 0	 3


\begin{flushleft}
TXL722	 Mechanics of Spinning Processes	
\end{flushleft}


3	 0	 0	 3


\begin{flushleft}
TXL724	 Textured Yarn Technology	
\end{flushleft}


3	 0	 0	 3


\begin{flushleft}
TXL725	 Mechanics of Spinning Machines	
\end{flushleft}


3	 0	 0	 3


\begin{flushleft}
TXL734	 Nonwoven Science and Engineering	
\end{flushleft}


3	 0	 0	 3


\begin{flushleft}
TXL740	 Science \& App. of Nanotechnology in Textiles	 3	 0	0	3
\end{flushleft}


\begin{flushleft}
TXL741	 Environment Management in Textile and	
\end{flushleft}


3	 0	 0	 3


	


\begin{flushleft}
Allied Industries
\end{flushleft}


\begin{flushleft}
TXL750	 Science of Clothing Comfort	
\end{flushleft}


3	 0	 0	 3


\begin{flushleft}
TXL752	 Design of Functional Clothing	
\end{flushleft}


3	 0	 0	 3


\begin{flushleft}
TXL773	 Medical Textiles	
\end{flushleft}


3	 0	0	3


\begin{flushleft}
TXL774	 Process Control in Yarn \& Fabric Manufacturing	3	 0	0	3
\end{flushleft}


\begin{flushleft}
TXL775	 Technical Textiles	
\end{flushleft}


3	 0	0	3


\begin{flushleft}
TXL776	 Design \& Manuf. of Text. Reinforced Composites	3	 0	0	3
\end{flushleft}


\begin{flushleft}
TXL777	 Product Design and Development	
\end{flushleft}


3	 0	 0	 3


\begin{flushleft}
TXL781	 Project Appraisal and Finance	
\end{flushleft}


3	 0	 0	 3


\begin{flushleft}
TXL782	 Production and Operations Management in	 3	 0	 0	 3
\end{flushleft}


	


\begin{flushleft}
Textile Industry
\end{flushleft}


\begin{flushleft}
TXL783	 Design of Experiments and Statistical Techniques	3	 0	0	3
\end{flushleft}


\begin{flushleft}
TXV701	 Process Cont. and Econ. in Manmade Fibre Prod.	 1	 0	0	1
\end{flushleft}


\begin{flushleft}
TXV702	 Management of Textile Business	
\end{flushleft}


1	 0	 0	 1


\begin{flushleft}
TXV703	 Special Module in Textile Technology	
\end{flushleft}


1	 0	 0	 1


\begin{flushleft}
TXV704	 Special Module in Yarn Manufacture	
\end{flushleft}


1	 0	 0	 1


\begin{flushleft}
TXV705	 Special Module in Fabric Manufacture	
\end{flushleft}


1	 0	 0	 1


\begin{flushleft}
TXV706	 Special Module in Fibre Science	
\end{flushleft}


1	 0	 0	 1


\begin{flushleft}
TXV707	 Special Module in Textile Chemical Processing	1	 0	0	1
\end{flushleft}





\begin{flushleft}
Institute Core: Engineering Arts and Sciences
\end{flushleft}


\begin{flushleft}
APL100	 Engineering Mechanics	
\end{flushleft}


3	 1	0	4


\begin{flushleft}
COL100	 Introduction to Computer Science	
\end{flushleft}


3	 0	 2	 4


\begin{flushleft}
CVL100	Environmental Science	
\end{flushleft}


2	 0	0	2


\begin{flushleft}
ELL100	 Introduction to Electrical Engineering	
\end{flushleft}


3	 0	 2	 4


\begin{flushleft}
MCP100	Engineering Visualization	
\end{flushleft}


0	 0	4	2


\begin{flushleft}
MCP101	 Product Realization through Manufacturing	 0	 0	 4	 2
\end{flushleft}


	


\begin{flushleft}
Total Credits				18
\end{flushleft}


\begin{flushleft}
Programme-Linked Basic / Engineering Arts / Sciences Core
\end{flushleft}


\begin{flushleft}
APL102	 Introduction to Materials Science	
\end{flushleft}


3	 0	 2	 4


	


\begin{flushleft}
and Engineering
\end{flushleft}


\begin{flushleft}
APL103	 Experimental Methods	
\end{flushleft}


3	 0	2	4


\begin{flushleft}
APL 105	 Mechanics of Solids and Fluids	
\end{flushleft}


3	 1	 0	 4


	


\begin{flushleft}
Total Credits				12
\end{flushleft}


\begin{flushleft}
Humanities and Social Sciences
\end{flushleft}


\begin{flushleft}
Courses from Humanities, Social Sciences and Management 	
\end{flushleft}


\begin{flushleft}
offered under this category				
\end{flushleft}





15





\begin{flushleft}
Departmental Core
\end{flushleft}


\begin{flushleft}
TXL110	 Polymer Chemistry	
\end{flushleft}


\begin{flushleft}
TXL111	 Textile Fibres	
\end{flushleft}


\begin{flushleft}
TXL211	 Structure and Physical Properties of Fibres	
\end{flushleft}


\begin{flushleft}
TXL212	 Manufactured Fibre Technology	
\end{flushleft}


\begin{flushleft}
TXP212	 Manufactured Fibre Technology Lab	
\end{flushleft}


\begin{flushleft}
TXL221	 Yarn Manufacture-I	
\end{flushleft}


\begin{flushleft}
TXP221	 Yarn Manufacture Laboratory-I	
\end{flushleft}


\begin{flushleft}
TXL222	 Yarn Manufacture-II	
\end{flushleft}


\begin{flushleft}
TXP222	 Yarn Manufacture Laboratory-II	
\end{flushleft}





3	 0	0	3


2	 0	2	3


3	 0	 0	 3


3	 0	0	3


0	 0	 2	 1


3	 0	0	3


0	 0	 2	 1


3	 0	0	3


0	 0	 2	 1





68





\newpage
69





\begin{flushleft}
Semester
\end{flushleft}





\begin{flushleft}
VIII
\end{flushleft}





\begin{flushleft}
VII
\end{flushleft}





\begin{flushleft}
VI
\end{flushleft}





\begin{flushleft}
V
\end{flushleft}





\begin{flushleft}
IV
\end{flushleft}





\begin{flushleft}
III
\end{flushleft}





\begin{flushleft}
II
\end{flushleft}





\begin{flushleft}
I
\end{flushleft}





3





3





0 0


\begin{flushleft}
TXL361
\end{flushleft}





3





3





3





3





0





2





0 0


\begin{flushleft}
DE 5 (4)
\end{flushleft}





0 0


\begin{flushleft}
DE 3 (3)
\end{flushleft}





4





3





3





\begin{flushleft}
Evaluation of Textile
\end{flushleft}


\begin{flushleft}
Materials
\end{flushleft}





3





\begin{flushleft}
Yarn Manufacture -- II
\end{flushleft}





0 0


\begin{flushleft}
TXL222
\end{flushleft}





3





0 0


\begin{flushleft}
TXL371
\end{flushleft}





3





3





3





3





0





0





3





0 0 3


\begin{flushleft}
OC 1 (3)
\end{flushleft}





1 0 4


\begin{flushleft}
DE 4 (3)
\end{flushleft}





\begin{flushleft}
Theory of Textile
\end{flushleft}


\begin{flushleft}
Structures
\end{flushleft}





3





\begin{flushleft}
Fabric Manufacture -- II
\end{flushleft}





0 0


\begin{flushleft}
TXL232
\end{flushleft}





3





0 2


\begin{flushleft}
TXL231
\end{flushleft}





3





2





\begin{flushleft}
Fabric Manufacture -- I
\end{flushleft}





4





\begin{flushleft}
Structure and Physical
\end{flushleft}


\begin{flushleft}
Properties of Fibres
\end{flushleft}





0 2


\begin{flushleft}
TXL211
\end{flushleft}





\begin{flushleft}
Textile Fibres
\end{flushleft}





3





3





\begin{flushleft}
Course-3
\end{flushleft}





\begin{flushleft}
Introduction to
\end{flushleft}


\begin{flushleft}
Chemistry
\end{flushleft}





0 0 3


\begin{flushleft}
CML100
\end{flushleft}





1 0 4


\begin{flushleft}
MTL101
\end{flushleft}





\begin{flushleft}
Calculus
\end{flushleft}





0





0 4 2


\begin{flushleft}
CMP100
\end{flushleft}





\begin{flushleft}
Physics Laboratory
\end{flushleft}





\begin{flushleft}
PYP100
\end{flushleft}





\begin{flushleft}
Course-5
\end{flushleft}





\begin{flushleft}
Linear Algebra and
\end{flushleft}


\begin{flushleft}
Chemistry Laboratory
\end{flushleft}


\begin{flushleft}
Differential Equations
\end{flushleft}





3





\begin{flushleft}
Course-4
\end{flushleft}





\begin{flushleft}
MTL100
\end{flushleft}





\begin{flushleft}
Course-6
\end{flushleft}





0





0





4





2





\begin{flushleft}
Product Realization
\end{flushleft}


\begin{flushleft}
through
\end{flushleft}


\begin{flushleft}
Manufacturing
\end{flushleft}





\begin{flushleft}
MCP101
\end{flushleft}





0





\begin{flushleft}
Course-7
\end{flushleft}





0





2





\begin{flushleft}
Introduction to
\end{flushleft}


\begin{flushleft}
Engineering
\end{flushleft}


\begin{flushleft}
(Non-graded)
\end{flushleft}





\begin{flushleft}
NIN100
\end{flushleft}





1





\begin{flushleft}
Course-8
\end{flushleft}





\begin{flushleft}
NLN100
\end{flushleft}





\begin{flushleft}
Course-9
\end{flushleft}





0 1 0.5


\begin{flushleft}
NEN100
\end{flushleft}





0





0


2


\begin{flushleft}
NLN100
\end{flushleft}





1





\begin{flushleft}
Professional Ethics and
\end{flushleft}


\begin{flushleft}
Language and Writing Skills-2
\end{flushleft}


\begin{flushleft}
Social Responsibility-2
\end{flushleft}


\begin{flushleft}
(Non-Graded)
\end{flushleft}


\begin{flushleft}
(Non-graded)
\end{flushleft}





0





\begin{flushleft}
Professional Ethics and
\end{flushleft}


\begin{flushleft}
Language and Writing Skills-1
\end{flushleft}


\begin{flushleft}
Social Responsibility-1
\end{flushleft}


\begin{flushleft}
(Non-Graded)
\end{flushleft}


\begin{flushleft}
(Non-graded)
\end{flushleft}





\begin{flushleft}
NEN100
\end{flushleft}





\begin{flushleft}
T
\end{flushleft}





\begin{flushleft}
P
\end{flushleft}





9.5 1 13 17.0 3 28.5





\begin{flushleft}
L
\end{flushleft}





\begin{flushleft}
Credits
\end{flushleft}





0 0


\begin{flushleft}
TXL241
\end{flushleft}





3





0 0


\begin{flushleft}
TXL242
\end{flushleft}





3





0 0


\begin{flushleft}
TXL372
\end{flushleft}





3





3





0





2





0





0





0 8


\begin{flushleft}
OC 2(3)
\end{flushleft}





\begin{flushleft}
B.Tech Project
\end{flushleft}





0 0


\begin{flushleft}
TXD411
\end{flushleft}





3





4





2





\begin{flushleft}
Speciality Yarns and
\end{flushleft}


\begin{flushleft}
Fabrics
\end{flushleft}





3





\begin{flushleft}
Technology of Textile
\end{flushleft}


\begin{flushleft}
Coloration
\end{flushleft}





3





\begin{flushleft}
CVL100
\end{flushleft}





3





0 2 4


\begin{flushleft}
SBL100
\end{flushleft}


0 2 4


\begin{flushleft}
TXL212
\end{flushleft}





3





3





3





3





0





2





4





0 0 3


\begin{flushleft}
OC 3 (4)
\end{flushleft}





0 0 3


\begin{flushleft}
HUL3XX
\end{flushleft}





0 0 3


\begin{flushleft}
DE 1 (3)
\end{flushleft}





\begin{flushleft}
Manufactured Fibre
\end{flushleft}


\begin{flushleft}
Technology
\end{flushleft}





3





0 0 2


\begin{flushleft}
TXL221
\end{flushleft}





3





3





3





3





0





0





3





1 0 4


\begin{flushleft}
DE 2 (3)
\end{flushleft}





1 0 4


\begin{flushleft}
HUL2XX
\end{flushleft}





0 0 3


\begin{flushleft}
HUL2XX
\end{flushleft}





\begin{flushleft}
Yarn Manufacture -- I
\end{flushleft}





2





\begin{flushleft}
Experimental Methods Environmental Science
\end{flushleft}





\begin{flushleft}
APL103
\end{flushleft}





\begin{flushleft}
Technology of Textile Introductory Biology
\end{flushleft}


\begin{flushleft}
Preparation \& Finishing
\end{flushleft}


\begin{flushleft}
for Engineers
\end{flushleft}





3





\begin{flushleft}
Polymer Chemistry
\end{flushleft}





\begin{flushleft}
TXL130
\end{flushleft}





1 0 4


\begin{flushleft}
TXP231
\end{flushleft}


0 2 1


\begin{flushleft}
TXP222
\end{flushleft}


0 2 1


\begin{flushleft}
TXP212
\end{flushleft}





0





0





2





1





\begin{flushleft}
Manufactured Fibre
\end{flushleft}


\begin{flushleft}
Technology Lab
\end{flushleft}





0





\begin{flushleft}
Yarn Manufacture
\end{flushleft}


\begin{flushleft}
Laboratory -- II
\end{flushleft}





0





\begin{flushleft}
Fabric Manufacture
\end{flushleft}


\begin{flushleft}
Laboratory -- I
\end{flushleft}





3





\begin{flushleft}
HUL2XX
\end{flushleft}





0 2


\begin{flushleft}
TXP221
\end{flushleft}


0 2


\begin{flushleft}
TXP232
\end{flushleft}





1





1





0 2


\begin{flushleft}
TXP361
\end{flushleft}





1





0





0





2





1





\begin{flushleft}
Evaluation of Textiles
\end{flushleft}


\begin{flushleft}
Lab
\end{flushleft}





0





\begin{flushleft}
Fabric Manufacture
\end{flushleft}


\begin{flushleft}
Laboratory -- II
\end{flushleft}





0





\begin{flushleft}
Yarn Manufacture
\end{flushleft}


\begin{flushleft}
Laboratory -- I
\end{flushleft}





0





\begin{flushleft}
Introduction to Textile
\end{flushleft}


\begin{flushleft}
Technology
\end{flushleft}


\begin{flushleft}
(Non-graded)
\end{flushleft}





\begin{flushleft}
TXN101
\end{flushleft}





0 3 1.5


\begin{flushleft}
TXP242
\end{flushleft}


0





0





3





1.5





\begin{flushleft}
Technology of Textile
\end{flushleft}


\begin{flushleft}
Coloration Lab
\end{flushleft}





0





\begin{flushleft}
Tech. for Tex. Prep. And
\end{flushleft}


\begin{flushleft}
Finishing Lab
\end{flushleft}





\begin{flushleft}
TXP241
\end{flushleft}


3





1





0





4





\begin{flushleft}
Mechanics of Solids and
\end{flushleft}


\begin{flushleft}
Fluids
\end{flushleft}





\begin{flushleft}
APL105
\end{flushleft}





\begin{flushleft}
TOTAL=145.0
\end{flushleft}





12.0 0 4 14.0 0 16.0 145.0





12.0 0 8 16.0 0 20.0





14.0 2 4 18.0 0 20.0





15.0 1 7 19.5 0 23.0





18.0 1 9 23.5 0 28.0





16.0 1 6 20.0 1 25.0





12 2 6 17.0 2 23.0


4 3 0 2 4 3 0 0 3 3 1 0 4 0 0 4 2


0 0 1 0.5 0


0


2


1


\begin{flushleft}
Note: Courses 1-6 above are attended in the given order by half of all first year students. The other half of First year students attend the Courses 1-6 of II semester first.
\end{flushleft}





\begin{flushleft}
Introduction to Materials
\end{flushleft}


\begin{flushleft}
Science and Engineering
\end{flushleft}





0





\begin{flushleft}
TXL111
\end{flushleft}





1





\begin{flushleft}
Introduction to
\end{flushleft}


\begin{flushleft}
Computer Science
\end{flushleft}





0.5 0 3 2


\begin{flushleft}
COL100
\end{flushleft}





\begin{flushleft}
APL102
\end{flushleft}





3





\begin{flushleft}
Engineering Mechanics
\end{flushleft}





4





\begin{flushleft}
Introduction to
\end{flushleft}


\begin{flushleft}
Engineering
\end{flushleft}


\begin{flushleft}
Visualization
\end{flushleft}





\begin{flushleft}
Introduction to Electrical
\end{flushleft}


\begin{flushleft}
Engineering
\end{flushleft}





0 2


\begin{flushleft}
APL100
\end{flushleft}





\begin{flushleft}
Electromagnetic Waves
\end{flushleft}


\begin{flushleft}
and Quantum
\end{flushleft}


\begin{flushleft}
Mechanics
\end{flushleft}





\begin{flushleft}
MCP100
\end{flushleft}





\begin{flushleft}
Course-1
\end{flushleft}





3





\begin{flushleft}
PYL100
\end{flushleft}





\begin{flushleft}
Course-2
\end{flushleft}





\begin{flushleft}
ELL100
\end{flushleft}





\begin{flushleft}
Non-Graded Units
\end{flushleft}





\begin{flushleft}
B.Tech. in Textile Technology	TT1
\end{flushleft}


\begin{flushleft}
Contact Hours
\end{flushleft}





\begin{flushleft}
\newpage
7. CAPABILITY-LINKED OPTIONS FOR
\end{flushleft}


\begin{flushleft}
UNDERGRADUATE STUDENTS
\end{flushleft}


\begin{flushleft}
As described in Section 4.8, students with CGPA higher than 7.0 and / or earned credits higher than 20 per semester
\end{flushleft}


\begin{flushleft}
are eligible to register for additional credits towards the following Capability-linked options. They can make use of
\end{flushleft}


\begin{flushleft}
these additional credits in two blocks of 20 credits to opt for
\end{flushleft}


	





\begin{flushleft}
(a)	
\end{flushleft}





\begin{flushleft}
Minor/Interdisciplinary Area Specialization
\end{flushleft}





	





\begin{flushleft}
(b)	
\end{flushleft}





\begin{flushleft}
Departmental Specialization
\end{flushleft}





\begin{flushleft}
A student based on his/her performance and interest can choose either one on both. Successful completion of
\end{flushleft}


\begin{flushleft}
minor area credits and / or Interdisciplinary / Departmental Specialization will be indicated on the degree.
\end{flushleft}


\begin{flushleft}
When a student opts for such a specialization and / or a minor area, he / she can use 10 open category (OC) credits
\end{flushleft}


\begin{flushleft}
(mandatory degree requirement) towards the specialization and/or minor area requirements. For example, a student
\end{flushleft}


\begin{flushleft}
registered for B.Tech (Chemical Engg.) and opting for minor area in Computer Science, can opt for courses prescribed
\end{flushleft}


\begin{flushleft}
for the minor area, as part of mandatory 10 credits requirements under OC. He/she will need to do additional 10
\end{flushleft}


\begin{flushleft}
credits for completing the Minor area requirements.
\end{flushleft}


\begin{flushleft}
A set of pre-defined courses of total 20 credits in a focus area comprises a Departmental Specialization if the
\end{flushleft}


\begin{flushleft}
courses belong to the parent Department of an undergraduate programme, or a Minor / Interdisciplinary Area
\end{flushleft}


\begin{flushleft}
Specialization if the courses belong to a different Department / Centre / School. Additional conditions and details
\end{flushleft}


\begin{flushleft}
are given in this section.
\end{flushleft}


\begin{flushleft}
If any course of a Minor / Interdisciplinary area overlaps with any core course (DC or PC category courses) or elective
\end{flushleft}


\begin{flushleft}
course (DE or PE category courses) of the student's programme, then credits from this course will not count towards
\end{flushleft}


\begin{flushleft}
the minor area credit requirements, though this course may contribute towards satisfying the requirement of the
\end{flushleft}


\begin{flushleft}
Minor / Interdisciplinary area. In such a case, the requirement of 20 credits must be completed by taking other courses
\end{flushleft}


\begin{flushleft}
of the Minor Area or Departmental / Interdisciplinary specialization. A student interested in opting for a Capabilitylinked option can register for the same online, on a first-come first served basis, after he / she completes at least
\end{flushleft}


\begin{flushleft}
2 courses, preferably the core courses (wherever applicable) of the Minor Area / Interdisciplinary / Departmental
\end{flushleft}


\begin{flushleft}
Specialization being applied for.
\end{flushleft}


\begin{flushleft}
Minor Area in Atmospheric Sciences (Centre for
\end{flushleft}


\begin{flushleft}
Atmospheric Sciences)
\end{flushleft}





\begin{flushleft}
SBP200	 Introduction to Practical Modern Biology	
\end{flushleft}


	





\begin{flushleft}
Minor Area Core
\end{flushleft}





\begin{flushleft}
Minor Area Electives
\end{flushleft}





\begin{flushleft}
ASL310	 Fundamentals of Atmosphere and Ocean	 3	 0	 2	 4
\end{flushleft}


\begin{flushleft}
ASL320	 Climate Change: Impacts, Adaptation and 	 3	 0	 2	 4
\end{flushleft}


\begin{flushleft}
	Mitigation
\end{flushleft}





\begin{flushleft}
SBD301	Mini Project	
\end{flushleft}


\begin{flushleft}
SBL701	 Biometry	
\end{flushleft}


\begin{flushleft}
SBL702	 Systems Biology	
\end{flushleft}


\begin{flushleft}
SBL704	 Human Virology	
\end{flushleft}


\begin{flushleft}
SBL707	 Bacterial Pathogenesis	
\end{flushleft}


\begin{flushleft}
SBL708	 Epigenetics in Health and Disease	
\end{flushleft}


\begin{flushleft}
SBL705	 Biology of Proteins	
\end{flushleft}


\begin{flushleft}
SBL703	 Advanced Cell Biology	
\end{flushleft}


\begin{flushleft}
SBL706	 Biologics	
\end{flushleft}


\begin{flushleft}
SBL709	 Marine Bioprospecting	
\end{flushleft}


\begin{flushleft}
SBL710	 Chemical Biology	
\end{flushleft}





	





\begin{flushleft}
Total Credits			8
\end{flushleft}





\begin{flushleft}
Minor Area Electives
\end{flushleft}


\begin{flushleft}
ASD330 	 Mini Project	
\end{flushleft}


0	 0	 12 6		


\begin{flushleft}
ASL410	 Numerical Simulation of Atmospheric	
\end{flushleft}


3	 0	 2	 4


	


\begin{flushleft}
and Oceanic Phenomena
\end{flushleft}


\begin{flushleft}
ASL733 	 Physics of the Atmosphere	
\end{flushleft}


3	 0	 0	 3


\begin{flushleft}
ASL734 	 Dynamics of the Atmosphere	
\end{flushleft}


3	 0	 0	 3


\begin{flushleft}
ASL735 	 Atmospheric Chemistry and Air Pollution	
\end{flushleft}


3	 0	 0	 3	


\begin{flushleft}
ASL736 	 Science of Climate Change	
\end{flushleft}


3	 0	 0	 3	


\begin{flushleft}
ASL737 	 Physical and Dynamical Oceanography	
\end{flushleft}


3	 0	 0	 3


\begin{flushleft}
ASL750 	 Boundary Layer Meteorology	
\end{flushleft}


3	 0	 0	 3		


\begin{flushleft}
ASL752 	 Mesoscale Meteorology	
\end{flushleft}


3	 0	 0	 3


\begin{flushleft}
ASL753 	Atmospheric Aerosols	
\end{flushleft}


3	0	0	3	


\begin{flushleft}
ASL754 	 Cloud Physics	
\end{flushleft}


3	 0	 0	 3			


\begin{flushleft}
ASL755 	 Remote Sensing of the Atmosphere and Ocean	3	0	0	3	
\end{flushleft}


\begin{flushleft}
ASL756 	 Synoptic Meteorology	
\end{flushleft}


3	 0	 0	 3			


\begin{flushleft}
ASL757 	 Tropical Weather and Climate	
\end{flushleft}


3	 0	 0	 3			


\begin{flushleft}
ASL758 	 General Circulation of the Atmosphere	
\end{flushleft}


3	 0	 0	 3		


\begin{flushleft}
ASL759 	 Land-Atmosphere Interactions	
\end{flushleft}


3	 0	 0	 3		


\begin{flushleft}
ASL760 	 Renewable Energy Meteorology	
\end{flushleft}


3	 0	 0	 3		


\begin{flushleft}
ASL761 	 Earth System Modelling	
\end{flushleft}


3	 0	 0	 3		


\begin{flushleft}
ASL762 	 Air-Sea Interaction	
\end{flushleft}


3	 0	 0	 3			


\begin{flushleft}
ASL763 	 Coastal Ocean and Estuarine Processes	 3	 0	 0	 3	
\end{flushleft}


\begin{flushleft}
ASL822 	 Climate Variability	
\end{flushleft}


3	 0	 0	 3		


\begin{flushleft}
ASL823 	 Geophysical Fluid Dynamics 	
\end{flushleft}


3	 0	 0	 3





0	0	6	3


3	0	0	3


3	0	0	3


3	0	0	3


3	0	0	3


3	 0	 0	 3


3	 0	 0	 3


3	 0	 0	 3


3	0	0	3


3	0	0	3


3	0	0	3





\begin{flushleft}
Minor Area in Business Management (Department of
\end{flushleft}


\begin{flushleft}
Management Studies)
\end{flushleft}


\begin{flushleft}
Minor Area Core (All four courses leading to 12 credits)
\end{flushleft}


\begin{flushleft}
MSL301	 Organizational \& People Management	
\end{flushleft}


3	0	0	3


\begin{flushleft}
MSL302	 Managerial Accounting \& Financial Management	3	0	0	3
\end{flushleft}


\begin{flushleft}
MSL303	Marketing Management	
\end{flushleft}


3	0	0	3


\begin{flushleft}
MSL304	Managing Operations	
\end{flushleft}


3	0	0	3


	





\begin{flushleft}
Total Credits				12
\end{flushleft}





\begin{flushleft}
Minor Area Electives (9 credits required)
\end{flushleft}


\begin{flushleft}
MSL704	
\end{flushleft}


\begin{flushleft}
MSL709	
\end{flushleft}


\begin{flushleft}
MSL710	
\end{flushleft}


\begin{flushleft}
MSL711	
\end{flushleft}


\begin{flushleft}
MSL712	
\end{flushleft}


\begin{flushleft}
MSL713	
\end{flushleft}


\begin{flushleft}
MSL714	
\end{flushleft}


\begin{flushleft}
MSL715	
\end{flushleft}


\begin{flushleft}
MSL716	
\end{flushleft}


\begin{flushleft}
MSL717	
\end{flushleft}


\begin{flushleft}
MSL719	
\end{flushleft}


\begin{flushleft}
MSL720	
\end{flushleft}





\begin{flushleft}
Minor Area in Biological Sciences (Kusuma School
\end{flushleft}


\begin{flushleft}
of Biological Sciences)
\end{flushleft}


\begin{flushleft}
Minor Area Core
\end{flushleft}


\begin{flushleft}
SBL201	 High-Dimensional Biology	
\end{flushleft}





0	 0	 4	 2





\begin{flushleft}
Total Credits				5
\end{flushleft}





3	0	0	3





70





\begin{flushleft}
Science \& Technology Policy Systems	
\end{flushleft}


3	 0	 0	 3


\begin{flushleft}
Business Research Methods	
\end{flushleft}


1.5	 0	 0	 1.5


\begin{flushleft}
Creative Problem Solving	
\end{flushleft}


3	 0	 0	 3


\begin{flushleft}
Strategic Management	
\end{flushleft}


3	 0	 0	 3


\begin{flushleft}
Ethics \& Values Based Leadership	
\end{flushleft}


1.5	 0	 0	 1.5


\begin{flushleft}
Information Systems Management	
\end{flushleft}


3	 0	 0	 3


\begin{flushleft}
Organizational Dynamics and Environment	 3	 0	 0	 3
\end{flushleft}


\begin{flushleft}
Quality and Environment Management Systems	 3	 0	0	 3
\end{flushleft}


\begin{flushleft}
Fundamentals of Management Systems	
\end{flushleft}


3	 0	 0	 3


\begin{flushleft}
Business Systems Analysis \& Design	
\end{flushleft}


3	 0	 0	 3


\begin{flushleft}
Statistics for Management	
\end{flushleft}


3	 0	 0	 3


\begin{flushleft}
Macroeconomic Environment of Business	 3	 0	 0	 3
\end{flushleft}





\begin{flushleft}
\newpage
Courses of Study 2017-2018
\end{flushleft}





\begin{flushleft}
MSL721	 Econometrics	
\end{flushleft}


3	 0	0	 3


\begin{flushleft}
MSL724	 Business Communication	
\end{flushleft}


1.5	 0	 0	 1.5


\begin{flushleft}
MSL725	 Business Negotiations	
\end{flushleft}


1.5	 0	 0	 1.5


\begin{flushleft}
MSL727	 Interpersonal Behavior \& Team Dynamics	 1.5	 0	 0	 1.5
\end{flushleft}


\begin{flushleft}
MSL729	 Individual Behavior in Organization	
\end{flushleft}


1.5	 0	 0	 1.5


\begin{flushleft}
MSL730	 Managing With Power	
\end{flushleft}


1.5	 0	 0	 1.5


\begin{flushleft}
MSL731	 Developing Self Awareness	
\end{flushleft}


1.5	 0	 0	 1.5


\begin{flushleft}
MSL733	 Organization Theory	
\end{flushleft}


1.5	0	0	 1.5


\begin{flushleft}
MSL734	 Management of Small \& Medium Scale 	
\end{flushleft}


3	 0	 0 3


	


\begin{flushleft}
Industrial Enterprises
\end{flushleft}


\begin{flushleft}
MSL740	 Quantitative Methods in Management	
\end{flushleft}


3	 0	 0	 3


\begin{flushleft}
MSL780	 Managerial Economics	
\end{flushleft}


1.5	 0	 0	 1.5


\begin{flushleft}
MSL801	Technology Forecasting \& Assessment	
\end{flushleft}


3	 0	 0	 3


\begin{flushleft}
MSL802	 Management of Intellectual Property Rights	 3	 0	 0	 3
\end{flushleft}


\begin{flushleft}
MSL804	 Procurement Management	
\end{flushleft}


3	 0	 0	 3


\begin{flushleft}
MSL805	 Services Operations Management	
\end{flushleft}


3	 0	 0	 3


\begin{flushleft}
MSL806	 Mergers \& Acquisitions	
\end{flushleft}


3	 0	 0	 3


\begin{flushleft}
MSL807	 Selected Topics in Strategic Management	 1	 0	 0	 1
\end{flushleft}


\begin{flushleft}
MSL808	 Systems Thinking 	
\end{flushleft}


3	 0	 0	 3


\begin{flushleft}
MSL809	 Cyber Security: Managing Risks	
\end{flushleft}


3	 0	 0	 3


\begin{flushleft}
MSL810	 Advanced Data Mining for Business Decisions	1.5	 0	 0	 1.5
\end{flushleft}


\begin{flushleft}
MSL811	 Management Control Systems	
\end{flushleft}


3	 0	 0	 3


\begin{flushleft}
MSL812	 Flexible Systems Management	
\end{flushleft}


3	 0	 0	 3


\begin{flushleft}
MSL813	 Systems Methodology for Management	
\end{flushleft}


3	 0	 0	 3


\begin{flushleft}
MSL814	 Data Visualization	
\end{flushleft}


1.5	 0	 0	 1.5


\begin{flushleft}
MSL815	 Decision Support and Expert Systems	
\end{flushleft}


3	 0	 0	 3


\begin{flushleft}
MSL816	 Total Quality Management	
\end{flushleft}


3	 0	 0	 3


\begin{flushleft}
MSL817	 Systems Waste \& Sustainability	
\end{flushleft}


3	 0	 0	 3


\begin{flushleft}
MSL818	 Industrial Waste Management	
\end{flushleft}


3	 0	 0	 3


\begin{flushleft}
MSL819	 Business Process Re-engineering	
\end{flushleft}


3	 0	 0	 3


\begin{flushleft}
MSL820	 Global Business Environment	
\end{flushleft}


3	 0	 0	 3


\begin{flushleft}
MSL821	 Strategy Execution Excellence	
\end{flushleft}


3	 0	 0	 3


\begin{flushleft}
MSL822	 International Business	
\end{flushleft}


3	 0	 0	 3


\begin{flushleft}
MSL823	 Strategic Change \& Flexibility	
\end{flushleft}


3	 0	 0	 3


\begin{flushleft}
MSL824	 Policy Dynamics \& Learning Organization	 3	 0	 0	 3
\end{flushleft}


\begin{flushleft}
MSL825	 Strategies in Functional Management	
\end{flushleft}


3	 0	 0	 3


\begin{flushleft}
MSL826	 Business Ethics	
\end{flushleft}


3	 0	 0	 3


\begin{flushleft}
MSL827	 International Competitiveness	
\end{flushleft}


3	 0	 0	 3


\begin{flushleft}
MSL828	 Global Strategic Management	
\end{flushleft}


3	 0	 0	 3


\begin{flushleft}
MSL829	 Current \& Emerging Issues in Strategic Management	3	 0	 0	 3
\end{flushleft}


\begin{flushleft}
MSL830	 Organizational Structure and Processes	
\end{flushleft}


3	 0	 0	 3


\begin{flushleft}
MSL831	 Management of Change	
\end{flushleft}


3	 0	 0	 3


\begin{flushleft}
MSL832	 Managing Innovation for Organizational 	
\end{flushleft}


3	 0	 0	 3


\begin{flushleft}
	Effectiveness
\end{flushleft}


\begin{flushleft}
MSL833	 Organizational Development	
\end{flushleft}


3	 0	 0	 3


\begin{flushleft}
MSL834	 Managing Diversity at Workplace	
\end{flushleft}


1.5	 0	 0	 1.5


\begin{flushleft}
MSL835	 Labor Legislation and Industrial Relations	 3	 0	 0	 3
\end{flushleft}


\begin{flushleft}
MSL836	 International Human Resources Management	 1.5	0	0	 1.5
\end{flushleft}


\begin{flushleft}
MSL839	 Current \& Emerging Issues in	
\end{flushleft}


3	 0	 0	 3


	


\begin{flushleft}
Organizational Management
\end{flushleft}


\begin{flushleft}
MSL840	 Manufacturing Strategy	
\end{flushleft}


3	 0	 0	 3


\begin{flushleft}
MSL841	 Supply Chain Analytics	
\end{flushleft}


3	 0	 0	 3


\begin{flushleft}
MSL842	 Supply Chain Modeling	
\end{flushleft}


3	 0	 0	 3


\begin{flushleft}
MSL843	 Supply Chain Logistics Management	
\end{flushleft}


3	 0	 0	 3


\begin{flushleft}
MSL844	 Systems Reliability, Safety and	
\end{flushleft}


3	 0	 0	 3


	


\begin{flushleft}
Maintenance Management
\end{flushleft}


\begin{flushleft}
MSL845	 Total Project Systems Management	
\end{flushleft}


3	 0	 0	 3


\begin{flushleft}
MSL846	 Total Productivity Management	
\end{flushleft}


3	 0	 0	 3


\begin{flushleft}
MSL847	 Advanced Methods for Management Research	 3	 0	 0 	 3
\end{flushleft}


\begin{flushleft}
MSL848	 Applied Operations Research	
\end{flushleft}


3	 0	 0	 3


\begin{flushleft}
MSL849	 Current \& Emerging Issues in	
\end{flushleft}


3	 0	 0	 3


	


\begin{flushleft}
Manufacturing Management
\end{flushleft}


\begin{flushleft}
MSL850	 Management of Information Technology	
\end{flushleft}


3	 0	 0	 3


\begin{flushleft}
MSL851	 Strategic Alliance	
\end{flushleft}


1.5	 0	 0	 1.5


\begin{flushleft}
MSL852	 Network System: Applications and Management	3	 0	 0	 3
\end{flushleft}


\begin{flushleft}
MSL853	 Software Project Management	
\end{flushleft}


3	 0	 0	 3


\begin{flushleft}
MSL854	 Big Data Analytics \& Data Science	
\end{flushleft}


1.5	 0	 0	 1.5


\begin{flushleft}
MSL855	 Electronic Commerce	
\end{flushleft}


3	 0	 0	 3


\begin{flushleft}
MSL856	 Business Intelligence	
\end{flushleft}


3	 0	 0	 3


\begin{flushleft}
MSL858	 Business Process Management with IT	
\end{flushleft}


1.5	 0	 0	 1.5





\begin{flushleft}
MSL859	 Current and Emerging Issues in IT Mgmt.	 3	 0	0	 3
\end{flushleft}


\begin{flushleft}
MSL861	 Market Research	
\end{flushleft}


3	 0	 0	 3


\begin{flushleft}
MSL862	 Product Management	
\end{flushleft}


3	 0	 0	 3


\begin{flushleft}
MSL863	 Advertising and Sales Promotion Mgmt.	
\end{flushleft}


3	 0	 0	 3


\begin{flushleft}
MSL864	 Corporate Communication	
\end{flushleft}


3	 0	 0	 3


\begin{flushleft}
MSL865	 Sales Management 	
\end{flushleft}


3	 0	 0	 3


\begin{flushleft}
MSL866	 International Marketing	
\end{flushleft}


3	 0	 0	 3


\begin{flushleft}
MSL867	 Industrial Marketing Management	
\end{flushleft}


3	 0	 0	 3


\begin{flushleft}
MSL868	 Digital Research Methods	
\end{flushleft}


1.5	 0	 0	 1.5


\begin{flushleft}
MSL869	 Current \& Emerging Issues in Marketing	
\end{flushleft}


3	 0	 0	 3


\begin{flushleft}
MSL870	 Corporate Governance	
\end{flushleft}


1.5	 0	 0	 1.5


\begin{flushleft}
MSL871	 Banking and Financial Services	
\end{flushleft}


1.5	 0	 0	 1.5


\begin{flushleft}
MSL872	 Working Capital Management 	
\end{flushleft}


3	 0	 0	 3


\begin{flushleft}
MSL873	Security Analysis \& Portfolio Management 	3	 0	 0	 3
\end{flushleft}


\begin{flushleft}
MSL874	 Indian Financial System	
\end{flushleft}


1.5	 0	 0	 1.5


\begin{flushleft}
MSL875	 International Financial Management	
\end{flushleft}


3	 0	 0	 3


\begin{flushleft}
MSL876	 Economics of Digital Business	
\end{flushleft}


1.5	 0	 0	 1.5


\begin{flushleft}
MSL877	 Electronic Government	
\end{flushleft}


1.5	 0	 0	 1.5


\begin{flushleft}
MSL878	Electronic Payments	
\end{flushleft}


1.5	 0	 0	 1.5


\begin{flushleft}
MSL879	 Current \& Emerging Issues in Finance 	
\end{flushleft}


3	 0	 0	 3


\begin{flushleft}
MSL880	 Selected Topics in Management Methodology	 3	 0	0	 3
\end{flushleft}


\begin{flushleft}
MSL881	 Management of Public Sector Enterprises	 3	 0	 0	 3
\end{flushleft}


	


\begin{flushleft}
in India
\end{flushleft}


\begin{flushleft}
MSL882	 Enterprise Cloud Computing	
\end{flushleft}


1.5	 0	 0	 1.5


\begin{flushleft}
MSL883	 ICTs, Development and Business	
\end{flushleft}


1.5	 0	 0	 1.5


\begin{flushleft}
MSL884	 Information System Strategy	
\end{flushleft}


3	 0	 0	 3


\begin{flushleft}
MSL885	 Digital Marketing-Analytics \& Optimization	 3	 0	 0	 3
\end{flushleft}


\begin{flushleft}
MSL886	 IT Consulting \& Practice	
\end{flushleft}


3	 0	 0	 3


\begin{flushleft}
MSL887	 Mobile Commerce	
\end{flushleft}


3	 0	 0	 3


\begin{flushleft}
MSL888	 Data Warehousing for Business Decision	
\end{flushleft}


1.5	 0	 0	 1.5


\begin{flushleft}
MSL889	 Current \& Emerging Issues in Public Sector	 3	 0	 0	 3
\end{flushleft}


\begin{flushleft}
	Management
\end{flushleft}


\begin{flushleft}
MSL891	 Data Analytics using SPSS	
\end{flushleft}


1.5	 0	 0	 1.5


\begin{flushleft}
MSL892	 Predictive Analytics	
\end{flushleft}


1.5	 0	 0	 1.5


\begin{flushleft}
MSL893	 Public Policy Issues in the Information Age	 1.5	 0	 0	 1.5
\end{flushleft}


\begin{flushleft}
MSL894	 Social Media \& Business Practices	
\end{flushleft}


3	 0	 0	 3


\begin{flushleft}
MSL895	 Advanced Data Analysis for Management 	 3	 0	 0 	 3
\end{flushleft}


\begin{flushleft}
MSL896	 International Economic Policy	
\end{flushleft}


3	 0	 0	 3


\begin{flushleft}
MSL897	 Consultancy Process \& Skills	
\end{flushleft}


3	 0	 0	 3


\begin{flushleft}
MSL898	 Consultancy Professional Practice	
\end{flushleft}


3	 0	 0 3


\begin{flushleft}
MSL899	 Current \& Emerging Issues in Consultancy	 3	 0	 0	 3
\end{flushleft}


\begin{flushleft}
	Management
\end{flushleft}


\begin{flushleft}
MTL732	 Financial Mathematics	
\end{flushleft}


4	 3	 1 0





\begin{flushleft}
Minor Area in Computational Mechanics (Department
\end{flushleft}


\begin{flushleft}
of Applied Mechanics)
\end{flushleft}


\begin{flushleft}
Minor Area Electives
\end{flushleft}


\begin{flushleft}
APD311	
\end{flushleft}


\begin{flushleft}
APL300	
\end{flushleft}


\begin{flushleft}
APL310	
\end{flushleft}


\begin{flushleft}
APL340	
\end{flushleft}


\begin{flushleft}
APL360	
\end{flushleft}


\begin{flushleft}
APL380	
\end{flushleft}


\begin{flushleft}
APL410	
\end{flushleft}


\begin{flushleft}
APL440	
\end{flushleft}


\begin{flushleft}
APL705	
\end{flushleft}


\begin{flushleft}
APL710	
\end{flushleft}


\begin{flushleft}
APL736	
\end{flushleft}





\begin{flushleft}
Project 	
\end{flushleft}


\begin{flushleft}
Computational Mechanics 	
\end{flushleft}


\begin{flushleft}
Constitutive Modelling 	
\end{flushleft}


\begin{flushleft}
Chaos 	
\end{flushleft}


\begin{flushleft}
Engineering Fluid Flows 	
\end{flushleft}


\begin{flushleft}
Biomechanics 	
\end{flushleft}


\begin{flushleft}
Computational Fluid Dynamics 	
\end{flushleft}


\begin{flushleft}
Parallel Processing in Computational Mechanics	
\end{flushleft}


\begin{flushleft}
Finite Element Method	
\end{flushleft}


\begin{flushleft}
Computer Aided Design	
\end{flushleft}


\begin{flushleft}
Multiscale Modelling of Crystalline Materials	
\end{flushleft}





0	


3	


3	


3	


3	


3	


3	


3	


3	


3	


3	





0	 8	 4


0	 2	 4


0	 2	 4


0	 2	 4


1	 0	 4


0	 2	 4


0	 2	 4


0	2	 4


0	 2	 4


0	 2	 4


0	2	 4





\begin{flushleft}
Minor Area in Materials Engineering (Department of
\end{flushleft}


\begin{flushleft}
Applied Mechanics)
\end{flushleft}


\begin{flushleft}
Minor Area Electives
\end{flushleft}


\begin{flushleft}
APD310	
\end{flushleft}


\begin{flushleft}
APL102	
\end{flushleft}


\begin{flushleft}
APL736	
\end{flushleft}


\begin{flushleft}
APL750	
\end{flushleft}


\begin{flushleft}
APL753	
\end{flushleft}


\begin{flushleft}
APL756	
\end{flushleft}


\begin{flushleft}
APL759	
\end{flushleft}


\begin{flushleft}
APL763	
\end{flushleft}


	





71





\begin{flushleft}
Mini Project 	
\end{flushleft}


\begin{flushleft}
Introducing to Materials Science	
\end{flushleft}


\begin{flushleft}
Multiscale Modelling of Crystalline Materials	
\end{flushleft}


\begin{flushleft}
Modern Engineering Materials 	
\end{flushleft}


\begin{flushleft}
Properties and Selection of Engineering Materials	
\end{flushleft}


\begin{flushleft}
Microstructural Characterization of Materials	
\end{flushleft}


\begin{flushleft}
Phase Transformations 	
\end{flushleft}


\begin{flushleft}
Micro \& Nanoscale Mechanical Behaviour	
\end{flushleft}


\begin{flushleft}
of Materials 	
\end{flushleft}





0	


3	


3	


3	


3	


3	


3	


3	





0	 6	 3


0	2	 4


0	2	 4


0	 0	 3


0	0	 3


0	2	 4


0	 0	 3


0	 2	 4





\begin{flushleft}
\newpage
Courses of Study 2017-2018
\end{flushleft}





\begin{flushleft}
APL764	 Mechanical Behaviour of Biomaterials 	
\end{flushleft}


3	 0	 0	 3


\begin{flushleft}
APL765	 Fracture Mechanics 	
\end{flushleft}


3	 0	 0	 3


\begin{flushleft}
APL767	 Engineering Failure Analysis and Prevention	 3	 0	 0	3
\end{flushleft}





\begin{flushleft}
COL812	
\end{flushleft}


\begin{flushleft}
COL818	
\end{flushleft}


\begin{flushleft}
COL819	
\end{flushleft}


\begin{flushleft}
COL821	
\end{flushleft}


\begin{flushleft}
COL829	
\end{flushleft}


\begin{flushleft}
COL830	
\end{flushleft}


\begin{flushleft}
COL831	
\end{flushleft}


\begin{flushleft}
COL832	
\end{flushleft}


\begin{flushleft}
COL851	
\end{flushleft}


\begin{flushleft}
COL852	
\end{flushleft}


\begin{flushleft}
COL860	
\end{flushleft}


\begin{flushleft}
COL861	
\end{flushleft}


\begin{flushleft}
COL862	
\end{flushleft}


\begin{flushleft}
COL863	
\end{flushleft}


\begin{flushleft}
COL864	
\end{flushleft}


\begin{flushleft}
COL865	
\end{flushleft}


\begin{flushleft}
COL866	
\end{flushleft}


\begin{flushleft}
COL867	
\end{flushleft}


\begin{flushleft}
COL868	
\end{flushleft}


\begin{flushleft}
COL869	
\end{flushleft}


\begin{flushleft}
COL870	
\end{flushleft}


\begin{flushleft}
COL871	
\end{flushleft}


	


\begin{flushleft}
COL872	
\end{flushleft}


\begin{flushleft}
COD891	
\end{flushleft}


\begin{flushleft}
COD892	
\end{flushleft}


\begin{flushleft}
COD893	
\end{flushleft}


\begin{flushleft}
COR310	
\end{flushleft}


\begin{flushleft}
COS310	
\end{flushleft}


\begin{flushleft}
COV877	
\end{flushleft}


\begin{flushleft}
COV878	
\end{flushleft}


\begin{flushleft}
COV879	
\end{flushleft}


\begin{flushleft}
COV880	
\end{flushleft}


\begin{flushleft}
COV881	
\end{flushleft}


\begin{flushleft}
COV882	
\end{flushleft}


\begin{flushleft}
COV883	
\end{flushleft}


\begin{flushleft}
COV884	
\end{flushleft}


\begin{flushleft}
COV885	
\end{flushleft}


\begin{flushleft}
COV886	
\end{flushleft}


\begin{flushleft}
COV887	
\end{flushleft}


\begin{flushleft}
COV888	
\end{flushleft}


\begin{flushleft}
COV889	
\end{flushleft}





\begin{flushleft}
Minor Area Non Departmental Electives in Material
\end{flushleft}


\begin{flushleft}
Science
\end{flushleft}


\begin{flushleft}
Minor Area Electives
\end{flushleft}


\begin{flushleft}
MCL336	
\end{flushleft}


\begin{flushleft}
MCL769	
\end{flushleft}


\begin{flushleft}
MCL780	
\end{flushleft}


\begin{flushleft}
MCL787	
\end{flushleft}


\begin{flushleft}
MCL791	
\end{flushleft}


\begin{flushleft}
PTL702	
\end{flushleft}





\begin{flushleft}
Advances in Wedding	
\end{flushleft}


\begin{flushleft}
Metal Forming Analysis	
\end{flushleft}


\begin{flushleft}
Casting Technology	
\end{flushleft}


\begin{flushleft}
Welding Science and Technology	
\end{flushleft}


\begin{flushleft}
Processing and Mechanics of Composites	
\end{flushleft}


\begin{flushleft}
Polymer Science and Technology	
\end{flushleft}





3	


3	


3	


3	


3	


3	





0	 2	 4


0	 2	 4


0	 2	 4


0	 2	 4


0	 2	 4


0	 0	3





\begin{flushleft}
Minor Area in Computer Science (Department of
\end{flushleft}


\begin{flushleft}
Computer Science and Engineering)
\end{flushleft}


\begin{flushleft}
Note : A student needs to do a minimum of three courses out of Minor
\end{flushleft}


\begin{flushleft}
Area Core and remaining courses from Minor Area Electives.
\end{flushleft}


\begin{flushleft}
Minor Area Core
\end{flushleft}


\begin{flushleft}
COL106	
\end{flushleft}


\begin{flushleft}
COL202	
\end{flushleft}


\begin{flushleft}
COL215	
\end{flushleft}


\begin{flushleft}
COL216	
\end{flushleft}


\begin{flushleft}
COL226	
\end{flushleft}


\begin{flushleft}
COP290	
\end{flushleft}


\begin{flushleft}
COL331	
\end{flushleft}


\begin{flushleft}
COL333	
\end{flushleft}


\begin{flushleft}
COL334	
\end{flushleft}


\begin{flushleft}
COL351	
\end{flushleft}


\begin{flushleft}
COL352	
\end{flushleft}


\begin{flushleft}
COL362	
\end{flushleft}


\begin{flushleft}
COL380	
\end{flushleft}


	





\begin{flushleft}
Data Structures and Algorithms	
\end{flushleft}


3	 0	 4	 5


\begin{flushleft}
Discrete Mathematical Structures 	
\end{flushleft}


3	 1	 0	 4


\begin{flushleft}
Digital Logic and System Design	
\end{flushleft}


3	 0	 4	 5


\begin{flushleft}
Computer Architecture	
\end{flushleft}


3	 0	 2	 4


\begin{flushleft}
Programming Languages	
\end{flushleft}


3	 0	 4	 5


\begin{flushleft}
Design Practices	
\end{flushleft}


0	 0	 6	 3


\begin{flushleft}
Operating Systems	
\end{flushleft}


3	 0	 4	 5


\begin{flushleft}
Principles of Artificial Intelligence*	
\end{flushleft}


3	 0	 2	 4


\begin{flushleft}
Computer Networks	
\end{flushleft}


3	 0	 2	 4


\begin{flushleft}
Analysis and Design of Algorithms	
\end{flushleft}


3	 1	 0	 4


\begin{flushleft}
Introduction to Automata and Theory of Computation	 3	 0	 0	3
\end{flushleft}


\begin{flushleft}
Introduction to Database Mgmt Systems*	
\end{flushleft}


3	 0	 2	 4


\begin{flushleft}
Introduction to Parallel and Distributed Programming	 2	 0	 2	3
\end{flushleft}


\begin{flushleft}
Total Credits (any three above courses)			 12-15
\end{flushleft}





\begin{flushleft}
Minor Area Electives
\end{flushleft}


\begin{flushleft}
COD300	
\end{flushleft}


\begin{flushleft}
COD310	
\end{flushleft}


\begin{flushleft}
COP315	
\end{flushleft}


\begin{flushleft}
COL341	
\end{flushleft}


\begin{flushleft}
COL718	
\end{flushleft}


\begin{flushleft}
COL719	
\end{flushleft}


\begin{flushleft}
COL722	
\end{flushleft}


\begin{flushleft}
COL724	
\end{flushleft}


\begin{flushleft}
COL726	
\end{flushleft}


\begin{flushleft}
COL728	
\end{flushleft}


\begin{flushleft}
COL729	
\end{flushleft}


\begin{flushleft}
COL730	
\end{flushleft}


\begin{flushleft}
COL732	
\end{flushleft}


\begin{flushleft}
COL733	
\end{flushleft}


\begin{flushleft}
COL740	
\end{flushleft}


\begin{flushleft}
COL750	
\end{flushleft}


\begin{flushleft}
COL751	
\end{flushleft}


\begin{flushleft}
COL752	
\end{flushleft}


\begin{flushleft}
COL753	
\end{flushleft}


\begin{flushleft}
COL754	
\end{flushleft}


\begin{flushleft}
COL756	
\end{flushleft}


\begin{flushleft}
COL757	
\end{flushleft}


\begin{flushleft}
COL758	
\end{flushleft}


\begin{flushleft}
COL759	
\end{flushleft}


\begin{flushleft}
COL760	
\end{flushleft}


\begin{flushleft}
COL762	
\end{flushleft}


\begin{flushleft}
COL765	
\end{flushleft}


\begin{flushleft}
COL768	
\end{flushleft}


\begin{flushleft}
COL770	
\end{flushleft}


\begin{flushleft}
COL772	
\end{flushleft}


\begin{flushleft}
COL774	
\end{flushleft}


\begin{flushleft}
COL776	
\end{flushleft}


\begin{flushleft}
COL780	
\end{flushleft}


\begin{flushleft}
COL781	
\end{flushleft}


\begin{flushleft}
COL783	
\end{flushleft}


\begin{flushleft}
COL786	
\end{flushleft}


\begin{flushleft}
COL788	
\end{flushleft}





\begin{flushleft}
Design Project (Non-Graded)	
\end{flushleft}


0	


\begin{flushleft}
Mini Project	
\end{flushleft}


0	


\begin{flushleft}
Embedded System Design Project	
\end{flushleft}


0	


\begin{flushleft}
Machine Learning	
\end{flushleft}


3	


\begin{flushleft}
Architecture of High Performance Computers	 3	
\end{flushleft}


\begin{flushleft}
Synthesis of Digital Systems	
\end{flushleft}


3	


\begin{flushleft}
Introduction to Compressed Sensing	
\end{flushleft}


3	


\begin{flushleft}
Advanced Computer Networks	
\end{flushleft}


3	


\begin{flushleft}
Numerical Algorithms	
\end{flushleft}


3	


\begin{flushleft}
Compiler Design	
\end{flushleft}


3	


\begin{flushleft}
Compiler Optimization	
\end{flushleft}


3	


\begin{flushleft}
Parallel Programming	
\end{flushleft}


3	


\begin{flushleft}
Virtualization and Cloud Computing	
\end{flushleft}


3	


\begin{flushleft}
Cloud Computing Technology Fundamentals	 3	
\end{flushleft}


\begin{flushleft}
Software Engineering	
\end{flushleft}


3	


\begin{flushleft}
Foundations of Automatic Verification	
\end{flushleft}


3	


\begin{flushleft}
Algorithmic Graph Theory	
\end{flushleft}


3	


\begin{flushleft}
Geometric Algorithms	
\end{flushleft}


3	


\begin{flushleft}
Complexity Theory	
\end{flushleft}


3	


\begin{flushleft}
Approximation Algorithms	
\end{flushleft}


3	


\begin{flushleft}
Mathematical Programming	
\end{flushleft}


3	


\begin{flushleft}
Model Centric Algorithm Design	
\end{flushleft}


3	


\begin{flushleft}
Advanced Algorithms	
\end{flushleft}


3	


\begin{flushleft}
Cryptography \& Computer Security	
\end{flushleft}


3	


\begin{flushleft}
Advanced Data Management	
\end{flushleft}


3	


\begin{flushleft}
Database Implementation	
\end{flushleft}


3	


\begin{flushleft}
Introduction to Logic and Functional Programming	 3	
\end{flushleft}


\begin{flushleft}
Wireless Networks	
\end{flushleft}


3	


\begin{flushleft}
Advanced Artificial Intelligence	
\end{flushleft}


3	


\begin{flushleft}
Natural Language Processing	
\end{flushleft}


3	


\begin{flushleft}
Machine Learning	
\end{flushleft}


3	


\begin{flushleft}
Learning Probabilistic Graphical Models	
\end{flushleft}


3	


\begin{flushleft}
Computer Vision	
\end{flushleft}


3	


\begin{flushleft}
Computer Graphics	
\end{flushleft}


3	


\begin{flushleft}
Digital Image Analysis	
\end{flushleft}


3	


\begin{flushleft}
Advanced Functional Brain Imaging	
\end{flushleft}


3	


\begin{flushleft}
Advanced Topics in Embedded Computing	 3	
\end{flushleft}





0	 4	 2


0	 6	 3


1	 6	 4


0	 2	 4


0	 2	4	


0	 2	 4


0	 0	 3


0	 2	 4


0	 2	 4


0	 3	 4.5


0	 3	 4.5


0	 2	 4


0	 2	 4


0	 2	4


0	 2	 4


0	 2	 4


0	 0	 3


0	 0	 3


0	 0	 3


0	 0	 3


0	 0	 3


0	 2	 4


0	 2	 4


0	 0	 3


0	 2	 4


0	 2	 4


0	 2	4


0	 2	 4


0	 2	 4


0	 2	 4


0	 2	 4


0	 2	 4


0	 2	 4


0	 3	 4.5


0	 3	 4.5


0	 2	 4


0	 0	 3





\begin{flushleft}
System Level Design and Modelling	
\end{flushleft}


3	


\begin{flushleft}
Principles of Multiprocessor Systems	
\end{flushleft}


3	


\begin{flushleft}
Advanced Distributed Systems	
\end{flushleft}


3	


\begin{flushleft}
Reconfigurable Computing	
\end{flushleft}


3	


\begin{flushleft}
Advanced Computer Graphics	
\end{flushleft}


3	


\begin{flushleft}
Distributed Computing	
\end{flushleft}


3	


\begin{flushleft}
Semantics of Programming Languages	
\end{flushleft}


3	


\begin{flushleft}
Proofs and Types	
\end{flushleft}


3	


\begin{flushleft}
Special Topics in Operating Systems	
\end{flushleft}


3	


\begin{flushleft}
Special Topics in Compilers	
\end{flushleft}


3	


\begin{flushleft}
Special Topics in Parallel Computation	
\end{flushleft}


3	


\begin{flushleft}
Special Topics in Hardware Systems	
\end{flushleft}


3	


\begin{flushleft}
Special Topics in Software Systems	
\end{flushleft}


3	


\begin{flushleft}
Special Topics in Theoretical Computer Science	 3	
\end{flushleft}


\begin{flushleft}
Special Topics in Artificial Intelligence	
\end{flushleft}


3	


\begin{flushleft}
Special Topics in Computer Applications	
\end{flushleft}


3	


\begin{flushleft}
Special Topics in Algorithms	
\end{flushleft}


3	


\begin{flushleft}
Special Topics in High Speed Networks	
\end{flushleft}


3	


\begin{flushleft}
Special Topics in Database Systems	
\end{flushleft}


3	


\begin{flushleft}
Special Topics in Concurrency	
\end{flushleft}


3	


\begin{flushleft}
Special Topics in Machine Learning	
\end{flushleft}


3	


\begin{flushleft}
Special Topics in Programming 	
\end{flushleft}


3	


\begin{flushleft}
Languages and Compilers
\end{flushleft}


\begin{flushleft}
Special Topics in Cryptography	
\end{flushleft}


3	


\begin{flushleft}
Minor Project	
\end{flushleft}


0	


\begin{flushleft}
M.Tech. Project Part-I	
\end{flushleft}


0	


\begin{flushleft}
M.Tech. Project Part- II	
\end{flushleft}


0	


\begin{flushleft}
Professional Practices (CS)	
\end{flushleft}


1	


\begin{flushleft}
Independent Study (CS)	
\end{flushleft}


0	


\begin{flushleft}
Special Module on Visual Computing	
\end{flushleft}


1	


\begin{flushleft}
Special Module in Machine Learning	
\end{flushleft}


1	


\begin{flushleft}
Special Module in Financial Algorithms	
\end{flushleft}


2	


\begin{flushleft}
Special Module in Parallel Computation	
\end{flushleft}


1	


\begin{flushleft}
Special Module in Hardware Systems	
\end{flushleft}


1	


\begin{flushleft}
Special Module in Software Systems	
\end{flushleft}


1	


\begin{flushleft}
Special Module in Theoretical Computer Science	1	
\end{flushleft}


\begin{flushleft}
Special Module in Artificial Intelligence	
\end{flushleft}


1	


\begin{flushleft}
Special Module in Computer Applications	
\end{flushleft}


1	


\begin{flushleft}
Special Module in Algorithms	
\end{flushleft}


1	


\begin{flushleft}
Special Module in High Speed Networks	
\end{flushleft}


1	


\begin{flushleft}
Special Module in Database Systems	
\end{flushleft}


1	


\begin{flushleft}
Special Module in Concurrency	
\end{flushleft}


1	





0	 0	 3


0	 2	 4


0	 2	 4


0	 0	 3


0	 2	 4


0	 0	 3


0	 0	 3


0	 0	 3


0	 0	 3


0	 0	 3


0	 0	 3


0	 0	 3


0	 0	 3	


0	 0	3


0	 0	 3


0	 0	 3


0	 0	 3


0	 0	 3


0	 0	 3


0	 0	 3


0	 0	 3


0	 0	 3


0	 0	 3


0	 6	 3


0	 14	7


0	 28	14


0	 2	 2


3	 0	 3


0	 0	 1


0	 0	 1


0	 0	 2	


0	 0	 1


0	 0	 1


0	 0	 1


0	 0	1


0	 0	 1


0	 0	 1


0	 0	 1


0	 0	 1


0	 0	 1


0	 0	 1





\begin{flushleft}
Minor Area in Cogeneration and Energy Efficiency
\end{flushleft}


\begin{flushleft}
(Centre for Energy Studies)
\end{flushleft}


\begin{flushleft}
Minor Area Core
\end{flushleft}


\begin{flushleft}
ESL748	
\end{flushleft}


\begin{flushleft}
ESL784	
\end{flushleft}


\begin{flushleft}
ESL785	
\end{flushleft}


	





\begin{flushleft}
Economics of Energy Conservation	
\end{flushleft}


3	 0	 0	


\begin{flushleft}
Cogeneration and Energy Efficiency	
\end{flushleft}


3	 0	 0	


\begin{flushleft}
Energy Analysis	
\end{flushleft}


3	 0	 0	


\begin{flushleft}
Total Credits				
\end{flushleft}





3


3	


3	


9





\begin{flushleft}
Minor Area Electives
\end{flushleft}


\begin{flushleft}
ESL714	
\end{flushleft}


\begin{flushleft}
ESL718	
\end{flushleft}


\begin{flushleft}
ESL722	
\end{flushleft}


\begin{flushleft}
ESL726	
\end{flushleft}


\begin{flushleft}
ESL776	
\end{flushleft}


\begin{flushleft}
ESL875	
\end{flushleft}


\begin{flushleft}
ESL786	
\end{flushleft}





\begin{flushleft}
Electrical Power Plant Engineering	
\end{flushleft}


\begin{flushleft}
Power Generation, Transmission and Distribution	
\end{flushleft}


\begin{flushleft}
Integrated Energy Systems	
\end{flushleft}


\begin{flushleft}
Waste Heat Recovery	
\end{flushleft}


\begin{flushleft}
Industrial Energy and Environmental Analysis	
\end{flushleft}


\begin{flushleft}
Alternative Fuels for Transportation	
\end{flushleft}


\begin{flushleft}
Exergy Analysis	
\end{flushleft}





3	


3	


3	


3	


3	


3	


3	





0	 0	 3	


0	 0	3	


0	 0	 3	


0	 0	 3	


0	 0	3	


0	 0	 3	


0	 0	 3	





\begin{flushleft}
Minor Area in Renewable Energy (Centre for Energy
\end{flushleft}


\begin{flushleft}
Studies)
\end{flushleft}


\begin{flushleft}
Minor Area Electives
\end{flushleft}


\begin{flushleft}
ESP713	
\end{flushleft}


\begin{flushleft}
ESL731	
\end{flushleft}


\begin{flushleft}
ESL732	
\end{flushleft}


\begin{flushleft}
ESL742 	
\end{flushleft}


	


\begin{flushleft}
ESL755	
\end{flushleft}


\begin{flushleft}
ESL768	
\end{flushleft}


\begin{flushleft}
ESL770	
\end{flushleft}





72





\begin{flushleft}
Energy Laboratories	
\end{flushleft}


\begin{flushleft}
Biomass - A Renewable Resource	
\end{flushleft}


\begin{flushleft}
Bioconversion and Processing of Waste	
\end{flushleft}


\begin{flushleft}
Economics and Financing of Renewable	
\end{flushleft}


\begin{flushleft}
Energy Systems
\end{flushleft}


\begin{flushleft}
Solar Photovoltaic Devices and Systems	
\end{flushleft}


\begin{flushleft}
Wind Energy and Hydro Power Systems	
\end{flushleft}


\begin{flushleft}
Solar Energy Utilization	
\end{flushleft}





0	


3	


3	


3	





0	


0	


0	


0	





6	


0	


0	


0	





3


3


3


3





3	 0	 0	 3


3	 0	 0	 3


3	 0	 0	 3





\begin{flushleft}
\newpage
Courses of Study 2017-2018
\end{flushleft}





\begin{flushleft}
ESL840	 Solar Architecture	
\end{flushleft}


\begin{flushleft}
ESL875	 Alternative Fuels for Transportation	
\end{flushleft}


\begin{flushleft}
ESL880	 Solar Thermal Power Generation	
\end{flushleft}





3	 0	 0	 3


3	 0	 0	 3


3	 0	 0	 3	





\begin{flushleft}
CLL774	
\end{flushleft}


\begin{flushleft}
CLL775	
\end{flushleft}


\begin{flushleft}
CLL776	
\end{flushleft}


\begin{flushleft}
CLL777	
\end{flushleft}





\begin{flushleft}
Minor Area in Technologies for Sustainable Rural
\end{flushleft}


\begin{flushleft}
Development (Centre for Rural Development and
\end{flushleft}


\begin{flushleft}
Technology)
\end{flushleft}


\begin{flushleft}
Biomass Production	
\end{flushleft}


\begin{flushleft}
Rural Resources and Livelihoods	
\end{flushleft}


\begin{flushleft}
Rural Energy Systems	
\end{flushleft}


\begin{flushleft}
Technologies for Water and Waste	
\end{flushleft}


\begin{flushleft}
Technology Alternatives for Rural Development	
\end{flushleft}


\begin{flushleft}
Food Quality and Safety	
\end{flushleft}





3	


3	


3	


3	


3	


3	





	





\begin{flushleft}
Total Credits				 9
\end{flushleft}


\begin{flushleft}
Rural Industrialization Policies	
\end{flushleft}


3	


\begin{flushleft}
Programmes and Cases
\end{flushleft}


\begin{flushleft}
Herbal, Medicinal and Aromatic Plants	
\end{flushleft}


3	


\begin{flushleft}
Technology for Utilization of Wastelands	
\end{flushleft}


3	


\begin{flushleft}
and Weeds
\end{flushleft}


\begin{flushleft}
Successful Forms of Grassroot Organizations	 3	
\end{flushleft}


\begin{flushleft}
Women, Technology and Development	
\end{flushleft}


3	


\begin{flushleft}
Minor Project	
\end{flushleft}


0	





\begin{flushleft}
CLD412	 Major Project in Energy and Environment	
\end{flushleft}


	


\begin{flushleft}
CLL705	
\end{flushleft}


\begin{flushleft}
CLL706	
\end{flushleft}


\begin{flushleft}
CLL720	
\end{flushleft}


\begin{flushleft}
CLL721	
\end{flushleft}


\begin{flushleft}
CLL722	
\end{flushleft}


\begin{flushleft}
CLL723	
\end{flushleft}


\begin{flushleft}
CLL724	
\end{flushleft}


	


\begin{flushleft}
CLL725	
\end{flushleft}


\begin{flushleft}
CLL726	
\end{flushleft}


\begin{flushleft}
CLL727	
\end{flushleft}


\begin{flushleft}
CLL728	
\end{flushleft}


\begin{flushleft}
CLL733	
\end{flushleft}


\begin{flushleft}
CLL734	
\end{flushleft}


\begin{flushleft}
CLL735	
\end{flushleft}


	


\begin{flushleft}
CLL736	
\end{flushleft}


	


\begin{flushleft}
CLL743	
\end{flushleft}


\begin{flushleft}
CLL768	
\end{flushleft}


	


\begin{flushleft}
CLL769	
\end{flushleft}


\begin{flushleft}
CLL793	
\end{flushleft}


\begin{flushleft}
CLL794	
\end{flushleft}





0	 0	 3


0	 0	 3


0	 0	 3


0	 0	3


0	 0	 3


0	 6	 3





\begin{flushleft}
Minor Area / Specialization Electives
\end{flushleft}





	





0	 0	 10	5		





\begin{flushleft}
Total Credits				 5
\end{flushleft}





\begin{flushleft}
Minor Area / Specialization Core
\end{flushleft}


\begin{flushleft}
CLL296	 Nano-engineering of Soft Materials	
\end{flushleft}


\begin{flushleft}
CLL730	 Structure, Transport and Reactions in	
\end{flushleft}


	


\begin{flushleft}
BioNano Systems
\end{flushleft}


\begin{flushleft}
CLL742	 Experimental Characterization of 	
\end{flushleft}


\begin{flushleft}
	BioMacromolecules
\end{flushleft}


\begin{flushleft}
CLL767	 Structures and Properties of Polymers 	
\end{flushleft}


\begin{flushleft}
CLL775	 Polymerization Process Modeling	
\end{flushleft}


\begin{flushleft}
CLL778	 Interfacial Behaviour and Transport 	
\end{flushleft}


	


\begin{flushleft}
of Biomolecules
\end{flushleft}


\begin{flushleft}
CLL779	 Molecular Biotechnology and 	
\end{flushleft}


	


\begin{flushleft}
in-vitro Diagnostics
\end{flushleft}


\begin{flushleft}
CLL780	 Bioprocessing and Bioseparations	
\end{flushleft}


\begin{flushleft}
CLL781	 Process Operations Scheduling	
\end{flushleft}


\begin{flushleft}
CLL786	 Fine Chemicals Technology	
\end{flushleft}


\begin{flushleft}
CLL791	 Chemical Product and Process Integration	
\end{flushleft}


\begin{flushleft}
CLL792	 Chemical Product Development and	
\end{flushleft}


\begin{flushleft}
	Commercialization
\end{flushleft}


\begin{flushleft}
CLL793	 Membrane Science and Engineering	
\end{flushleft}


\begin{flushleft}
SBL705	 Biology of Proteins	
\end{flushleft}





3	 0	 0	 3


3	 0	 0	 3





	





3	 0	 0	 3


3	 0	 0	 3





0	 0	 10	5	





\begin{flushleft}
Total Credits				 5
\end{flushleft}





\begin{flushleft}
Minor Area / Specialization Electives
\end{flushleft}


\begin{flushleft}
CLL296	
\end{flushleft}


\begin{flushleft}
CLL766	
\end{flushleft}


\begin{flushleft}
CLL767	
\end{flushleft}


\begin{flushleft}
CLL771	
\end{flushleft}


\begin{flushleft}
CLL772	
\end{flushleft}


\begin{flushleft}
CLL773	
\end{flushleft}





\begin{flushleft}
Nano-engineering of Soft Materials	
\end{flushleft}


\begin{flushleft}
Interfacial Engineering	
\end{flushleft}


\begin{flushleft}
Structures and Properties of Polymers 	
\end{flushleft}


\begin{flushleft}
Introduction to Complex Fluids	
\end{flushleft}


\begin{flushleft}
Transport Phenomena in Complex Fluids	
\end{flushleft}


\begin{flushleft}
Thermodynamics of Complex Fluids	
\end{flushleft}





3	


3	


3	


3	


3	


3	





0	


0	


0	


0	


0	


0	





0	


0	


0	


0	


0	


0	





3	


3	


3	


3	


3	


3	


3	





0	 0	 3


0	 0	 3


0	 0	 3


0	 0	 3


0	 0	3


0	 0	 3	


0	 0	 3





3	


3	


3	


3	


3	


3	


3	





0	 0	 3


0	 0	 3


0	 0	3


0	 0	 3


0	 0	 3


0	 0	 3


0	 0	 3





3	 0	 0	 3	


3	 0	 0	 3		


2	 0	 2	 3


2	 0	 2	3	


3	 0	 0	 3		


3	 0	 0	 3		





0	 0	 10	5





\begin{flushleft}
Total Credits				 5	
\end{flushleft}





\begin{flushleft}
CLL390	 Process Utilities and Pipeline Design	
\end{flushleft}


3	


\begin{flushleft}
CLL475	 Safety and Hazards in Process Industries	
\end{flushleft}


3	


\begin{flushleft}
CLL477	 Materials of Construction	
\end{flushleft}


3	


\begin{flushleft}
CLL707	 Population Balance Modeling	
\end{flushleft}


3	


\begin{flushleft}
CLL733	 Industrial Multiphase Reactors	
\end{flushleft}


3	


\begin{flushleft}
CLL734	 Process Intensification and Novel Reactors	 3	
\end{flushleft}


\begin{flushleft}
CLL735	 Design of Multicomponent Separation	
\end{flushleft}


3	


\begin{flushleft}
	Processes
\end{flushleft}


\begin{flushleft}
CLL736	 Experimental Characterization of 	
\end{flushleft}


3	


	


\begin{flushleft}
Multiphase Reactors
\end{flushleft}


\begin{flushleft}
CLL761	 Chemical Engineering Mathematics	
\end{flushleft}


3	


\begin{flushleft}
CLL762	 Advanced Computational Techniques in	
\end{flushleft}


2	


	


\begin{flushleft}
Chemical Engineering
\end{flushleft}


\begin{flushleft}
CLL768	 Fundamentals of Computational Fluid Dynamics	 2	
\end{flushleft}


\begin{flushleft}
CLL769	 Applications of Computational Fluid Dynamics	 2	
\end{flushleft}


\begin{flushleft}
CLL781	 Process Operations Scheduling	
\end{flushleft}


3	


\begin{flushleft}
CLL782	 Process Optimization	
\end{flushleft}


3	


\begin{flushleft}
CLL783	 Advanced Process Control	
\end{flushleft}


3	


\begin{flushleft}
CLL784	 Process Modeling and Simulation	
\end{flushleft}


3	


\begin{flushleft}
CLL785	 Evolutionary Optimization	
\end{flushleft}


3	


\begin{flushleft}
CLL791	 Chemical Product and Process Integration	 3	
\end{flushleft}


\begin{flushleft}
CLL792	 Chemical Product Development and 	
\end{flushleft}


3	


\begin{flushleft}
	Commercialization
\end{flushleft}


\begin{flushleft}
CLL793	 Membrane Science and Engineering	
\end{flushleft}


3	





\begin{flushleft}
Minor Area / Specialization Core
\end{flushleft}


	





\begin{flushleft}
Petroleum Reservoir Engineering	
\end{flushleft}


\begin{flushleft}
Petroleum Production Engineering	
\end{flushleft}


\begin{flushleft}
Principles of Electrochemical Engineering	
\end{flushleft}


\begin{flushleft}
Electrochemical Methods	
\end{flushleft}


\begin{flushleft}
Electrochemical Conversion and Storage Devices	
\end{flushleft}


\begin{flushleft}
Hydrogen Energy and Fuel Cell Technology	
\end{flushleft}


\begin{flushleft}
Environmental Engineering and 	
\end{flushleft}


\begin{flushleft}
Waste Management
\end{flushleft}


\begin{flushleft}
Air Pollution Control Engineering	
\end{flushleft}


\begin{flushleft}
Molecular Modeling of Catalytic Reactions	
\end{flushleft}


\begin{flushleft}
Heterogeneous Catalysis and Catalytic Reactors	
\end{flushleft}


\begin{flushleft}
Biomass Conversion and Utilization	
\end{flushleft}


\begin{flushleft}
Industrial Multiphase Reactors	
\end{flushleft}


\begin{flushleft}
Process Intensification and Novel Reactors	
\end{flushleft}


\begin{flushleft}
Design of Multicomponent 	
\end{flushleft}


\begin{flushleft}
Separation Processes
\end{flushleft}


\begin{flushleft}
Experimental Characterization of 	
\end{flushleft}


\begin{flushleft}
Multiphase Reactors
\end{flushleft}


\begin{flushleft}
Petrochemicals Technology	
\end{flushleft}


\begin{flushleft}
Fundamentals of Computational 	
\end{flushleft}


\begin{flushleft}
Fluid Dynamics
\end{flushleft}


\begin{flushleft}
Applications of Computational Fluid Dynamics	
\end{flushleft}


\begin{flushleft}
Membrane Science and Engineering	
\end{flushleft}


\begin{flushleft}
Petroleum Refinery Engineering	
\end{flushleft}





\begin{flushleft}
Minor Area / Specialization Electives
\end{flushleft}





3	


3	


3	


3	


3





\begin{flushleft}
Minor Area / Departmental Specialization in Complex
\end{flushleft}


\begin{flushleft}
Fluids and Materials (Department of Chemical
\end{flushleft}


\begin{flushleft}
Engineering)
\end{flushleft}


\begin{flushleft}
CLD413	 Major Project in Complex Fluids	
\end{flushleft}





\begin{flushleft}
Total Credits				 5	
\end{flushleft}





\begin{flushleft}
CLD414	 Major Project in Process Engineering,	
\end{flushleft}


	


\begin{flushleft}
Modeling and Optimization
\end{flushleft}





3	 0	 0	 3	


0	


0	


0	


0	


0	





0	 0	 10	5	





\begin{flushleft}
Minor Area / Specialization Core
\end{flushleft}





3	 0	 0	 3


3	 0	 0	 3


3	 0	 0	 3





0	


0	


0	


0	


0	





3	


3	


3	


3





\begin{flushleft}
Minor Area / Departmental Specialization in Process
\end{flushleft}


\begin{flushleft}
Engineering, Modelling and Optimization (Department
\end{flushleft}


\begin{flushleft}
of Chemical Engineering)
\end{flushleft}





3	 0	 0	 3





3	


3	


3	


3	


3	





0	


0	


0	


0	





\begin{flushleft}
Minor Area / Specialization Electives
\end{flushleft}





\begin{flushleft}
Minor Area / Departmental Specialization in
\end{flushleft}


\begin{flushleft}
Biopharmaceuticals and Fine Chemicals (Department
\end{flushleft}


\begin{flushleft}
of Chemical Engineering)
\end{flushleft}


\begin{flushleft}
CLD415	 Major Project in Biopharmaceuticals and 	
\end{flushleft}


	


\begin{flushleft}
Fine Chemicals
\end{flushleft}





0	


0	


0	


0	





\begin{flushleft}
Minor Area / Specialization Core
\end{flushleft}





0	 0	 3


0	 0	 3


0	 0	 3


0	 0	 3


0	 0	3


0	 0	 3





\begin{flushleft}
Minor Area Electives
\end{flushleft}


\begin{flushleft}
RDL701	
\end{flushleft}


	


\begin{flushleft}
RDL726	
\end{flushleft}


\begin{flushleft}
RDL740	
\end{flushleft}


	


\begin{flushleft}
RDL801	
\end{flushleft}


\begin{flushleft}
RDL807	
\end{flushleft}


\begin{flushleft}
RDD750	
\end{flushleft}





3	


3	


3	


3	





\begin{flushleft}
Minor Area / Departmental Specialization in Energy
\end{flushleft}


\begin{flushleft}
and Environment (Department of Chemical
\end{flushleft}


\begin{flushleft}
Engineering)
\end{flushleft}





\begin{flushleft}
Minor Area Core (Any three of the following courses)
\end{flushleft}


\begin{flushleft}
RDL700	
\end{flushleft}


\begin{flushleft}
RDL705	
\end{flushleft}


\begin{flushleft}
RDL722	
\end{flushleft}


\begin{flushleft}
RDL724	
\end{flushleft}


\begin{flushleft}
RDL730	
\end{flushleft}


\begin{flushleft}
RDL760	
\end{flushleft}





\begin{flushleft}
Simulation Techniques for Complex Fluids	
\end{flushleft}


\begin{flushleft}
Polymerization Process Modeling	
\end{flushleft}


\begin{flushleft}
Granular Materials	
\end{flushleft}


\begin{flushleft}
Complex Fluids Technology	
\end{flushleft}





3	


3	


3	


3	


3	


3	





73





0	


0	


0	


0	


0	


0	


0	





0	


0	


0	


0	


0	


0	


0	





3	


3


3


3


3			


3	


3





0	 0	 3	


0	 0	 3	


0	 2	 3	


0	 2	3


0	 2	3	


0	 0	 3	


0	 0	 3	


0	 0	 3		


0	 0	 3		


0	 0	 3		


0	 0	 3		


0	 0	 3


0	 0	 3	





\begin{flushleft}
\newpage
Courses of Study 2017-2018
\end{flushleft}





\begin{flushleft}
Minor Area / Departmental Specialization in Nanoscience and Technology (Department of Physics)
\end{flushleft}





\begin{flushleft}
Core 2
\end{flushleft}


\begin{flushleft}
JRL310	 Robotics Technology 	
\end{flushleft}


\begin{flushleft}
Core 3
\end{flushleft}


\begin{flushleft}
JRD301	 Mini Project in Robotics	
\end{flushleft}





\begin{flushleft}
Minor Area / Specialization Core
\end{flushleft}





\begin{flushleft}
PYL112	 Quantum Mechanics	
\end{flushleft}


3	 1	 0	 4


\begin{flushleft}
PYL201	 Fundamentals of Dielectrics \& Semiconductors	 3	 1	 0	4
\end{flushleft}


	


\begin{flushleft}
Total Credits				 8
\end{flushleft}





	





3	


3	


2	


2	


3	


3	


3	


2	


1	





0	 0	 3


0	 0	 3


0	 0	 2


0	 0	 2


0	 0	 3


0	 0	3


0	 0	 3


0	 0	 2


0	 0	 1





\begin{flushleft}
Minor Area / Specialization Core					
\end{flushleft}





\begin{flushleft}
PYL112	 Quantum Mechanics	
\end{flushleft}


3	 1	 0	 4	


\begin{flushleft}
PYL115	 Applied Optics	
\end{flushleft}


3	 1	 0	 4	


	


\begin{flushleft}
Total Credits				 8	
\end{flushleft}


\begin{flushleft}
Minor Area / Specialization Electives
\end{flushleft}


\begin{flushleft}
Lasers	
\end{flushleft}


\begin{flushleft}
Semiconductor Optoelectronics	
\end{flushleft}


\begin{flushleft}
Fourier Optics and Holography	
\end{flushleft}


\begin{flushleft}
Quantum Electronics	
\end{flushleft}


\begin{flushleft}
Ultrafast Laser Systems and Applications 	
\end{flushleft}


\begin{flushleft}
Fiber and Integrated Optics	
\end{flushleft}


\begin{flushleft}
Project III	
\end{flushleft}


\begin{flushleft}
Engineering Optics	
\end{flushleft}


\begin{flushleft}
Selected Topics in Photonics	
\end{flushleft}


\begin{flushleft}
Special Topics in Photonics	
\end{flushleft}





3	


3	


3	


3	


3	


3	


0	


3	


2	


1	





\begin{flushleft}
Interdisciplinary Specialization in Biodesign
\end{flushleft}





0	 0	3	


0	 0	 3	


0	 0	 3	


0	 0	 3	


0	 0	 3	


0	 0	 3	


0	 8	 4	


0	 0	 3	


0	 0	 2	


0	 0	 1





\begin{flushleft}
Specialization Core					
\end{flushleft}


\begin{flushleft}
BML741	 Medical Device Design	
\end{flushleft}


2	 0	 4	 4	


\begin{flushleft}
BMD742	 Minor Biodesign Project	
\end{flushleft}


0	 0	 8	 4	


	


\begin{flushleft}
Total Credits				 8	
\end{flushleft}


\begin{flushleft}
Specialization Electives					
\end{flushleft}


\begin{flushleft}
APL380	 Biomechanics 	
\end{flushleft}


3	


\begin{flushleft}
BML700	 Intro. to Basic Medical Sciences for Engineers	 3	
\end{flushleft}


\begin{flushleft}
BML710	 Industrial Biomaterial Technology	
\end{flushleft}


3	


\begin{flushleft}
BML720	 Medical Imaging	
\end{flushleft}


3	


\begin{flushleft}
BML735	 Biomedical Signal and Image processing	
\end{flushleft}


2	


\begin{flushleft}
BML736	 Application of Mathematics in Biomedical Engg.	2	
\end{flushleft}


\begin{flushleft}
BML743	 Special Topics in Biodesign	
\end{flushleft}


3	


\begin{flushleft}
BML750	 Point of Care Medical Diagnostic Devices	
\end{flushleft}


3	


\begin{flushleft}
BML770	 Fundamentals of Biomechanics 	
\end{flushleft}


3	


\begin{flushleft}
BML771	 Orthopaedic Device Design 	
\end{flushleft}


3	


\begin{flushleft}
BML772	 Biofabrication	
\end{flushleft}


3	


\begin{flushleft}
BML810	 Tissue Engineering	
\end{flushleft}


3	


\begin{flushleft}
BML820	Biomaterials	
\end{flushleft}


3	


\begin{flushleft}
BML830	 Biosensor Technology	
\end{flushleft}


3	


\begin{flushleft}
CLL779	 Molecular Biotechnology and in-vitro Diagnostics	 3	
\end{flushleft}


\begin{flushleft}
MCL442	 ThermoFluid Analysis of Biosystems	
\end{flushleft}


3	


\begin{flushleft}
TXL773	 Medical Textiles	
\end{flushleft}


3	





\begin{flushleft}
Interdisciplinary Specialization in Robotics
\end{flushleft}





0	 2	 4	


0	 0	3	


0	 0	 3	


0	 0	 3


0	 2	 3


0	 0	2


0	 0	 3	


0	 0	 3	


0	 0	 3	


0	 0	 3		


0	 0	 3		


0	 0	 3	


0	 0	3	


0	 2	 4		


0	 0	3	


0	 0	 3	


0	 0	 3	





\begin{flushleft}
Specialization Core					
\end{flushleft}


\begin{flushleft}
Core 1
\end{flushleft}


\begin{flushleft}
MCL111+	 Kinematics and Dynamics of Machines	
\end{flushleft}


3	 0	


\begin{flushleft}
MCL212\#	 Control Theory and Applications	
\end{flushleft}


3	 0	


\begin{flushleft}
ELL225\#	 Control Engineering-I	
\end{flushleft}


3	 1	


\begin{flushleft}
COP315*	Embedded System Design Project	
\end{flushleft}


0	 1	


\begin{flushleft}
ELL365*	 Embedded Systems	
\end{flushleft}


3	 0	


	


\begin{flushleft}
Total Credits			
\end{flushleft}





\begin{flushleft}
Total Credits			
\end{flushleft}





13/14





\begin{flushleft}
Specialization Electives
\end{flushleft}





\begin{flushleft}
Minor Area / Departmental Specialization in Photonics
\end{flushleft}


\begin{flushleft}
Technology (Department of Physics)
\end{flushleft}





\begin{flushleft}
PYL311	
\end{flushleft}


\begin{flushleft}
PYL312	
\end{flushleft}


\begin{flushleft}
PYL313	
\end{flushleft}


\begin{flushleft}
PYL411	
\end{flushleft}


\begin{flushleft}
PYL412	
\end{flushleft}


\begin{flushleft}
PYL413	
\end{flushleft}


\begin{flushleft}
PYD414	
\end{flushleft}


\begin{flushleft}
PYL414	
\end{flushleft}


\begin{flushleft}
PYV418	
\end{flushleft}


\begin{flushleft}
PYV419	
\end{flushleft}





0	 0	 14	7





\begin{flushleft}
Since the course may have pre-requisites, plan in advance.
\end{flushleft}


\begin{flushleft}
A student is required to complete (one of the core 1 course), (core 2 course)
\end{flushleft}


\begin{flushleft}
and (core 3 course).
\end{flushleft}





\begin{flushleft}
Minor Area / Specialization Electives
\end{flushleft}


\begin{flushleft}
PYL321	 Low Dimensional Physics	
\end{flushleft}


\begin{flushleft}
PYL322	 Nanoscale Fabrication	
\end{flushleft}


\begin{flushleft}
PYL323	 Nanoscale Microscopy	
\end{flushleft}


\begin{flushleft}
PYL324	 Spectroscopy of Nanomaterials	
\end{flushleft}


\begin{flushleft}
PYL421	 Functional Nanostructures	
\end{flushleft}


\begin{flushleft}
PYL422	 Spintronics	
\end{flushleft}


\begin{flushleft}
PYL423	 Nanoscale Energy Materials \& Devices	
\end{flushleft}


\begin{flushleft}
PYV428	 Selected Topics in Nanotechnology	
\end{flushleft}


\begin{flushleft}
PYV429	 Special Topics in Nanotechnology	
\end{flushleft}





3	 0	 0	 3





2	 4


2	 4


0	 4


6	 4


0	 3


13/14





\begin{flushleft}
*Students of ME1/ME2 to take only one of these courses as core.
\end{flushleft}


\begin{flushleft}
\#Students of CS1/CS5 to take only one of these courses as core.
\end{flushleft}


\begin{flushleft}
+Core for EE1/EE3 students only.
\end{flushleft}


\begin{flushleft}
Other Students can select any one of the Core one courses mentioned above.
\end{flushleft}





74





\begin{flushleft}
COL106	 Data Structures	
\end{flushleft}


3	


\begin{flushleft}
COL333	 Principles of Artificial Intelligence	
\end{flushleft}


3	


\begin{flushleft}
COL341	 Machine Learning	
\end{flushleft}


3	


\begin{flushleft}
COL351	 Analysis and Design of Algorithms	
\end{flushleft}


3	


\begin{flushleft}
COL671	 Artificial Intelligence	
\end{flushleft}


3	


\begin{flushleft}
COL740	 Software Engineering	
\end{flushleft}


3	


\begin{flushleft}
COL752	 Geometric Algorithms	
\end{flushleft}


3	


\begin{flushleft}
COL774	 Machine Learning	
\end{flushleft}


3	


\begin{flushleft}
COL770	 Advanced Artificial Intelligence	
\end{flushleft}


3	


\begin{flushleft}
COL780	 Computer Vision	
\end{flushleft}


3	


\begin{flushleft}
COL783	 Digital Image Analysis	
\end{flushleft}


3	


\begin{flushleft}
COL864	 Special Topics in Artificial Intelligence	
\end{flushleft}


3	


\begin{flushleft}
COL870	 Special Topics in Machine Learning	
\end{flushleft}


3	


\begin{flushleft}
ELL406	 Robotics and Automation 	
\end{flushleft}


3	


\begin{flushleft}
ELL409	 Machine Intelligence and Learning	
\end{flushleft}


3	


\begin{flushleft}
ELL703	 Optimal Control Theory	
\end{flushleft}


3	


\begin{flushleft}
ELL715	 Digital Image Processing	
\end{flushleft}


3	


\begin{flushleft}
ELL767	 Mechatronics	
\end{flushleft}


3	


\begin{flushleft}
ELL787	 Embedded Systems and Applications	
\end{flushleft}


3	


\begin{flushleft}
ELL791	 Neural Systems and Learning Machines	
\end{flushleft}


3	


\begin{flushleft}
ELL793	 Computer Vision	
\end{flushleft}


3	


\begin{flushleft}
ELL798	 Agent Technology	
\end{flushleft}


3	


\begin{flushleft}
MTL342	 Analysis and Design of Algorithms	
\end{flushleft}


3	


\begin{flushleft}
MTL509	 Numerical Analysis	
\end{flushleft}


3	


\begin{flushleft}
MTL729	 Computational Algebra and its Applications	 3	
\end{flushleft}


\begin{flushleft}
MTL744	 Mathematical Theory of Coding	
\end{flushleft}


3	


\begin{flushleft}
MTL811	 Mathematical Foundation of Artificial Intelligence	 3	
\end{flushleft}


\begin{flushleft}
MTL851	 Applied Numerical Analysis 	
\end{flushleft}


3	


\begin{flushleft}
MCL731	 Analytical Dynamics	
\end{flushleft}


3	


\begin{flushleft}
MCL738	 Dynamics of Multibody Systems	
\end{flushleft}


2	


\begin{flushleft}
MCL745	 Robotics	
\end{flushleft}


3	


\begin{flushleft}
MCL749	 Mechatronics Product Design	
\end{flushleft}


3	


\begin{flushleft}
MCL837	 Advanced Mechanisms	
\end{flushleft}


2	


\begin{flushleft}
MCL845	 Advanced Robotics	
\end{flushleft}


2	





0	 2	 4	


0	 2	 4	


0	 2	 4	


1	 0	 4


0	 2	 4


0	 2	 4	


0	 0	 3	


0	 2	 4	


0	 2	 4


0	 2	 4	


0	 3	 4.5


0	 0	 3	


0	 0	 3


0	 0	 3


0	 2	 4


0	 0	 3


0	 2	 4


0	 0	3


0	 0	 3


0	 2	 4


0	 0	 3


0	 0	 3


1	 0	 4


1	 0	 4


0	 0	 3


0	 0	 3


0	0	 3


0	 0	 3


0	 0	 3


0	 2	 3


0	2	 4


0	 2	 4


0	 2	 3


0	 2	 3





\begin{flushleft}
Departmental Specialization in Applications and
\end{flushleft}


\begin{flushleft}
Information Technology (Department of Computer
\end{flushleft}


\begin{flushleft}
Science and Engineering)
\end{flushleft}


\begin{flushleft}
Specialization Core					
\end{flushleft}


\begin{flushleft}
COD494	 B.Tech. Project Part-II	
\end{flushleft}


0	 0	 16	8	


\begin{flushleft}
COL703	 Logic for Computer Science	
\end{flushleft}


3	 0	 2	 4	


	


\begin{flushleft}
Total Credits				 12
\end{flushleft}


\begin{flushleft}
Specialization Electives
\end{flushleft}


\begin{flushleft}
COL333	
\end{flushleft}


\begin{flushleft}
COL362	
\end{flushleft}


\begin{flushleft}
COL722	
\end{flushleft}


\begin{flushleft}
COL757	
\end{flushleft}


\begin{flushleft}
COL760	
\end{flushleft}


\begin{flushleft}
COL762	
\end{flushleft}


\begin{flushleft}
COL765	
\end{flushleft}


	


\begin{flushleft}
COL770	
\end{flushleft}


\begin{flushleft}
COL786	
\end{flushleft}


\begin{flushleft}
COL865	
\end{flushleft}


\begin{flushleft}
COL869	
\end{flushleft}


\begin{flushleft}
COV885	
\end{flushleft}


\begin{flushleft}
COV888	
\end{flushleft}


\begin{flushleft}
COV889	
\end{flushleft}


\begin{flushleft}
SIL769	
\end{flushleft}


\begin{flushleft}
SIL801	
\end{flushleft}


\begin{flushleft}
SIL802	
\end{flushleft}


\begin{flushleft}
SIV813	
\end{flushleft}





\begin{flushleft}
Principles of Artificial Intelligence*	
\end{flushleft}


\begin{flushleft}
Introduction to Database Mgmt. Systems*	
\end{flushleft}


\begin{flushleft}
Introduction to Compressed Sensing	
\end{flushleft}


\begin{flushleft}
Model Centric Algorithm Design	
\end{flushleft}


\begin{flushleft}
Advanced Data Management	
\end{flushleft}


\begin{flushleft}
Database Implementation	
\end{flushleft}


\begin{flushleft}
Introduction to Logic and	
\end{flushleft}


\begin{flushleft}
Functional Programming
\end{flushleft}


\begin{flushleft}
Advanced Artificial Intelligence	
\end{flushleft}


\begin{flushleft}
Advanced Functional Brain Imaging	
\end{flushleft}


\begin{flushleft}
Special Topics in Computer Applications	
\end{flushleft}


\begin{flushleft}
Special Topics in Concurrency	
\end{flushleft}


\begin{flushleft}
Special Module in Computer Applications	
\end{flushleft}


\begin{flushleft}
Special Module in Database Systems	
\end{flushleft}


\begin{flushleft}
Special Module in Concurrency	
\end{flushleft}


\begin{flushleft}
Internet Traffic-Measurement, Modeling \& Analysis	
\end{flushleft}


\begin{flushleft}
Special Topics in Multimedia System	
\end{flushleft}


\begin{flushleft}
Special Topics in Web Based Computing	
\end{flushleft}


\begin{flushleft}
Applications of Computer in Medicines	
\end{flushleft}





3	


3	


3	


3	


3	


3	


3	





0	


0	


0	


0	


0	


0	


0	





2	


2	


0	


2	


2	


2	


2	





4	


4	


3	


4	


4	


4	


4	





3	


3	


3	


3	


1	


1	


1	


3	


3	


3	


1	





0	 2	 4	


0	 2	 4	


0	 0	 3	


0	 0	 3	


0	 0	 1	


0	 0	 1	


0	 0	 1	


0	 2	4	


0	 0	 3	


0	 0	 3	


0	 0	 1	





\begin{flushleft}
\newpage
Courses of Study 2017-2018
\end{flushleft}





\begin{flushleft}
SIV861	
\end{flushleft}


	


\begin{flushleft}
SIV864	
\end{flushleft}


\begin{flushleft}
SIV871	
\end{flushleft}


\begin{flushleft}
SIV889	
\end{flushleft}


	


\begin{flushleft}
SIV895	
\end{flushleft}





\begin{flushleft}
Information and Comm Technologies 	
\end{flushleft}


1	


\begin{flushleft}
for Development
\end{flushleft}


\begin{flushleft}
Special Module on Media Processing \& Communication	1	
\end{flushleft}


\begin{flushleft}
Special Module in Computational Neuroscience	1	
\end{flushleft}


\begin{flushleft}
Special Module in Human 	
\end{flushleft}


1	


\begin{flushleft}
Computer Interaction
\end{flushleft}


\begin{flushleft}
Special Module on Intelligent Information Processing	1	
\end{flushleft}





\begin{flushleft}
Departmental Specialization in Software Systems
\end{flushleft}


\begin{flushleft}
(Department of Computer Science and Engineering)
\end{flushleft}





0	 0	 1	


0	 0	 1	


0	 0	1


0	 0	 1





\begin{flushleft}
Specialization Core					
\end{flushleft}


\begin{flushleft}
COD494	 B.Tech. Project Part 2	
\end{flushleft}


\begin{flushleft}
COL703	 Logic for Computer Science	
\end{flushleft}





0	 0	 1





	





\begin{flushleft}
Departmental Specialization in Architecture and
\end{flushleft}


\begin{flushleft}
Embedded Systems (Department of Computer
\end{flushleft}


\begin{flushleft}
Science and Engineering)
\end{flushleft}





	





\begin{flushleft}
COL724	
\end{flushleft}


\begin{flushleft}
COL728	
\end{flushleft}


\begin{flushleft}
COL729	
\end{flushleft}


\begin{flushleft}
COL730	
\end{flushleft}


\begin{flushleft}
COL732	
\end{flushleft}


\begin{flushleft}
COL733	
\end{flushleft}


\begin{flushleft}
COL740	
\end{flushleft}


\begin{flushleft}
COL768	
\end{flushleft}


\begin{flushleft}
COL819	
\end{flushleft}


\begin{flushleft}
COL851	
\end{flushleft}


\begin{flushleft}
COL852	
\end{flushleft}


\begin{flushleft}
COL860	
\end{flushleft}


\begin{flushleft}
COL862	
\end{flushleft}


\begin{flushleft}
COL867	
\end{flushleft}


\begin{flushleft}
COL871	
\end{flushleft}


	


\begin{flushleft}
COV876	
\end{flushleft}


	


\begin{flushleft}
COV880	
\end{flushleft}


\begin{flushleft}
COV882	
\end{flushleft}


\begin{flushleft}
COV887	
\end{flushleft}


\begin{flushleft}
SIL765	
\end{flushleft}


\begin{flushleft}
SIL769	
\end{flushleft}


	





0	 0	 16	8	


3	 0	 2	 4	





\begin{flushleft}
Total Credits				 12
\end{flushleft}





\begin{flushleft}
Specialization Electives
\end{flushleft}


\begin{flushleft}
COP315	
\end{flushleft}


\begin{flushleft}
COL718	
\end{flushleft}


\begin{flushleft}
COL719	
\end{flushleft}


\begin{flushleft}
COL788	
\end{flushleft}


\begin{flushleft}
COL812	
\end{flushleft}


\begin{flushleft}
COL818	
\end{flushleft}


\begin{flushleft}
COL821	
\end{flushleft}


\begin{flushleft}
COL861	
\end{flushleft}


\begin{flushleft}
COV881	
\end{flushleft}





\begin{flushleft}
Embedded System Design Project	
\end{flushleft}


\begin{flushleft}
Architecture of High Performance Computers	
\end{flushleft}


\begin{flushleft}
Synthesis of Digital Systems	
\end{flushleft}


\begin{flushleft}
Advanced Topics in Embedded Computing	
\end{flushleft}


\begin{flushleft}
System Level Design and Modelling	
\end{flushleft}


\begin{flushleft}
Principles of Multiprocessor Systems	
\end{flushleft}


\begin{flushleft}
Reconfigurable Computing	
\end{flushleft}


\begin{flushleft}
Special Topics in Hardware Systems	
\end{flushleft}


\begin{flushleft}
Special Module in Hardware Systems	
\end{flushleft}





0	


3	


3	


3	


3	


3	


3	


3	


1	





1	 6	 4	


0	 2	4	


0	 2	 4	


0	 0	 3	


0	 0	 3	


0	 2	 4	


0	 0	 3	


0	 0	 3	


0	 0	 1	





\begin{flushleft}
Departmental Specialization in Data Analytics and
\end{flushleft}


\begin{flushleft}
Artificial Intelligence (Department of Computer
\end{flushleft}


\begin{flushleft}
Science and Engineering)
\end{flushleft}


\begin{flushleft}
Specialization Core
\end{flushleft}


\begin{flushleft}
COD494	 B.Tech. Project Part 2	
\end{flushleft}


\begin{flushleft}
COL703	 Logic for Computer Science	
\end{flushleft}


	





0	 0	 16	8	


3	 0	 2	 4	





\begin{flushleft}
Principles of Artificial Intelligence*	
\end{flushleft}


\begin{flushleft}
Machine Learning	
\end{flushleft}


\begin{flushleft}
Introduction to Database Mgmt. Systems*	
\end{flushleft}


\begin{flushleft}
Advanced Data Management	
\end{flushleft}


\begin{flushleft}
Database Implementation	
\end{flushleft}


\begin{flushleft}
Introduction to Logic and Functional Programming	
\end{flushleft}


\begin{flushleft}
Advanced Artificial Intelligence	
\end{flushleft}


\begin{flushleft}
Natural Language Processing	
\end{flushleft}


\begin{flushleft}
Machine Learning	
\end{flushleft}


\begin{flushleft}
Learning Probabilistic Graphical Models	
\end{flushleft}


\begin{flushleft}
Advanced Functional Brain Imaging	
\end{flushleft}


\begin{flushleft}
Special Topics in Artificial Intelligence	
\end{flushleft}


\begin{flushleft}
Special Topics in Database Systems	
\end{flushleft}


\begin{flushleft}
Special Topics in Concurrency	
\end{flushleft}


\begin{flushleft}
Special Topics in Machine Learning	
\end{flushleft}


\begin{flushleft}
Special Module in Machine Learning	
\end{flushleft}


\begin{flushleft}
Special Module in Artificial Intelligence	
\end{flushleft}


\begin{flushleft}
Special Module in Database Systems	
\end{flushleft}


\begin{flushleft}
Special Module in Concurrency	
\end{flushleft}





3	


3	


3	


3	


3	


3	


3	


3	


3	


3	


3	


3	


3	


3	


3	


1	


1	


1	


1	





\begin{flushleft}
COD494	 B.Tech. Project Part 2	
\end{flushleft}


\begin{flushleft}
COL703	 Logic for Computer Science	
\end{flushleft}





0	 2	 4	


0	 2	 4	


0	 2	 4	


0	 2	 4	


0	 2	 4	


0	 2	4	


0	 2	 4	


0	 2	 4	


0	 2	 4	


0	 2	 4	


0	 2	 4	


0	 0	 3	


0	 0	 3	


0	 0	 3	


0	 0	 3	


0	 0	 1	


0	 0	 1	


0	 0	 1	


0	 0	 1	





	


\begin{flushleft}
COL726	
\end{flushleft}


\begin{flushleft}
COL730	
\end{flushleft}


\begin{flushleft}
COL750	
\end{flushleft}


\begin{flushleft}
COL751	
\end{flushleft}


\begin{flushleft}
COL752	
\end{flushleft}


\begin{flushleft}
COL753	
\end{flushleft}


\begin{flushleft}
COL754	
\end{flushleft}


\begin{flushleft}
COL756	
\end{flushleft}


\begin{flushleft}
COL757	
\end{flushleft}


\begin{flushleft}
COL758	
\end{flushleft}


\begin{flushleft}
COL759	
\end{flushleft}


\begin{flushleft}
COL830	
\end{flushleft}


\begin{flushleft}
COL831	
\end{flushleft}


\begin{flushleft}
COL832	
\end{flushleft}


\begin{flushleft}
COL860	
\end{flushleft}


\begin{flushleft}
COL863	
\end{flushleft}


\begin{flushleft}
COL866	
\end{flushleft}


\begin{flushleft}
COL872	
\end{flushleft}


\begin{flushleft}
COV879	
\end{flushleft}


\begin{flushleft}
COV883	
\end{flushleft}


\begin{flushleft}
COV886	
\end{flushleft}





0	 0	 16	8	


3	 0	 2	 4	





\begin{flushleft}
Total Credits				 12
\end{flushleft}


\begin{flushleft}
Computer Vision	
\end{flushleft}


\begin{flushleft}
Computer Graphics	
\end{flushleft}


\begin{flushleft}
Digital Image Analysis	
\end{flushleft}


\begin{flushleft}
Advanced Computer Graphics	
\end{flushleft}


\begin{flushleft}
Special Module on Visual Computing	
\end{flushleft}


\begin{flushleft}
Special Topics in Multimedia System	
\end{flushleft}





1	 0	 0	 1


1	


1	


1	


3	


3	





0	


0	


0	


0	


0	





0	


0	


0	


2	


2	





1	


1	


1	


4	


4





3	


3	


3	


3	


1	


3	





0	


0	


0	


0	


0	


0	





2	


3	


3	


2	


0	


0	





0	 0	 16	8	


3	 0	 2	 4	





\begin{flushleft}
Total Credits				 12
\end{flushleft}


\begin{flushleft}
Numerical Algorithms	
\end{flushleft}


\begin{flushleft}
Parallel Programming	
\end{flushleft}


\begin{flushleft}
Foundations of Automatic Verification	
\end{flushleft}


\begin{flushleft}
Algorithmic Graph Theory	
\end{flushleft}


\begin{flushleft}
Geometric Algorithms	
\end{flushleft}


\begin{flushleft}
Complexity Theory	
\end{flushleft}


\begin{flushleft}
Approximation Algorithms	
\end{flushleft}


\begin{flushleft}
Mathematical Programming	
\end{flushleft}


\begin{flushleft}
Model Centric Algorithm Design	
\end{flushleft}


\begin{flushleft}
Advanced Algorithms	
\end{flushleft}


\begin{flushleft}
Cryptography \& Computer Security	
\end{flushleft}


\begin{flushleft}
Distributed Computing	
\end{flushleft}


\begin{flushleft}
Semantics of Programming Languages	
\end{flushleft}


\begin{flushleft}
Proofs and Types	
\end{flushleft}


\begin{flushleft}
Special Topics in Parallel Computation	
\end{flushleft}


\begin{flushleft}
Special Topics in Theoretical Computer Science	
\end{flushleft}


\begin{flushleft}
Special Topics in Algorithms	
\end{flushleft}


\begin{flushleft}
Special Topics in Cryptography	
\end{flushleft}


\begin{flushleft}
Special Module in Financial Algorithms	
\end{flushleft}


\begin{flushleft}
Special Module in Theoretical Computer Science	
\end{flushleft}


\begin{flushleft}
Special Module in Algorithms	
\end{flushleft}





3	


3	


3	


3	


3	


3	


3	


3	


3	


3	


3	


3	


3	


3	


3	


3	


3	


3	


2	


1	


1	





0	 2	 4	


0	 2	 4	


0	 2	 4	


0	 0	 3	


0	 0	 3	


0	 0	 3	


0	 0	 3	


0	 0	 3	


0	 2	 4	


0	 2	 4	


0	 0	 3	


0	 0	 3	


0	 0	 3	


0	 0	 3	


0	 0	 3	


0	 0	3	


0	 0	 3	


0	 0	 3	


0	 0	 2	


0	 0	1	


0	 0	 1





\begin{flushleft}
Departmental Specialization in Environmental
\end{flushleft}


\begin{flushleft}
Engineering (Department of Civil Engineering)
\end{flushleft}





\begin{flushleft}
Specialization Electives				
\end{flushleft}


\begin{flushleft}
COL780	
\end{flushleft}


\begin{flushleft}
COL781	
\end{flushleft}


\begin{flushleft}
COL783	
\end{flushleft}


\begin{flushleft}
COL829	
\end{flushleft}


\begin{flushleft}
COV877	
\end{flushleft}


\begin{flushleft}
SIL801	
\end{flushleft}





0	 2	 4	


0	 3	 4.5


0	 3	 4.5


0	 2	 4	


0	 2	 4	


0	 2	4	


0	 2	 4	


0	 2	 4	


0	 2	 4	


0	 0	 3	


0	 0	 3	


0	 0	 3	


0	 0	 3	


0	 0	 3	


0	 0	 3	





\begin{flushleft}
Specialization Electives					
\end{flushleft}





\begin{flushleft}
Specialization Core					
\end{flushleft}





	





3	


3	


3	


3	


3	


3	


3	


3	


3	


3	


3	


3	


3	


3	


3	





\begin{flushleft}
Specialization Core					
\end{flushleft}





\begin{flushleft}
Departmental Specialization in Graphics and Vision
\end{flushleft}


\begin{flushleft}
(Department of Computer Science and Engineering)
\end{flushleft}


\begin{flushleft}
COD494	 B.Tech. Project Part 2	
\end{flushleft}


\begin{flushleft}
COL703	 Logic for Computer Science	
\end{flushleft}





\begin{flushleft}
Advanced Computer Networks	
\end{flushleft}


\begin{flushleft}
Compiler Design	
\end{flushleft}


\begin{flushleft}
Compiler Optimization	
\end{flushleft}


\begin{flushleft}
Parallel Programming	
\end{flushleft}


\begin{flushleft}
Virtualization and Cloud Computing	
\end{flushleft}


\begin{flushleft}
Cloud Computing Technology Fundamentals	
\end{flushleft}


\begin{flushleft}
Software Engineering	
\end{flushleft}


\begin{flushleft}
Wireless Networks	
\end{flushleft}


\begin{flushleft}
Advanced Distributed Systems	
\end{flushleft}


\begin{flushleft}
Special Topics in Operating Systems	
\end{flushleft}


\begin{flushleft}
Special Topics in Compilers	
\end{flushleft}


\begin{flushleft}
Special Topics in Parallel Computation	
\end{flushleft}


\begin{flushleft}
Special Topics in Software Systems	
\end{flushleft}


\begin{flushleft}
Special Topics in High Speed Networks	
\end{flushleft}


\begin{flushleft}
Special Topics in Programming 	
\end{flushleft}


\begin{flushleft}
Languages and Compilers
\end{flushleft}


\begin{flushleft}
Special Module on Automated Reasoning	
\end{flushleft}


\begin{flushleft}
Methods for Program Analysis
\end{flushleft}


\begin{flushleft}
Special Module in Parallel Computation	
\end{flushleft}


\begin{flushleft}
Special Module in Software Systems	
\end{flushleft}


\begin{flushleft}
Special Module in High Speed Networks	
\end{flushleft}


\begin{flushleft}
Networks \& System Security	
\end{flushleft}


\begin{flushleft}
Internet Traffic -Measurement,	
\end{flushleft}


\begin{flushleft}
Modeling \& Analysis
\end{flushleft}





\begin{flushleft}
Departmental Specialization in Theoretical Computer
\end{flushleft}


\begin{flushleft}
Science (Dept. of Computer Science and Engineering)
\end{flushleft}





\begin{flushleft}
Total Credits				 12
\end{flushleft}





\begin{flushleft}
Specialization Electives
\end{flushleft}


\begin{flushleft}
COL333	
\end{flushleft}


\begin{flushleft}
COL341	
\end{flushleft}


\begin{flushleft}
COL362	
\end{flushleft}


\begin{flushleft}
COL760	
\end{flushleft}


\begin{flushleft}
COL762	
\end{flushleft}


\begin{flushleft}
COL765	
\end{flushleft}


\begin{flushleft}
COL770	
\end{flushleft}


\begin{flushleft}
COL772	
\end{flushleft}


\begin{flushleft}
COL774	
\end{flushleft}


\begin{flushleft}
COL776	
\end{flushleft}


\begin{flushleft}
COL786	
\end{flushleft}


\begin{flushleft}
COL864	
\end{flushleft}


\begin{flushleft}
COL868	
\end{flushleft}


\begin{flushleft}
COL869	
\end{flushleft}


\begin{flushleft}
COL870	
\end{flushleft}


\begin{flushleft}
COV878	
\end{flushleft}


\begin{flushleft}
COV884	
\end{flushleft}


\begin{flushleft}
COV888	
\end{flushleft}


\begin{flushleft}
COV889	
\end{flushleft}





\begin{flushleft}
Total Credits				 12	
\end{flushleft}





\begin{flushleft}
Specialization Electives					
\end{flushleft}





\begin{flushleft}
Specialization Core
\end{flushleft}


\begin{flushleft}
COD494	 B.Tech. Project Part-II	
\end{flushleft}


\begin{flushleft}
COL703	 Logic for Computer Science	
\end{flushleft}





0	 0	 16	8	


3	 0	 2	 4	





\begin{flushleft}
Specialization Core					
\end{flushleft}





4	


4.5


4.5


4	


1	


3





75





\begin{flushleft}
CVD412	
\end{flushleft}


\begin{flushleft}
CVL313	
\end{flushleft}


\begin{flushleft}
CVL721	
\end{flushleft}


\begin{flushleft}
CVL724	
\end{flushleft}





\begin{flushleft}
B.Tech. Project Part-II	
\end{flushleft}


\begin{flushleft}
Air and Noise Pollution	
\end{flushleft}


\begin{flushleft}
Solid Waste Engineering 	
\end{flushleft}


\begin{flushleft}
Environmental Systems Analysis 	
\end{flushleft}





0	


3	


3	


3	





0	


0	


0	


0	





12	6


0	 3	


0	 3	


2	 4	





	





\begin{flushleft}
Total Credits				 16
\end{flushleft}





\begin{flushleft}
\newpage
Courses of Study 2017-2018
\end{flushleft}





\begin{flushleft}
Departmental Specialization in Water Resources
\end{flushleft}


\begin{flushleft}
Engineering (Department of Civil Engineering)
\end{flushleft}





\begin{flushleft}
Specialization Electives (8 Credits)
\end{flushleft}


\begin{flushleft}
CVL311	
\end{flushleft}


\begin{flushleft}
CVL312	
\end{flushleft}


\begin{flushleft}
CVL727	
\end{flushleft}


\begin{flushleft}
CVL820	
\end{flushleft}


\begin{flushleft}
CVL822	
\end{flushleft}


	


\begin{flushleft}
CVL823	
\end{flushleft}


\begin{flushleft}
CVL824	
\end{flushleft}





\begin{flushleft}
Industrial Waste Management	
\end{flushleft}


3	


\begin{flushleft}
Environmental Assessment Methodologies	 3	
\end{flushleft}


\begin{flushleft}
Environmental Risk Assessment 	
\end{flushleft}


3	


\begin{flushleft}
Environmental Impact Assessment 	
\end{flushleft}


3	


\begin{flushleft}
Emerging Technologies for 	
\end{flushleft}


3	


\begin{flushleft}
Environmental Management
\end{flushleft}


\begin{flushleft}
Thermal Techniques for Waste Mgmt. 	
\end{flushleft}


3	


\begin{flushleft}
Life Cycle Analysis \& Design for Environment 	3	
\end{flushleft}





0	


0	


0	


0	


0	





0	


0	


0	


0	


0	





3	


3	


3


3	


3	





\begin{flushleft}
Specialization Core
\end{flushleft}





0	 0	 3	


0	 0	 3		





\begin{flushleft}
Departmental Specialization in Geotechnical
\end{flushleft}


\begin{flushleft}
Engineering (Department of Civil Engineering)
\end{flushleft}


\begin{flushleft}
B.Tech. Project Part-II	
\end{flushleft}


\begin{flushleft}
Ground Engineering	
\end{flushleft}


\begin{flushleft}
Rock Engineering	
\end{flushleft}


\begin{flushleft}
Soil Dynamics	
\end{flushleft}


\begin{flushleft}
Environmental Geotechniques and 	
\end{flushleft}


\begin{flushleft}
Geosynthetics
\end{flushleft}





0	


3	


3	


3	


3	





0	


0	


0	


0	


0	





	





\begin{flushleft}
Total Credits				 18
\end{flushleft}


\begin{flushleft}
Design of Foundations \& Retaining Structures	
\end{flushleft}


\begin{flushleft}
Stability of Slopes	
\end{flushleft}


\begin{flushleft}
FEM in Geotechnical Engineering	
\end{flushleft}


\begin{flushleft}
Geotechnical Design Studio	
\end{flushleft}


\begin{flushleft}
Underground Structures	
\end{flushleft}





3	


2	


3	


0	


2	





12	6


0	 3


0	 3


0	 3


0	 3





	


\begin{flushleft}
MCL322	
\end{flushleft}


\begin{flushleft}
MCL421	
\end{flushleft}


\begin{flushleft}
MCL422	
\end{flushleft}


\begin{flushleft}
MCL721	
\end{flushleft}


\begin{flushleft}
MCL722	
\end{flushleft}


\begin{flushleft}
MCL723	
\end{flushleft}


\begin{flushleft}
MCL724	
\end{flushleft}


\begin{flushleft}
MCL725	
\end{flushleft}


\begin{flushleft}
MCL726	
\end{flushleft}





0	


0	


0	


0	


0	


0	


0	


0	


0	


0	


0	





\begin{flushleft}
TXD402	
\end{flushleft}


\begin{flushleft}
TXL710	
\end{flushleft}


\begin{flushleft}
TXL719	
\end{flushleft}


\begin{flushleft}
TXL734	
\end{flushleft}


\begin{flushleft}
TXL740	
\end{flushleft}


\begin{flushleft}
TXL752	
\end{flushleft}


\begin{flushleft}
TXL773	
\end{flushleft}


\begin{flushleft}
TXL775	
\end{flushleft}


\begin{flushleft}
TXL776	
\end{flushleft}





	





\begin{flushleft}
Total Credits				 16
\end{flushleft}





0	 12	6


0	 2	3


0	 2	 3		


0	 2	 4





\begin{flushleft}
Specialization Electives (8 Credits)					
\end{flushleft}


\begin{flushleft}
CVL361	
\end{flushleft}


\begin{flushleft}
CVL461	
\end{flushleft}


\begin{flushleft}
CVL462	
\end{flushleft}


	


\begin{flushleft}
CVL743	
\end{flushleft}


\begin{flushleft}
CVL744	
\end{flushleft}


\begin{flushleft}
CVL746	
\end{flushleft}


\begin{flushleft}
CVL841	
\end{flushleft}


\begin{flushleft}
CVL842	
\end{flushleft}


\begin{flushleft}
CVL847	
\end{flushleft}





\begin{flushleft}
Introduction to Railway Engineering	
\end{flushleft}


\begin{flushleft}
Logistics and Freight Transport	
\end{flushleft}


\begin{flushleft}
Introduction to Intelligent 	
\end{flushleft}


\begin{flushleft}
Transportation Systems
\end{flushleft}


\begin{flushleft}
Airport Planning and Design 	
\end{flushleft}


\begin{flushleft}
Transportation Infrastructure Design	
\end{flushleft}


\begin{flushleft}
Public Transportation Systems 	
\end{flushleft}


\begin{flushleft}
Advanced Transportation Modelling 	
\end{flushleft}


\begin{flushleft}
Geometric Design of Roads 	
\end{flushleft}


\begin{flushleft}
Transportation Economics 	
\end{flushleft}





0	


0	


0	


0	


0	


0	





0	


0	


0	


2	


2	


0	





0	


0	


0	


2	


4	


2	


2	


0	





2	


2	


2	


3	


3	


3	


3		


3	





0	 0	 14	7	


3	 0	 2	 4	





\begin{flushleft}
Total Credits				 11
\end{flushleft}


\begin{flushleft}
Power Train Design	
\end{flushleft}


\begin{flushleft}
Automotive Structural Design	
\end{flushleft}


\begin{flushleft}
Design of Brake Systems	
\end{flushleft}


\begin{flushleft}
Automotive Prime Movers	
\end{flushleft}


\begin{flushleft}
Mechanical Design of Prime Mover Elements	
\end{flushleft}


\begin{flushleft}
Vehicle Dynamics	
\end{flushleft}


\begin{flushleft}
Biomechanics of Trauma in Automotive Design	
\end{flushleft}


\begin{flushleft}
Design Electronic Assist Systems in Automobiles	
\end{flushleft}


\begin{flushleft}
Design of Steering Systems	
\end{flushleft}





3	


2	


2	


3	


3	


3	


3	


3	


3	





0	 0	 3	


0	 2	 3	


0	 2	 3	


0	 0	 3	


0	 0	3	


0	 0	 3	


0	 0	3	


0	 0	3	


0	 0	 3





\begin{flushleft}
Major Project Part-II	
\end{flushleft}


0	


\begin{flushleft}
High Performance and Specialty Fibres	
\end{flushleft}


3	


\begin{flushleft}
Functional and Smart Textiles	
\end{flushleft}


3	


\begin{flushleft}
Nonwoven Science and Engineering	
\end{flushleft}


3	


\begin{flushleft}
Science \& App. of Nanotechnology in Textiles	3	
\end{flushleft}


\begin{flushleft}
Design of Functional Clothing	
\end{flushleft}


3	


\begin{flushleft}
Medical Textiles	
\end{flushleft}


3	


\begin{flushleft}
Technical Textiles	
\end{flushleft}


3	


\begin{flushleft}
Design \& Manuf. of Text. Reinforced Composites	3	
\end{flushleft}





0	 16	8	


0	 0	 3	


0	 0	 3	


0	 0	 3	


0	 0	3	


0	 0	 3	


0	 0	 3	


0	 0	 3	


0	 0	3





\begin{flushleft}
MCL756	
\end{flushleft}


\begin{flushleft}
MCL760	
\end{flushleft}


\begin{flushleft}
TXD402	
\end{flushleft}


\begin{flushleft}
TXL381	
\end{flushleft}


\begin{flushleft}
TXL781	
\end{flushleft}


\begin{flushleft}
TXL782	
\end{flushleft}


	


\begin{flushleft}
TXL783	
\end{flushleft}


\begin{flushleft}
TXV702	
\end{flushleft}





\begin{flushleft}
Supply Chain Management	
\end{flushleft}


\begin{flushleft}
Project Management	
\end{flushleft}


\begin{flushleft}
Major Project Part-II	
\end{flushleft}


\begin{flushleft}
Costing and its Application in Textiles	
\end{flushleft}


\begin{flushleft}
Costing, Project Formulation and Appraisal	
\end{flushleft}


\begin{flushleft}
Production and Operations 	
\end{flushleft}


\begin{flushleft}
Management in Textile Industry
\end{flushleft}


\begin{flushleft}
Design of Experiments and Statistical Techniques	
\end{flushleft}


\begin{flushleft}
Management of Textile Business	
\end{flushleft}





3	


3	


0	


3	


3	


3	





0	


0	


0	


1	


0	


0	





0	 3	


0	 3	


16	8	


0	 4	


0	 3	


0	 3	





3	 0	 0	3	


1	 0	 0	 1	





\begin{flushleft}
Departmental Specialization in Appliance Engineering
\end{flushleft}


\begin{flushleft}
(Department of Electrical Engineering)
\end{flushleft}





3	 0	 0	 3	


3	 0	 0	 3	


3	 0	 0	 3	


3	


3	


3	


2	


2	


3	





0	


0	


0	


0	


0	


0	


0	


0	





\begin{flushleft}
Specialization Electives					
\end{flushleft}





\begin{flushleft}
Specialization Core					
\end{flushleft}


0	


2	


2	


3	





2	


2	


2	


2	


1	


2	


2	


3	





\begin{flushleft}
Departmental Specialization in Textile Business
\end{flushleft}


\begin{flushleft}
Management (Department of Textile Technology)
\end{flushleft}





\begin{flushleft}
Departmental Specialization in Transportation
\end{flushleft}


\begin{flushleft}
Engineering (Department of Civil Engineering)
\end{flushleft}


\begin{flushleft}
B.Tech. Project Part-II	
\end{flushleft}


\begin{flushleft}
Pavement Materials and Design of Pavements	
\end{flushleft}


\begin{flushleft}
Urban and Regional Transportation Planning	
\end{flushleft}


\begin{flushleft}
Traffic Engineering 	
\end{flushleft}





2	 0	 2	 3	





\begin{flushleft}
Specialization Electives
\end{flushleft}





3


3


3


3


3


3


3


3


3


3


3	





\begin{flushleft}
CVD412	
\end{flushleft}


\begin{flushleft}
CVL740	
\end{flushleft}


\begin{flushleft}
CVL741	
\end{flushleft}


\begin{flushleft}
CVL742	
\end{flushleft}





\begin{flushleft}
Fundamentals of Geographic 	
\end{flushleft}


\begin{flushleft}
Information Systems
\end{flushleft}


\begin{flushleft}
Water Resources Systems	
\end{flushleft}


\begin{flushleft}
Urban Hydrology	
\end{flushleft}


\begin{flushleft}
Frequency Analysis in Hydrology	
\end{flushleft}


\begin{flushleft}
Fundamentals of Remote Sensing	
\end{flushleft}


\begin{flushleft}
Computational Aspects in Water Resources	
\end{flushleft}


\begin{flushleft}
River Mechanics	
\end{flushleft}


\begin{flushleft}
Geo-informatics	
\end{flushleft}


\begin{flushleft}
Mechanics of Sediment Transport 	
\end{flushleft}





\begin{flushleft}
Departmental Specialization in Technical and
\end{flushleft}


\begin{flushleft}
Innovative Textiles (Department of Textile Technology)
\end{flushleft}





2	 1	 0	 3	


0	


0	


0	


0	


0	


0	


0	


0	


0	


0	


0	





0	 2	


0	 3	


2	 3	


0	 2	


12	6





\begin{flushleft}
Specialization Electives					
\end{flushleft}





\begin{flushleft}
Specialization Electives (6 Credits)					
\end{flushleft}





3	


3	


3	


3	


3	


3	


3	


3	


3	


3	


3	





\begin{flushleft}
Total Credits				 16
\end{flushleft}





\begin{flushleft}
MCD412	 B.Tech. Project-II	
\end{flushleft}


\begin{flushleft}
MCL321	 Automotive Systems	
\end{flushleft}





\begin{flushleft}
B.Tech. Project Part-II	
\end{flushleft}


0	 0	 12	6


\begin{flushleft}
Structural Design	
\end{flushleft}


3	 0	 0	 3	


\begin{flushleft}
Structural Analysis-III	
\end{flushleft}


3	 0	 0	 3	


\begin{flushleft}
Prestressed Concrete \& Industrial Structures	 3	 0	 0	3	
\end{flushleft}


\begin{flushleft}
Solid Mechanics in Structural Engineering 	 3	 0	 0	 3
\end{flushleft}


\begin{flushleft}
Total Credits				 18
\end{flushleft}


\begin{flushleft}
Analytical and Numerical Methods	
\end{flushleft}


\begin{flushleft}
for Struct. Engineering
\end{flushleft}


\begin{flushleft}
Concrete Mechanics	
\end{flushleft}


\begin{flushleft}
Design of Bridge Structures	
\end{flushleft}


\begin{flushleft}
Design of Masonry Structures	
\end{flushleft}


\begin{flushleft}
Design of Tall Buildings	
\end{flushleft}


\begin{flushleft}
Prestressed and Composite Structures	
\end{flushleft}


\begin{flushleft}
Advanced Concrete Technology	
\end{flushleft}


\begin{flushleft}
Structural Safety and Reliability	
\end{flushleft}


\begin{flushleft}
Theory of Plates and Shells	
\end{flushleft}


\begin{flushleft}
Theory of Structural Stability	
\end{flushleft}


\begin{flushleft}
Design of Offshore Structures	
\end{flushleft}


\begin{flushleft}
Wind Resistant Design of Structures	
\end{flushleft}





0	


0	


0	


0	


0	





\begin{flushleft}
Specialization Core					
\end{flushleft}





\begin{flushleft}
Specialization Core					
\end{flushleft}





\begin{flushleft}
CVL763	
\end{flushleft}


	


\begin{flushleft}
CVL765	
\end{flushleft}


\begin{flushleft}
CVL766	
\end{flushleft}


\begin{flushleft}
CVL768	
\end{flushleft}


\begin{flushleft}
CVL769	
\end{flushleft}


\begin{flushleft}
CVL770	
\end{flushleft}


\begin{flushleft}
CVL771	
\end{flushleft}


\begin{flushleft}
CVL857	
\end{flushleft}


\begin{flushleft}
CVL858	
\end{flushleft}


\begin{flushleft}
CVL859	
\end{flushleft}


\begin{flushleft}
CVL862	
\end{flushleft}


\begin{flushleft}
CVL866	
\end{flushleft}





2	


3	


2	


2	


0	





\begin{flushleft}
Departmental Specialization in Automotive Design
\end{flushleft}


\begin{flushleft}
(Department of Mechanical Engineering)
\end{flushleft}





0	 0	3	


0	 0	 2


0	 0	 3	


0	 4	 2	


0	 0	 2	





\begin{flushleft}
Departmental Specialization in Structural Engineering
\end{flushleft}


\begin{flushleft}
(Department of Civil Engineering)
\end{flushleft}


\begin{flushleft}
CVD412	
\end{flushleft}


\begin{flushleft}
CVL441	
\end{flushleft}


\begin{flushleft}
CVL442	
\end{flushleft}


\begin{flushleft}
CVL443	
\end{flushleft}


\begin{flushleft}
CVL758	
\end{flushleft}


	





	


\begin{flushleft}
CVL284	
\end{flushleft}


	


\begin{flushleft}
CVL383	
\end{flushleft}


\begin{flushleft}
CVL384	
\end{flushleft}


\begin{flushleft}
CVL385	
\end{flushleft}


\begin{flushleft}
CVL386	
\end{flushleft}


\begin{flushleft}
CVP484	
\end{flushleft}


\begin{flushleft}
CVL485	
\end{flushleft}


\begin{flushleft}
CVL486	
\end{flushleft}


\begin{flushleft}
CVL837	
\end{flushleft}





\begin{flushleft}
Specialization Electives (6 Credits)					
\end{flushleft}


\begin{flushleft}
CVL431	
\end{flushleft}


\begin{flushleft}
CVL432	
\end{flushleft}


\begin{flushleft}
CVL433	
\end{flushleft}


\begin{flushleft}
CVL434	
\end{flushleft}


\begin{flushleft}
CVL435	
\end{flushleft}





\begin{flushleft}
Groundwater	
\end{flushleft}


\begin{flushleft}
Water Resources Management	
\end{flushleft}


\begin{flushleft}
Water Power Engineering	
\end{flushleft}


\begin{flushleft}
Groundwater \& Surface-water Pollution	
\end{flushleft}


\begin{flushleft}
B.Tech. Project Part-II	
\end{flushleft}





\begin{flushleft}
Specialization Electives (8 Credits)					
\end{flushleft}





\begin{flushleft}
Specialization Core					
\end{flushleft}


\begin{flushleft}
CVD412	
\end{flushleft}


\begin{flushleft}
CVL421	
\end{flushleft}


\begin{flushleft}
CVL422	
\end{flushleft}


\begin{flushleft}
CVL423	
\end{flushleft}


\begin{flushleft}
CVP424	
\end{flushleft}


	





\begin{flushleft}
CVL382	
\end{flushleft}


\begin{flushleft}
CVL481	
\end{flushleft}


\begin{flushleft}
CVL482	
\end{flushleft}


\begin{flushleft}
CVL483	
\end{flushleft}


\begin{flushleft}
CVD412	
\end{flushleft}





\begin{flushleft}
Specialization Electives						
\end{flushleft}


\begin{flushleft}
ELD451	
\end{flushleft}


\begin{flushleft}
ELL319	
\end{flushleft}


\begin{flushleft}
ELL365	
\end{flushleft}


\begin{flushleft}
ELL450	
\end{flushleft}


\begin{flushleft}
ELL754	
\end{flushleft}


\begin{flushleft}
ELL756	
\end{flushleft}


\begin{flushleft}
ELL762	
\end{flushleft}





3


3	


3	


3	


3	


3	





76





\begin{flushleft}
BTP Part-II	
\end{flushleft}


\begin{flushleft}
Digital Signal Processing	
\end{flushleft}


\begin{flushleft}
Embedded Systems	
\end{flushleft}


\begin{flushleft}
Special Topics in AE--I	
\end{flushleft}


\begin{flushleft}
Permanent Magnet Machines	
\end{flushleft}


\begin{flushleft}
Special Electrical Machines	
\end{flushleft}


\begin{flushleft}
Intelligent Motor Controllers	
\end{flushleft}





0	


3	


3	


3	


3	


3	


3	





0	


0	


0	


0	


0	


0	


0	





16	8


2	 4


0	 3


0	 3		


0	 3		


0	 3


0	 3





\begin{flushleft}
\newpage
Courses of Study 2017-2018
\end{flushleft}





\begin{flushleft}
ELL766	 Appliance System	
\end{flushleft}


\begin{flushleft}
ELL767	 Mechatronics	
\end{flushleft}


\begin{flushleft}
ELV750	 Special Modules in AE--I	
\end{flushleft}





3	 0	 0	 3	


3	 0	 0	 3		


1	 0	 0	 1





\begin{flushleft}
Departmental Specialization in Cognitive and
\end{flushleft}


\begin{flushleft}
Intelligent Systems (Department of Electrical Engg.)
\end{flushleft}


\begin{flushleft}
Specialization Electives						
\end{flushleft}


\begin{flushleft}
ELD457	 BTP Part-II 	
\end{flushleft}


0	 0	 16	8	


\begin{flushleft}
ELL409	 Machine Intelligence and Learning	
\end{flushleft}


3	 0	 2	 4


\begin{flushleft}
ELL457	 Special Topics in C\&IS -- I	
\end{flushleft}


3	 0	 0	 3


\begin{flushleft}
ELL704	 Advanced Robotics	
\end{flushleft}


3	0	0	3		


\begin{flushleft}
ELL707	 Systems Biology	
\end{flushleft}


3	0	0	3


\begin{flushleft}
ELL715	 Digital Image Processing	
\end{flushleft}


3	 0	 2	 4


\begin{flushleft}
ELL741	 Neuromorphic Engineering	
\end{flushleft}


3	0	0	3


\begin{flushleft}
ELL762	 Intelligent Motor Controllers	
\end{flushleft}


3	 0	 0	 3


\begin{flushleft}
ELL779	 Forecasting Techniques for Power Systems	 3	 0	 0	 3
\end{flushleft}


\begin{flushleft}
ELL784	 Introduction to Machine Learning	
\end{flushleft}


3	 0	 0	 3


\begin{flushleft}
ELL786	 Multimedia Systems	
\end{flushleft}


3	0	0	3


\begin{flushleft}
ELL788	 Computational Cognition and Perception	 3	 0	 0	 3	
\end{flushleft}


\begin{flushleft}
ELL789	 Intelligent Systems	
\end{flushleft}


3	0	0	3


\begin{flushleft}
ELL791	 Neural Systems and Learning Machines	
\end{flushleft}


3	 0	 2	 4


\begin{flushleft}
ELL793	 Computer Vision	
\end{flushleft}


3	0	0	3


\begin{flushleft}
ELL794	 Human-Computer Interface	
\end{flushleft}


3	0	0	3


\begin{flushleft}
ELL795	 Swarm Intelligence	
\end{flushleft}


3	0	0	3


\begin{flushleft}
ELL796	 Signals and Systems in Biology	
\end{flushleft}


3	0	0	3	


\begin{flushleft}
ELL798	 Agent Technology	
\end{flushleft}


3	0	0	3	


\begin{flushleft}
ELL799	 Natural Computing	
\end{flushleft}


3	0	0	3


\begin{flushleft}
ELV780	 Special Modules in Computers	
\end{flushleft}


1	 0	 0	 1	





\begin{flushleft}
ELL719	 Detection and Estimation Theory	
\end{flushleft}


\begin{flushleft}
ELL720	 Advanced Digital Signal Processing	
\end{flushleft}


\begin{flushleft}
ELL784	 Introduction to Machine Learning	
\end{flushleft}


\begin{flushleft}
ELL786	 Multimedia Systems	
\end{flushleft}


\begin{flushleft}
ELL793	 Computer Vision	
\end{flushleft}


\begin{flushleft}
ELL794	 Human-Computer Interface	
\end{flushleft}


\begin{flushleft}
ELV781	 Special Modules in IP -- I	
\end{flushleft}


\begin{flushleft}
CRL707	 Human and Machine Speech Communications	
\end{flushleft}





3	 0	 0	 3		


3	 0	 0	 3		


3	 0	 0	 3		


3	0	0	3		


3	0	0	3	


3	0	0	3	


1	 0	 0	 1		


3	0	0	3





\begin{flushleft}
Departmental Specialization in Nano-electronic and
\end{flushleft}


\begin{flushleft}
Photonic Systems (Department of Electrical Engg.)
\end{flushleft}


\begin{flushleft}
Specialization Electives						
\end{flushleft}


\begin{flushleft}
ELD456	 BTP Part-II 	
\end{flushleft}


\begin{flushleft}
ELL456	 Special Topics in NE\&PS -- I	
\end{flushleft}


\begin{flushleft}
ELL730	 IC Technology	
\end{flushleft}


\begin{flushleft}
ELL732	 Micro and Nanoelectronics	
\end{flushleft}


\begin{flushleft}
ELL737	 Flexible Electronics	
\end{flushleft}


\begin{flushleft}
ELL738	 Micro and Nano photonics	
\end{flushleft}


\begin{flushleft}
ELL739	 Advanced semiconductor devices	
\end{flushleft}


\begin{flushleft}
ELL740	 Compact Modeling of Semiconductor Devices	
\end{flushleft}


\begin{flushleft}
ELL741	 Neuromorphic Engineering	
\end{flushleft}


\begin{flushleft}
ELL742	 Introduction to MEMS Design	
\end{flushleft}


\begin{flushleft}
ELL743	 Photovoltaics	
\end{flushleft}


\begin{flushleft}
ELL744	 Electronic and Photonic Nanomaterials	
\end{flushleft}


\begin{flushleft}
ELL745	 Quantum Electronics	
\end{flushleft}


\begin{flushleft}
ELV731	 Special Modules in NE\&PS -- I	
\end{flushleft}





0	 0	 16	8


3	 0	 0	 3


3	0	0	3


3	 0	 0	 3


3	0	0	3


3	 0	 0	 3


3	 0	 0	 3


3	0	2	4


3	0	0	3		


3	 0	 0	 3


3	0	0	3


3	 0	 0	 3


3	0	0	3


1	 0	 0	 1





\begin{flushleft}
Departmental Specialization in Smart Grid and
\end{flushleft}


\begin{flushleft}
Renewable Energy (Department of Electrical Engg.)
\end{flushleft}





\begin{flushleft}
Departmental Specialization in Communication
\end{flushleft}


\begin{flushleft}
Systems and Networking (Dept. of Electrical Engg.)
\end{flushleft}





\begin{flushleft}
Specialization Electives						
\end{flushleft}


\begin{flushleft}
ELD452	 BTP Part-II 	
\end{flushleft}


\begin{flushleft}
ELL402	 Computer Communications	
\end{flushleft}


\begin{flushleft}
ELL417	 Renewable Energy Systems	
\end{flushleft}


\begin{flushleft}
ELL765	 Smart Grid Technology	
\end{flushleft}


\begin{flushleft}
ELL770	 Power System Analysis	
\end{flushleft}


\begin{flushleft}
ELL771	 Special Topics in SG\&RE -- I	
\end{flushleft}


\begin{flushleft}
ELL772	 Planning and Operation of Smart Grid	
\end{flushleft}


\begin{flushleft}
ELL773	 High Voltage DC Transmission	
\end{flushleft}


\begin{flushleft}
ELL774	 Flexible AC Transmission Systems	
\end{flushleft}


\begin{flushleft}
ELL775	 Power System Dynamics	
\end{flushleft}


\begin{flushleft}
ELL789	 Intelligent Systems	
\end{flushleft}


\begin{flushleft}
ELV451	 Special Modules in SG\&RE -- I	
\end{flushleft}





\begin{flushleft}
Specialization Electives						
\end{flushleft}


\begin{flushleft}
ELD458	 BTP Part-II 	
\end{flushleft}


0	 0	 16	8


\begin{flushleft}
ELL411	 Digital Communications	
\end{flushleft}


3	0	2	4	


\begin{flushleft}
ELL458	 Special Topics in CS\&N -- I	
\end{flushleft}


3	 0	 0	 3	


\begin{flushleft}
ELL713	 Microwave theory and techniques	
\end{flushleft}


3	 0	 0	 3


\begin{flushleft}
ELL714	 Basic Information Theory	
\end{flushleft}


3	0	0	3


\begin{flushleft}
ELL717	 Optical Communication Systems	
\end{flushleft}


3	 0	 0	 3


\begin{flushleft}
ELL723	 Broadband Communication Systems	
\end{flushleft}


3	 0	 0	 3


\begin{flushleft}
ELL725	 Wireless Communications	
\end{flushleft}


3	0	0	3	


\begin{flushleft}
ELL785	 Computer Communication Networks	
\end{flushleft}


3	 0	 0	 3	


\begin{flushleft}
ELV720	 Special Modules in CS\&N -- I	
\end{flushleft}


1	 0	 0	 1	





\begin{flushleft}
Departmental Specialization in Electric Transportation
\end{flushleft}


\begin{flushleft}
(Department of Electrical Engineering)
\end{flushleft}





0	 0	 16	8


3	0	0	3	


3	 0	 0	 3


3	 0	 0	 3	


3	0	0	3


3	 0	 0	 3	


3	 0	 0	 3	


3	 0	 0	 3	


3	0	0	3	


3	 0	 0	 3	


3	0	0	3	


1	 0	 0	 1





\begin{flushleft}
Departmental Specialization in Systems and Control
\end{flushleft}


\begin{flushleft}
(Department of Electrical Engineering)
\end{flushleft}





\begin{flushleft}
Specialization Electives						
\end{flushleft}


\begin{flushleft}
ELD454	 BTP Part-II 	
\end{flushleft}


0	 0	 16	8


\begin{flushleft}
ELL334	 DSP based Control of drives	
\end{flushleft}


3	 0	 2	 4	


\begin{flushleft}
ELL450	 Special Topics in AE -- I	
\end{flushleft}


3	 0	 0	 3	


\begin{flushleft}
ELL454	 Special Topics in ET -- I	
\end{flushleft}


3	 0	 0	 3	


\begin{flushleft}
ELL750	 Modeling of electrical machines	
\end{flushleft}


3	 0	 0	 3	


\begin{flushleft}
ELL754	 Permanent Magnet Machines	
\end{flushleft}


3	 0	 0	 3	


\begin{flushleft}
ELL755	 Variable Reluctance Machines	
\end{flushleft}


3	 0	 0	 3	


\begin{flushleft}
ELL764	 Electric vehicles	
\end{flushleft}


3	0	0	3	


\begin{flushleft}
ELV753	 Special Modules in ET -- I	
\end{flushleft}


1	 0	 0	 1





\begin{flushleft}
Specialization Electives						
\end{flushleft}


\begin{flushleft}
ELD450	 BTP Part-II	
\end{flushleft}


\begin{flushleft}
ELL436	 Digital Control	
\end{flushleft}


\begin{flushleft}
ELL700	 Linear Systems Theory	
\end{flushleft}


\begin{flushleft}
ELL702	 Nonlinear Systems	
\end{flushleft}


\begin{flushleft}
ELL703	 Optimal Control Theory	
\end{flushleft}


\begin{flushleft}
ELL704	 Advanced Robotics	
\end{flushleft}


\begin{flushleft}
ELL705	 Stochastic Filtering and Identification	
\end{flushleft}


\begin{flushleft}
ELL707	 Systems Biology	
\end{flushleft}


\begin{flushleft}
ELL708	 Selected Topics in Systems and Control	
\end{flushleft}


\begin{flushleft}
ELL762	 Intelligent Motor Controllers	
\end{flushleft}


\begin{flushleft}
ELV700	 Special Modules in Systems and Control 	
\end{flushleft}





\begin{flushleft}
Departmental Specialization in Energy-Efficient
\end{flushleft}


\begin{flushleft}
Technologies (Department of Electrical Engineering)
\end{flushleft}


\begin{flushleft}
Specialization Electives						
\end{flushleft}


\begin{flushleft}
ELD453	 BTP Part-II 	
\end{flushleft}


0	 0	 16	8


\begin{flushleft}
ELL408	 Low Power Circuit Design	
\end{flushleft}


3	 0	 0	 3	


\begin{flushleft}
ELL453	 Special Topics in EET -- I	
\end{flushleft}


3	 0	 0	 3	


\begin{flushleft}
ELL721	 Introduction to Telecommunication Systems	3	0	0	3
\end{flushleft}


\begin{flushleft}
ELL743	 Photovoltaics	
\end{flushleft}


3	0	0	3	


\begin{flushleft}
ELL757	 Energy Efficient Motors	
\end{flushleft}


3	 0	 0	 3	


\begin{flushleft}
ELL763	 Advanced Electrical Drives	
\end{flushleft}


3	 0	 0	 3	


\begin{flushleft}
ELL765	 Smart Grid Technology	
\end{flushleft}


3	 0	 0	 3	


\begin{flushleft}
ELL797	 Energy Efficient Computing	
\end{flushleft}


3	 0	 0	 3	


\begin{flushleft}
ELV752	 Special Modules in EET -- I	
\end{flushleft}


1	 0	 0	 1	





0	 0	 16	8


3	0	0	3


3	0	0	3


3	0	0	3


3	0	0	3


3	0	0	3


3	 0	 0	 3


3	0	0	3


3	 0	 0	 3


3	 0	 0	 3


1	 0	 0	 1	





\begin{flushleft}
Departmental Specialization in VLSI and Embedded
\end{flushleft}


\begin{flushleft}
Systems (Department of Electrical Engineering)
\end{flushleft}


\begin{flushleft}
Specialization Electives						
\end{flushleft}


\begin{flushleft}
ELD455	 BTP Part-II 	
\end{flushleft}


\begin{flushleft}
ELL365	 Embedded Systems	
\end{flushleft}


\begin{flushleft}
ELL455	 Special Topics in V\&ES -- I	
\end{flushleft}


\begin{flushleft}
ELL720	 Advanced Digital Signal Processing	
\end{flushleft}


\begin{flushleft}
ELL730	 IC Technology	
\end{flushleft}


\begin{flushleft}
ELL731	 Mixed signal circuit design	
\end{flushleft}


\begin{flushleft}
ELL733	 Digital ASIC Design	
\end{flushleft}


\begin{flushleft}
ELL734	 MOS VLSI Design	
\end{flushleft}


\begin{flushleft}
ELL735	 Analog Integrated Circuits	
\end{flushleft}


\begin{flushleft}
ELL736	 Solid State Imaging Sensors	
\end{flushleft}


\begin{flushleft}
ELL740	 Compact Modeling of Semiconductor Devices	
\end{flushleft}


\begin{flushleft}
ELL741	 Neuromorphic Engineering	
\end{flushleft}


\begin{flushleft}
ELL747	 Active and Passive Filter Design	
\end{flushleft}


\begin{flushleft}
ELL748	 System-on-Chip Design and Test	
\end{flushleft}


\begin{flushleft}
ELL749	 Semiconductor Memory Design	
\end{flushleft}


\begin{flushleft}
ELV730	 Special Modules in V\&ES -- I	
\end{flushleft}





\begin{flushleft}
Departmental Specialization in Information
\end{flushleft}


\begin{flushleft}
Processing (Department of Electrical Engineering)
\end{flushleft}


\begin{flushleft}
Specialization Electives						
\end{flushleft}


\begin{flushleft}
ELD459	 BTP Part-II 	
\end{flushleft}


0	 0	 16	8	


\begin{flushleft}
ELL459	 Special Topics in IP -- I	
\end{flushleft}


3	 0	 0	 3	


\begin{flushleft}
ELL460	 Special Topics in IP -- II	
\end{flushleft}


3	 0	 0	 3	


\begin{flushleft}
ELL714	 Basic Information Theory	
\end{flushleft}


3	0	0	3		


\begin{flushleft}
ELL715	 Digital Image processing	
\end{flushleft}


3	 0	 2	 4		


\begin{flushleft}
ELL718	 Statistical signal processing	
\end{flushleft}


3	 0	 0	 3		





77





0	 0	 16	8


3	0	0	3		


3	 0	 0	 3


3	 0	 0	 3	


3	0	0	3	


3	 0	 0	 3	


3	0	2	4


3	 0	 0	 3	


3	 0	 0	 3	


3	 0	 0	 3	


3	0	2	4


3	0	0	3	


3	 0	 0	 3	


3	 0	 0	 3	


3	 0	 0	 3	


1	 0	 0	 1





\begin{flushleft}
\newpage
Courses of Study 2017-2018
\end{flushleft}





\begin{flushleft}
8. NON-GRADED CORE FOR UNDERGRADUATE STUDENTS
\end{flushleft}


\begin{flushleft}
In order to synergize formal academics with informal outside-class-room learning experience, mechanisms for
\end{flushleft}


\begin{flushleft}
earning non-graded units have been introduced in the undergraduate curriculum in 2013. In order to earn these
\end{flushleft}


\begin{flushleft}
units, a student will need to involve himself/herself in activities beyond the classroom engagements. For earning 1
\end{flushleft}


\begin{flushleft}
unit a student will typically need to work for 2-3 hours per week (28-42 hours per semester) in on-campus activities.
\end{flushleft}


\begin{flushleft}
In case of project / design / internship activities, the student engagement expected is typically 20 man-days of work
\end{flushleft}


\begin{flushleft}
per non-graded unit. A student would not be allowed to earn credits as well as non-graded units for the same
\end{flushleft}


\begin{flushleft}
effort - it is important that the efforts towards earning non-graded units should be distinct from that spent on earning
\end{flushleft}


\begin{flushleft}
credits. Also, the effort for earning different components of the non-graded units also should be distinct, i.e., the
\end{flushleft}


\begin{flushleft}
same effort would not be evaluated for more than one non-graded activity.
\end{flushleft}


\begin{flushleft}
Non-graded core of the undergraduate curriculum comprises the following components:
\end{flushleft}


	


	





1.	


2.	





\begin{flushleft}
Introduction to Engineering \& Programme	
\end{flushleft}


\begin{flushleft}
Language and Writing Skills	
\end{flushleft}





\begin{flushleft}
:	 02 units
\end{flushleft}


\begin{flushleft}
:	 02 units
\end{flushleft}





	





3.	





\begin{flushleft}
NCC / NSO / NSS	
\end{flushleft}





:	





	





4.	





\begin{flushleft}
Professional Ethics and Social Responsibility	
\end{flushleft}





\begin{flushleft}
:	 02 units
\end{flushleft}





	





5.	





\begin{flushleft}
Communication Skills / Seminar	
\end{flushleft}





:	





\begin{flushleft}
02 units
\end{flushleft}





	





6.	





\begin{flushleft}
Design / Practical Experience	
\end{flushleft}





:	





\begin{flushleft}
05 units
\end{flushleft}





\begin{flushleft}
02 units
\end{flushleft}





\begin{flushleft}
:	 15 units
\end{flushleft}





\begin{flushleft}
		Total	
\end{flushleft}





\begin{flushleft}
These 15 units form a compulsory graduation requirement for all the undergraduate (B.Tech. as well as Dual
\end{flushleft}


\begin{flushleft}
degree) programmes. A student will need to earn these 15 units over the duration of the programme with special
\end{flushleft}


\begin{flushleft}
consideration and requirements for each component as detailed in the following sections. Each component would
\end{flushleft}


\begin{flushleft}
be constituted by one or more non-graded courses, and a student will need to get an {`}S' grade in these courses
\end{flushleft}


\begin{flushleft}
to earn the respective non-graded unit(s). Incomplete status in such courses will be indicated by a {`}Z' grade. The
\end{flushleft}


\begin{flushleft}
student would be required either to repeat the course / activity or continue with the project / internship until such time
\end{flushleft}


\begin{flushleft}
that the evaluating faculty member / committee is satisfied with the effort to award an {`}S' grade. No partial / fractional
\end{flushleft}


\begin{flushleft}
units can be awarded. For example, if a particular activity carries 2 units, a student cannot be awarded 1 unit or
\end{flushleft}


\begin{flushleft}
fractional units for incomplete work, but would need to repeat / complete the work to the satisfaction of the evaluating
\end{flushleft}


\begin{flushleft}
faculty member / committee to become eligible for award of 2 units.
\end{flushleft}





\begin{flushleft}
8.1 Introduction to Engineering and Programme
\end{flushleft}


\begin{flushleft}
This non-graded component is aimed at orienting and exciting students in the subject of engineering in general
\end{flushleft}


\begin{flushleft}
and their respective disciplines in particular. The objectives of the component are:
\end{flushleft}


$\bullet$	





\begin{flushleft}
Exposing students to {``}Engineering'' as a profession that creates wealth for nations, and as a vehicle for
\end{flushleft}


\begin{flushleft}
economic growth.
\end{flushleft}





$\bullet$	





\begin{flushleft}
Exposing students to Science/ Engineering as a medium through which one can address problems facing
\end{flushleft}


\begin{flushleft}
the society including some of the grand challenges.
\end{flushleft}





$\bullet$	





\begin{flushleft}
Excite students by enabling them to appreciate the role and enormous impact of research in science/
\end{flushleft}


\begin{flushleft}
engineering on our day to day lives.
\end{flushleft}





$\bullet$	





\begin{flushleft}
Enlighten students about the various career options available to them.
\end{flushleft}





$\bullet$	





\begin{flushleft}
Make students aware of the issues involved in engineering a product, and help them appreciate why
\end{flushleft}


\begin{flushleft}
the process of design and innovation leading to products and systems is both personally satisfying and
\end{flushleft}


\begin{flushleft}
professionally rewarding.
\end{flushleft}





$\bullet$	





\begin{flushleft}
Excite students about potential role models and successful alumni in engineering profession.
\end{flushleft}





$\bullet$	





\begin{flushleft}
Motivate students to take up some co-curricular activities on their own during their stay in the Institute.
\end{flushleft}





\begin{flushleft}
The activities to realize the above-mentioned objectives as part of this non-graded component include:
\end{flushleft}


$\bullet$	





\begin{flushleft}
Understanding engineering through product dissection and reverse engineering. (The products given to
\end{flushleft}


\begin{flushleft}
students to dissect could be physical in form or in the form of videos).
\end{flushleft}





$\bullet$	





\begin{flushleft}
Screening of videos that bring out the strong relation between science / engineering and societal needs.
\end{flushleft}





$\bullet$	





\begin{flushleft}
Conducting design and innovation contests among students.
\end{flushleft}





$\bullet$	





\begin{flushleft}
Solving science / engineering puzzles in the class.
\end{flushleft}





$\bullet$	





\begin{flushleft}
Lectures by successful industrialists, alumni and entrepreneurs about their journey.
\end{flushleft}


78





\begin{flushleft}
\newpage
Courses of Study 2017-2018
\end{flushleft}





$\bullet$	


$\bullet$	


$\bullet$	


$\bullet$	


$\bullet$	


$\bullet$	


$\bullet$	


$\bullet$	





\begin{flushleft}
Exposure to successful research cases from the Institute and the impact of the same.
\end{flushleft}


\begin{flushleft}
Exposure to successful products / innovations from the Institute which have reached people/ industry/ society.
\end{flushleft}


\begin{flushleft}
Some interesting demonstrations in laboratories.
\end{flushleft}


\begin{flushleft}
Hands-on exercises in laboratories including use of breadboard circuits, Lego sets, robot kits, balsa
\end{flushleft}


\begin{flushleft}
bridge engineering kits, fibre optics kits, mobile apps etc.
\end{flushleft}


\begin{flushleft}
Industry visits
\end{flushleft}


\begin{flushleft}
Visits to on-going exhibitions in the city
\end{flushleft}


\begin{flushleft}
Do-it-yourself projects in teams
\end{flushleft}


\begin{flushleft}
Lectures by faculty, visitors, alumni on some exciting topics.
\end{flushleft}





\begin{flushleft}
This non-graded unit is administered in the form of two non-graded courses of one unit each:
\end{flushleft}


\begin{flushleft}
(i)	 NIN100 Introduction to Engineering in the first semester of the undergraduate programme, and
\end{flushleft}


\begin{flushleft}
(ii)	 XXN101 / XXN111 Introduction to $<$the respective engineering discipline$>$ in the third semester.
\end{flushleft}


\begin{flushleft}
Here, XX stands for the Course code prefix of the Department offering the undergraduate programme and $<$the
\end{flushleft}


\begin{flushleft}
respective engineering discipline$>$ stands for the name of the undergraduate programme to which the student
\end{flushleft}


\begin{flushleft}
belongs. Table 8.1 lists the courses corresponding to the different undergraduate programmes:
\end{flushleft}


\begin{flushleft}
Table 8.1 : List of Introduction to Engineering Courses Offered by Departments
\end{flushleft}


\begin{flushleft}
S. No. Programme Code(s) Course Code
\end{flushleft}





\begin{flushleft}
Course Title
\end{flushleft}





1





\begin{flushleft}
BB1, BB5
\end{flushleft}





\begin{flushleft}
BBN101
\end{flushleft}





\begin{flushleft}
Introduction to Biochemical Engineering and Biotechnology
\end{flushleft}





2





\begin{flushleft}
CE1
\end{flushleft}





\begin{flushleft}
CVN101
\end{flushleft}





\begin{flushleft}
Introduction to Civil Engineering
\end{flushleft}





3





\begin{flushleft}
CH1, CH7
\end{flushleft}





\begin{flushleft}
CLN101
\end{flushleft}





\begin{flushleft}
Introduction to Chemical Engineering
\end{flushleft}





4





\begin{flushleft}
CS1, CS5
\end{flushleft}





\begin{flushleft}
CON101
\end{flushleft}





\begin{flushleft}
Introduction to Computer Science and Engineering
\end{flushleft}





5





\begin{flushleft}
EE1
\end{flushleft}





\begin{flushleft}
ELN101
\end{flushleft}





\begin{flushleft}
Introduction to Electrical Engineering
\end{flushleft}





6





\begin{flushleft}
EE3
\end{flushleft}





\begin{flushleft}
ELN111
\end{flushleft}





\begin{flushleft}
Introduction to Electrical Engineering -- Power and Automation
\end{flushleft}





7





\begin{flushleft}
ME1
\end{flushleft}





\begin{flushleft}
MCN101
\end{flushleft}





\begin{flushleft}
Introduction to Mechanical Engineering
\end{flushleft}





8





\begin{flushleft}
ME2
\end{flushleft}





\begin{flushleft}
MCN111
\end{flushleft}





\begin{flushleft}
Introduction to Production and Industrial Engineering
\end{flushleft}





9





\begin{flushleft}
MT1, MT6
\end{flushleft}





\begin{flushleft}
MTN101
\end{flushleft}





\begin{flushleft}
Introduction to Mathematics and Computing
\end{flushleft}





10





\begin{flushleft}
PH1
\end{flushleft}





\begin{flushleft}
PYN101
\end{flushleft}





\begin{flushleft}
Introduction to Engineering Physics
\end{flushleft}





11





\begin{flushleft}
TT1
\end{flushleft}





\begin{flushleft}
TXN101
\end{flushleft}





\begin{flushleft}
Introduction to Textile Technology
\end{flushleft}





\begin{flushleft}
Course coordinator of NIN100 would be identified by the Dean Academics. For all the Departmental courses listed
\end{flushleft}


\begin{flushleft}
in Table 8.1, the departments offering UG programmes would identify the course coordinators. It is necessary to
\end{flushleft}


\begin{flushleft}
get a satisfactory (S) grade in both these courses for completing the degree requirements. Attendance would be
\end{flushleft}


\begin{flushleft}
one of the main criteria for evaluation. Apart from this, active participation and quiz based evaluation etc. would
\end{flushleft}


\begin{flushleft}
also be used as a basis to decide {`}S' or {`}Z' grade. The grades of NIN100 would be moderated by Dean Academics,
\end{flushleft}


\begin{flushleft}
and those of the Departmental courses would be moderated in the respective Departments. In case a student is
\end{flushleft}


\begin{flushleft}
awarded {`}Z' grade he/she would need to repeat the course in the subsequent year(s).
\end{flushleft}





\begin{flushleft}
8.2	 Language and Writing Skills
\end{flushleft}


\begin{flushleft}
All students, in the first two semesters, are required to undergo exercises designed to impart language skillsenhancing their ability of listening comprehension, reading and writing in English. These exercises would be tailored
\end{flushleft}


\begin{flushleft}
according to the background of the students. The English language ability of the students would be assessed through
\end{flushleft}


\begin{flushleft}
a test to be conducted in the beginning, typically during their admission and orientation period. The students would
\end{flushleft}


\begin{flushleft}
also be exposed to principles of English grammar and nuances of technical writing. Textual material and lectures
\end{flushleft}


\begin{flushleft}
would focus on the relationship between Engineering, Humanities and Social Sciences.
\end{flushleft}


\begin{flushleft}
This component is also administered in the form of two courses, each of one unit: NLN100 Language and Writing
\end{flushleft}


\begin{flushleft}
Skills--I in the first semester and NLN101 Language and Writing Skills--II in the second semester. Course coordinators
\end{flushleft}


\begin{flushleft}
for these courses are identified by the Dean Academics in consultation with Head, Humanities and Social Sciences.
\end{flushleft}


\begin{flushleft}
Assessment of a student towards S grade in each of these courses would typically be on the basis of attendance,
\end{flushleft}


\begin{flushleft}
participation and performance in the exercises. A student could also be prescribed self learning exercises or additional
\end{flushleft}


\begin{flushleft}
practice sessions during vacations as requirement for securing S grade. Student's involvement, during regular semester,
\end{flushleft}


\begin{flushleft}
would typically be two hours per week. The grades of these courses would be moderated by the Dean Academics.
\end{flushleft}


79





\begin{flushleft}
\newpage
Courses of Study 2017-2018
\end{flushleft}





\begin{flushleft}
8.3	 NCC/ NSO/ NSS
\end{flushleft}


\begin{flushleft}
A student is required to choose one of NCC/NSO/NSS by during his/her first semester, and complete the requirements
\end{flushleft}


\begin{flushleft}
within the first four registered semesters. Students will be required to earn 2 non-graded units from one of these
\end{flushleft}


\begin{flushleft}
activities, by completing 100 hours of work. The faculty coordinators of NCC / NSO / NSS decide and announce the
\end{flushleft}


\begin{flushleft}
policies on earning non-graded units in these activities from time to time. A student must complete the 100 hours of
\end{flushleft}


\begin{flushleft}
activities in one of these three options by the end of the fourth registered semester or the summer after the fourth
\end{flushleft}


\begin{flushleft}
semester, failing which he/ she would not be allowed to register for the fifth semester.
\end{flushleft}





\begin{flushleft}
8.4	 Professional Ethics and Social Responsibility
\end{flushleft}


\begin{flushleft}
There is increasing consensus worldwide that engineering ethics should be incorporated into the engineering
\end{flushleft}


\begin{flushleft}
curriculum to provide students with an exposure to the kind of professional ethical dilemmas they might face on
\end{flushleft}


\begin{flushleft}
an individual basis as well as in the larger context of ethical aspects of technology development. Workshops,
\end{flushleft}


\begin{flushleft}
discussion/ debates, use of theatre-in-education, case-study based approaches, etc. are often used for illustration
\end{flushleft}


\begin{flushleft}
and discussion of engineering ethics and such inputs could be provided in a stand-alone manner, integrated into
\end{flushleft}


\begin{flushleft}
existing courses or both. The objective of this non-graded component is to sensitize students about Professional
\end{flushleft}


\begin{flushleft}
Ethics and Social Responsibility (PESR) through a combination of the above-mentioned approaches, supplemented
\end{flushleft}


\begin{flushleft}
by discussion fora and supplementary materials, to help students to become ethical professionals. A student
\end{flushleft}


\begin{flushleft}
is required to complete this non-graded component in the first six registered semesters of the undergraduate
\end{flushleft}


\begin{flushleft}
programme, through activities divided into four courses. The courses NEN100 and NEN101 together constitute
\end{flushleft}


\begin{flushleft}
one non-graded unit and the course NEN300 along with one of the three alternatives NEN201 / NEN202 / NEN203
\end{flushleft}


\begin{flushleft}
constitutes the second non-graded unit for PESR:
\end{flushleft}


\begin{flushleft}
i)	 NEN100 Professional Ethics and Social Responsibility - I
\end{flushleft}


\begin{flushleft}
ii)	 NEN101 Professional Ethics and Social Responsibility - II
\end{flushleft}


\begin{flushleft}
iii)	 One of the following three courses:
\end{flushleft}


\begin{flushleft}
		 a.	 NEN201 PESR Internships
\end{flushleft}


\begin{flushleft}
		 b.	 NEN202 PESR Workshops
\end{flushleft}


\begin{flushleft}
		 c.	 NEN203 PESR Projects
\end{flushleft}


\begin{flushleft}
iv)	 NEN300 Case Studies in Professional Ethics
\end{flushleft}


\begin{flushleft}
NEN100 and NEN101 are compulsory for all students, and these courses involve interactive sessions of a group
\end{flushleft}


\begin{flushleft}
of about 20 students with a faculty mentor in the first and second semesters respectively. The student will earn
\end{flushleft}


\begin{flushleft}
one unit by getting S grade in both these courses.
\end{flushleft}


\begin{flushleft}
The second unit under PESR has two parts. For the first part, the students can choose to participate in any one
\end{flushleft}


\begin{flushleft}
out of a large variety of activities relevant to the core themes of PESR. With the considerable amount of flexibility
\end{flushleft}


\begin{flushleft}
allowed in the choice of activities, each student should be able to identify an activity of interest to him/ her under
\end{flushleft}


\begin{flushleft}
the purview of PESR. These activities have been divided into three broad categories, viz., (a) PESR internships (b)
\end{flushleft}


\begin{flushleft}
PESR workshops (c) PESR projects, each of which corresponds to a separate course number NEN201, NEN202
\end{flushleft}


\begin{flushleft}
and NEN203 respectively. After a student has got S grades in NEN100 and NEN101, the student can register for
\end{flushleft}


\begin{flushleft}
NEN201/202/203. The second part of this unit, Case Studies in Professional Ethics, is compulsory, and is offered
\end{flushleft}


\begin{flushleft}
under the course number NEN300.
\end{flushleft}


\begin{flushleft}
Under NEN201 PESR Internships, students can take up field work outside the Institute during summer/ winter
\end{flushleft}


\begin{flushleft}
vacation with organizations within the country. These organizations could be NGOs or CSR units of corporate houses.
\end{flushleft}


\begin{flushleft}
The students could also choose to work with organizations in their home towns. These internships must involve an
\end{flushleft}


\begin{flushleft}
exposure to the life of communities through field work. Before going for an internship with such an organization,
\end{flushleft}


\begin{flushleft}
the student will have to submit an online request to the Institute level PESR Committee, specifying the internship
\end{flushleft}


\begin{flushleft}
duration, nature of work and other details and requirements, and take prior approval. The student will be able to get
\end{flushleft}


\begin{flushleft}
an S grade in this course only if the student has attended the internship for at least the number of days specified in
\end{flushleft}


\begin{flushleft}
Table 8.2 and has satisfied the requirements committed in the prior approval. A documentary proof of the same from
\end{flushleft}


\begin{flushleft}
the organization should be submitted online by the student to the PESR Committee. No credit will be given to the
\end{flushleft}


\begin{flushleft}
student if he/she attends the internship for less number of days than specified in the prior approval as per Table 8.2.
\end{flushleft}


\begin{flushleft}
If a student gets selected in one of the nation building initiatives organized by reputed organizations (Examples:
\end{flushleft}


\begin{flushleft}
Participating in Jagriti Yatra, working as a summer fellow with Rakshak foundation etc.) and successfully completes
\end{flushleft}


\begin{flushleft}
the same, he/she would be considered to have completed the requirements under NEN201 and hence would be
\end{flushleft}


\begin{flushleft}
awarded {`}S' grade in the same. Even to exercise this option, it is mandatory that a prior approval of the PESR
\end{flushleft}


\begin{flushleft}
committee should be sought online, and a documentary proof of completion of the activity should be submitted
\end{flushleft}


80





\begin{flushleft}
\newpage
Courses of Study 2017-2018
\end{flushleft}





\begin{flushleft}
online to the PESR committee, as specified above for internships.
\end{flushleft}


\begin{flushleft}
Under NEN202 PESR Workshops, students can participate in one or more workshops of duration of 3-8 days,
\end{flushleft}


\begin{flushleft}
approved by the PESR Committee and Dean Academics. These workshops would be organized on campus by a faculty
\end{flushleft}


\begin{flushleft}
coordinator and would be conducted by resource persons from within or outside the Institute. These workshops could
\end{flushleft}


\begin{flushleft}
be pertaining to any of the themes relevant to PESR and could be held during mid-semester break / summer / winter
\end{flushleft}


\begin{flushleft}
vacation or even long weekends during the semester. The students must follow the procedure announced by the faculty
\end{flushleft}


\begin{flushleft}
coordinator to register for the workshop. The S grade for attending a workshop will be awarded only if the student
\end{flushleft}


\begin{flushleft}
attends all sessions of the workshop on all the days for its full duration. The faculty coordinator organizing the workshop
\end{flushleft}


\begin{flushleft}
would submit a list of all such students to the PESR committee for award of S grade in NEN202 PESR Workshops.
\end{flushleft}


\begin{flushleft}
Under NEN203 PESR Projects, the students can take up projects under the guidance of one or more faculty members
\end{flushleft}


\begin{flushleft}
to make positive contribution to campus life. This could include promoting wholesome practices on campus such
\end{flushleft}


\begin{flushleft}
as: ethical practices particularly among students through specially directed efforts; peer assistance for the students
\end{flushleft}


\begin{flushleft}
in need of help academically or otherwise; sustainable practices on campus like resource conservation, waste
\end{flushleft}


\begin{flushleft}
management, use of renewable resources and the like; working on technology for a social cause etc. This work
\end{flushleft}


\begin{flushleft}
could be done during a semester or mid-semester break or summer / winter vacation. The student must submit a
\end{flushleft}


\begin{flushleft}
project proposal online, with explicit statement of deliverables, through his / her faculty supervisor(s), for approval
\end{flushleft}


\begin{flushleft}
by the PESR committee. If the work is taken up in a team, each student's share of work must be defined in the
\end{flushleft}


\begin{flushleft}
proposal. It is expected that each student puts in at least 50 person-hours of effort in the project. On completion
\end{flushleft}


\begin{flushleft}
of the project, the students should submit a completion request online, again through the faculty supervisor, who
\end{flushleft}


\begin{flushleft}
should certify that each student has completed his / her share of the deliverables and each student has put in at
\end{flushleft}


\begin{flushleft}
least 50 person-hours of work into the project. The PESR Committee may also decide to evaluate the project by
\end{flushleft}


\begin{flushleft}
additional means as deemed fit.
\end{flushleft}


\begin{flushleft}
The work done under NEN201/ NEN202/ NEN203 would also be evaluated, the mechanism for which will be decided
\end{flushleft}


\begin{flushleft}
by the PESR committee and notified to students accordingly .
\end{flushleft}


\begin{flushleft}
If a student has a confirmed/ approved registration in an internship/ workshop or a project but does not turn up for
\end{flushleft}


\begin{flushleft}
the same, he/ she can be penalized by the PESR committee with an increase in the number of PESR units to be
\end{flushleft}


\begin{flushleft}
completed by the student for the degree requirements.
\end{flushleft}


\begin{flushleft}
In NEN300, every student will work on at least two case studies related to professional ethics, followed by discussions
\end{flushleft}


\begin{flushleft}
on the same, moderated by a faculty member. The details on how to select the case studies and the mode of discussions
\end{flushleft}


\begin{flushleft}
and their evaluation would be decided by the PESR Committee from time to time and notified to the students.
\end{flushleft}


\begin{flushleft}
Table 8.2 summarizes the requirements of the non-graded component on Professional Ethics and Social
\end{flushleft}


\begin{flushleft}
Responsibility.
\end{flushleft}


\begin{flushleft}
Table 8.2 : Summary of Requirements of the Non-Graded Component on Professional Ethics and
\end{flushleft}


\begin{flushleft}
Social Responsibility
\end{flushleft}


\begin{flushleft}
S.
\end{flushleft}


\begin{flushleft}
No.
\end{flushleft}





\begin{flushleft}
Course
\end{flushleft}





\begin{flushleft}
Period of
\end{flushleft}


\begin{flushleft}
Activity
\end{flushleft}





\begin{flushleft}
Description
\end{flushleft}





\begin{flushleft}
A
\end{flushleft}





\begin{flushleft}
Part-1: Regular Classroom Contact
\end{flushleft}





1.





\begin{flushleft}
NEN100
\end{flushleft}


\begin{flushleft}
Professional
\end{flushleft}


\begin{flushleft}
Ethics and
\end{flushleft}


\begin{flushleft}
Social
\end{flushleft}


\begin{flushleft}
Responsibility--I
\end{flushleft}





\begin{flushleft}
1st Semester
\end{flushleft}





\begin{flushleft}
NEN101
\end{flushleft}


\begin{flushleft}
Professional
\end{flushleft}


\begin{flushleft}
Ethics and
\end{flushleft}


\begin{flushleft}
Social
\end{flushleft}


\begin{flushleft}
Responsibility--II
\end{flushleft}





\begin{flushleft}
2nd Semester
\end{flushleft}





2.





\begin{flushleft}
B
\end{flushleft}





\begin{flushleft}
Requirement for S grade
\end{flushleft}





\begin{flushleft}
Regular sessions of 1.5-2
\end{flushleft}


\begin{flushleft}
hours with a faculty mentor.
\end{flushleft}


\begin{flushleft}
Activities in the sessions to
\end{flushleft}


\begin{flushleft}
be decided by the faculty.
\end{flushleft}





\begin{flushleft}
15 hours in regular sessions
\end{flushleft}





\begin{flushleft}
Regular sessions of 1.5-2
\end{flushleft}


\begin{flushleft}
hours with a faculty mentor.
\end{flushleft}


\begin{flushleft}
Activities in the sessions to
\end{flushleft}


\begin{flushleft}
be decided by the faculty.
\end{flushleft}





\begin{flushleft}
15 hours in regular sessions
\end{flushleft}





\begin{flushleft}
No.
\end{flushleft}


\begin{flushleft}
of
\end{flushleft}


\begin{flushleft}
units
\end{flushleft}





1





1





\begin{flushleft}
Part-2: Case Studies and Practical / Field Activity
\end{flushleft}


\begin{flushleft}
$\bullet$  Each student should register for NEN300 and ANY ONE of NEN201, NEN202, NEN203
\end{flushleft}


\begin{flushleft}
$\bullet$  To be completed before the beginning of 7th semester
\end{flushleft}


\begin{flushleft}
$\bullet$  Activities as listed below followed by A PRESENTATION IN AN EVALUATION SESSION
\end{flushleft}


\begin{flushleft}
$\bullet$  Satisfaction of ALL the requirements set out by the respective in-charge / resource persons / faculty.
\end{flushleft}


\begin{flushleft}
NO PART CREDIT.
\end{flushleft}


81





\begin{flushleft}
\newpage
Courses of Study 2017-2018
\end{flushleft}





3.





\begin{flushleft}
NEN201 : PESR summer/winter Engagement with
\end{flushleft}


\begin{flushleft}
Internships
\end{flushleft}


\begin{flushleft}
vacations
\end{flushleft}


\begin{flushleft}
communities/NGOs
\end{flushleft}


\begin{flushleft}
OUTSIDE IITD involving
\end{flushleft}


\begin{flushleft}
technical or non-technical
\end{flushleft}


\begin{flushleft}
work or internship with the
\end{flushleft}


\begin{flushleft}
CSR unit of an industry
\end{flushleft}


\begin{flushleft}
involving field work.
\end{flushleft}





\begin{flushleft}
For residential internship/
\end{flushleft}


\begin{flushleft}
camp with an organisation:
\end{flushleft}


\begin{flushleft}
6-8 days of stay in the camp.
\end{flushleft}


\begin{flushleft}
For non-residential internship:
\end{flushleft}


\begin{flushleft}
20 working days of internship.
\end{flushleft}


\begin{flushleft}
Prior approval of PESR
\end{flushleft}


\begin{flushleft}
committee specifying the
\end{flushleft}


\begin{flushleft}
type and length of the
\end{flushleft}


\begin{flushleft}
internship / camp; S grade
\end{flushleft}


\begin{flushleft}
to be awarded only for
\end{flushleft}


\begin{flushleft}
full duration. No credit for
\end{flushleft}


\begin{flushleft}
attending the internship for
\end{flushleft}


\begin{flushleft}
less number of days than that
\end{flushleft}


\begin{flushleft}
specified in approval.
\end{flushleft}





4.





\begin{flushleft}
NEN202 : PESR Mid-semester
\end{flushleft}


\begin{flushleft}
Workshops
\end{flushleft}


\begin{flushleft}
breaks/
\end{flushleft}


\begin{flushleft}
summer/winter
\end{flushleft}


\begin{flushleft}
vacations.
\end{flushleft}





\begin{flushleft}
Participation in intense
\end{flushleft}


\begin{flushleft}
ON-CAMPUS workshops
\end{flushleft}


\begin{flushleft}
approved by Dean
\end{flushleft}


\begin{flushleft}
Academics, of 3-8 days
\end{flushleft}


\begin{flushleft}
duration, conducted by
\end{flushleft}


\begin{flushleft}
professional resource
\end{flushleft}


\begin{flushleft}
persons, with special
\end{flushleft}


\begin{flushleft}
emphasis on themes related
\end{flushleft}


\begin{flushleft}
to PESR.
\end{flushleft}





\begin{flushleft}
Completion of either a single
\end{flushleft}


\begin{flushleft}
workshop of at least 6 days'
\end{flushleft}


\begin{flushleft}
duration OR two workshops
\end{flushleft}


\begin{flushleft}
of at least 3 days' duration.
\end{flushleft}


\begin{flushleft}
S grade to be awarded only
\end{flushleft}


\begin{flushleft}
for attending the workshop
\end{flushleft}


\begin{flushleft}
for full duration. No credit for
\end{flushleft}


\begin{flushleft}
attending the workshop for
\end{flushleft}


\begin{flushleft}
less number of days.
\end{flushleft}





5.





\begin{flushleft}
NEN203: PESR
\end{flushleft}


\begin{flushleft}
Projects
\end{flushleft}





\begin{flushleft}
Summer/
\end{flushleft}


\begin{flushleft}
winter vacation
\end{flushleft}


\begin{flushleft}
/mid-semester
\end{flushleft}


\begin{flushleft}
break or
\end{flushleft}


\begin{flushleft}
during a
\end{flushleft}


\begin{flushleft}
semester.
\end{flushleft}





\begin{flushleft}
Taking up on-campus
\end{flushleft}


\begin{flushleft}
projects under the guidance
\end{flushleft}


\begin{flushleft}
of a faculty mentor, related
\end{flushleft}


\begin{flushleft}
to any of the topics relevant
\end{flushleft}


\begin{flushleft}
to PESR, such as (but not
\end{flushleft}


\begin{flushleft}
limited to)
\end{flushleft}


\begin{flushleft}
A. Promoting ethical
\end{flushleft}


\begin{flushleft}
practices on campus in
\end{flushleft}


\begin{flushleft}
various spheres particularly
\end{flushleft}


\begin{flushleft}
related to student life on
\end{flushleft}


\begin{flushleft}
campus.
\end{flushleft}


\begin{flushleft}
B. Strengthening the existing
\end{flushleft}


\begin{flushleft}
systems and designing and
\end{flushleft}


\begin{flushleft}
implementing new ones for
\end{flushleft}


\begin{flushleft}
an active student community
\end{flushleft}


\begin{flushleft}
participation in addressing
\end{flushleft}


\begin{flushleft}
the academic as well as
\end{flushleft}


\begin{flushleft}
other problems of student
\end{flushleft}


\begin{flushleft}
community.
\end{flushleft}


\begin{flushleft}
C. Developing socially
\end{flushleft}


\begin{flushleft}
relevant technologies
\end{flushleft}


\begin{flushleft}
D. Promoting Sustainable
\end{flushleft}


\begin{flushleft}
Practices in hostels,
\end{flushleft}


\begin{flushleft}
academic area, residential
\end{flushleft}


\begin{flushleft}
areas etc., involving activities
\end{flushleft}


\begin{flushleft}
pertaining to conservation of
\end{flushleft}


\begin{flushleft}
water / electricity / paper / other
\end{flushleft}


\begin{flushleft}
resources, waste
\end{flushleft}


\begin{flushleft}
management, promoting use
\end{flushleft}


\begin{flushleft}
of bicycles, etc.
\end{flushleft}





\begin{flushleft}
Prior approval of project
\end{flushleft}


\begin{flushleft}
proposal by PESR committee
\end{flushleft}


\begin{flushleft}
explicitly specifying
\end{flushleft}


\begin{flushleft}
deliverables and work share
\end{flushleft}


\begin{flushleft}
of each student in case of
\end{flushleft}


\begin{flushleft}
group projects; Completion
\end{flushleft}


\begin{flushleft}
of the project deliverables
\end{flushleft}


\begin{flushleft}
identified in the proposal - It
\end{flushleft}


\begin{flushleft}
must involve at least 50 hours
\end{flushleft}


\begin{flushleft}
of work by each student.
\end{flushleft}





\begin{flushleft}
5th or 6th
\end{flushleft}


\begin{flushleft}
registered
\end{flushleft}


\begin{flushleft}
semester
\end{flushleft}





\begin{flushleft}
Work on two case studies
\end{flushleft}


\begin{flushleft}
on professional ethics;
\end{flushleft}


\begin{flushleft}
participate in discussions
\end{flushleft}


\begin{flushleft}
moderated by a faculty
\end{flushleft}


\begin{flushleft}
member.
\end{flushleft}





\begin{flushleft}
Recommendation of S grade
\end{flushleft}


\begin{flushleft}
by the faculty member(s)
\end{flushleft}


\begin{flushleft}
moderating the discussions.
\end{flushleft}





6.





\begin{flushleft}
NEN300 Case
\end{flushleft}


\begin{flushleft}
Studies in
\end{flushleft}


\begin{flushleft}
Professional
\end{flushleft}


\begin{flushleft}
Ethics
\end{flushleft}





82





1





\begin{flushleft}
\newpage
Courses of Study 2017-2018
\end{flushleft}





\begin{flushleft}
8.5	 Communication Skills / Seminar
\end{flushleft}


\begin{flushleft}
The objective of this non-graded component is to provide the students with an opportunity to develop their skills
\end{flushleft}


\begin{flushleft}
in preparing write-ups and / or making presentations, and reading / listening to others' write-ups / presentations.
\end{flushleft}


\begin{flushleft}
A student would be required to earn these non-graded units between their 5th and 8th semesters. This component
\end{flushleft}


\begin{flushleft}
would be administered in two parts:
\end{flushleft}


\begin{flushleft}
(i)	 A set of topic specific seminar courses (XXQ301, XXQ302, etc.) introduced by the parent Department
\end{flushleft}


\begin{flushleft}
of each student (for example ELQ301 -- Seminar on Embedded Systems -- 1 unit). These courses
\end{flushleft}


\begin{flushleft}
would be non-credit electives, offered in each semester. These seminar sessions would be held for
\end{flushleft}


\begin{flushleft}
two hours per week. Many such courses could run in parallel. Students need to register for at least
\end{flushleft}


\begin{flushleft}
one such course in his / her parent department for earning one unit.
\end{flushleft}


\begin{flushleft}
(ii)	 Students should earn the remaining one unit through any one of the following means:
\end{flushleft}


	





\begin{flushleft}
a.	 By registering and completing an additional seminar course (XYQ301, XYQ302 etc.) offered by
\end{flushleft}


\begin{flushleft}
any other Department / Centre / School.
\end{flushleft}





	





\begin{flushleft}
b.	 By participating in optional seminars which may be part of regular courses; for example regular
\end{flushleft}


\begin{flushleft}
{`}L' courses can have an optional seminar component (e.g. ELL711 Optical Communications can
\end{flushleft}


\begin{flushleft}
have optional seminar component of 1 unit). This would, like any other seminar course, need to
\end{flushleft}


\begin{flushleft}
have seminar sessions of 2 hours duration every week for a whole semester. In such a case, a
\end{flushleft}


\begin{flushleft}
student should register for XXQ30y, and the course coordinator would send recommendations for
\end{flushleft}


\begin{flushleft}
{`}S' grades to the Dean Academics, duly moderated by the Moderation Committee of the concerned
\end{flushleft}


\begin{flushleft}
Department / Centre / School.
\end{flushleft}





	





\begin{flushleft}
c.	 By participating in special workshops on Communication Skills approved by Dean Academics.
\end{flushleft}


\begin{flushleft}
The faculty coordinator in charge of the workshop would submit a list of students who completed
\end{flushleft}


\begin{flushleft}
the activity with 100\% attendance in all sessions on all days of the workshop for award of S grade
\end{flushleft}


\begin{flushleft}
in NQN301.
\end{flushleft}





	





\begin{flushleft}
d.	 By submitting documentary evidence of excellence in debating and/or writing as certified by faculty
\end{flushleft}


\begin{flushleft}
in-charge of these activities, to the Dean Academics. In all such cases, the student should submit
\end{flushleft}


\begin{flushleft}
documentary evidence online, as detailed below.
\end{flushleft}





		





$\bullet$	





\begin{flushleft}
A student who wins first, second or third position in any event / competition conducted at interhostel level, by BRCA or by BSP or by BSW would qualify for this option. The event / competition
\end{flushleft}


\begin{flushleft}
must be either a debate / declamation / extempore. Since many such events do not have
\end{flushleft}


\begin{flushleft}
certificates issued, the student must submit a letter signed by the warden or the president of
\end{flushleft}


\begin{flushleft}
the respective board (in case of BRCA, president of the club would also suffice) stating the
\end{flushleft}


\begin{flushleft}
date, time, venue of the event / competition along with the number of participants and position
\end{flushleft}


\begin{flushleft}
secured. In case number of participants is less than 20, the event shall not be counted.
\end{flushleft}





		





$\bullet$	





\begin{flushleft}
A student who performs as a compere for any of the Institute functions (only those listed in
\end{flushleft}


\begin{flushleft}
the Institute calendar). The student will need to produce a signed letter from the faculty incharge of the Institute function stating the student's role as compere. The letter must include
\end{flushleft}


\begin{flushleft}
the date, time, venue and duration of the event. Any event lasting less than 1 hour will not
\end{flushleft}


\begin{flushleft}
be counted.
\end{flushleft}





		





$\bullet$	





\begin{flushleft}
A student winning a technical paper presentation award during TRYST will need to submit
\end{flushleft}


\begin{flushleft}
a copy of the certificate and the abstract of the paper presented. Technical publications in
\end{flushleft}


\begin{flushleft}
Journals or Conferences would also be considered, provided (i) the number of authors of
\end{flushleft}


\begin{flushleft}
the paper does not exceed 2 and (ii) the faculty member supervising the work certifies that
\end{flushleft}


\begin{flushleft}
the paper was written by the concerned student.
\end{flushleft}





\begin{flushleft}
A minimum of three such documents certified by the Faculty in charge of the Board / Club / Activity as mentioned above
\end{flushleft}


\begin{flushleft}
would qualify a student to earn one unit of Communication Skills / Seminar. In each case, before recommending
\end{flushleft}


\begin{flushleft}
the award of non-graded units for the above activities, the Faculty in charge of the Board / Club / Activity should
\end{flushleft}


\begin{flushleft}
keep in mind that a student engagement / effort (including preparations and the actual event) of 28-42 hours
\end{flushleft}


\begin{flushleft}
would be necessary for the award of one non-graded unit.
\end{flushleft}


\begin{flushleft}
In cases of options (i), (ii) a and (ii) b above, the faculty member in charge of the course should ensure that the
\end{flushleft}


\begin{flushleft}
student has 100\% attendance in the seminars, and has done a satisfactory task of his / her contribution to the
\end{flushleft}


83





\begin{flushleft}
\newpage
Courses of Study 2017-2018
\end{flushleft}





\begin{flushleft}
course: the write-up, presentation, etc. before awarding an {`}S' grade. These grades would be moderated by the
\end{flushleft}


\begin{flushleft}
respective Department / Centre / School. In case of unavoidable absence of up to 3 seminar sessions, appropriate
\end{flushleft}


\begin{flushleft}
compensation mechanism should be announced by the faculty member at the beginning of the course. For
\end{flushleft}


\begin{flushleft}
absence beyond 3 sessions, S graded cannot be awarded.
\end{flushleft}


\begin{flushleft}
An Institute level Committee for Communication Skills / Seminar would be appointed by the Dean Academics.
\end{flushleft}


\begin{flushleft}
The convener of this committee would serve as the course coordinator of NQN301. This committee would
\end{flushleft}


\begin{flushleft}
moderate the non-graded units for Communication Skills / Seminar recommended for activities other than the
\end{flushleft}


\begin{flushleft}
courses XXQ30y.
\end{flushleft}


\begin{flushleft}
A student needs to secure an S grade in both parts of the communication skills / seminar non-graded component
\end{flushleft}


\begin{flushleft}
to complete graduation requirements.
\end{flushleft}





\begin{flushleft}
8.6	 Design / Practical Experience
\end{flushleft}


\begin{flushleft}
The objective of this non-graded component is to give opportunities to students to learn in an informal setting.
\end{flushleft}


\begin{flushleft}
This mode of learning, is often more effective than conventional lectures / laboratory work. Second and even
\end{flushleft}


\begin{flushleft}
more important objective of this non-graded component is to inculcate design thinking among students and
\end{flushleft}


\begin{flushleft}
facilitate them to gain some design immersion experience. Design / Practical Experience (DPE) component can
\end{flushleft}


\begin{flushleft}
promote learning by doing which does two important things: Firstly, it allows students to immerse themselves
\end{flushleft}


\begin{flushleft}
in the environment in which work is to be done, so that they can understand the values and expectations of the
\end{flushleft}


\begin{flushleft}
target beneficiaries. Secondly it enables a fresh look at problems, not only at the ways of defining them, but
\end{flushleft}


\begin{flushleft}
also at the ways to solve those including skill-sets that are required to address them. A shift from problem based
\end{flushleft}


\begin{flushleft}
learning (acquisition of knowledge) to project based learning (application of knowledge), in which the projects
\end{flushleft}


\begin{flushleft}
are grounded in problems outside the classrooms and laboratories, in everyday scenarios. Thus, DPE bridges
\end{flushleft}


\begin{flushleft}
division between the curricular and the co-curricular components, and encourages the curiosity and involvement
\end{flushleft}


\begin{flushleft}
that arises from total absorption in a subject of interest.
\end{flushleft}


\begin{flushleft}
As a part of this requirement, every student is expected to earn a minimum of five non-graded units of DPE to
\end{flushleft}


\begin{flushleft}
complete the degree requirements. To earn one unit of DPE, a student is expected to put in 28-42 hours of
\end{flushleft}


\begin{flushleft}
effort or 20 working days depending on the type of activity. These units can be earned in multiple ways during
\end{flushleft}


\begin{flushleft}
the semester as well as during vacation and mid-semester breaks:
\end{flushleft}


$\bullet$	





\begin{flushleft}
Courses with design focus without any regular graded credits, which are designated to give design/
\end{flushleft}


\begin{flushleft}
practical experience units.
\end{flushleft}





$\bullet$	





\begin{flushleft}
Courses (core or elective) with optional design / practical experience component.
\end{flushleft}





$\bullet$	





\begin{flushleft}
Summer/semester internships by students in R\&D / Industry / Universities in India or abroad.
\end{flushleft}





$\bullet$	





\begin{flushleft}
Summer / winter / semester projects under the guidance of faculty of the Institute.
\end{flushleft}





$\bullet$	





\begin{flushleft}
Participation in design / innovation projects by Innovation Center / CAIC, etc.
\end{flushleft}





$\bullet$	





\begin{flushleft}
One time activity such as design / practical experience workshop / course / event during semester / 
\end{flushleft}


\begin{flushleft}
vacation / mid-semester breaks, etc.
\end{flushleft}





\begin{flushleft}
DPE activities are not restricted to design of physical products but can also include system level design and
\end{flushleft}


\begin{flushleft}
experience. For example a team of students who under the supervision of faculty in collaboration with an NGO,
\end{flushleft}


\begin{flushleft}
would like to design a new financial inclusion system for marginalized section of population too can earn design/
\end{flushleft}


\begin{flushleft}
practical experience units.
\end{flushleft}


\begin{flushleft}
The operational modalities of implementing the above-mentioned activities so that students can earn the required
\end{flushleft}


\begin{flushleft}
non-graded units, are presented in the following paragraphs.
\end{flushleft}





\begin{flushleft}
8.6.1 Management of Non-graded DPE Units
\end{flushleft}


\begin{flushleft}
Each Department offering UG programme(s) would constitute a DPE Committee with a Departmental DPE
\end{flushleft}


\begin{flushleft}
Coordinator to manage the non-graded Design / Practical Experience units.
\end{flushleft}


\begin{flushleft}
a)	 The Departmental DPE Committee would coordinate with T\&P Unit to identify and vet industries for
\end{flushleft}


\begin{flushleft}
internships.
\end{flushleft}


84





\begin{flushleft}
\newpage
Courses of Study 2017-2018
\end{flushleft}





\begin{flushleft}
b)	 The committee would also examine other types of internship (in Universities, research laboratories,
\end{flushleft}


\begin{flushleft}
start-ups etc.) requested by students and approve or deny as per a policy defined by the Department.
\end{flushleft}


\begin{flushleft}
c)	 Students of the Department desirous of earning non-graded DPE units through any other mechanism
\end{flushleft}


\begin{flushleft}
listed above should request permission of this committee before embarking on the activity. The
\end{flushleft}


\begin{flushleft}
committee would also decide on the award of non-graded DPE units for all such activities for the
\end{flushleft}


\begin{flushleft}
students of the Department through appropriate evaluation mechanisms.
\end{flushleft}


\begin{flushleft}
d)	 The committee would be responsible to evaluate the design activities carried out by the students during
\end{flushleft}


\begin{flushleft}
internships and recommending award of the non-graded DPE units, or continuation of the internship
\end{flushleft}


\begin{flushleft}
activity for more days to become eligible for the units, as per the efforts of the students during the
\end{flushleft}


\begin{flushleft}
internship. DPE Committee will moderate all Design units awarded to students of that Department. The
\end{flushleft}


\begin{flushleft}
Departmental DPE Coordinator also has responsibility of ensuring that units earned by heterogeneous
\end{flushleft}


\begin{flushleft}
activities meet the requirements in terms of learning efforts and experience.
\end{flushleft}


\begin{flushleft}
e)	 The Dean Academics will appoint an Institute DPE Coordinator for Design / Practical Experience
\end{flushleft}


\begin{flushleft}
units.
\end{flushleft}


\begin{flushleft}
f)	
\end{flushleft}





\begin{flushleft}
Departmental DPE Coordinators, Institute DPE Coordinator and Associate Dean Academics-Curriculum
\end{flushleft}


\begin{flushleft}
together will form an institute level committee to moderate the non-graded units awarded under
\end{flushleft}


\begin{flushleft}
interdisciplinary work including the activities carried out by students in Departments / Centers / Schools
\end{flushleft}


\begin{flushleft}
not offering UG programmes. This committee would also review and modify policies as well as modalities
\end{flushleft}


\begin{flushleft}
administering DPE units.
\end{flushleft}





\begin{flushleft}
8.6.2 Activities Covered Under Design / Practical Experience
\end{flushleft}


\begin{flushleft}
8.6.2.1. Specialized Courses Related to Design / Practical Experience (Maximum 2 Units)
\end{flushleft}


\begin{flushleft}
Departments/Departments/Centres/Schools may offer a basket of courses that will not have any credits associated
\end{flushleft}


\begin{flushleft}
with them but will have only Design / Practical Experience units linked to them. In other words, on successful
\end{flushleft}


\begin{flushleft}
completion of such courses the students will earn only DPE units but no graded credits. These courses offered
\end{flushleft}


\begin{flushleft}
by Departments/ Centers/Schools can be of one unit (28-42 hours of student effort) or two units (56-84 hours
\end{flushleft}


\begin{flushleft}
of student effort). Faculty offering these courses will award these units on successful completion of the course
\end{flushleft}


\begin{flushleft}
requirements, and the same would be moderated by the Departmental Committee for DPE in case of Departments
\end{flushleft}


\begin{flushleft}
offering undergraduate programmes. For other Departments / Centres / Schools, the moderation would be done
\end{flushleft}


\begin{flushleft}
by the Institute level DPE committee.
\end{flushleft}





\begin{flushleft}
8.6.2.2. Semester / Summer / Winter Projects Under the Guidance of Institute Faculty
\end{flushleft}


\begin{flushleft}
(Maximum 2 Units)
\end{flushleft}


\begin{flushleft}
Some of the co-curricular activities in the Institute that pertain to team based product building such as Robotics,
\end{flushleft}


\begin{flushleft}
Automobile, IGEM, Aero-modelling etc. can also be considered for earning DPE units. Students who successfully
\end{flushleft}


\begin{flushleft}
complete SURA / DISA projects related to design / practical experience will also be eligible for DPE units. Besides,
\end{flushleft}


\begin{flushleft}
students may also opt for working on semester / summer / winter projects involving design/practical experience
\end{flushleft}


\begin{flushleft}
activity under the guidance of faculty of the institute. In order to be evaluated for DPE Units in such cases, a
\end{flushleft}


\begin{flushleft}
student should register for XXD35y Minor Design Project floated by the parent Department XX of the student. In
\end{flushleft}


\begin{flushleft}
case the project is interdisciplinary or it is offered by faculty of other Departments / Centres / Schools, the faculty
\end{flushleft}


\begin{flushleft}
supervisor of the project may advise the students to register for NDN35y Minor Design Project. In either case, the
\end{flushleft}


\begin{flushleft}
project would be evaluated by the faculty supervisor. For award of DPE units, XXD35y would be moderated by
\end{flushleft}


\begin{flushleft}
the Departmental Committee for DPE while NDN35y would be moderated by the Institute level DPE committee.
\end{flushleft}


\begin{flushleft}
The courses XXD351 -- XXD355 would be Minor Design Projects with 1 non-graded DPE unit, and XXD356 --
\end{flushleft}


\begin{flushleft}
XXD358 would be Minor Design Projects with 2 non-graded units each. Courses NDN351 -- 358 would also
\end{flushleft}


\begin{flushleft}
follow a similar definition.
\end{flushleft}





\begin{flushleft}
8.6.2.3. Regular Courses with Optional Design / Practical Experience Component
\end{flushleft}


\begin{flushleft}
(Maximum 2 Units)
\end{flushleft}


\begin{flushleft}
Course coordinators of regular core and elective courses can also offer optional design component in their
\end{flushleft}


\begin{flushleft}
courses. A proposal for this should be sent to the Departmental DPE committee prior to the commencement of the
\end{flushleft}


85





\begin{flushleft}
\newpage
Courses of Study 2017-2018
\end{flushleft}





\begin{flushleft}
course by the Course Coordinator. This would be notified to students by the Departmental DPE committee and
\end{flushleft}


\begin{flushleft}
also announced to the students by the course coordinator. Successful completion of the course will give graded
\end{flushleft}


\begin{flushleft}
credits to students and at the same time they will be eligible for earning (1 or 2) design units if they successfully
\end{flushleft}


\begin{flushleft}
complete the optional DPE component. The course coordinator will recommend these DPE units on successful
\end{flushleft}


\begin{flushleft}
completion of the assigned work. This would be moderated by the Departmental Committee for DPE. In case
\end{flushleft}


\begin{flushleft}
the course is offered by Departments / Centres / Schools which do not offer a UG programme, the notification prior
\end{flushleft}


\begin{flushleft}
to beginning of the course and moderation after the end of the course would be done by the Institute level DPE
\end{flushleft}


\begin{flushleft}
committee. In order to be evaluated for DPE Units, a student should register for XXD35y Minor Design Project
\end{flushleft}


\begin{flushleft}
or NDN35y Minor Design Project as the case may be.
\end{flushleft}





\begin{flushleft}
8.6.2.4. Summer Internships (2 Units)
\end{flushleft}


\begin{flushleft}
Students can undertake a minimum of 40 working days of internship to earn two design practical experience
\end{flushleft}


\begin{flushleft}
units during summer vacations in Industry, R\&D institutions or Universities in India or abroad. This would be
\end{flushleft}


\begin{flushleft}
administered by the Departmental Committee for DPE with the help of the Training and Placement (T\&P) unit.
\end{flushleft}


\begin{flushleft}
The Departmental DPE Committee would also be responsible for appointing a faculty supervisor for the internship.
\end{flushleft}


\begin{flushleft}
Students can proceed with the internship after the Departmental Committee for DPE approves the same. Design
\end{flushleft}


\begin{flushleft}
units for the internship would be awarded by the Departmental Committee after evaluation at the end of internship
\end{flushleft}


\begin{flushleft}
period. Rules governing administration of internships are given in section 8.6.3. In case an internship pertains
\end{flushleft}


\begin{flushleft}
to areas of expertise outside those of the parent Department, the DPE Committee may co-opt faculty members
\end{flushleft}


\begin{flushleft}
from other Departments / Centres / Schools for evaluating / supervising such internships.
\end{flushleft}





\begin{flushleft}
8.6.2.5. One-Semester Internship (Maximum 5 Units)
\end{flushleft}


\begin{flushleft}
Students can opt for one semester internship in Industry, R\&D institutions or Universities in India or abroad, for
\end{flushleft}


\begin{flushleft}
a minimum of 100 working days, by appropriately planning for completion of credit requirements for the degree.
\end{flushleft}


\begin{flushleft}
The student can also opt for a break in coursework for a semester to initiate or work for his / her start up. These
\end{flushleft}


\begin{flushleft}
are the only two activities upon successful completion of which students would be eligible for 5 DPE units. It is
\end{flushleft}


\begin{flushleft}
mandatory that student's work during the one-semester internship is supervised by two mentors, one from the
\end{flushleft}


\begin{flushleft}
institute (appointed by the DPE Committee of the student's Department) and another from the host organization.
\end{flushleft}


\begin{flushleft}
In case of semester break for a start up, students will work under the mentorship of a faculty member of the
\end{flushleft}


\begin{flushleft}
Institute. Students desiring to opt for one semester internship or semester break for start-up as mentioned above
\end{flushleft}


\begin{flushleft}
are required to plan well in advance and submit a project proposal in consultation with their supervisors (in case
\end{flushleft}


\begin{flushleft}
of internship) or faculty mentor (in case of start-ups). Students can proceed with the internship / startup activity
\end{flushleft}


\begin{flushleft}
only after the Departmental Committee for DPE approves the same. DPE units for the activity would be awarded
\end{flushleft}


\begin{flushleft}
by the Departmental DPE Committee after evaluation at the end of the internship / startup period. In case an
\end{flushleft}


\begin{flushleft}
internship / startup pertains to areas of expertise outside those of the parent Department, the DPE Committee
\end{flushleft}


\begin{flushleft}
may co-opt faculty members from other Departments / Centres / Schools for evaluating / supervising such activities.
\end{flushleft}


\begin{flushleft}
Details of the procedure are given in section 8.6.3 on internships.
\end{flushleft}


\begin{flushleft}
A semester in which a student earns DPE units through semester-long internship or start-up as discussed above
\end{flushleft}


\begin{flushleft}
would be counted as a registered semester for graduation requirements. In case the DPE committee does not
\end{flushleft}


\begin{flushleft}
approve the award of 5 units for such activity, the semester would not be counted as a registered semester.
\end{flushleft}





\begin{flushleft}
8.6.2.6. One Time Design / Practical Experience Module (1 Unit)
\end{flushleft}


\begin{flushleft}
One time DPE modules can be offered by Institute faculty as well as working professionals who would like to
\end{flushleft}


\begin{flushleft}
engage students in a workshop / course related to design / practical experience. A proposal for such a module
\end{flushleft}


\begin{flushleft}
should be sent by faculty member coordinating the course through the concerned Department / Centre / School
\end{flushleft}


\begin{flushleft}
to the Institute DPE Committee for approval. These modules can be typically of 28-42 hours duration, and
\end{flushleft}


\begin{flushleft}
may be offered during mid-semester breaks, winter/summer vacations and even during non-class hours
\end{flushleft}


\begin{flushleft}
during the semester.
\end{flushleft}


\begin{flushleft}
Table 8.3 summarizes the information presented in section 8.6.2. Detailed rules pertaining to internships and
\end{flushleft}


\begin{flushleft}
their administration are given in section 8.6.3.
\end{flushleft}





86





\begin{flushleft}
\newpage
Courses of Study 2017-2018
\end{flushleft}





\begin{flushleft}
Table 8.3 : Implementation and Evaluation Plan for Design / Practical Experience Units
\end{flushleft}





\begin{flushleft}
Activity
\end{flushleft}


\begin{flushleft}
Courses with design focus
\end{flushleft}


\begin{flushleft}
(which are primarily design
\end{flushleft}


\begin{flushleft}
courses or have significant
\end{flushleft}


\begin{flushleft}
design component)
\end{flushleft}





\begin{flushleft}
Norms for the Activity
\end{flushleft}


\begin{flushleft}
Courses offered as per
\end{flushleft}


\begin{flushleft}
Institute procedure
\end{flushleft}





\begin{flushleft}
Criteria for awarding Units
\end{flushleft}


\begin{flushleft}
Registration by the student
\end{flushleft}


\begin{flushleft}
in the respective course;
\end{flushleft}


\begin{flushleft}
Evaluation by course coordinator;
\end{flushleft}


\begin{flushleft}
Moderation by DPE committee of
\end{flushleft}


\begin{flushleft}
Department / Institute
\end{flushleft}





\begin{flushleft}
Courses with optional
\end{flushleft}


\begin{flushleft}
Course Coordinator
\end{flushleft}


\begin{flushleft}
design/practical experience provides intimation to
\end{flushleft}


\begin{flushleft}
component
\end{flushleft}


\begin{flushleft}
Departmental / Institute DPE
\end{flushleft}


\begin{flushleft}
Committee about offering
\end{flushleft}


\begin{flushleft}
optional design units prior to
\end{flushleft}


\begin{flushleft}
commencement of the course
\end{flushleft}


\begin{flushleft}
4-week project with
\end{flushleft}


\begin{flushleft}
Institute Faculty during
\end{flushleft}


\begin{flushleft}
winter/ summer (20
\end{flushleft}


\begin{flushleft}
working days)
\end{flushleft}





\begin{flushleft}
Notification of projects by
\end{flushleft}


\begin{flushleft}
DPE Committee of Student's
\end{flushleft}


\begin{flushleft}
Department / Institute
\end{flushleft}





\begin{flushleft}
8-week project with
\end{flushleft}


\begin{flushleft}
Institute Faculty including
\end{flushleft}


\begin{flushleft}
SURA, DISA, etc. (40
\end{flushleft}


\begin{flushleft}
working days)
\end{flushleft}





\begin{flushleft}
Notification of projects by
\end{flushleft}


\begin{flushleft}
DPE Committee of Student's
\end{flushleft}


\begin{flushleft}
Department / Institute
\end{flushleft}


\begin{flushleft}
OR
\end{flushleft}


\begin{flushleft}
Announcement and selection
\end{flushleft}


\begin{flushleft}
by appropriate Institute
\end{flushleft}


\begin{flushleft}
bodies
\end{flushleft}





\begin{flushleft}
8-week internship
\end{flushleft}


\begin{flushleft}
during summer with
\end{flushleft}


\begin{flushleft}
Industry / R\&D / University
\end{flushleft}


\begin{flushleft}
(40 working days)
\end{flushleft}





\begin{flushleft}
Arranged by T\&P or selfarranged by the student in
\end{flushleft}


\begin{flushleft}
coordination with T\&P
\end{flushleft}





\begin{flushleft}
One semester internship
\end{flushleft}


\begin{flushleft}
(100 working days) or One
\end{flushleft}


\begin{flushleft}
semester break for own
\end{flushleft}


\begin{flushleft}
start-ups (singly or jointly)
\end{flushleft}





\begin{flushleft}
Arranged by T\&P or selfarranged by the student in
\end{flushleft}


\begin{flushleft}
coordination with T\&P for
\end{flushleft}


\begin{flushleft}
internships
\end{flushleft}





87





\begin{flushleft}
Student to raise request online
\end{flushleft}


\begin{flushleft}
for prior permission, forwarded
\end{flushleft}


\begin{flushleft}
by course coordinator; Prior
\end{flushleft}


\begin{flushleft}
Approval by DPE coordinator;
\end{flushleft}


\begin{flushleft}
Evaluation by course coordinator;
\end{flushleft}


\begin{flushleft}
Moderation by DPE committee of
\end{flushleft}


\begin{flushleft}
Department / Institute
\end{flushleft}


\begin{flushleft}
Student to raise request online for
\end{flushleft}


\begin{flushleft}
prior permission; Prior approval
\end{flushleft}


\begin{flushleft}
by DPE Committee of Student's
\end{flushleft}


\begin{flushleft}
Department; Evaluation by
\end{flushleft}


\begin{flushleft}
Faculty Supervisor of the project;
\end{flushleft}


\begin{flushleft}
Completion approval request
\end{flushleft}


\begin{flushleft}
by student forwarded through
\end{flushleft}


\begin{flushleft}
supervisor; Moderation by DPE
\end{flushleft}


\begin{flushleft}
committee of Department / Institute
\end{flushleft}


\begin{flushleft}
Student to raise request online for
\end{flushleft}


\begin{flushleft}
prior permission; Prior approval
\end{flushleft}


\begin{flushleft}
by DPE Committee of Student's
\end{flushleft}


\begin{flushleft}
Department; Evaluation by Faculty
\end{flushleft}


\begin{flushleft}
Mentor of the project / appropriate
\end{flushleft}


\begin{flushleft}
committee; Completion
\end{flushleft}


\begin{flushleft}
approval request by student
\end{flushleft}


\begin{flushleft}
forwarded through supervisor;
\end{flushleft}


\begin{flushleft}
Moderation by DPE committee of
\end{flushleft}


\begin{flushleft}
Department / Institute
\end{flushleft}


\begin{flushleft}
Student to raise request online for
\end{flushleft}


\begin{flushleft}
prior permission; Prior approval
\end{flushleft}


\begin{flushleft}
by DPE Committee of Student's
\end{flushleft}


\begin{flushleft}
Department; Monitoring by
\end{flushleft}


\begin{flushleft}
Internship supervisor; Completion
\end{flushleft}


\begin{flushleft}
approval request by student
\end{flushleft}


\begin{flushleft}
forwarded through supervisor;
\end{flushleft}


\begin{flushleft}
Evaluation and Moderation by DPE
\end{flushleft}


\begin{flushleft}
committee of Department / Institute
\end{flushleft}


\begin{flushleft}
Student to raise request online for
\end{flushleft}


\begin{flushleft}
prior permission; Prior approval
\end{flushleft}


\begin{flushleft}
of Institute DPE Committee on
\end{flushleft}


\begin{flushleft}
recommendation from DPE
\end{flushleft}


\begin{flushleft}
committee of Student's Department;
\end{flushleft}


\begin{flushleft}
Monitoring by Internship supervisor;
\end{flushleft}


\begin{flushleft}
Completion approval request
\end{flushleft}


\begin{flushleft}
by student forwarded through
\end{flushleft}


\begin{flushleft}
supervisor; Evaluation and
\end{flushleft}


\begin{flushleft}
Moderation by DPE committee of
\end{flushleft}


\begin{flushleft}
Department / Institute
\end{flushleft}





\begin{flushleft}
No. of
\end{flushleft}


\begin{flushleft}
Units
\end{flushleft}


\begin{flushleft}
Min
\end{flushleft}





\begin{flushleft}
Max
\end{flushleft}





1





2





1





2





1





1





2





2





2





2





5





5





\begin{flushleft}
\newpage
Courses of Study 2017-2018
\end{flushleft}





\begin{flushleft}
Participation in design/
\end{flushleft}


\begin{flushleft}
project activity under the
\end{flushleft}


\begin{flushleft}
supervision of faculty
\end{flushleft}


\begin{flushleft}
during semester
\end{flushleft}





\begin{flushleft}
Notification of projects by
\end{flushleft}


\begin{flushleft}
DPE Committee of Student's
\end{flushleft}


\begin{flushleft}
Department / Institute
\end{flushleft}





\begin{flushleft}
Participation in design/
\end{flushleft}


\begin{flushleft}
practical/ experience
\end{flushleft}


\begin{flushleft}
workshop/course/event
\end{flushleft}


\begin{flushleft}
organized by industry/
\end{flushleft}


\begin{flushleft}
other institutions or visitors
\end{flushleft}


\begin{flushleft}
including visiting faculty
\end{flushleft}


\begin{flushleft}
Participation in design/
\end{flushleft}


\begin{flushleft}
innovation activities of
\end{flushleft}


\begin{flushleft}
clubs (eg. Robotics, IGEM,
\end{flushleft}


\begin{flushleft}
etc.)
\end{flushleft}





\begin{flushleft}
Proposal for activity to
\end{flushleft}


\begin{flushleft}
be recommended by
\end{flushleft}


\begin{flushleft}
Department DPE Committee
\end{flushleft}


\begin{flushleft}
and approved by Institute
\end{flushleft}


\begin{flushleft}
DPE Committee
\end{flushleft}


\begin{flushleft}
Notification by the Faculty incharge of the corresponding
\end{flushleft}


\begin{flushleft}
activity
\end{flushleft}





\begin{flushleft}
Student to raise request online for
\end{flushleft}


\begin{flushleft}
prior permission; Prior approval
\end{flushleft}


\begin{flushleft}
by DPE Committee of Student's
\end{flushleft}


\begin{flushleft}
Department; Evaluation by Faculty
\end{flushleft}


\begin{flushleft}
Mentor of the project; Completion
\end{flushleft}


\begin{flushleft}
approval request by student
\end{flushleft}


\begin{flushleft}
forwarded through supervisor;
\end{flushleft}


\begin{flushleft}
Moderation by DPE committee of
\end{flushleft}


\begin{flushleft}
Department / Institute
\end{flushleft}


\begin{flushleft}
Registration by the student
\end{flushleft}


\begin{flushleft}
in the activity; Evaluation by
\end{flushleft}


\begin{flushleft}
Faculty Coordinator and Visiting
\end{flushleft}


\begin{flushleft}
Faculty offering the course if any;
\end{flushleft}


\begin{flushleft}
Moderation by DPE committee of
\end{flushleft}


\begin{flushleft}
Department / Institute
\end{flushleft}


\begin{flushleft}
Student to raise request online for
\end{flushleft}


\begin{flushleft}
prior permission; Prior approval
\end{flushleft}


\begin{flushleft}
by DPE Committee of Student's
\end{flushleft}


\begin{flushleft}
Department; Evaluation by
\end{flushleft}


\begin{flushleft}
faculty in-charge of activity / clubs;
\end{flushleft}


\begin{flushleft}
Completion approval request
\end{flushleft}


\begin{flushleft}
by student forwarded through
\end{flushleft}


\begin{flushleft}
supervisor; Moderation by Institute
\end{flushleft}


\begin{flushleft}
DPE committee
\end{flushleft}





1





2





1





1





1





2





$\bullet$	





\begin{flushleft}
A student cannot register for more than 3 non-graded DPE units per summer semester or per registered
\end{flushleft}


\begin{flushleft}
semester in which a student is on regular academic activity. To take part in activities that can result in more
\end{flushleft}


\begin{flushleft}
than 3 DPE units, a student has to take the semester off from regular courses.
\end{flushleft}





$\bullet$	





\begin{flushleft}
A single activity cannot be evaluated for more than one purpose. For example, the same project cannot
\end{flushleft}


\begin{flushleft}
be submitted for graded credits as well as for design units.
\end{flushleft}





\begin{flushleft}
8.6.3	 Rules Governing Internship
\end{flushleft}


\begin{flushleft}
i)	
\end{flushleft}





\begin{flushleft}
Internships for DPE units are permitted only in one of the two following formats:
\end{flushleft}





	





\begin{flushleft}
a.	 Summer internship of 40 days duration, in which a student can earn 2 DPE units.
\end{flushleft}





	





\begin{flushleft}
b.	 Semester-long internship of 100 days duration, in which a student can earn 5 DPE units.
\end{flushleft}





	





\begin{flushleft}
No other format of internship would be considered for the award of DPE units. DPE units would be
\end{flushleft}


\begin{flushleft}
awarded only if training for the stipulated number of working days, as mentioned above, is completed
\end{flushleft}


\begin{flushleft}
to the satisfaction of the concerned Departmental DPE Committee. DPE units would not be awarded
\end{flushleft}


\begin{flushleft}
against partial completion of the internship duration.
\end{flushleft}





\begin{flushleft}
ii)	 A student can choose from one of the following options in order to complete the Non-Graded component
\end{flushleft}


\begin{flushleft}
of Design / Practical Experience:
\end{flushleft}


	





\begin{flushleft}
a.	 One semester internship, accounting for 5 DPE units.
\end{flushleft}





	





\begin{flushleft}
b.	
\end{flushleft}





\begin{flushleft}
One summer internship, accounting for 2 DPE units and 3 DPE units from other activities at the Institute
\end{flushleft}





	





\begin{flushleft}
c.	
\end{flushleft}





\begin{flushleft}
Two summer internships, accounting for a total of 4 DPE units, and 1 DPE unit from other activities
\end{flushleft}


\begin{flushleft}
at the Institute
\end{flushleft}





	





\begin{flushleft}
d.	
\end{flushleft}





\begin{flushleft}
One summer internship accounting for 2 DPE units and one semester internship, accounting for 5
\end{flushleft}


\begin{flushleft}
DPE units.
\end{flushleft}





	





\begin{flushleft}
e.	 No internships: all DPE units can be earned through design / project activities at the Institute
\end{flushleft}





\begin{flushleft}
iii)	 A student can do at most two internships for DPE units, during his/her stay at the Institute. If any student
\end{flushleft}


\begin{flushleft}
does more than two internships, DPE units will be awarded for the first two registered internships only.
\end{flushleft}


\begin{flushleft}
iv)	 Summer internships are allowed in the summer after the 4th registered semester of the student or later.
\end{flushleft}


\begin{flushleft}
Semester Internships are permitted from the seventh registered semester or later.
\end{flushleft}


88





\begin{flushleft}
\newpage
Courses of Study 2017-2018
\end{flushleft}





\begin{flushleft}
v)	 Internships are permitted in industry, research laboratories or academic institutions involved in research,
\end{flushleft}


\begin{flushleft}
development and/or technology transfer. Any student opting for semester long internship may also be
\end{flushleft}


\begin{flushleft}
allowed to work on a start-up. All internships must be approved by the departmental DPE committee in
\end{flushleft}


\begin{flushleft}
advance. In the case of non-industry internships, the work should be research / development / practice
\end{flushleft}


\begin{flushleft}
oriented, and not classroom course work.
\end{flushleft}


\begin{flushleft}
vi)	 In all cases, for award of DPE units, after completion of the internship, the work must be evaluated by
\end{flushleft}


\begin{flushleft}
the DPE committee of the student's Department. In case the work is found wanting in any respect, the
\end{flushleft}


\begin{flushleft}
student(s) will be advised to do more work and reappear before the committee. In any case, partial award
\end{flushleft}


\begin{flushleft}
of DPE units would not be allowed.
\end{flushleft}


\begin{flushleft}
vii)	 Both for self-arranged internships and for internships arranged through T\&P Unit, administration and
\end{flushleft}


\begin{flushleft}
correspondence would be handled by the Training and Placement Unit. For self-arranged internships,
\end{flushleft}


\begin{flushleft}
any documentation regarding the bona fide status of students (while applying for training) will be provided
\end{flushleft}


\begin{flushleft}
by UG section. T\&P Unit will process the internship case of the student once the student submits all
\end{flushleft}


\begin{flushleft}
departmental approvals and the confirmed offer letter from the company to T\&P.
\end{flushleft}





\begin{flushleft}
8.6.3.1. Registration Procedure for Internships
\end{flushleft}


\begin{flushleft}
Summer Internships:
\end{flushleft}


\begin{flushleft}
i)	
\end{flushleft}





\begin{flushleft}
At the beginning of first semester of each academic year, the data of all students who have earned at
\end{flushleft}


\begin{flushleft}
least 30 credits would be automatically enrolled by the T\&P unit for internship in the subsequent summer.
\end{flushleft}





\begin{flushleft}
ii)	 At the beginning of the internship in the following summer, the student must have completed 50 credits
\end{flushleft}


\begin{flushleft}
to be eligible.
\end{flushleft}


\begin{flushleft}
iii)	 T\&P unit would allow the students to opt out of the process for allocation of internships until a specified
\end{flushleft}


\begin{flushleft}
deadline, if a student would like to try for self-arranged internship.
\end{flushleft}


\begin{flushleft}
iv)	 Students who do not opt out of the process are considered for allocation of internship by the T\&P unit.
\end{flushleft}


\begin{flushleft}
If a student is selected for an internship through T\&P, he/she is bound to accept the internship. If the
\end{flushleft}


\begin{flushleft}
student does not take up or complete the internship, he/she will be debarred from all further T\&P activities
\end{flushleft}


\begin{flushleft}
including further internship opportunities and placement procedure. This is to discourage non-serious
\end{flushleft}


\begin{flushleft}
students from depriving other students of the opportunity, and damaging the reputation of IIT Delhi with
\end{flushleft}


\begin{flushleft}
the companies offering internships through T\&P.
\end{flushleft}


\begin{flushleft}
v)	 The T\&P Unit would handle correspondences and training certificates of all internships, both self-arranged
\end{flushleft}


\begin{flushleft}
and those arranged by the T\&P unit.
\end{flushleft}


\begin{flushleft}
vi)	 T\&P Unit will try and arrange internships for as many students as it can. However, it may not be possible
\end{flushleft}


\begin{flushleft}
for the T\&P Unit to arrange internships for all the students who participate in the process.
\end{flushleft}


\begin{flushleft}
vii)	 The T\&P unit typically starts the process of selections for internships in August and ends in FebruaryMarch. The exact dates would be notified by the T\&P unit each year.
\end{flushleft}


\begin{flushleft}
viii)	 T\&P unit would also notify students about the deadlines to submit documents related to self-arranged
\end{flushleft}


\begin{flushleft}
internship. Only those students who submit relevant papers by this deadline will be considered for the
\end{flushleft}


\begin{flushleft}
internship.
\end{flushleft}


\begin{flushleft}
ix)	 At the end of the process, T\&P will send a list of students whose internships are to be approved, to
\end{flushleft}


\begin{flushleft}
the respective departments. The internship coordinators of the department will then enroll the students
\end{flushleft}


\begin{flushleft}
on the online portal for non-graded units, in one of the two courses XXT200 or XXT300. A student will
\end{flushleft}


\begin{flushleft}
be enrolled in XXT200 if it is his/her first summer internship. Otherwise the student will be enrolled in
\end{flushleft}


\begin{flushleft}
XXT300.
\end{flushleft}


\begin{flushleft}
Semester Internship:
\end{flushleft}


\begin{flushleft}
i)	
\end{flushleft}





\begin{flushleft}
Semester internship, as mentioned in section 8.6.2.5, is permitted in the seventh registered semester
\end{flushleft}


\begin{flushleft}
or later, for students with at least 75 earned credits.
\end{flushleft}





\begin{flushleft}
ii)	 A student needs to submit an online request for prior approval of semester internship. The request for
\end{flushleft}


\begin{flushleft}
internship will evaluated by the DPE committee of the student's parent Department and approved by
\end{flushleft}


\begin{flushleft}
the Institute DPE committee upon recommendation of the former.
\end{flushleft}


\begin{flushleft}
iii)	 Process of monitoring / mentoring the internship is described in section 8.6.2.5. Upon completion, the
\end{flushleft}


\begin{flushleft}
student should submit an online request for approval of the completion of the internship through the
\end{flushleft}


\begin{flushleft}
supervisor and Departmental DPE committee to the Institute DPE committee. The grade for semester
\end{flushleft}


\begin{flushleft}
internship is awarded by the Institute DPE committee.
\end{flushleft}


89





\begin{flushleft}
\newpage
Courses of Study 2017-2018
\end{flushleft}





\begin{flushleft}
The list of courses offered in connection with non-graded units listed in sections 8.1-8.6 along with the respective
\end{flushleft}


\begin{flushleft}
pre-requisites is summarized in Table 8.4.
\end{flushleft}


\begin{flushleft}
Table 8.4 : Summary of courses for non-graded unit
\end{flushleft}


\begin{flushleft}
S. No.
\end{flushleft}





\begin{flushleft}
Course Number
\end{flushleft}





\begin{flushleft}
Course Name and / or Description
\end{flushleft}





\begin{flushleft}
Pre-Requisites
\end{flushleft}





\begin{flushleft}
No. of Units
\end{flushleft}





\begin{flushleft}
Introduction to Engineering \& Programme: 02 units
\end{flushleft}


1





\begin{flushleft}
NIN100
\end{flushleft}





\begin{flushleft}
Introduction to Engineering in the first semester
\end{flushleft}





2





\begin{flushleft}
XXN101/ XXN111
\end{flushleft}





\begin{flushleft}
Introduction to $<$the respective engineering
\end{flushleft}


\begin{flushleft}
programme$>$ in the third semester
\end{flushleft}





----





1





\begin{flushleft}
NIN100
\end{flushleft}





1





\begin{flushleft}
Language and Writing Skills: 02 units
\end{flushleft}


3





\begin{flushleft}
NLN100
\end{flushleft}





\begin{flushleft}
Language and Writing Skills -- I in I semester
\end{flushleft}





----





1





4





\begin{flushleft}
NLN101
\end{flushleft}





\begin{flushleft}
Language and Writing Skills -- II in II semester
\end{flushleft}





\begin{flushleft}
NLN100
\end{flushleft}





1





\begin{flushleft}
NCC / NSO / NSS: 02 units
\end{flushleft}


5





\begin{flushleft}
NCN100
\end{flushleft}





\begin{flushleft}
NCC
\end{flushleft}





----





2





6





\begin{flushleft}
NPN100
\end{flushleft}





\begin{flushleft}
NSO
\end{flushleft}





----





2





7





\begin{flushleft}
NSN100
\end{flushleft}





\begin{flushleft}
NSS
\end{flushleft}





----





2





\begin{flushleft}
Professional Ethics and Social Responsibility: 02 units
\end{flushleft}


8





\begin{flushleft}
NEN100
\end{flushleft}





\begin{flushleft}
Professional Ethics and Social Responsibility -- I
\end{flushleft}


\begin{flushleft}
in first semester -- 15 hours
\end{flushleft}





9





\begin{flushleft}
NEN101
\end{flushleft}





\begin{flushleft}
Professional Ethics and Social Responsibility -- II
\end{flushleft}


\begin{flushleft}
in second semester -- 15 hours
\end{flushleft}





10





\begin{flushleft}
NEN201
\end{flushleft}





\begin{flushleft}
PESR Internships: 20 working days followed by a
\end{flushleft}


\begin{flushleft}
presentation and evaluation
\end{flushleft}





11





\begin{flushleft}
NEN202
\end{flushleft}





\begin{flushleft}
PESR Workshops: 40 hours followed by
\end{flushleft}


\begin{flushleft}
presentation and evaluation
\end{flushleft}





\begin{flushleft}
NEN203
\end{flushleft}





\begin{flushleft}
PESR projects: 50 man hours of work followed by
\end{flushleft}


\begin{flushleft}
presentation and evaluation
\end{flushleft}





\begin{flushleft}
NEN300
\end{flushleft}





\begin{flushleft}
Case Studies in Professional Ethics
\end{flushleft}





12


13





----





\begin{flushleft}
NEN100
\end{flushleft}





\begin{flushleft}
(Any one of the
\end{flushleft}


\begin{flushleft}
three) NEN101
\end{flushleft}





\begin{flushleft}
(Both)
\end{flushleft}


1





1





\begin{flushleft}
NEN101
\end{flushleft}





\begin{flushleft}
Communication Skills / Seminar: 02 units
\end{flushleft}


14





\begin{flushleft}
XXQ301, XXQ302, etc.
\end{flushleft}





\begin{flushleft}
Topic specific Seminar courses introduced by
\end{flushleft}


\begin{flushleft}
parent Department
\end{flushleft}





\begin{flushleft}
EC 50
\end{flushleft}





15





\begin{flushleft}
XYQ301, XYQ302, etc.
\end{flushleft}





\begin{flushleft}
Additional Seminar courses introduced by any
\end{flushleft}


\begin{flushleft}
other Department/ Centre/ School
\end{flushleft}





\begin{flushleft}
EC 50
\end{flushleft}





16





\begin{flushleft}
NQN301
\end{flushleft}





\begin{flushleft}
Seminar component of regular courses OR
\end{flushleft}


\begin{flushleft}
Three extracurricular activities involving
\end{flushleft}


\begin{flushleft}
communication skills
\end{flushleft}





1





\begin{flushleft}
EC 50
\end{flushleft}





\begin{flushleft}
(Any one)
\end{flushleft}


1





\begin{flushleft}
Design/ Practical Experience: 05 units
\end{flushleft}


17





\begin{flushleft}
XXD351
\end{flushleft}





\begin{flushleft}
Minor Design Project -- 1
\end{flushleft}





\begin{flushleft}
EC 30
\end{flushleft}





1





18





\begin{flushleft}
XXD352
\end{flushleft}





\begin{flushleft}
Minor Design Project -- 2
\end{flushleft}





\begin{flushleft}
EC 30
\end{flushleft}





1





19





\begin{flushleft}
XXD353
\end{flushleft}





\begin{flushleft}
Minor Design Project -- 3
\end{flushleft}





\begin{flushleft}
EC 30
\end{flushleft}





1





20





\begin{flushleft}
XXD354
\end{flushleft}





\begin{flushleft}
Minor Design Project -- 4
\end{flushleft}





\begin{flushleft}
EC 30
\end{flushleft}





1





21





\begin{flushleft}
XXD355
\end{flushleft}





\begin{flushleft}
Minor Design Project -- 5
\end{flushleft}





\begin{flushleft}
EC 30
\end{flushleft}





1





90





\begin{flushleft}
\newpage
Courses of Study 2017-2018
\end{flushleft}





22





\begin{flushleft}
XXD356
\end{flushleft}





\begin{flushleft}
Minor Design Project -- 6
\end{flushleft}





\begin{flushleft}
EC 30
\end{flushleft}





2





23





\begin{flushleft}
XXD357
\end{flushleft}





\begin{flushleft}
Minor Design Project -- 7
\end{flushleft}





\begin{flushleft}
EC 30
\end{flushleft}





2





24





\begin{flushleft}
XXD358
\end{flushleft}





\begin{flushleft}
Minor Design Project -- 8
\end{flushleft}





\begin{flushleft}
EC 30
\end{flushleft}





2





25





\begin{flushleft}
NDN351
\end{flushleft}





\begin{flushleft}
Minor Design Project -- 1
\end{flushleft}





\begin{flushleft}
EC 30
\end{flushleft}





1





26





\begin{flushleft}
NDN352
\end{flushleft}





\begin{flushleft}
Minor Design Project -- 2
\end{flushleft}





\begin{flushleft}
EC 30
\end{flushleft}





1





27





\begin{flushleft}
NDN353
\end{flushleft}





\begin{flushleft}
Minor Design Project -- 3
\end{flushleft}





\begin{flushleft}
EC 30
\end{flushleft}





1





28





\begin{flushleft}
NDN354
\end{flushleft}





\begin{flushleft}
Minor Design Project -- 4
\end{flushleft}





\begin{flushleft}
EC 30
\end{flushleft}





1





29





\begin{flushleft}
NDN355
\end{flushleft}





\begin{flushleft}
Minor Design Project -- 5
\end{flushleft}





\begin{flushleft}
EC 30
\end{flushleft}





1





30





\begin{flushleft}
NDN356
\end{flushleft}





\begin{flushleft}
Minor Design Project -- 6
\end{flushleft}





\begin{flushleft}
EC 30
\end{flushleft}





2





31





\begin{flushleft}
NDN357
\end{flushleft}





\begin{flushleft}
Minor Design Project -- 7
\end{flushleft}





\begin{flushleft}
EC 30
\end{flushleft}





2





32





\begin{flushleft}
NDN358
\end{flushleft}





\begin{flushleft}
Minor Design Project -- 8
\end{flushleft}





\begin{flushleft}
EC 30
\end{flushleft}





2





33





\begin{flushleft}
XXT200
\end{flushleft}





\begin{flushleft}
Summer Internship -- 1
\end{flushleft}





\begin{flushleft}
EC 30
\end{flushleft}





2





34





\begin{flushleft}
XXT300
\end{flushleft}





\begin{flushleft}
Summer Internship -- 2
\end{flushleft}





\begin{flushleft}
XXT200
\end{flushleft}





2





35





\begin{flushleft}
XXT400
\end{flushleft}





\begin{flushleft}
Semester Internship
\end{flushleft}





\begin{flushleft}
EC 75
\end{flushleft}





5





\begin{flushleft}
In all above course descriptions, XX and XY stand for the two-letter prefix corresponding to course numbers of
\end{flushleft}


\begin{flushleft}
academic units (Table 1 of Chapter 1): example, XXD351 corresponding to Department of Textile Technology
\end{flushleft}


\begin{flushleft}
would be TXL351.
\end{flushleft}





\begin{flushleft}
8.7 Overlapping Activities
\end{flushleft}


\begin{flushleft}
Many of the activities listed in sections 8.1-8.6 could also qualify as valid activities under other non-graded
\end{flushleft}


\begin{flushleft}
components: for example, an internship under NEN201 may qualify to be a valid NSS activity; a technical project
\end{flushleft}


\begin{flushleft}
done as part of NEN203 may qualify to be submitted for DPE units under XXD35y, etc. Some of the technical
\end{flushleft}


\begin{flushleft}
projects may also qualify as valid activities under Minor/ Mini/ Major projects towards earning graded credits. In
\end{flushleft}


\begin{flushleft}
this regard, the following would be strictly followed:
\end{flushleft}


\begin{flushleft}
a)	 In case a project is evaluated for graded credits or for any other non-graded activity, it would not be
\end{flushleft}


\begin{flushleft}
allowed to be re-submitted for any other non-graded unit. While submitting the completion request of
\end{flushleft}


\begin{flushleft}
the project online, a student should submit an undertaking to this effect, approved online by the faculty
\end{flushleft}


\begin{flushleft}
supervisor of the project.
\end{flushleft}


\begin{flushleft}
b)	 Additional work which is not evaluated for such projects, either done prior to such projects or done after
\end{flushleft}


\begin{flushleft}
the completion of such projects, could be considered. In such a case, prior written permission must be
\end{flushleft}


\begin{flushleft}
taken from the concerned committee (PESR, DPE, etc.), explicitly describing components of work being
\end{flushleft}


\begin{flushleft}
submitted for the other graded / non-graded evaluations and for the current submission separately. In
\end{flushleft}


\begin{flushleft}
this regard, note the following examples:
\end{flushleft}


\begin{flushleft}
(i)	 The workshops organised by NSS and under NEN202 would be generally distinct. Under NEN202,
\end{flushleft}


\begin{flushleft}
workshops would have minimum duration of 3 days and would be designated as {``}PESR WORKSHOP''.
\end{flushleft}


\begin{flushleft}
Workshops organized by NSS would not be counted for NEN202 and vice-versa.
\end{flushleft}


\begin{flushleft}
(ii)	 In case a student desires to do internship as part of NSS activities as well as under PESR through
\end{flushleft}


\begin{flushleft}
NEN201, each of these internships must have distinct time spans and special approvals of the PESR
\end{flushleft}


\begin{flushleft}
Committee and the NSS Coordinator are required for the same.
\end{flushleft}





91





\begin{flushleft}
\newpage
9. POSTGRADUATE PROGRAMME STRUCTURES
\end{flushleft}





\begin{flushleft}
\newpage
Programme Code:CYS
\end{flushleft}





\begin{flushleft}
Master of Science in Chemistry
\end{flushleft}


\begin{flushleft}
Department of Chemistry
\end{flushleft}





\begin{flushleft}
The overall credits structure
\end{flushleft}


\begin{flushleft}
Category
\end{flushleft}





\begin{flushleft}
PC
\end{flushleft}





\begin{flushleft}
PE
\end{flushleft}





\begin{flushleft}
OC
\end{flushleft}





\begin{flushleft}
Total
\end{flushleft}





\begin{flushleft}
Credits
\end{flushleft}





60





9





6





75





\begin{flushleft}
Program Core
\end{flushleft}





\begin{flushleft}
CMP511	Lab I	
\end{flushleft}


\begin{flushleft}
CMP512	Lab II	
\end{flushleft}


\begin{flushleft}
CMP521	Lab III	
\end{flushleft}


\begin{flushleft}
CMP522	Lab IV	
\end{flushleft}





\begin{flushleft}
CMD631	 Project Part I	
\end{flushleft}


0	 0	 12	6


\begin{flushleft}
CMD641	 Project Part II	
\end{flushleft}


0	 0	 20	10


\begin{flushleft}
CML511	 Quantum Chemistry	
\end{flushleft}


3	0	0	 3


\begin{flushleft}
CML512	 Stereochemistry \& Organic Reaction	
\end{flushleft}


3	 0	 0	 3


\begin{flushleft}
	Mechanisms
\end{flushleft}


\begin{flushleft}
CML513	 Photochemistry \& Pericyclic Reactions	
\end{flushleft}


3	 0	 0	 3


\begin{flushleft}
CML514	 Main Group Chemistry	
\end{flushleft}


3	 0	 0	 3


\begin{flushleft}
CML515	 Instrumental Methods of Analysis	
\end{flushleft}


3	 0	 0	 3


\begin{flushleft}
CML521	Molecular Thermodynamics	
\end{flushleft}


3	0	0	 3


\begin{flushleft}
CML522	 Chemical Dynamics \& Surface Chemistry	
\end{flushleft}


3	 0	 0	 3


\begin{flushleft}
CML523	Organic Synthesis	
\end{flushleft}


3	0	0	 3


\begin{flushleft}
CML524	 Transition and Inner Transition Metal Chemistry	3	0	0	 3
\end{flushleft}


\begin{flushleft}
CML525	 Basic Organometalic Chemistry	
\end{flushleft}


3	 0	 0	 3


\begin{flushleft}
CML526	 Structure \& Function of Cellular Biomolecules	 3	 0	 0	 3
\end{flushleft}


\begin{flushleft}
CML631	Molecular Biochemistry	
\end{flushleft}


3	0	0	 3





\begin{flushleft}
Total Credits						
\end{flushleft}





\begin{flushleft}
CML661	 Solid state chemistry	
\end{flushleft}


\begin{flushleft}
CML662	 Statistical Mechanics \& Molecular	
\end{flushleft}


	


\begin{flushleft}
Simulation Methods
\end{flushleft}


\begin{flushleft}
CML664	Microbial Biochemistry	
\end{flushleft}


\begin{flushleft}
CML665	Food Chemistry	
\end{flushleft}


\begin{flushleft}
CML671	 Applied organometallic Chemistry	
\end{flushleft}


\begin{flushleft}
CML672	Inorganic Polymers	
\end{flushleft}


\begin{flushleft}
CML673	 Structural Methods of Inorganic Compounds	
\end{flushleft}


\begin{flushleft}
CML739	Applied Biocatalysis	
\end{flushleft}





(3-0-0) 3





\begin{flushleft}
Stereochemistry
\end{flushleft}


\begin{flushleft}
\& Organic
\end{flushleft}


\begin{flushleft}
Reaction
\end{flushleft}


\begin{flushleft}
Mechanisms
\end{flushleft}





\begin{flushleft}
CML513
\end{flushleft}





\begin{flushleft}
Photochemistry
\end{flushleft}


\begin{flushleft}
\& Pericyclic
\end{flushleft}


\begin{flushleft}
Reactions
\end{flushleft}





(3-0-0) 3





\begin{flushleft}
CML514
\end{flushleft}





\begin{flushleft}
Main Group
\end{flushleft}


\begin{flushleft}
Chemistry
\end{flushleft}





\begin{flushleft}
CML515
\end{flushleft}





(3-0-0) 3





\begin{flushleft}
Instrumental Methods
\end{flushleft}


\begin{flushleft}
of Analysis
\end{flushleft}





\begin{flushleft}
CML524
\end{flushleft}





\begin{flushleft}
CML525
\end{flushleft}





3	0	0	 3


3	0	0	 3


3	 0	 0	 3


3	0	0	 3


3	 0	 0	 3


3	0	0	 3





\begin{flushleft}
Contact h/week
\end{flushleft}


\begin{flushleft}
L
\end{flushleft}





\begin{flushleft}
T
\end{flushleft}





\begin{flushleft}
P
\end{flushleft}





\begin{flushleft}
Total
\end{flushleft}





\begin{flushleft}
Credits
\end{flushleft}





\begin{flushleft}
Quantum
\end{flushleft}


\begin{flushleft}
Chemistry
\end{flushleft}





3	 0	 0	 3


3	 0	 0	 3





\begin{flushleft}
Lecture
\end{flushleft}


\begin{flushleft}
courses
\end{flushleft}





\begin{flushleft}
CML511 CML512
\end{flushleft}





60





\begin{flushleft}
Program Electives
\end{flushleft}





\begin{flushleft}
Courses
\end{flushleft}


\begin{flushleft}
(Number, Abbreviated Title, L-T-P, credits)
\end{flushleft}





\begin{flushleft}
Sem.
\end{flushleft}





\begin{flushleft}
I
\end{flushleft}





	





0	0	4	 2


0	0	4	 2


0	0	4	 2


0	0	4	 2





5





15





0





8





23





19





6





18





0





8





26





22





4





12





0





12





24





18





2





6





0





20





26





16





\begin{flushleft}
CMP501 CMP502
\end{flushleft}


\begin{flushleft}
Lab-I
\end{flushleft}





(0-0-4) 2





\begin{flushleft}
Lab-II
\end{flushleft}





(0-0-4) 2





(3-0-0) 3





(3-0-0) 3


\begin{flushleft}
CML521 CML522
\end{flushleft}


\begin{flushleft}
II
\end{flushleft}





\begin{flushleft}
Molecular
\end{flushleft}


\begin{flushleft}
Thermodynamics
\end{flushleft}





(3-0-0) 3





\begin{flushleft}
Chemical
\end{flushleft}


\begin{flushleft}
Dynamics
\end{flushleft}


\begin{flushleft}
\& Surface
\end{flushleft}


\begin{flushleft}
Chemistry
\end{flushleft}





\begin{flushleft}
CML523
\end{flushleft}


\begin{flushleft}
Organic
\end{flushleft}


\begin{flushleft}
Synthesis
\end{flushleft}





(3-0-0) 3





(3-0-0) 3





\begin{flushleft}
Transition
\end{flushleft}


\begin{flushleft}
and Inner
\end{flushleft}


\begin{flushleft}
Transition
\end{flushleft}


\begin{flushleft}
Metal
\end{flushleft}


\begin{flushleft}
Chemistry
\end{flushleft}





(3-0-0) 3





\begin{flushleft}
Basic
\end{flushleft}


\begin{flushleft}
Organometalic
\end{flushleft}


\begin{flushleft}
Chemistry
\end{flushleft}





(3-0-0) 3





\begin{flushleft}
CML526 CMP503 CMP504
\end{flushleft}


\begin{flushleft}
Structure \&
\end{flushleft}


\begin{flushleft}
Function
\end{flushleft}


\begin{flushleft}
of Cellular
\end{flushleft}


\begin{flushleft}
Biomolecules
\end{flushleft}





\begin{flushleft}
Lab-III
\end{flushleft}





\begin{flushleft}
Lab-IV
\end{flushleft}





(0-0-4) 2





(0-0-4) 2





(3-0-0) 3





\begin{flushleft}
Summer
\end{flushleft}





\begin{flushleft}
III
\end{flushleft}





\begin{flushleft}
CML631 PE-1
\end{flushleft}


\begin{flushleft}
Molecular (3-0-0) 3
\end{flushleft}


\begin{flushleft}
Biochemistry
\end{flushleft}





\begin{flushleft}
PE-2
\end{flushleft}


(3-0-0) 3





\begin{flushleft}
OE-1
\end{flushleft}


(3-0-0) 3





\begin{flushleft}
CMD611
\end{flushleft}


\begin{flushleft}
Project
\end{flushleft}


\begin{flushleft}
Part-I
\end{flushleft}





(0-0-12) 6





(3-0-0) 3


\begin{flushleft}
IV
\end{flushleft}





\begin{flushleft}
PE-3
\end{flushleft}


(3-0-0) 3





\begin{flushleft}
OE-2
\end{flushleft}


(3-0-0) 3





\begin{flushleft}
CMD621
\end{flushleft}





\begin{flushleft}
Project Part-II
\end{flushleft}





(0-0-20) 10





\begin{flushleft}
Total = 75
\end{flushleft}


93





\begin{flushleft}
\newpage
Programme Code: MAS
\end{flushleft}





\begin{flushleft}
Master of Science in Mathematics
\end{flushleft}


\begin{flushleft}
Department of Mathematics
\end{flushleft}


\begin{flushleft}
The overall credits structure
\end{flushleft}


\begin{flushleft}
Category
\end{flushleft}





\begin{flushleft}
PC
\end{flushleft}





\begin{flushleft}
PE
\end{flushleft}





\begin{flushleft}
OC
\end{flushleft}





\begin{flushleft}
Total
\end{flushleft}





\begin{flushleft}
Credits
\end{flushleft}





57





12





6





75


\begin{flushleft}
MTL743	 Fourier Analysis	
\end{flushleft}


3	0	0	3


\begin{flushleft}
MTL744	 Mathematical Theory of Coding	
\end{flushleft}


3	0	0	3


\begin{flushleft}
MTL745	 Advanced Matrix Theory 	
\end{flushleft}


3	0	0	3


\begin{flushleft}
MTL746	 Methods of Applied Mathematics	
\end{flushleft}


3	0	0	3


\begin{flushleft}
MTL747	 Mathematical Logic	
\end{flushleft}


3	0	0	3


\begin{flushleft}
MTL751	 Symbolic Dynamics	
\end{flushleft}


3	0	0	3


\begin{flushleft}
MTL755	 Algebraic Geometry	
\end{flushleft}


3	 0	 0	 3


\begin{flushleft}
MTL756	 Lie Algebras and Lie Groups	
\end{flushleft}


3	 0	 0	 3


\begin{flushleft}
MTL757	 Introduction to Algebraic Topology	
\end{flushleft}


3	0	0	3


\begin{flushleft}
MTL760	 Advanced Algorithms	
\end{flushleft}


3	0	0	3


\begin{flushleft}
MTL761	 Basic Ergodic Theory	
\end{flushleft}


3	0	0	3


\begin{flushleft}
MTL762	 Probability Theory	
\end{flushleft}


3	0	0	3


\begin{flushleft}
MTL763	 Introduction to Game Theory	
\end{flushleft}


3	 0	 0	 3


\begin{flushleft}
MTL766	 Multivariate Statistical Methods	
\end{flushleft}


3	0	0	3


\begin{flushleft}
MTL768	 Graph Theory	
\end{flushleft}


3	 0	 0	 3


\begin{flushleft}
MTL773	 Wavelets and Applications 	
\end{flushleft}


3	0	0	3


\begin{flushleft}
MTL781	 Finite Element Theory and Applications	
\end{flushleft}


3	0	0	3


\begin{flushleft}
MTL785	 Natural Language Processing	
\end{flushleft}


3	0	0	3


\begin{flushleft}
MTL792	 Modern Methods in Partial Differential equations	 3	0	0	3
\end{flushleft}


\begin{flushleft}
MTL793	 Numerical Methods for Hyperbolic PDEs	
\end{flushleft}


3	 0	 0	 3


\begin{flushleft}
MTL794	 Advanced Probability Theory	
\end{flushleft}


3	0	0	3


\begin{flushleft}
MTL795	 Numerical Method for Partial Differential Equations	3	1	0	4
\end{flushleft}


\begin{flushleft}
MTV791	 Special Module in Dynamical System	
\end{flushleft}


1	 0	 0	 1


\begin{flushleft}
MTL843	 Mathematical Modeling of Credit Risk	
\end{flushleft}


3	0	0	3


\begin{flushleft}
MTL851	 Applied Numerical Analysis	
\end{flushleft}


3	0	0	3


\begin{flushleft}
MTL854	 Interpolation and Approximation	
\end{flushleft}


3	0	0	3


\begin{flushleft}
MTL855	 Multiple Decision Procedures in Ranking	
\end{flushleft}


3	0	0	3


	


\begin{flushleft}
and Selection
\end{flushleft}


\begin{flushleft}
MTL860	 Linear Algebra	
\end{flushleft}


3	0	0	3


\begin{flushleft}
MTL863	 Algebraic Number Theory	
\end{flushleft}


3	0	0	3


\begin{flushleft}
MTV874	 Analysis	
\end{flushleft}


3	0	0	3


\begin{flushleft}
MTL882	 Applied Analysis	
\end{flushleft}


3	0	0	3


\begin{flushleft}
MTL883	 Physical Fluid Mechanics	
\end{flushleft}


3	0	0	3


\begin{flushleft}
MTL888	 Boundary Elements Methods with Computer 	 3	0	0	3 		
\end{flushleft}


	


\begin{flushleft}
Implementation	
\end{flushleft}





\begin{flushleft}
Program Electives
\end{flushleft}


\begin{flushleft}
MTD702	Project-II	
\end{flushleft}


0	0	12	


6


\begin{flushleft}
MTL625	 Principles of Optimization Theory	
\end{flushleft}


3	0	0	3


\begin{flushleft}
MTL704	 Numerical Optimization	
\end{flushleft}


3	0	0	3


\begin{flushleft}
MTL712	 Computational Methods for Differential Equations	 3	0	2	4
\end{flushleft}


\begin{flushleft}
MTL717	 Fuzzy Sets and Applications 	
\end{flushleft}


3	 0	 0	 3


\begin{flushleft}
MTL720	 Neurocomputing and Applications	
\end{flushleft}


3	0	0	3


\begin{flushleft}
MTL725	 Stochastic Processes and its Applications 	
\end{flushleft}


3	 0	 0	 3


\begin{flushleft}
MTL728	 Category Theory	
\end{flushleft}


3	0	0	3


\begin{flushleft}
MTL729	 Computational Algebra and its Applications	 3	0	0	3
\end{flushleft}


\begin{flushleft}
MTL730	 Cryptography	
\end{flushleft}


3	0	0	3


\begin{flushleft}
MTL731	 Introduction to Chaotic Dynamical Systems	 3	 0	 0	 3
\end{flushleft}


\begin{flushleft}
MTL732	 Financial Mathematics	
\end{flushleft}


3	0	0	3


\begin{flushleft}
MTL733	 Stochastic of Finance	
\end{flushleft}


3	0	0	3


\begin{flushleft}
MTL735	 Advanced Number Theory	
\end{flushleft}


3	0	0	3


\begin{flushleft}
MTL737	 Differential Geometry	
\end{flushleft}


3	 0	 0	 3


\begin{flushleft}
MTL738	 Commutative Algebra	
\end{flushleft}


3	0	0	3


\begin{flushleft}
MTL739	 Representation of Finite Groups	
\end{flushleft}


3	 0	 0	 3


\begin{flushleft}
MTL741	 Fractal Geometry	
\end{flushleft}


3	 0	 0	 3


\begin{flushleft}
MTL742	 Operator Theory	
\end{flushleft}


3	0	0	3





\begin{flushleft}
Courses
\end{flushleft}


\begin{flushleft}
(Number, Abbreviated Title, L-T-P, credits)
\end{flushleft}





\begin{flushleft}
Sem.
\end{flushleft}





\begin{flushleft}
I
\end{flushleft}





\begin{flushleft}
II
\end{flushleft}





\begin{flushleft}
MTL501
\end{flushleft}





\begin{flushleft}
MTL502
\end{flushleft}





\begin{flushleft}
MTL503
\end{flushleft}





(3-1-0) 4





(3-1-0) 4





(3-1-0) 4





\begin{flushleft}
Ordinary differential
\end{flushleft}


\begin{flushleft}
Equations
\end{flushleft}





\begin{flushleft}
MTL506
\end{flushleft}





\begin{flushleft}
MTL507
\end{flushleft}





\begin{flushleft}
MTL508
\end{flushleft}





\begin{flushleft}
Algebra
\end{flushleft}





\begin{flushleft}
Complex
\end{flushleft}


\begin{flushleft}
Analysis
\end{flushleft}





\begin{flushleft}
Linear Algebra
\end{flushleft}





(3-1-0) 4





\begin{flushleft}
Mathematical
\end{flushleft}


\begin{flushleft}
Programming
\end{flushleft}





\begin{flushleft}
MTL601
\end{flushleft}





\begin{flushleft}
MTL602
\end{flushleft}





\begin{flushleft}
MTL603
\end{flushleft}





(3-1-0) 4





(3-1-0) 4





(3-1-0) 4





\begin{flushleft}
DE-2
\end{flushleft}





\begin{flushleft}
DE-3
\end{flushleft}





\begin{flushleft}
DE-4
\end{flushleft}





(3-1-0) 4





\begin{flushleft}
Topology
\end{flushleft}





\begin{flushleft}
Real Analysis
\end{flushleft}





(3-1-0) 4





\begin{flushleft}
MTL504
\end{flushleft}





\begin{flushleft}
MTL505
\end{flushleft}





(3-1-0) 4





(3-1-0) 4





\begin{flushleft}
MTL509
\end{flushleft}





\begin{flushleft}
MTL510
\end{flushleft}





\begin{flushleft}
Numerical Analysis
\end{flushleft}





\begin{flushleft}
Computer
\end{flushleft}


\begin{flushleft}
Programming
\end{flushleft}





(3-1-0) 4





\begin{flushleft}
Measure and
\end{flushleft}


\begin{flushleft}
Integration
\end{flushleft}





\begin{flushleft}
DE-1
\end{flushleft}





\begin{flushleft}
MAD701
\end{flushleft}





\begin{flushleft}
Contact h/week
\end{flushleft}


\begin{flushleft}
L
\end{flushleft}





\begin{flushleft}
T
\end{flushleft}





\begin{flushleft}
P
\end{flushleft}





\begin{flushleft}
Total
\end{flushleft}





\begin{flushleft}
Credits
\end{flushleft}





\begin{flushleft}
MTD701	Project-I	
\end{flushleft}


0	0	10	


5


\begin{flushleft}
MTL501	 Algebra	
\end{flushleft}


3	1	0	4


\begin{flushleft}
MTL502	 Linear Algebra	
\end{flushleft}


3	1	0	4


\begin{flushleft}
MTL503	 Real Analysis	
\end{flushleft}


3	1	0	4


\begin{flushleft}
MTL504	 Ordinary Differential Equations	
\end{flushleft}


3	1	0	4


\begin{flushleft}
MTL505	 Computer Programming	
\end{flushleft}


3	1	0	4


\begin{flushleft}
MTL506	 Complex Analysis	
\end{flushleft}


3	1	0	4


\begin{flushleft}
MTL507	 Topology	
\end{flushleft}


3	1	0	4


\begin{flushleft}
MTL508	 Mathematical Programming	
\end{flushleft}


3	1	0	4


\begin{flushleft}
MTL509	 Numerical Analysis	
\end{flushleft}


3	1	0	4


\begin{flushleft}
MTL510	 Measure and Integration	
\end{flushleft}


3	1	0	4


\begin{flushleft}
MTL601	 Probability and Statistics	
\end{flushleft}


3	1	0	4


\begin{flushleft}
MTL602	 Functional Analysis	
\end{flushleft}


3	1	0	4


\begin{flushleft}
MTL603	 Partial Differential Equations	
\end{flushleft}


3	1	0	4


	


\begin{flushleft}
Total Credits				57
\end{flushleft}





\begin{flushleft}
Lecture
\end{flushleft}


\begin{flushleft}
courses
\end{flushleft}





\begin{flushleft}
Program Core
\end{flushleft}





5





15





5





0





20





20





5





15





5





0





20





20





4





12





3





10





25





20





5





15





0





0





15





15





(3-1-0) 4





\begin{flushleft}
Summer
\end{flushleft}


\begin{flushleft}
III
\end{flushleft}


\begin{flushleft}
IV
\end{flushleft}





\begin{flushleft}
Probability and
\end{flushleft}


\begin{flushleft}
Statistics
\end{flushleft}





\begin{flushleft}
Functional
\end{flushleft}


\begin{flushleft}
Analysis
\end{flushleft}





\begin{flushleft}
Partial Differential
\end{flushleft}


\begin{flushleft}
Equations
\end{flushleft}





\begin{flushleft}
Project-I
\end{flushleft}





(0-0-10) 5


\begin{flushleft}
OC-1
\end{flushleft}





\begin{flushleft}
OC-2
\end{flushleft}





\begin{flushleft}
Total = 75
\end{flushleft}


94





\begin{flushleft}
\newpage
Programme Code: PHS
\end{flushleft}





\begin{flushleft}
Master of Science in Physics
\end{flushleft}


\begin{flushleft}
Department of Physics
\end{flushleft}





\begin{flushleft}
The overall credits structure
\end{flushleft}


\begin{flushleft}
Category
\end{flushleft}





\begin{flushleft}
PC
\end{flushleft}





\begin{flushleft}
PE
\end{flushleft}





\begin{flushleft}
OC
\end{flushleft}





\begin{flushleft}
Total
\end{flushleft}





\begin{flushleft}
Credits
\end{flushleft}





62





12





6





80





\begin{flushleft}
Optional Departmental specialization : Additional 6 credits : Total Credits : 86 with specialization
\end{flushleft}





\begin{flushleft}
Program Core
\end{flushleft}


\begin{flushleft}
PYD561 	Project-I	
\end{flushleft}


0	0	6	3


\begin{flushleft}
PYD562 	Project-II	
\end{flushleft}


0	0	12	


6


\begin{flushleft}
PYL551 	 Classical Mechanics 	
\end{flushleft}


3	 1	 0	 4


\begin{flushleft}
PYL552 	Electrodynamics 	
\end{flushleft}


3	1	0	4


\begin{flushleft}
PYL553 	 Mathematical Physics
\end{flushleft}


	


3	 1	 0	 4


\begin{flushleft}
PYL555 	Quantum Mechanics-I	
\end{flushleft}


3	1	0	4


\begin{flushleft}
PYL556	 Quantum Mechanics-II	
\end{flushleft}


3	0	0	3


\begin{flushleft}
PYL557 	 Electronics 	
\end{flushleft}


3	 1	 0	 4


\begin{flushleft}
PYL558 	 Statistical Mechanics 	
\end{flushleft}


3	 1	 0	 4


\begin{flushleft}
PYL560	 Applied Optics
\end{flushleft}


	


3	 1	 0	 4


\begin{flushleft}
PYL563 	 Solid State Physics
\end{flushleft}


	


3	 1	 0	 4


\begin{flushleft}
PYL567 	 Atomic and Molecular Physics 	
\end{flushleft}


3	 0	 0	 3


\begin{flushleft}
PYL569 	 Nuclear and Particle Physics 	
\end{flushleft}


3	 0	 0	 3


\begin{flushleft}
PYP561 	Laboratory-I	
\end{flushleft}


0	0	8	4


\begin{flushleft}
PYP562 	Laboratory-II	
\end{flushleft}


0	0	8	4


\begin{flushleft}
PYP563 	 Advanced Laboratory 	
\end{flushleft}


0	 0	 8	 4


	


\begin{flushleft}
Total Credits				62	
\end{flushleft}


\begin{flushleft}
Program Electives
\end{flushleft}





\begin{flushleft}
PYL551
\end{flushleft}





\begin{flushleft}
Classical
\end{flushleft}


\begin{flushleft}
Mechanics
\end{flushleft}





(3-1-0) 4


\begin{flushleft}
PYL552
\end{flushleft}





\begin{flushleft}
Electrodynamics
\end{flushleft}





(3-1-0) 4





\begin{flushleft}
PYL553
\end{flushleft}





\begin{flushleft}
PYL555
\end{flushleft}





\begin{flushleft}
Mathematical Quantum
\end{flushleft}


\begin{flushleft}
Physics
\end{flushleft}


\begin{flushleft}
Mechanics
\end{flushleft}





(3-1-0) 4


\begin{flushleft}
PYL556
\end{flushleft}





\begin{flushleft}
Quantum
\end{flushleft}


\begin{flushleft}
Mechanics-II
\end{flushleft}





(3-1-0) 4


\begin{flushleft}
PYL558
\end{flushleft}





\begin{flushleft}
Statistical
\end{flushleft}


\begin{flushleft}
Mechanics
\end{flushleft}





\begin{flushleft}
PYL557
\end{flushleft}





\begin{flushleft}
PYP561
\end{flushleft}





(3-1-0) 4





(0-0-8) 4





\begin{flushleft}
PYL560
\end{flushleft}





\begin{flushleft}
PYP562
\end{flushleft}





\begin{flushleft}
Electronics
\end{flushleft}





\begin{flushleft}
Applied
\end{flushleft}


\begin{flushleft}
Optics
\end{flushleft}





(3-0-0) 3





(3-1-0) 4





(3-1-0) 4





\begin{flushleft}
PYL567
\end{flushleft}





\begin{flushleft}
PYL569
\end{flushleft}





\begin{flushleft}
PYP563
\end{flushleft}





3	 0	 0	 3


3	0	0	3


3	 0	 0	 3


3	 0	 0	 3


3	 0	 0	 3


3	 0	 0	 3


2	0	0	2


3	0	0	3


3	0	0	3





\begin{flushleft}
Laboratory-I
\end{flushleft}





3	0	0	3


3	0	0	3


3	0	0	3


3	 0	 0	 3


3	 0	 0	 3


3	 0	 0	 3


3	0	0	3


3	 0	 0	 3


3	 0	 0	 3


3	0	0	3


3	 0	 0	 3





\begin{flushleft}
Contact h/week
\end{flushleft}


\begin{flushleft}
L
\end{flushleft}





\begin{flushleft}
T
\end{flushleft}





\begin{flushleft}
P
\end{flushleft}





\begin{flushleft}
Total
\end{flushleft}





\begin{flushleft}
Credits
\end{flushleft}





3	 0	 0	 3


3	0	0	3


3	0	0	3


3	0	0	3


3	0	0	3





\begin{flushleft}
Courses
\end{flushleft}


\begin{flushleft}
(Number, Abbreviated Title, L-T-P, credits)
\end{flushleft}





\begin{flushleft}
Sem.
\end{flushleft}





\begin{flushleft}
II
\end{flushleft}





\begin{flushleft}
PYL651	 Advanced Solid State Physics	
\end{flushleft}


\begin{flushleft}
PYL652	 Magnetism and Spintronics	
\end{flushleft}


\begin{flushleft}
PYL702	 Physics of Semiconductor Devices	
\end{flushleft}


\begin{flushleft}
PYL704	 Science and Technology of Thin Films 	
\end{flushleft}


\begin{flushleft}
PYL707 	 Characterization Techniques for Materials 	
\end{flushleft}


\begin{flushleft}
PYL727	 Energy Materials and Devices 	
\end{flushleft}


\begin{flushleft}
PYL728 	Quantum Heterostructures	
\end{flushleft}


\begin{flushleft}
PYL739	 Computational Techniques for Solid	
\end{flushleft}


	


\begin{flushleft}
State Materials
\end{flushleft}


\begin{flushleft}
PYL740	 Advanced Condensed Matter Theory	
\end{flushleft}





\begin{flushleft}
PYL657	 Plasma Physics	
\end{flushleft}


\begin{flushleft}
PYL658	 Advanced Plasma Physics	
\end{flushleft}


\begin{flushleft}
PYL740	 Advanced Condensed Matter Theory	
\end{flushleft}


\begin{flushleft}
PYL741	 Field Theory and Quantum Electrodynamics	
\end{flushleft}


\begin{flushleft}
PYL742	 General Relativity and Introductory	
\end{flushleft}


\begin{flushleft}
	Astrophysics
\end{flushleft}


\begin{flushleft}
PYL743	 Group Theory and its Applications	
\end{flushleft}


\begin{flushleft}
PYL744	 High Energy Physics	
\end{flushleft}


\begin{flushleft}
PYL745 	 Advanced Statistical Mechanics	
\end{flushleft}


\begin{flushleft}
PYL746 	 Non-equilibrium Statistical Mechanics with	
\end{flushleft}


	


\begin{flushleft}
Interdisciplinary Applications
\end{flushleft}


\begin{flushleft}
PYL748	 Quantum Optics	
\end{flushleft}


\begin{flushleft}
PYL749	 Quantum Information and Computation	
\end{flushleft}





\begin{flushleft}
Specialization in Photonics Min. 12 credits
\end{flushleft}





\begin{flushleft}
I
\end{flushleft}





\begin{flushleft}
Specialization in Condensed Matter Physics Min. 12 credits
\end{flushleft}





\begin{flushleft}
Specialization in Theoretical Physics Min. 12 credits
\end{flushleft}





0	0	6	3


3	0	0	3


3	0	0	3


3	0	0	3


3	 0	 0	 3


3	 0	 0	 3


3	0	0	3





\begin{flushleft}
PYL650	 Fiber and Integrated Optics	
\end{flushleft}


\begin{flushleft}
PYL655	 Laser Physics	
\end{flushleft}


\begin{flushleft}
PYL659	 Laser Spectroscopy	
\end{flushleft}


\begin{flushleft}
PYL747 	Non-linear Optics	
\end{flushleft}


\begin{flushleft}
PYL748	 Quantum Optics	
\end{flushleft}





3	 0	 0	 3


3	 0	 0	 3


3	0	0	3


3	0	0	3


3	 0	 0	 3


3	0	0	3


3	 0	 0	 3





\begin{flushleft}
Lecture
\end{flushleft}


\begin{flushleft}
courses
\end{flushleft}





\begin{flushleft}
PYD658 	Mini Project	
\end{flushleft}


\begin{flushleft}
PYL653 	Semiconductor Electronics	
\end{flushleft}


\begin{flushleft}
PYL656 	Microwaves	
\end{flushleft}


\begin{flushleft}
PYL705	 Nanostructured Materials 	
\end{flushleft}


\begin{flushleft}
PYL723 	 Vacuum Science and Cryogenics 	
\end{flushleft}


\begin{flushleft}
PYL725 	 Physics of Amorphous Materials	
\end{flushleft}


\begin{flushleft}
PYL792	 Optical Electronics	
\end{flushleft}





\begin{flushleft}
PYL749	 Quantum Information and Computation	
\end{flushleft}


\begin{flushleft}
PYL760	 Biomedical optics and Bio-photonics	
\end{flushleft}


\begin{flushleft}
PYL761	 Liquid Crystals	
\end{flushleft}


\begin{flushleft}
PYL762	 Statistical Optics	
\end{flushleft}


\begin{flushleft}
PYL770	 Ultra-fast optics and applications	
\end{flushleft}


\begin{flushleft}
PYL793	 Photonic Devices	
\end{flushleft}


\begin{flushleft}
PYL892	 Guided Wave Photonic Sensors 	
\end{flushleft}





4





12





4





8





24





20





5





15





3





8





26





22





5-6





15-18





1





14





30-33





23-26





3-4





9-12





0





12





21-24





15-18





\begin{flushleft}
PYL563
\end{flushleft}





\begin{flushleft}
Laboratory-II Solid State
\end{flushleft}


\begin{flushleft}
Physics
\end{flushleft}


(0-0-8) 4





(3-1-0) 4





\begin{flushleft}
Summer
\end{flushleft}





\begin{flushleft}
III
\end{flushleft}





\begin{flushleft}
IV
\end{flushleft}





\begin{flushleft}
PE(I)
\end{flushleft}


(3-0-0) 3





\begin{flushleft}
PYD562
\end{flushleft}


\begin{flushleft}
Project-II
\end{flushleft}





(0-0-12) 6





\begin{flushleft}
Atomic and
\end{flushleft}


\begin{flushleft}
Molecular
\end{flushleft}


\begin{flushleft}
Physics
\end{flushleft}





(3-0-0) 3


\begin{flushleft}
PE-3
\end{flushleft}


(3-0-0) 3





\begin{flushleft}
Nuclear
\end{flushleft}


\begin{flushleft}
Advanced
\end{flushleft}


\begin{flushleft}
and Particle Laboratory
\end{flushleft}


\begin{flushleft}
Physics
\end{flushleft}


(0-0-8) 4





(3-0-0) 3


\begin{flushleft}
PE-4
\end{flushleft}


(3-0-0) 3





\begin{flushleft}
OE-2
\end{flushleft}


(3-0-0) 3





\begin{flushleft}
PYD561
\end{flushleft}


\begin{flushleft}
Project-I
\end{flushleft}





(0-0-6) 3





\begin{flushleft}
PE-2
\end{flushleft}


(3-0-0) 3





\begin{flushleft}
DS-2
\end{flushleft}


(3-0-0) 3





\begin{flushleft}
OE-1
\end{flushleft}


\begin{flushleft}
DS-1
\end{flushleft}


(3-0-0) 3 (3-0-0) 3





\begin{flushleft}
Total = 75-81
\end{flushleft}


95





\begin{flushleft}
\newpage
Programme Code: JDS
\end{flushleft}





\begin{flushleft}
Master of Design in Industrial Design
\end{flushleft}


\begin{flushleft}
Interdisciplinary Programme
\end{flushleft}


\begin{flushleft}
The overall credits structure
\end{flushleft}


\begin{flushleft}
Category
\end{flushleft}





\begin{flushleft}
PC
\end{flushleft}





\begin{flushleft}
PE
\end{flushleft}





\begin{flushleft}
OE
\end{flushleft}





\begin{flushleft}
Total
\end{flushleft}





\begin{flushleft}
Credits
\end{flushleft}





39





9





3





51





\begin{flushleft}
Program Core
\end{flushleft}





\begin{flushleft}
Program Electives
\end{flushleft}





\begin{flushleft}
DSD792	Design Project-I	
\end{flushleft}


0	0	6	3	


\begin{flushleft}
DSD891	Design Project-II	
\end{flushleft}


0	0	12	


6	


\begin{flushleft}
DSD892	 Industry/ Research Design Project 	
\end{flushleft}


0	 0	 18	9


\begin{flushleft}
DSL710	 Framework of Design	
\end{flushleft}


2	0	0	2	


\begin{flushleft}
DSL732	 Adv. Mat. Processes \& Die Design	
\end{flushleft}


2	 0	 2	 3


\begin{flushleft}
DSL751	 Form and Aesthetics 	
\end{flushleft}


2	0	2	3


\begin{flushleft}
DSP711	 Computer Aided Product Detailing	
\end{flushleft}


1	0	4	3


\begin{flushleft}
DSP721	 Design and Innovation Methods	
\end{flushleft}


1	 0	 4	 3


\begin{flushleft}
DSP722	Applied Ergonomics	
\end{flushleft}


1	0	2	2


\begin{flushleft}
DSP731	 Communication and presentation skills	
\end{flushleft}


1	 0	 4	 3	


\begin{flushleft}
DSP741	 Product Interface \& Design	
\end{flushleft}


1	 0	 2	 2


\begin{flushleft}
DSR761	Social Immersion (Non-credit)	
\end{flushleft}


0	0	2	0


\begin{flushleft}
DSR801	Summer Internship (Non-credit)	
\end{flushleft}


0	0	4	0


	


\begin{flushleft}
Total Credits				39	
\end{flushleft}





\begin{flushleft}
DSL782	 Design for Usability	
\end{flushleft}


\begin{flushleft}
DSL810	 Special Topics in Design-I	
\end{flushleft}


\begin{flushleft}
DSL820	 Special Topics in Design-II	
\end{flushleft}


\begin{flushleft}
DSL841	 Design Management and Professional	
\end{flushleft}


\begin{flushleft}
	Practice
\end{flushleft}


\begin{flushleft}
DSP712	 Exhibitions and Environmental Design	
\end{flushleft}


\begin{flushleft}
DSR762	Vehicle Design 	
\end{flushleft}


\begin{flushleft}
DSR772	Transportation Design	
\end{flushleft}


\begin{flushleft}
DSR812	Media Studies 	
\end{flushleft}


\begin{flushleft}
DSR822	Design for Sustainability	
\end{flushleft}


\begin{flushleft}
DSR832	 Design for User Experience	
\end{flushleft}


\begin{flushleft}
DSR852	Strategic Design Management	
\end{flushleft}


\begin{flushleft}
DSR862	 Design in Indian Context	
\end{flushleft}


\begin{flushleft}
DSV820	 Special Modules in Design	
\end{flushleft}





\begin{flushleft}
I
\end{flushleft}





\begin{flushleft}
Framework of
\end{flushleft}


\begin{flushleft}
Design
\end{flushleft}





(2-0-0) 2





\begin{flushleft}
DSP721
\end{flushleft}





(1-0-4) 3





\begin{flushleft}
DSP741
\end{flushleft}





\begin{flushleft}
Product
\end{flushleft}


\begin{flushleft}
Interface \&
\end{flushleft}


\begin{flushleft}
Design
\end{flushleft}





(1-0-2) 2





\begin{flushleft}
L
\end{flushleft}





\begin{flushleft}
T
\end{flushleft}





\begin{flushleft}
P
\end{flushleft}





\begin{flushleft}
Total
\end{flushleft}





2





7





0





12





19





13





2





6-7





0





14-16





21-22





14





2





4-6





0





6-10





12-14





12





1





3





0





18





21





12





\begin{flushleft}
DSL751
\end{flushleft}





\begin{flushleft}
Form and
\end{flushleft}


\begin{flushleft}
Aesthetics
\end{flushleft}





(2-0-2) 3





\begin{flushleft}
DSR761
\end{flushleft}





\begin{flushleft}
Social Immersion (Non-credit core)
\end{flushleft}





\begin{flushleft}
DSP711
\end{flushleft}





\begin{flushleft}
Computer
\end{flushleft}


\begin{flushleft}
Aided Product
\end{flushleft}


\begin{flushleft}
Detailing
\end{flushleft}





(1-0-4) 3





\begin{flushleft}
DSP722
\end{flushleft}





\begin{flushleft}
DSL732
\end{flushleft}





\begin{flushleft}
Applied
\end{flushleft}


\begin{flushleft}
Ergonomics
\end{flushleft}





\begin{flushleft}
Adv. Mat.
\end{flushleft}


\begin{flushleft}
Processes \&
\end{flushleft}


\begin{flushleft}
Die Design
\end{flushleft}





(1-0-2) 2





(2-0-2) 3





\begin{flushleft}
DSD792
\end{flushleft}


\begin{flushleft}
Design
\end{flushleft}


\begin{flushleft}
Project-I
\end{flushleft}





(0-0-6) 3





\begin{flushleft}
PE-1
\end{flushleft}


(2-0-2/3-0-0) 3





\begin{flushleft}
DSR801
\end{flushleft}





\begin{flushleft}
Summer
\end{flushleft}





\begin{flushleft}
Summer Intership (Non-credit core)
\end{flushleft}





\begin{flushleft}
DSD891
\end{flushleft}


\begin{flushleft}
III
\end{flushleft}





\begin{flushleft}
Communication
\end{flushleft}


\begin{flushleft}
and
\end{flushleft}


\begin{flushleft}
presentation
\end{flushleft}


\begin{flushleft}
skills
\end{flushleft}





(1-0-4) 3





\begin{flushleft}
Winter
\end{flushleft}





\begin{flushleft}
II
\end{flushleft}





\begin{flushleft}
DSP731
\end{flushleft}





\begin{flushleft}
Design and
\end{flushleft}


\begin{flushleft}
Innovation
\end{flushleft}


\begin{flushleft}
Methods
\end{flushleft}





\begin{flushleft}
Contact h/week
\end{flushleft}





\begin{flushleft}
Credits
\end{flushleft}





\begin{flushleft}
DSL710
\end{flushleft}





2	 0	 2	 3


2	0	2	3


2	0	2	3


2	0	2	3


2	0	2	3


3	 0	 0	 3


2	0	2	3


3	 0	 0	 3


1	 0	 0	 1





\begin{flushleft}
Lecture
\end{flushleft}


\begin{flushleft}
courses
\end{flushleft}





\begin{flushleft}
Courses
\end{flushleft}


\begin{flushleft}
(Number, Abbreviated Title, L-T-P, credits)
\end{flushleft}





\begin{flushleft}
Sem.
\end{flushleft}





2	0	2	3


3	0	0	3


3	0	0	3


3	 0	 0	 3





\begin{flushleft}
Design
\end{flushleft}


\begin{flushleft}
Project-II
\end{flushleft}





\begin{flushleft}
DE-2
\end{flushleft}


(2-0-2/3-0-0) 3





\begin{flushleft}
DE-3
\end{flushleft}


(2-0-2/3-0-0) 3





(0-0-12) 6


\begin{flushleft}
DSD892
\end{flushleft}


\begin{flushleft}
IV
\end{flushleft}





\begin{flushleft}
Industry/
\end{flushleft}


\begin{flushleft}
Research
\end{flushleft}


\begin{flushleft}
Design Project
\end{flushleft}





\begin{flushleft}
OE
\end{flushleft}


(3-0-0) 3





(0-0-18) 9





\begin{flushleft}
Total = 51
\end{flushleft}


96





\begin{flushleft}
\newpage
Programme Code: SMG
\end{flushleft}





\begin{flushleft}
Master of Business Administration
\end{flushleft}


\begin{flushleft}
Department of Management
\end{flushleft}


\begin{flushleft}
The overall credits structure
\end{flushleft}


\begin{flushleft}
Programme Core
\end{flushleft}


\begin{flushleft}
PC
\end{flushleft}


\begin{flushleft}
(Total 36 Credits)
\end{flushleft}





\begin{flushleft}
Category
\end{flushleft}





\begin{flushleft}
Streamed Electives
\end{flushleft}


\begin{flushleft}
SE
\end{flushleft}


\begin{flushleft}
(Total 12 credits)
\end{flushleft}





\begin{flushleft}
Common Core
\end{flushleft}


\begin{flushleft}
CC
\end{flushleft}





\begin{flushleft}
Unique Core
\end{flushleft}


\begin{flushleft}
UC
\end{flushleft}





\begin{flushleft}
Analytical Skills Stream
\end{flushleft}


\begin{flushleft}
AS
\end{flushleft}





\begin{flushleft}
People Skills Stream
\end{flushleft}


\begin{flushleft}
PS
\end{flushleft}





30





6





6





6





\begin{flushleft}
Credits
\end{flushleft}


\begin{flushleft}
Program Core
\end{flushleft}





\begin{flushleft}
Streamed Electives (SE)
\end{flushleft}


\begin{flushleft}
Streamed Electives consist of Analytical Skills (AS) Stream and People
\end{flushleft}


\begin{flushleft}
Skills (PS) Stream. The total credits of Streamed Electives would be
\end{flushleft}


\begin{flushleft}
12 -- 6 from AS and 6 from PS.
\end{flushleft}


\begin{flushleft}
a) Analytical Skills (AS) Stream
\end{flushleft}


\begin{flushleft}
MSL719*	 Statistics for Management	
\end{flushleft}


3	 0	 0	 3


\begin{flushleft}
MSL721*	Econometrics	
\end{flushleft}


3	 0	 0	 3


\begin{flushleft}
MSL740	 Quantitative Methods in Management	
\end{flushleft}


3	 0	 0	 3


\begin{flushleft}
MTL732	 Financial Mathematics	
\end{flushleft}


3	 1	 0	 4


\begin{flushleft}
* These are new courses which have been designed and/or modified
\end{flushleft}


\begin{flushleft}
as a part of the curriculum review.
\end{flushleft}


\begin{flushleft}
b) People Skills (PS) Stream
\end{flushleft}


\begin{flushleft}
MSL710	 Creative Problem Solving	
\end{flushleft}


3	 0	 0	 3


\begin{flushleft}
MSL724*	 Business Communication	
\end{flushleft}


1.5	 0	 0	 1.5


\begin{flushleft}
MSL725*	 Business Negotiations	
\end{flushleft}


1.5	 0	 0	 1.5


\begin{flushleft}
MSL727*	 Interpersonal Behavior \& Team Dynamics	 1.5	 0	 0	 1.5
\end{flushleft}


\begin{flushleft}
MSL729*	 Individual Behavior in Organization	
\end{flushleft}


1.5	 0	 0	 1.5


\begin{flushleft}
MSL730*	 Managing With Power	
\end{flushleft}


1.5	 0	 0	 1.5


\begin{flushleft}
MSL731*	 Developing Self Awareness	
\end{flushleft}


1.5	 0	 0	 1.5


\begin{flushleft}
MSL733*	 Organization Theory	
\end{flushleft}


1.5	 0	 0	 1.5


\begin{flushleft}
* These are new courses which have been designed and/or modified
\end{flushleft}


\begin{flushleft}
as a part of the curriculum review.
\end{flushleft}


\begin{flushleft}
Non-credit Core (NC)
\end{flushleft}


\begin{flushleft}
MST893	 Corporate Sector Attachment	
\end{flushleft}


0	 0	 4	 2


\begin{flushleft}
MST894*	 Social Sector Attachment	
\end{flushleft}


0	 0	 2	 1


\begin{flushleft}
* This is a new course which has been designed as a part of the
\end{flushleft}


\begin{flushleft}
curriculum review.
\end{flushleft}


\begin{flushleft}
Program Electives (PE)
\end{flushleft}


3	


3	


3	


3	


3	


3	





0	 0	


0	 0	


0	 0	


0	 0	


0	 0	


0	 0	





\begin{flushleft}
Programme
\end{flushleft}


\begin{flushleft}
Electives
\end{flushleft}


\begin{flushleft}
PE
\end{flushleft}





\begin{flushleft}
Total
\end{flushleft}





3





24





72





\begin{flushleft}
MSL801	Technology Forecasting \& Assessment	
\end{flushleft}


3	 0	 0	 3


\begin{flushleft}
MSV801 	 Selected Topics in OB \& HR Management	 1	 0	 0	1
\end{flushleft}


\begin{flushleft}
MSV802 	 Selected Topics in Finance	
\end{flushleft}


1	 0	 0	1


\begin{flushleft}
MSL803	 Technical Entrepreneurship	
\end{flushleft}


3	 0	 0	 3


\begin{flushleft}
MSV803 	 Selected Topics in Information Technology	 1	 0	 0	1
\end{flushleft}


\begin{flushleft}
	Management
\end{flushleft}


\begin{flushleft}
MSV804 	 Selected Topics in Operations Management	 1	 0	 0	1
\end{flushleft}


\begin{flushleft}
MSV805 	 Selected Topics in Economics	
\end{flushleft}


1	 0	 0	1


\begin{flushleft}
MSL806*	 Mergers \& Acquisitions	
\end{flushleft}


3	 0	 0	 3


\begin{flushleft}
MSV806 	 Selected Topics in Marketing Management	 1	 0	 0	1
\end{flushleft}


\begin{flushleft}
MSL807*	 Selected Topics in Strategic Management	 1	 0	 0	 1
\end{flushleft}


\begin{flushleft}
MSL808*	 Systems Thinking 	
\end{flushleft}


3	 0	 0	 3


\begin{flushleft}
MSL812	 Flexible Systems Management	
\end{flushleft}


3	 0	 0	 3


\begin{flushleft}
MSL813	 Systems Methodology for Management	
\end{flushleft}


3	 0	 0	 3


\begin{flushleft}
MSL817	 Systems Waste \& Sustainability	
\end{flushleft}


3	 0	 0	 3


\begin{flushleft}
MSL819	 Business Process Re-engineering	
\end{flushleft}


3	 0	 0	 3


\begin{flushleft}
MSL820	 Global Business Environment	
\end{flushleft}


3	 0	 0	 3


\begin{flushleft}
MSL821*	 Strategy Execution Excellence	
\end{flushleft}


3	 0	 0	 3


\begin{flushleft}
MSL 822	 International Business	
\end{flushleft}


3	 0	 0	 3


\begin{flushleft}
MSL823	 Strategic Change \& Flexibility	
\end{flushleft}


3	 0	 0	 3


\begin{flushleft}
MSL 824	 Policy Dynamics \& Learning Organization	 3	 0	 0	 3
\end{flushleft}


\begin{flushleft}
MSL 825	 Strategies in Functional Management	
\end{flushleft}


3	 0	 0	 3


\begin{flushleft}
MSL826	 Business Ethics	
\end{flushleft}


3	 0	 0	 3


\begin{flushleft}
MSL827	 International Competitiveness	
\end{flushleft}


3	 0	 0	 3


\begin{flushleft}
MSL828	 Global Strategic Management	
\end{flushleft}


3	 0	 0	 3


\begin{flushleft}
MSL829	 Current \& Emerging Issues in Strategic Management	3	 0	 0	 3
\end{flushleft}


\begin{flushleft}
MSL851*	 Strategic Alliance	
\end{flushleft}


1.5	 0	 0	 1.5


\begin{flushleft}
MSL714	 Organizational Dynamics and Environment	 3	 0	 0	 3
\end{flushleft}


\begin{flushleft}
MSL830	 Organizational Structure and Processes	
\end{flushleft}


3	 0	 0	 3


\begin{flushleft}
MSL831	 Management of Change	
\end{flushleft}


3	 0	 0	 3


\begin{flushleft}
MSL832	 Managing Innovation for Organizational 	
\end{flushleft}


3	 0	 0	 3


\begin{flushleft}
	Effectiveness
\end{flushleft}


\begin{flushleft}
MSL833	 Organizational Development	
\end{flushleft}


3	 0	 0	 3


\begin{flushleft}
MSL834*	 Managing Diversity at Workplace	
\end{flushleft}


1.5	 0	 0	 1.5


\begin{flushleft}
MSL836*	 International Human Resources Management	 1.5	0	0	 1.5
\end{flushleft}


\begin{flushleft}
MSL839	 Current \& Emerging Issues in	
\end{flushleft}


3	 0	 0	 3


	


\begin{flushleft}
Organizational Management
\end{flushleft}


\begin{flushleft}
MSL804*	 Procurement Management	
\end{flushleft}


3	 0	 0	 3


\begin{flushleft}
MSL805*	 Services Operations Management	
\end{flushleft}


3	 0	 0	 3


\begin{flushleft}
MSL715	 Quality and Environment Management Systems	3	 0	 0	 3
\end{flushleft}


\begin{flushleft}
MSL816	 Total Quality Management	
\end{flushleft}


3	 0	 0	 3


\begin{flushleft}
MSL818	 Industrial Waste Management	
\end{flushleft}


3	 0	 0	 3


\begin{flushleft}
MSL840	 Manufacturing Strategy	
\end{flushleft}


3	 0	 0	 3


\begin{flushleft}
MSL841*	 Supply Chain Analytics	
\end{flushleft}


3	 0	 0	 3


\begin{flushleft}
MSL842*	 Supply Chain Modeling	
\end{flushleft}


3	 0	 0	 3


\begin{flushleft}
MSL843	 Supply Chain Logistics Management	
\end{flushleft}


3	 0	 0	 3


\begin{flushleft}
MSL844	 Systems Reliability, Safety and Maintenance	3	 0	 0	 3
\end{flushleft}


\begin{flushleft}
	Management
\end{flushleft}


\begin{flushleft}
MSL845	 Total Project Systems Management	
\end{flushleft}


3	 0	 0	 3


\begin{flushleft}
MSL846	 Total Productivity Management	
\end{flushleft}


3	 0	 0	 3


\begin{flushleft}
MSL848*	 Applied Operations Research	
\end{flushleft}


3	 0	 0	 3


\begin{flushleft}
MSL849	 Current \& Emerging Issues in Manufacturing	3	 0	 0	 3
\end{flushleft}


\begin{flushleft}
	Management
\end{flushleft}


\begin{flushleft}
MSL809*	 Cyber Security: Managing Risks	
\end{flushleft}


3	 0	 0	 3


\begin{flushleft}
MSL810*	 Advanced Data Mining for Business Decisions	1.5	 0	 0	 1.5
\end{flushleft}


\begin{flushleft}
MSL814*	 Data Visualization	
\end{flushleft}


1.5	 0	 0	 1.5


\begin{flushleft}
MSL815	 Decision Support and Expert Systems	
\end{flushleft}


3	 0	 0	 3


\begin{flushleft}
MSL850	 Management of Information Technology	
\end{flushleft}


3	 0	 0	 3


\begin{flushleft}
MSL852	 Network System: Applications and Management	3	 0	 0	 3
\end{flushleft}


\begin{flushleft}
MSL853*	 Software Project Management	
\end{flushleft}


3	 0	 0	 3





\begin{flushleft}
MSL705*	 HRM Systems	
\end{flushleft}


1.5	 0	 0	 1.5


\begin{flushleft}
MSL706**	Business Laws	
\end{flushleft}


3	 0	 0	 3


\begin{flushleft}
MSL707*	 Management Accounting	
\end{flushleft}


3	 0	 0	 3


\begin{flushleft}
MSL708*	 Financial Management	
\end{flushleft}


3	 0	 0	 3


\begin{flushleft}
MSL709*	 Business Research Methods	
\end{flushleft}


1.5	 0	 0	 1.5


\begin{flushleft}
MSL711*	 Strategic Management	
\end{flushleft}


3	 0	 0	 3


\begin{flushleft}
MSL712*	 Ethics \& Values Based Leadership	
\end{flushleft}


1.5	 0	 0	 1.5


\begin{flushleft}
MSL713*	 Information Systems Management	
\end{flushleft}


3	 0	 0	 3


\begin{flushleft}
MSL720*	 Macroeconomic Environment of Business	 3	 0	 0	 3
\end{flushleft}


\begin{flushleft}
MSL745	 Operations Management	
\end{flushleft}


3	 0	 0	 3


\begin{flushleft}
MSL760	 Marketing Management	
\end{flushleft}


3	 0	 0	 3


\begin{flushleft}
MSL780*	 Managerial Economics	
\end{flushleft}


1.5	 0	 0	 1.5


\begin{flushleft}
MSD890	 Major Project (Unique Core)	
\end{flushleft}


0	 0	 12	6


\begin{flushleft}
Notes:
\end{flushleft}


\begin{flushleft}
The UC will include the major project which would focus on a research
\end{flushleft}


\begin{flushleft}
driven application of skills acquired in a particular functional area,
\end{flushleft}


\begin{flushleft}
through the programme.				
\end{flushleft}


\begin{flushleft}
* These are new courses which have been designed and/or modified
\end{flushleft}


\begin{flushleft}
as a part of the curriculum review.
\end{flushleft}


\begin{flushleft}
** MSL706 was initially an elective, MSL887. This course's content is
\end{flushleft}


\begin{flushleft}
the same, only the number has been changed to now reflect a core
\end{flushleft}


\begin{flushleft}
course.
\end{flushleft}


	


\begin{flushleft}
Total Credits				 36
\end{flushleft}





\begin{flushleft}
MSL716	 Fundamentals of Management Systems	
\end{flushleft}


\begin{flushleft}
MSL717*	 Business Systems Analysis \& Design	
\end{flushleft}


\begin{flushleft}
MSL811	 Management Control Systems	
\end{flushleft}


\begin{flushleft}
MSL802	 Management of Intellectual Property Rights	
\end{flushleft}


\begin{flushleft}
MSL835	 Labor Legislation and Industrial Relations	
\end{flushleft}


\begin{flushleft}
MSL704	 Science \& Technology Policy Systems	
\end{flushleft}





\begin{flushleft}
Non-credit Core
\end{flushleft}


\begin{flushleft}
NC
\end{flushleft}





3


3


3


3


3


3





97





\begin{flushleft}
\newpage
MSL707
\end{flushleft}


\begin{flushleft}
Mgmt.
\end{flushleft}


\begin{flushleft}
Accounting
\end{flushleft}





(3-0-0) 3





\begin{flushleft}
SE
\end{flushleft}


\begin{flushleft}
AS-2
\end{flushleft}


\begin{flushleft}
MSL719
\end{flushleft}





\begin{flushleft}
Statistics for
\end{flushleft}


\begin{flushleft}
Mgmt.
\end{flushleft}





\begin{flushleft}
MSL760 MSL713
\end{flushleft}


\begin{flushleft}
Marketing
\end{flushleft}


\begin{flushleft}
Mgmt.
\end{flushleft}





(3-0-0) 3





(3-0-0) 3





\begin{flushleft}
MSL709
\end{flushleft}





(3-0-0) 3





(1.5-0-0)


1.5





\begin{flushleft}
IV
\end{flushleft}





\begin{flushleft}
Ethics \&
\end{flushleft}


\begin{flushleft}
Values
\end{flushleft}


\begin{flushleft}
Based
\end{flushleft}


\begin{flushleft}
Leadership
\end{flushleft}





\begin{flushleft}
SE
\end{flushleft}


\begin{flushleft}
SE
\end{flushleft}


\begin{flushleft}
PS-1
\end{flushleft}


\begin{flushleft}
AS-1
\end{flushleft}


\begin{flushleft}
MSL724 MSL740
\end{flushleft}


\begin{flushleft}
Business
\end{flushleft}


\begin{flushleft}
Comm.
\end{flushleft}





(1.5-0-0) 1.5 (1.5-0-0)


1.5


\begin{flushleft}
MST894
\end{flushleft}





\begin{flushleft}
Quantitative
\end{flushleft}


\begin{flushleft}
Methods
\end{flushleft}


\begin{flushleft}
in Mgmt.
\end{flushleft}





\begin{flushleft}
L
\end{flushleft}





\begin{flushleft}
T
\end{flushleft}





\begin{flushleft}
P Total
\end{flushleft}





\begin{flushleft}
Credits
\end{flushleft}





\begin{flushleft}
Contact h/week
\end{flushleft}





8





19.5 0





15 20.5 19.5





8





19.5 0





0





(3-0-0) 3





\begin{flushleft}
Social Sector Attachment
\end{flushleft}





\begin{flushleft}
MSL708
\end{flushleft}





\begin{flushleft}
MSL780
\end{flushleft}





\begin{flushleft}
MSL705 MSL745
\end{flushleft}





\begin{flushleft}
MSL711
\end{flushleft}





\begin{flushleft}
Financial
\end{flushleft}


\begin{flushleft}
Mgmt.
\end{flushleft}





\begin{flushleft}
Managerial
\end{flushleft}


\begin{flushleft}
Economics
\end{flushleft}





\begin{flushleft}
HRM
\end{flushleft}


\begin{flushleft}
Systems
\end{flushleft}





\begin{flushleft}
Operations
\end{flushleft}


\begin{flushleft}
Mgmt.
\end{flushleft}





\begin{flushleft}
Strategic
\end{flushleft}


\begin{flushleft}
Mgmt.
\end{flushleft}





(3-0-0) 3





(3-0-0) 3





(3-0-0) 3





(1.5-0-0) 1.5 (1.5-0-0)


1.5





\begin{flushleft}
PE
\end{flushleft}


(3-0-0) 3





\begin{flushleft}
SE
\end{flushleft}


\begin{flushleft}
PS-2
\end{flushleft}


(1.5-0-0)


1.5





\begin{flushleft}
MSL720
\end{flushleft}


\begin{flushleft}
Macroeconomic
\end{flushleft}


\begin{flushleft}
Environment
\end{flushleft}


\begin{flushleft}
of Business
\end{flushleft}





19.5





19





(3-0-0) 3


\begin{flushleft}
MST893
\end{flushleft}





\begin{flushleft}
Summer
\end{flushleft}





\begin{flushleft}
III
\end{flushleft}





\begin{flushleft}
MSL712
\end{flushleft}





\begin{flushleft}
Information Business
\end{flushleft}


\begin{flushleft}
Systems
\end{flushleft}


\begin{flushleft}
Research
\end{flushleft}


\begin{flushleft}
Mgmt.
\end{flushleft}


\begin{flushleft}
Methods
\end{flushleft}





\begin{flushleft}
Winter
\end{flushleft}





\begin{flushleft}
II
\end{flushleft}





\begin{flushleft}
MSL867	 Industrial Marketing Management	
\end{flushleft}


3	 0	 0	 3


\begin{flushleft}
MSL869	 Current \& Emerging Issues in Marketing	
\end{flushleft}


3	 0	 0	 3


\begin{flushleft}
MSL870*	 Corporate Governance	
\end{flushleft}


1.5	 0	 0	 1.5


\begin{flushleft}
MSL871*	 Banking and Financial Services	
\end{flushleft}


1.5	 0	 0	 1.5


\begin{flushleft}
MSL872	 Working Capital Management 	
\end{flushleft}


3	 0	 0	 3


\begin{flushleft}
MSL873	Security Analysis \& Portfolio Management 	3	 0	 0	 3
\end{flushleft}


\begin{flushleft}
MSL874*	 Indian Financial System	
\end{flushleft}


1.5	 0	 0	 1.5


\begin{flushleft}
MSL875	 International Financial Management	
\end{flushleft}


3	 0	 0	 3


\begin{flushleft}
MSL879	 Current \& Emerging Issues in Finance 	
\end{flushleft}


3	 0	 0	 3


\begin{flushleft}
MSL734	 Management of Small \& Medium Scale 	
\end{flushleft}


3	 0	 0 3


	


\begin{flushleft}
Industrial Enterprises
\end{flushleft}


\begin{flushleft}
MSL847	 Advanced Methods for Management Research	 3	 0	 0 	 3
\end{flushleft}


\begin{flushleft}
MSL880	 Selected Topics in Management Methodology	3	 0	 0	 3
\end{flushleft}


\begin{flushleft}
MSL881	 Management of Public Sector Enterprises	 3	 0	 0	 3
\end{flushleft}


	


\begin{flushleft}
in India
\end{flushleft}


\begin{flushleft}
MSL889	 Current \& Emerging Issues in Public Sector	 3	 0	 0	 3
\end{flushleft}


\begin{flushleft}
	Management
\end{flushleft}


\begin{flushleft}
MSL897	 Consultancy Process \& Skills	
\end{flushleft}


3	 0	 0	 3


\begin{flushleft}
MSL898	 Consultancy Professional Practice	
\end{flushleft}


3	 0	 0 3


\begin{flushleft}
MSL899	 Current \& Emerging Issues in Consultancy	 3	 0	 0	 3
\end{flushleft}


\begin{flushleft}
	Management
\end{flushleft}


\begin{flushleft}
MSL895*	 Advanced Data Analysis for Management 	 3	 0	 0 	 3
\end{flushleft}


\begin{flushleft}
MSL896*	 International Economic Policy	
\end{flushleft}


3	 0	 0	 3


\begin{flushleft}
* These are new courses which have been designed and/or modified
\end{flushleft}


\begin{flushleft}
as a part of the curriculum review.
\end{flushleft}





\begin{flushleft}
Courses
\end{flushleft}


\begin{flushleft}
(Number, Abbreviated Title, L-T-P, credits)
\end{flushleft}





\begin{flushleft}
Sem.
\end{flushleft}





\begin{flushleft}
I
\end{flushleft}





1.5


3


3


1.5


3


1.5


1.5


1.5


1.5


1.5


1.5


3


3


3


3


1.5


1.5


1.5


1.5


3


3


3


3


3


3


3





\begin{flushleft}
Lecture
\end{flushleft}


\begin{flushleft}
courses
\end{flushleft}





\begin{flushleft}
MSL854*	 Big Data Analytics \& Data Science	
\end{flushleft}


1.5	 0	 0	


\begin{flushleft}
MSL855*	 Electronic Commerce	
\end{flushleft}


3	 0	 0	


\begin{flushleft}
MSL856*	 Business Intelligence	
\end{flushleft}


3	 0	 0	


\begin{flushleft}
MSL858*	 Business Process Management with IT	
\end{flushleft}


1.5	 0	 0	


\begin{flushleft}
MSL859	 Current and Emerging Issues in IT Management	3	 0	 0	
\end{flushleft}


\begin{flushleft}
MSL868*	 Digital Research Methods	
\end{flushleft}


1.5	 0	 0	


\begin{flushleft}
MSL876*	 Economics of Digital Business	
\end{flushleft}


1.5	 0	 0	


\begin{flushleft}
MSL877*	 Electronic Government	
\end{flushleft}


1.5	 0	 0	


\begin{flushleft}
MSL878*	Electronic Payments	
\end{flushleft}


1.5	 0	 0	


\begin{flushleft}
MSL882*	 Enterprise Cloud Computing	
\end{flushleft}


1.5	 0	 0	


\begin{flushleft}
MSL883*	 ICTs, Development and Business	
\end{flushleft}


1.5	 0	 0	


\begin{flushleft}
MSL884*	 Information System Strategy	
\end{flushleft}


3	 0	 0	


\begin{flushleft}
MSL885*	 Digital Marketing-Analytics \& Optimization	 3	 0	 0	
\end{flushleft}


\begin{flushleft}
MSL886*	 IT Consulting \& Practice	
\end{flushleft}


3	 0	 0	


\begin{flushleft}
MSL887*	 Mobile Commerce	
\end{flushleft}


3	 0	 0	


\begin{flushleft}
MSL888*	 Data Warehousing for Business Decision	
\end{flushleft}


1.5	 0	 0	


\begin{flushleft}
MSL891*	 Data Analytics using SPSS	
\end{flushleft}


1.5	 0	 0	


\begin{flushleft}
MSL892*	 Predictive Analytics	
\end{flushleft}


1.5	 0	 0	


\begin{flushleft}
MSL893*	 Public Policy Issues in the Information Age	 1.5	 0	 0	
\end{flushleft}


\begin{flushleft}
MSL894*	 Social Media \& Business Practices	
\end{flushleft}


3	 0	 0	


\begin{flushleft}
MSL861	 Market Research	
\end{flushleft}


3	 0	 0	


\begin{flushleft}
MSL862	 Product Management	
\end{flushleft}


3	 0	 0	


\begin{flushleft}
MSL863	 Advertising and Sales Promotion Management	3	 0	 0	
\end{flushleft}


\begin{flushleft}
MSL864*	 Corporate Communication	
\end{flushleft}


3	 0	 0	


\begin{flushleft}
MSL865	 Sales Management 	
\end{flushleft}


3	 0	 0	


\begin{flushleft}
MSL866	 International Marketing	
\end{flushleft}


3	 0	 0	





\begin{flushleft}
Corporate Sector Attachment
\end{flushleft}





\begin{flushleft}
SE
\end{flushleft}


\begin{flushleft}
SE
\end{flushleft}


\begin{flushleft}
MSL706
\end{flushleft}


\begin{flushleft}
Business
\end{flushleft}


\begin{flushleft}
PS-3
\end{flushleft}


\begin{flushleft}
PS-4
\end{flushleft}


\begin{flushleft}
Laws
\end{flushleft}


(1.5-0-0) 1.5 (1.5-0-0) 1.5


(3-0-0) 3


\begin{flushleft}
MSD890
\end{flushleft}


\begin{flushleft}
Major
\end{flushleft}


\begin{flushleft}
Project
\end{flushleft}





\begin{flushleft}
PE
\end{flushleft}


\begin{flushleft}
(Credits 9-12)
\end{flushleft}





15-18





\begin{flushleft}
PE
\end{flushleft}


\begin{flushleft}
(Credits 9-12)
\end{flushleft}





15-18





(0-0-12) 6


\begin{flushleft}
SE = Streamed Electives, AS = Analytical Skills Stream, PS = People Skills Stream, FE = Focus Electives, PE = Programme Electives
\end{flushleft}





\begin{flushleft}
Total = 72
\end{flushleft}


98





\begin{flushleft}
\newpage
Programme Code: SMT
\end{flushleft}





\begin{flushleft}
Master of Business Administration (Telecom Management)
\end{flushleft}


\begin{flushleft}
Department of Management
\end{flushleft}


\begin{flushleft}
The overall credits structure
\end{flushleft}


\begin{flushleft}
Programme Core
\end{flushleft}


\begin{flushleft}
PC
\end{flushleft}


\begin{flushleft}
(Total 36 Credits)
\end{flushleft}





\begin{flushleft}
Category
\end{flushleft}





\begin{flushleft}
Streamed Electives
\end{flushleft}


\begin{flushleft}
SE
\end{flushleft}


\begin{flushleft}
(Total 12 credits)
\end{flushleft}





\begin{flushleft}
Common Core
\end{flushleft}


\begin{flushleft}
CC
\end{flushleft}





\begin{flushleft}
Unique Core
\end{flushleft}


\begin{flushleft}
UC
\end{flushleft}





\begin{flushleft}
Analytical Skills
\end{flushleft}


\begin{flushleft}
Stream
\end{flushleft}


\begin{flushleft}
AS
\end{flushleft}





\begin{flushleft}
People Skills
\end{flushleft}


\begin{flushleft}
Stream
\end{flushleft}


\begin{flushleft}
PS
\end{flushleft}





30





6





6





6





\begin{flushleft}
Credits
\end{flushleft}





\begin{flushleft}
Focus
\end{flushleft}


\begin{flushleft}
Electives
\end{flushleft}


\begin{flushleft}
FE
\end{flushleft}





\begin{flushleft}
Non-credit
\end{flushleft}


\begin{flushleft}
Core
\end{flushleft}


\begin{flushleft}
NC
\end{flushleft}





\begin{flushleft}
Programme
\end{flushleft}


\begin{flushleft}
Electives
\end{flushleft}


\begin{flushleft}
PE
\end{flushleft}





\begin{flushleft}
Total
\end{flushleft}





6





3





18





72





\begin{flushleft}
Program Core (PC)
\end{flushleft}





\begin{flushleft}
Program Electives (PE)
\end{flushleft}





\begin{flushleft}
MSL705*	 HRM Systems	
\end{flushleft}


1.5	 0	 0	 1.5


\begin{flushleft}
MSL706**	Business Laws	
\end{flushleft}


3	 0	 0	 3


\begin{flushleft}
MSL707*	 Management Accounting	
\end{flushleft}


3	 0	 0	 3


\begin{flushleft}
MSL708*	 Financial Management	
\end{flushleft}


3	 0	 0	 3


\begin{flushleft}
MSL709*	 Business Research Methods	
\end{flushleft}


1.5	 0	 0	 1.5


\begin{flushleft}
MSL711*	 Strategic Management	
\end{flushleft}


3	 0	 0	 3


\begin{flushleft}
MSL712*	 Ethics \& Values Based Leadership	
\end{flushleft}


1.5	 0	 0	 1.5


\begin{flushleft}
MSL713*	 Information Systems Management	
\end{flushleft}


3	 0	 0	 3


\begin{flushleft}
MSL720*	 Macroeconomic Environment of Business	 3	 0	 0	 3
\end{flushleft}


\begin{flushleft}
MSL745	 Operations Management	
\end{flushleft}


3	 0	 0	 3


\begin{flushleft}
MSL760	 Marketing Management	
\end{flushleft}


3	 0	 0	 3


\begin{flushleft}
MSL780*	 Managerial Economics	
\end{flushleft}


1.5	 0	 0	 1.5


\begin{flushleft}
MSD891	 Major Project (Unique Core)	
\end{flushleft}


0	 0	 12	6


\begin{flushleft}
Notes:
\end{flushleft}


\begin{flushleft}
The UC will include the major project which would focus on a research
\end{flushleft}


\begin{flushleft}
driven application of skills acquired in a particular functional area,
\end{flushleft}


\begin{flushleft}
through the programme.				
\end{flushleft}


\begin{flushleft}
* These are new courses which have been designed and/or modified
\end{flushleft}


\begin{flushleft}
as a part of the curriculum review.
\end{flushleft}


\begin{flushleft}
** MSL706 was initially an elective, MSL887. This course's content is
\end{flushleft}


\begin{flushleft}
the same, only the number has been changed to now reflect a core
\end{flushleft}


\begin{flushleft}
course.
\end{flushleft}


	


\begin{flushleft}
Total Credits				 36
\end{flushleft}





\begin{flushleft}
MSL716	 Fundamentals of Management Systems	
\end{flushleft}


3	 0	 0	 3


\begin{flushleft}
MSL717*	 Business Systems Analysis \& Design	
\end{flushleft}


3	 0	 0	 3


\begin{flushleft}
MSL811	 Management Control Systems	
\end{flushleft}


3	 0	 0	 3


\begin{flushleft}
MSL802	 Management of Intellectual Property Rights	 3	 0	 0	 3
\end{flushleft}


\begin{flushleft}
MSL835	 Labor Legislation and Industrial Relations	 3	 0	 0	 3
\end{flushleft}


\begin{flushleft}
MSL704	 Science \& Technology Policy Systems	
\end{flushleft}


3	 0	 0	 3


\begin{flushleft}
MSL801	Technology Forecasting \& Assessment	
\end{flushleft}


3	 0	 0	 3


\begin{flushleft}
MSL803	 Technical Entrepreneurship	
\end{flushleft}


3	 0	 0	 3


\begin{flushleft}
MSL806*	 Mergers \& Acquisitions	
\end{flushleft}


3	 0	 0	 3


\begin{flushleft}
MSL807*	 Selected Topics in Strategic Management	 1	 0	 0	 1
\end{flushleft}


\begin{flushleft}
MSL808*	 Systems Thinking 	
\end{flushleft}


3	 0	 0	 3


\begin{flushleft}
MSL812	 Flexible Systems Management	
\end{flushleft}


3	 0	 0	 3


\begin{flushleft}
MSL813	 Systems Methodology for Management	
\end{flushleft}


3	 0	 0	 3


\begin{flushleft}
MSL817	 Systems Waste \& Sustainability	
\end{flushleft}


3	 0	 0	 3


\begin{flushleft}
MSL819	 Business Process Re-engineering	
\end{flushleft}


3	 0	 0	 3


\begin{flushleft}
MSL820	 Global Business Environment	
\end{flushleft}


3	 0	 0	 3


\begin{flushleft}
MSL821*	 Strategy Execution Excellence	
\end{flushleft}


3	 0	 0	 3


\begin{flushleft}
MSL 822	 International Business	
\end{flushleft}


3	 0	 0	 3


\begin{flushleft}
MSL823	 Strategic Change \& Flexibility	
\end{flushleft}


3	 0	 0	 3


\begin{flushleft}
MSL 824	 Policy Dynamics \& Learning Organization	 3	 0	 0	 3
\end{flushleft}


\begin{flushleft}
MSL 825	 Strategies in Functional Management	
\end{flushleft}


3	 0	 0	 3


\begin{flushleft}
MSL826	 Business Ethics	
\end{flushleft}


3	 0	 0	 3


\begin{flushleft}
MSL827	 International Competitiveness	
\end{flushleft}


3	 0	 0	 3


\begin{flushleft}
MSL828	 Global Strategic Management	
\end{flushleft}


3	 0	 0	 3


\begin{flushleft}
MSL829	 Current \& Emerging Issues in Strategic Management	3	 0	 0	 3
\end{flushleft}


\begin{flushleft}
MSL851*	 Strategic Alliance	
\end{flushleft}


1.5	 0	 0	 1.5


\begin{flushleft}
MSL714	 Organizational Dynamics and Environment	 3	 0	 0	 3
\end{flushleft}


\begin{flushleft}
MSL830	 Organizational Structure and Processes	
\end{flushleft}


3	 0	 0	 3


\begin{flushleft}
MSL831	 Management of Change	
\end{flushleft}


3	 0	 0	 3


\begin{flushleft}
MSL832	 Managing Innovation for Organizational 	
\end{flushleft}


3	 0	 0	 3


\begin{flushleft}
	Effectiveness
\end{flushleft}


\begin{flushleft}
MSL833	 Organizational Development	
\end{flushleft}


3	 0	 0	 3


\begin{flushleft}
MSL834*	 Managing Diversity at Workplace	
\end{flushleft}


1.5	 0	 0	 1.5


\begin{flushleft}
MSL836*	 International Human Resources Management	 1.5	0	0	 1.5
\end{flushleft}


\begin{flushleft}
MSL839	 Current \& Emerging Issues in	
\end{flushleft}


3	 0	 0	 3


	


\begin{flushleft}
Organizational Management
\end{flushleft}


\begin{flushleft}
MSL804*	 Procurement Management	
\end{flushleft}


3	 0	 0	 3


\begin{flushleft}
MSL805*	 Services Operations Management	
\end{flushleft}


3	 0	 0	 3


\begin{flushleft}
MSL715	 Quality and Environment Management Systems	 3	 0	0	 3
\end{flushleft}


\begin{flushleft}
MSL816	 Total Quality Management	
\end{flushleft}


3	 0	 0	 3


\begin{flushleft}
MSL818	 Industrial Waste Management	
\end{flushleft}


3	 0	 0	 3


\begin{flushleft}
MSL840	 Manufacturing Strategy	
\end{flushleft}


3	 0	 0	 3


\begin{flushleft}
MSL841*	 Supply Chain Analytics	
\end{flushleft}


3	 0	 0	 3


\begin{flushleft}
MSL842*	 Supply Chain Modeling	
\end{flushleft}


3	 0	 0	 3


\begin{flushleft}
MSL843	 Supply Chain Logistics Management	
\end{flushleft}


3	 0	 0	 3


\begin{flushleft}
MSL844	 Systems Reliability, Safety and Maintenance	3	 0	0	 3
\end{flushleft}


\begin{flushleft}
	Management
\end{flushleft}


\begin{flushleft}
MSL845	 Total Project Systems Management	
\end{flushleft}


3	 0	 0	 3


\begin{flushleft}
MSL846	 Total Productivity Management	
\end{flushleft}


3	 0	 0	 3


\begin{flushleft}
MSL848*	 Applied Operations Research	
\end{flushleft}


3	 0	 0	 3


\begin{flushleft}
MSL849	 Current \& Emerging Issues in Manufacturing	3	 0	 0	 3
\end{flushleft}


\begin{flushleft}
	Management
\end{flushleft}


\begin{flushleft}
MSL809*	 Cyber Security: Managing Risks	
\end{flushleft}


3	 0	 0	 3


\begin{flushleft}
MSL810*	 Advanced Data Mining for Business Decisions	1.5	 0	 0	 1.5
\end{flushleft}


\begin{flushleft}
MSL814*	 Data Visualization	
\end{flushleft}


1.5	 0	 0	 1.5


\begin{flushleft}
MSL815	 Decision Support and Expert Systems	
\end{flushleft}


3	 0	 0	 3


\begin{flushleft}
MSL850	 Management of Information Technology	
\end{flushleft}


3	 0	 0	 3





\begin{flushleft}
Streamed Electives (SE)
\end{flushleft}


\begin{flushleft}
Streamed Electives consist of Analytical Skills (AS) Stream and People
\end{flushleft}


\begin{flushleft}
Skills (PS) Stream. The total credits of Streamed Electives would be
\end{flushleft}


\begin{flushleft}
12 -- 6 from AS and 6 from PS.
\end{flushleft}


\begin{flushleft}
a) Analytical Skills (AS) Stream
\end{flushleft}


\begin{flushleft}
MSL719*	 Statistics for Management	
\end{flushleft}


3	 0	 0	 3


\begin{flushleft}
MSL721*	Econometrics	
\end{flushleft}


3	 0	 0	 3


\begin{flushleft}
MSL740	 Quantitative Methods in Management	
\end{flushleft}


3	 0	 0	 3


\begin{flushleft}
MTL732	 Financial Mathematics	
\end{flushleft}


3	 1	 0	 4


\begin{flushleft}
* These are new courses which have been designed and/or modified
\end{flushleft}


\begin{flushleft}
as a part of the curriculum review.
\end{flushleft}


\begin{flushleft}
b) People Skills (PS) Stream
\end{flushleft}


\begin{flushleft}
MSL710	 Creative Problem Solving	
\end{flushleft}


3	 0	 0	 3


\begin{flushleft}
MSL724*	 Business Communication	
\end{flushleft}


1.5	 0	 0	 1.5


\begin{flushleft}
MSL725*	 Business Negotiations	
\end{flushleft}


1.5	 0	 0	 1.5


\begin{flushleft}
MSL727*	 Interpersonal Behavior \& Team Dynamics	 1.5	 0	 0	 1.5
\end{flushleft}


\begin{flushleft}
MSL729*	 Individual Behavior in Organization	
\end{flushleft}


1.5	 0	 0	 1.5


\begin{flushleft}
MSL730*	 Managing With Power	
\end{flushleft}


1.5	 0	 0	 1.5


\begin{flushleft}
MSL731*	 Developing Self Awareness	
\end{flushleft}


1.5	 0	 0	 1.5


\begin{flushleft}
MSL733*	 Organization Theory	
\end{flushleft}


1.5	 0	 0	 1.5


\begin{flushleft}
* These are new courses which have been designed and/or modified
\end{flushleft}


\begin{flushleft}
as a part of the curriculum review.
\end{flushleft}


\begin{flushleft}
Focus Electives (FE)
\end{flushleft}


\begin{flushleft}
MSL723	
\end{flushleft}


\begin{flushleft}
MSL726	
\end{flushleft}


\begin{flushleft}
MSL728	
\end{flushleft}


\begin{flushleft}
EEL767	
\end{flushleft}





\begin{flushleft}
Telecommunication Systems	
\end{flushleft}


3	


\begin{flushleft}
Telecom System Analysis, Planning \& Design	3	
\end{flushleft}


\begin{flushleft}
International Telecommunication Management	 3	
\end{flushleft}


\begin{flushleft}
Telecom Systems	
\end{flushleft}


3	





0	 0	 3


0	0	 3


0	0	 3


0	 0	 3





\begin{flushleft}
Non-credit Core (NC)
\end{flushleft}


\begin{flushleft}
MST893	 Corporate Sector Attachment	
\end{flushleft}


0	 0	 4	 2


\begin{flushleft}
MST894*	 Social Sector Attachment	
\end{flushleft}


0	 0	 2	 1


\begin{flushleft}
* This is a new course which has been designed as a part of the
\end{flushleft}


\begin{flushleft}
curriculum review.
\end{flushleft}





99





\begin{flushleft}
\newpage
MSL707
\end{flushleft}


\begin{flushleft}
Mgmt.
\end{flushleft}


\begin{flushleft}
Accounting
\end{flushleft}





(3-0-0) 3





\begin{flushleft}
SE
\end{flushleft}


\begin{flushleft}
AS-2
\end{flushleft}


\begin{flushleft}
MSL719
\end{flushleft}





\begin{flushleft}
Statistics for
\end{flushleft}


\begin{flushleft}
Mgmt.
\end{flushleft}





\begin{flushleft}
MSL760 MSL713
\end{flushleft}


\begin{flushleft}
Marketing
\end{flushleft}


\begin{flushleft}
Mgmt.
\end{flushleft}





(3-0-0) 3





(3-0-0) 3





\begin{flushleft}
MSL709
\end{flushleft}





(3-0-0) 3





(1.5-0-0)


1.5





\begin{flushleft}
Winter
\end{flushleft}





\begin{flushleft}
II
\end{flushleft}





\begin{flushleft}
IV
\end{flushleft}





\begin{flushleft}
Ethics \&
\end{flushleft}


\begin{flushleft}
Values
\end{flushleft}


\begin{flushleft}
Based
\end{flushleft}


\begin{flushleft}
Leadership
\end{flushleft}





\begin{flushleft}
SE
\end{flushleft}


\begin{flushleft}
SE
\end{flushleft}


\begin{flushleft}
PS-1
\end{flushleft}


\begin{flushleft}
AS-1
\end{flushleft}


\begin{flushleft}
MSL724 MSL740
\end{flushleft}


\begin{flushleft}
Business
\end{flushleft}


\begin{flushleft}
Comm.
\end{flushleft}





(1.5-0-0) 1.5 (1.5-0-0)


1.5


\begin{flushleft}
MST894
\end{flushleft}





\begin{flushleft}
Quantitative
\end{flushleft}


\begin{flushleft}
Methods
\end{flushleft}


\begin{flushleft}
in Mgmt.
\end{flushleft}





\begin{flushleft}
L
\end{flushleft}





\begin{flushleft}
T
\end{flushleft}





\begin{flushleft}
P Total
\end{flushleft}





8





19.5 0





15 20.5 19.5





8





19.5 0





0





(3-0-0) 3





\begin{flushleft}
Social Sector Attachment
\end{flushleft}





\begin{flushleft}
MSL708
\end{flushleft}





\begin{flushleft}
MSL780
\end{flushleft}





\begin{flushleft}
MSL705 MSL745
\end{flushleft}





\begin{flushleft}
MSL711
\end{flushleft}





\begin{flushleft}
Financial
\end{flushleft}


\begin{flushleft}
Mgmt.
\end{flushleft}





\begin{flushleft}
Managerial
\end{flushleft}


\begin{flushleft}
Economics
\end{flushleft}





\begin{flushleft}
HRM
\end{flushleft}


\begin{flushleft}
Systems
\end{flushleft}





\begin{flushleft}
Operations
\end{flushleft}


\begin{flushleft}
Mgmt.
\end{flushleft}





\begin{flushleft}
Strategic
\end{flushleft}


\begin{flushleft}
Mgmt.
\end{flushleft}





(3-0-0) 3





(3-0-0) 3





(3-0-0) 3





(1.5-0-0) 1.5 (1.5-0-0)


1.5





\begin{flushleft}
FE-1
\end{flushleft}


(3-0-0) 3





\begin{flushleft}
SE
\end{flushleft}


\begin{flushleft}
PS-2
\end{flushleft}


(1.5-0-0)


1.5





\begin{flushleft}
MSL720
\end{flushleft}


\begin{flushleft}
Macroeconomic
\end{flushleft}


\begin{flushleft}
Environment
\end{flushleft}


\begin{flushleft}
of Business
\end{flushleft}





19.5





19





(3-0-0) 3


\begin{flushleft}
MST893
\end{flushleft}





\begin{flushleft}
Summer
\end{flushleft}





\begin{flushleft}
III
\end{flushleft}





\begin{flushleft}
MSL712
\end{flushleft}





\begin{flushleft}
Information Business
\end{flushleft}


\begin{flushleft}
Systems
\end{flushleft}


\begin{flushleft}
Research
\end{flushleft}


\begin{flushleft}
Mgmt.
\end{flushleft}


\begin{flushleft}
Methods
\end{flushleft}





\begin{flushleft}
Contact h/week
\end{flushleft}





\begin{flushleft}
Credits
\end{flushleft}





\begin{flushleft}
Courses
\end{flushleft}


\begin{flushleft}
(Number, Abbreviated Title, L-T-P, credits)
\end{flushleft}





\begin{flushleft}
Sem.
\end{flushleft}





\begin{flushleft}
I
\end{flushleft}





\begin{flushleft}
MSL866	 International Marketing	
\end{flushleft}


3	 0	 0	 3


\begin{flushleft}
MSL867	 Industrial Marketing Management	
\end{flushleft}


3	 0	 0	 3


\begin{flushleft}
MSL869	 Current \& Emerging Issues in Marketing	
\end{flushleft}


3	 0	 0	 3


\begin{flushleft}
MSL870*	 Corporate Governance	
\end{flushleft}


1.5	 0	 0	 1.5


\begin{flushleft}
MSL871*	 Banking and Financial Services	
\end{flushleft}


1.5	 0	 0	 1.5


\begin{flushleft}
MSL872	 Working Capital Management 	
\end{flushleft}


3	 0	 0	 3


\begin{flushleft}
MSL873	 Security Analysis \& Portfolio Management 	3	 0	 0	 3
\end{flushleft}


\begin{flushleft}
MSL874*	 Indian Financial System	
\end{flushleft}


1.5	 0	 0	 1.5


\begin{flushleft}
MSL875	 International Financial Management	
\end{flushleft}


3	 0	 0	 3


\begin{flushleft}
MSL879	 Current \& Emerging Issues in Finance 	
\end{flushleft}


3	 0	 0	 3


\begin{flushleft}
MSL734	 Management of Small \& Medium Scale 	
\end{flushleft}


3	 0	 0 3


	


\begin{flushleft}
Industrial Enterprises
\end{flushleft}


\begin{flushleft}
MSL847	 Advanced Methods for Management Research	 3	 0	 0 	 3
\end{flushleft}


\begin{flushleft}
MSL880	 Selected Topics in Management Methodology	3	 0	 0	 3
\end{flushleft}


\begin{flushleft}
MSL881	 Management of Public Sector Enterprises	 3	 0	 0	 3
\end{flushleft}


	


\begin{flushleft}
in India
\end{flushleft}


\begin{flushleft}
MSL889	 Current \& Emerging Issues in Public Sector	 3	 0	 0	 3
\end{flushleft}


\begin{flushleft}
	Management
\end{flushleft}


\begin{flushleft}
MSL897	 Consultancy Process \& Skills	
\end{flushleft}


3	 0	 0	 3


\begin{flushleft}
MSL898	 Consultancy Professional Practice	
\end{flushleft}


3	 0	 0 3


\begin{flushleft}
MSL899	 Current \& Emerging Issues in Consultancy	 3	 0	 0	 3
\end{flushleft}


\begin{flushleft}
	Management
\end{flushleft}


\begin{flushleft}
MSL895*	 Advanced Data Analysis for Management 	 3	 0	 0 	 3
\end{flushleft}


\begin{flushleft}
MSL896*	International Economic Policy	
\end{flushleft}


3	 0	 0	 3


\begin{flushleft}
* These are new courses which have been designed and/or modified
\end{flushleft}


\begin{flushleft}
as a part of the curriculum review.
\end{flushleft}





\begin{flushleft}
Lecture
\end{flushleft}


\begin{flushleft}
courses
\end{flushleft}





\begin{flushleft}
MSL852	 Network System: Applications and Management	3	 0	 0	 3
\end{flushleft}


\begin{flushleft}
MSL853*	 Software Project Management	
\end{flushleft}


3	 0	 0	 3


\begin{flushleft}
MSL854*	 Big Data Analytics \& Data Science	
\end{flushleft}


1.5	 0	 0	 1.5


\begin{flushleft}
MSL855*	 Electronic Commerce	
\end{flushleft}


3	 0	 0	 3


\begin{flushleft}
MSL856*	 Business Intelligence	
\end{flushleft}


3	 0	 0	 3


\begin{flushleft}
MSL858*	 Business Process Management with IT	
\end{flushleft}


1.5	 0	 0	 1.5


\begin{flushleft}
MSL859	 Current and Emerging Issues in IT Management	3	 0	 0	 3
\end{flushleft}


\begin{flushleft}
MSL868*	 Digital Research Methods	
\end{flushleft}


1.5	 0	 0	 1.5


\begin{flushleft}
MSL876*	 Economics of Digital Business	
\end{flushleft}


1.5	 0	 0	 1.5


\begin{flushleft}
MSL877*	 Electronic Government	
\end{flushleft}


1.5	 0	 0	 1.5


\begin{flushleft}
MSL878*	Electronic Payments	
\end{flushleft}


1.5	 0	 0	 1.5


\begin{flushleft}
MSL882*	 Enterprise Cloud Computing	
\end{flushleft}


1.5	 0	 0	 1.5


\begin{flushleft}
MSL883*	 ICTs, Development and Business	
\end{flushleft}


1.5	 0	 0	 1.5


\begin{flushleft}
MSL884*	 Information System Strategy	
\end{flushleft}


3	 0	 0	 3


\begin{flushleft}
MSL885*	 Digital Marketing-Analytics \& Optimization	 3	 0	 0	 3
\end{flushleft}


\begin{flushleft}
MSL886*	 IT Consulting \& Practice	
\end{flushleft}


3	 0	 0	 3


\begin{flushleft}
MSL887*	 Mobile Commerce	
\end{flushleft}


3	 0	 0	 3


\begin{flushleft}
MSL888*	 Data Warehousing for Business Decision	
\end{flushleft}


1.5	 0	 0	 1.5


\begin{flushleft}
MSL891*	 Data Analytics using SPSS	
\end{flushleft}


1.5	 0	 0	 1.5


\begin{flushleft}
MSL892*	 Predictive Analytics	
\end{flushleft}


1.5	 0	 0	 1.5


\begin{flushleft}
MSL893*	 Public Policy Issues in the Information Age	 1.5	 0	 0	 1.5
\end{flushleft}


\begin{flushleft}
MSL894*	 Social Media \& Business Practices	
\end{flushleft}


3	 0	 0	 3


\begin{flushleft}
MSL861	 Market Research	
\end{flushleft}


3	 0	 0	 3


\begin{flushleft}
MSL862	 Product Management	
\end{flushleft}


3	 0	 0	 3


\begin{flushleft}
MSL863	 Advertising and Sales Promotion Management	 3	 0	0	 3
\end{flushleft}


\begin{flushleft}
MSL864*	 Corporate Communication	
\end{flushleft}


3	 0	 0	 3


\begin{flushleft}
MSL865	 Sales Management 	
\end{flushleft}


3	 0	 0	 3





\begin{flushleft}
Corporate Sector Attachment
\end{flushleft}





\begin{flushleft}
SE
\end{flushleft}


\begin{flushleft}
SE
\end{flushleft}


\begin{flushleft}
MSL706 FE-2
\end{flushleft}


\begin{flushleft}
Business
\end{flushleft}


\begin{flushleft}
PS-3
\end{flushleft}


\begin{flushleft}
PS-4
\end{flushleft}


(3-0-0) 3


\begin{flushleft}
Laws
\end{flushleft}


(1.5-0-0) 1.5 (1.5-0-0) 1.5


(3-0-0) 3


\begin{flushleft}
MSD891
\end{flushleft}


\begin{flushleft}
Major
\end{flushleft}


\begin{flushleft}
Project
\end{flushleft}





\begin{flushleft}
PE
\end{flushleft}


\begin{flushleft}
(Credits 9-12)
\end{flushleft}





15-18





\begin{flushleft}
PE
\end{flushleft}


\begin{flushleft}
(Credits 9-12)
\end{flushleft}





15-18





(0-0-12) 6


\begin{flushleft}
SE = Streamed Electives, AS = Analytical Skills Stream, PS = People Skills Stream, FE = Focus Electives, PE = Programme Electives
\end{flushleft}





\begin{flushleft}
Total = 72
\end{flushleft}


100





\begin{flushleft}
\newpage
Programme Code: SMN
\end{flushleft}





\begin{flushleft}
Master of Business Administration (Technology Management)
\end{flushleft}


\begin{flushleft}
Department of Management
\end{flushleft}


\begin{flushleft}
The overall credits structure
\end{flushleft}


\begin{flushleft}
Programme Core
\end{flushleft}


\begin{flushleft}
PC
\end{flushleft}


\begin{flushleft}
(Total 36 Credits)
\end{flushleft}





\begin{flushleft}
Category
\end{flushleft}





\begin{flushleft}
Streamed Electives
\end{flushleft}


\begin{flushleft}
SE
\end{flushleft}


\begin{flushleft}
(Total 12 credits)
\end{flushleft}





\begin{flushleft}
Common Core
\end{flushleft}


\begin{flushleft}
CC
\end{flushleft}





\begin{flushleft}
Unique Core
\end{flushleft}


\begin{flushleft}
UC
\end{flushleft}





\begin{flushleft}
Analytical Skills
\end{flushleft}


\begin{flushleft}
Stream
\end{flushleft}


\begin{flushleft}
AS
\end{flushleft}





\begin{flushleft}
People Skills
\end{flushleft}


\begin{flushleft}
Stream
\end{flushleft}


\begin{flushleft}
PS
\end{flushleft}





30





6





6





6





\begin{flushleft}
Credits
\end{flushleft}





\begin{flushleft}
Focus
\end{flushleft}


\begin{flushleft}
Electives
\end{flushleft}


\begin{flushleft}
FE
\end{flushleft}





\begin{flushleft}
Non-credit
\end{flushleft}


\begin{flushleft}
Core
\end{flushleft}


\begin{flushleft}
NC
\end{flushleft}





\begin{flushleft}
Programme
\end{flushleft}


\begin{flushleft}
Electives
\end{flushleft}


\begin{flushleft}
PE
\end{flushleft}





\begin{flushleft}
Total
\end{flushleft}





6





3





18





72





\begin{flushleft}
Program Core (PC)
\end{flushleft}





\begin{flushleft}
Program Electives (PE)
\end{flushleft}





\begin{flushleft}
MSL705*	 HRM Systems	
\end{flushleft}


1.5	 0	 0	 1.5


\begin{flushleft}
MSL706**	Business Laws	
\end{flushleft}


3	 0	 0	 3


\begin{flushleft}
MSL707*	 Management Accounting	
\end{flushleft}


3	 0	 0	 3


\begin{flushleft}
MSL708*	 Financial Management	
\end{flushleft}


3	 0	 0	 3


\begin{flushleft}
MSL709*	 Business Research Methods	
\end{flushleft}


1.5	 0	 0	 1.5


\begin{flushleft}
MSL711*	 Strategic Management	
\end{flushleft}


3	 0	 0	 3


\begin{flushleft}
MSL712*	 Ethics \& Values Based Leadership	
\end{flushleft}


1.5	 0	 0	 1.5


\begin{flushleft}
MSL713*	 Information Systems Management	
\end{flushleft}


3	 0	 0	 3


\begin{flushleft}
MSL720*	 Macroeconomic Environment of Business	 3	 0	 0	 3
\end{flushleft}


\begin{flushleft}
MSL745	 Operations Management	
\end{flushleft}


3	 0	 0	 3


\begin{flushleft}
MSL760	 Marketing Management	
\end{flushleft}


3	 0	 0	 3


\begin{flushleft}
MSL780*	 Managerial Economics	
\end{flushleft}


1.5	 0	 0	 1.5


\begin{flushleft}
MSD892	 Major Project (Unique Core)	
\end{flushleft}


0	 0	 12	6


\begin{flushleft}
Notes:
\end{flushleft}


\begin{flushleft}
The UC will include the major project which would focus on a research
\end{flushleft}


\begin{flushleft}
driven application of skills acquired in a particular functional area,
\end{flushleft}


\begin{flushleft}
through the programme.				
\end{flushleft}


\begin{flushleft}
* These are new courses which have been designed and/or modified
\end{flushleft}


\begin{flushleft}
as a part of the curriculum review.
\end{flushleft}


\begin{flushleft}
** MSL706 was initially an elective, MSL887. This course's content is
\end{flushleft}


\begin{flushleft}
the same, only the number has been changed to now reflect a core
\end{flushleft}


\begin{flushleft}
course.
\end{flushleft}


	


\begin{flushleft}
Total Credits				 36
\end{flushleft}





\begin{flushleft}
MSL716	 Fundamentals of Management Systems	
\end{flushleft}


3	 0	 0	 3


\begin{flushleft}
MSL717*	 Business Systems Analysis \& Design	
\end{flushleft}


3	 0	 0	 3


\begin{flushleft}
MSL811	 Management Control Systems	
\end{flushleft}


3	 0	 0	 3


\begin{flushleft}
MSL802	 Management of Intellectual Property Rights	 3	 0	 0	 3
\end{flushleft}


\begin{flushleft}
MSL835	 Labor Legislation and Industrial Relations	 3	 0	 0	 3
\end{flushleft}


\begin{flushleft}
MSL704	 Science \& Technology Policy Systems	
\end{flushleft}


3	 0	 0	 3


\begin{flushleft}
MSL801	 Technology Forecasting \& Assessment	
\end{flushleft}


3	 0	 0	 3


\begin{flushleft}
MSL803	 Technical Entrepreneurship	
\end{flushleft}


3	 0	 0	 3


\begin{flushleft}
MSL806*	 Mergers \& Acquisitions	
\end{flushleft}


3	 0	 0	 3


\begin{flushleft}
MSL807*	 Selected Topics in Strategic Management	 1	 0	 0	 1
\end{flushleft}


\begin{flushleft}
MSL808*	 Systems Thinking 	
\end{flushleft}


3	 0	 0	 3


\begin{flushleft}
MSL812	 Flexible Systems Management	
\end{flushleft}


3	 0	 0	 3


\begin{flushleft}
MSL813	 Systems Methodology for Management	
\end{flushleft}


3	 0	 0	 3


\begin{flushleft}
MSL817	 Systems Waste \& Sustainability	
\end{flushleft}


3	 0	 0	 3


\begin{flushleft}
MSL819	 Business Process Re-engineering	
\end{flushleft}


3	 0	 0	 3


\begin{flushleft}
MSL820	 Global Business Environment	
\end{flushleft}


3	 0	 0	 3


\begin{flushleft}
MSL821*	 Strategy Execution Excellence	
\end{flushleft}


3	 0	 0	 3


\begin{flushleft}
MSL822	 International Business	
\end{flushleft}


3	 0	 0	 3


\begin{flushleft}
MSL823	 Strategic Change \& Flexibility	
\end{flushleft}


3	 0	 0	 3


\begin{flushleft}
MSL824	 Policy Dynamics \& Learning Organization	 3	 0	 0	 3
\end{flushleft}


\begin{flushleft}
MSL825	 Strategies in Functional Management	
\end{flushleft}


3	 0	 0	 3


\begin{flushleft}
MSL826	 Business Ethics	
\end{flushleft}


3	 0	 0	 3


\begin{flushleft}
MSL827	 International Competitiveness	
\end{flushleft}


3	 0	 0	 3


\begin{flushleft}
MSL828	 Global Strategic Management	
\end{flushleft}


3	 0	 0	 3


\begin{flushleft}
MSL829	 Current \& Emerging Issues in Strategic Mgmt.	3	 0	 0	 3
\end{flushleft}


\begin{flushleft}
MSL851*	 Strategic Alliance	
\end{flushleft}


1.5	 0	 0	 1.5


\begin{flushleft}
MSL714	 Organizational Dynamics and Environment	 3	 0	 0	 3
\end{flushleft}


\begin{flushleft}
MSL830	 Organizational Structure and Processes	
\end{flushleft}


3	 0	 0	 3


\begin{flushleft}
MSL831	 Management of Change	
\end{flushleft}


3	 0	 0	 3


\begin{flushleft}
MSL832	 Managing Innovation for Organizational 	
\end{flushleft}


3	 0	 0	 3


\begin{flushleft}
	Effectiveness
\end{flushleft}


\begin{flushleft}
MSL833	 Organizational Development	
\end{flushleft}


3	 0	 0	 3


\begin{flushleft}
MSL834*	 Managing Diversity at Workplace	
\end{flushleft}


1.5	 0	 0	 1.5


\begin{flushleft}
MSL836*	 International Human Resources Management	 1.5	0	0	 1.5
\end{flushleft}


\begin{flushleft}
MSL839	 Current \& Emerging Issues in	
\end{flushleft}


3	 0	 0	 3


	


\begin{flushleft}
Organizational Management
\end{flushleft}


\begin{flushleft}
MSL804*	 Procurement Management	
\end{flushleft}


3	 0	 0	 3


\begin{flushleft}
MSL805*	 Services Operations Management	
\end{flushleft}


3	 0	 0	 3


\begin{flushleft}
MSL715	 Quality and Environment Management Systems	 3	 0	0	 3
\end{flushleft}


\begin{flushleft}
MSL816	 Total Quality Management	
\end{flushleft}


3	 0	 0	 3


\begin{flushleft}
MSL818	 Industrial Waste Management	
\end{flushleft}


3	 0	 0	 3


\begin{flushleft}
MSL840	 Manufacturing Strategy	
\end{flushleft}


3	 0	 0	 3


\begin{flushleft}
MSL841*	 Supply Chain Analytics	
\end{flushleft}


3	 0	 0	 3


\begin{flushleft}
MSL842*	 Supply Chain Modeling	
\end{flushleft}


3	 0	 0	 3


\begin{flushleft}
MSL843	 Supply Chain Logistics Management	
\end{flushleft}


3	 0	 0	 3


\begin{flushleft}
MSL844	 Systems Reliability, Safety and Maintenance Mgmt.	 3	 0	0	 3
\end{flushleft}


\begin{flushleft}
MSL845	 Total Project Systems Management	
\end{flushleft}


3	 0	 0	 3


\begin{flushleft}
MSL846	 Total Productivity Management	
\end{flushleft}


3	 0	 0	 3


\begin{flushleft}
MSL848*	 Applied Operations Research	
\end{flushleft}


3	 0	 0	 3


\begin{flushleft}
MSL849	 Current \& Emerging Issues in Manufacturing Mgmt.	 3	 0	0	 3
\end{flushleft}


\begin{flushleft}
MSL809*	 Cyber Security: Managing Risks	
\end{flushleft}


3	 0	 0	 3


\begin{flushleft}
MSL810*	 Advanced Data Mining for Business Decisions	1.5	 0	 0	 1.5
\end{flushleft}


\begin{flushleft}
MSL814*	 Data Visualization	
\end{flushleft}


1.5	 0	 0	 1.5


\begin{flushleft}
MSL815	 Decision Support and Expert Systems	
\end{flushleft}


3	 0	 0	 3


\begin{flushleft}
MSL850	 Management of Information Technology	
\end{flushleft}


3	 0	 0	 3


\begin{flushleft}
MSL852	 Network System: Applications and Management	3	 0	 0	 3
\end{flushleft}


\begin{flushleft}
MSL853*	 Software Project Management	
\end{flushleft}


3	 0	 0	 3


\begin{flushleft}
MSL854*	 Big Data Analytics \& Data Science	
\end{flushleft}


1.5	 0	 0	 1.5


\begin{flushleft}
MSL855*	 Electronic Commerce	
\end{flushleft}


3	 0	 0	 3





\begin{flushleft}
Streamed Electives (SE)
\end{flushleft}


\begin{flushleft}
Streamed Electives consist of Analytical Skills (AS) Stream and People
\end{flushleft}


\begin{flushleft}
Skills (PS) Stream. The total credits of Streamed Electives would be
\end{flushleft}


\begin{flushleft}
12 -- 6 from AS and 6 from PS.
\end{flushleft}


\begin{flushleft}
a) Analytical Skills (AS) Stream
\end{flushleft}


\begin{flushleft}
MSL719*	 Statistics for Management	
\end{flushleft}


3	 0	 0	 3


\begin{flushleft}
MSL721*	Econometrics	
\end{flushleft}


3	 0	 0	 3


\begin{flushleft}
MSL740	 Quantitative Methods in Management	
\end{flushleft}


3	 0	 0	 3


\begin{flushleft}
MTL732	 Financial Mathematics	
\end{flushleft}


3	 1	 0	 4


\begin{flushleft}
* These are new courses which have been designed and/or modified
\end{flushleft}


\begin{flushleft}
as a part of the curriculum review.
\end{flushleft}


\begin{flushleft}
b) People Skills (PS) Stream
\end{flushleft}


\begin{flushleft}
MSL710	 Creative Problem Solving	
\end{flushleft}


3	 0	 0	 3


\begin{flushleft}
MSL724*	 Business Communication	
\end{flushleft}


1.5	 0	 0	 1.5


\begin{flushleft}
MSL725*	 Business Negotiations	
\end{flushleft}


1.5	 0	 0	 1.5


\begin{flushleft}
MSL727*	 Interpersonal Behavior \& Team Dynamics	 1.5	 0	 0	 1.5
\end{flushleft}


\begin{flushleft}
MSL729*	 Individual Behavior in Organization	
\end{flushleft}


1.5	 0	 0	 1.5


\begin{flushleft}
MSL730*	 Managing With Power	
\end{flushleft}


1.5	 0	 0	 1.5


\begin{flushleft}
MSL731*	 Developing Self Awareness	
\end{flushleft}


1.5	 0	 0	 1.5


\begin{flushleft}
MSL733*	 Organization Theory	
\end{flushleft}


1.5	 0	 0	 1.5


\begin{flushleft}
* These are new courses which have been designed and/or modified
\end{flushleft}


\begin{flushleft}
as a part of the curriculum review.
\end{flushleft}


\begin{flushleft}
Focus Electives (FE)
\end{flushleft}


\begin{flushleft}
MSL700	 Fundamentals of Management of Technology	3	
\end{flushleft}


\begin{flushleft}
MSL701	 Strategic Technology Management	
\end{flushleft}


3	


\begin{flushleft}
MSL702	 Management of Innovation and R\&D	
\end{flushleft}


3	


\begin{flushleft}
MSL703	 Management of Technology Transfer and	
\end{flushleft}


3	


\begin{flushleft}
	Absorption
\end{flushleft}





0	0	 3


0	 0	 3


0	 0	 3


0	 0	 3





\begin{flushleft}
Non-credit Core (NC)
\end{flushleft}


\begin{flushleft}
MSC894*	Seminar	
\end{flushleft}


0	 0	 6	 3


\begin{flushleft}
* This is a new course which has been designed as a part of the
\end{flushleft}


\begin{flushleft}
curriculum review.
\end{flushleft}





101





\begin{flushleft}
\newpage
MSL707
\end{flushleft}





\begin{flushleft}
Management
\end{flushleft}


\begin{flushleft}
Accounting
\end{flushleft}





(3-0-0) 3





\begin{flushleft}
SE
\end{flushleft}


\begin{flushleft}
AS-2
\end{flushleft}


\begin{flushleft}
MSL719
\end{flushleft}





\begin{flushleft}
Statistics for
\end{flushleft}


\begin{flushleft}
Management
\end{flushleft}





\begin{flushleft}
MSL760
\end{flushleft}





\begin{flushleft}
Marketing
\end{flushleft}


\begin{flushleft}
Management
\end{flushleft}





(3-0-0) 3





\begin{flushleft}
MSL709
\end{flushleft}


\begin{flushleft}
Business
\end{flushleft}


\begin{flushleft}
Research
\end{flushleft}


\begin{flushleft}
Methods
\end{flushleft}





\begin{flushleft}
MSL705
\end{flushleft}





\begin{flushleft}
MSL712
\end{flushleft}





(1.5-0-0) 1.5





(1.5-0-0) 1.5





\begin{flushleft}
SE
\end{flushleft}


\begin{flushleft}
PS-2
\end{flushleft}


(1.5-0-0) 1.5





\begin{flushleft}
SE
\end{flushleft}


\begin{flushleft}
AS-1
\end{flushleft}


\begin{flushleft}
MSL740
\end{flushleft}





\begin{flushleft}
HRM
\end{flushleft}


\begin{flushleft}
Systems
\end{flushleft}





(1.5-0-0) 1.5





\begin{flushleft}
Ethics and Values
\end{flushleft}


\begin{flushleft}
Based Leadership
\end{flushleft}





\begin{flushleft}
Contact h/week
\end{flushleft}


\begin{flushleft}
L
\end{flushleft}





\begin{flushleft}
T
\end{flushleft}





\begin{flushleft}
P
\end{flushleft}





\begin{flushleft}
Total
\end{flushleft}





\begin{flushleft}
Credits
\end{flushleft}





\begin{flushleft}
Courses
\end{flushleft}


\begin{flushleft}
(Number, Abbreviated Title, L-T-P, credits)
\end{flushleft}





\begin{flushleft}
Sem.
\end{flushleft}





\begin{flushleft}
I
\end{flushleft}





\begin{flushleft}
MSL869	 Current \& Emerging Issues in Marketing	
\end{flushleft}


3	 0	 0	 3


\begin{flushleft}
MSL870*	 Corporate Governance	
\end{flushleft}


1.5	 0	 0	 1.5


\begin{flushleft}
MSL871*	 Banking and Financial Services	
\end{flushleft}


1.5	 0	 0	 1.5


\begin{flushleft}
MSL872	 Working Capital Management 	
\end{flushleft}


3	 0	 0	 3


\begin{flushleft}
MSL873	 Security Analysis \& Portfolio Management 	3	 0	 0	 3
\end{flushleft}


\begin{flushleft}
MSL874*	 Indian Financial System	
\end{flushleft}


1.5	 0	 0	 1.5


\begin{flushleft}
MSL875	 International Financial Management	
\end{flushleft}


3	 0	 0	 3


\begin{flushleft}
MSL879	 Current \& Emerging Issues in Finance 	
\end{flushleft}


3	 0	 0	 3


\begin{flushleft}
MSL734	 Management of Small \& Medium Scale 	
\end{flushleft}


3	 0	 0 3


	


\begin{flushleft}
Industrial Enterprises
\end{flushleft}


\begin{flushleft}
MSL847	 Advanced Methods for Management Research	 3	 0	 0 	 3
\end{flushleft}


\begin{flushleft}
MSL880	 Selected Topics in Management Methodology	3	 0	 0	 3
\end{flushleft}


\begin{flushleft}
MSL881	 Management of Public Sector Enterprises	 3	 0	 0	 3
\end{flushleft}


	


\begin{flushleft}
in India
\end{flushleft}


\begin{flushleft}
MSL889	 Current \& Emerging Issues in Public Sector	 3	 0	 0	 3
\end{flushleft}


\begin{flushleft}
	Management
\end{flushleft}


\begin{flushleft}
MSL897	 Consultancy Process \& Skills	
\end{flushleft}


3	 0	 0	 3


\begin{flushleft}
MSL898	 Consultancy Professional Practice	
\end{flushleft}


3	 0	 0 3


\begin{flushleft}
MSL899	 Current \& Emerging Issues in Consultancy	 3	 0	 0	 3
\end{flushleft}


\begin{flushleft}
	Management
\end{flushleft}


\begin{flushleft}
MSL895*	 Advanced Data Analysis for Management 	 3	 0	 0 	 3
\end{flushleft}


\begin{flushleft}
MSL896*	 International Economic Policy	
\end{flushleft}


3	 0	 0	 3


\begin{flushleft}
* These are new courses which have been designed and/or modified
\end{flushleft}


\begin{flushleft}
as a part of the curriculum review.
\end{flushleft}





\begin{flushleft}
Lecture
\end{flushleft}


\begin{flushleft}
courses
\end{flushleft}





\begin{flushleft}
MSL856*	 Business Intelligence	
\end{flushleft}


3	 0	 0	 3


\begin{flushleft}
MSL858*	 Business Process Management with IT	
\end{flushleft}


1.5	 0	 0	 1.5


\begin{flushleft}
MSL859	 Current and Emerging Issues in IT Mgmt.	 3	 0	 0	 3
\end{flushleft}


\begin{flushleft}
MSL868*	 Digital Research Methods	
\end{flushleft}


1.5	 0	 0	 1.5


\begin{flushleft}
MSL876*	 Economics of Digital Business	
\end{flushleft}


1.5	 0	 0	 1.5


\begin{flushleft}
MSL877*	 Electronic Government	
\end{flushleft}


1.5	 0	 0	 1.5


\begin{flushleft}
MSL878*	 Electronic Payments	
\end{flushleft}


1.5	 0	 0	 1.5


\begin{flushleft}
MSL882*	 Enterprise Cloud Computing	
\end{flushleft}


1.5	 0	 0	 1.5


\begin{flushleft}
MSL883*	 ICTs, Development and Business	
\end{flushleft}


1.5	 0	 0	 1.5


\begin{flushleft}
MSL884*	 Information System Strategy	
\end{flushleft}


3	 0	 0	 3


\begin{flushleft}
MSL885*	 Digital Marketing-Analytics \& Optimization	 3	 0	 0	 3
\end{flushleft}


\begin{flushleft}
MSL886*	 IT Consulting \& Practice	
\end{flushleft}


3	 0	 0	 3


\begin{flushleft}
MSL887*	 Mobile Commerce	
\end{flushleft}


3	 0	 0	 3


\begin{flushleft}
MSL888*	 Data Warehousing for Business Decision	
\end{flushleft}


1.5	 0	 0	 1.5


\begin{flushleft}
MSL891*	 Data Analytics using SPSS	
\end{flushleft}


1.5	 0	 0	 1.5


\begin{flushleft}
MSL892*	 Predictive Analytics	
\end{flushleft}


1.5	 0	 0	 1.5


\begin{flushleft}
MSL893*	 Public Policy Issues in the Information Age	 1.5	 0	 0	 1.5
\end{flushleft}


\begin{flushleft}
MSL894*	 Social Media \& Business Practices	
\end{flushleft}


3	 0	 0	 3


\begin{flushleft}
MSL861	 Market Research	
\end{flushleft}


3	 0	 0	 3


\begin{flushleft}
MSL862	 Product Management	
\end{flushleft}


3	 0	 0	 3


\begin{flushleft}
MSL863	 Advertising and Sales Promotion Management	 3	 0	0	 3
\end{flushleft}


\begin{flushleft}
MSL864*	 Corporate Communication	
\end{flushleft}


3	 0	 0	 3


\begin{flushleft}
MSL865	 Sales Management 	
\end{flushleft}


3	 0	 0	 3


\begin{flushleft}
MSL866	 International Marketing	
\end{flushleft}


3	 0	 0	 3


\begin{flushleft}
MSL867	 Industrial Marketing Management	
\end{flushleft}


3	 0	 0	 3





6





13.5





0





0





13.5





13.5





6





13





0





1





14





13.5





5





13.5





1





0





13.5





14





(3-0-0) 3





\begin{flushleft}
II
\end{flushleft}





\begin{flushleft}
MSL708
\end{flushleft}





\begin{flushleft}
MSL780
\end{flushleft}





(3-0-0) 3





(1.5-0-0) 1.5





\begin{flushleft}
Financial
\end{flushleft}


\begin{flushleft}
Management
\end{flushleft}





\begin{flushleft}
Managerial
\end{flushleft}


\begin{flushleft}
Economics
\end{flushleft}





\begin{flushleft}
MSL713
\end{flushleft}





\begin{flushleft}
Information
\end{flushleft}


\begin{flushleft}
Systems
\end{flushleft}


\begin{flushleft}
Management
\end{flushleft}





(3-0-0) 3





\begin{flushleft}
SE
\end{flushleft}


\begin{flushleft}
PS-1
\end{flushleft}


\begin{flushleft}
MSL724
\end{flushleft}





\begin{flushleft}
Business
\end{flushleft}


\begin{flushleft}
Communication
\end{flushleft}





(1.5-0-0) 1.5





\begin{flushleft}
Quantitative Methods
\end{flushleft}


\begin{flushleft}
in Management
\end{flushleft}





(3-0-0) 3





\begin{flushleft}
Summer
\end{flushleft}





\begin{flushleft}
MSL745
\end{flushleft}


\begin{flushleft}
III
\end{flushleft}





\begin{flushleft}
IV
\end{flushleft}





\begin{flushleft}
Operations
\end{flushleft}


\begin{flushleft}
Management
\end{flushleft}





\begin{flushleft}
Strategic
\end{flushleft}


\begin{flushleft}
Management
\end{flushleft}





(3-0-0) 3





(3-0-0) 3





\begin{flushleft}
MSL706
\end{flushleft}





\begin{flushleft}
SE
\end{flushleft}


\begin{flushleft}
PS-4
\end{flushleft}


(1.5-0-0) 1.5





\begin{flushleft}
Business
\end{flushleft}


\begin{flushleft}
Laws
\end{flushleft}





(3-0-0) 3





\begin{flushleft}
SE
\end{flushleft}


\begin{flushleft}
PS-3
\end{flushleft}


(1.5-0-0) 1.5





\begin{flushleft}
FE-1
\end{flushleft}


(3-0-0) 3





\begin{flushleft}
MSL720
\end{flushleft}





\begin{flushleft}
Macroeconomic Environment of
\end{flushleft}


\begin{flushleft}
Business
\end{flushleft}





(3-0-0) 3





\begin{flushleft}
FE-2
\end{flushleft}


(3-0-0) 3





\begin{flushleft}
PE
\end{flushleft}


\begin{flushleft}
(Credits 6-9)
\end{flushleft}





13.516.5





\begin{flushleft}
MSC894
\end{flushleft}





\begin{flushleft}
Summer
\end{flushleft}





\begin{flushleft}
Seminar
\end{flushleft}





\begin{flushleft}
V
\end{flushleft}


\begin{flushleft}
VI
\end{flushleft}





\begin{flushleft}
MSL711
\end{flushleft}





\begin{flushleft}
MSD892
\end{flushleft}





\begin{flushleft}
Major Project
\end{flushleft}





(0-0-12) 6





\begin{flushleft}
PE
\end{flushleft}


\begin{flushleft}
(Credits 15-18)
\end{flushleft}


\begin{flushleft}
PE
\end{flushleft}


\begin{flushleft}
(Credits 6-9)
\end{flushleft}





15-18





12-15





\begin{flushleft}
SE = Streamed Electives, AS = Analytical Skills Stream, PS = People Skills Stream, FE = Focus Electives, PE = Programme Electives
\end{flushleft}





\begin{flushleft}
Total = 72
\end{flushleft}


102





\begin{flushleft}
\newpage
Programme Code: AMA
\end{flushleft}





\begin{flushleft}
Master of Technology in Engineering Analysis and Design
\end{flushleft}


\begin{flushleft}
Department of Applied Mechanics
\end{flushleft}


\begin{flushleft}
The overall credits structure
\end{flushleft}


\begin{flushleft}
Category
\end{flushleft}





\begin{flushleft}
PC
\end{flushleft}





\begin{flushleft}
PE
\end{flushleft}





\begin{flushleft}
OC
\end{flushleft}





\begin{flushleft}
Total
\end{flushleft}





\begin{flushleft}
Credits
\end{flushleft}





34





12





6





52





\begin{flushleft}
Program Core (PC)
\end{flushleft}


\begin{flushleft}
APL700	
\end{flushleft}


\begin{flushleft}
APL701	
\end{flushleft}


\begin{flushleft}
APL703	
\end{flushleft}


\begin{flushleft}
APL753	
\end{flushleft}


\begin{flushleft}
APL775	
\end{flushleft}





\begin{flushleft}
Experimental Methods for Solids and Fluids	
\end{flushleft}


\begin{flushleft}
Continuum Mechanics	
\end{flushleft}


\begin{flushleft}
Engineering Mathematic and Computation	
\end{flushleft}


\begin{flushleft}
Properties and Selection of Engg. Materials	
\end{flushleft}


\begin{flushleft}
Design Methods	
\end{flushleft}





2	


3	


3	


3	


3	





0	 2	


0	 0	


0	 2	


0	 0	


0	 0	





3


3


4


3


3





3	


3	


3	


3	


2	





0	 2	


0	 0	


0	 0	


0	 0	


0	 4	





4


3


3


3


4





\begin{flushleft}
Product Design (Program Electives)
\end{flushleft}


\begin{flushleft}
APL710	
\end{flushleft}


\begin{flushleft}
APL767	
\end{flushleft}


\begin{flushleft}
APL771	
\end{flushleft}


\begin{flushleft}
APL774	
\end{flushleft}


\begin{flushleft}
APL776	
\end{flushleft}


	


\begin{flushleft}
APL871	
\end{flushleft}


\begin{flushleft}
MCL741	
\end{flushleft}


\begin{flushleft}
MCL749	
\end{flushleft}





\begin{flushleft}
Computer Aided Design	
\end{flushleft}


\begin{flushleft}
Engineering Failure Analysis and Prevention	
\end{flushleft}


\begin{flushleft}
Design Optimization and Decision Theory	
\end{flushleft}


\begin{flushleft}
Modeling \& Analysis of Mechanical Systems	
\end{flushleft}


\begin{flushleft}
Product Design and Feasibility Study	
\end{flushleft}


\begin{flushleft}
(Stream Core)
\end{flushleft}


\begin{flushleft}
Product Reliability	
\end{flushleft}


\begin{flushleft}
Control Engineering	
\end{flushleft}


\begin{flushleft}
Mechatronics Product Design	
\end{flushleft}





3	 0	 0	 3


3	 0	 2	 4


3	 0	 2	 4





\begin{flushleft}
Engineering Mechanics (Program Electives)
\end{flushleft}


\begin{flushleft}
APL705	 Finite Element Method	
\end{flushleft}


\begin{flushleft}
APL711	 Advanced Fluid Mechanics	
\end{flushleft}





3	 0	 2	 4


3	 0	 0	 3





\begin{flushleft}
APL713	 Turbulence and its Modeling	
\end{flushleft}


\begin{flushleft}
APL715	 Physics of Turbulent Flows	
\end{flushleft}


\begin{flushleft}
APL716	 Fluid Transportation Systems 	
\end{flushleft}


\begin{flushleft}
APL720	 Computational Fluid Dynamics	
\end{flushleft}


\begin{flushleft}
APL734	 Advanced Dynamics	
\end{flushleft}


\begin{flushleft}
APL765	 Fracture Mechanics	
\end{flushleft}


\begin{flushleft}
APL796	 Advanced Solid Mechanics	
\end{flushleft}


\begin{flushleft}
APL831	 Theory of Plates and Shells	
\end{flushleft}


\begin{flushleft}
APL835	 Mechanics of Composite Materials	
\end{flushleft}





3	 0	


3	 0	


3	 0	


3	 0	


3	0	


3	0	


3	 0	


3	 0	


3	 0	





0	 3


0	 3


0	 3


2	 4


0	3


0	3


0	 3


0	 3


0	 3





\begin{flushleft}
Materials (Program Electives)
\end{flushleft}


\begin{flushleft}
APL750	
\end{flushleft}


\begin{flushleft}
APL756	
\end{flushleft}


\begin{flushleft}
APL759	
\end{flushleft}


\begin{flushleft}
APL763	
\end{flushleft}


	


\begin{flushleft}
APL764	
\end{flushleft}


\begin{flushleft}
APL765	
\end{flushleft}


\begin{flushleft}
APL767	
\end{flushleft}


\begin{flushleft}
APLXX	
\end{flushleft}





\begin{flushleft}
Modern Engineering Materials 	
\end{flushleft}


\begin{flushleft}
Microstructural Characterization of Materials	
\end{flushleft}


\begin{flushleft}
Phase Transformations 	
\end{flushleft}


\begin{flushleft}
Micro \& Nanoscale Mechanical Behaviour	
\end{flushleft}


\begin{flushleft}
of Materials 	
\end{flushleft}


\begin{flushleft}
Mechanical Behaviour of Biomaterials 	
\end{flushleft}


\begin{flushleft}
Fracture Mechanics 	
\end{flushleft}


\begin{flushleft}
Engineering Failure Analysis and Prevention	
\end{flushleft}


\begin{flushleft}
Selected Topics in Material Engineering	
\end{flushleft}





3	


3	


3	


3	





0	 0	 3


0	2	 4


0	 0	 3


0	 2	 4





3	


3	


3	


3	





0	 0	 3


0	 0	 3


0	 0	3


0	 0	3





\begin{flushleft}
Semester wise course breakup for three streams
\end{flushleft}





\begin{flushleft}
APL775
\end{flushleft}


\begin{flushleft}
I
\end{flushleft}





\begin{flushleft}
Design
\end{flushleft}


\begin{flushleft}
Methods
\end{flushleft}





(3-0-0) 3





\begin{flushleft}
APL753
\end{flushleft}





\begin{flushleft}
Properties \&
\end{flushleft}


\begin{flushleft}
Selection of
\end{flushleft}


\begin{flushleft}
Engg. Materials
\end{flushleft}





\begin{flushleft}
APL703
\end{flushleft}





\begin{flushleft}
Engineering
\end{flushleft}


\begin{flushleft}
Mathematics
\end{flushleft}


\begin{flushleft}
\& Computation
\end{flushleft}





(3-0-0) 3





(3-0-2) 4





\begin{flushleft}
APL701
\end{flushleft}





\begin{flushleft}
Continuum
\end{flushleft}


\begin{flushleft}
Mechanics
\end{flushleft}





(3-0-0) 3





\begin{flushleft}
Contact h/week
\end{flushleft}


\begin{flushleft}
L
\end{flushleft}





\begin{flushleft}
T
\end{flushleft}





\begin{flushleft}
P
\end{flushleft}





\begin{flushleft}
Total
\end{flushleft}





5





14





0





4





18





16





4





12





0





0





12





12





2





6





0





12





18





12





0





0





0





24





24





12





\begin{flushleft}
Credits
\end{flushleft}





\begin{flushleft}
Lecture
\end{flushleft}


\begin{flushleft}
courses
\end{flushleft}





\begin{flushleft}
Courses
\end{flushleft}


\begin{flushleft}
(Number, Abbreviated Title, L-T-P, credits)
\end{flushleft}





\begin{flushleft}
Sem.
\end{flushleft}





\begin{flushleft}
APL700
\end{flushleft}





\begin{flushleft}
Experimental
\end{flushleft}


\begin{flushleft}
Methods
\end{flushleft}


\begin{flushleft}
for Solids \&
\end{flushleft}


\begin{flushleft}
Fluids
\end{flushleft}





(2-0-2) 3





\begin{flushleft}
Summer
\end{flushleft}


\begin{flushleft}
II
\end{flushleft}





\begin{flushleft}
PE-1
\end{flushleft}





\begin{flushleft}
PE-2
\end{flushleft}





\begin{flushleft}
PE-3
\end{flushleft}





\begin{flushleft}
III
\end{flushleft}





\begin{flushleft}
OE2
\end{flushleft}





\begin{flushleft}
AMD811
\end{flushleft}





\begin{flushleft}
PE-4
\end{flushleft}





\begin{flushleft}
IV
\end{flushleft}





\begin{flushleft}
AMD812
\end{flushleft}





\begin{flushleft}
OE-1
\end{flushleft}





\begin{flushleft}
Total = 52
\end{flushleft}


103





\begin{flushleft}
\newpage
Master of Technology in Chemical Engineering
\end{flushleft}





\begin{flushleft}
Programme Code: CHE
\end{flushleft}





\begin{flushleft}
Department of Chemical Engineering
\end{flushleft}


\begin{flushleft}
The overall credits structure
\end{flushleft}


\begin{flushleft}
Category
\end{flushleft}





\begin{flushleft}
PC
\end{flushleft}





\begin{flushleft}
PE
\end{flushleft}





\begin{flushleft}
OE
\end{flushleft}





\begin{flushleft}
Total
\end{flushleft}





\begin{flushleft}
Credits
\end{flushleft}





37





12





3





52





\begin{flushleft}
Program Core
\end{flushleft}





\begin{flushleft}
CLL705	 Petroleum Reservoir Engineering	
\end{flushleft}


3	 0	


\begin{flushleft}
CLL706	 Petroleum Production Engineering	
\end{flushleft}


3	 0	


\begin{flushleft}
CLL707	 Population Balance Modeling	
\end{flushleft}


3	 0	


\begin{flushleft}
CLL720	 Principles of Electrochemical Engineering	
\end{flushleft}


3	 0	


\begin{flushleft}
CLL721	 Electrochemical Methods	
\end{flushleft}


3	0	


\begin{flushleft}
CLL722	 Electrochemical Conversion and Storage	
\end{flushleft}


3	 0	


\begin{flushleft}
	Devices
\end{flushleft}


\begin{flushleft}
CLL723	 Hydrogen Energy and Fuel Cell Technology	 3	 0	
\end{flushleft}


\begin{flushleft}
CLL724	 Environmental Engineering and Waste	
\end{flushleft}


3	 0	


\begin{flushleft}
	Management
\end{flushleft}


\begin{flushleft}
CLL725	 Air Pollution Control Engineering	
\end{flushleft}


3	 0	


\begin{flushleft}
CLL726	 Molecular Modeling of Catalytic Reactions	 3	 0	
\end{flushleft}


\begin{flushleft}
CLL727	 Heterogeneous Catalysis and Catalytic Reactors	3	0	
\end{flushleft}


\begin{flushleft}
CLL728	 Biomass Conversion and Utilization	
\end{flushleft}


3	 0	


\begin{flushleft}
CLL730	 Structure, Transport and Reactions 	
\end{flushleft}


3	 0	


	


\begin{flushleft}
in BioNano Systems	
\end{flushleft}


\begin{flushleft}
CLL732	 Advanced Chemical Engineering	
\end{flushleft}


3	 0	


\begin{flushleft}
	Thermodynamics
\end{flushleft}


\begin{flushleft}
CLL734	 Process Intensification and Novel Reactors	 3	 0	
\end{flushleft}


\begin{flushleft}
CLL735	 Design of Multicomponent Separation Processes	3	0	
\end{flushleft}


\begin{flushleft}
CLL736	 Experimental Characterization of Multiphase	 3	 0	
\end{flushleft}


\begin{flushleft}
	Reactors
\end{flushleft}





\begin{flushleft}
CLL701
\end{flushleft}





\begin{flushleft}
Modelling of
\end{flushleft}


\begin{flushleft}
Transport Processes
\end{flushleft}





(2-0-0) 2


\begin{flushleft}
CLL731
\end{flushleft}





\begin{flushleft}
II
\end{flushleft}





0	 3


0	 3


0	 3


0	 3


0	3


0	 3


0	 3


0	 3


0	 3


0	3


0	 3





\begin{flushleft}
Courses
\end{flushleft}


\begin{flushleft}
(Number, Abbreviated Title, L-T-P, credits)
\end{flushleft}





\begin{flushleft}
Sem.
\end{flushleft}





\begin{flushleft}
I
\end{flushleft}





0	 3


0	 3


0	 3


0	 3


0	3


0	 3





\begin{flushleft}
Advanced Transport
\end{flushleft}


\begin{flushleft}
Phenomena
\end{flushleft}





(3-0-0) 3





\begin{flushleft}
CLL702
\end{flushleft}





\begin{flushleft}
Principles of
\end{flushleft}


\begin{flushleft}
Thermodynamics,
\end{flushleft}


\begin{flushleft}
Reaction Kinetics
\end{flushleft}


\begin{flushleft}
and Reactors
\end{flushleft}





\begin{flushleft}
CLL703
\end{flushleft}





(2-0-0) 2





(3-0-0) 3





\begin{flushleft}
CLL733
\end{flushleft}





\begin{flushleft}
CLD771
\end{flushleft}





\begin{flushleft}
Industrial
\end{flushleft}


\begin{flushleft}
Multiphase
\end{flushleft}


\begin{flushleft}
Reactors
\end{flushleft}





\begin{flushleft}
PE-1
\end{flushleft}


(3-0-0) 3





\begin{flushleft}
Principles of
\end{flushleft}


\begin{flushleft}
Thermodynamics,
\end{flushleft}


\begin{flushleft}
Reaction Kinetics
\end{flushleft}


\begin{flushleft}
and Reactors
\end{flushleft}





\begin{flushleft}
PE-3
\end{flushleft}


(3-0-0) 3





\begin{flushleft}
Minor Project
\end{flushleft}





(0-0-6) 3





0	3


0	 3


2	3


2	3


0	 3


0	 3


0	 3


0	 3


0	 3


0	 3


0	3


0	 3


0	 3


0	 3


0	 3


0	3


0	 3


0	 3


0	3


0	3


0	 3


0	 3


0	


0	


0	


0	


0	


0	





3


3


3


3


1


2





\begin{flushleft}
L
\end{flushleft}





\begin{flushleft}
T
\end{flushleft}





\begin{flushleft}
P
\end{flushleft}





\begin{flushleft}
Total
\end{flushleft}





5





13





0





0





13





13





3





9





0





8





17





13





2





6





0





18





22





14





0





0





0





24





24





12





\begin{flushleft}
Contact h/week
\end{flushleft}





\begin{flushleft}
PE-2
\end{flushleft}


(3-0-0) 3





\begin{flushleft}
CLP704
\end{flushleft}





\begin{flushleft}
Tech. Commu.
\end{flushleft}


\begin{flushleft}
Chem. Engineers
\end{flushleft}





(0-0-2) 1





(3-0-0) 3





0	3


0	 3


2	3





\begin{flushleft}
Credits
\end{flushleft}





\begin{flushleft}
Program Electives
\end{flushleft}





0	3





\begin{flushleft}
Lecture
\end{flushleft}


\begin{flushleft}
courses
\end{flushleft}





\begin{flushleft}
CLL742	 Experimental Characterization	
\end{flushleft}


3	0	


	


\begin{flushleft}
of BioMacromolecules
\end{flushleft}


\begin{flushleft}
CLL743	 Petrochemicals Technology	
\end{flushleft}


3	0	


\begin{flushleft}
CLL761	 Chemical Engineering Mathematics	
\end{flushleft}


3	 0	


\begin{flushleft}
CLL762	 Advanced Computational Techniques	
\end{flushleft}


2	0	


	


\begin{flushleft}
in Chemical Engineering
\end{flushleft}


\begin{flushleft}
CLL766	 Interfacial Engineering	
\end{flushleft}


3	0	


\begin{flushleft}
CLL767	 Structures and Properties of Polymers 	
\end{flushleft}


3	 0	


\begin{flushleft}
CLL768	 Fundamentals of Computational Fluid Dynamics	2	0	
\end{flushleft}


\begin{flushleft}
CLL769	 Applications of Computational Fluid Dynamics 	2	0	
\end{flushleft}


\begin{flushleft}
CLL771	 Introduction to Complex Fluids	
\end{flushleft}


3	 0	


\begin{flushleft}
CLL772	 Transport Phenomena in Complex Fluids	
\end{flushleft}


3	 0	


\begin{flushleft}
CLL773	 Thermodynamics of Complex Fluids	
\end{flushleft}


3	 0	


\begin{flushleft}
CLL774	 Simulation Techniques for Complex Fluids	 3	 0	
\end{flushleft}


\begin{flushleft}
CLL775	 Polymerization Process Modeling	
\end{flushleft}


3	 0	


\begin{flushleft}
CLL776	 Granular Materials	
\end{flushleft}


3	 0	


\begin{flushleft}
CLL777	 Complex Fluids Technology	
\end{flushleft}


3	0	


\begin{flushleft}
CLL778	 Interfacial Behaviour and Transport 	
\end{flushleft}


3	 0	


	


\begin{flushleft}
of Biomolecules	
\end{flushleft}


\begin{flushleft}
CLL779	 Molecular Biotechnology and in-vitro	
\end{flushleft}


3	 0	


\begin{flushleft}
	Diagnostics
\end{flushleft}


\begin{flushleft}
CLL780	 Bioprocessing and Bioseparations	
\end{flushleft}


3	 0	


\begin{flushleft}
CLL781	 Process Operations Scheduling	
\end{flushleft}


3	 0	


\begin{flushleft}
CLL782	 Process Optimization	
\end{flushleft}


3	0	


\begin{flushleft}
CLL783	 Advanced Process Control	
\end{flushleft}


3	 0	


\begin{flushleft}
CLL784	 Process Modeling and Simulation	
\end{flushleft}


3	 0	


\begin{flushleft}
CLL785	 Evolutionary Optimization	
\end{flushleft}


3	0	


\begin{flushleft}
CLL786	 Fine Chemicals Technology	
\end{flushleft}


3	0	


\begin{flushleft}
CLL791	 Chemical Product and Process Integration	 3	 0	
\end{flushleft}


\begin{flushleft}
CLL792	 Chemical Product Development	
\end{flushleft}


3	 0	


	


\begin{flushleft}
and Commercialization
\end{flushleft}


\begin{flushleft}
CLL793	 Membrane Science and Engineering	
\end{flushleft}


3	 0	


\begin{flushleft}
CLL794	 Petroleum Refinery Engineering	
\end{flushleft}


3	 0	


\begin{flushleft}
CLL798	 Selected Topics in Chemical Engineering-I	 3	 0	
\end{flushleft}


\begin{flushleft}
CLL799	 Selected Topics in Chemical Engineering-II	 3	 0	
\end{flushleft}


\begin{flushleft}
CLV796	 Current Topics in Chemical Engineering	
\end{flushleft}


1	 0	


\begin{flushleft}
CLV797	 Recent Advances in Chemical Engineering	 2	 0	
\end{flushleft}





\begin{flushleft}
CLD771	 Minor Project	
\end{flushleft}


0	0	 6	3


\begin{flushleft}
CLD781	 Major Project Part-I	
\end{flushleft}


0	 0	 16	8


\begin{flushleft}
CLD782	 Major Project Part-II	
\end{flushleft}


0	 0	 24	12


\begin{flushleft}
CLL701	 Modelling of Transport Processes	
\end{flushleft}


2	 0	 0	 2


\begin{flushleft}
CLL702	 Principles of Thermodynamics, Reaction	
\end{flushleft}


2	 0	 0	 2


	


\begin{flushleft}
Kinetics and Reactors
\end{flushleft}


\begin{flushleft}
CLL703	 Process Engineering	
\end{flushleft}


3	0	 0	3


\begin{flushleft}
CLP704	 Technical Communication for Chemical	
\end{flushleft}


0	 0	 2	 1


\begin{flushleft}
	Engineers
\end{flushleft}


\begin{flushleft}
CLL731	 Advanced Transport Phenomena	
\end{flushleft}


3	0	 0	3


\begin{flushleft}
CLL733	 Industrial Multiphase Reactors	
\end{flushleft}


3	 0	 0	 3


	


\begin{flushleft}
Total Credits				37
\end{flushleft}





\begin{flushleft}
Summer
\end{flushleft}


\begin{flushleft}
III
\end{flushleft}





\begin{flushleft}
CLD781
\end{flushleft}





\begin{flushleft}
PE-4
\end{flushleft}


(3-0-0) 3





\begin{flushleft}
Major Project Part-I
\end{flushleft}





(0-0-16) 8


\begin{flushleft}
IV
\end{flushleft}





\begin{flushleft}
CLD782
\end{flushleft}


\begin{flushleft}
Major Project Part-II
\end{flushleft}





\begin{flushleft}
OE-1
\end{flushleft}


(3-0-0) 3





(0-0-24) 12





\begin{flushleft}
Total = 52
\end{flushleft}


104





\begin{flushleft}
\newpage
Programme Code: CYM
\end{flushleft}





\begin{flushleft}
Master of Technology in Molecular Engineering : Chemical Synthesis and Analysis
\end{flushleft}


\begin{flushleft}
Department of Chemistry
\end{flushleft}





\begin{flushleft}
The overall credits structure
\end{flushleft}


\begin{flushleft}
PC
\end{flushleft}





\begin{flushleft}
PE
\end{flushleft}





\begin{flushleft}
OE
\end{flushleft}





\begin{flushleft}
Total
\end{flushleft}





\begin{flushleft}
Credits
\end{flushleft}





42





12





-





54


\begin{flushleft}
Program Electives
\end{flushleft}





\begin{flushleft}
CMD806	 Major Project Part-I	
\end{flushleft}


0	 0	 18	9


\begin{flushleft}
CMD807	 Major Project Part-II	
\end{flushleft}


0	 0	 18	9


\begin{flushleft}
CML721	 Design and Synthesis of Organic Molecules	 3	 0	 0	 3
\end{flushleft}


\begin{flushleft}
CML724	 Synthesis of Industrially Important Inorganic	 3	 0	 0	 3
\end{flushleft}


\begin{flushleft}
	Materials
\end{flushleft}


\begin{flushleft}
CML726	 Cheminformatics and Molecular Modelling	 3	 0	 0	 3
\end{flushleft}


\begin{flushleft}
CMP728	Instrumentation Laboratory	
\end{flushleft}


0	0	 6	3


\begin{flushleft}
CML729	Material Characterization	
\end{flushleft}


3	0	 0	3


\begin{flushleft}
CML731	 Chemical Separation and Electroanalytical	 3	 0	 0	 3
\end{flushleft}


\begin{flushleft}
	Methods
\end{flushleft}


\begin{flushleft}
CML737	Applied Spectroscopy	
\end{flushleft}


3	0	 0	3


\begin{flushleft}
CMP722	 Synthesis of Organic and Inorganic
\end{flushleft}


		


\begin{flushleft}
Compounds	
\end{flushleft}


0	0	 6	3


	


\begin{flushleft}
Total Credits				42
\end{flushleft}





\begin{flushleft}
CMD799	Minor project	
\end{flushleft}


\begin{flushleft}
CML723	 Principles and practice of NMR and	
\end{flushleft}


	


\begin{flushleft}
Optical Spectroscopy
\end{flushleft}


\begin{flushleft}
CML733	 Chemistry of Industrial Catalysts	
\end{flushleft}


\begin{flushleft}
CML734	 Chemistry of Nanostructured Materials	
\end{flushleft}


\begin{flushleft}
CML738	 Applications of p-block elements and their	
\end{flushleft}


\begin{flushleft}
	compounds
\end{flushleft}


\begin{flushleft}
CML739	Applied Biocatalysis	
\end{flushleft}


\begin{flushleft}
CML740	 Chemistry of Heterocyclic Compounds	
\end{flushleft}


\begin{flushleft}
CML741	 Organo and organometallic catalysis	
\end{flushleft}


\begin{flushleft}
CML742	 Reagents in Synthetic Transformations	
\end{flushleft}


\begin{flushleft}
CML801	 Molecular Modelling and	
\end{flushleft}


	


\begin{flushleft}
Simulations: Concepts and Techniques
\end{flushleft}





\begin{flushleft}
I
\end{flushleft}





\begin{flushleft}
CML721
\end{flushleft}


\begin{flushleft}
Design \&
\end{flushleft}


\begin{flushleft}
Synthesis
\end{flushleft}





\begin{flushleft}
III
\end{flushleft}





\begin{flushleft}
CMP722
\end{flushleft}





(3-0-0) 3





(3-0-0) 3





(0-0-6) 3





\begin{flushleft}
CML724
\end{flushleft}





\begin{flushleft}
CML729
\end{flushleft}





\begin{flushleft}
CML737
\end{flushleft}





\begin{flushleft}
CMP728
\end{flushleft}





(3-0-0) 3





(3-0-0) 3





(3-0-0) 3





(0-0-6) 3





\begin{flushleft}
CMD805
\end{flushleft}





\begin{flushleft}
PE/OE-3
\end{flushleft}


(3-0-0) 3





\begin{flushleft}
PE/OE-4
\end{flushleft}


(3-0-0) 3





\begin{flushleft}
Inorganic
\end{flushleft}


\begin{flushleft}
Materials
\end{flushleft}





\begin{flushleft}
Major Project
\end{flushleft}


\begin{flushleft}
Part-I
\end{flushleft}





\begin{flushleft}
Cheminformatics
\end{flushleft}





\begin{flushleft}
CML731
\end{flushleft}





\begin{flushleft}
Separation \&
\end{flushleft}


\begin{flushleft}
Electroanalytical
\end{flushleft}





(3-0-0) 3


\begin{flushleft}
II
\end{flushleft}





\begin{flushleft}
CML726
\end{flushleft}





\begin{flushleft}
Material
\end{flushleft}


\begin{flushleft}
Characterization
\end{flushleft}





\begin{flushleft}
Applied
\end{flushleft}


\begin{flushleft}
Spectroscopy
\end{flushleft}





\begin{flushleft}
Lab on
\end{flushleft}


\begin{flushleft}
Synthesis
\end{flushleft}





\begin{flushleft}
Instru.
\end{flushleft}


\begin{flushleft}
Lab.
\end{flushleft}





3	 0	 0	 3


3	 0	 0	 3


3	 0	 0	 3


3	0	


3	 0	


3	 0	


3	 0	


3	 0	





0	3


0	 3


0	 3


0	 3


0	 3





\begin{flushleft}
L
\end{flushleft}





\begin{flushleft}
PE/OE-1
\end{flushleft}


(3-0-0) 3





4





12





0





6





18





15





\begin{flushleft}
PE/OE-2
\end{flushleft}


(3-0-0) 3





4





12





0





6





18





15





2





6





0





18





24





12





0





0





0





18





18





12





\begin{flushleft}
Courses
\end{flushleft}


\begin{flushleft}
(Number, Abbreviated Title, L-T-P, credits)
\end{flushleft}





\begin{flushleft}
Sem.
\end{flushleft}





0	0	 6	3


3	 0	 0	 3





\begin{flushleft}
Lecture
\end{flushleft}


\begin{flushleft}
courses
\end{flushleft}





\begin{flushleft}
Program Core
\end{flushleft}





\begin{flushleft}
Contact h/week
\end{flushleft}


\begin{flushleft}
T
\end{flushleft}





\begin{flushleft}
P
\end{flushleft}





\begin{flushleft}
Total
\end{flushleft}





\begin{flushleft}
Credits
\end{flushleft}





\begin{flushleft}
Category
\end{flushleft}





(0-0-12) 6


\begin{flushleft}
IV
\end{flushleft}





\begin{flushleft}
CMD807
\end{flushleft}





\begin{flushleft}
Major Project
\end{flushleft}


\begin{flushleft}
Part-II
\end{flushleft}





(0-0-24) 12





\begin{flushleft}
Total = 54
\end{flushleft}


105





\begin{flushleft}
\newpage
Programme Code: CEC
\end{flushleft}





\begin{flushleft}
Master of Technology in Construction Technology and Management
\end{flushleft}


\begin{flushleft}
Department of Civil Engineering
\end{flushleft}


\begin{flushleft}
The overall credits structure
\end{flushleft}


\begin{flushleft}
Category
\end{flushleft}





\begin{flushleft}
PC
\end{flushleft}





\begin{flushleft}
PE
\end{flushleft}





\begin{flushleft}
OE
\end{flushleft}





\begin{flushleft}
Total
\end{flushleft}





\begin{flushleft}
Credits
\end{flushleft}





37.5





15





0





52.5





\begin{flushleft}
Program Core
\end{flushleft}


\begin{flushleft}
CVC771	 Seminar In Construction Technology and	
\end{flushleft}


0	 0	 2	 0


\begin{flushleft}
	Management-I
\end{flushleft}


\begin{flushleft}
CVC772	 Seminar In Construction Technology and	
\end{flushleft}


0	 0	 2	 0


\begin{flushleft}
	Management-II
\end{flushleft}


\begin{flushleft}
CVD772	 Major Project Part-I (CEC)	
\end{flushleft}


0	 0	 18	9


\begin{flushleft}
CVD773	 Major Project Part-II (CEC)	
\end{flushleft}


0	 0	 24	12


\begin{flushleft}
CVL772	 Construction Project Management 	
\end{flushleft}


3	 0	 0	 3


\begin{flushleft}
CVL773	 Quantitative Methods in Construction 	
\end{flushleft}


3	 0	 0	 3


\begin{flushleft}
	Management
\end{flushleft}


\begin{flushleft}
CVL774	 Construction Contract Management	
\end{flushleft}


3	 0	 0	 3


\begin{flushleft}
CVL775	 Construction Economics and Finance	
\end{flushleft}


3	 0	 0	 3


\begin{flushleft}
CVL776	 Construction Practices and Equipment	
\end{flushleft}


3	 0	 0	 3


\begin{flushleft}
CVP772	 Computational Laboratory for Construction	 0	 0	 3	 1.5
\end{flushleft}


\begin{flushleft}
	Management
\end{flushleft}


	


\begin{flushleft}
Total Credits			37.5
\end{flushleft}


\begin{flushleft}
Program Electives for All Background
\end{flushleft}


\begin{flushleft}
CVD771	 Minor Project (CEC)	
\end{flushleft}


\begin{flushleft}
CVS771	 Independent Study (CEC)	
\end{flushleft}


\begin{flushleft}
MCL754	 Operations Planning and Control 	
\end{flushleft}


\begin{flushleft}
MCL756	 Supply Chain Management 	
\end{flushleft}


\begin{flushleft}
MCL757	Logistics 	
\end{flushleft}


\begin{flushleft}
MCL771	 Value Engineering and Life Cycle Costing 	
\end{flushleft}


\begin{flushleft}
MSL705	 HRM Systems 	
\end{flushleft}


\begin{flushleft}
MSL804	 Procurement Management 	
\end{flushleft}


\begin{flushleft}
MSL822	 International Business 	
\end{flushleft}


\begin{flushleft}
MSL846	 Total Productivity Management 	
\end{flushleft}


\begin{flushleft}
MCL772	 Reliability Engineering 	
\end{flushleft}





0	 0	


0	 3	


3	 0	


3	 0	


3	0	


3	 0	


2	 0	


3	 0	


3	 0	


3	 0	


3	 0	





6	 3


0	 3


0	 3


0	 3


0	3


0	 3


0	 1.5


0	 3


0	 3


0	 3


0	 3





\begin{flushleft}
Program Electives for Civil Engineering Background
\end{flushleft}


\begin{flushleft}
EEL747	 Electrical Systems for Construction Industries	3	0	
\end{flushleft}


\begin{flushleft}
CVL702 	 Ground Improvement and Geosynthetics 	
\end{flushleft}


3	 0	


\begin{flushleft}
CVL714	 Field Exploration and Geotechnical Processes	 3	0	
\end{flushleft}


\begin{flushleft}
CVL715	 Excavation Methods and Underground	
\end{flushleft}


3	 0	


	


\begin{flushleft}
Space Technology
\end{flushleft}


\begin{flushleft}
CVL727	 Environmental risk assessment
\end{flushleft}


	


3	 0	


\begin{flushleft}
CVL747	 Transportation Safety and Environment	
\end{flushleft}


3	 0	


\begin{flushleft}
CVL750	 Intelligent Transportation Systems	
\end{flushleft}


3	0	


\begin{flushleft}
CVL765	 Concrete Mechanics	
\end{flushleft}


3	0	


\begin{flushleft}
CVL771	 Advanced Concrete Technology 	
\end{flushleft}


3	 0	


\begin{flushleft}
CVL777	 Building Science	
\end{flushleft}


3	0	


\begin{flushleft}
CVL778	 Building Services and Maintenance	
\end{flushleft}


3	 0	


\begin{flushleft}
	Management
\end{flushleft}


\begin{flushleft}
CVL779	 Formwork for Concrete Structures	
\end{flushleft}


3	 0	


\begin{flushleft}
CVL820	 Environmental Impact Assessment
\end{flushleft}


	


3	 0	


\begin{flushleft}
CVL838	 Geographic Information Systems	
\end{flushleft}


2	 0	


\begin{flushleft}
CVL840	 Planning and Design of Sustainable	
\end{flushleft}


3	 0	


	


\begin{flushleft}
Transport Systems
\end{flushleft}


\begin{flushleft}
CVL871	 Durability and Repair of Concrete Structures	 3	 0	
\end{flushleft}


\begin{flushleft}
CVL872	 Infrastructure Development and	
\end{flushleft}


3	 0	


\begin{flushleft}
	Management
\end{flushleft}


\begin{flushleft}
CVL873	 Fire Engineering and Design	
\end{flushleft}


3	 0	


\begin{flushleft}
CVL874	 Quality and Safety in Construction	
\end{flushleft}


3	 0	


\begin{flushleft}
CVL875	 Sustainable Materials and Green Buildings	 3	 0	
\end{flushleft}





2	4


0	 3


0	3


0	 3


0	 3


0	 3


0	3


0	3


0	 3


0	3


0	 3


0	


0	


2	


0	





3


3


3


3





0	 3


0	 3


0	 3


0	 3


0	 3





\begin{flushleft}
Program Electives for Electrical Engineering Background
\end{flushleft}


\begin{flushleft}
ELL700	
\end{flushleft}


\begin{flushleft}
ELL712	
\end{flushleft}


\begin{flushleft}
ELL750	
\end{flushleft}


\begin{flushleft}
ELL751	
\end{flushleft}


\begin{flushleft}
ELL752	
\end{flushleft}


\begin{flushleft}
ELL753	
\end{flushleft}


\begin{flushleft}
ELL754	
\end{flushleft}





\begin{flushleft}
Linear Systems Theory	
\end{flushleft}


\begin{flushleft}
Digital Communications	
\end{flushleft}


\begin{flushleft}
Modelling of Electrical Machines 	
\end{flushleft}


\begin{flushleft}
Power Electronic Converters	
\end{flushleft}


\begin{flushleft}
Electric Drive System	
\end{flushleft}


\begin{flushleft}
Physical Phenomena in Electrical Machines 	
\end{flushleft}


\begin{flushleft}
Permanent Magnet Machines 	
\end{flushleft}





3	0	


3	0	


3	 0	


3	 0	


3	 0	


3	 0	


3	 0	





0	3


0	3


0	 3


0	 3


0	 3


0	 3


0	 3





\begin{flushleft}
ELL755	 Variable Reluctance Machines 	
\end{flushleft}


3	 0	


\begin{flushleft}
ELL756	 Special Electrical Machines	
\end{flushleft}


3	 0	


\begin{flushleft}
ELL757	 Energy Efficient Motors 	
\end{flushleft}


3	 0	


\begin{flushleft}
ELL758	 Power Quality 	
\end{flushleft}


3	 0	


\begin{flushleft}
ELL759	 Power Electronic Converters for Renewable 	 3	 0	
\end{flushleft}


	


\begin{flushleft}
Energy Systems
\end{flushleft}


\begin{flushleft}
ELL760	 Switched Mode Power Conversion 	
\end{flushleft}


3	 0	


\begin{flushleft}
ELL761	 Power Electronics for Utility Interface 	
\end{flushleft}


3	 0	


\begin{flushleft}
ELL762	 Intelligent Motor Controllers 	
\end{flushleft}


3	 0	


\begin{flushleft}
ELL763	 Advanced Electric Drives 	
\end{flushleft}


3	 0	


\begin{flushleft}
ELL764	 Electric Vehicles 	
\end{flushleft}


3	 0	


\begin{flushleft}
ELL765	 Smart Grid Technology	
\end{flushleft}


3	 0	


\begin{flushleft}
ELL766	 Appliance Systems	
\end{flushleft}


3	0	


\begin{flushleft}
ELL767	 Mechatronics	
\end{flushleft}


3	0	


\begin{flushleft}
ELL770	 Power System Analysis	
\end{flushleft}


3	0	


\begin{flushleft}
ELL771	 Advanced Power System Protection	
\end{flushleft}


3	 0	


\begin{flushleft}
ELL772	 Planning and Operation of a Smart Grid	
\end{flushleft}


3	 0	


\begin{flushleft}
ELL773	 High Voltage DC Transmission	
\end{flushleft}


3	 0	


\begin{flushleft}
ELL774	 Flexible AC Transmission system	
\end{flushleft}


3	0	


\begin{flushleft}
ELL775	 Power System Dynamics	
\end{flushleft}


3	 0	


\begin{flushleft}
ELL776	 Advanced Power System Optimization	
\end{flushleft}


3	 0	


\begin{flushleft}
ELL777	 Power System operation and control 	
\end{flushleft}


3	 0	


\begin{flushleft}
ELL778	 Dynamic Modelling And Control	
\end{flushleft}


3	 0	


	


\begin{flushleft}
of Sustainable Energy Systems
\end{flushleft}


\begin{flushleft}
ELL850	 Digital Control of Power Electronics 	
\end{flushleft}


3	 0	


	


\begin{flushleft}
and Drive Systems
\end{flushleft}


\begin{flushleft}
ELL851	 Computer Aided Design of 	
\end{flushleft}


3	0	


	


\begin{flushleft}
Electrical Machines
\end{flushleft}


\begin{flushleft}
ELL852	 Condition Monitoring of Electrical Machines 	 3	 0	
\end{flushleft}


\begin{flushleft}
ELL853	 Advanced Topics in Electrical Machines 	
\end{flushleft}


3	 0	


\begin{flushleft}
ELL854	 Selected Topics in Electrical Machines	
\end{flushleft}


3	 0	


\begin{flushleft}
ELL855	 High Power Converters 	
\end{flushleft}


3	 0	


\begin{flushleft}
ELL856	 Advanced Topics in Power Electronics	
\end{flushleft}


3	 0	


\begin{flushleft}
ELL857	 Selected Topics in Power Electronics	
\end{flushleft}


3	 0	


\begin{flushleft}
ELL858	 Advanced Topics in Electric Drives 	
\end{flushleft}


3	 0	


\begin{flushleft}
ELL859	 Selected Topics in Electric Drives 	
\end{flushleft}


3	 0	


\begin{flushleft}
ELL870	 Restructured Power System 	
\end{flushleft}


3	 0	


\begin{flushleft}
ELL871	 Distribution System Operation and Planning	 3	 0	
\end{flushleft}


\begin{flushleft}
ELL872	 Selected Topics in Power System	
\end{flushleft}


3	 0	


\begin{flushleft}
ELL873	 Power System Transient	
\end{flushleft}


3	0	


\begin{flushleft}
ELL874	 Power System Reliability	
\end{flushleft}


3	 0	


\begin{flushleft}
ELP850	 Electrical Machines Laboratory 	
\end{flushleft}


0	 0	


\begin{flushleft}
ELP851	 Power Electronics Laboratory 	
\end{flushleft}


0	 0	


\begin{flushleft}
ELP852	 Electrical Drives Laboratory 	
\end{flushleft}


0	 0	


\begin{flushleft}
ELP853	 DSP Based Control of Power Electronics 	
\end{flushleft}


0	 0	


	


\begin{flushleft}
and Drives Laboratory
\end{flushleft}


\begin{flushleft}
ELP854	 Electrical Machines CAD Laboratory 	
\end{flushleft}


0	 1	


\begin{flushleft}
ELP855	 Smart Grids Laboratory 	
\end{flushleft}


0	 1	


\begin{flushleft}
ELP870	 Power System Lab I	
\end{flushleft}


0	 1	


\begin{flushleft}
ELP871	 Power System Lab II	
\end{flushleft}


0	 1	


\begin{flushleft}
ESL718	 Power Generation, Transmission and	
\end{flushleft}


3	 0	


\begin{flushleft}
	Distribution
\end{flushleft}


\begin{flushleft}
ESL732	 Bioconversion and Processing of Waste	
\end{flushleft}


3	 0	


\begin{flushleft}
ESL734	 Nuclear Energy	
\end{flushleft}


3	0	


\begin{flushleft}
ESL740	 Non-conventional Sources of Energy 	
\end{flushleft}


3	 0	


\begin{flushleft}
ESL746	 Hydrogen Energy	
\end{flushleft}


3	0	


\begin{flushleft}
ESL768 	 Wind Energy and Hydro Power Systems 	
\end{flushleft}


3	 0	


\begin{flushleft}
ESL770 	 Solar Energy Utilization 	
\end{flushleft}


3	 0	


\begin{flushleft}
ESL870 	 Fusion Energy 	
\end{flushleft}


3	 0	





0	


0	


0	


0	


0	





3


3


3


3


3





0	 3


0	 3


0	 3


0	 3


0	 3


0	 3


0	3


0	3


0	3


0	 3


0	 3


0	 3


0	3


0	 3


0	 3


0	 3


0	 3


0	 3


0	3


0	 3


0	 3


0	 3


0	 3


0	 3


0	 3


0	 3


0	 3


0	 3


0	 3


0	 3


0	3


0	 3


3	 1.5


3	 1.5


3	 1.5


3	 1.5


4	


4	


4	


4	


0	





3


3


3


3


3





0	 3


0	3


0	 3


0	3


0	 3


0	 3


0	 3





\begin{flushleft}
Program Electives for Mechanical Engineering Background
\end{flushleft}


\begin{flushleft}
EEL747	
\end{flushleft}


\begin{flushleft}
ESL768	
\end{flushleft}


\begin{flushleft}
ITL709 	
\end{flushleft}


\begin{flushleft}
ITL752 	
\end{flushleft}





106





\begin{flushleft}
Electrical Systems for Construction Industries	3	0	
\end{flushleft}


\begin{flushleft}
Wind Engery \& Hydro Power System	
\end{flushleft}


3	0	


\begin{flushleft}
Maintenance Planning and Control 	
\end{flushleft}


3	 0	


\begin{flushleft}
Bulk Materials Handling
\end{flushleft}


	


2	 0	





2	4


0	3


0	 3


2	 3





\begin{flushleft}
\newpage
MCL784	Computer Aided Manufacturing	
\end{flushleft}


\begin{flushleft}
MCL785	 Advanced Machining Processes	
\end{flushleft}





3	0	 2	4


3	 0	 0	 3





\begin{flushleft}
MCL787	 Welding Science and Technology	
\end{flushleft}





3	 0	 2	 4





\begin{flushleft}
MCL788	Surface Engineering	
\end{flushleft}





3	0	 2	4





\begin{flushleft}
MCL791	 Processing and Mechanics of Composite	
\end{flushleft}


\begin{flushleft}
	Materials
\end{flushleft}





3	 0	 2	 4





\begin{flushleft}
MCL792	 Injection Molding and Mold Design	
\end{flushleft}





2	 0	 2	 3





\begin{flushleft}
MCL818	 Heating, Ventilating and Air-conditioning	
\end{flushleft}





3	 0	 0	 3





\begin{flushleft}
MCL866	 Maintenance management 	
\end{flushleft}





3	 0	 0	 3





\begin{flushleft}
Construction Project
\end{flushleft}


\begin{flushleft}
Management
\end{flushleft}





(3-0-0) 3





\begin{flushleft}
CVL773
\end{flushleft}





\begin{flushleft}
Quantitative
\end{flushleft}


\begin{flushleft}
Methods in
\end{flushleft}


\begin{flushleft}
Construction
\end{flushleft}


\begin{flushleft}
Management
\end{flushleft}





\begin{flushleft}
CVP772
\end{flushleft}





\begin{flushleft}
Computational
\end{flushleft}


\begin{flushleft}
Laboratory for
\end{flushleft}


\begin{flushleft}
Construction	
\end{flushleft}


\begin{flushleft}
Management
\end{flushleft}





\begin{flushleft}
CVC771
\end{flushleft}





\begin{flushleft}
Seminar In
\end{flushleft}


\begin{flushleft}
Construction
\end{flushleft}


\begin{flushleft}
Technology and
\end{flushleft}


\begin{flushleft}
Management-I
\end{flushleft}





(3-0-0) 3





(0-0-3) 1.5





(0-0-2) 0





\begin{flushleft}
CVL775
\end{flushleft}





\begin{flushleft}
CVL776
\end{flushleft}





\begin{flushleft}
CVL774
\end{flushleft}





\begin{flushleft}
CVC772
\end{flushleft}





(3-0-0) 3





(3-0-0) 3





(3-0-0) 3





\begin{flushleft}
PE-4
\end{flushleft}


(3-0-0) 3





\begin{flushleft}
PE-5
\end{flushleft}


(3-0-0) 3





\begin{flushleft}
Construction
\end{flushleft}


\begin{flushleft}
Economics and
\end{flushleft}


\begin{flushleft}
Finance
\end{flushleft}





\begin{flushleft}
Construction
\end{flushleft}


\begin{flushleft}
Practices and
\end{flushleft}


\begin{flushleft}
Equipment
\end{flushleft}





\begin{flushleft}
Construction
\end{flushleft}


\begin{flushleft}
Contract
\end{flushleft}


\begin{flushleft}
Management
\end{flushleft}





\begin{flushleft}
Seminar In
\end{flushleft}


\begin{flushleft}
Construction
\end{flushleft}


\begin{flushleft}
Technology and
\end{flushleft}


\begin{flushleft}
Management-II
\end{flushleft}





\begin{flushleft}
PE-1
\end{flushleft}


(3-0-0) 3





\begin{flushleft}
Contact h/week
\end{flushleft}


\begin{flushleft}
L
\end{flushleft}





\begin{flushleft}
T
\end{flushleft}





\begin{flushleft}
P
\end{flushleft}





\begin{flushleft}
Total
\end{flushleft}





\begin{flushleft}
Credits
\end{flushleft}





\begin{flushleft}
CVL772
\end{flushleft}





\begin{flushleft}
II
\end{flushleft}





2	 4


4	 3


2	 4


0	 3


2	4


0	 3


2	 4


2	4


2	 4


2	 4





\begin{flushleft}
Courses
\end{flushleft}


\begin{flushleft}
(Number, Abbreviated Title, L-T-P, credits)
\end{flushleft}





\begin{flushleft}
Sem.
\end{flushleft}





\begin{flushleft}
I
\end{flushleft}





3	 0	


1	 0	


3	 0	


3	 0	


3	0	


3	 0	


3	 0	


3	0	


3	 0	


3	 0	





\begin{flushleft}
Lecture
\end{flushleft}


\begin{flushleft}
courses
\end{flushleft}





\begin{flushleft}
MCL749	 Mechatronics Product Design	
\end{flushleft}


\begin{flushleft}
MCL751	 Industrial Engineering Systems 	
\end{flushleft}


\begin{flushleft}
MCL753	 Manufacturing Informatics 	
\end{flushleft}


\begin{flushleft}
MCL755	 Service System Design 	
\end{flushleft}


\begin{flushleft}
MCL769	Metal Forming Analysis	
\end{flushleft}


\begin{flushleft}
MCL776	 Advances in Metal Forming	
\end{flushleft}


\begin{flushleft}
MCL778	 Design and Metallurgy of Welded Joints	
\end{flushleft}


\begin{flushleft}
MCL780	Casting Technology	
\end{flushleft}


\begin{flushleft}
MCL781	 Machining Processes and Analysis	
\end{flushleft}


\begin{flushleft}
MCL783	 Automation in Manufacturing	
\end{flushleft}





4





12





0





5





17





13.5





4





12





0





2





14





12





2





6





0





18





24





15





0





0





0





24





24





12





\begin{flushleft}
PE-2
\end{flushleft}


(3-0-0) 3





\begin{flushleft}
PE-3
\end{flushleft}


(3-0-0) 3





(0-0-2) 0





\begin{flushleft}
Summer
\end{flushleft}





\begin{flushleft}
CVD772
\end{flushleft}


\begin{flushleft}
III
\end{flushleft}





\begin{flushleft}
Major Project Part-I
\end{flushleft}


\begin{flushleft}
(CEC)
\end{flushleft}





(0-0-18) 9


\begin{flushleft}
CVD773
\end{flushleft}


\begin{flushleft}
IV
\end{flushleft}





\begin{flushleft}
Major Project Part-II
\end{flushleft}


\begin{flushleft}
(CEC)
\end{flushleft}





(0-0-24) 12





\begin{flushleft}
Total = 52.5
\end{flushleft}


107





\begin{flushleft}
\newpage
Programme Code: CEG
\end{flushleft}





\begin{flushleft}
Master of Technology in Geotechnical and Geoenvironmental Engineering
\end{flushleft}


\begin{flushleft}
Department of Civil Engineering
\end{flushleft}


\begin{flushleft}
The overall credits structure
\end{flushleft}


\begin{flushleft}
Category
\end{flushleft}





\begin{flushleft}
PC
\end{flushleft}





\begin{flushleft}
PE
\end{flushleft}





\begin{flushleft}
OE
\end{flushleft}





\begin{flushleft}
Total
\end{flushleft}





\begin{flushleft}
Credits
\end{flushleft}





36





12





0





48


\begin{flushleft}
Program Electives
\end{flushleft}





\begin{flushleft}
Program Core
\end{flushleft}





	





\begin{flushleft}
Total Credits				36
\end{flushleft}





\begin{flushleft}
CVL700
\end{flushleft}





\begin{flushleft}
Engineering
\end{flushleft}


\begin{flushleft}
Behaviour of Soils
\end{flushleft}





\begin{flushleft}
CVL701
\end{flushleft}





\begin{flushleft}
CVP700
\end{flushleft}





(3-0-0) 3





\begin{flushleft}
Site Investigation
\end{flushleft}


\begin{flushleft}
and
\end{flushleft}


\begin{flushleft}
Foundation Design
\end{flushleft}





\begin{flushleft}
CVL702
\end{flushleft}





\begin{flushleft}
CVL703
\end{flushleft}





\begin{flushleft}
CVP800
\end{flushleft}





(3-0-0) 3





(3-0-0) 3





\begin{flushleft}
CVD700
\end{flushleft}





\begin{flushleft}
Ground Improvement
\end{flushleft}


\begin{flushleft}
and Geosynthetics
\end{flushleft}





\begin{flushleft}
II
\end{flushleft}





\begin{flushleft}
CVD700	Minor Project	
\end{flushleft}


0	0	


\begin{flushleft}
CVL704	 Finite Element Method in Geotechnical Engg.	3	0	
\end{flushleft}


\begin{flushleft}
CVL705	 Slopes and Retaining Structures	
\end{flushleft}


3	 0	


\begin{flushleft}
CVL706	 Soil Dynamics and Earthquake	
\end{flushleft}


3	 0	


	


\begin{flushleft}
Geotechnical Engineering
\end{flushleft}


\begin{flushleft}
CVL707	 Soil-Structure Interaction Analysis	
\end{flushleft}


3	0	


\begin{flushleft}
CVL708	 Geotechnology of Waste Disposal Facilities	 3	 0	
\end{flushleft}


\begin{flushleft}
CVL709	 Offshore Geotechnical Engineering	
\end{flushleft}


3	 0	


\begin{flushleft}
CVL800	 Emerging Topics in Geotechnical Engineering	3	0	
\end{flushleft}


\begin{flushleft}
CVL801	 Constitutive Modelling in Geotechnics	
\end{flushleft}


3	 0	


\begin{flushleft}
CVS800	Independent Study	
\end{flushleft}


0	3	





12	6


24	12


0	 3


0	 3


0	 3


0	 3


6	 3


6	3





\begin{flushleft}
Courses
\end{flushleft}


\begin{flushleft}
(Number, Abbreviated Title, L-T-P, credits)
\end{flushleft}





\begin{flushleft}
Sem.
\end{flushleft}





\begin{flushleft}
I
\end{flushleft}





0	 0	


0	 0	


3	 0	


3	 0	


3	 0	


3	 0	


0	 0	


0	0	





\begin{flushleft}
Soil Engineering Lab
\end{flushleft}





(0-0-6) 3





\begin{flushleft}
PE-1
\end{flushleft}


(3-0-0) 3





\begin{flushleft}
Contact h/week
\end{flushleft}





6	3


0	3


0	 3


0	 3


0	3


0	 3


0	 3


0	3


0	 3


0	3





\begin{flushleft}
L
\end{flushleft}





\begin{flushleft}
T
\end{flushleft}





\begin{flushleft}
P
\end{flushleft}





\begin{flushleft}
Total
\end{flushleft}





\begin{flushleft}
Credits
\end{flushleft}





\begin{flushleft}
Major Project Part-I	
\end{flushleft}


\begin{flushleft}
Major Project Part-II	
\end{flushleft}


\begin{flushleft}
Engineering Behaviour of Soils	
\end{flushleft}


\begin{flushleft}
Site Investigation and Foundation Design	
\end{flushleft}


\begin{flushleft}
Ground Improvement and Geosynthetics 	
\end{flushleft}


\begin{flushleft}
Geoenvironmental Engineering	
\end{flushleft}


\begin{flushleft}
Soil Engineering Lab	
\end{flushleft}


\begin{flushleft}
Geoenvironmental and Geotechnical	
\end{flushleft}


\begin{flushleft}
Engineering Lab
\end{flushleft}





\begin{flushleft}
Lecture
\end{flushleft}


\begin{flushleft}
courses
\end{flushleft}





\begin{flushleft}
CVD800	
\end{flushleft}


\begin{flushleft}
CVD801	
\end{flushleft}


\begin{flushleft}
CVL700 	
\end{flushleft}


\begin{flushleft}
CVL701 	
\end{flushleft}


\begin{flushleft}
CVL702 	
\end{flushleft}


\begin{flushleft}
CVL703 	
\end{flushleft}


\begin{flushleft}
CVP700	
\end{flushleft}


\begin{flushleft}
CVP800	
\end{flushleft}


	





3





9





0





6





15





12





4





12





0





6





18





15





1





3





0





12





15





9





0





0





0





24





24





12





(3-0-0) 3


\begin{flushleft}
Geoenvironmental
\end{flushleft}


\begin{flushleft}
Engineering
\end{flushleft}





\begin{flushleft}
Geoenvironmental and
\end{flushleft}


\begin{flushleft}
Geotechnical Engg. Lab/
\end{flushleft}


\begin{flushleft}
Minor Project
\end{flushleft}


\begin{flushleft}
(for Part Time Students)
\end{flushleft}





\begin{flushleft}
PE-2
\end{flushleft}


(3-0-0) 3





\begin{flushleft}
PE-3
\end{flushleft}


(3-0-0) 3





(0-0-6) 3


\begin{flushleft}
Summer
\end{flushleft}


\begin{flushleft}
III
\end{flushleft}





\begin{flushleft}
IV
\end{flushleft}





\begin{flushleft}
CVD800
\end{flushleft}





\begin{flushleft}
Major Project Part-I
\end{flushleft}





(0-0-12) 6





\begin{flushleft}
PE-4
\end{flushleft}


(3-0-0) 3





\begin{flushleft}
CVD801
\end{flushleft}





\begin{flushleft}
Major Project Part-II
\end{flushleft}





(0-0-24) 12





\begin{flushleft}
Total = 48
\end{flushleft}


108





\begin{flushleft}
\newpage
Master of Technology in Transportation Engineering
\end{flushleft}





\begin{flushleft}
Programme Code: CEP
\end{flushleft}





\begin{flushleft}
Department of Civil Engineering
\end{flushleft}


\begin{flushleft}
The overall credits structure
\end{flushleft}


\begin{flushleft}
Category
\end{flushleft}





\begin{flushleft}
PC
\end{flushleft}





\begin{flushleft}
PE
\end{flushleft}





\begin{flushleft}
OC
\end{flushleft}





\begin{flushleft}
Total
\end{flushleft}





\begin{flushleft}
Credits
\end{flushleft}





36





18





0





54





\begin{flushleft}
Including 6 Credits of Restricted Electives
\end{flushleft}





\begin{flushleft}
Program Core
\end{flushleft}





\begin{flushleft}
Program Electives
\end{flushleft}





	





\begin{flushleft}
Total Credits				36
\end{flushleft}





\begin{flushleft}
CVL743	 Airport Planning and Design	
\end{flushleft}


3	 0	


\begin{flushleft}
CVL744	 Transportation Infrastructure Design	
\end{flushleft}


2	 0	


\begin{flushleft}
CVL745	 Modeling of Pavement Materials	
\end{flushleft}


2	 0	


\begin{flushleft}
CVL746	 Public Transportation Systems	
\end{flushleft}


3	0	


\begin{flushleft}
CVL747	 Transportation Safety and Environment	
\end{flushleft}


3	 0	


\begin{flushleft}
CVL750	 Intelligent Transportation Systems	
\end{flushleft}


3	0	


\begin{flushleft}
CVL840	 Planning and Design of Sustainable	
\end{flushleft}


3	 0	


	


\begin{flushleft}
Transport Systems	
\end{flushleft}


\begin{flushleft}
CVL841	 Advanced Transportation Modelling	
\end{flushleft}


2	0	


\begin{flushleft}
CVL842	 Geometric Design of Roads	
\end{flushleft}


2	 0	


\begin{flushleft}
CVL844	 Transportation Infrastructure Management	 3	 0	
\end{flushleft}


\begin{flushleft}
CVL845	 Viscoelastic Behavior of Bituminous Materials	3	0	
\end{flushleft}


\begin{flushleft}
CVL846	 Transportation System Management	
\end{flushleft}


3	 0	


\begin{flushleft}
CVL847	 Transportation Economics	
\end{flushleft}


3	0	


\begin{flushleft}
CVL849	 Traffic Flow Modelling	
\end{flushleft}


3	 0	


\begin{flushleft}
CVL850	 Transportation Logistics	
\end{flushleft}


3	0	


\begin{flushleft}
CVL851	 Special Topics in Transportation Engineering	 3	 0	
\end{flushleft}


\begin{flushleft}
CVS754	Independent Study	
\end{flushleft}


0	3	





18	9


24	12


2	4


2	 4


2	 4


6	3





\begin{flushleft}
Restricted Electives (6 Credits)
\end{flushleft}


\begin{flushleft}
CVL763	
\end{flushleft}


	


\begin{flushleft}
CVL729	
\end{flushleft}


	


\begin{flushleft}
MCL761	
\end{flushleft}


\begin{flushleft}
CVL731	
\end{flushleft}


	


\begin{flushleft}
CVS753	
\end{flushleft}





\begin{flushleft}
Analytical \& Numerical Methods in 	
\end{flushleft}


\begin{flushleft}
Structural Engineering
\end{flushleft}


\begin{flushleft}
Environmental Statistics and 	
\end{flushleft}


\begin{flushleft}
Experimental Design
\end{flushleft}


\begin{flushleft}
Probability and Statistics	
\end{flushleft}


\begin{flushleft}
Optimization Techniques in 	
\end{flushleft}


\begin{flushleft}
Water Resources
\end{flushleft}


\begin{flushleft}
Minor Project in Transportation Engineering	
\end{flushleft}





\begin{flushleft}
CVL741
\end{flushleft}





\begin{flushleft}
II
\end{flushleft}





\begin{flushleft}
Urban \&
\end{flushleft}


\begin{flushleft}
Regional Transport
\end{flushleft}





\begin{flushleft}
Planning
\end{flushleft}





\begin{flushleft}
CVL740
\end{flushleft}





\begin{flushleft}
Pavement Materials
\end{flushleft}


\begin{flushleft}
and Design of
\end{flushleft}


\begin{flushleft}
Pavements
\end{flushleft}





(3-0-2) 4





(3-0-2) 4





\begin{flushleft}
PE-1
\end{flushleft}


(2-0-2) 3





\begin{flushleft}
PE-2
\end{flushleft}


(2-0-2) 3





\begin{flushleft}
IV
\end{flushleft}





3	 0	 0	 3


3	 0	 0	 3


0	 0	 6	 3





\begin{flushleft}
RE-1*
\end{flushleft}


\begin{flushleft}
(3-0-0 or
\end{flushleft}


2-0-2) 3





\begin{flushleft}
CVL742
\end{flushleft}





\begin{flushleft}
Traffic
\end{flushleft}


\begin{flushleft}
Engineering
\end{flushleft}





(3-0-2) 4


\begin{flushleft}
PE-3
\end{flushleft}


(2-0-2) 3





\begin{flushleft}
RE-1\#
\end{flushleft}


\begin{flushleft}
(3-0-0 or
\end{flushleft}


2-0-2) 3





\begin{flushleft}
Contact h/week
\end{flushleft}





2	3


2	 3


0	 3


0	3


0	 3


0	3


0	 3


0	3


0	 3


0	3





\begin{flushleft}
L
\end{flushleft}





\begin{flushleft}
T
\end{flushleft}





\begin{flushleft}
P
\end{flushleft}





\begin{flushleft}
Total
\end{flushleft}





\begin{flushleft}
Min. 11
\end{flushleft}


\begin{flushleft}
Max. 12
\end{flushleft}





0





\begin{flushleft}
Min. 6
\end{flushleft}


\begin{flushleft}
Max. 8
\end{flushleft}





18





15





\begin{flushleft}
Min. 6
\end{flushleft}


\begin{flushleft}
Max. 8
\end{flushleft}





12





12





\begin{flushleft}
Min. 8
\end{flushleft}


\begin{flushleft}
Max. 9
\end{flushleft}





\begin{flushleft}
CVS852
\end{flushleft}





\begin{flushleft}
Summer
\end{flushleft}





\begin{flushleft}
III
\end{flushleft}





2	 0	 2	 3





\begin{flushleft}
Courses
\end{flushleft}


\begin{flushleft}
(Number, Abbreviated Title, L-T-P, credits)
\end{flushleft}





\begin{flushleft}
Sem.
\end{flushleft}





\begin{flushleft}
I
\end{flushleft}





3	 0	 0	 3





0	 3


2	 3


2	 3


0	3


0	 3


0	3


0	 3





\begin{flushleft}
Credits
\end{flushleft}





\begin{flushleft}
Major Project Part-I	
\end{flushleft}


0	 0	


\begin{flushleft}
Major Project Part-II	
\end{flushleft}


0	 0	


\begin{flushleft}
Pavement Materials and Design of Pavements	3	0	
\end{flushleft}


\begin{flushleft}
Urban and Regional Transportation Planning	 3	 0	
\end{flushleft}


\begin{flushleft}
Traffic Engineering	
\end{flushleft}


3	 0	


\begin{flushleft}
Advanced Topics in Transportation Engineering	 0	0	
\end{flushleft}





\begin{flushleft}
Lecture
\end{flushleft}


\begin{flushleft}
courses
\end{flushleft}





\begin{flushleft}
CVD853	
\end{flushleft}


\begin{flushleft}
CVD854	
\end{flushleft}


\begin{flushleft}
CVL740	
\end{flushleft}


\begin{flushleft}
CVL741	
\end{flushleft}


\begin{flushleft}
CVL742	
\end{flushleft}


\begin{flushleft}
CVS852	
\end{flushleft}





\begin{flushleft}
Advanced Topics in Transportation Engineering
\end{flushleft}





3





(0-0-6) 3


\begin{flushleft}
PE-4
\end{flushleft}


(2-0-2) 3





\begin{flushleft}
CVD853
\end{flushleft}





\begin{flushleft}
Major Project Part-I
\end{flushleft}





(0-0-18) 9





\begin{flushleft}
CVD854
\end{flushleft}





\begin{flushleft}
Major Project Part-II
\end{flushleft}





(0-0-24) 12





0





0





6





6





12





3





0





18





24





12





\begin{flushleft}
*	 Should be listed in restricted elective course category.
\end{flushleft}


\begin{flushleft}
\#	Any course (relevant to research area) offered in that semester with consent of thesis supervisor. Alternatively minor project can be opted.
\end{flushleft}





\begin{flushleft}
Total = 54
\end{flushleft}


109





\begin{flushleft}
\newpage
Programme Code: CES
\end{flushleft}





\begin{flushleft}
Master of Technology in Structural Engineering
\end{flushleft}


\begin{flushleft}
Department of Civil Engineering
\end{flushleft}


\begin{flushleft}
The overall credits structure
\end{flushleft}


\begin{flushleft}
PC
\end{flushleft}





\begin{flushleft}
PE
\end{flushleft}





\begin{flushleft}
OC
\end{flushleft}





\begin{flushleft}
Total
\end{flushleft}





\begin{flushleft}
Credits
\end{flushleft}





42





12





0





54





\begin{flushleft}
CVD757	 Major Project Part-I (CES)	
\end{flushleft}


0	 0	 18	9


\begin{flushleft}
CVD758	 Major Project Part-II (CES)	
\end{flushleft}


0	 0	 18	9


\begin{flushleft}
CVL756	 Advanced Structural Analysis	
\end{flushleft}


3	0	 0	3


\begin{flushleft}
CVL757	 Finite Element Methods in Structural	
\end{flushleft}


2	 0	 2	 3


\begin{flushleft}
	Engineering
\end{flushleft}


\begin{flushleft}
CVL758	 Solid Mechanics in Structural Engineering	 3	 0	 0	 3
\end{flushleft}


\begin{flushleft}
CVL759	 Structural Dynamics	
\end{flushleft}


3	0	 0	3


\begin{flushleft}
CVL760	 Theory of Concrete Structures	
\end{flushleft}


3	 0	 0	 3


\begin{flushleft}
CVL761	 Theory of Steel Structures	
\end{flushleft}


3	 0	 0	 3


\begin{flushleft}
CVL762	 Earthquake Analysis and Design	
\end{flushleft}


3	 0	 0	 3


\begin{flushleft}
CVP756	 Structural Engineering Laboratory	
\end{flushleft}


0	 0	 6	 3


	


\begin{flushleft}
Total Credits				42
\end{flushleft}


\begin{flushleft}
Program Electives
\end{flushleft}


\begin{flushleft}
CVD756	 Minor Project in Structural Engineering	
\end{flushleft}


\begin{flushleft}
CVL763	 Analytical and Numerical Methods for	
\end{flushleft}


	


\begin{flushleft}
Structural Engineering
\end{flushleft}


\begin{flushleft}
CVL764	 Blast Resistant Design of Structures	
\end{flushleft}


\begin{flushleft}
CVL765	 Concrete Mechanics	
\end{flushleft}


\begin{flushleft}
CVL766	 Design of Bridge Structures	
\end{flushleft}





\begin{flushleft}
II
\end{flushleft}





2	 0	 2	 3


3	0	 0	3


3	 0	 0	 3





\begin{flushleft}
Courses
\end{flushleft}


\begin{flushleft}
(Number, Abbreviated Title, L-T-P, credits)
\end{flushleft}





\begin{flushleft}
Sem.
\end{flushleft}





\begin{flushleft}
I
\end{flushleft}





0	 0	 6	 3


3	 0	 0	 3





\begin{flushleft}
CVL767	 Design of Fiber Reinforced Composite	
\end{flushleft}


\begin{flushleft}
	Structures
\end{flushleft}


\begin{flushleft}
CVL768	 Design of Masonry Structures	
\end{flushleft}


\begin{flushleft}
CVL769	 Design of Tall Buildings	
\end{flushleft}


\begin{flushleft}
CVL770	 Prestressed and Composite Structures	
\end{flushleft}


\begin{flushleft}
CVL856	 Strengthening and Retrofitting of Structures	
\end{flushleft}


\begin{flushleft}
CVL857	 Structural Safety and Reliability	
\end{flushleft}


\begin{flushleft}
CVL858	 Theory of Plates and Shells	
\end{flushleft}


\begin{flushleft}
CVL859	 Theory of Structural Stability	
\end{flushleft}


\begin{flushleft}
CVL860	 Advanced Finite Element Method and	
\end{flushleft}


\begin{flushleft}
	Programming
\end{flushleft}


\begin{flushleft}
CVL861	 Analysis and Design of Machine Foundations	
\end{flushleft}


\begin{flushleft}
CVL862	 Design of Offshore Structures	
\end{flushleft}


\begin{flushleft}
CVL863	 General Continuum Mechanics	
\end{flushleft}


\begin{flushleft}
CVL864	 Structural Health Monitoring	
\end{flushleft}


\begin{flushleft}
CVL865	 Structural Vibration Control	
\end{flushleft}


\begin{flushleft}
CVL866	 Wind Resistant Design of Structures	
\end{flushleft}


\begin{flushleft}
CVS756	 Independent Study (CES)	
\end{flushleft}


\begin{flushleft}
CVL771	 Advanced Concrete Technology	
\end{flushleft}


\begin{flushleft}
CVL873	 Fire Engineering and Design	
\end{flushleft}


\begin{flushleft}
CVL779	 Formwork for Concrete Structures	
\end{flushleft}





\begin{flushleft}
Lecture
\end{flushleft}


\begin{flushleft}
courses
\end{flushleft}





\begin{flushleft}
Program Core
\end{flushleft}





\begin{flushleft}
CVL756
\end{flushleft}





\begin{flushleft}
CVL759
\end{flushleft}





(3-0-0) 3





(3-0-0) 3





\begin{flushleft}
Finite Element
\end{flushleft}


\begin{flushleft}
Methods in Structural
\end{flushleft}


\begin{flushleft}
Engineering
\end{flushleft}





(2-0-2) 3





(3-0-0) 3





\begin{flushleft}
CVP756
\end{flushleft}





\begin{flushleft}
CVL762
\end{flushleft}





\begin{flushleft}
CVL760
\end{flushleft}





\begin{flushleft}
CVL761
\end{flushleft}





\begin{flushleft}
Advanced
\end{flushleft}


\begin{flushleft}
Structural Analysis
\end{flushleft}





\begin{flushleft}
Structural
\end{flushleft}


\begin{flushleft}
Engineering
\end{flushleft}


\begin{flushleft}
Laboratory
\end{flushleft}





\begin{flushleft}
Structural
\end{flushleft}


\begin{flushleft}
Dynamics
\end{flushleft}





\begin{flushleft}
Earthquake
\end{flushleft}


\begin{flushleft}
Analysis
\end{flushleft}


\begin{flushleft}
and Design
\end{flushleft}





\begin{flushleft}
CVL757
\end{flushleft}





\begin{flushleft}
Theory of Concrete
\end{flushleft}


\begin{flushleft}
Structures
\end{flushleft}





(3-0-0) 3





(0-0-6) 3





(3-0-0) 3





\begin{flushleft}
CVD757
\end{flushleft}





\begin{flushleft}
PE-3
\end{flushleft}


\begin{flushleft}
PE-4
\end{flushleft}


\begin{flushleft}
(3-0-0) 3 or (3-0-0) 3 or
\end{flushleft}


(2-0-2) 3


(2-0-2) 3





\begin{flushleft}
CVL758
\end{flushleft}





\begin{flushleft}
Solid Mechanics
\end{flushleft}


\begin{flushleft}
in Structural
\end{flushleft}


\begin{flushleft}
Engineering
\end{flushleft}





\begin{flushleft}
Theory of Steel
\end{flushleft}


\begin{flushleft}
Structures
\end{flushleft}





(3-0-0) 3





\begin{flushleft}
PE-1
\end{flushleft}


(3-0-0) 3


\begin{flushleft}
or
\end{flushleft}


(2-0-2) 3


\begin{flushleft}
PE-2
\end{flushleft}


(3-0-0) 3


\begin{flushleft}
or
\end{flushleft}


(2-0-2) 3





3	 0	 0	 3


3	


3	


2	


3	


3	


3	


3	


2	





0	


0	


0	


0	


0	


0	


0	


0	





0	


0	


2	


0	


0	


0	


0	


2	





3


3


3


3


3


3


3


3





2	0	


3	 0	


3	 0	


2	 0	


3	 0	


3	 0	


0	 3	


3	0	


3	 0	


3	 0	





2	3


0	 3


0	 3


2	 3


0	 3


0	 3


0	 3


0	3


0	 3


0	 3





\begin{flushleft}
Credits
\end{flushleft}





\begin{flushleft}
Category
\end{flushleft}





\begin{flushleft}
Contact h/week
\end{flushleft}


\begin{flushleft}
L
\end{flushleft}





\begin{flushleft}
T
\end{flushleft}





\begin{flushleft}
P
\end{flushleft}





\begin{flushleft}
Total
\end{flushleft}





5





(13,


14)





0





(2,4)





(16,


17)





15





4





(11,


12)





0





(6,8)





(18,


19)





15





2





(4,6)





0





(18,


22)





(24,


26)





15





0





0





0





18





18





9





\begin{flushleft}
Summer
\end{flushleft}


\begin{flushleft}
III
\end{flushleft}





\begin{flushleft}
Major Project Part
\end{flushleft}


\begin{flushleft}
I (CES)
\end{flushleft}





(0-0-18) 9


\begin{flushleft}
CVD758
\end{flushleft}


\begin{flushleft}
IV
\end{flushleft}





\begin{flushleft}
Major Project Part
\end{flushleft}


\begin{flushleft}
II (CES)
\end{flushleft}





(0-0-18) 9





\begin{flushleft}
Total = 54
\end{flushleft}


110





\begin{flushleft}
\newpage
Programme Code: CET
\end{flushleft}





\begin{flushleft}
Master of Technology in Construction Engineering and Management
\end{flushleft}


\begin{flushleft}
Department of Civil Engineering
\end{flushleft}


\begin{flushleft}
The overall credits structure
\end{flushleft}


\begin{flushleft}
Category
\end{flushleft}





\begin{flushleft}
PC
\end{flushleft}





\begin{flushleft}
PE
\end{flushleft}





\begin{flushleft}
OC
\end{flushleft}





\begin{flushleft}
Total
\end{flushleft}





\begin{flushleft}
Credits
\end{flushleft}





42





12





0





54


\begin{flushleft}
Program Electives
\end{flushleft}





	





0	


0	


3	


3	


3	





0	


0	


0	


0	


0	





18	9


24	12


0	 3


0	 3


0	 3





3	


3	


3	


0	


0	





0	


0	


0	


0	


0	





0	


0	


0	


3	


3	





3


3


3


1.5


1.5





\begin{flushleft}
Total Credits				42
\end{flushleft}





\begin{flushleft}
Courses
\end{flushleft}


\begin{flushleft}
(Number, Abbreviated Title, L-T-P, credits)
\end{flushleft}





\begin{flushleft}
Sem.
\end{flushleft}





\begin{flushleft}
CVL772
\end{flushleft}


\begin{flushleft}
I
\end{flushleft}





\begin{flushleft}
Construction Project
\end{flushleft}


\begin{flushleft}
Management
\end{flushleft}





(3-0-0) 3





\begin{flushleft}
CVL773
\end{flushleft}





\begin{flushleft}
Quantitative
\end{flushleft}


\begin{flushleft}
Methods in
\end{flushleft}


\begin{flushleft}
Construction 	
\end{flushleft}


\begin{flushleft}
Management
\end{flushleft}





(3-0-0) 3


\begin{flushleft}
CVL775
\end{flushleft}


\begin{flushleft}
II
\end{flushleft}





\begin{flushleft}
CVD776	Minor Project (CET)	
\end{flushleft}


0	0	6	3


\begin{flushleft}
CVL765	Concrete Mechanics	
\end{flushleft}


3	0	0	3


\begin{flushleft}
CVL777	Building Science	
\end{flushleft}


3	0	0	3


\begin{flushleft}
CVL778	 Building Services and Maintenance	
\end{flushleft}


3	 0	 0	 3


\begin{flushleft}
	Management
\end{flushleft}


\begin{flushleft}
CVL779	 Formwork for Concrete Structures	
\end{flushleft}


3	 0	 0	 3


\begin{flushleft}
CVL871	 Durability and Repair of Concrete Structures	 3	 0	 0	 3
\end{flushleft}


\begin{flushleft}
CVL872	 Infrastructure Development and Management	 3	 0	 0	 3
\end{flushleft}


\begin{flushleft}
CVL873	 Fire Engineering and Design	
\end{flushleft}


3	 0	 0	 3


\begin{flushleft}
CVL874	 Quality and Safety in Construction	
\end{flushleft}


3	 0	 0	 3


\begin{flushleft}
CVL875	 Sustainable Materials and Green Buildings	 3	 0	 0	 3
\end{flushleft}


\begin{flushleft}
CVS776	Independent Study (CET)	
\end{flushleft}


0	3	0	3





\begin{flushleft}
Construction
\end{flushleft}


\begin{flushleft}
Economics and Finance
\end{flushleft}





(3-0-0) 3





\begin{flushleft}
CVL776
\end{flushleft}





\begin{flushleft}
Construction
\end{flushleft}


\begin{flushleft}
Practices and
\end{flushleft}


\begin{flushleft}
Equipment
\end{flushleft}





\begin{flushleft}
CVL771
\end{flushleft}





\begin{flushleft}
Advanced
\end{flushleft}


\begin{flushleft}
Concrete
\end{flushleft}


\begin{flushleft}
Technology
\end{flushleft}





\begin{flushleft}
CVP772
\end{flushleft}





(3-0-0) 3





\begin{flushleft}
Computational
\end{flushleft}


\begin{flushleft}
Laboratory for
\end{flushleft}


\begin{flushleft}
Construction	
\end{flushleft}


\begin{flushleft}
Management
\end{flushleft}





\begin{flushleft}
CVL774
\end{flushleft}





\begin{flushleft}
CVP771
\end{flushleft}





\begin{flushleft}
Construction
\end{flushleft}


\begin{flushleft}
Contract
\end{flushleft}


\begin{flushleft}
Management
\end{flushleft}





(3-0-0) 3





(3-0-0) 3





\begin{flushleft}
PE-3
\end{flushleft}


(3-0-0) 3





\begin{flushleft}
PE-4
\end{flushleft}


(3-0-0) 3





\begin{flushleft}
Contact h/week
\end{flushleft}


\begin{flushleft}
L
\end{flushleft}





\begin{flushleft}
T
\end{flushleft}





\begin{flushleft}
P
\end{flushleft}





\begin{flushleft}
Total
\end{flushleft}





\begin{flushleft}
Credits
\end{flushleft}





\begin{flushleft}
CVD777	 Major Project Part-I (CET)	
\end{flushleft}


\begin{flushleft}
CVD778	 Major Project Part-II (CET)	
\end{flushleft}


\begin{flushleft}
CVL771	 Advanced Concrete Technology 	
\end{flushleft}


\begin{flushleft}
CVL772	 Construction Project Management 	
\end{flushleft}


\begin{flushleft}
CVL773	 Quantitative Methods in Construction 	
\end{flushleft}


\begin{flushleft}
	Management
\end{flushleft}


\begin{flushleft}
CVL774	 Construction Contract Management	
\end{flushleft}


\begin{flushleft}
CVL775	 Construction Economics and Finance	
\end{flushleft}


\begin{flushleft}
CVL776	 Construction Practices and Equipment	
\end{flushleft}


\begin{flushleft}
CVP771	 Construction Technology Laboratory	
\end{flushleft}


\begin{flushleft}
CVP772	 Computational Laboratory for Construction	
\end{flushleft}


\begin{flushleft}
	Management
\end{flushleft}





\begin{flushleft}
Lecture
\end{flushleft}


\begin{flushleft}
courses
\end{flushleft}





\begin{flushleft}
Program Core
\end{flushleft}





4





12





0





3





15





13.5





4





12





0





3





15





13.5





2





6





0





18





24





15





0





0





0





24





24





12





\begin{flushleft}
PE-1
\end{flushleft}


(3-0-0) 3





(0-0-3) 1.5


\begin{flushleft}
Construction
\end{flushleft}


\begin{flushleft}
Technology
\end{flushleft}


\begin{flushleft}
Laboratory
\end{flushleft}





\begin{flushleft}
PE-2
\end{flushleft}


(3-0-0) 3





(0-0-3) 1.5





\begin{flushleft}
Summer
\end{flushleft}





\begin{flushleft}
CVD777
\end{flushleft}


\begin{flushleft}
III
\end{flushleft}





\begin{flushleft}
Major Project Part-I
\end{flushleft}


\begin{flushleft}
(CET)
\end{flushleft}





(0-0-18) 9


\begin{flushleft}
CVD778
\end{flushleft}


\begin{flushleft}
IV
\end{flushleft}





\begin{flushleft}
Major Project Part-II
\end{flushleft}


\begin{flushleft}
(CET)
\end{flushleft}





(0-0-24) 12





\begin{flushleft}
Total = 54
\end{flushleft}


111





\begin{flushleft}
\newpage
Programme Code: CEU
\end{flushleft}





\begin{flushleft}
Master of Technology in Rock Engineering and Underground Structures
\end{flushleft}


\begin{flushleft}
Department of Civil Engineering
\end{flushleft}


\begin{flushleft}
The overall credits structure
\end{flushleft}


\begin{flushleft}
Category
\end{flushleft}





\begin{flushleft}
PC
\end{flushleft}





\begin{flushleft}
PE
\end{flushleft}





\begin{flushleft}
OC
\end{flushleft}





\begin{flushleft}
Total
\end{flushleft}





\begin{flushleft}
Credits
\end{flushleft}





36





12





0





48





\begin{flushleft}
Program Core
\end{flushleft}





\begin{flushleft}
Program Electives
\end{flushleft}





\begin{flushleft}
CVD810	 Major Project Part-I	
\end{flushleft}


0	 0	 12	6


\begin{flushleft}
CVD811	 Major Project Part-II	
\end{flushleft}


0	 0	 24	12


\begin{flushleft}
CVL710	 Engineering Properties of Rocks and Rock	 3	 0	 0	 3
\end{flushleft}


\begin{flushleft}
	Masses
\end{flushleft}


\begin{flushleft}
CVL711	 Structural Geology	
\end{flushleft}


3	 0	 0	 3


\begin{flushleft}
CVL712	Slopes and Foundations	
\end{flushleft}


3	0	0	3


\begin{flushleft}
CVL713	 Analysis and Design of Underground	
\end{flushleft}


3	 0	 0	 3


\begin{flushleft}
	Structures
\end{flushleft}


\begin{flushleft}
CVP710	Rock Mechanics Laboratory-I	
\end{flushleft}


0	0	6	3


\begin{flushleft}
CVP810	Rock Mechanics Laboratory-II	
\end{flushleft}


0	0	6	3


	


\begin{flushleft}
Total Credits				36
\end{flushleft}





\begin{flushleft}
CVD710	Minor Project 	
\end{flushleft}


0	0	6	3


\begin{flushleft}
CVL704 	 Finite Element Method in Geotechnical	
\end{flushleft}


3	 0	 0	 3


\begin{flushleft}
	Engineering
\end{flushleft}


\begin{flushleft}
CVL714	 Field Exploration and Geotechnical Processes	3	0	0	3
\end{flushleft}


\begin{flushleft}
CVL715	 Excavation Methods and Underground	
\end{flushleft}


3	 0	 0	 3


	


\begin{flushleft}
Space Technology
\end{flushleft}


\begin{flushleft}
CVL716	Environmental Rock Engineering	
\end{flushleft}


3	0	0	3


\begin{flushleft}
CVL810	 Emerging Topics in Rock Engineering and 	 3	 0	 0	 3
\end{flushleft}


	


\begin{flushleft}
Underground Structures
\end{flushleft}


\begin{flushleft}
CVL811	 Numerical and Computer Methods 	
\end{flushleft}


3	 0	 0	 3


	


\begin{flushleft}
in Geomechanics
\end{flushleft}


\begin{flushleft}
CVS810	Independent Study	
\end{flushleft}


0	0	6	3





\begin{flushleft}
I
\end{flushleft}





\begin{flushleft}
Engineering Properties
\end{flushleft}


\begin{flushleft}
of Rocks and Rock	
\end{flushleft}


\begin{flushleft}
Masses
\end{flushleft}





\begin{flushleft}
CVL711
\end{flushleft}





\begin{flushleft}
Structural Geology
\end{flushleft}


(3-0-0) 3





\begin{flushleft}
CVP710
\end{flushleft}





\begin{flushleft}
Rock Mechanics
\end{flushleft}


\begin{flushleft}
Laboratory-I
\end{flushleft}


(0-0-6) 3





\begin{flushleft}
L
\end{flushleft}





\begin{flushleft}
T
\end{flushleft}





\begin{flushleft}
P
\end{flushleft}





\begin{flushleft}
Total
\end{flushleft}





\begin{flushleft}
Credits
\end{flushleft}





\begin{flushleft}
CVL710
\end{flushleft}





\begin{flushleft}
Contact h/week
\end{flushleft}





\begin{flushleft}
Lecture
\end{flushleft}


\begin{flushleft}
courses
\end{flushleft}





\begin{flushleft}
Courses
\end{flushleft}


\begin{flushleft}
(Number, Abbreviated Title, L-T-P, credits)
\end{flushleft}





\begin{flushleft}
Sem.
\end{flushleft}





3





9





0





6





15





12





4





12





0





6





18





15





1





3





0





12





15





9





0





0





0





24





24





12





\begin{flushleft}
PE-1
\end{flushleft}





(3-0-0) 3





(3-0-0) 3


\begin{flushleft}
CVL712
\end{flushleft}


\begin{flushleft}
II
\end{flushleft}





\begin{flushleft}
Slopes and Foundations
\end{flushleft}





(3-0-0) 3





\begin{flushleft}
CVL713
\end{flushleft}





\begin{flushleft}
Analysis and Design
\end{flushleft}


\begin{flushleft}
of Underground	
\end{flushleft}


\begin{flushleft}
Structures
\end{flushleft}





\begin{flushleft}
CVP810
\end{flushleft}





\begin{flushleft}
Rock Mechanics
\end{flushleft}


\begin{flushleft}
Laboratory-II
\end{flushleft}


(0-0-6) 3





\begin{flushleft}
PE-2
\end{flushleft}


(3-0-0) 3





\begin{flushleft}
PE-3
\end{flushleft}


(3-0-0) 3





(3-0-0) 3


\begin{flushleft}
Summer
\end{flushleft}


\begin{flushleft}
III
\end{flushleft}





\begin{flushleft}
IV
\end{flushleft}





\begin{flushleft}
CVD810
\end{flushleft}





\begin{flushleft}
Major Project Part-I
\end{flushleft}





(0-0-12) 6





\begin{flushleft}
PE-4
\end{flushleft}


(3-0-0) 3





\begin{flushleft}
CVD811
\end{flushleft}





\begin{flushleft}
Major Project Part-II
\end{flushleft}





(0-0-24) 12





\begin{flushleft}
Total = 48
\end{flushleft}


112





\begin{flushleft}
\newpage
Programme Code: CEV
\end{flushleft}





\begin{flushleft}
Master of Technology in Environmental Engineering and Management
\end{flushleft}


\begin{flushleft}
Department of Civil Engineering
\end{flushleft}


\begin{flushleft}
The overall credits structure
\end{flushleft}


\begin{flushleft}
PC
\end{flushleft}





\begin{flushleft}
PE
\end{flushleft}





\begin{flushleft}
OC
\end{flushleft}





\begin{flushleft}
Total
\end{flushleft}





\begin{flushleft}
Credits
\end{flushleft}





39





9





6





54





\begin{flushleft}
CVD720	
\end{flushleft}


\begin{flushleft}
CVD721	
\end{flushleft}


\begin{flushleft}
CVD726	
\end{flushleft}


\begin{flushleft}
CVL720	
\end{flushleft}


\begin{flushleft}
CVL721	
\end{flushleft}


\begin{flushleft}
CVL722	
\end{flushleft}


\begin{flushleft}
CVL723	
\end{flushleft}


\begin{flushleft}
CVL724	
\end{flushleft}


\begin{flushleft}
CVL725	
\end{flushleft}


	





\begin{flushleft}
Major Thesis Part-I	
\end{flushleft}


0	 0	 12	6


\begin{flushleft}
Major Thesis Part-II	
\end{flushleft}


0	 0	 24	12


\begin{flushleft}
Minor Project	
\end{flushleft}


0	 0	 6	 3


\begin{flushleft}
Air Pollution and Control	
\end{flushleft}


3	 0	 0	 3


\begin{flushleft}
Solid Waste Engineering	
\end{flushleft}


3	 0	 0	 3


\begin{flushleft}
Water Engineering	
\end{flushleft}


3	 0	 0	 3


\begin{flushleft}
Wastewater Engineering	
\end{flushleft}


3	 0	 0	 3


\begin{flushleft}
Environmental Systems Analysis	
\end{flushleft}


3	 0	 0	 3


\begin{flushleft}
Environmental Chemistry and Microbiology	 1	 0	 4	 3
\end{flushleft}


\begin{flushleft}
Total Credits				 39
\end{flushleft}





\begin{flushleft}
Program Electives
\end{flushleft}


\begin{flushleft}
CVL727	
\end{flushleft}


\begin{flushleft}
CVL728	
\end{flushleft}


\begin{flushleft}
CVL729	
\end{flushleft}


	





\begin{flushleft}
Environmental risk assessment
\end{flushleft}


\begin{flushleft}
Environmental Quality Modeling	
\end{flushleft}


\begin{flushleft}
Environmental Statistics and 	
\end{flushleft}


\begin{flushleft}
Experimental Design
\end{flushleft}





\begin{flushleft}
CVL725
\end{flushleft}





\begin{flushleft}
Environmental Chemistry
\end{flushleft}


\begin{flushleft}
and Microbiology
\end{flushleft}





(1-0-4) 3


\begin{flushleft}
CVL721
\end{flushleft}


\begin{flushleft}
II
\end{flushleft}





\begin{flushleft}
Solid Waste
\end{flushleft}


\begin{flushleft}
Engineering
\end{flushleft}





(3-0-0) 3


\begin{flushleft}
Summer
\end{flushleft}





3	 0	 0	 3


3	 0	 0	 3


2	 0	 2	 3





\begin{flushleft}
Courses
\end{flushleft}


\begin{flushleft}
(Number, Abbreviated Title, L-T-P, credits)
\end{flushleft}





\begin{flushleft}
Sem.
\end{flushleft}





\begin{flushleft}
I
\end{flushleft}





	





\begin{flushleft}
CVL820	 Environmental Impact Assessment
\end{flushleft}


	


3	


\begin{flushleft}
CVL821	 Industrial Waste Management and Audit 	 3	
\end{flushleft}


\begin{flushleft}
CVL822	 Emerging Technologies for Environmental	 3	
\end{flushleft}


\begin{flushleft}
	Management
\end{flushleft}


\begin{flushleft}
CVL823	 Thermal Techniques for Waste Management	 3	
\end{flushleft}


\begin{flushleft}
CVL824	 Life Cycle Analysis and Design	
\end{flushleft}


3	


	


\begin{flushleft}
for Environment
\end{flushleft}


\begin{flushleft}
CVL825	 Fundamental of Aerosol: Health	
\end{flushleft}


3	


	


\begin{flushleft}
and Climate Change
\end{flushleft}


\begin{flushleft}
CVL826	 Quantitative Microbial Risk Assessment	
\end{flushleft}


1	


\begin{flushleft}
CVL827	 Environmental Implications of Engineered	
\end{flushleft}


2	


\begin{flushleft}
	Nanomaterials
\end{flushleft}


\begin{flushleft}
CVL828	 Water Distribution and Sewerage	
\end{flushleft}


3	


	


\begin{flushleft}
Network Design
\end{flushleft}


\begin{flushleft}
CVP820	 Advanced Air Pollution Laboratory	
\end{flushleft}


1	


\begin{flushleft}
CVP821	 Advanced Water and Wastewater Laboratory	1	
\end{flushleft}


\begin{flushleft}
CVS720	 Independent Study	
\end{flushleft}


0	





\begin{flushleft}
CVL722
\end{flushleft}





\begin{flushleft}
Water Engineering
\end{flushleft}





\begin{flushleft}
CVL720
\end{flushleft}





(3-0-0) 3





\begin{flushleft}
Air pollution
\end{flushleft}


\begin{flushleft}
and control
\end{flushleft}





\begin{flushleft}
CVL724
\end{flushleft}





\begin{flushleft}
CVL723
\end{flushleft}





\begin{flushleft}
Environmental
\end{flushleft}


\begin{flushleft}
Systems Analysis
\end{flushleft}





(3-0-0) 3





\begin{flushleft}
PE-1
\end{flushleft}


\begin{flushleft}
(3-0-0) 3 or
\end{flushleft}


\begin{flushleft}
(2-0-2) 3 or
\end{flushleft}


(1-0-4)





(3-0-0) 3





\begin{flushleft}
PE-2
\end{flushleft}


\begin{flushleft}
(3-0-0) 3 or
\end{flushleft}


\begin{flushleft}
(2-0-2) 3 or
\end{flushleft}


(1-0-4)





\begin{flushleft}
Wastewater
\end{flushleft}


\begin{flushleft}
Engineering
\end{flushleft}


(3-0-0) 3





\begin{flushleft}
IV
\end{flushleft}





\begin{flushleft}
OE-1
\end{flushleft}


(3-0-0) 3





0	 0	 3


0	 0	 3


0	 0	 3


0	 0	 3


0	 0	 3


0	 0	 3


0	 0	 1


0	 0	 2


0	 0	 3


0	 4	 3


0	 4	 3


3	 0	 3





\begin{flushleft}
Contact h/week
\end{flushleft}


\begin{flushleft}
L
\end{flushleft}





\begin{flushleft}
T
\end{flushleft}





\begin{flushleft}
P
\end{flushleft}





\begin{flushleft}
Total
\end{flushleft}





4





(8,


10)





0





(4,8)





(14,


16)





12





5





(13,


15)





0





(0,4)





(15,


17)





15





\begin{flushleft}
CVD726 Minor project (0-0-6) 3
\end{flushleft}





3





\begin{flushleft}
CVD800
\end{flushleft}


\begin{flushleft}
III
\end{flushleft}





\begin{flushleft}
Lecture
\end{flushleft}


\begin{flushleft}
courses
\end{flushleft}





\begin{flushleft}
Program Core
\end{flushleft}





\begin{flushleft}
Credits
\end{flushleft}





\begin{flushleft}
Category
\end{flushleft}





\begin{flushleft}
PE-3
\end{flushleft}


\begin{flushleft}
(3-0-0) 3 or
\end{flushleft}


\begin{flushleft}
(2-0-2) 3 or
\end{flushleft}


(1-0-4)





\begin{flushleft}
Major Thesis Part-I
\end{flushleft}





(0-0-12) 6


\begin{flushleft}
CVD801
\end{flushleft}





\begin{flushleft}
Major Thesis Part-II
\end{flushleft}





(0-0-24) 12





\begin{flushleft}
OE-2
\end{flushleft}


(3-0-0) 3





2





(4,6)





0





(12,


16)





(18,


20)





12





0





0





0





24





24





12





\begin{flushleft}
Total = 54
\end{flushleft}


113





\begin{flushleft}
\newpage
Programme Code: CEW
\end{flushleft}





\begin{flushleft}
Master of Technology in Water Resources Engineering
\end{flushleft}


\begin{flushleft}
Department of Civil Engineering
\end{flushleft}


\begin{flushleft}
The overall credits structure
\end{flushleft}


\begin{flushleft}
PE
\end{flushleft}





\begin{flushleft}
OC
\end{flushleft}





\begin{flushleft}
Total
\end{flushleft}





39





15





0





54





\begin{flushleft}
Program Core
\end{flushleft}





\begin{flushleft}
Program Electives
\end{flushleft}





\begin{flushleft}
CVD831	
\end{flushleft}


\begin{flushleft}
CVD832	
\end{flushleft}


\begin{flushleft}
CVL730	
\end{flushleft}


\begin{flushleft}
CVL731	
\end{flushleft}


\begin{flushleft}
CVL732	
\end{flushleft}


\begin{flushleft}
CVL733	
\end{flushleft}


\begin{flushleft}
CVL734	
\end{flushleft}


\begin{flushleft}
CVL735	
\end{flushleft}


\begin{flushleft}
CVP730	
\end{flushleft}


\begin{flushleft}
CVP731	
\end{flushleft}





\begin{flushleft}
Major Project Part-I	
\end{flushleft}


0	 0	 12	6


\begin{flushleft}
Major Project Part-II	
\end{flushleft}


0	 0	 24	12


\begin{flushleft}
Hydrologic Processes and Modeling	
\end{flushleft}


3	 0	 0	 3


\begin{flushleft}
Optimization Techniques in Water Resources	3	 0	0	 3
\end{flushleft}


\begin{flushleft}
Groundwater Hydrology	
\end{flushleft}


3	 0	 0	 3


\begin{flushleft}
Stochastic Hydrology	
\end{flushleft}


2	 0	 2	 3


\begin{flushleft}
Advanced Hydraulics	
\end{flushleft}


3	 0	 0	 3


\begin{flushleft}
Finite Element in Water Resources	
\end{flushleft}


3	 0	 0	 3


\begin{flushleft}
Simulation Laboratory-I	
\end{flushleft}


0	 0	 3	 1.5


\begin{flushleft}
Simulation Laboratory-II	
\end{flushleft}


0	 0	 3	 1.5





	





\begin{flushleft}
Total Credits				
\end{flushleft}





\begin{flushleft}
CVL736	
\end{flushleft}


\begin{flushleft}
CVL737	
\end{flushleft}


\begin{flushleft}
CVL738	
\end{flushleft}


\begin{flushleft}
CVL830	
\end{flushleft}


\begin{flushleft}
CVL831	
\end{flushleft}


\begin{flushleft}
CVL832	
\end{flushleft}


\begin{flushleft}
CVL833	
\end{flushleft}


\begin{flushleft}
CVL834	
\end{flushleft}


\begin{flushleft}
CVL835	
\end{flushleft}


\begin{flushleft}
CVL836	
\end{flushleft}


\begin{flushleft}
CVL837	
\end{flushleft}


\begin{flushleft}
CVL838	
\end{flushleft}


\begin{flushleft}
CVL839	
\end{flushleft}


\begin{flushleft}
CVS730	
\end{flushleft}


\begin{flushleft}
CVS830	
\end{flushleft}





\begin{flushleft}
Courses
\end{flushleft}


\begin{flushleft}
(Number, Abbreviated Title, L-T-P, credits)
\end{flushleft}





\begin{flushleft}
Sem.
\end{flushleft}





\begin{flushleft}
CVL730
\end{flushleft}


\begin{flushleft}
I
\end{flushleft}





\begin{flushleft}
CVL731
\end{flushleft}





\begin{flushleft}
CVL732
\end{flushleft}





(3-0-0) 3





(3-0-0) 3





\begin{flushleft}
Stochastic
\end{flushleft}


\begin{flushleft}
Hyd.
\end{flushleft}





\begin{flushleft}
CVL733
\end{flushleft}





\begin{flushleft}
CVL734
\end{flushleft}





\begin{flushleft}
CVP730
\end{flushleft}





\begin{flushleft}
CVP731
\end{flushleft}





(3-0-0) 3





(3-0-0) 3





(0-0-3) 1.5





(0-0-3) 1.5





\begin{flushleft}
Hyd.
\end{flushleft}


\begin{flushleft}
Process
\end{flushleft}





(3-0-0) 3





\begin{flushleft}
II
\end{flushleft}





\begin{flushleft}
Summer
\end{flushleft}





\begin{flushleft}
Adv.
\end{flushleft}


\begin{flushleft}
Hydraulics
\end{flushleft}





\begin{flushleft}
Opt. Tech.
\end{flushleft}





\begin{flushleft}
Finite
\end{flushleft}


\begin{flushleft}
Element
\end{flushleft}





\begin{flushleft}
IV
\end{flushleft}





\begin{flushleft}
GW Hyd.
\end{flushleft}





\begin{flushleft}
Sim. Lab-I
\end{flushleft}





\begin{flushleft}
CVL735
\end{flushleft}


(2-0-2) 3





\begin{flushleft}
Sim. Lab-II
\end{flushleft}





\begin{flushleft}
PE-1
\end{flushleft}


(3-0-0) 3


\begin{flushleft}
or
\end{flushleft}


(2-0-2) 3


\begin{flushleft}
PE-2
\end{flushleft}


(3-0-0) 3


\begin{flushleft}
or
\end{flushleft}


(2-0-2) 3





\begin{flushleft}
PE-3
\end{flushleft}


(3-0-0) 3


\begin{flushleft}
or
\end{flushleft}


(2-0-2) 3





\begin{flushleft}
L
\end{flushleft}





\begin{flushleft}
T
\end{flushleft}





\begin{flushleft}
P
\end{flushleft}





\begin{flushleft}
Total
\end{flushleft}





5





14/13





0





2/4





16/17





15





4





12-10





0





6-10





18-20





15





\begin{flushleft}
Contact h/week
\end{flushleft}





\begin{flushleft}
Major Project Part I (CEW)
\end{flushleft}





\begin{flushleft}
CVD831
\end{flushleft}


\begin{flushleft}
III
\end{flushleft}





39





\begin{flushleft}
Soft Computing Techniques in Water Resources	2	 0	2	 3
\end{flushleft}


\begin{flushleft}
Environmental Dynamics and Management 	 3	 0	 0	 3
\end{flushleft}


\begin{flushleft}
Economic Aspects of Water Resources Development	 3	 0	0	 3
\end{flushleft}


\begin{flushleft}
Groundwater Flow and Pollution Modeling	 3	 0	 0	 3
\end{flushleft}


\begin{flushleft}
Surface Water Quality Modeling and Control 	 3	 0	 0	 3
\end{flushleft}


\begin{flushleft}
Hydroelectric Engineering	
\end{flushleft}


3	 0	 0	 3


\begin{flushleft}
Water Resources Systems	
\end{flushleft}


3	 0	 0	 3


\begin{flushleft}
Urban Water Infrastructure	
\end{flushleft}


3	 0	 0	 3


\begin{flushleft}
Eco-hydraulics and Hydrology 	
\end{flushleft}


3	 0	 0	 3


\begin{flushleft}
Advanced Hydrologic Land Surface Processes	 3	 0	0	 3
\end{flushleft}


\begin{flushleft}
Mechanics of Sediment Transport	
\end{flushleft}


2	 0	 2	 3


\begin{flushleft}
Geographic Information Systems	
\end{flushleft}


2	 0	 2	 3


\begin{flushleft}
Hydrologic Applications of Remote Sensing	 2	 0	 2	 3
\end{flushleft}


\begin{flushleft}
Minor Project	
\end{flushleft}


0	 0	 6	 3


\begin{flushleft}
Independent Study	
\end{flushleft}


0	 3	 0	 3





\begin{flushleft}
Credits
\end{flushleft}





\begin{flushleft}
PC
\end{flushleft}





\begin{flushleft}
Credits
\end{flushleft}





\begin{flushleft}
Lecture
\end{flushleft}


\begin{flushleft}
courses
\end{flushleft}





\begin{flushleft}
Category
\end{flushleft}





\begin{flushleft}
Major Project Part-I
\end{flushleft}





(0-0-12) 6





0





\begin{flushleft}
PE-4
\end{flushleft}


(3-0-0) 3


\begin{flushleft}
or
\end{flushleft}


(2-0-2) 3





\begin{flushleft}
PE-5
\end{flushleft}


(3-0-0) 3


\begin{flushleft}
or
\end{flushleft}


(2-0-2) 3





\begin{flushleft}
CVD832
\end{flushleft}





\begin{flushleft}
Major Project Part-II
\end{flushleft}





(0-0-24) 12





2





6-4





0





12-16





18-20





12





0





0





0





24





24





12





\begin{flushleft}
Total = 54
\end{flushleft}


114





\begin{flushleft}
\newpage
Programme Code: MCS
\end{flushleft}





\begin{flushleft}
Master of Technology in Computer Science and Engineering
\end{flushleft}


\begin{flushleft}
Department of Computer Science and Engineering
\end{flushleft}


\begin{flushleft}
The overall credits structure
\end{flushleft}


\begin{flushleft}
Category
\end{flushleft}





\begin{flushleft}
PC
\end{flushleft}





\begin{flushleft}
PE
\end{flushleft}





\begin{flushleft}
OC
\end{flushleft}





\begin{flushleft}
Total
\end{flushleft}





\begin{flushleft}
Credits
\end{flushleft}





21





27-33





\begin{flushleft}
4 cr in leiu of PE for MTech by course work
\end{flushleft}





48-54





\begin{flushleft}
Program Core
\end{flushleft}





\begin{flushleft}
COL781	 Computer Graphics	
\end{flushleft}


\begin{flushleft}
COL783	Digital Image Analysis	
\end{flushleft}


\begin{flushleft}
COL829	 Advanced Computer Graphics	
\end{flushleft}


\begin{flushleft}
COV877	 Special Module on Visual Computing	
\end{flushleft}


\begin{flushleft}
SIL801	 Special Topics in Multimedia System	
\end{flushleft}





\begin{flushleft}
COD891	Minor Project	
\end{flushleft}


0	0	6	3


\begin{flushleft}
COD892	 M.Tech. Project Part-I 	
\end{flushleft}


0	 0	 14	7


\begin{flushleft}
COL702	Advanced Data Structures	
\end{flushleft}


3	0	2	4


\begin{flushleft}
COL765	 Logic \& Functional Programming	
\end{flushleft}


3	 0	 2	 4


\begin{flushleft}
COP701	Software Systems Laboratory	
\end{flushleft}


0	0	6	3


	


	


\begin{flushleft}
Total Credits				21
\end{flushleft}





\begin{flushleft}
3. Software Systems (SS)
\end{flushleft}


\begin{flushleft}
COL724	Advanced Computer Networks	
\end{flushleft}


3	0	2	4


\begin{flushleft}
COL728	Compiler Design	
\end{flushleft}


3	0	3	4.5


\begin{flushleft}
COL729	Compiler Optimization	
\end{flushleft}


3	0	3	4.5


\begin{flushleft}
COL730	Parallel Programming	
\end{flushleft}


3	0	2	4


\begin{flushleft}
COL732	 Virtualization and Cloud Computing	
\end{flushleft}


3	 0	 2	 4


\begin{flushleft}
COL733	Cloud Computing Technology Fundamentals	3	0	2	4
\end{flushleft}


\begin{flushleft}
COL740	Software Engineering	
\end{flushleft}


3	0	2	4


\begin{flushleft}
COL768	Wireless Networks	
\end{flushleft}


3	0	2	4


\begin{flushleft}
COL819	 Advanced Distributed Systems	
\end{flushleft}


3	0	2	4


\begin{flushleft}
COL851	 Special Topics in Operating Systems	
\end{flushleft}


3	 0	 0	 3


\begin{flushleft}
COL852	Special Topics in Compliers	
\end{flushleft}


3	0	0	3


\begin{flushleft}
COL860	 Special Topics in Parallel Computation	
\end{flushleft}


3	 0	 0	 3


\begin{flushleft}
COL862	 Special Topics in Software Systems	
\end{flushleft}


3	 0	 0	 3


\begin{flushleft}
COL867	 Special Topics in High Speed Networks	
\end{flushleft}


3	 0	 0	 3


\begin{flushleft}
COL871	 Special Topics in programming languages \&	 3	 0	 0	 3
\end{flushleft}


\begin{flushleft}
	Compliers
\end{flushleft}


\begin{flushleft}
COV880	 Special Module in Parallel Computation	
\end{flushleft}


1	 0	 0	 1


\begin{flushleft}
COV882	 Special Module in Software Systems	
\end{flushleft}


1	 0	 0	 1


\begin{flushleft}
COV887	 Special Module in High Speed Networks	
\end{flushleft}


1	 0	 0	 1


\begin{flushleft}
SIL765	 Networks \& System Security	
\end{flushleft}


3	 0	 2	 4


\begin{flushleft}
SIL769	 Internet Traffic - Measurement, Modeling \& 	 3	 0	 2	 4
\end{flushleft}


\begin{flushleft}
	Analysis
\end{flushleft}





\begin{flushleft}
Bridge Courses - Min. 6 credits, may be waived in exceptional 		
\end{flushleft}


\begin{flushleft}
cases on recommendation by DRC
\end{flushleft}


3	 0	 2	 4


3	 0	 2	 4


3	 0	 2	 4	


3	0	2	4





\begin{flushleft}
Program Electives (PE)
\end{flushleft}


0	 0	 28	14


0	3	0	3





\begin{flushleft}
Specialization Streams - At least 6 credits from PE; Project work
\end{flushleft}


\begin{flushleft}
relevant to specialization
\end{flushleft}


\begin{flushleft}
1. Architecture \& Embedded Systems (AES)
\end{flushleft}


\begin{flushleft}
COL718	 Architecture of High Performance Computers	3	0	2	4
\end{flushleft}


\begin{flushleft}
COL719	 Synthesis of Digital Systems	
\end{flushleft}


3	 0	 2	 4


\begin{flushleft}
COL788	Embedded Computing	
\end{flushleft}


3	0	0	3


\begin{flushleft}
COL812	 System Level Design and Modelling	
\end{flushleft}


3	 0	 0	 3


\begin{flushleft}
COL818	 Principles of Multiprocessor Systems	
\end{flushleft}


3	 0	 2	 4


\begin{flushleft}
COL821	 Reconfigurable Computing	
\end{flushleft}


3	 0	 0	 3


\begin{flushleft}
COL861	 Special Topics in Hardware Systems	
\end{flushleft}


3	 0	 0	 3


\begin{flushleft}
COP745	 Digital System Design Laboratory	
\end{flushleft}


0	 0	 6	 3


\begin{flushleft}
COV881	 Special Module in Hardware Systems	
\end{flushleft}


1	 0	 0	 1


\begin{flushleft}
2. Graphic \& Vision (GV)
\end{flushleft}


\begin{flushleft}
COL726	Numerical Algorithms	
\end{flushleft}


\begin{flushleft}
COL780	Computer Vision	
\end{flushleft}





3	0	2	4


3	0	2	4





\begin{flushleft}
Courses
\end{flushleft}


\begin{flushleft}
(Number, Abbreviated Title, L-T-P, credits)
\end{flushleft}





\begin{flushleft}
Sem.
\end{flushleft}





\begin{flushleft}
COL702
\end{flushleft}


\begin{flushleft}
I
\end{flushleft}





\begin{flushleft}
4. Theoretical Computer Science (TH)
\end{flushleft}


\begin{flushleft}
COL703	 Logic for CS (LCS)	
\end{flushleft}


\begin{flushleft}
COL726	Numerical Algorithms	
\end{flushleft}


\begin{flushleft}
COL730	Parallel Programming	
\end{flushleft}


\begin{flushleft}
COL750	 Foundations of Automatic Verification	
\end{flushleft}


\begin{flushleft}
COL751	 Algorithmic Graph Theory	
\end{flushleft}





\begin{flushleft}
Advanced Data Structures
\end{flushleft}





(3-0-2) 4





\begin{flushleft}
Bridge
\end{flushleft}


\begin{flushleft}
Course-1
\end{flushleft}


(3-4)





\begin{flushleft}
COL765
\end{flushleft}





\begin{flushleft}
Software Lab
\end{flushleft}





(0-0-6) 3





(3-0-2) 4





\begin{flushleft}
PE-1
\end{flushleft}


(3-4)





\begin{flushleft}
Bridge
\end{flushleft}


\begin{flushleft}
Course-2
\end{flushleft}


(3-4)





\begin{flushleft}
COD891
\end{flushleft}





\begin{flushleft}
III
\end{flushleft}





\begin{flushleft}
PE-3
\end{flushleft}


(3-4)





\begin{flushleft}
PE-4
\end{flushleft}


(3-4)





\begin{flushleft}
COD892
\end{flushleft}





\begin{flushleft}
L
\end{flushleft}





\begin{flushleft}
T
\end{flushleft}





\begin{flushleft}
P
\end{flushleft}





\begin{flushleft}
Total
\end{flushleft}





2-3





6-9





0





10





16-19





11-15





2-3





6-9





0





6-8





12-17





9-15





2-6





2





6





18





24





13-15





3





9





0





6





15





9-12





0





0





0





28





28





14





\begin{flushleft}
Contact h/week
\end{flushleft}





\begin{flushleft}
COP701
\end{flushleft}





\begin{flushleft}
Logic and Functional
\end{flushleft}


\begin{flushleft}
Programming
\end{flushleft}





\begin{flushleft}
II
\end{flushleft}





3	 0	 2	 4


3	0	2	4


3	0	2	4


3	 0	 2	 4


3	 0	 0	 3





\begin{flushleft}
Credits
\end{flushleft}





\begin{flushleft}
COD893	 Major Project Part-II	
\end{flushleft}


\begin{flushleft}
COS799	Independent Study	
\end{flushleft}





\begin{flushleft}
Lecture
\end{flushleft}


\begin{flushleft}
courses
\end{flushleft}





\begin{flushleft}
COL632	 Introduction to Data Base Systems	
\end{flushleft}


\begin{flushleft}
COL633	 Resources Management in Computer Systems
\end{flushleft}


\begin{flushleft}
COL671	 Artificial Intelligence	
\end{flushleft}


\begin{flushleft}
COL672	Computer Networks	
\end{flushleft}





3	 0	 3	 4.5


3	0	3	4.5


3	 0	 2	 4


1	 0	 0	 1


3	 0	 0	 3





\begin{flushleft}
PE-2
\end{flushleft}


(3-4)





\begin{flushleft}
Minor Project
\end{flushleft}





(3-0-0) 3


\begin{flushleft}
M.Tech. Project Part-I
\end{flushleft}





(0-0-14) 7


\begin{flushleft}
M.Tech. Option I Course Work
\end{flushleft}





\begin{flushleft}
IV
\end{flushleft}





\begin{flushleft}
PE-5
\end{flushleft}


(3-4)





\begin{flushleft}
PE-6
\end{flushleft}


(3-4)





\begin{flushleft}
PE-7 / OC
\end{flushleft}


(3-4)





\begin{flushleft}
(OR)
\end{flushleft}


\begin{flushleft}
IV
\end{flushleft}





\begin{flushleft}
Summer
\end{flushleft}





\begin{flushleft}
M.Tech. Option II Thesis*
\end{flushleft}





\begin{flushleft}
COD893
\end{flushleft}


\begin{flushleft}
M.Tech. Project Part-II (MTP-II)
\end{flushleft}





(0-0-28)





\begin{flushleft}
MTP-II (Contd.)
\end{flushleft}





\begin{flushleft}
*Thesis option has a requirement of Min. CGPA 7.5 at the end 3rd sem and B Grade in COD892.
\end{flushleft}


\begin{flushleft}
In exceptional cases DRC may waive the CGPA requirement
\end{flushleft}





\begin{flushleft}
Total = 48
\end{flushleft}


115





\begin{flushleft}
\newpage
Programme Code: MCS
\end{flushleft}





\begin{flushleft}
Master of Technology in Computer Science and Engineering
\end{flushleft}


\begin{flushleft}
Department of Computer Science and Engineering
\end{flushleft}


\begin{flushleft}
COL752	 Geometric Algorithms	
\end{flushleft}


3	 0	 0	 3


\begin{flushleft}
COL753	Complexity Theory	
\end{flushleft}


3	0	0	3


\begin{flushleft}
COL754	Approximation Algorithms	
\end{flushleft}


3	0	0	3


\begin{flushleft}
COL756	Mathematical Programming	
\end{flushleft}


3	0	0	3


\begin{flushleft}
COL757	Model Centric Algorithm Design	
\end{flushleft}


3	0	2	4


\begin{flushleft}
COL758	Advanced Algorithms	
\end{flushleft}


3	0	2	4


\begin{flushleft}
COL759	 Cryptography \& Computer Security	
\end{flushleft}


3	 0	 0	 3


\begin{flushleft}
COL830	Distributed Computing	
\end{flushleft}


3	0	0	3


\begin{flushleft}
COL831	 Semantics of Programming Languages	
\end{flushleft}


3	 0	 0	 3


\begin{flushleft}
COL832	Proofs and Types	
\end{flushleft}


3	0	0	3


\begin{flushleft}
COL860	 Special Topics in Parallel Computation	
\end{flushleft}


3	 0	 0	 3


\begin{flushleft}
COL863	 Special Topics in Theoretical Computer Science	 3	0	0	3
\end{flushleft}


\begin{flushleft}
COL866	Special Topics in Algorithms	
\end{flushleft}


3	0	0	3


\begin{flushleft}
COL872	Special Topics in Cryptography	
\end{flushleft}


3	0	0	3


\begin{flushleft}
COV879	 Special Module in Financial Algorithms	
\end{flushleft}


2	 0	 0	 2


\begin{flushleft}
COV883	 Special Module in Theoretical Computer Science	1	0	0	1
\end{flushleft}


\begin{flushleft}
COV886	Special Module in Algorithms	
\end{flushleft}


1	0	0	1


\begin{flushleft}
5. Data Analysis \& AI (DAAI)
\end{flushleft}


\begin{flushleft}
COL726	Numerical Algorithms	
\end{flushleft}


\begin{flushleft}
COL760	Advanced Data Management	
\end{flushleft}


\begin{flushleft}
COL762	Database Implementation	
\end{flushleft}


\begin{flushleft}
COL770	 Advanced Artificial Intelligence	
\end{flushleft}


\begin{flushleft}
COL772	Natural Language Processing	
\end{flushleft}


\begin{flushleft}
COL774	Machine Learning	
\end{flushleft}


\begin{flushleft}
COL776	 Learning Probabilistic Graphical Models	
\end{flushleft}


\begin{flushleft}
COL786	 Advanced Functional Brain Imaging	
\end{flushleft}


\begin{flushleft}
COL864	 Special Topics in Artificial Intelligence	
\end{flushleft}


\begin{flushleft}
COL868	 Special Topics in Database Systems	
\end{flushleft}


\begin{flushleft}
COL869	Special Topics in Concurrency	
\end{flushleft}





3	0	2	4


3	0	2	4


3	0	2	4


3	 0	 2	 4


3	0	2	4


3	0	2	4


3	 0	 2	 4


3	 0	 2	 4


3	 0	 0	 3


3	 0	 0	 3


3	0	0	3





\begin{flushleft}
COL870	
\end{flushleft}


\begin{flushleft}
COV878	
\end{flushleft}


\begin{flushleft}
COV884	
\end{flushleft}


\begin{flushleft}
COV888	
\end{flushleft}


\begin{flushleft}
COV889	
\end{flushleft}





\begin{flushleft}
Special Topics in Machine Learning	
\end{flushleft}


\begin{flushleft}
Special Module in Machine Learning	
\end{flushleft}


\begin{flushleft}
Special Module in Artificial Intelligence	
\end{flushleft}


\begin{flushleft}
Special Module in Database Systems	
\end{flushleft}


\begin{flushleft}
Special Module in Concurrency	
\end{flushleft}





3	


1	


1	


1	


1	





0	


0	


0	


0	


0	





0	


0	


0	


0	


0	





3


1


1


1


1





\begin{flushleft}
6. Application \& IT (ITA)
\end{flushleft}


\begin{flushleft}
COL722	 Introduction to Compressed Sensing	
\end{flushleft}


3	 0	 0	 3


\begin{flushleft}
COL757	Model Centric Algorithm Design	
\end{flushleft}


3	0	2	4


\begin{flushleft}
COL760	Advanced Data Management	
\end{flushleft}


3	0	2	4


\begin{flushleft}
COL762	Database Implementation	
\end{flushleft}


3	0	2	4


\begin{flushleft}
COL770	 Advanced Artificial Intelligence	
\end{flushleft}


3	 0	 2	 4


\begin{flushleft}
COL786	 Advanced Functional Brain Imaging	
\end{flushleft}


3	 0	 2	 4


\begin{flushleft}
COL865	Special Topics in Computer Applications	
\end{flushleft}


3	0	0	3


\begin{flushleft}
COL869	Special Topics in Concurrency	
\end{flushleft}


3	0	0	3


\begin{flushleft}
COV885	 Special Module in Computer Applications	
\end{flushleft}


1	 0	 0	 1


\begin{flushleft}
COV888	 Special Module in Database Systems	
\end{flushleft}


1	 0	 0	 1


\begin{flushleft}
COV889	 Special Module in Concurrency	
\end{flushleft}


1	 0	 0	 1


\begin{flushleft}
SIL769	 Internet Traffic - Measurement, Modeling \& 	 3	 0	 2	 4
\end{flushleft}


\begin{flushleft}
	Analysis
\end{flushleft}


\begin{flushleft}
SIL801	 Special Topics in Multimedia System	
\end{flushleft}


3	 0	 0	 3


\begin{flushleft}
SIL802	 Special Topics in Web Based Computing	
\end{flushleft}


3	 0	 0	 3


\begin{flushleft}
SIV813	 Applications of Computer in Medicines	
\end{flushleft}


1	 0	 0	 1


\begin{flushleft}
SIV861	 Information and Communication	
\end{flushleft}


1	0	0	1


	


\begin{flushleft}
Technologies for Development
\end{flushleft}


\begin{flushleft}
SIV864	 Special Module on Media Processing	
\end{flushleft}


1	 0	 0	 1


\begin{flushleft}
	Communication
\end{flushleft}


\begin{flushleft}
SIV871	 Special Module in Computational Neuroscience	
\end{flushleft}


1	0	0	1	


\begin{flushleft}
SIV895	 Special Module on Intelligent Information	
\end{flushleft}


1	 0	 0	 1


\begin{flushleft}
	Processing
\end{flushleft}





116





\begin{flushleft}
\newpage
Programme Code: EEA
\end{flushleft}





\begin{flushleft}
Master of Technology in Control and Automation
\end{flushleft}


\begin{flushleft}
Department of Electrical Engineering
\end{flushleft}


\begin{flushleft}
The overall credits structure
\end{flushleft}


\begin{flushleft}
Category
\end{flushleft}





\begin{flushleft}
PC
\end{flushleft}





\begin{flushleft}
PE
\end{flushleft}





\begin{flushleft}
OC
\end{flushleft}





\begin{flushleft}
Total
\end{flushleft}





\begin{flushleft}
Credits
\end{flushleft}





24





18





6





48





\begin{flushleft}
Program Core
\end{flushleft}


\begin{flushleft}
ELD801	
\end{flushleft}


\begin{flushleft}
ELL700	
\end{flushleft}


\begin{flushleft}
ELL701	
\end{flushleft}


\begin{flushleft}
ELL702	
\end{flushleft}


\begin{flushleft}
ELL703	
\end{flushleft}


\begin{flushleft}
ELL705	
\end{flushleft}


\begin{flushleft}
ELP800	
\end{flushleft}


\begin{flushleft}
ELP801	
\end{flushleft}


	





\begin{flushleft}
Major Project Part-I	
\end{flushleft}


0	 0	 12	6


\begin{flushleft}
Linear Systems Theory	
\end{flushleft}


3	 0	 0	 3


\begin{flushleft}
Mathematical Methods in Control	
\end{flushleft}


3	 0	 0	 3


\begin{flushleft}
Nonlinear Systems	
\end{flushleft}


3	 0	 0	 3


\begin{flushleft}
Optimal Control Theory	
\end{flushleft}


3	 0	 0	 3


\begin{flushleft}
Stochastic Filtering and Identification	
\end{flushleft}


3	 0	 0	 3


\begin{flushleft}
Control Systems Laboratory	
\end{flushleft}


0	 0	 2	 1


\begin{flushleft}
Advanced Control Laboratory	
\end{flushleft}


0	 0	 4	 2


\begin{flushleft}
Total Credits				 24
\end{flushleft}





\begin{flushleft}
Program Electives
\end{flushleft}





\begin{flushleft}
II
\end{flushleft}





0	 6	 3


0	 24	12


0	 0	 3


0	 0	 3


0	 0	 3


0	 0	 3


0	 0	 3


0	 0	 3


0	 0	 3


0	 0	 3


0	 0	 3


0	 0	 3


0	 0	 3


0	0	 3


0	 0	 3


0	 0	 3





\begin{flushleft}
ELL700
\end{flushleft}





\begin{flushleft}
ELL701
\end{flushleft}





(3-0-0)





(3-0-0)





\begin{flushleft}
ELL703
\end{flushleft}





\begin{flushleft}
ELL705
\end{flushleft}





\begin{flushleft}
Linear Systems
\end{flushleft}


\begin{flushleft}
Theory
\end{flushleft}





\begin{flushleft}
Optimal Control
\end{flushleft}


\begin{flushleft}
Theory
\end{flushleft}





(3-0-0)





\begin{flushleft}
Mathematical
\end{flushleft}


\begin{flushleft}
Methods in Control
\end{flushleft}





\begin{flushleft}
Stochastic Filtering
\end{flushleft}


\begin{flushleft}
and Identification
\end{flushleft}





(3-0-0)





0	 2	


0	 0	


0	 0	


0	 0	


0	 2	


0	 0	


0	 0	


0	 0	


0	 0	





4


3


3


3


4


3


3


3


3





3	


3	


3	


3	


3	


3	





0	 0	 3


0	 0	 3


0	 0	 3


0	 0	 3


0	0	 3


0	 0	 3





3	


3	


2	


3	





0	 0	


0	 0	


0	 2	


0	 0	





3


3


3


3





3	


3	


3	


3	


1	





0	 0	


0	 0	


0	 0	


0	 0	


0	 0	





3


3


3


3


1





\begin{flushleft}
L
\end{flushleft}





\begin{flushleft}
T
\end{flushleft}





\begin{flushleft}
P
\end{flushleft}





\begin{flushleft}
Total
\end{flushleft}





4





12





0





2





14





13





\begin{flushleft}
PE
\end{flushleft}


(3-0-0)





3





9





0





4





13





11





\begin{flushleft}
PE
\end{flushleft}


(3-0-0)





\begin{flushleft}
OE
\end{flushleft}


(3-0-0)





2





6





0





12





18





12





\begin{flushleft}
PE
\end{flushleft}


(3-0-0)





\begin{flushleft}
OE
\end{flushleft}


(3-0-0)





4





12





0





0





12





12





0





0





0





24





24





12





2





6





0





12





18





12





\begin{flushleft}
Courses
\end{flushleft}


\begin{flushleft}
(Number, Abbreviated Title, L-T-P, credits)
\end{flushleft}





\begin{flushleft}
Sem.
\end{flushleft}





\begin{flushleft}
I
\end{flushleft}





0	


0	


3	


3	


3	


3	


3	


3	


3	


3	


3	


3	


3	


3	


3	


3	





2	


3	


3	


3	


3	


3	


3	


3	


3	





\begin{flushleft}
Credits
\end{flushleft}





\begin{flushleft}
Minor Project (EEA)	
\end{flushleft}


\begin{flushleft}
Major Project Part-II	
\end{flushleft}


\begin{flushleft}
Advanced Robotics	
\end{flushleft}


\begin{flushleft}
Numerical Optimization	
\end{flushleft}


\begin{flushleft}
Systems Biology	
\end{flushleft}


\begin{flushleft}
Selected Topics in Systems and Control 	
\end{flushleft}


\begin{flushleft}
Design Aspects in Control	
\end{flushleft}


\begin{flushleft}
Sensors \& Tranducers	
\end{flushleft}


\begin{flushleft}
Basic Information Theory	
\end{flushleft}


\begin{flushleft}
Advanced Digital Signal Processing	
\end{flushleft}


\begin{flushleft}
Introduction to Chaotic Dynamical System	
\end{flushleft}


\begin{flushleft}
Intelligent Motor Controllers 	
\end{flushleft}


\begin{flushleft}
Smart Grid Technology	
\end{flushleft}


\begin{flushleft}
Mechatronics	
\end{flushleft}


\begin{flushleft}
Power System Dynamics	
\end{flushleft}


\begin{flushleft}
Dynamic Modelling And Control of	
\end{flushleft}


\begin{flushleft}
Sustainable Energy Systems
\end{flushleft}





\begin{flushleft}
Automation Manufacturing	
\end{flushleft}


\begin{flushleft}
Introduction to Machine Learning	
\end{flushleft}


\begin{flushleft}
Embedded Systems and Applications	
\end{flushleft}


\begin{flushleft}
Intelligent Systems	
\end{flushleft}


\begin{flushleft}
Neural Systems and Learning Machines	
\end{flushleft}


\begin{flushleft}
Computer Vision	
\end{flushleft}


\begin{flushleft}
Swarm Intelligence	
\end{flushleft}


\begin{flushleft}
Signals and Systems in Biology	
\end{flushleft}


\begin{flushleft}
Numerical Linear Algebra and Optimization	
\end{flushleft}


\begin{flushleft}
in Engineering
\end{flushleft}


\begin{flushleft}
Nonlinear Control 	
\end{flushleft}


\begin{flushleft}
Adaptive and Learning Control	
\end{flushleft}


\begin{flushleft}
Model Reduction in Control	
\end{flushleft}


\begin{flushleft}
Robust Control	
\end{flushleft}


\begin{flushleft}
Networked and Multi-Agent Control Systems	
\end{flushleft}


\begin{flushleft}
Modeling and Control of Distributed	
\end{flushleft}


\begin{flushleft}
Parameter Systems
\end{flushleft}


\begin{flushleft}
Stochastic Control	
\end{flushleft}


\begin{flushleft}
Advanced Topics in Systems and Control 	
\end{flushleft}


\begin{flushleft}
Advanced Robotics	
\end{flushleft}


\begin{flushleft}
Digital Control of Power Electronics and 	
\end{flushleft}


\begin{flushleft}
Drive Systems
\end{flushleft}


\begin{flushleft}
Embedded Intelligence	
\end{flushleft}


\begin{flushleft}
Advanced Machine Learning	
\end{flushleft}


\begin{flushleft}
Computational Neuroscience	
\end{flushleft}


\begin{flushleft}
Cyber-Physical Systems	
\end{flushleft}


\begin{flushleft}
Special Module in Systems and Control	
\end{flushleft}





\begin{flushleft}
Lecture
\end{flushleft}


\begin{flushleft}
courses
\end{flushleft}





\begin{flushleft}
ELD800	
\end{flushleft}


\begin{flushleft}
ELD802	
\end{flushleft}


\begin{flushleft}
ELL704	
\end{flushleft}


\begin{flushleft}
MTL704	
\end{flushleft}


\begin{flushleft}
ELL707	
\end{flushleft}


\begin{flushleft}
ELL708	
\end{flushleft}


\begin{flushleft}
ELL709	
\end{flushleft}


\begin{flushleft}
DSL711	
\end{flushleft}


\begin{flushleft}
ELL714	
\end{flushleft}


\begin{flushleft}
ELL720	
\end{flushleft}


\begin{flushleft}
MTL731	
\end{flushleft}


\begin{flushleft}
ELL762	
\end{flushleft}


\begin{flushleft}
ELL765	
\end{flushleft}


\begin{flushleft}
ELL767	
\end{flushleft}


\begin{flushleft}
ELL775	
\end{flushleft}


\begin{flushleft}
ELL778	
\end{flushleft}


	





\begin{flushleft}
MCL783	
\end{flushleft}


\begin{flushleft}
ELL784	
\end{flushleft}


\begin{flushleft}
ELL787	
\end{flushleft}


\begin{flushleft}
ELL789	
\end{flushleft}


\begin{flushleft}
ELL791	
\end{flushleft}


\begin{flushleft}
ELL793	
\end{flushleft}


\begin{flushleft}
ELL795	
\end{flushleft}


\begin{flushleft}
ELL796	
\end{flushleft}


\begin{flushleft}
ELL800	
\end{flushleft}


	


\begin{flushleft}
ELL801	
\end{flushleft}


\begin{flushleft}
ELL802	
\end{flushleft}


\begin{flushleft}
ELL803	
\end{flushleft}


\begin{flushleft}
ELL804	
\end{flushleft}


\begin{flushleft}
ELL805	
\end{flushleft}


\begin{flushleft}
ELL806	
\end{flushleft}


	


\begin{flushleft}
ELL807	
\end{flushleft}


\begin{flushleft}
ELL808	
\end{flushleft}


\begin{flushleft}
MCL845	
\end{flushleft}


\begin{flushleft}
ELL850	
\end{flushleft}


	


\begin{flushleft}
ELL883	
\end{flushleft}


\begin{flushleft}
ELL888	
\end{flushleft}


\begin{flushleft}
ELL890	
\end{flushleft}


\begin{flushleft}
ELL893	
\end{flushleft}


\begin{flushleft}
ELV700	
\end{flushleft}





\begin{flushleft}
ELL702
\end{flushleft}





\begin{flushleft}
Nonlinear Systems
\end{flushleft}





(3-0-0)





\begin{flushleft}
ELP800
\end{flushleft}


\begin{flushleft}
Control
\end{flushleft}


\begin{flushleft}
Systems
\end{flushleft}


\begin{flushleft}
Lab
\end{flushleft}





\begin{flushleft}
OE
\end{flushleft}


(3-0-0)





\begin{flushleft}
Contact h/week
\end{flushleft}





(0-0-2)


\begin{flushleft}
ELP801
\end{flushleft}





\begin{flushleft}
Advanced
\end{flushleft}


\begin{flushleft}
Control Lab
\end{flushleft}





(0-0-4)





\begin{flushleft}
Summer
\end{flushleft}


\begin{flushleft}
III
\end{flushleft}





\begin{flushleft}
(Project based)
\end{flushleft}


\begin{flushleft}
OR
\end{flushleft}





\begin{flushleft}
III
\end{flushleft}





\begin{flushleft}
(Course based)
\end{flushleft}





\begin{flushleft}
IV
\end{flushleft}





\begin{flushleft}
(Project based)
\end{flushleft}


\begin{flushleft}
OR
\end{flushleft}





\begin{flushleft}
IV
\end{flushleft}





\begin{flushleft}
(Course based)
\end{flushleft}





\begin{flushleft}
ELD801
\end{flushleft}





\begin{flushleft}
Major Project Part-I
\end{flushleft}





(0-0-12)


\begin{flushleft}
PE
\end{flushleft}


(3-0-0)





\begin{flushleft}
PE
\end{flushleft}


(3-0-0)





\begin{flushleft}
ELD802
\end{flushleft}





\begin{flushleft}
Major Project Part-II
\end{flushleft}





(0-0-24)


\begin{flushleft}
ELD801
\end{flushleft}





\begin{flushleft}
Major Project Part-I
\end{flushleft}





(0-0-12)





\begin{flushleft}
PE
\end{flushleft}


(3-0-0)





\begin{flushleft}
PE
\end{flushleft}


(3-0-0)





\begin{flushleft}
Total = 48
\end{flushleft}


117





\begin{flushleft}
\newpage
Master of Technology in Communication Engineering
\end{flushleft}





\begin{flushleft}
Programme Code: EEE
\end{flushleft}





\begin{flushleft}
Department of Electrical Engineering
\end{flushleft}


\begin{flushleft}
The overall credits structure
\end{flushleft}


\begin{flushleft}
Category
\end{flushleft}





\begin{flushleft}
PC
\end{flushleft}





\begin{flushleft}
PE
\end{flushleft}





\begin{flushleft}
OE
\end{flushleft}





\begin{flushleft}
Total
\end{flushleft}





\begin{flushleft}
Credits
\end{flushleft}





24





18





6





48





\begin{flushleft}
Streamed Electives (EEE) in (Communication Systems)
\end{flushleft}


\begin{flushleft}
ELD810	 Minor Project (Communication Engineering)	 0	 0	 6	 3
\end{flushleft}


\begin{flushleft}
ELD812	 Major Project Part-II	
\end{flushleft}


0	 0	 24	 12


\begin{flushleft}
ELL701	 Mathematical Methods in Control	
\end{flushleft}


3	 0	 0	 3


\begin{flushleft}
ELL710	 Coding Theory	
\end{flushleft}


3	0	0	3


\begin{flushleft}
ELL714	 Basic Information Theory	
\end{flushleft}


3	0	0	3


\begin{flushleft}
ELL716	 Telecommunication Switching and Transmission	3	0	0	3
\end{flushleft}


\begin{flushleft}
ELL717	 Optical Communication Systems	
\end{flushleft}


3	 0	 0	 3


\begin{flushleft}
ELL720	 Advanced Digital Signal Processing	
\end{flushleft}


3	 0	 0	 3


\begin{flushleft}
ELL722	 Antenna Theory and Techniques	
\end{flushleft}


3	 0	 0	 3


\begin{flushleft}
ELL723	 Broadband Communication Systems	
\end{flushleft}


3	 0	 0	 3


\begin{flushleft}
ELL724	 Computational Electromagnetics	
\end{flushleft}


3	0	0	3


\begin{flushleft}
ELL725	 Wireless Communications	
\end{flushleft}


3	0	0	3


\begin{flushleft}
ELL730	 I.C. Technology	
\end{flushleft}


3	0	0	3


\begin{flushleft}
ELL732	 Micro and Nanoelectronics	
\end{flushleft}


3	 0	 0	 3


\begin{flushleft}
ELL734	 MOS VLSI design	
\end{flushleft}


3	 0	 0	 3


\begin{flushleft}
ELL735	 Analog Integrated Circuits	
\end{flushleft}


3	 0	 0	 3


\begin{flushleft}
ELL785	 Computer Communication Networks 	
\end{flushleft}


3	 0	 0	 3


\begin{flushleft}
ELL810	 Cyber Security and Information Assurance	 3	 0	 0	 3
\end{flushleft}


\begin{flushleft}
ELL812	 Microwave Propagation and Systems	
\end{flushleft}


3	 0	 0	 3


\begin{flushleft}
ELL813	 Advanced Information Theory	
\end{flushleft}


3	0	0	3


\begin{flushleft}
ELL814	 Wireless Optical Communications	
\end{flushleft}


3	 0	 0	 3


\begin{flushleft}
ELL815	 MIMO Wireless Communications	
\end{flushleft}


3	 0	 0	 3


\begin{flushleft}
ELL816	 Satellite Communication	
\end{flushleft}


3	0	0	3


\begin{flushleft}
ELL818	 Telecommunication Technologies	
\end{flushleft}


3	0	0	3


\begin{flushleft}
ELL821	 Selected Topics in Communication Systems	 3	 0	 0	 3
\end{flushleft}


	


\begin{flushleft}
and Networking-I
\end{flushleft}





\begin{flushleft}
II
\end{flushleft}





\begin{flushleft}
Streamed Electives (EEE) in (Information Processing)
\end{flushleft}


\begin{flushleft}
ELD810	 Minor Project (Communication Engineering)	 0	 0	 6	 3
\end{flushleft}


\begin{flushleft}
ELD812	 Major Project Part-II	
\end{flushleft}


0	 0	 24	 12


\begin{flushleft}
ELL701	 Mathematical Methods in Control	
\end{flushleft}


3	 0	 0	 3


\begin{flushleft}
ELL714	 Basic Information Theory	
\end{flushleft}


3	0	0	3


\begin{flushleft}
ELL715	 Digital Image Processing	
\end{flushleft}


3	 0	 2	 4


\begin{flushleft}
ELL718	 Statistical Signal Processing	
\end{flushleft}


3	 0	 0	 3


\begin{flushleft}
ELL720	 Advanced Digital Signal Processing	
\end{flushleft}


3	 0	 0	 3


\begin{flushleft}
ELL784	 Introduction to Machine Learning	
\end{flushleft}


3	 0	 0	 3


\begin{flushleft}
ELL786	 Multimedia Systems	
\end{flushleft}


3	0	0	3


\begin{flushleft}
ELL792	 Computer Graphics	
\end{flushleft}


3	 0	 0	 3


\begin{flushleft}
ELL793	 Computer Vision	
\end{flushleft}


3	0	0	3


\begin{flushleft}
ELL794	 Human-Computer Interface	
\end{flushleft}


3	0	0	3


\begin{flushleft}
ELL823	 Selected Topics in Information Processing-I	 3	 0	 0	 3
\end{flushleft}


\begin{flushleft}
ELL824	 Selected Topics in Information Processing-II	 3	 0	 0	 3
\end{flushleft}


\begin{flushleft}
ELV781	 Special Modules in Information Processing-I	 1	 0	 0	 1
\end{flushleft}


\begin{flushleft}
ELV823	 Special Modules in Information Processing-II	 1	 0	 0	 1
\end{flushleft}


\begin{flushleft}
CRL704	 Sensor Array Signal Processing	
\end{flushleft}


3	 0	 0	 3


\begin{flushleft}
CRL707	 Human \& Machine Speech Communication	 3	 0	 0	 3
\end{flushleft}





\begin{flushleft}
L
\end{flushleft}





\begin{flushleft}
T
\end{flushleft}





\begin{flushleft}
P
\end{flushleft}





\begin{flushleft}
Total
\end{flushleft}





3





9





1





4





14





12





\begin{flushleft}
PE-2
\end{flushleft}


(3-0-0)





3





9





1





4





14





12





\begin{flushleft}
OE-2
\end{flushleft}


(3-0-0)





2





6





0





12





18





12





0





0





0





24





24





12





4





12





0





0





12





12





\begin{flushleft}
Courses
\end{flushleft}


\begin{flushleft}
(Number, Abbreviated Title, L-T-P, credits)
\end{flushleft}





\begin{flushleft}
Sem.
\end{flushleft}





\begin{flushleft}
I
\end{flushleft}





\begin{flushleft}
ELL822	 Selected Topics in Communication Systems	 3	 0	 0	 3
\end{flushleft}


	


\begin{flushleft}
and Networking-II
\end{flushleft}


\begin{flushleft}
ELL833	 CMOS RF IC Design	
\end{flushleft}


3	 0	 0	 3


\begin{flushleft}
ELL894	 Network Performance Modeling and Analysis	3	0	0	3
\end{flushleft}


\begin{flushleft}
ELP718	 Telecommunication Software Laboratory	
\end{flushleft}


0	 1	 4	 3


\begin{flushleft}
ELP721	Embedded Telecommunication Systems	
\end{flushleft}


0	1	4	3


\begin{flushleft}
	Laboratory
\end{flushleft}


\begin{flushleft}
ELV710	 Special Module in Cyber Security	
\end{flushleft}


1	 0	 0	 1


\begin{flushleft}
ELV720	 Special Module in Communication Systems	 1	 0	 0	 1
\end{flushleft}


	


\begin{flushleft}
and Networking-I
\end{flushleft}


\begin{flushleft}
ELV821	 Special Module in Communication Systems	 1	 0	 0	 1
\end{flushleft}


	


\begin{flushleft}
and Networking-II
\end{flushleft}


\begin{flushleft}
CRL708	 Sonar Systems Engineering	
\end{flushleft}


3	 0	 0	 3


\begin{flushleft}
CRL709	 Underwater Electronic Systems	
\end{flushleft}


3	 0	 0	 3


\begin{flushleft}
CRL712	 RF and Microwave Active Circuits	
\end{flushleft}


3	 0	 0	 3


\begin{flushleft}
CRL715	 Radiating Systems for RF Communication	 3	 0	 0	 3
\end{flushleft}





\begin{flushleft}
ELL 711
\end{flushleft}





\begin{flushleft}
ELL712
\end{flushleft}





(3-0-0)





(3-0-0)





\begin{flushleft}
ELL719
\end{flushleft}





\begin{flushleft}
ELP725
\end{flushleft}





\begin{flushleft}
Signal theory
\end{flushleft}





\begin{flushleft}
Detection and
\end{flushleft}


\begin{flushleft}
Estimation Theory
\end{flushleft}





(3-0-0)





\begin{flushleft}
Digital Comm.
\end{flushleft}





\begin{flushleft}
Wireless Comm. Lab.
\end{flushleft}





(0-1-4)





\begin{flushleft}
ELL 713
\end{flushleft}





\begin{flushleft}
Microwave Theory
\end{flushleft}


\begin{flushleft}
and Techniques
\end{flushleft}





(3-0-0)


\begin{flushleft}
PE-1
\end{flushleft}


(3-0-0)





\begin{flushleft}
Contact h/week
\end{flushleft}





\begin{flushleft}
Credits
\end{flushleft}





\begin{flushleft}
ELD811	 Major Project Part-I (Communication Engineering)	 0	 0	 12	6
\end{flushleft}


\begin{flushleft}
ELL711	 Signal Theory	
\end{flushleft}


3	0	0	3


\begin{flushleft}
ELL712	 Digital Communications	
\end{flushleft}


3	0	0	3


\begin{flushleft}
ELL713	 Microwave Theory and Techniques	
\end{flushleft}


3	 0	 0	 3


\begin{flushleft}
ELL719	 Detection and Estimation Theory	
\end{flushleft}


3	 0	 0	 3


\begin{flushleft}
ELP719	Microwave Laboratory	
\end{flushleft}


0	1	4	3


\begin{flushleft}
ELP725	 Wireless Communication Laboratory	
\end{flushleft}


0	 1	 4	 3


	


\begin{flushleft}
Total Credits				24
\end{flushleft}





\begin{flushleft}
Lecture
\end{flushleft}


\begin{flushleft}
courses
\end{flushleft}





\begin{flushleft}
Program Core
\end{flushleft}





\begin{flushleft}
ELP 719
\end{flushleft}





\begin{flushleft}
Microwave Lab.
\end{flushleft}





(0-1-4)





\begin{flushleft}
Summer
\end{flushleft}


\begin{flushleft}
III
\end{flushleft}


\begin{flushleft}
IV
\end{flushleft}


\begin{flushleft}
(Project
\end{flushleft}


\begin{flushleft}
based) OR
\end{flushleft}


\begin{flushleft}
IV
\end{flushleft}


\begin{flushleft}
(Course
\end{flushleft}


\begin{flushleft}
based)
\end{flushleft}





\begin{flushleft}
ELD811
\end{flushleft}





\begin{flushleft}
OE-1
\end{flushleft}


(3-0-0)





\begin{flushleft}
Major Project Part-I
\end{flushleft}





(0-0-12) 6


\begin{flushleft}
ELD812
\end{flushleft}





\begin{flushleft}
Major Project Part-II
\end{flushleft}





(0-0-24) 12


\begin{flushleft}
PE-3
\end{flushleft}


(3-0-0)





\begin{flushleft}
PE-4
\end{flushleft}


(3-0-0)





\begin{flushleft}
PE-5
\end{flushleft}


(3-0-0)





\begin{flushleft}
PE-6
\end{flushleft}


(3-0-0)





\begin{flushleft}
Total = 48
\end{flushleft}


118





\begin{flushleft}
\newpage
Programme Code: EEN
\end{flushleft}





\begin{flushleft}
Master of Technology in Integrated Electronics and Circuits
\end{flushleft}


\begin{flushleft}
Department of Electrical Engineering
\end{flushleft}


\begin{flushleft}
The overall credits structure
\end{flushleft}


\begin{flushleft}
Category
\end{flushleft}





\begin{flushleft}
PC
\end{flushleft}





\begin{flushleft}
PE
\end{flushleft}





\begin{flushleft}
OC
\end{flushleft}





\begin{flushleft}
Total
\end{flushleft}





\begin{flushleft}
Credits
\end{flushleft}





24





18





6





48





\begin{flushleft}
Major Project Part-I	
\end{flushleft}


0	 0	 12	6


\begin{flushleft}
(Integrated Electronic Circuits)
\end{flushleft}


\begin{flushleft}
I.C. Technology	
\end{flushleft}


3	 0	 0	 3


\begin{flushleft}
Micro and Nanoelectronics	
\end{flushleft}


3	 0	 0	 3


\begin{flushleft}
MOS VLSI design	
\end{flushleft}


3	 0	 0	 3


\begin{flushleft}
Analog Integrated Circuits	
\end{flushleft}


3	 0	 0	 3


\begin{flushleft}
IEC Laboratory-I	
\end{flushleft}


0	 0	 6	 3


\begin{flushleft}
IEC Laboratory-II	
\end{flushleft}


0	 0	 6	 3


\begin{flushleft}
Total Credits				 24
\end{flushleft}





\begin{flushleft}
Streamed Electives (EEN) in (VLSI Design)
\end{flushleft}


\begin{flushleft}
COL719	 Synthesis of Digital Systems	
\end{flushleft}


3	


\begin{flushleft}
ELD830	 Minor Project	
\end{flushleft}


0	


\begin{flushleft}
ELD832	 Major Project Part-II	
\end{flushleft}


0	


\begin{flushleft}
ELL720	 Advanced Digital Signal Processing	
\end{flushleft}


3	


\begin{flushleft}
ELL731	 Mixed Signal Circuit Design	
\end{flushleft}


3	


\begin{flushleft}
ELL733	 Digital ASIC Design	
\end{flushleft}


3	


\begin{flushleft}
ELL736	 Solid State Imaging Sensors	
\end{flushleft}


3	


\begin{flushleft}
ELL737	 Flexible Electronics	
\end{flushleft}


3	


\begin{flushleft}
ELL740	 Compact Modeling of Semiconductor Devices	3	
\end{flushleft}


\begin{flushleft}
ELL741	 Neuromorphic Engineering	
\end{flushleft}


3	


\begin{flushleft}
ELL747	 Active and Passive Filter Design	
\end{flushleft}


3	


\begin{flushleft}
ELL748	 System-on-Chip Design and Test	
\end{flushleft}


3	


\begin{flushleft}
ELL749	 Semiconductor Memory Design	
\end{flushleft}


3	


\begin{flushleft}
ELL782	 Computer Architecture	
\end{flushleft}


3	


\begin{flushleft}
ELL791	 Neural Systems and Learning Machines	
\end{flushleft}


3	


\begin{flushleft}
ELL830	 Issues in Deep Submicron VLSI Design	
\end{flushleft}


3	


\begin{flushleft}
ELL831	 CAD for VLSI, MEMS, and Nanoassembly	 3	
\end{flushleft}


\begin{flushleft}
ELL832	 Selected Topics in IEC-I	
\end{flushleft}


3	


\begin{flushleft}
ELL833	 CMOS RF IC Design	
\end{flushleft}


3	


\begin{flushleft}
ELL834	 Selected Topics in IEC-II	
\end{flushleft}


3	


\begin{flushleft}
ELP830	 Semiconductor Processing Laboratory	
\end{flushleft}


0	


\begin{flushleft}
ELV734	 Special Module in Scientific Writing for	
\end{flushleft}


1	


\begin{flushleft}
	Research
\end{flushleft}


\begin{flushleft}
ELV830	 Special Module in Low Power IC Design	
\end{flushleft}


1	


\begin{flushleft}
ELV831	 Special Module in VLSI Testing	
\end{flushleft}


1	


\begin{flushleft}
ELV832	 Special Module in Machine Learning	
\end{flushleft}


1	





0	 2	 4


0	 6	 3


0	 24	12


0	 0	 3


0	 0	 3


0	 2	 4


0	 0	 3


0	 0	 3


0	 0	 3


0	 0	 3


0	 0	 3


0	 0	 3


0	 0	 3


0	 0	 3


0	 2	 4


0	 0	 3


0	 0	 3


0	 0	 3


0	 0	 3


0	 0	 3


0	 6	 3


0	 0	 1


0	 0	 1


0	 0	 1


0	 0	 1





\begin{flushleft}
Streamed Electives (EEN) in (Nanoelectronics and Photonics)
\end{flushleft}


\begin{flushleft}
ELD830	
\end{flushleft}


\begin{flushleft}
ELD832	
\end{flushleft}


\begin{flushleft}
ELL737	
\end{flushleft}


\begin{flushleft}
ELL738	
\end{flushleft}





\begin{flushleft}
Minor Project	
\end{flushleft}


\begin{flushleft}
Major Project Part-II	
\end{flushleft}


\begin{flushleft}
Flexible Electronics	
\end{flushleft}


\begin{flushleft}
Micro and Nano Photonics	
\end{flushleft}





0	


0	


3	


3	





0	


0	


0	


0	





6	 3


24	12


0	 3


0	 3





\begin{flushleft}
ELL739	 Advanced Semiconductor Devices	
\end{flushleft}


3	 0	 0	 3


\begin{flushleft}
ELL740	 Compact Modeling of Semiconductor Devices	 3	 0	0	 3
\end{flushleft}


\begin{flushleft}
ELL741	 Neuromorphic Engineering	
\end{flushleft}


3	 0	 0	 3


\begin{flushleft}
ELL742	 Introduction to MEMS Design	
\end{flushleft}


3	 0	 0	 3


\begin{flushleft}
ELL743	 Photovoltaics	
\end{flushleft}


3	 0	0	 3


\begin{flushleft}
ELL744	 Electronic and Photonic Nanomaterials	
\end{flushleft}


3	 0	 0	 3


\begin{flushleft}
ELL745	 Quantum Electronics	
\end{flushleft}


3	 0	 0	 3


\begin{flushleft}
ELL746	 Biomedical Electronics 	
\end{flushleft}


3	 0	 0	 3


\begin{flushleft}
ELL749	 Semiconductor Memory Design	
\end{flushleft}


3	 0	 0	 3


\begin{flushleft}
ELL791	 Neural Systems and Learning Machines	
\end{flushleft}


3	 0	 2	 4


\begin{flushleft}
ELL830	 Issues in Deep Submicron VLSI Design	
\end{flushleft}


3	 0	 0	 3


\begin{flushleft}
ELL832	 Selected Topics in IEC-I	
\end{flushleft}


3	 0	 0	 3


\begin{flushleft}
ELL834	 Selected Topics in IEC-II	
\end{flushleft}


3	 0	 0	 3


\begin{flushleft}
ELP830	 Semiconductor Processing Laboratory	
\end{flushleft}


0	 0	 6	 3


\begin{flushleft}
ELP833	 Device and Materials Characterization	
\end{flushleft}


0	 0	 6	 3


\begin{flushleft}
	Laboratory
\end{flushleft}


\begin{flushleft}
ELV734	 Special Module in Scientific Writing for Research	1	 0	 0	 1
\end{flushleft}


\begin{flushleft}
ELV833	 Special Module in Semiconductor Business	 1	 0	 0	 1
\end{flushleft}


\begin{flushleft}
	Management
\end{flushleft}


\begin{flushleft}
ELV834	 Special Module in Nanoelectronics	
\end{flushleft}


1	 0	 0	 1


\begin{flushleft}
Streamed Electives (EEN) in (Embedded Intelligent Systems)
\end{flushleft}


\begin{flushleft}
COL719	
\end{flushleft}


\begin{flushleft}
COL788	
\end{flushleft}


\begin{flushleft}
ELD830	
\end{flushleft}


\begin{flushleft}
ELD832	
\end{flushleft}


\begin{flushleft}
ELL720	
\end{flushleft}


\begin{flushleft}
ELL731	
\end{flushleft}


\begin{flushleft}
ELL733	
\end{flushleft}


\begin{flushleft}
ELL736	
\end{flushleft}


\begin{flushleft}
ELL748	
\end{flushleft}


\begin{flushleft}
ELL782	
\end{flushleft}


\begin{flushleft}
ELL784	
\end{flushleft}


\begin{flushleft}
ELL787	
\end{flushleft}


\begin{flushleft}
ELL789	
\end{flushleft}


\begin{flushleft}
ELL791	
\end{flushleft}


\begin{flushleft}
ELL830	
\end{flushleft}


\begin{flushleft}
ELL831	
\end{flushleft}


\begin{flushleft}
ELL832	
\end{flushleft}


\begin{flushleft}
ELL834	
\end{flushleft}


\begin{flushleft}
ELL883	
\end{flushleft}


\begin{flushleft}
ELV734	
\end{flushleft}


\begin{flushleft}
ELV831	
\end{flushleft}


\begin{flushleft}
ELV832	
\end{flushleft}





\begin{flushleft}
Synthesis of Digital Systems	
\end{flushleft}


3	


\begin{flushleft}
Advanced Topics in Embedded Computing	 3	
\end{flushleft}


\begin{flushleft}
Minor Project	
\end{flushleft}


0	


\begin{flushleft}
Major Project Part-II	
\end{flushleft}


0	


\begin{flushleft}
Advanced Digital Signal Processing	
\end{flushleft}


3	


\begin{flushleft}
Mixed Signal Circuit Design	
\end{flushleft}


3	


\begin{flushleft}
Digital ASIC Design	
\end{flushleft}


3	


\begin{flushleft}
Solid State Imaging Sensors	
\end{flushleft}


3	


\begin{flushleft}
System-on-Chip Design and Test	
\end{flushleft}


3	


\begin{flushleft}
Computer Architecture	
\end{flushleft}


3	


\begin{flushleft}
Introduction to Machine Learning	
\end{flushleft}


3	


\begin{flushleft}
Embedded Systems and Applications	
\end{flushleft}


3	


\begin{flushleft}
Intelligent Systems	
\end{flushleft}


3	


\begin{flushleft}
Neural Systems and Learning Machines	
\end{flushleft}


3	


\begin{flushleft}
Issues in Deep Submicron VLSI Design	
\end{flushleft}


3	


\begin{flushleft}
CAD for VLSI, MEMS, and Nanoassembly	 3	
\end{flushleft}


\begin{flushleft}
Selected Topics in IEC-I	
\end{flushleft}


3	


\begin{flushleft}
Selected Topics in IEC-II	
\end{flushleft}


3	


\begin{flushleft}
Embedded Intelligence	
\end{flushleft}


3	


\begin{flushleft}
Special Module in Scientific Writing for Research	1	
\end{flushleft}


\begin{flushleft}
Special Module in VLSI Testing	
\end{flushleft}


1	


\begin{flushleft}
Special Module in Machine Learning	
\end{flushleft}


1	





\begin{flushleft}
Courses
\end{flushleft}


\begin{flushleft}
(Number, Abbreviated Title, L-T-P, credits)
\end{flushleft}





\begin{flushleft}
Sem.
\end{flushleft}





\begin{flushleft}
ELL732
\end{flushleft}


\begin{flushleft}
I
\end{flushleft}





\begin{flushleft}
Micro and
\end{flushleft}


\begin{flushleft}
Nanoelectronics
\end{flushleft}





\begin{flushleft}
II
\end{flushleft}





(3-0-0)


\begin{flushleft}
PE
\end{flushleft}


(3-0-0)





\begin{flushleft}
ELL735
\end{flushleft}





\begin{flushleft}
Analog
\end{flushleft}


\begin{flushleft}
Integrated Circuits
\end{flushleft}





(3-0-0)


\begin{flushleft}
ELP832
\end{flushleft}





\begin{flushleft}
IEC Lab-II
\end{flushleft}





(0-0-6)





0	


0	


0	


0	


0	


0	


0	


0	


0	


0	


0	


0	


0	


0	


0	


0	


0	


0	


0	


0	


0	


0	





\begin{flushleft}
Contact h/week
\end{flushleft}





2	 4


0	 3


6	 3


24	12


0	 3


0	 3


2	 4


0	 3


0	 3


0	 3


0	 3


0	 3


0	 3


2	 4


0	 3


0	 3


0	 3


0	 3


0	 3


0	 1


0	 1


0	 1





\begin{flushleft}
L
\end{flushleft}





\begin{flushleft}
T
\end{flushleft}





\begin{flushleft}
P
\end{flushleft}





\begin{flushleft}
Total
\end{flushleft}





\begin{flushleft}
Credits
\end{flushleft}





\begin{flushleft}
ELD831	
\end{flushleft}


	


\begin{flushleft}
ELL730	
\end{flushleft}


\begin{flushleft}
ELL732	
\end{flushleft}


\begin{flushleft}
ELL734	
\end{flushleft}


\begin{flushleft}
ELL735	
\end{flushleft}


\begin{flushleft}
ELP831	
\end{flushleft}


\begin{flushleft}
ELP832	
\end{flushleft}


	





\begin{flushleft}
Lecture
\end{flushleft}


\begin{flushleft}
courses
\end{flushleft}





\begin{flushleft}
Program Core
\end{flushleft}





3





9





0





6





15





12





\begin{flushleft}
ELL734
\end{flushleft}





\begin{flushleft}
ELP831
\end{flushleft}





(3-0-0)





(0-0-6)





\begin{flushleft}
ELL730
\end{flushleft}





\begin{flushleft}
PE / OE
\end{flushleft}


(3-0-0)





3





9





0





6





15





12





\begin{flushleft}
PE / OE
\end{flushleft}


(3-0-0)





2





6





0





12





18





12





0





0





0





24





24





12





4





12





0





0





12





12





\begin{flushleft}
MOS VLSI Design
\end{flushleft}





\begin{flushleft}
I.C. Technology
\end{flushleft}





(3-0-0)





\begin{flushleft}
IEC Lab-I
\end{flushleft}





\begin{flushleft}
Summer
\end{flushleft}


\begin{flushleft}
III
\end{flushleft}





\begin{flushleft}
ELD831
\end{flushleft}





\begin{flushleft}
PE / OE
\end{flushleft}


(3-0-0)





\begin{flushleft}
Major Project Part-I
\end{flushleft}





(0-0-12)


\begin{flushleft}
IV
\end{flushleft}





\begin{flushleft}
(Project based)
\end{flushleft}


\begin{flushleft}
OR
\end{flushleft}





\begin{flushleft}
IV
\end{flushleft}





\begin{flushleft}
(Course based)
\end{flushleft}





\begin{flushleft}
ELD832
\end{flushleft}





\begin{flushleft}
Major Project Part-II
\end{flushleft}





(0-0-24)


\begin{flushleft}
PE / OE
\end{flushleft}


(3-0-0)





\begin{flushleft}
PE / OE
\end{flushleft}


(3-0-0)





\begin{flushleft}
PE / OE
\end{flushleft}


(3-0-0)





\begin{flushleft}
PE / OE
\end{flushleft}


(3-0-0)





\begin{flushleft}
Total = 48
\end{flushleft}


119





\begin{flushleft}
\newpage
Programme Code: EEP
\end{flushleft}





\begin{flushleft}
Master of Technology in Power Electronics, Electrical Machines and Drives
\end{flushleft}


\begin{flushleft}
Department of Electrical Engineering
\end{flushleft}


\begin{flushleft}
The overall credits structure
\end{flushleft}


\begin{flushleft}
PE
\end{flushleft}





\begin{flushleft}
OC
\end{flushleft}





\begin{flushleft}
Total
\end{flushleft}





24





18





6





48





\begin{flushleft}
Program Core
\end{flushleft}


\begin{flushleft}
ELD851	
\end{flushleft}


\begin{flushleft}
ELL750	
\end{flushleft}


\begin{flushleft}
ELL751	
\end{flushleft}


\begin{flushleft}
ELL752	
\end{flushleft}


\begin{flushleft}
ELL850	
\end{flushleft}


	


\begin{flushleft}
ELP850	
\end{flushleft}


\begin{flushleft}
ELP851	
\end{flushleft}


\begin{flushleft}
ELP852	
\end{flushleft}


\begin{flushleft}
ELP853	
\end{flushleft}


	





\begin{flushleft}
Major Project Part-I	
\end{flushleft}


\begin{flushleft}
Modelling of Electrical Machines 	
\end{flushleft}


\begin{flushleft}
Power Electronic Converters	
\end{flushleft}


\begin{flushleft}
Electric Drive System	
\end{flushleft}


\begin{flushleft}
Digital Control of Power Electronics and 	
\end{flushleft}


\begin{flushleft}
Drive Systems
\end{flushleft}


\begin{flushleft}
Electrical Machines Laboratory 	
\end{flushleft}


\begin{flushleft}
Power Electronics Laboratory 	
\end{flushleft}


\begin{flushleft}
Electrical Drives Laboratory 	
\end{flushleft}


\begin{flushleft}
DSP Based Control of Power Electronics 	
\end{flushleft}


\begin{flushleft}
and Drives Laboratory
\end{flushleft}





0	


3	


3	


3	


3	





0	


0	


0	


0	


0	





12	6


0	 3


0	 3


0	 3


0	 3





0	


0	


0	


0	





0	


0	


0	


0	





3	


3	


3	


3	





	





\begin{flushleft}
Total Credits				
\end{flushleft}





1.5


1.5


1.5


1.5


24





\begin{flushleft}
Program Electives
\end{flushleft}


\begin{flushleft}
ELD850	
\end{flushleft}


\begin{flushleft}
ELD852	
\end{flushleft}


\begin{flushleft}
ELL700	
\end{flushleft}


\begin{flushleft}
ELL703	
\end{flushleft}


\begin{flushleft}
ELL704	
\end{flushleft}


\begin{flushleft}
ELL706	
\end{flushleft}


\begin{flushleft}
ELL720	
\end{flushleft}


\begin{flushleft}
ELL753	
\end{flushleft}


\begin{flushleft}
ELL754	
\end{flushleft}


\begin{flushleft}
ELL755	
\end{flushleft}


\begin{flushleft}
ELL756	
\end{flushleft}


\begin{flushleft}
ELL757	
\end{flushleft}





\begin{flushleft}
Minor Project	
\end{flushleft}


\begin{flushleft}
Major Project Part-II	
\end{flushleft}


\begin{flushleft}
Linear Systems Theory	
\end{flushleft}


\begin{flushleft}
Optimal Control Theory	
\end{flushleft}


\begin{flushleft}
Advanced Robotics	
\end{flushleft}


\begin{flushleft}
Digital Control	
\end{flushleft}


\begin{flushleft}
Advanced Digital Signal Processing	
\end{flushleft}


\begin{flushleft}
Physical Phenomena in Electrical Machines 	
\end{flushleft}


\begin{flushleft}
Permanent Magnet Machines 	
\end{flushleft}


\begin{flushleft}
Variable Reluctance Machines 	
\end{flushleft}


\begin{flushleft}
Special Electrical Machines	
\end{flushleft}


\begin{flushleft}
Energy Efficient Motors 	
\end{flushleft}





\begin{flushleft}
ELL750
\end{flushleft}





\begin{flushleft}
II
\end{flushleft}





0	


0	


0	


0	


0	


0	


0	


0	


0	


0	


0	


0	





6	 3


24	12


0	 3


0	 3


0	 3


0	 3


0	 3


0	 3


0	 3


0	 3


0	 3


0	 3





\begin{flushleft}
Power Quality 	
\end{flushleft}


3	 0	 0	 3


\begin{flushleft}
Power Electronic Converters for Renewable 	 3	 0	 0	 3
\end{flushleft}


\begin{flushleft}
Energy Systems
\end{flushleft}


\begin{flushleft}
Switched Mode Power Conversion 	
\end{flushleft}


3	 0	 0	 3


\begin{flushleft}
Power Electronics for Utility Interface 	
\end{flushleft}


3	 0	 0	 3


\begin{flushleft}
Intelligent Motor Controllers 	
\end{flushleft}


3	 0	 0	 3


\begin{flushleft}
Advanced Electric Drives 	
\end{flushleft}


3	 0	 0	 3


\begin{flushleft}
Electric Vehicles 	
\end{flushleft}


3	 0	 0	 3


\begin{flushleft}
Smart Grid Technology	
\end{flushleft}


3	 0	 0	 3


\begin{flushleft}
Appliance Systems	
\end{flushleft}


3	 0	 0	 3


\begin{flushleft}
Mechatronics	
\end{flushleft}


3	 0	0	 3


\begin{flushleft}
Computer Aided Design of Power	
\end{flushleft}


3	 0	 0	 3


\begin{flushleft}
Electronic Systems
\end{flushleft}


\begin{flushleft}
Embedded Systems and Applications	
\end{flushleft}


3	 0	 0	 3


\begin{flushleft}
Neural Systems and Learning Machines	
\end{flushleft}


3	 0	 2	 4


\begin{flushleft}
Computer Aided Design of Electrical Machines	3	 0	0	 3
\end{flushleft}


\begin{flushleft}
Condition Monitoring of Electrical Machines 	 3	 0	 0	 3
\end{flushleft}


\begin{flushleft}
Advanced Topics in Electrical Machines 	
\end{flushleft}


3	 0	 0	 3


\begin{flushleft}
Selected Topics in Electrical Machines	
\end{flushleft}


3	 0	 0	 3


\begin{flushleft}
High Power Converters 	
\end{flushleft}


3	 0	 0	 3


\begin{flushleft}
Advanced Topics in Power Electronics	
\end{flushleft}


3	 0	 0	 3


\begin{flushleft}
Selected Topics in Power Electronics	
\end{flushleft}


3	 0	 0	 3


\begin{flushleft}
Advanced Topics in Electric Drives 	
\end{flushleft}


3	 0	 0	 3


\begin{flushleft}
Selected Topics in Electric Drives 	
\end{flushleft}


3	 0	 0	 3


\begin{flushleft}
Electrical Machines CAD Laboratory 	
\end{flushleft}


0	 1	 4	 3


\begin{flushleft}
Smart Grids Laboratory 	
\end{flushleft}


0	 1	 4	 3


\begin{flushleft}
Industrial Training and Seminar	
\end{flushleft}


0	 0	 6	 3





\begin{flushleft}
Courses
\end{flushleft}


\begin{flushleft}
(Number, Abbreviated Title, L-T-P, credits)
\end{flushleft}





\begin{flushleft}
Sem.
\end{flushleft}





\begin{flushleft}
I
\end{flushleft}





0	


0	


3	


3	


3	


3	


3	


3	


3	


3	


3	


3	





\begin{flushleft}
ELL758	
\end{flushleft}


\begin{flushleft}
ELL759	
\end{flushleft}


	


\begin{flushleft}
ELL760	
\end{flushleft}


\begin{flushleft}
ELL761	
\end{flushleft}


\begin{flushleft}
ELL762	
\end{flushleft}


\begin{flushleft}
ELL763	
\end{flushleft}


\begin{flushleft}
ELL764	
\end{flushleft}


\begin{flushleft}
ELL765	
\end{flushleft}


\begin{flushleft}
ELL766	
\end{flushleft}


\begin{flushleft}
ELL767	
\end{flushleft}


\begin{flushleft}
ELL768	
\end{flushleft}


	


\begin{flushleft}
ELL787	
\end{flushleft}


\begin{flushleft}
ELL791	
\end{flushleft}


\begin{flushleft}
ELL851	
\end{flushleft}


\begin{flushleft}
ELL852	
\end{flushleft}


\begin{flushleft}
ELL853	
\end{flushleft}


\begin{flushleft}
ELL854	
\end{flushleft}


\begin{flushleft}
ELL855	
\end{flushleft}


\begin{flushleft}
ELL856	
\end{flushleft}


\begin{flushleft}
ELL857	
\end{flushleft}


\begin{flushleft}
ELL858	
\end{flushleft}


\begin{flushleft}
ELL859	
\end{flushleft}


\begin{flushleft}
ELP854	
\end{flushleft}


\begin{flushleft}
ELP855	
\end{flushleft}


\begin{flushleft}
ELT850	
\end{flushleft}





\begin{flushleft}
Modelling of
\end{flushleft}


\begin{flushleft}
Electrical Machines
\end{flushleft}





\begin{flushleft}
ELL751
\end{flushleft}





\begin{flushleft}
Power Electronic
\end{flushleft}


\begin{flushleft}
Converters
\end{flushleft}





\begin{flushleft}
ELP850
\end{flushleft}





\begin{flushleft}
Electrical Machines
\end{flushleft}


\begin{flushleft}
Laboratory
\end{flushleft}





\begin{flushleft}
ELP851
\end{flushleft}





\begin{flushleft}
Power Electronics
\end{flushleft}


\begin{flushleft}
Laboratory
\end{flushleft}





(3-0-0)





(3-0-0)





(0-0-3)





(0-0-3)





\begin{flushleft}
ELL752
\end{flushleft}





\begin{flushleft}
ELL850
\end{flushleft}





\begin{flushleft}
ELP852
\end{flushleft}





\begin{flushleft}
ELP853
\end{flushleft}





\begin{flushleft}
Electric Drive
\end{flushleft}


\begin{flushleft}
System
\end{flushleft}





(3-0-0)





\begin{flushleft}
Digital Control of Power
\end{flushleft}


\begin{flushleft}
Electronics and
\end{flushleft}


\begin{flushleft}
Drive Systems
\end{flushleft}





(3-0-0)





\begin{flushleft}
Electrical Drives
\end{flushleft}


\begin{flushleft}
Laboratory
\end{flushleft}





(0-0-3)





\begin{flushleft}
DSP Based Control
\end{flushleft}


\begin{flushleft}
of Power Electronics
\end{flushleft}


\begin{flushleft}
and Drives
\end{flushleft}


\begin{flushleft}
Laboratory
\end{flushleft}





\begin{flushleft}
PE/OE
\end{flushleft}


(3-0-0)*





\begin{flushleft}
L
\end{flushleft}





\begin{flushleft}
T
\end{flushleft}





\begin{flushleft}
P
\end{flushleft}





\begin{flushleft}
Total
\end{flushleft}





3





9





0





6





15





12





3





9





0





6





15





12





2





6





0





12





18





12





0





0





0





24





24





12





4





12





0





0





12





12





2





6





0





12





18





12





\begin{flushleft}
Contact h/week
\end{flushleft}





\begin{flushleft}
Credits
\end{flushleft}





\begin{flushleft}
PC
\end{flushleft}





\begin{flushleft}
Credits
\end{flushleft}





\begin{flushleft}
Lecture
\end{flushleft}


\begin{flushleft}
courses
\end{flushleft}





\begin{flushleft}
Category
\end{flushleft}





\begin{flushleft}
PE/OE
\end{flushleft}


(3-0-0)*





(0-0-3)


\begin{flushleft}
Project Based
\end{flushleft}





\begin{flushleft}
ELD851
\end{flushleft}


\begin{flushleft}
III
\end{flushleft}





\begin{flushleft}
IV
\end{flushleft}





\begin{flushleft}
Major Project Part-I
\end{flushleft}





(0-0-12)





\begin{flushleft}
PE/OE
\end{flushleft}


(3-0-0)*





\begin{flushleft}
PE/OE
\end{flushleft}


(3-0-0)*





\begin{flushleft}
ELD852
\end{flushleft}





\begin{flushleft}
Major Project Part-II
\end{flushleft}





(0-0-24)


\begin{flushleft}
(OR) Course Based
\end{flushleft}





\begin{flushleft}
III
\end{flushleft}


\begin{flushleft}
IV
\end{flushleft}





\begin{flushleft}
PE/OE
\end{flushleft}


(3-0-0)





\begin{flushleft}
PE/OE
\end{flushleft}


(3-0-0)





\begin{flushleft}
PE/OE
\end{flushleft}


(3-0-0)





\begin{flushleft}
ELD851
\end{flushleft}





\begin{flushleft}
PE/OE
\end{flushleft}


(3-0-0)





\begin{flushleft}
PE/OE
\end{flushleft}


(3-0-0)





\begin{flushleft}
Major Project Part-I
\end{flushleft}





(0-0-12)





\begin{flushleft}
PE/OE
\end{flushleft}


(3-0-0)





\begin{flushleft}
Total = 48
\end{flushleft}


120





\begin{flushleft}
\newpage
Programme Code: EES
\end{flushleft}





\begin{flushleft}
Master of Technology in Power Systems
\end{flushleft}


\begin{flushleft}
Department of Electrical Engineering
\end{flushleft}


\begin{flushleft}
The overall credits structure
\end{flushleft}


\begin{flushleft}
PE
\end{flushleft}





\begin{flushleft}
OC
\end{flushleft}





\begin{flushleft}
Total
\end{flushleft}





24





18





6





48





\begin{flushleft}
Program Core
\end{flushleft}


\begin{flushleft}
ELD871	 Major Project Part-I 	
\end{flushleft}


0	 0	 12	6


\begin{flushleft}
ELL770	 Power System Analysis	
\end{flushleft}


3	0	 0	3


\begin{flushleft}
ELL771	 Advanced Power System Protection	
\end{flushleft}


3	 0	 0	 3


\begin{flushleft}
ELL775	 Power System Dynamics	
\end{flushleft}


3	 0	 0	 3


\begin{flushleft}
ELL776	 Advanced Power System Optimization	
\end{flushleft}


3	 0	 0	 3


\begin{flushleft}
ELP870	 Power System Lab-I	
\end{flushleft}


0	 1	 4	 3


\begin{flushleft}
ELP871 	 Power System Lab-II	
\end{flushleft}


0	 1	 4	 3


	


\begin{flushleft}
Total Credits				24
\end{flushleft}


\begin{flushleft}
Program Electives
\end{flushleft}


\begin{flushleft}
ELD870	 Minor Project-I	
\end{flushleft}


\begin{flushleft}
ELD872	 Major Project Part-II	
\end{flushleft}


\begin{flushleft}
ELL700	 Linear Systems Theory	
\end{flushleft}


\begin{flushleft}
ELL712	 Digital Communications	
\end{flushleft}





\begin{flushleft}
ELL 770
\end{flushleft}





\begin{flushleft}
II
\end{flushleft}





6	3


24	12


0	3


0	3





\begin{flushleft}
Power Quality 	
\end{flushleft}


3	 0	


\begin{flushleft}
Power Electronic Converters for Renewable 	 3	 0	
\end{flushleft}


\begin{flushleft}
Energy Systems
\end{flushleft}


\begin{flushleft}
Planning and Operation of a Smart Grid	
\end{flushleft}


3	 0	


\begin{flushleft}
High Voltage DC Transmission	
\end{flushleft}


3	 0	


\begin{flushleft}
Flexible AC Transmission System	
\end{flushleft}


3	0	


\begin{flushleft}
Power System operation and control 	
\end{flushleft}


3	 0	


\begin{flushleft}
Dynamic Modelling And Control of	
\end{flushleft}


3	 0	


\begin{flushleft}
Sustainable Energy Systems
\end{flushleft}


\begin{flushleft}
Forecasting Techniques for Power System	 3	 0	
\end{flushleft}


\begin{flushleft}
Restructured Power System 	
\end{flushleft}


3	 0	


\begin{flushleft}
Distribution System Operation and Planning 	 3	 0	
\end{flushleft}


\begin{flushleft}
Selected Topics in Power System	
\end{flushleft}


3	 0	


\begin{flushleft}
Power System Transient	
\end{flushleft}


3	0	


\begin{flushleft}
Power System Reliability	
\end{flushleft}


3	 0	





\begin{flushleft}
Courses
\end{flushleft}


\begin{flushleft}
(number, Abbreviated Title, L-T-P, credits)
\end{flushleft}





\begin{flushleft}
Sem.
\end{flushleft}





\begin{flushleft}
I
\end{flushleft}





0	0	


0	 0	


3	0	


3	0	





\begin{flushleft}
ELL758	
\end{flushleft}


\begin{flushleft}
ELL759	
\end{flushleft}


	


\begin{flushleft}
ELL772	
\end{flushleft}


\begin{flushleft}
ELL773	
\end{flushleft}


\begin{flushleft}
ELL774	
\end{flushleft}


\begin{flushleft}
ELL777	
\end{flushleft}


\begin{flushleft}
ELL778	
\end{flushleft}


	


\begin{flushleft}
ELL779	
\end{flushleft}


\begin{flushleft}
ELL870	
\end{flushleft}


\begin{flushleft}
ELL871	
\end{flushleft}


\begin{flushleft}
ELL872	
\end{flushleft}


\begin{flushleft}
ELL873	
\end{flushleft}


\begin{flushleft}
ELL874	
\end{flushleft}





\begin{flushleft}
Power System
\end{flushleft}


\begin{flushleft}
Analysis
\end{flushleft}





\begin{flushleft}
ELL771
\end{flushleft}





\begin{flushleft}
Advanced Power
\end{flushleft}


\begin{flushleft}
System Protection
\end{flushleft}





\begin{flushleft}
ELL775
\end{flushleft}





\begin{flushleft}
Power System
\end{flushleft}


\begin{flushleft}
Dynamics
\end{flushleft}





\begin{flushleft}
Contact h/week
\end{flushleft}





0	 3


0	 3


0	 3


0	 3


0	3


0	 3


0	 3


0	 3


0	 3


0	 3


0	 3


0	3


0	 3





\begin{flushleft}
L
\end{flushleft}





\begin{flushleft}
T
\end{flushleft}





\begin{flushleft}
P
\end{flushleft}





\begin{flushleft}
Total
\end{flushleft}





\begin{flushleft}
Credits
\end{flushleft}





\begin{flushleft}
PC
\end{flushleft}





\begin{flushleft}
Credits
\end{flushleft}





\begin{flushleft}
Lecture
\end{flushleft}


\begin{flushleft}
courses
\end{flushleft}





\begin{flushleft}
Category
\end{flushleft}





3





9





1





4





14





12





\begin{flushleft}
ELP870
\end{flushleft}





\begin{flushleft}
Power System
\end{flushleft}


\begin{flushleft}
Lab-I
\end{flushleft}





(3-0-0)





(3-0-0)





(3-0-0)





(0-1-4)





\begin{flushleft}
ELL776
\end{flushleft}





\begin{flushleft}
ELP871
\end{flushleft}





\begin{flushleft}
PE/OE
\end{flushleft}


(3-0-0)





\begin{flushleft}
PE/OE
\end{flushleft}


(3-0-0)





3





9





1





4





14





12





(3-0-0)





(0-1-4)


\begin{flushleft}
PE/OE
\end{flushleft}


(3-0-0)





\begin{flushleft}
PE/OE
\end{flushleft}


(3-0-0)





2





6





0





12





18





12





0





0





0





24





24





12





4





12





0





0





12





12





\begin{flushleft}
Advanced Power System
\end{flushleft}


\begin{flushleft}
Optimization
\end{flushleft}





\begin{flushleft}
Power System
\end{flushleft}


\begin{flushleft}
Lab-II
\end{flushleft}





\begin{flushleft}
Summer
\end{flushleft}


\begin{flushleft}
III
\end{flushleft}


\begin{flushleft}
IV
\end{flushleft}





\begin{flushleft}
(Project based)
\end{flushleft}


\begin{flushleft}
OR
\end{flushleft}





\begin{flushleft}
IV
\end{flushleft}





\begin{flushleft}
(Course based)
\end{flushleft}





\begin{flushleft}
ELD871
\end{flushleft}





\begin{flushleft}
Major Project Part-I
\end{flushleft}





(0-0-12)


\begin{flushleft}
ELD871
\end{flushleft}





\begin{flushleft}
Major Project Part-II
\end{flushleft}





(0-0-24)


\begin{flushleft}
PE/OE
\end{flushleft}


(3-0-0)





\begin{flushleft}
PE/OE
\end{flushleft}


(3-0-0)





\begin{flushleft}
PE/OE
\end{flushleft}


(3-0-0)





\begin{flushleft}
PE/OE
\end{flushleft}


(3-0-0)





\begin{flushleft}
Total = 48
\end{flushleft}


121





\begin{flushleft}
\newpage
Master of Technology in Computer Technology
\end{flushleft}





\begin{flushleft}
Programme Code: EET
\end{flushleft}





\begin{flushleft}
Department of Electrical Engineering
\end{flushleft}


\begin{flushleft}
The overall credits structure
\end{flushleft}


\begin{flushleft}
Category
\end{flushleft}





\begin{flushleft}
PC
\end{flushleft}





\begin{flushleft}
PE
\end{flushleft}





\begin{flushleft}
OC
\end{flushleft}





\begin{flushleft}
Total
\end{flushleft}





\begin{flushleft}
Credits
\end{flushleft}





21





24/27





3/6





51





\begin{flushleft}
Program Core
\end{flushleft}


\begin{flushleft}
ELD780	Minor Project	
\end{flushleft}


0	0	4	2


\begin{flushleft}
ELD880	 Major Project Part-I	
\end{flushleft}


0	 0	 12	 6


\begin{flushleft}
ELL780	 Mathematical Foundations of Computer 	
\end{flushleft}


3	 0	 0	 3


\begin{flushleft}
	Technology	
\end{flushleft}


\begin{flushleft}
ELL781	 Software Fundamentals for Computer	
\end{flushleft}


3	 0	 0	 3


\begin{flushleft}
	Technology
\end{flushleft}


\begin{flushleft}
ELL782	 Computer Architecture	
\end{flushleft}


3	0	0	3


\begin{flushleft}
ELL783	 Operating Systems	
\end{flushleft}


3	0	2	4


	


\begin{flushleft}
Total Credits				21
\end{flushleft}


\begin{flushleft}
Program Electives
\end{flushleft}


\begin{flushleft}
ELD881	
\end{flushleft}


\begin{flushleft}
ELL880	
\end{flushleft}


\begin{flushleft}
ELL881	
\end{flushleft}


\begin{flushleft}
ELV752	
\end{flushleft}


\begin{flushleft}
ELV780	
\end{flushleft}





\begin{flushleft}
Major Project Part-II	
\end{flushleft}


\begin{flushleft}
Special Topics in Computers-I	
\end{flushleft}


\begin{flushleft}
Special Topics in Computers-II	
\end{flushleft}


\begin{flushleft}
Special Modules in EET -- I	
\end{flushleft}


\begin{flushleft}
Special Module in Computers	
\end{flushleft}





0	


3	


3	


1	


1	





0	


0	


0	


0	


0	





24	 12


0	 3


0	 3


0	 1


0	 1





\begin{flushleft}
ELL748	 System-on-Chip Design and Test	
\end{flushleft}


\begin{flushleft}
ELL766	 Appliance Systems	
\end{flushleft}


\begin{flushleft}
ELL767	 Mechatronics	
\end{flushleft}


\begin{flushleft}
ELL785	 Computer Communication Networks 	
\end{flushleft}


\begin{flushleft}
ELL786	 Multimedia Systems	
\end{flushleft}


\begin{flushleft}
ELL790	 Digital Hardware Design	
\end{flushleft}


\begin{flushleft}
ELL791	 Neural Systems and Learning Machines	
\end{flushleft}


\begin{flushleft}
ELL797	 Energy-Efficient Computing	
\end{flushleft}


\begin{flushleft}
ELL802	 Adaptive and Learning Control	
\end{flushleft}


\begin{flushleft}
ELL883	 Embedded Intelligence	
\end{flushleft}


\begin{flushleft}
ELL887	 Cloud Computing	
\end{flushleft}


\begin{flushleft}
ELL898	 Pervasive Computing	
\end{flushleft}


\begin{flushleft}
ELL899	 Testing and Fault Tolerance	
\end{flushleft}


\begin{flushleft}
ELP780	Software Lab	
\end{flushleft}


\begin{flushleft}
ELP781	 Digital Systems Lab	
\end{flushleft}


\begin{flushleft}
ELP831	IEC Laboratory-I	
\end{flushleft}





3	 0	 0	 3


3	0	0	3


3	0	0	3


3	 0	 0	 3


3	0	0	3


3	 0	 0	 3


3	 0	 2	 4


3	 0	 0	 3


3	 0	 0	 3


3	0	0	3


3	0	0	3


3	0	0	3


3	 0	 0	 3


0	1	4	3


0	 1	 4	 3


0	0	6	3





\begin{flushleft}
Streamed Electives (EET) in (Cognitive and Intelligent Systems)
\end{flushleft}





\begin{flushleft}
Streamed Electives (EET) in (Computer Communication and
\end{flushleft}


\begin{flushleft}
Networks)
\end{flushleft}





\begin{flushleft}
Required Electives
\end{flushleft}


\begin{flushleft}
ELL784	 Introduction to Machine Learning	
\end{flushleft}


\begin{flushleft}
ELL786	 Multimedia Systems	
\end{flushleft}





\begin{flushleft}
Required Electives
\end{flushleft}


\begin{flushleft}
ELL785	 Computer Communication Networks 	
\end{flushleft}


\begin{flushleft}
ELL786	 Multimedia Systems	
\end{flushleft}





3	 0	 0	 3


3	0	0	3





\begin{flushleft}
Other Electives
\end{flushleft}


\begin{flushleft}
ELL704	 Advanced Robotics	
\end{flushleft}


3	0	0	3


\begin{flushleft}
ELL707	 Systems Biology	
\end{flushleft}


3	0	0	3


\begin{flushleft}
ELL715	 Digital Image Processing	
\end{flushleft}


3	 0	 2	 4


\begin{flushleft}
ELL741	 Neuromorphic Engineering	
\end{flushleft}


3	0	0	3


\begin{flushleft}
ELL785	 Computer Communication Networks 	
\end{flushleft}


3	 0	 0	 3


\begin{flushleft}
ELL787	 Embedded Systems and Applications	
\end{flushleft}


3	 0	 0	 3


\begin{flushleft}
ELL788	 Computational Perception and Cognition	
\end{flushleft}


3	 0	 0	 3


\begin{flushleft}
ELL789	 Intelligent Systems	
\end{flushleft}


3	0	0	3


\begin{flushleft}
ELL791	 Neural Systems and Learning Machines	
\end{flushleft}


3	 0	 2	 4


\begin{flushleft}
ELL793	 Computer Vision	
\end{flushleft}


3	0	0	3


\begin{flushleft}
ELL794	 Human-Computer Interface	
\end{flushleft}


3	0	0	3


\begin{flushleft}
ELL795	 Swarm Intelligence	
\end{flushleft}


3	0	0	3


\begin{flushleft}
ELL796	 Signals and Systems in Biology	
\end{flushleft}


3	 0	 0	 3


\begin{flushleft}
ELL798	 Agent Technologies	
\end{flushleft}


3	0	0	3


\begin{flushleft}
ELL799	 Natural Computing	
\end{flushleft}


3	0	0	3


\begin{flushleft}
ELL882	 Large-Scale Machine Learning	
\end{flushleft}


3	 0	 0	 3


\begin{flushleft}
ELL883	 Embedded Intelligence	
\end{flushleft}


3	0	0	3


\begin{flushleft}
ELL884	 Information Retrieval	
\end{flushleft}


3	0	0	3


\begin{flushleft}
ELL885	 Machine Learning for Computational Finance	3	0	0	3
\end{flushleft}


\begin{flushleft}
ELL886	 Big Data Systems	
\end{flushleft}


3	 0	 0	 3


\begin{flushleft}
ELL887	 Cloud Computing	
\end{flushleft}


3	0	0	3


\begin{flushleft}
ELL888	 Advanced Machine Learning	
\end{flushleft}


3	 0	 0	 3


\begin{flushleft}
ELL890	 Computational Neuroscience	
\end{flushleft}


3	0	0	3


\begin{flushleft}
ELL891	 Computational Linguistics	
\end{flushleft}


3	0	0	3


\begin{flushleft}
ELL893	 Cyber-Physical Systems	
\end{flushleft}


3	0	0	3


\begin{flushleft}
Streamed Electives (EET) in (Embedded Intelligent Systems)
\end{flushleft}


\begin{flushleft}
Required Electives
\end{flushleft}


\begin{flushleft}
ELL784	 Introduction to Machine Learning	
\end{flushleft}


\begin{flushleft}
ELL787	 Embedded Systems and Applications	
\end{flushleft}





3	 0	 0	 3


3	 0	 0	 3





\begin{flushleft}
Other Electives
\end{flushleft}


\begin{flushleft}
COL719	 Synthesis of Digital Systems	
\end{flushleft}


\begin{flushleft}
COL812	 System Level Design and Modelling	
\end{flushleft}


\begin{flushleft}
ELL704	 Advanced Robotics	
\end{flushleft}


\begin{flushleft}
ELL710	 Coding Theory	
\end{flushleft}


\begin{flushleft}
ELL720	 Advanced Digital Signal Processing	
\end{flushleft}


\begin{flushleft}
ELL728	 Optoelectronic Instrumentation	
\end{flushleft}


\begin{flushleft}
ELL731	 Mixed Signal Circuit Design	
\end{flushleft}


\begin{flushleft}
ELL733	 Digital ASIC Design	
\end{flushleft}


\begin{flushleft}
ELL734	 MOS VLSI design	
\end{flushleft}


\begin{flushleft}
ELL735	 Analog Integrated Circuits	
\end{flushleft}





3	 0	 2	 4


3	 0	 0	 3


3	0	0	3


3	0	0	3


3	 0	 0	 3


3	0	0	3


3	 0	 0	 3


3	0	2	4


3	 0	 0	 3


3	 0	 0	 3





\begin{flushleft}
Other Electives
\end{flushleft}


\begin{flushleft}
ELL710	 Coding Theory	
\end{flushleft}


\begin{flushleft}
ELL711	 Signal Theory	
\end{flushleft}


\begin{flushleft}
ELL712	 Digital Communications	
\end{flushleft}


\begin{flushleft}
ELL714	 Basic Information Theory	
\end{flushleft}


\begin{flushleft}
ELL716	 Telecommunication Switching and	
\end{flushleft}


\begin{flushleft}
	Transmission
\end{flushleft}


\begin{flushleft}
ELL717	 Optical Communication Systems	
\end{flushleft}


\begin{flushleft}
ELL723	 Broadband Communication Systems	
\end{flushleft}


\begin{flushleft}
ELL725	 Wireless Communications	
\end{flushleft}


\begin{flushleft}
ELL784	 Introduction to Machine Learning	
\end{flushleft}


\begin{flushleft}
ELL787	 Embedded Systems and Applications	
\end{flushleft}


\begin{flushleft}
ELL797	 Energy-Efficient Computing	
\end{flushleft}


\begin{flushleft}
ELL813	 Advanced Information Theory	
\end{flushleft}


\begin{flushleft}
ELL816	 Satellite Communication	
\end{flushleft}


\begin{flushleft}
ELL817	 Access Networks	
\end{flushleft}


\begin{flushleft}
ELL818	 Telecommunication Technologies	
\end{flushleft}


\begin{flushleft}
ELL820	 Photonic Switching and Networking	
\end{flushleft}


\begin{flushleft}
ELL887	 Cloud Computing	
\end{flushleft}


\begin{flushleft}
ELL889	 Protocol Engineering	
\end{flushleft}


\begin{flushleft}
ELL892	 Internet Technologies	
\end{flushleft}


\begin{flushleft}
ELL894	 Network Performance Modeling	
\end{flushleft}


	


\begin{flushleft}
and Analysis
\end{flushleft}


\begin{flushleft}
ELL895	 Network Security	
\end{flushleft}


\begin{flushleft}
ELL896	 Mobile Computing	
\end{flushleft}


\begin{flushleft}
ELL897	 Network Management	
\end{flushleft}


\begin{flushleft}
ELL898	 Pervasive Computing	
\end{flushleft}


\begin{flushleft}
ELP720	 Telecommunication Networks Laboratory	
\end{flushleft}


\begin{flushleft}
ELP780	Software Lab	
\end{flushleft}


\begin{flushleft}
ELP781	 Digital Systems Lab	
\end{flushleft}


\begin{flushleft}
ELP782	 Computer Networks Lab	
\end{flushleft}


\begin{flushleft}
ELP821	Advanced Telecommunication Networks	
\end{flushleft}


\begin{flushleft}
	Laboratory
\end{flushleft}


\begin{flushleft}
ELP822	 Network Software Laboratory	
\end{flushleft}





3	 0	 0	 3


3	0	0	3


3	0	0	3


3	0	0	3


3	0	0	3


3	0	0	3


3	 0	 0	 3


3	 0	 0	 3


3	 0	 0	 3


3	0	0	3


3	 0	 0	 3


3	 0	 0	 3


3	 0	 0	 3


3	0	0	3


3	0	0	3


3	0	0	3


3	0	0	3


3	 0	 0	 3


3	0	0	3


3	0	0	3


3	0	0	3


3	 0	 0	 3


3	0	0	3


3	0	0	3


3	0	0	3


3	0	0	3


0	 1	 4	 3


0	1	4	3


0	 1	 4	 3


0	 1	 4	 3


0	1	4	3


0	 1	 4	 3





\begin{flushleft}
Streamed Electives (EET) in (Multimedia Information Processing)
\end{flushleft}


\begin{flushleft}
Required Electives
\end{flushleft}


\begin{flushleft}
ELL786	 Multimedia Systems	
\end{flushleft}


\begin{flushleft}
ELL787	 Embedded Systems and Applications	
\end{flushleft}





3	0	0	3


3	 0	 0	 3





\begin{flushleft}
Other Electives
\end{flushleft}


\begin{flushleft}
ELL710	 Coding Theory	
\end{flushleft}





3	0	0	3





122





\begin{flushleft}
\newpage
Streamed Electives (EET) in (Internet Technologies)
\end{flushleft}


\begin{flushleft}
Required Electives
\end{flushleft}


\begin{flushleft}
ELL784	 Introduction to Machine Learning	
\end{flushleft}


\begin{flushleft}
ELL785	 Computer Communication Networks 	
\end{flushleft}





\begin{flushleft}
Courses
\end{flushleft}


\begin{flushleft}
(Number, Abbreviated Title, L-T-P, credits)
\end{flushleft}





\begin{flushleft}
Sem.
\end{flushleft}





\begin{flushleft}
ELL780
\end{flushleft}


\begin{flushleft}
I
\end{flushleft}





\begin{flushleft}
II
\end{flushleft}





3	 0	 0	 3


3	 0	 0	 3





\begin{flushleft}
Mathematical
\end{flushleft}


\begin{flushleft}
Foundations of
\end{flushleft}


\begin{flushleft}
Computer Technology
\end{flushleft}





\begin{flushleft}
ELL781
\end{flushleft}





(3-0-0)





\begin{flushleft}
Software
\end{flushleft}


\begin{flushleft}
Fundamentals for
\end{flushleft}


\begin{flushleft}
Computer
\end{flushleft}


\begin{flushleft}
Technology
\end{flushleft}





\begin{flushleft}
ELL783
\end{flushleft}





\begin{flushleft}
ELD780
\end{flushleft}





(3-0-2)





(0-0-4)





\begin{flushleft}
Operating Systems
\end{flushleft}





\begin{flushleft}
ELL782
\end{flushleft}





\begin{flushleft}
Computer
\end{flushleft}


\begin{flushleft}
Architecture
\end{flushleft}





\begin{flushleft}
PE-1
\end{flushleft}


(3-0-0)





(3-0-0)





\begin{flushleft}
L
\end{flushleft}





\begin{flushleft}
T
\end{flushleft}





\begin{flushleft}
P
\end{flushleft}





\begin{flushleft}
Total
\end{flushleft}





\begin{flushleft}
Credits
\end{flushleft}





\begin{flushleft}
Other Electives
\end{flushleft}


\begin{flushleft}
ELL723	 Broadband Communication Systems	
\end{flushleft}


3	 0	 0	 3


\begin{flushleft}
ELL765	 Smart Grid Technology	
\end{flushleft}


3	 0	 0	 3


\begin{flushleft}
ELL766	 Appliance Systems	
\end{flushleft}


3	0	0	3


\begin{flushleft}
ELL772	 Planning and Operation of a Smart Grid	
\end{flushleft}


3	 0	 0	 3


\begin{flushleft}
ELL786	 Multimedia Systems	
\end{flushleft}


3	0	0	3


\begin{flushleft}
ELL787	 Embedded Systems and Applications	
\end{flushleft}


3	 0	 0	 3


\begin{flushleft}
ELL797	 Energy-Efficient Computing	
\end{flushleft}


3	 0	 0	 3


\begin{flushleft}
ELL798	 Agent Technologies	
\end{flushleft}


3	0	0	3


\begin{flushleft}
ELL884	 Information Retrieval	
\end{flushleft}


3	0	0	3


\begin{flushleft}
ELL887	 Cloud Computing	
\end{flushleft}


3	0	0	3


\begin{flushleft}
ELL892	 Internet Technologies	
\end{flushleft}


3	0	0	3


\begin{flushleft}
ELL895	 Network Security	
\end{flushleft}


3	0	0	3


\begin{flushleft}
ELL896	 Mobile Computing	
\end{flushleft}


3	0	0	3


\begin{flushleft}
ELL898	 Pervasive Computing	
\end{flushleft}


3	0	0	3


\begin{flushleft}
ELP721	 Embedded Telecommunication Systems Laboratory	0	1	4	3
\end{flushleft}


\begin{flushleft}
ELP780	Software Lab	
\end{flushleft}


0	1	4	3


\begin{flushleft}
ELP781	 Digital Systems Lab	
\end{flushleft}


0	 1	 4	 3


\begin{flushleft}
ELP782	 Computer Networks Lab	
\end{flushleft}


0	 1	 4	 3


\begin{flushleft}
ELP855	 Smart Grids Laboratory 	
\end{flushleft}


0	 1	 4	 3





3	0	0	3


3	0	0	3


3	 0	 2	 4


3	 0	 0	 3


3	 0	 0	 3


3	 0	 0	 3


3	 0	 0	 3


3	 0	 0	 3


3	 0	 0	 3


3	 0	 0	 3


3	0	0	3


3	0	0	3


3	 0	 0	 3


3	 0	 0	 3





\begin{flushleft}
Lecture
\end{flushleft}


\begin{flushleft}
courses
\end{flushleft}





\begin{flushleft}
ELL711	 Signal Theory	
\end{flushleft}


\begin{flushleft}
ELL714	 Basic Information Theory	
\end{flushleft}


\begin{flushleft}
ELL715	 Digital Image Processing	
\end{flushleft}


\begin{flushleft}
ELL718	 Statistical Signal Processing	
\end{flushleft}


\begin{flushleft}
ELL719	 Detection and Estimation Theory	
\end{flushleft}


\begin{flushleft}
ELL720	 Advanced Digital Signal Processing	
\end{flushleft}


\begin{flushleft}
ELL784	 Introduction to Machine Learning	
\end{flushleft}


\begin{flushleft}
ELL785	 Computer Communication Networks 	
\end{flushleft}


\begin{flushleft}
ELL788	 Computational Perception and Cognition	
\end{flushleft}


\begin{flushleft}
ELL792	 Computer Graphics	
\end{flushleft}


\begin{flushleft}
ELL793	 Computer Vision	
\end{flushleft}


\begin{flushleft}
ELL813	 Advanced Information Theory	
\end{flushleft}


\begin{flushleft}
ELL882	 Large-Scale Machine Learning	
\end{flushleft}


\begin{flushleft}
CRL707	 Human \& Machine Speech Communication	
\end{flushleft}





5





15





0





0





15





15





4





9





0





6





15





12





2





6





0





12





18





12





4





12





0





0





12





12





0





0





0





24





24





12





2





6





0





12





18





12





\begin{flushleft}
Contact h/week
\end{flushleft}





\begin{flushleft}
PE-2
\end{flushleft}


(3-0-0)





(3-0-0)


\begin{flushleft}
Minor Project
\end{flushleft}





\begin{flushleft}
PE-3
\end{flushleft}


(3-0-0)





\begin{flushleft}
PE-4
\end{flushleft}


(3-0-0)





\begin{flushleft}
Summer: [PC-6] ELD880 Major Project Part 1 (for M.Tech with Dissertation)
\end{flushleft}


\begin{flushleft}
III
\end{flushleft}





\begin{flushleft}
(M.Tech. with
\end{flushleft}


\begin{flushleft}
DIssertation)
\end{flushleft}


\begin{flushleft}
OR
\end{flushleft}





\begin{flushleft}
III
\end{flushleft}





\begin{flushleft}
(M.Tech.
\end{flushleft}


\begin{flushleft}
without
\end{flushleft}


\begin{flushleft}
Dissertation)
\end{flushleft}





\begin{flushleft}
IV
\end{flushleft}





\begin{flushleft}
(M.Tech. with
\end{flushleft}


\begin{flushleft}
Dissertation)
\end{flushleft}


\begin{flushleft}
OR
\end{flushleft}





\begin{flushleft}
IV
\end{flushleft}





\begin{flushleft}
(M.Tech.
\end{flushleft}


\begin{flushleft}
without
\end{flushleft}


\begin{flushleft}
Dissertation)
\end{flushleft}





\begin{flushleft}
ELD880
\end{flushleft}





\begin{flushleft}
Major Project Part-I
\end{flushleft}





(0-0-12)


\begin{flushleft}
PE-5
\end{flushleft}


(3-0-0)





\begin{flushleft}
PE-5
\end{flushleft}


(3-0-0)





\begin{flushleft}
OE-1
\end{flushleft}


(3-0-0)





\begin{flushleft}
PE-6
\end{flushleft}


(3-0-0)





\begin{flushleft}
PE-7/OE-1
\end{flushleft}


(3-0-0)





\begin{flushleft}
PE-8/OE-2
\end{flushleft}


(3-0-0)





\begin{flushleft}
ELD881
\end{flushleft}





\begin{flushleft}
Major Project Part-II
\end{flushleft}





(0-0-24)


\begin{flushleft}
ELD880
\end{flushleft}





\begin{flushleft}
Major Project Part-I
\end{flushleft}





(0-0-12)





\begin{flushleft}
PE-7/OE-1
\end{flushleft}


(3-0-0)





\begin{flushleft}
PE-8/OE-2
\end{flushleft}


(3-0-0)





\begin{flushleft}
Total = 51
\end{flushleft}


123





\begin{flushleft}
\newpage
Master of Technology in Industrial Engineering
\end{flushleft}





\begin{flushleft}
Programme Code: MEE
\end{flushleft}





\begin{flushleft}
Department of Mechanical Engineering
\end{flushleft}


\begin{flushleft}
The overall credits structure
\end{flushleft}


\begin{flushleft}
Category
\end{flushleft}





\begin{flushleft}
PC
\end{flushleft}





\begin{flushleft}
PE
\end{flushleft}





\begin{flushleft}
OE
\end{flushleft}





\begin{flushleft}
Total
\end{flushleft}





\begin{flushleft}
Credits
\end{flushleft}





36





12





3





51





\begin{flushleft}
Program Core
\end{flushleft}





\begin{flushleft}
Streamed Electives (MEE) in (Product Life Cycle Management)
\end{flushleft}





\begin{flushleft}
MCD861	 M.Tech. Project Part-I 	
\end{flushleft}


0	 0	 24	12


\begin{flushleft}
MCD862	 M.Tech. Project Part-II 	
\end{flushleft}


0	 0	 24	12


\begin{flushleft}
MCL751	 Industrial Engineering Systems 	
\end{flushleft}


1	 0	 4	 3


\begin{flushleft}
MCL754	 Operations Planning and Control 	
\end{flushleft}


3	 0	 0	 3


\begin{flushleft}
MCL761	 Probability and Statistics 	
\end{flushleft}


3	 0	 0	 3


\begin{flushleft}
MCL765	 Operations Research 	
\end{flushleft}


3	 0	 0	 3


	


\begin{flushleft}
Total Credits				36
\end{flushleft}





\begin{flushleft}
MCL771	 Value Engineering and Life Cycle Costing 	
\end{flushleft}


\begin{flushleft}
MCL772	 Reliability Engineering 	
\end{flushleft}


\begin{flushleft}
MCL773	 Quality Systems 	
\end{flushleft}





\begin{flushleft}
Streamed Electives (MEE) in (Operations Management)
\end{flushleft}





\begin{flushleft}
MCL761
\end{flushleft}





\begin{flushleft}
MCL765
\end{flushleft}





(3-0-0) 3





(3-0-0) 3





\begin{flushleft}
Industrial
\end{flushleft}


\begin{flushleft}
Engineering
\end{flushleft}


\begin{flushleft}
Systems
\end{flushleft}





\begin{flushleft}
MCL754
\end{flushleft}





\begin{flushleft}
PE-2
\end{flushleft}


(3-0-0) 3





\begin{flushleft}
PE-3
\end{flushleft}


(3-0-0) 3





\begin{flushleft}
Probability and
\end{flushleft}


\begin{flushleft}
Statistics
\end{flushleft}





\begin{flushleft}
Operations Planning
\end{flushleft}


\begin{flushleft}
and Control
\end{flushleft}





\begin{flushleft}
Operations
\end{flushleft}


\begin{flushleft}
Research
\end{flushleft}





\begin{flushleft}
MCL751
\end{flushleft}





(1-0-4) 3





0	 3


0	 3


0	3


0	3


0	 3


0	 3


0	 3





\begin{flushleft}
L
\end{flushleft}





\begin{flushleft}
T
\end{flushleft}





\begin{flushleft}
P
\end{flushleft}





\begin{flushleft}
Total
\end{flushleft}





\begin{flushleft}
Credits
\end{flushleft}





\begin{flushleft}
II
\end{flushleft}





2	 4


0	3


0	 3


0	 3





\begin{flushleft}
Courses
\end{flushleft}


\begin{flushleft}
(Number, Abbreviated Title, L-T-P, credits)
\end{flushleft}





\begin{flushleft}
Sem.
\end{flushleft}





\begin{flushleft}
I
\end{flushleft}





3	 0	


3	0	


3	 0	


3	 0	





3	 0	


3	 0	


3	0	


3	0	


3	 0	


3	 0	


3	 0	





\begin{flushleft}
Lecture
\end{flushleft}


\begin{flushleft}
courses
\end{flushleft}





\begin{flushleft}
MCL755	 Service System Design 	
\end{flushleft}


\begin{flushleft}
MCL756	 Supply Chain Management 	
\end{flushleft}


\begin{flushleft}
MCL757	Logistics 	
\end{flushleft}


\begin{flushleft}
MCL759	Entrepreneurship 	
\end{flushleft}


\begin{flushleft}
MCL760	 Project Management 	
\end{flushleft}


\begin{flushleft}
MCL775	 Special Topics in IE 	
\end{flushleft}


\begin{flushleft}
MCL866	 Maintenance management 	
\end{flushleft}





\begin{flushleft}
Streamed Electives (MEE) in (Analytics and Optimization)
\end{flushleft}


\begin{flushleft}
MCL753	 Manufacturing Informatics 	
\end{flushleft}


\begin{flushleft}
MCL758	Optimization 	
\end{flushleft}


\begin{flushleft}
MCL770	 Stochastic Modeling and Simulation 	
\end{flushleft}


\begin{flushleft}
MCL865	 Advanced Operations Research 	
\end{flushleft}





3	 0	 0	 3


3	 0	 0	 3


3	 0	 0	 3





4





10





0





4





14





12





4





12





0





0





12





12





1





3





0





24





27





15





0





0





0





24





24





12





\begin{flushleft}
Contact h/week
\end{flushleft}





\begin{flushleft}
PE-1
\end{flushleft}





\begin{flushleft}
(from PLM
\end{flushleft}


\begin{flushleft}
Stream)
\end{flushleft}





(3-0-0) 3


\begin{flushleft}
PE-4
\end{flushleft}


(3-0-0) 3





(3-0-0) 3


\begin{flushleft}
Professional Project Activity In Summer Vacation
\end{flushleft}





\begin{flushleft}
III
\end{flushleft}





\begin{flushleft}
IV
\end{flushleft}





\begin{flushleft}
MCD861
\end{flushleft}





\begin{flushleft}
Major Porject Part-I
\end{flushleft}





(0-0-24) 12





\begin{flushleft}
OE-1
\end{flushleft}


(3-0-0) 3





\begin{flushleft}
MCD862
\end{flushleft}





\begin{flushleft}
Major Porject Part-II
\end{flushleft}





(0-0-24) 12





\begin{flushleft}
Total = 51
\end{flushleft}


124





\begin{flushleft}
\newpage
Master of Technology in Mechanical Design
\end{flushleft}





\begin{flushleft}
Programme Code: MEM
\end{flushleft}





\begin{flushleft}
Department of Mechanical Engineering
\end{flushleft}


\begin{flushleft}
The overall credits structure
\end{flushleft}


\begin{flushleft}
Category
\end{flushleft}





\begin{flushleft}
PC
\end{flushleft}





\begin{flushleft}
PE
\end{flushleft}





\begin{flushleft}
OE
\end{flushleft}





\begin{flushleft}
Total
\end{flushleft}





\begin{flushleft}
Credits
\end{flushleft}





32





22





0





54





\begin{flushleft}
Program Core
\end{flushleft}





\begin{flushleft}
MCL738	Dynamics of Multibody Systems	
\end{flushleft}


\begin{flushleft}
MCL740	Lubrication	
\end{flushleft}


\begin{flushleft}
MCL743	 Plant Equipment Design	
\end{flushleft}


\begin{flushleft}
MCL744	 Design for Manufacture and Assembly	
\end{flushleft}


\begin{flushleft}
MCL746	 Design for Noise Vibration and Harshness	
\end{flushleft}


\begin{flushleft}
MCL747	 Design of Precision Machines	
\end{flushleft}


\begin{flushleft}
MCL749	 Mechatronics Product Design	
\end{flushleft}


\begin{flushleft}
MCL777	 Machine Tool Design	
\end{flushleft}


\begin{flushleft}
MCL834	Vibroacoustics	
\end{flushleft}


\begin{flushleft}
MCL837	 Advanced Mechanisms	
\end{flushleft}


\begin{flushleft}
MCL838	 Rotor Dynamics	
\end{flushleft}


\begin{flushleft}
MCL840	 Experimental Modal Analysis and Dynamic	
\end{flushleft}


\begin{flushleft}
	Design
\end{flushleft}


\begin{flushleft}
MCL845	 Advanced Robotics	
\end{flushleft}


\begin{flushleft}
MCL848	 Special topics in Systems Design-I	
\end{flushleft}


\begin{flushleft}
MCL849	 Special topics in Systems Design-II	
\end{flushleft}


\begin{flushleft}
MCS830	 Independent Study	
\end{flushleft}


\begin{flushleft}
MCV849	 Special Module in Systems Design	
\end{flushleft}





\begin{flushleft}
MCL730	 Designing with advanced materials	
\end{flushleft}


\begin{flushleft}
MCL733	 Vibration and Noise	
\end{flushleft}


\begin{flushleft}
MCL736	 Automotive Design	
\end{flushleft}


\begin{flushleft}
MCL741	 Control Engineering	
\end{flushleft}


\begin{flushleft}
MCL745	Robotics	
\end{flushleft}


\begin{flushleft}
MCL748	 Tribological Systems Design	
\end{flushleft}





3	 0	 2	 4


3	 0	 2	 4


3	 0	 2	 4


3	 0	 2	 4


3	 0	2	 4


3	 0	 2	 4





\begin{flushleft}
Streamed Electives MEM - (A2) (atleast 10 credits)
\end{flushleft}


\begin{flushleft}
MCL728	Nanotribology	
\end{flushleft}


3	 0	0	 3


\begin{flushleft}
MCL729	Nanomechanics	
\end{flushleft}


2	 0	2	 3


\begin{flushleft}
MCL737	 Biomechanics of trauma and automotive design	 3	 0	0	 3
\end{flushleft}





\begin{flushleft}
Courses
\end{flushleft}


\begin{flushleft}
(Number, Abbreviated Title, L-T-P, credits)
\end{flushleft}





\begin{flushleft}
Sem.
\end{flushleft}





\begin{flushleft}
I
\end{flushleft}





\begin{flushleft}
II
\end{flushleft}





\begin{flushleft}
APL701
\end{flushleft}





\begin{flushleft}
MCL731
\end{flushleft}





(3-0-0) 3





(3-0-0) 3





\begin{flushleft}
CAD and
\end{flushleft}


\begin{flushleft}
Finite Element
\end{flushleft}


\begin{flushleft}
Analysis
\end{flushleft}





\begin{flushleft}
PE-1
\end{flushleft}


(3-0-2) 4





\begin{flushleft}
PE-2
\end{flushleft}


(3-0-2) 4





\begin{flushleft}
PE-3
\end{flushleft}


(3-0-2) 4





\begin{flushleft}
MCD831
\end{flushleft}





\begin{flushleft}
PE-5
\end{flushleft}


(3-0-2) 4





\begin{flushleft}
PE-6
\end{flushleft}


(3-0-0) 3





\begin{flushleft}
Continuum
\end{flushleft}


\begin{flushleft}
Mechanics
\end{flushleft}





\begin{flushleft}
Analytical
\end{flushleft}


\begin{flushleft}
Dynamics
\end{flushleft}





\begin{flushleft}
MCL735
\end{flushleft}





(3-0-2) 4





2	


2	


3	


0	


1	





0	


0	


0	


3	


0	





2	


0	


0	


0	


0	





3


2


3


3


1





\begin{flushleft}
L
\end{flushleft}





\begin{flushleft}
T
\end{flushleft}





\begin{flushleft}
P
\end{flushleft}





\begin{flushleft}
Total
\end{flushleft}





\begin{flushleft}
Credits
\end{flushleft}





\begin{flushleft}
Streamed Electives MEM - (A1) (atleast 12 credits)
\end{flushleft}





2	 0	 2	 3


3	 0	0	 3


3	 0	 0	 3


3	 0	 2	 4


3	 0	 2	 4


2	 0	 2	 3


3	 0	 2	 4


3	 0	 2	 4


2	 0	2	 3


2	 0	 2	 3


3	 0	 2	 4


2	 0	 2	 3





\begin{flushleft}
Lecture
\end{flushleft}


\begin{flushleft}
courses
\end{flushleft}





\begin{flushleft}
APL701	 Continuum Mechanics	
\end{flushleft}


3	 0	 0	 3


\begin{flushleft}
MCD831	 Major Project Part-I	
\end{flushleft}


0	 0	 12	 6


\begin{flushleft}
MCD832	 Major Project Part-II	
\end{flushleft}


0	 0	 24	 12


\begin{flushleft}
MCL731	 Analytical Dynamics	
\end{flushleft}


3	 0	 0	 3


\begin{flushleft}
MCL735	 CAD and Finite Element Analysis 	
\end{flushleft}


3	 0	 2	 4


\begin{flushleft}
MCL742	 Design \& Optimization	
\end{flushleft}


3	 0	 2	 4


	


\begin{flushleft}
Total Credits				 32
\end{flushleft}





4





12





0





4





16





14





4





12





0





6





18





15





2





6





0





14





0





13





0





0





0





24





24





12





\begin{flushleft}
Contact h/week
\end{flushleft}





\begin{flushleft}
MCL742
\end{flushleft}





\begin{flushleft}
Design \&
\end{flushleft}


\begin{flushleft}
Optimization
\end{flushleft}





(3-0-2) 4


\begin{flushleft}
PE-4
\end{flushleft}


(3-0-0) 3





\begin{flushleft}
Summer
\end{flushleft}





\begin{flushleft}
III
\end{flushleft}





\begin{flushleft}
IV
\end{flushleft}





\begin{flushleft}
Major Project Part-I
\end{flushleft}





(0-0-12) 6


\begin{flushleft}
MCD832
\end{flushleft}





\begin{flushleft}
Major Project Part-II
\end{flushleft}





(0-0-24) 12





\begin{flushleft}
Total = 54
\end{flushleft}


125





\begin{flushleft}
\newpage
Master of Technology in Production Engineering
\end{flushleft}





\begin{flushleft}
Programme Code: MEP
\end{flushleft}





\begin{flushleft}
Department of Mechanical Engineering
\end{flushleft}


\begin{flushleft}
The overall credits structure
\end{flushleft}


\begin{flushleft}
Category
\end{flushleft}





\begin{flushleft}
PC
\end{flushleft}





\begin{flushleft}
PE
\end{flushleft}





\begin{flushleft}
OE
\end{flushleft}





\begin{flushleft}
Total
\end{flushleft}





\begin{flushleft}
Credits
\end{flushleft}





31





18





0





49





\begin{flushleft}
Program Core
\end{flushleft}





\begin{flushleft}
MCL751	 Industrial Engineering Systems 	
\end{flushleft}


\begin{flushleft}
MCL753	Manufacturing Informatics 	
\end{flushleft}


\begin{flushleft}
MCL754	 Operations Planning and Control 	
\end{flushleft}


\begin{flushleft}
MCL773	Quality Systems 	
\end{flushleft}


\begin{flushleft}
MCL776	 Advances in Metal Forming	
\end{flushleft}


\begin{flushleft}
MCL777	Machine Tool Design	
\end{flushleft}


\begin{flushleft}
MCL778	 Design and Metallurgy of Welded Joints	
\end{flushleft}


\begin{flushleft}
MCL780	Casting Technology	
\end{flushleft}


\begin{flushleft}
MCL783	Automation in Manufacturing	
\end{flushleft}


\begin{flushleft}
MCL785	Advanced Machining Processes	
\end{flushleft}


\begin{flushleft}
MCL788	Surface Engineering	
\end{flushleft}


\begin{flushleft}
MCL791	 Processing and Mechanics of Composite	
\end{flushleft}


\begin{flushleft}
	Materials
\end{flushleft}


\begin{flushleft}
MCL792	 Injection Molding and Mold Design	
\end{flushleft}


\begin{flushleft}
MCL796	Additive Manufacturing	
\end{flushleft}


\begin{flushleft}
MCP790	Process Engineering	
\end{flushleft}





\begin{flushleft}
Total Credits				31
\end{flushleft}





	





\begin{flushleft}
Program Electives
\end{flushleft}


\begin{flushleft}
MCD882	 Major Project Part-II	
\end{flushleft}


\begin{flushleft}
MCL729	Nanomechanics 	
\end{flushleft}


\begin{flushleft}
MCL749	Mechatronics Product Design	
\end{flushleft}


\begin{flushleft}
MCL750	 Product Design and Manufacturing	
\end{flushleft}





\begin{flushleft}
Courses
\end{flushleft}


\begin{flushleft}
(Number, Abbreviated Title, L-T-P, credits)
\end{flushleft}





\begin{flushleft}
Sem.
\end{flushleft}





\begin{flushleft}
MCL781
\end{flushleft}


\begin{flushleft}
I
\end{flushleft}





\begin{flushleft}
II
\end{flushleft}





0	 0	 24	12


3	0	0	3


3	0	2	4


1	 0	 4	 3





\begin{flushleft}
Machining Processes
\end{flushleft}


\begin{flushleft}
and Analysis
\end{flushleft}





\begin{flushleft}
MCL787
\end{flushleft}





(3-0-2) 4


\begin{flushleft}
MCL705
\end{flushleft}





\begin{flushleft}
MCL784
\end{flushleft}





\begin{flushleft}
MCL786
\end{flushleft}





(3-0-2) 4





(2-0-2) 3





\begin{flushleft}
Experimental
\end{flushleft}


\begin{flushleft}
Methods
\end{flushleft}





(3-0-2) 4





\begin{flushleft}
CAM
\end{flushleft}





2	 0	 2	 3


3	0	2	4


2	0	4	4





\begin{flushleft}
L
\end{flushleft}





\begin{flushleft}
T
\end{flushleft}





\begin{flushleft}
P
\end{flushleft}





\begin{flushleft}
Total
\end{flushleft}





3





9





0





6





15





12





4





10





0





6





16





13





2





6





0





12





18





12





0





0





0





24





24





12





\begin{flushleft}
Contact h/week
\end{flushleft}





\begin{flushleft}
MCL769
\end{flushleft}





\begin{flushleft}
Welding
\end{flushleft}


\begin{flushleft}
Science and
\end{flushleft}


\begin{flushleft}
Technology
\end{flushleft}





(3-0-2) 4





1	 0	 4	 3


3	0	2	4


3	 0	 0	 3


3	0	0	3


3	 0	 0	 3


3	0	2	4


3	 0	 2	 4


3	0	2	4


3	0	2	4


3	0	0	3


3	0	2	4


3	 0	 2	 4





\begin{flushleft}
Credits
\end{flushleft}





0	0	12	


6


3	0	2	4


3	0	2	4


3	0	2	4


2	0	0	2


3	0	2	4


2	0	2	3


3	0	2	4





\begin{flushleft}
Lecture
\end{flushleft}


\begin{flushleft}
courses
\end{flushleft}





\begin{flushleft}
MCD881	 Major Project Part-I	
\end{flushleft}


\begin{flushleft}
MCL705	Experimental Methods	
\end{flushleft}


\begin{flushleft}
MCL769	Metal Forming Analysis	
\end{flushleft}


\begin{flushleft}
MCL781	Machining Processes and Analysis	
\end{flushleft}


\begin{flushleft}
MCL782	Computational Methods	
\end{flushleft}


\begin{flushleft}
MCL784	Computer Aided Manufacturing	
\end{flushleft}


\begin{flushleft}
MCL786	Metrology 	
\end{flushleft}


\begin{flushleft}
MCL787	Welding Science and Technology	
\end{flushleft}





\begin{flushleft}
Metal
\end{flushleft}


\begin{flushleft}
Forming Analysis
\end{flushleft}





(3-0-2) 4


\begin{flushleft}
Metrology
\end{flushleft}





\begin{flushleft}
MCL782
\end{flushleft}





\begin{flushleft}
Computational
\end{flushleft}


\begin{flushleft}
Methods
\end{flushleft}





(2-0-0) 3





\begin{flushleft}
Professional Project Activity In Summer Vacation
\end{flushleft}





\begin{flushleft}
MCD881
\end{flushleft}


\begin{flushleft}
III
\end{flushleft}





\begin{flushleft}
Major Project
\end{flushleft}


\begin{flushleft}
Part-I (Core)
\end{flushleft}





\begin{flushleft}
PE-1
\end{flushleft}


(3-0-0) 3





\begin{flushleft}
PE-2
\end{flushleft}


(3-0-0) 3





(0-0-12) 6


\begin{flushleft}
MCD882
\end{flushleft}


\begin{flushleft}
IV
\end{flushleft}





\begin{flushleft}
Major Project Part-II
\end{flushleft}


\begin{flushleft}
(PE)
\end{flushleft}





(0-0-24) 12





\begin{flushleft}
Total = 49
\end{flushleft}


126





\begin{flushleft}
\newpage
Programme Code: MET
\end{flushleft}





\begin{flushleft}
Master of Technology in Thermal Engineering
\end{flushleft}


\begin{flushleft}
Department of Mechanical Engineering
\end{flushleft}


\begin{flushleft}
The overall credits structure
\end{flushleft}


\begin{flushleft}
Category
\end{flushleft}





\begin{flushleft}
PC
\end{flushleft}





\begin{flushleft}
PE
\end{flushleft}





\begin{flushleft}
OE
\end{flushleft}





\begin{flushleft}
Total
\end{flushleft}





\begin{flushleft}
Credits
\end{flushleft}





36





12





3





51





\begin{flushleft}
Non-graded Core
\end{flushleft}





\begin{flushleft}
MCL812	Combustion 	
\end{flushleft}


\begin{flushleft}
MCL813	 Computational Heat Transfer 	
\end{flushleft}


\begin{flushleft}
MCL814	 Convective Heat Transfer 	
\end{flushleft}


\begin{flushleft}
MCL815	 Fire Dynamics and Engineering 	
\end{flushleft}


\begin{flushleft}
MCL816	 Gas Dynamics 	
\end{flushleft}


\begin{flushleft}
MCL817	 Heat Exchangers 	
\end{flushleft}


\begin{flushleft}
MCL818	 Heating, Ventilating and Air-conditioning	
\end{flushleft}


\begin{flushleft}
MCL819	 Lattice Boltzmann method 	
\end{flushleft}


\begin{flushleft}
MCL820	 Micro/nano scale heat transfer 	
\end{flushleft}


\begin{flushleft}
MCL821	 Radiative Heat Transfer 	
\end{flushleft}


\begin{flushleft}
MCL822	 Steam and Gas Turbines 	
\end{flushleft}


\begin{flushleft}
MCL823	 Thermal Design 	
\end{flushleft}


\begin{flushleft}
MCL824	Turbocompressors 	
\end{flushleft}


\begin{flushleft}
MCL825	 Design of Wind Power Farms 	
\end{flushleft}





\begin{flushleft}
Program Core
\end{flushleft}


\begin{flushleft}
MCD811	 Major Project Part-I (Thermal Engineering)	 0	 0	 16	8
\end{flushleft}


\begin{flushleft}
MCD812	 Major Project Part-II (Thermal Engineering)	 0	 0	 24	12
\end{flushleft}


\begin{flushleft}
MCL701	 Advanced Thermodynamics	
\end{flushleft}


3	 0	 0	 3


\begin{flushleft}
MCL702	 Advanced Fluid Mechanics	
\end{flushleft}


3	 0	 0	 3


\begin{flushleft}
MCL703	 Advanced Heat and Mass Transfer	
\end{flushleft}


3	 0	 0	 3


\begin{flushleft}
MCL704	 Applied Mathematics for Thermofluids	
\end{flushleft}


3	 0	 0	 3


\begin{flushleft}
MCL705	Experimental Methods	
\end{flushleft}


3	 0	2	 4


	


\begin{flushleft}
Total Credits				 36
\end{flushleft}


\begin{flushleft}
Program Electives
\end{flushleft}


\begin{flushleft}
MCL811	 Advanced Power Generation Systems 	
\end{flushleft}





\begin{flushleft}
Courses
\end{flushleft}


\begin{flushleft}
(Number, Abbreviated Title, L-T-P, credits)
\end{flushleft}





\begin{flushleft}
Sem.
\end{flushleft}





\begin{flushleft}
MCL701
\end{flushleft}


\begin{flushleft}
I
\end{flushleft}





\begin{flushleft}
II
\end{flushleft}


\begin{flushleft}
Summer
\end{flushleft}





\begin{flushleft}
Adv.
\end{flushleft}


\begin{flushleft}
Thermodynamics
\end{flushleft}





\begin{flushleft}
MCL702
\end{flushleft}





\begin{flushleft}
Adv. Fluid
\end{flushleft}


\begin{flushleft}
Mechanics
\end{flushleft}





\begin{flushleft}
MCL703
\end{flushleft}





\begin{flushleft}
Adv. Heat
\end{flushleft}


\begin{flushleft}
\& Mass Transfer
\end{flushleft}





(3-0-0) 3





(3-0-0) 3





(3-0-0) 3





\begin{flushleft}
MCL705
\end{flushleft}





\begin{flushleft}
PE-1
\end{flushleft}


(3-0-0) 3





\begin{flushleft}
PE-2
\end{flushleft}


(3-0-0) 3





\begin{flushleft}
Exptl Methods
\end{flushleft}





(3-0-2) 4





\begin{flushleft}
Major Project Part-I
\end{flushleft}


\begin{flushleft}
(MET)
\end{flushleft}





\begin{flushleft}
PE-4
\end{flushleft}


(3-0-0) 3





\begin{flushleft}
Contact h/week
\end{flushleft}





0	


0	


0	


0	


0	


0	


0	


0	


0	


0	


0	


0	


0	


0	





0	


2	


0	


4	


2	


0	


0	


0	


2	


0	


2	


2	


0	


2	





3


4


3


4


4


3


3


3


4


3


4


4


3


4





\begin{flushleft}
L
\end{flushleft}





\begin{flushleft}
T
\end{flushleft}





\begin{flushleft}
P
\end{flushleft}





\begin{flushleft}
Total
\end{flushleft}





4





12





0





0





12





12





4





12





0





2





14





13





\begin{flushleft}
MCL704
\end{flushleft}





\begin{flushleft}
Applied Math.
\end{flushleft}





(3-0-0) 3


\begin{flushleft}
PE-3
\end{flushleft}


(3-0-0) 3





0





\begin{flushleft}
Professional Project Activity (compulsory audit)
\end{flushleft}





\begin{flushleft}
MED811
\end{flushleft}


\begin{flushleft}
III
\end{flushleft}





3	 0	 0	 3





3	


3	


3	


2	


3	


3	


3	


3	


3	


3	


3	


3	


3	


3	





\begin{flushleft}
Credits
\end{flushleft}





0	 0	 6	 3





\begin{flushleft}
Lecture
\end{flushleft}


\begin{flushleft}
courses
\end{flushleft}





\begin{flushleft}
MCD800	 Professional Project Activity 	
\end{flushleft}





\begin{flushleft}
OE-1
\end{flushleft}


(3-0-0) 3





0





2





6





0





12





18





14





0





0





0





24





24





12





(0-0-16) 8


\begin{flushleft}
MED812
\end{flushleft}


\begin{flushleft}
IV
\end{flushleft}





\begin{flushleft}
Major Project Part-II
\end{flushleft}


\begin{flushleft}
(MET)
\end{flushleft}





(0-0-24) 12





\begin{flushleft}
Total = 51
\end{flushleft}


127





\begin{flushleft}
\newpage
Programme Code: PHA
\end{flushleft}





\begin{flushleft}
Master of Technology in Applied Optics
\end{flushleft}


\begin{flushleft}
Department of Physics
\end{flushleft}





\begin{flushleft}
The overall credits structure
\end{flushleft}


\begin{flushleft}
PE
\end{flushleft}





\begin{flushleft}
OE
\end{flushleft}





\begin{flushleft}
Total
\end{flushleft}





39





9





3





51





\begin{flushleft}
Program Core
\end{flushleft}


\begin{flushleft}
PYD801	
\end{flushleft}


\begin{flushleft}
PYD802	
\end{flushleft}


\begin{flushleft}
PYL751	
\end{flushleft}


\begin{flushleft}
PYL752	
\end{flushleft}


\begin{flushleft}
PYL753	
\end{flushleft}


\begin{flushleft}
PYL755	
\end{flushleft}


\begin{flushleft}
PYL756	
\end{flushleft}


\begin{flushleft}
PYP761	
\end{flushleft}


\begin{flushleft}
PYP762	
\end{flushleft}


	





\begin{flushleft}
PYL760	
\end{flushleft}


\begin{flushleft}
PYL770	
\end{flushleft}


\begin{flushleft}
PYL771	
\end{flushleft}


\begin{flushleft}
PYL772	
\end{flushleft}


\begin{flushleft}
PYL780	
\end{flushleft}


\begin{flushleft}
PYL791	
\end{flushleft}


\begin{flushleft}
PYL792	
\end{flushleft}


\begin{flushleft}
PYL795	
\end{flushleft}


\begin{flushleft}
PYL858	
\end{flushleft}


\begin{flushleft}
PYL879	
\end{flushleft}


\begin{flushleft}
PYL881	
\end{flushleft}


\begin{flushleft}
PYL882	
\end{flushleft}


\begin{flushleft}
PYL883	
\end{flushleft}


\begin{flushleft}
PYL892	
\end{flushleft}


\begin{flushleft}
PYP763	
\end{flushleft}


\begin{flushleft}
PYP764	
\end{flushleft}


\begin{flushleft}
PYS855	
\end{flushleft}





\begin{flushleft}
Major Project Part-I	
\end{flushleft}


0	 0	 12	 6


\begin{flushleft}
Major Project Part-II	
\end{flushleft}


0	 0	 24	 12


\begin{flushleft}
Optical sources, photometry and metrology	 3	 0	 0	 3
\end{flushleft}


\begin{flushleft}
Laser systems and applications	
\end{flushleft}


3	 0	 0	 3


\begin{flushleft}
Optical systems design	
\end{flushleft}


3	 0	 0	 3


\begin{flushleft}
Basic optics and optical instrumentation	
\end{flushleft}


3	 0	 0	 3


\begin{flushleft}
Fourier optics and holography	
\end{flushleft}


3	 0	 0	 3


\begin{flushleft}
Optical fabrication and metrology laboratory	 0	 0	 6	 3
\end{flushleft}


\begin{flushleft}
Advanced optics laboratory 	
\end{flushleft}


0	 0	 6	 3


\begin{flushleft}
Total Credits				 39
\end{flushleft}





\begin{flushleft}
Program Electives
\end{flushleft}


\begin{flushleft}
PYL757	 Statistical and Quantum optics 	
\end{flushleft}


3	 0	 0	 3


\begin{flushleft}
PYL758	 Advanced Quantum optics and applications	 3	 0	 0	 3
\end{flushleft}


\begin{flushleft}
PYL759	 Computational optical imaging	
\end{flushleft}


3	 0	 0	 3





\begin{flushleft}
Courses
\end{flushleft}


\begin{flushleft}
(Number, Abbreviated Title, L-T-P, credits)
\end{flushleft}





\begin{flushleft}
Sem.
\end{flushleft}





\begin{flushleft}
I
\end{flushleft}





\begin{flushleft}
II
\end{flushleft}





\begin{flushleft}
Biomedical optics and Bio-photonics	
\end{flushleft}


\begin{flushleft}
Ultra-fast optics and applications	
\end{flushleft}


\begin{flushleft}
Green Photonics	
\end{flushleft}


\begin{flushleft}
Plasmonic sensors	
\end{flushleft}


\begin{flushleft}
Diffractive and micro optics	
\end{flushleft}


\begin{flushleft}
Fiber Optics	
\end{flushleft}


\begin{flushleft}
Optical Electronics	
\end{flushleft}


\begin{flushleft}
Optics and Lasers	
\end{flushleft}


\begin{flushleft}
Advanced Holographic techniques	
\end{flushleft}


\begin{flushleft}
Selected Topics in Applied Optics	
\end{flushleft}


\begin{flushleft}
Selected Topics -- I 	
\end{flushleft}


\begin{flushleft}
Selected Topics -- II 	
\end{flushleft}


\begin{flushleft}
Minor Project 	
\end{flushleft}


\begin{flushleft}
Guided Wave Photonic Sensors 	
\end{flushleft}


\begin{flushleft}
Computational Optics laboratory	
\end{flushleft}


\begin{flushleft}
Advanced Optical Workshop	
\end{flushleft}


\begin{flushleft}
Independent Study	
\end{flushleft}





\begin{flushleft}
PYL755
\end{flushleft}





\begin{flushleft}
PYL751
\end{flushleft}





(3-0-0) 3





(3-0-0) 3





\begin{flushleft}
PYL752
\end{flushleft}





\begin{flushleft}
PYL756
\end{flushleft}





\begin{flushleft}
Basic Optics
\end{flushleft}


\begin{flushleft}
and Optical
\end{flushleft}


\begin{flushleft}
Instrumentation
\end{flushleft}





\begin{flushleft}
Laser Systems
\end{flushleft}


\begin{flushleft}
and Application
\end{flushleft}





\begin{flushleft}
Optical Sources,
\end{flushleft}


\begin{flushleft}
Photometry and
\end{flushleft}


\begin{flushleft}
Metrology
\end{flushleft}





\begin{flushleft}
Fourier Optics
\end{flushleft}


\begin{flushleft}
and Holography
\end{flushleft}





(3-0-0) 3





(3-0-0) 3





\begin{flushleft}
OE-1
\end{flushleft}


(3-0-0) 3





\begin{flushleft}
PYD801
\end{flushleft}





\begin{flushleft}
PYL753
\end{flushleft}





\begin{flushleft}
Optical Systems
\end{flushleft}


\begin{flushleft}
Design
\end{flushleft}





(3-0-0) 3





\begin{flushleft}
PYP761
\end{flushleft}





\begin{flushleft}
Optical
\end{flushleft}


\begin{flushleft}
Fabrication
\end{flushleft}


\begin{flushleft}
and Metrology
\end{flushleft}


\begin{flushleft}
Laboratory
\end{flushleft}





3	


3	


3	


3	


3	


3	


3	


3	


3	


3	


1	


1	


0	


3	


0	


0	


0	





0	


0	


0	


0	


0	


0	


0	


0	


0	


0	


0	


0	


0	


0	


0	


0	


3	





\begin{flushleft}
Contact h/week
\end{flushleft}





0	


0	


0	


0	


0	


0	


0	


0	


0	


0	


0	


0	


6	


0	


6	


6	


0	





3


3


3


3


3


3


3


3


3


3


1


1


3


3


3


3


3





\begin{flushleft}
L
\end{flushleft}





\begin{flushleft}
T
\end{flushleft}





\begin{flushleft}
P
\end{flushleft}





\begin{flushleft}
Total
\end{flushleft}





4





12





0





6





18





15





4





12





0





6





18





15





1





3





0





12





15





9





0





0





0





24





24





12





\begin{flushleft}
Credits
\end{flushleft}





\begin{flushleft}
PC
\end{flushleft}





\begin{flushleft}
Credits
\end{flushleft}





\begin{flushleft}
Lecture
\end{flushleft}


\begin{flushleft}
courses
\end{flushleft}





\begin{flushleft}
Category
\end{flushleft}





\begin{flushleft}
PE-1
\end{flushleft}


(3-0-0) 3





(0-0-6) 3


\begin{flushleft}
PYP762
\end{flushleft}





\begin{flushleft}
Advanced Optics
\end{flushleft}


\begin{flushleft}
Laboratory
\end{flushleft}





\begin{flushleft}
PE-2
\end{flushleft}


(3-0-0) 3





\begin{flushleft}
PE-3
\end{flushleft}


(3-0-0) 3





(0-0-6) 3





\begin{flushleft}
Summer
\end{flushleft}


\begin{flushleft}
III
\end{flushleft}





\begin{flushleft}
IV
\end{flushleft}





\begin{flushleft}
Maj. Proj. Part-I
\end{flushleft}





(0-0-12) 6





\begin{flushleft}
PYD802
\end{flushleft}





\begin{flushleft}
Maj. Proj. Part-II
\end{flushleft}





(0-0-24) 12





\begin{flushleft}
Total = 51
\end{flushleft}


128





\begin{flushleft}
\newpage
Master of Technology in Solid State Materials
\end{flushleft}





\begin{flushleft}
Programme Code: PHM
\end{flushleft}





\begin{flushleft}
Department of Physics
\end{flushleft}





\begin{flushleft}
The overall credits structure
\end{flushleft}


\begin{flushleft}
Category
\end{flushleft}





\begin{flushleft}
PC
\end{flushleft}





\begin{flushleft}
PE
\end{flushleft}





\begin{flushleft}
OE
\end{flushleft}





\begin{flushleft}
Total
\end{flushleft}





\begin{flushleft}
Credits
\end{flushleft}





39





9





3





51





\begin{flushleft}
Program Core
\end{flushleft}





\begin{flushleft}
Program Electives
\end{flushleft}





\begin{flushleft}
PYD801	
\end{flushleft}


\begin{flushleft}
PYD802	
\end{flushleft}


\begin{flushleft}
PYL701	
\end{flushleft}


\begin{flushleft}
PYL702	
\end{flushleft}


\begin{flushleft}
PYL703	
\end{flushleft}


\begin{flushleft}
PYL704	
\end{flushleft}


\begin{flushleft}
PYL705	
\end{flushleft}


\begin{flushleft}
PYP701	
\end{flushleft}


\begin{flushleft}
PYP702	
\end{flushleft}


	





\begin{flushleft}
PYL707 	 Characterization Techniques for Materials 	
\end{flushleft}


\begin{flushleft}
PYL723 	 Vacuum Science and Cryogenics 	
\end{flushleft}


\begin{flushleft}
PYL724	 Advances in Spintronics	
\end{flushleft}


\begin{flushleft}
PYL725	 Surface Physics and Analysis 	
\end{flushleft}


\begin{flushleft}
PYL726	 Semiconductor Device Technology 	
\end{flushleft}


\begin{flushleft}
PYL727	 Energy Materials and Devices 	
\end{flushleft}


\begin{flushleft}
PYL728 	 Quantum Heterostructures	
\end{flushleft}


\begin{flushleft}
PYL729	Nanoprobe Techniques	
\end{flushleft}


\begin{flushleft}
PYV759 	 Selected Topics in Solid State Materials 	
\end{flushleft}





\begin{flushleft}
PYL701
\end{flushleft}


\begin{flushleft}
I
\end{flushleft}





\begin{flushleft}
Physical
\end{flushleft}


\begin{flushleft}
Foundations
\end{flushleft}


\begin{flushleft}
of Materials
\end{flushleft}


\begin{flushleft}
Science
\end{flushleft}





(3-0-0) 3


\begin{flushleft}
PYL704
\end{flushleft}


\begin{flushleft}
II
\end{flushleft}





\begin{flushleft}
Science and
\end{flushleft}


\begin{flushleft}
Technology of
\end{flushleft}


\begin{flushleft}
Thin Films
\end{flushleft}





(3-0-0) 3





\begin{flushleft}
PYL702
\end{flushleft}





\begin{flushleft}
Physics of
\end{flushleft}


\begin{flushleft}
Semiconductor
\end{flushleft}


\begin{flushleft}
Devices
\end{flushleft}





\begin{flushleft}
PYL703
\end{flushleft}





\begin{flushleft}
Electronic
\end{flushleft}


\begin{flushleft}
Properties of
\end{flushleft}


\begin{flushleft}
Materials
\end{flushleft}





\begin{flushleft}
PYP701
\end{flushleft}





\begin{flushleft}
Solid State
\end{flushleft}


\begin{flushleft}
Materials
\end{flushleft}


\begin{flushleft}
Laboratory-I
\end{flushleft}





(3-0-0) 3





(3-0-0) 3





(0-0-6) 3





\begin{flushleft}
PYL705
\end{flushleft}





\begin{flushleft}
PYP702
\end{flushleft}





\begin{flushleft}
PE-2
\end{flushleft}


(3-0-0) 3





\begin{flushleft}
Nanostructured
\end{flushleft}


\begin{flushleft}
Materials
\end{flushleft}





(3-0-0) 3





\begin{flushleft}
Solid State
\end{flushleft}


\begin{flushleft}
Materials
\end{flushleft}


\begin{flushleft}
Laboratory-II
\end{flushleft}





0	


0	


0	


0	


0	


0	


0	


0	


0	





\begin{flushleft}
Contact h/week
\end{flushleft}





0	


0	


0	


0	


0	


0	


0	


0	


0	





3


3


3


3


3


3


2


1


1





\begin{flushleft}
L
\end{flushleft}





\begin{flushleft}
T
\end{flushleft}





\begin{flushleft}
P
\end{flushleft}





\begin{flushleft}
Total
\end{flushleft}





\begin{flushleft}
Credits
\end{flushleft}





\begin{flushleft}
Courses
\end{flushleft}


\begin{flushleft}
(Number, Abbreviated Title, L-T-P, credits)
\end{flushleft}





\begin{flushleft}
Sem.
\end{flushleft}





3	


3	


3	


3	


3	


3	


2	


1	


1	





\begin{flushleft}
Lecture
\end{flushleft}


\begin{flushleft}
courses
\end{flushleft}





\begin{flushleft}
Major Project Part-I	
\end{flushleft}


0	 0	 12	6


\begin{flushleft}
Major Project Part-II	
\end{flushleft}


0	 0	 24	12


\begin{flushleft}
Physical Foundations of Materials Science 	 3	 0	 0	 3
\end{flushleft}


\begin{flushleft}
Physics of Semiconductor Devices	
\end{flushleft}


3	 0	 0	 3


\begin{flushleft}
Electronic Properties of Materials	
\end{flushleft}


3	 0	 0	 3


\begin{flushleft}
Science and Technology of Thin Films 	
\end{flushleft}


3	 0	 0	 3


\begin{flushleft}
Nanostructured Materials 	
\end{flushleft}


3	 0	 0	 3


\begin{flushleft}
Solid State Materials Laboratory I	
\end{flushleft}


0	 0	 6	 3


\begin{flushleft}
Solid State Materials Laboratory II 	
\end{flushleft}


0	 0	 6	 3


\begin{flushleft}
Total Credits				 39
\end{flushleft}





4





12





0





6





18





15





4





12





0





6





18





15





1





3





0





12





15





9





0





0





0





24





24





12





\begin{flushleft}
PE-1
\end{flushleft}


(3-0-0) 3





\begin{flushleft}
PE-3
\end{flushleft}


(3-0-0) 3





(0-0-6) 3





\begin{flushleft}
Summer
\end{flushleft}


\begin{flushleft}
III
\end{flushleft}





\begin{flushleft}
OE-1
\end{flushleft}


(3-0-0) 3





\begin{flushleft}
PYD801
\end{flushleft}





\begin{flushleft}
Maj. Proj. Part-I
\end{flushleft}





(0-0-12) 6





\begin{flushleft}
PYD802
\end{flushleft}


\begin{flushleft}
IV
\end{flushleft}





\begin{flushleft}
Maj. Proj. Part-II
\end{flushleft}


\begin{flushleft}
+ Report
\end{flushleft}





(0-0-24) 12





	





\begin{flushleft}
Total = 51
\end{flushleft}


129





\begin{flushleft}
\newpage
Programme Code: TTE
\end{flushleft}





\begin{flushleft}
Master of Technology in Textile Engineering
\end{flushleft}


\begin{flushleft}
Department of Textile Technology
\end{flushleft}


\begin{flushleft}
The overall credits structure
\end{flushleft}


\begin{flushleft}
Category
\end{flushleft}





\begin{flushleft}
PC
\end{flushleft}





\begin{flushleft}
PE
\end{flushleft}





\begin{flushleft}
OE
\end{flushleft}





\begin{flushleft}
Total
\end{flushleft}





\begin{flushleft}
Credits
\end{flushleft}





42





12





0





54





\begin{flushleft}
Major Project Part-I (TXE)	
\end{flushleft}


0	 0	 12	6


\begin{flushleft}
Major Project Part-II (TXE)	
\end{flushleft}


0	 0	 24	12


\begin{flushleft}
Theory of Yarn Structure	
\end{flushleft}


3	 0	 0	 3


\begin{flushleft}
Mechanics of Spinning Processes	
\end{flushleft}


3	 0	 0	 3


\begin{flushleft}
Mechanics of Spinning Machines	
\end{flushleft}


3	 0	 0	 3


\begin{flushleft}
Theory of Fabric Structure	
\end{flushleft}


3	 0	 0	 3


\begin{flushleft}
Advanced Fabric Manufacturing Systems	
\end{flushleft}


3	 0	 0	 3


\begin{flushleft}
Technical Textiles	
\end{flushleft}


3	 0	 0	 3


\begin{flushleft}
Design of Experiments and Statistical Techniques	3	 0	 0	 3
\end{flushleft}


\begin{flushleft}
Mechanics of Textile Machines Laboratory	 0	 0	 2	 1
\end{flushleft}


\begin{flushleft}
Evaluation of Textile Materials	
\end{flushleft}


0	 0	 4	 2


\begin{flushleft}
Total Credits				 42
\end{flushleft}





\begin{flushleft}
Program Electives
\end{flushleft}


\begin{flushleft}
TXD809	
\end{flushleft}


\begin{flushleft}
TXL700	
\end{flushleft}


\begin{flushleft}
TXL710	
\end{flushleft}


\begin{flushleft}
TXL712	
\end{flushleft}


\begin{flushleft}
TXL719	
\end{flushleft}


\begin{flushleft}
TXL724	
\end{flushleft}





\begin{flushleft}
Mini Project (Textile Engineering)	
\end{flushleft}


0	


\begin{flushleft}
Modelling and Simulation in Fibrous Assemblies	2	
\end{flushleft}


\begin{flushleft}
High Performance \& Specialty. Fiber	
\end{flushleft}


3	


\begin{flushleft}
Polymer and Fibre Physics	
\end{flushleft}


3	


\begin{flushleft}
Functional \& Smart Textiles	
\end{flushleft}


3	


\begin{flushleft}
Textured Yarn Technology	
\end{flushleft}


3	





\begin{flushleft}
Courses
\end{flushleft}


\begin{flushleft}
(Number, Abbreviated Title, L-T-P, credits)
\end{flushleft}





\begin{flushleft}
Sem.
\end{flushleft}





\begin{flushleft}
TXL721
\end{flushleft}


\begin{flushleft}
I
\end{flushleft}





\begin{flushleft}
II
\end{flushleft}





0	 8	 4


0	 2	 3


0	 0	 3


0	 0	 3


0	 0	 3


0	 0	 3





\begin{flushleft}
TXL734	 Nonwoven Science and Engineering	
\end{flushleft}


3	


\begin{flushleft}
TXL740	 Science \& App. of Nanotechnology in Textiles	3	
\end{flushleft}


\begin{flushleft}
TXL750	 Science of Clothing Comfort	
\end{flushleft}


3	


\begin{flushleft}
TXL751	 Apparel Engineering and Quality Control	
\end{flushleft}


2	


\begin{flushleft}
TXL752	 Design of Functional Clothing	
\end{flushleft}


3	


\begin{flushleft}
TXL766	 Design and Manuf. of Textile Structural	
\end{flushleft}


3	


\begin{flushleft}
	Composites
\end{flushleft}


\begin{flushleft}
TXL771	 Electronics and Controls for Textile Industry	 3	
\end{flushleft}


\begin{flushleft}
TXL772 	 Computational Methods for Textiles 	
\end{flushleft}


2	


\begin{flushleft}
TXL773	 Medical Textiles	
\end{flushleft}


3	


\begin{flushleft}
TXL774	 Process Control in Yarn \& Fabric Manufacturing	3	
\end{flushleft}


\begin{flushleft}
TXL777	 Product Design and Development	
\end{flushleft}


3	


\begin{flushleft}
TXL781	 Project Appraisal and Finance	
\end{flushleft}


3	


\begin{flushleft}
TXL782	 Production and Operations Management in	 3	
\end{flushleft}


	


\begin{flushleft}
Textile Industry
\end{flushleft}


\begin{flushleft}
TXL784 	 Supply Chain Management in Textile Industry 	3	
\end{flushleft}


\begin{flushleft}
TXL807	 Seminar (Textile Engineering)	
\end{flushleft}


0	


\begin{flushleft}
TXS805	 Independent Study (Textile Engineering)	
\end{flushleft}


0	


\begin{flushleft}
TXV702	 Management of Textile Business	
\end{flushleft}


1	


\begin{flushleft}
TXV703	 Special Module in Textile Technology	
\end{flushleft}


1	


\begin{flushleft}
TXV704	 Special Module in Yarn Manufacture	
\end{flushleft}


1	


\begin{flushleft}
TXV705	 Special Module in Fabric Manufacture	
\end{flushleft}


1	





\begin{flushleft}
Theory of
\end{flushleft}


\begin{flushleft}
Yarn
\end{flushleft}


\begin{flushleft}
Structure
\end{flushleft}





\begin{flushleft}
TXL722
\end{flushleft}





\begin{flushleft}
Mechanics
\end{flushleft}


\begin{flushleft}
of Spinning
\end{flushleft}


\begin{flushleft}
Processes
\end{flushleft}





(3-0-0) 3





(3-0-0) 3





\begin{flushleft}
TXL775
\end{flushleft}





\begin{flushleft}
TXL725
\end{flushleft}





\begin{flushleft}
Technical
\end{flushleft}


\begin{flushleft}
Textiles
\end{flushleft}





(3-0-0) 3





\begin{flushleft}
Mechanics
\end{flushleft}


\begin{flushleft}
of Spinning
\end{flushleft}


\begin{flushleft}
Machines
\end{flushleft}





(3-0-0) 3





\begin{flushleft}
TXL731
\end{flushleft}





\begin{flushleft}
Theory of
\end{flushleft}


\begin{flushleft}
Fabric Structure
\end{flushleft}





\begin{flushleft}
TXL732
\end{flushleft}





(3-0-0) 3





\begin{flushleft}
Advanced Fabric
\end{flushleft}


\begin{flushleft}
Manufacturing
\end{flushleft}


\begin{flushleft}
Systems
\end{flushleft}





\begin{flushleft}
TXL783
\end{flushleft}





\begin{flushleft}
TXP761
\end{flushleft}





(3-0-0) 3





(0-0-4) 2





\begin{flushleft}
Design of Expt.
\end{flushleft}


\begin{flushleft}
and Stat. Tech.
\end{flushleft}





\begin{flushleft}
PE-1
\end{flushleft}


(3-0-0) 3





0	 2	


0	 2	


0	 0	


0	 0	


0	 0	


0	 0	


0	 0	





4


3


3


3


3


3


3





0	


2	


3	


0	


0	


0	


0	





3


2


3


1


1


1


1





\begin{flushleft}
Contact h/week
\end{flushleft}





0	


0	


0	


0	


0	


0	


0	





\begin{flushleft}
L
\end{flushleft}





\begin{flushleft}
T
\end{flushleft}





\begin{flushleft}
P
\end{flushleft}





\begin{flushleft}
Total
\end{flushleft}





5





15





0





0





15





15





4





12





0





6





18





15





2





6





0





12





18





12





0





0





0





24





24





12





(3-0-0) 3


\begin{flushleft}
Evaluation of
\end{flushleft}


\begin{flushleft}
Textile Materials
\end{flushleft}





\begin{flushleft}
TXP725
\end{flushleft}





\begin{flushleft}
Mechanics of
\end{flushleft}


\begin{flushleft}
Textile Machines
\end{flushleft}


\begin{flushleft}
Laboratory
\end{flushleft}





\begin{flushleft}
PE-2
\end{flushleft}


(3-0-0) 3





(0-0-2) 1





\begin{flushleft}
Summer
\end{flushleft}





0





\begin{flushleft}
TXD801
\end{flushleft}


\begin{flushleft}
III
\end{flushleft}





0	 0	 3


0	0	 3


0	 0	 3


0	 2	 3


0	 0	 3


0	 0	 3





\begin{flushleft}
Credits
\end{flushleft}





\begin{flushleft}
TXD801	
\end{flushleft}


\begin{flushleft}
TXD803	
\end{flushleft}


\begin{flushleft}
TXL721	
\end{flushleft}


\begin{flushleft}
TXL722	
\end{flushleft}


\begin{flushleft}
TXL725	
\end{flushleft}


\begin{flushleft}
TXL731	
\end{flushleft}


\begin{flushleft}
TXL732	
\end{flushleft}


\begin{flushleft}
TXL775	
\end{flushleft}


\begin{flushleft}
TXL783	
\end{flushleft}


\begin{flushleft}
TXP725	
\end{flushleft}


\begin{flushleft}
TXP761	
\end{flushleft}


	





\begin{flushleft}
Lecture
\end{flushleft}


\begin{flushleft}
courses
\end{flushleft}





\begin{flushleft}
Program Core
\end{flushleft}





\begin{flushleft}
Major Project
\end{flushleft}


\begin{flushleft}
Part-I (TXE)
\end{flushleft}





\begin{flushleft}
PE-3
\end{flushleft}


(3-0-0) 3





\begin{flushleft}
PE-4
\end{flushleft}


(3-0-0) 3





(0-0-12) 6


\begin{flushleft}
TXD803
\end{flushleft}


\begin{flushleft}
IV
\end{flushleft}





\begin{flushleft}
Major Project
\end{flushleft}


\begin{flushleft}
Part-II (TXE)
\end{flushleft}





(0-0-24) 12





\begin{flushleft}
Total = 54
\end{flushleft}


130





\begin{flushleft}
\newpage
Master of Technology in Textile Chemical Processing
\end{flushleft}





\begin{flushleft}
Programme Code: TTC
\end{flushleft}





\begin{flushleft}
Department of Textile Technology
\end{flushleft}


\begin{flushleft}
The overall credits structure
\end{flushleft}


\begin{flushleft}
Category
\end{flushleft}





\begin{flushleft}
PC
\end{flushleft}





\begin{flushleft}
PE
\end{flushleft}





\begin{flushleft}
OE
\end{flushleft}





\begin{flushleft}
Total
\end{flushleft}





\begin{flushleft}
Credits
\end{flushleft}





42





12





0





54





\begin{flushleft}
Program Core
\end{flushleft}


\begin{flushleft}
TXD805	 Major project part-I (TXC)	
\end{flushleft}


\begin{flushleft}
TXD806	 Major project part-II (TXC)	
\end{flushleft}


\begin{flushleft}
TXL712	 Polymer and Fibre Physics	
\end{flushleft}


\begin{flushleft}
TXL747	 Colour Science	
\end{flushleft}


\begin{flushleft}
TXL748	 Advances in Finishing of Textiles 	
\end{flushleft}


\begin{flushleft}
TXL749	 Theory and Practice of Dyeing	
\end{flushleft}


\begin{flushleft}
TXL753	 Advanced Textile Printing Technology	
\end{flushleft}


\begin{flushleft}
TXL754	 Sustainable Chemical Processing of Textiles	
\end{flushleft}


\begin{flushleft}
TXL783	 Design of Experiments and Statistical	
\end{flushleft}


\begin{flushleft}
	Techniques
\end{flushleft}


\begin{flushleft}
TXP748	 Textile Preparation and Finishing Lab	
\end{flushleft}


\begin{flushleft}
TXP749	 Textile Coloration Lab	
\end{flushleft}


\begin{flushleft}
TXP751	 Characterization and Evaluation of dyed	
\end{flushleft}


	


\begin{flushleft}
and finished textiles Lab
\end{flushleft}


\begin{flushleft}
TXS751	 Research Seminar	
\end{flushleft}


\begin{flushleft}
TXR752	 Professional Practices	
\end{flushleft}


\begin{flushleft}
TXT800	 Industrial Summer Training	
\end{flushleft}





0	


0	


3	


3	


3	


3	


2	


2	


3	





0	


0	


0	


0	


0	


0	


0	


0	


0	





12	6


24	12


0	 3


0	 3


0	 3


0	 3


0	 2


0	 2


0	 3





0	 0	 2	 1


0	 0	 2	 1


0	 0	 2	 1


0	 0	 2	 1


0	 0	 2	 1


\begin{flushleft}
Non credit
\end{flushleft}





\begin{flushleft}
Total Credits				
\end{flushleft}





	





42





\begin{flushleft}
Program Electives
\end{flushleft}


\begin{flushleft}
MSL760	
\end{flushleft}


\begin{flushleft}
MSL802	
\end{flushleft}


\begin{flushleft}
MSL816	
\end{flushleft}


\begin{flushleft}
TXD812	
\end{flushleft}


\begin{flushleft}
TXL711	
\end{flushleft}


\begin{flushleft}
TXL713	
\end{flushleft}





\begin{flushleft}
Marketing Management	
\end{flushleft}


\begin{flushleft}
Management of Intellectual Property Rights 	
\end{flushleft}


\begin{flushleft}
Total Quality Management	
\end{flushleft}


\begin{flushleft}
Mini Projects (TCP)	
\end{flushleft}


\begin{flushleft}
Polymer and Fibre Chemistry	
\end{flushleft}


\begin{flushleft}
Technology of Melt Spun Fibres 	
\end{flushleft}





2	


3	


2	


0	


3	


3	





0	


0	


0	


0	


0	


1	





2	


0	


2	


6	


0	


0	





3


3


3


3


3


4





\begin{flushleft}
TXL714	 Advanced Materials Characterization 	
\end{flushleft}


1	 0	 0	 1


\begin{flushleft}
	Techniques
\end{flushleft}


\begin{flushleft}
TXL715	 Technology of Solution Spun Fibres 	
\end{flushleft}


3	 0	 0	 3


\begin{flushleft}
TXL719	 Functional and Smart Textiles 	
\end{flushleft}


3	 0	 0	 3


\begin{flushleft}
TXL724	 Textured Yarn Technology 	
\end{flushleft}


3	 0	 0	 3


\begin{flushleft}
TXL740	 Science and Application of Nanotechnology 	 3	 0	 0	 3
\end{flushleft}


	


\begin{flushleft}
in Textiles
\end{flushleft}


\begin{flushleft}
TXL750	 Science of Clothing Comfort 	
\end{flushleft}


3	 0	 0	 3


\begin{flushleft}
TXL751	 Apparel Engineering and Quality Control 	
\end{flushleft}


2	 0	 2	 3


\begin{flushleft}
TXL752	 Design of Functional Clothing 	
\end{flushleft}


3	 0	 0	 3


\begin{flushleft}
TXL755	 Textile Wet Processing Machines: 	
\end{flushleft}


3	 0	 0	 3


	


\begin{flushleft}
Automation and Control*
\end{flushleft}


\begin{flushleft}
TXL756	 Textile Auxiliaries*	
\end{flushleft}


3	 0	 0	 3


\begin{flushleft}
TXL766	 Design and Manuf. of Textile Structural 	
\end{flushleft}


3	 0	 0	 3


\begin{flushleft}
	Composites
\end{flushleft}


\begin{flushleft}
TXL773	 Medical Textiles 	
\end{flushleft}


3	 0	 0	 3


\begin{flushleft}
TXL775	 Technical Textiles	
\end{flushleft}


3	 0	 0	 3


\begin{flushleft}
TXL777	 Product Design and Development 	
\end{flushleft}


3	 0	 0	 3


\begin{flushleft}
TXL781	 Project Appraisal and Finance	
\end{flushleft}


2	 1	 0	 3


\begin{flushleft}
TXL782	 Production and Operations Management 	
\end{flushleft}


3	 0	 0	 3


	


\begin{flushleft}
in Textile Industry
\end{flushleft}


\begin{flushleft}
TXL784	 Supply Chain Mgmt. in Textile Industry 	
\end{flushleft}


3	 0	 0	 3


\begin{flushleft}
TXP711	 Polymer and Fibre Chemistry Laboratory 	
\end{flushleft}


0	 2	 0	 1


\begin{flushleft}
TXP712	 Polymer and Fibre Physics Laboratory 	
\end{flushleft}


0	 0	 2	 1


\begin{flushleft}
TXP716	 Fibre Production and Post Spinning 	
\end{flushleft}


0	 0	 4	 2


	


\begin{flushleft}
Operation Laboratory
\end{flushleft}


\begin{flushleft}
TXP761	 Evaluation of Textile Materials	
\end{flushleft}


0	 0	 4	 2


\begin{flushleft}
TXS811	 Independent Study 	
\end{flushleft}


0	 3	 0	 3


\begin{flushleft}
TXV707	 Special Module-Textile Chemical Processing-1 	1	 0	0	 1
\end{flushleft}


\begin{flushleft}
* TCP PE Basket
\end{flushleft}





\begin{flushleft}
I
\end{flushleft}





\begin{flushleft}
Polymer
\end{flushleft}


\begin{flushleft}
\& Fibre
\end{flushleft}


\begin{flushleft}
Physics
\end{flushleft}





(3-0-0) 3


\begin{flushleft}
TXL748
\end{flushleft}


\begin{flushleft}
II
\end{flushleft}





\begin{flushleft}
Advances
\end{flushleft}


\begin{flushleft}
in Finishing
\end{flushleft}


\begin{flushleft}
of Textiles
\end{flushleft}





(3-0-0) 3


\begin{flushleft}
Summer
\end{flushleft}





\begin{flushleft}
Colour
\end{flushleft}


\begin{flushleft}
Science
\end{flushleft}





(3-0-0) 3





\begin{flushleft}
TXL754
\end{flushleft}





\begin{flushleft}
Sustainable
\end{flushleft}


\begin{flushleft}
Chemical
\end{flushleft}


\begin{flushleft}
Processing
\end{flushleft}


\begin{flushleft}
of Textiles
\end{flushleft}





(2-0-0) 2





\begin{flushleft}
TXL749
\end{flushleft}





\begin{flushleft}
Theory and
\end{flushleft}


\begin{flushleft}
Practice of
\end{flushleft}


\begin{flushleft}
Dyeing
\end{flushleft}





\begin{flushleft}
TXL753
\end{flushleft}





(3-0-0) 3





\begin{flushleft}
Advanced
\end{flushleft}


\begin{flushleft}
Textile
\end{flushleft}


\begin{flushleft}
Printing
\end{flushleft}


\begin{flushleft}
Technology
\end{flushleft}





\begin{flushleft}
TXL783
\end{flushleft}





\begin{flushleft}
TXP748
\end{flushleft}





\begin{flushleft}
Design of
\end{flushleft}


\begin{flushleft}
Experiments
\end{flushleft}


\begin{flushleft}
and Statistical
\end{flushleft}


\begin{flushleft}
Techniques
\end{flushleft}





(3-0-0) 3





(2-0-0) 2





\begin{flushleft}
TXP749
\end{flushleft}





\begin{flushleft}
Textile
\end{flushleft}


\begin{flushleft}
Coloration
\end{flushleft}


\begin{flushleft}
Lab
\end{flushleft}





(0-0-2) 1





\begin{flushleft}
TXP751
\end{flushleft}





\begin{flushleft}
Characterization
\end{flushleft}


\begin{flushleft}
of Chemicals
\end{flushleft}


\begin{flushleft}
and Finished
\end{flushleft}


\begin{flushleft}
Textiles Lab
\end{flushleft}





\begin{flushleft}
P
\end{flushleft}





\begin{flushleft}
Total
\end{flushleft}





4





11





0





6





17





14





5





14





0





2





16





15





2





6





0





14





18





13





0





0





0





24





24





12





\begin{flushleft}
TXR752
\end{flushleft}





\begin{flushleft}
Professional
\end{flushleft}


\begin{flushleft}
Practices
\end{flushleft}





(0-0-2) 1





(0-0-2) 1


\begin{flushleft}
PE-1
\end{flushleft}


(3-0-0) 3





\begin{flushleft}
Textile
\end{flushleft}


\begin{flushleft}
Preparation
\end{flushleft}


\begin{flushleft}
and Finishing
\end{flushleft}


\begin{flushleft}
Lab
\end{flushleft}





\begin{flushleft}
T
\end{flushleft}





\begin{flushleft}
PE-2
\end{flushleft}


(3-0-3) 3





(0-0-2) 1





\begin{flushleft}
TXT800 Industrial Summer Training
\end{flushleft}





\begin{flushleft}
TXD805
\end{flushleft}


\begin{flushleft}
III
\end{flushleft}





\begin{flushleft}
TXL747
\end{flushleft}





\begin{flushleft}
L
\end{flushleft}





\begin{flushleft}
Credits
\end{flushleft}





\begin{flushleft}
TXL712
\end{flushleft}





\begin{flushleft}
Contact h/week
\end{flushleft}





\begin{flushleft}
Lecture
\end{flushleft}


\begin{flushleft}
courses
\end{flushleft}





\begin{flushleft}
Courses
\end{flushleft}


\begin{flushleft}
(Number, Abbreviated Title, L-T-P, credits)
\end{flushleft}





\begin{flushleft}
Sem.
\end{flushleft}





\begin{flushleft}
Major
\end{flushleft}


\begin{flushleft}
Project
\end{flushleft}


\begin{flushleft}
Part-I (TCP)
\end{flushleft}





(0-0-12) 6





\begin{flushleft}
TXS751
\end{flushleft}


\begin{flushleft}
Research
\end{flushleft}


\begin{flushleft}
Seminar
\end{flushleft}





\begin{flushleft}
PE-3*
\end{flushleft}


(3-0-0) 3





(0-0-2) 1





\begin{flushleft}
PE-4
\end{flushleft}


(3-0-0) 3





\begin{flushleft}
TXD806
\end{flushleft}


\begin{flushleft}
IV
\end{flushleft}





\begin{flushleft}
Major
\end{flushleft}


\begin{flushleft}
Project
\end{flushleft}


\begin{flushleft}
Part-II
\end{flushleft}


\begin{flushleft}
(TCP)
\end{flushleft}





(0-0-24) 12


\begin{flushleft}
* From TCP PE Basket										Total
\end{flushleft}





131





= 54





\begin{flushleft}
\newpage
Programme Code: TTF
\end{flushleft}





\begin{flushleft}
Master of Technology in Fibre Science and Technology
\end{flushleft}


\begin{flushleft}
Department of Textile Technology
\end{flushleft}


\begin{flushleft}
The overall credits structure
\end{flushleft}


\begin{flushleft}
Category
\end{flushleft}





\begin{flushleft}
PC
\end{flushleft}





\begin{flushleft}
PE
\end{flushleft}





\begin{flushleft}
OE
\end{flushleft}





\begin{flushleft}
Total
\end{flushleft}





\begin{flushleft}
Credits
\end{flushleft}





42





12





0





54





\begin{flushleft}
Program Core
\end{flushleft}





\begin{flushleft}
TXL734	 Nonwoven Science and Engineering	
\end{flushleft}


3	 0	 0	 3


\begin{flushleft}
TXL740	 Science \& App. of Nanotechnology in Textiles	 3	0	0	3
\end{flushleft}


\begin{flushleft}
TXL741	 Env. Manag. in Textile and Allied Industries	 3	 0	 0	 3
\end{flushleft}


\begin{flushleft}
TXL747 	 Colour Science 	
\end{flushleft}


3	 0	 0	 3


\begin{flushleft}
TXL750	 Science of Clothing Comfort	
\end{flushleft}


3	 0	 0	 3


\begin{flushleft}
TXL752	 Design of Functional Clothing	
\end{flushleft}


3	 0	 0	 3


\begin{flushleft}
TXL754	 Sustainable Chemical Processing of Textiles	 2	 0	 0	 2
\end{flushleft}


\begin{flushleft}
TXL772 	 Computational Methods for Textiles 	
\end{flushleft}


2	 0	 2	 3


\begin{flushleft}
TXL773	 Medical Textiles	
\end{flushleft}


3	0	0	3


\begin{flushleft}
TXL775	 Technical Textiles	
\end{flushleft}


3	0	0	3


\begin{flushleft}
TXL777	 Product Design and Development	
\end{flushleft}


3	 0	 0	 3


\begin{flushleft}
TXL781	 Project Appraisal and Finance	
\end{flushleft}


3	0	0	3


\begin{flushleft}
TXL782	 Production and Operations Management in	 3	 0	 0	 3
\end{flushleft}


	


\begin{flushleft}
Textile Industry
\end{flushleft}


\begin{flushleft}
TXL783	 Design of Experiments and Statistical Techniques	3	0	0	3
\end{flushleft}


\begin{flushleft}
TXL784 	 Supply Chain Management in Textile Industry 	3	0	0	3
\end{flushleft}


\begin{flushleft}
TXS806 	 Independent Study (TTF) 	
\end{flushleft}


0	 3	 0	 3


\begin{flushleft}
TXV701	 Process Cont. and Econ. in Manmade Fibre Prod.	1	0	0	1
\end{flushleft}


\begin{flushleft}
TXV702	Management of Textile Business	
\end{flushleft}


1	0	0	1


\begin{flushleft}
TXV706	 Special Module in Fibre Science	
\end{flushleft}


1	 0	 0	 1


\begin{flushleft}
TXV707	 Special Module in Textile Chemical Processing	1	0	0	1
\end{flushleft}





\begin{flushleft}
TXL700	 Modelling and Simulation in Fibrous Assemblies	2	0	2	3
\end{flushleft}


\begin{flushleft}
TXL710	 High Performance \& Specialty. Fiber	
\end{flushleft}


3	 0	 0	 3


\begin{flushleft}
TXL719	 Functional \& Smart Textiles	
\end{flushleft}


3	0	0	3


\begin{flushleft}
TXL724	 Textured Yarn Technology	
\end{flushleft}


3	0	0	3





\begin{flushleft}
Courses
\end{flushleft}


\begin{flushleft}
(Number, Abbreviated Title, L-T-P, credits)
\end{flushleft}





\begin{flushleft}
Sem.
\end{flushleft}





\begin{flushleft}
I
\end{flushleft}





\begin{flushleft}
TXL711
\end{flushleft}





\begin{flushleft}
TXP711
\end{flushleft}





(3-0-0) 3





(0-0-2) 1





\begin{flushleft}
Polymer \&
\end{flushleft}


\begin{flushleft}
Chemistry
\end{flushleft}





\begin{flushleft}
TXL715
\end{flushleft}


\begin{flushleft}
II
\end{flushleft}





\begin{flushleft}
Technology
\end{flushleft}


\begin{flushleft}
of Soln Spun
\end{flushleft}


\begin{flushleft}
Fibres
\end{flushleft}





(3-0-0) 3





\begin{flushleft}
Polymer \& Fibre
\end{flushleft}


\begin{flushleft}
Chemistry Lab
\end{flushleft}





\begin{flushleft}
TXP716
\end{flushleft}





\begin{flushleft}
Fibre Production
\end{flushleft}


\begin{flushleft}
\& Post Spinning
\end{flushleft}


\begin{flushleft}
Operation Lab
\end{flushleft}





(0-0-4) 2





\begin{flushleft}
TXL712
\end{flushleft}


\begin{flushleft}
Polymer
\end{flushleft}


\begin{flushleft}
\& Fibre
\end{flushleft}


\begin{flushleft}
Physics
\end{flushleft}





(3-0-0) 3


\begin{flushleft}
TXL748
\end{flushleft}





\begin{flushleft}
Advances in
\end{flushleft}


\begin{flushleft}
Finishing of
\end{flushleft}


\begin{flushleft}
Textiles
\end{flushleft}





(3-0-0) 3





\begin{flushleft}
TXP712
\end{flushleft}





\begin{flushleft}
Polymer \& Fibre
\end{flushleft}


\begin{flushleft}
Physics Lab
\end{flushleft}





(0-0-2) 1





\begin{flushleft}
TXL713
\end{flushleft}





\begin{flushleft}
Technology
\end{flushleft}


\begin{flushleft}
of Melt Spun
\end{flushleft}


\begin{flushleft}
Fibres
\end{flushleft}





(3-1-0) 4





\begin{flushleft}
TXL714
\end{flushleft}





\begin{flushleft}
Characterization
\end{flushleft}


\begin{flushleft}
of advanced
\end{flushleft}


\begin{flushleft}
materials
\end{flushleft}





\begin{flushleft}
Contact h/week
\end{flushleft}


\begin{flushleft}
L
\end{flushleft}





\begin{flushleft}
T
\end{flushleft}





\begin{flushleft}
P
\end{flushleft}





\begin{flushleft}
Total
\end{flushleft}





\begin{flushleft}
Credits
\end{flushleft}





\begin{flushleft}
Program Electives
\end{flushleft}





\begin{flushleft}
Lecture
\end{flushleft}


\begin{flushleft}
courses
\end{flushleft}





\begin{flushleft}
TXD802 	 Major Project Part-I 	
\end{flushleft}


0	 0	 12	6


\begin{flushleft}
TXD804 	 Major Project Part-II 	
\end{flushleft}


0	 0	 24	12


\begin{flushleft}
TXL711 	 Polymer and Fibre Chemistry 	
\end{flushleft}


3	 0	 0	 3


\begin{flushleft}
TXL712	 Polymer and Fibre Physics	
\end{flushleft}


3	 0	 0	 3


\begin{flushleft}
TXL713 	 Technology of Melt Spun Fibres 	
\end{flushleft}


3	 1	 0	 4


\begin{flushleft}
TXL714 	 Advanced Materials Characterization Techniques 	
\end{flushleft}


1	0	0	1


\begin{flushleft}
TXL715 	 Technology of Solution Spun Fibres 	
\end{flushleft}


3	 0	 0	 3


\begin{flushleft}
TXL748 	 Advances in Finishing of Textiles 	
\end{flushleft}


3	 0	 0	 3


\begin{flushleft}
TXL749 	 Theory and Practice of Dyeing 	
\end{flushleft}


3	 0	 0	 3


\begin{flushleft}
TXP711 	 Polymer and Fibre Chemistry Laboratory 	
\end{flushleft}


0	 0	 2	 1


\begin{flushleft}
TXP712 	 Polymer and Fibre Physics Laboratory 	
\end{flushleft}


0	 0	 2	 1


\begin{flushleft}
TXP716 	 Fibre Production and Post Spinning 	
\end{flushleft}


0	 0	 4	 2


	


\begin{flushleft}
Operation Laboratory
\end{flushleft}


	


\begin{flushleft}
Total Credits				42
\end{flushleft}





4





12





1





4





17





15





5





12





0





6





18





15





2





6





0





12





18





12





0





0





0





24





24





12





\begin{flushleft}
TXL749
\end{flushleft}





\begin{flushleft}
Theory and
\end{flushleft}


\begin{flushleft}
Practice of
\end{flushleft}


\begin{flushleft}
Dyeing
\end{flushleft}





(3-0-0) 3





\begin{flushleft}
PE-1
\end{flushleft}


(3-0-0) 3





\begin{flushleft}
PE-2
\end{flushleft}


(3-0-0) 3





\begin{flushleft}
PE-3
\end{flushleft}


(3-0-0) 3





\begin{flushleft}
PE-4
\end{flushleft}


(3-0-0) 3





(1-0-0) 1





\begin{flushleft}
Summer
\end{flushleft}





\begin{flushleft}
TXD802
\end{flushleft}


\begin{flushleft}
III
\end{flushleft}





\begin{flushleft}
Maj. Proj. Part-I
\end{flushleft}


\begin{flushleft}
(TTF)
\end{flushleft}





(0-0-12) 6


\begin{flushleft}
TXD804
\end{flushleft}


\begin{flushleft}
IV
\end{flushleft}





\begin{flushleft}
Maj. Proj.
\end{flushleft}


\begin{flushleft}
Part-II
\end{flushleft}


\begin{flushleft}
(TTF)
\end{flushleft}





(0-0-24) 12





\begin{flushleft}
Total = 54
\end{flushleft}


132





\begin{flushleft}
\newpage
Programme Code: CRF
\end{flushleft}





\begin{flushleft}
Master of Technology in Radio Frequency Design and Technology
\end{flushleft}


\begin{flushleft}
Centre for Applied Research and Electronics
\end{flushleft}


\begin{flushleft}
The overall credits structure
\end{flushleft}


\begin{flushleft}
Category
\end{flushleft}





\begin{flushleft}
PC
\end{flushleft}





\begin{flushleft}
PE
\end{flushleft}





\begin{flushleft}
OE
\end{flushleft}





\begin{flushleft}
Total
\end{flushleft}





\begin{flushleft}
Credits
\end{flushleft}





24





24*/21**





0*/3**





48





\begin{flushleft}
* For students with M.Tech Dissertation
\end{flushleft}


\begin{flushleft}
** For students without M.Tech Dissertation
\end{flushleft}





\begin{flushleft}
Program Core
\end{flushleft}





\begin{flushleft}
ELL711	 Signal Theory	
\end{flushleft}


\begin{flushleft}
ELL712	 Digital Communications	
\end{flushleft}


\begin{flushleft}
ELL714	 Basic Information Theory	
\end{flushleft}


\begin{flushleft}
ELL718	 Statistical Signal Processing	
\end{flushleft}


\begin{flushleft}
ELL719	 Detection and Estimation Theory	
\end{flushleft}


\begin{flushleft}
ELL720	 Advanced Digital Signal Processing	
\end{flushleft}


\begin{flushleft}
ELL725	 Wireless Communications	
\end{flushleft}


\begin{flushleft}
ELL731	 Mixed Signal Circuit Design	
\end{flushleft}


\begin{flushleft}
ELL734	 MOS VLSI design	
\end{flushleft}


\begin{flushleft}
ELL735	 Analog Integrated Circuits	
\end{flushleft}


\begin{flushleft}
ELL784	 Introduction to Machine Learning	
\end{flushleft}


\begin{flushleft}
ELL815	 MIMO Wireless Communications	
\end{flushleft}


\begin{flushleft}
ELL833	 CMOS RF IC Design	
\end{flushleft}


\begin{flushleft}
ELP725	 Wireless Communication Laboratory	
\end{flushleft}


\begin{flushleft}
CRD802	Minor Project	
\end{flushleft}


\begin{flushleft}
CRD812	Major Project-II	
\end{flushleft}


\begin{flushleft}
CRD814	Major Project-III	
\end{flushleft}





3	0	0	3


3	0	0	3


3	0	0	3


3	0	0	3


3	0	0	3


3	 0	 0	 3


3	0	0	3


3	 0	 0	 3


3	0	0	3


3	0	0	3


3	 0	 0	 3


3	0	0	3


3	 0	 0	 3


0	1	4	3


0	0	6	3


0	0	24	


12


0	0	12	


6





\begin{flushleft}
Courses
\end{flushleft}


\begin{flushleft}
(number, Abbreviated Title, L-T-P, credits)
\end{flushleft}





\begin{flushleft}
Sem.
\end{flushleft}





\begin{flushleft}
CRL711
\end{flushleft}





\begin{flushleft}
CAD of RF and
\end{flushleft}


\begin{flushleft}
Microwave
\end{flushleft}


\begin{flushleft}
Circuits
\end{flushleft}





(3-0-2) 4


\begin{flushleft}
I
\end{flushleft}





\begin{flushleft}
CRL718
\end{flushleft}





\begin{flushleft}
RF and
\end{flushleft}


\begin{flushleft}
Microwave
\end{flushleft}


\begin{flushleft}
Measurement
\end{flushleft}


\begin{flushleft}
Lab
\end{flushleft}





\begin{flushleft}
PE-1
\end{flushleft}


(3-0-0) 3





3	 0	 0	 3


3	 0	 0	 3


3	0	0	3


3	0	0	3


3	0	0	3


3	0	0	3


3	0	0	3


1	 0	 4	 3


0	3	0	3


1	 0	 0	 1


1	0	0	1


1	0	0	1





\begin{flushleft}
L
\end{flushleft}





\begin{flushleft}
T
\end{flushleft}





\begin{flushleft}
P
\end{flushleft}





\begin{flushleft}
Total
\end{flushleft}





\begin{flushleft}
Credits
\end{flushleft}





\begin{flushleft}
Program Electives
\end{flushleft}





2	 0	 2	 3


2	 1	 0	 3


3	 0	 0	 3


3	0	0	3


3	 0	 0	 3


3	 0	 0	 3


3	0	0	3


3	0	0	3


3	 0	 0	 3


3	 0	 0	 3


3	 0	 0	 3


3	 0	 0	 3





\begin{flushleft}
Lecture
\end{flushleft}


\begin{flushleft}
courses
\end{flushleft}





\begin{flushleft}
CRL601	 Basics of Statistical Signal Analysis 	
\end{flushleft}


\begin{flushleft}
CRL611	 Basics of RF and Microwaves	
\end{flushleft}


\begin{flushleft}
CRL621	 Fundamentals of Semiconductor Devices 	
\end{flushleft}


\begin{flushleft}
CRL704	Sensor Array Signal Processing	
\end{flushleft}


\begin{flushleft}
CRL706	 Selected Topics in Radars and Sonars 	
\end{flushleft}


\begin{flushleft}
CRL707	 Human \& Machine Speech Communication	
\end{flushleft}


\begin{flushleft}
CRL708	Sonar Systems Engineering	
\end{flushleft}


\begin{flushleft}
CRL709	Underwater Electronic Systems	
\end{flushleft}


\begin{flushleft}
CRL712	 RF and Microwave Active Circuits	
\end{flushleft}


\begin{flushleft}
CRL715	 Radiating Systems for RF Communication	
\end{flushleft}


\begin{flushleft}
CRL722	 RF and Microwave Solid State Devices	
\end{flushleft}


\begin{flushleft}
CRL725	 Technology of RF and Microwave 	
\end{flushleft}


	


\begin{flushleft}
Solid State Devices
\end{flushleft}


\begin{flushleft}
CRL726	 RF MEMS Design and Technology	
\end{flushleft}


\begin{flushleft}
CRL727	 Introduction to Quantum Electron Devices	
\end{flushleft}


\begin{flushleft}
CRL729	Sensors and Transducers	
\end{flushleft}


\begin{flushleft}
CRL731	Selected Topics in RFDT-I	
\end{flushleft}


\begin{flushleft}
CRL732	Selected Topics in RFDT-II	
\end{flushleft}


\begin{flushleft}
CRL733	Selected Topics in RFDT-III	
\end{flushleft}


\begin{flushleft}
CRL734	Selected Topics in RFDT-IV	
\end{flushleft}


\begin{flushleft}
CRP723	 Fabrication Techniques for RF and	
\end{flushleft}


	


\begin{flushleft}
Microwave Devices
\end{flushleft}


\begin{flushleft}
CRS735	Independent Study	
\end{flushleft}


\begin{flushleft}
CRV741	 Acoustic Classification using Passive Sonar	
\end{flushleft}


\begin{flushleft}
CRV742	 Special Module in Radio Frequency Design	
\end{flushleft}


	


\begin{flushleft}
and Technology-I
\end{flushleft}


\begin{flushleft}
CRV743	 Special Module in Radio Frequency Design	
\end{flushleft}


	


\begin{flushleft}
and Technology-II
\end{flushleft}





\begin{flushleft}
CRD802	Minor Project	
\end{flushleft}


0	0	6	3


\begin{flushleft}
CRD811	Major Project-I	
\end{flushleft}


0	0	12	


6


\begin{flushleft}
CRL702	 Architectures and Algorithms for 	
\end{flushleft}


2	 0	 4	 4


	


\begin{flushleft}
DSP Systems
\end{flushleft}


\begin{flushleft}
CRL711	 CAD of RF and Microwave Circuits	
\end{flushleft}


3	 0	 2	 4


\begin{flushleft}
CRL724	 RF and Microwave Measurements	
\end{flushleft}


3	 0	 0	 3


\begin{flushleft}
CRP718	 RF and Microwave Measurement Lab	
\end{flushleft}


1	 0	 6	 4


	


\begin{flushleft}
Total Credits				24
\end{flushleft}





4





9-10





10





8-10





18-19





4





\begin{flushleft}
Contact h/week
\end{flushleft}





\begin{flushleft}
Bridge course : Any one of
\end{flushleft}


\begin{flushleft}
the following three :
\end{flushleft}


\begin{flushleft}
CRL601
\end{flushleft}





\begin{flushleft}
Basics of Stat. Signal Analysis
\end{flushleft}





(2-0-2)/


\begin{flushleft}
CRL611
\end{flushleft}





(1-0-6) 4





\begin{flushleft}
Basics of RF and Microwaves
\end{flushleft}





(2-1-0)/


\begin{flushleft}
CRL621
\end{flushleft}





\begin{flushleft}
Fund. of Semiconductor Devices
\end{flushleft}





(3-0-0)


\begin{flushleft}
CRL724
\end{flushleft}


\begin{flushleft}
II
\end{flushleft}





\begin{flushleft}
RF and
\end{flushleft}


\begin{flushleft}
Microwave
\end{flushleft}


\begin{flushleft}
Measurements
\end{flushleft}





(3-0-0)





\begin{flushleft}
CRL702
\end{flushleft}





\begin{flushleft}
Architectures
\end{flushleft}


\begin{flushleft}
and Algorithms
\end{flushleft}


\begin{flushleft}
for DSP Systems
\end{flushleft}





\begin{flushleft}
PE-2
\end{flushleft}


(3-0-0) 3





\begin{flushleft}
CRD802
\end{flushleft}


(0-0-6) 3





3





8





0





10





18





13





\begin{flushleft}
PE-3
\end{flushleft}


(3-0-0) 3





\begin{flushleft}
PE-4
\end{flushleft}


(3-0-0) 3





2





6





0





12





18





12





0





0





0





24





24





12





2





6





0





12





18





12





\begin{flushleft}
Minor Project
\end{flushleft}





(2-0-4) 4





\begin{flushleft}
Summer
\end{flushleft}


\begin{flushleft}
III
\end{flushleft}





\begin{flushleft}
CRD811
\end{flushleft}





\begin{flushleft}
Major Project-I
\end{flushleft}





(0-0-12)





\begin{flushleft}
CRD812
\end{flushleft}


(0-0-24)*





\begin{flushleft}
IV
\end{flushleft}


\begin{flushleft}
Project
\end{flushleft}


\begin{flushleft}
Option
\end{flushleft}


\begin{flushleft}
OR
\end{flushleft}


\begin{flushleft}
IV
\end{flushleft}


\begin{flushleft}
Course
\end{flushleft}


\begin{flushleft}
Option
\end{flushleft}





\begin{flushleft}
PE-5
\end{flushleft}


(3-0-0) 3





\begin{flushleft}
CRD814
\end{flushleft}





\begin{flushleft}
Major Project-III
\end{flushleft}





(0-0-12) 6





\begin{flushleft}
OE-1
\end{flushleft}


(3-0-0) 3





\begin{flushleft}
*	 Note : Minimum eligibility criterion for doing CRD812 (M.Tech. Project 2) in final semester leading to M.Tech. with Dissertation shall be B grade in CRD811.
\end{flushleft}


\begin{flushleft}
However, additional/higher criteria may be set CFB based on which CRC shall approve/disapprove this option for each student.
\end{flushleft}





\begin{flushleft}
Total = 51
\end{flushleft}


133





\begin{flushleft}
\newpage
Programme Code: AST
\end{flushleft}





\begin{flushleft}
Master of Technology in Atmospheric-Oceanic Science and Technology
\end{flushleft}


\begin{flushleft}
Centre for Atmospheric Sciences
\end{flushleft}


\begin{flushleft}
The overall credits structure
\end{flushleft}


\begin{flushleft}
Category
\end{flushleft}





\begin{flushleft}
PC
\end{flushleft}





\begin{flushleft}
PE
\end{flushleft}





\begin{flushleft}
OE
\end{flushleft}





\begin{flushleft}
Total
\end{flushleft}





\begin{flushleft}
Credits
\end{flushleft}





33





21





0





54





\begin{flushleft}
Program Core
\end{flushleft}





\begin{flushleft}
ASL758	 General Circulation of the Atmosphere	
\end{flushleft}


3	


\begin{flushleft}
ASL759	 Land-Atmosphere Interactions	
\end{flushleft}


3	


\begin{flushleft}
ASL760	 Renewable Energy Meteorology	
\end{flushleft}


3	


\begin{flushleft}
ASL761	 Earth System Modeling 	
\end{flushleft}


3	


\begin{flushleft}
ASL762	 Air-Sea Interaction 	
\end{flushleft}


3	


\begin{flushleft}
ASL763	 Coastal Ocean and Estuarine Processes	
\end{flushleft}


3	


\begin{flushleft}
ASL821	 Advanced Dynamic Meteorology	
\end{flushleft}


3	


\begin{flushleft}
ASL822	 Climate Variability	
\end{flushleft}


3	


\begin{flushleft}
ASL823	 Geophysical Fluid Dynamics	
\end{flushleft}


3	


\begin{flushleft}
ASL824	 Parameterization of Physical Processes 	
\end{flushleft}


3	


\begin{flushleft}
ASL826	 Ocean Modeling	
\end{flushleft}


2	


\begin{flushleft}
ASL827	 Advanced Dynamic Oceanography 	
\end{flushleft}


3	


\begin{flushleft}
ASL851	 Special Topics in Climate 	
\end{flushleft}


3	


\begin{flushleft}
ASL852	 Special Topics in Oceans 	
\end{flushleft}


3	


\begin{flushleft}
ASL853	 Special Topics in Atmosphere 	
\end{flushleft}


3	


\begin{flushleft}
ASL854	 Special Topics in Air Pollution Studies	
\end{flushleft}


3	


\begin{flushleft}
ASL856	 Special Topics in Atmospheric and Oceanic	 2	
\end{flushleft}


\begin{flushleft}
	Observations
\end{flushleft}


\begin{flushleft}
ASP825	 Mesoscale Modeling	
\end{flushleft}


0	


\begin{flushleft}
ASP855	 Special Topics in Atmosphere and Ocean 	 1	
\end{flushleft}


\begin{flushleft}
ASP867	 Special Module in Weather Forecasting	
\end{flushleft}


0	


\begin{flushleft}
ASP868	 Special Module in Atmospheric and Oceanic 	 0	
\end{flushleft}


\begin{flushleft}
	Observations
\end{flushleft}


\begin{flushleft}
ASS800	 Independent Study	
\end{flushleft}


0	


\begin{flushleft}
ASV862	 Special Module in Climate 	
\end{flushleft}


1	


\begin{flushleft}
ASV863	 Special Module in Oceans 	
\end{flushleft}


1	


\begin{flushleft}
ASV864	 Special Module in Atmosphere 	
\end{flushleft}


1	


\begin{flushleft}
ASV865	 Special Module in Air Pollution Studies	
\end{flushleft}


1	


\begin{flushleft}
ASV866	 Special Module in Atmosphere and Ocean 	 1	
\end{flushleft}





\begin{flushleft}
Program Electives
\end{flushleft}


\begin{flushleft}
ASC869	
\end{flushleft}


\begin{flushleft}
ASD882	
\end{flushleft}


\begin{flushleft}
ASL750	
\end{flushleft}


\begin{flushleft}
ASL751	
\end{flushleft}


\begin{flushleft}
ASL752	
\end{flushleft}


\begin{flushleft}
ASL753	
\end{flushleft}


\begin{flushleft}
ASL754	
\end{flushleft}


\begin{flushleft}
ASL755	
\end{flushleft}


\begin{flushleft}
ASL756	
\end{flushleft}


\begin{flushleft}
ASL757	
\end{flushleft}





\begin{flushleft}
Atmospheric and Oceanic Science Colloquium 	0	 1	 0	 1
\end{flushleft}


\begin{flushleft}
Project-II	
\end{flushleft}


0	 0	 24	12


\begin{flushleft}
Boundary Layer Meteorology 	
\end{flushleft}


3	 0	 0	 3


\begin{flushleft}
Dispersion of Air Pollutants 	
\end{flushleft}


3	 0	 0	 3


\begin{flushleft}
Mesoscale Meteorology	
\end{flushleft}


3	 0	 0	 3


\begin{flushleft}
Atmospheric Aerosols	
\end{flushleft}


3	 0	 0	 3


\begin{flushleft}
Cloud Physics	
\end{flushleft}


3	 0	 0	 3


\begin{flushleft}
Remote Sensing of the Atmosphere and Ocean	3	 0	 0	3
\end{flushleft}


\begin{flushleft}
Synoptic Meteorology	
\end{flushleft}


3	 0	 0	 3


\begin{flushleft}
Tropical Weather and Climate 	
\end{flushleft}


3	 0	 0	 3





\begin{flushleft}
Courses
\end{flushleft}


\begin{flushleft}
(Number, Abbreviated Title, L-T-P, credits)
\end{flushleft}





\begin{flushleft}
Sem.
\end{flushleft}





\begin{flushleft}
ASL730
\end{flushleft}


\begin{flushleft}
I
\end{flushleft}





\begin{flushleft}
II
\end{flushleft}





\begin{flushleft}
Introduction
\end{flushleft}


\begin{flushleft}
to Weather,
\end{flushleft}


\begin{flushleft}
Climate and Air
\end{flushleft}


\begin{flushleft}
Pollution
\end{flushleft}





\begin{flushleft}
ASP731
\end{flushleft}





\begin{flushleft}
ASL732
\end{flushleft}





(1-0-0) 1





(0-0-4) 2





\begin{flushleft}
Mathematical and
\end{flushleft}


\begin{flushleft}
Computational
\end{flushleft}


\begin{flushleft}
Methods for
\end{flushleft}


\begin{flushleft}
Atmospheric and
\end{flushleft}


\begin{flushleft}
Oceanic Sciences
\end{flushleft}





\begin{flushleft}
ASL736
\end{flushleft}





\begin{flushleft}
ASL737
\end{flushleft}





\begin{flushleft}
ASL738
\end{flushleft}





\begin{flushleft}
Science of Climate Change
\end{flushleft}





(3-0-0) 3





\begin{flushleft}
Data Analysis
\end{flushleft}


\begin{flushleft}
Methods for
\end{flushleft}


\begin{flushleft}
Atmospheric and
\end{flushleft}


\begin{flushleft}
Oceanic Sciences
\end{flushleft}





\begin{flushleft}
Physical and
\end{flushleft}


\begin{flushleft}
Dynamical
\end{flushleft}


\begin{flushleft}
Oceanography
\end{flushleft}





\begin{flushleft}
ASL733
\end{flushleft}





\begin{flushleft}
Physics of the
\end{flushleft}


\begin{flushleft}
Atmosphere
\end{flushleft}





(3-0-0) 3





\begin{flushleft}
ASL734
\end{flushleft}





\begin{flushleft}
Dynamics
\end{flushleft}


\begin{flushleft}
of the
\end{flushleft}


\begin{flushleft}
Atmosphere
\end{flushleft}





(3-0-0) 3





(2-0-2) 3





(3-0-0) 3





\begin{flushleft}
Numerical
\end{flushleft}


\begin{flushleft}
Modeling of the
\end{flushleft}


\begin{flushleft}
Atmosphere and
\end{flushleft}


\begin{flushleft}
Ocean
\end{flushleft}





\begin{flushleft}
PE-2
\end{flushleft}


(3-0-0) 3





\begin{flushleft}
PE-3
\end{flushleft}


(3-0-0) 3





(2-0-2) 3





\begin{flushleft}
ASP820
\end{flushleft}





\begin{flushleft}
Advanced
\end{flushleft}


\begin{flushleft}
Data Analysis
\end{flushleft}


\begin{flushleft}
for Weather
\end{flushleft}


\begin{flushleft}
and Climate
\end{flushleft}





0	


0	


0	


0	


0	


0	


0	


0	


0	


0	


0	


0	


0	


0	


0	


0	


0	





0	


0	


0	


0	


0	


0	


0	


0	


0	


0	


2	


0	


0	


0	


0	


0	


2	





3


3


3


3


3


3


3


3


3


3


3


3


3


3


3


3


3





0	


0	


0	


0	





6	


4	


2	


2	





3


3


1


1





3	


0	


0	


0	


0	


0	





0	


0	


0	


0	


0	


0	





3


1


1


1


1


1





\begin{flushleft}
Contact h/week
\end{flushleft}


\begin{flushleft}
L
\end{flushleft}





\begin{flushleft}
T
\end{flushleft}





\begin{flushleft}
P
\end{flushleft}





\begin{flushleft}
Total
\end{flushleft}





\begin{flushleft}
Credits
\end{flushleft}





\begin{flushleft}
Project-I	
\end{flushleft}


0	 0	 12	6


\begin{flushleft}
Introduction to Weather, Climate and Air Pollution	1	 0	 0	1
\end{flushleft}


\begin{flushleft}
Mathematical and Computational Methods	 2	 0	 2	 3
\end{flushleft}


\begin{flushleft}
for Atmospheric and Oceanic Sciences
\end{flushleft}


\begin{flushleft}
Physics of the Atmosphere	
\end{flushleft}


3	 0	 0	 3


\begin{flushleft}
Dynamics of the Atmosphere	
\end{flushleft}


3	 0	 0	 3


\begin{flushleft}
Atmospheric Chemistry and Air Pollution	
\end{flushleft}


3	 0	 0	 3


\begin{flushleft}
Science of Climate Change	
\end{flushleft}


3	 0	 0	 3


\begin{flushleft}
Physical and Dynamical Oceanography	
\end{flushleft}


3	 0	 0	 3


\begin{flushleft}
Numerical Modeling of the Atmosphere	
\end{flushleft}


2	 0	 2	 3


\begin{flushleft}
and Ocean
\end{flushleft}


\begin{flushleft}
Data Analysis Methods for Atmospheric and	 0	 0	 4	 2
\end{flushleft}


\begin{flushleft}
Oceanic Sciences
\end{flushleft}


\begin{flushleft}
Advanced Data Analysis for 	
\end{flushleft}


1	 0	 4	 3


\begin{flushleft}
Weather and Climate
\end{flushleft}


\begin{flushleft}
Total Credits				 33
\end{flushleft}





\begin{flushleft}
Lecture
\end{flushleft}


\begin{flushleft}
courses
\end{flushleft}





\begin{flushleft}
ASD881	
\end{flushleft}


\begin{flushleft}
ASL730	
\end{flushleft}


\begin{flushleft}
ASL732	
\end{flushleft}


	


\begin{flushleft}
ASL733	
\end{flushleft}


\begin{flushleft}
ASL734	
\end{flushleft}


\begin{flushleft}
ASL735	
\end{flushleft}


\begin{flushleft}
ASL736	
\end{flushleft}


\begin{flushleft}
ASL737	
\end{flushleft}


\begin{flushleft}
ASL738	
\end{flushleft}


	


\begin{flushleft}
ASP731	
\end{flushleft}


	


\begin{flushleft}
ASP820	
\end{flushleft}


	


	





5





12





0





6





18





15





5





12





0





6





18





15





2





6





0





12





18





12





0





0





0





24





24





12





4





12





0





0





12





12





\begin{flushleft}
ASL735
\end{flushleft}





\begin{flushleft}
Atmospheric
\end{flushleft}


\begin{flushleft}
Chemistry
\end{flushleft}


\begin{flushleft}
and Air
\end{flushleft}


\begin{flushleft}
Pollution
\end{flushleft}





(3-0-0) 3





\begin{flushleft}
PE-1
\end{flushleft}


(3-0-0) 3





(1-0-4) 3





\begin{flushleft}
Summer
\end{flushleft}





\begin{flushleft}
ASD881
\end{flushleft}


\begin{flushleft}
III
\end{flushleft}





\begin{flushleft}
IV
\end{flushleft}


\begin{flushleft}
Project
\end{flushleft}


\begin{flushleft}
Option
\end{flushleft}


\begin{flushleft}
OR
\end{flushleft}


\begin{flushleft}
IV
\end{flushleft}


\begin{flushleft}
Course
\end{flushleft}


\begin{flushleft}
Option
\end{flushleft}





\begin{flushleft}
Project-I (Core)
\end{flushleft}





(0-0-12) 6


\begin{flushleft}
ASD882
\end{flushleft}


\begin{flushleft}
Project-II
\end{flushleft}





(0-0-24) 12


\begin{flushleft}
PE-4
\end{flushleft}


(3-0-0) 3





\begin{flushleft}
PE-5
\end{flushleft}


(3-0-0) 3





\begin{flushleft}
PE-6
\end{flushleft}


(3-0-0) 3





\begin{flushleft}
PE-7
\end{flushleft}


(3-0-0) 3





\begin{flushleft}
Total = 54
\end{flushleft}


134





\begin{flushleft}
\newpage
Programme Code: BMT
\end{flushleft}





\begin{flushleft}
Master of Technology in Biomedical Engineering
\end{flushleft}


\begin{flushleft}
Centre for Biomedical Engineering
\end{flushleft}


\begin{flushleft}
The overall credits structure
\end{flushleft}


\begin{flushleft}
Core
\end{flushleft}





\begin{flushleft}
Elective
\end{flushleft}





\begin{flushleft}
Total
\end{flushleft}





\begin{flushleft}
Category
\end{flushleft}





\begin{flushleft}
BC
\end{flushleft}





\begin{flushleft}
CC
\end{flushleft}





\begin{flushleft}
CP
\end{flushleft}





\begin{flushleft}
PE
\end{flushleft}





\begin{flushleft}
OE
\end{flushleft}





\begin{flushleft}
Total
\end{flushleft}





\begin{flushleft}
Credits
\end{flushleft}





02





18





21





09





03





53





\begin{flushleft}
Bridge Courses (Core)
\end{flushleft}


1	


1	


1	


1	





0	


0	


0	


0	





0	


0	


0	


0	





	





\begin{flushleft}
Total Credits				
\end{flushleft}





\begin{flushleft}
BMD801	 Major Project-1	
\end{flushleft}


\begin{flushleft}
BMD802	 Major Project-2	
\end{flushleft}





1


1


0


1


2





\begin{flushleft}
Program Core
\end{flushleft}





\begin{flushleft}
BML770
\end{flushleft}





\begin{flushleft}
II
\end{flushleft}





0	


0	


0	


0	





3


3


3


2





2	 0	 0	 2


3	 0	 0	 3


0	 0	 4	 2





\begin{flushleft}
BML790	 Modern Medicine: An Engg. Perspective	
\end{flushleft}


\begin{flushleft}
BML810	 Tissue Engineering	
\end{flushleft}


\begin{flushleft}
BML741	 Medical Device Design	
\end{flushleft}


\begin{flushleft}
BML815	 Selected Topics in Biomedical Engineering	
\end{flushleft}


\begin{flushleft}
BML860	Nanomedicine	
\end{flushleft}


\begin{flushleft}
BML820	Biomaterials	
\end{flushleft}


\begin{flushleft}
BML735	 Biomedical Image and Signal Processing	
\end{flushleft}


\begin{flushleft}
BML850	 Cancer: Diagnosis and Therapy	
\end{flushleft}


\begin{flushleft}
BML830	 Biosensor Technology	
\end{flushleft}


\begin{flushleft}
BML750	 Point of Care Medical Diagnostic Devices	
\end{flushleft}


\begin{flushleft}
BML771	 Orthopaedic Device Design	
\end{flushleft}


\begin{flushleft}
BML772	Biofabrication	
\end{flushleft}


\begin{flushleft}
BML800	 Research Techniques in Biomedical Engg.	
\end{flushleft}





\begin{flushleft}
Courses
\end{flushleft}


\begin{flushleft}
(Number, Abbreviated Title, L-T-P, credits)
\end{flushleft}





\begin{flushleft}
Sem.
\end{flushleft}





\begin{flushleft}
I
\end{flushleft}





0	


0	


0	


0	





\begin{flushleft}
Fundamentals of
\end{flushleft}


\begin{flushleft}
Biomechanics
\end{flushleft}





\begin{flushleft}
BML710
\end{flushleft}





(3-0-0) 3





\begin{flushleft}
Industrial
\end{flushleft}


\begin{flushleft}
Biomaterial
\end{flushleft}


\begin{flushleft}
Technology
\end{flushleft}





\begin{flushleft}
BML736
\end{flushleft}





\begin{flushleft}
BML760
\end{flushleft}





\begin{flushleft}
Application of
\end{flushleft}


\begin{flushleft}
Mathematics
\end{flushleft}


\begin{flushleft}
in Biomedical
\end{flushleft}


\begin{flushleft}
Engineering
\end{flushleft}





(2-0-0) 2





(3-0-0) 3


\begin{flushleft}
Ethics, Safety
\end{flushleft}


\begin{flushleft}
and Regulatory
\end{flushleft}


\begin{flushleft}
Affairs
\end{flushleft}





(2-0-0) 2





\begin{flushleft}
BML720
\end{flushleft}


\begin{flushleft}
Medical
\end{flushleft}


\begin{flushleft}
Imaging
\end{flushleft}





\begin{flushleft}
PE-1\#
\end{flushleft}


(2-4)





\begin{flushleft}
BC-1
\end{flushleft}


(1)





\begin{flushleft}
BC-2
\end{flushleft}


(1)





\begin{flushleft}
BMP743
\end{flushleft}





\begin{flushleft}
PE-2\#
\end{flushleft}


(2-4)





\begin{flushleft}
PE-3\#
\end{flushleft}


(2-4)





(3-0-0) 3


\begin{flushleft}
BML740
\end{flushleft}





\begin{flushleft}
Biomedical
\end{flushleft}


\begin{flushleft}
Instrumentation
\end{flushleft}





(3-0-0) 3





39





\begin{flushleft}
Program Electives
\end{flushleft}


!





\begin{flushleft}
 Students shall take any two core bridge courses based on their
\end{flushleft}


\begin{flushleft}
background (Engg./ Biology) on suggestion of the program adviser.
\end{flushleft}


3	


3	


3	


2	





\begin{flushleft}
Total Credits				
\end{flushleft}





	





!





\begin{flushleft}
BML770	 Fundamentals of Biomechanics	
\end{flushleft}


\begin{flushleft}
BML710	 Industrial Biomaterial Technology	
\end{flushleft}


\begin{flushleft}
BML720	 Medical Imaging	
\end{flushleft}


\begin{flushleft}
BML736	 Application of Mathematics in Biomedical	
\end{flushleft}


\begin{flushleft}
	Engineering
\end{flushleft}


\begin{flushleft}
BML760	 Biomedical Ethics, Safety and Regulatory	
\end{flushleft}


\begin{flushleft}
	Affairs
\end{flushleft}


\begin{flushleft}
BML740	 Biomedical Instrumentation	
\end{flushleft}


\begin{flushleft}
BMP743	 Basic Biomedical Laboratory	
\end{flushleft}





0	 0	 0	 9


0	 0	 0	 12





\begin{flushleft}
Basic
\end{flushleft}


\begin{flushleft}
Biomedical
\end{flushleft}


\begin{flushleft}
Laboratory
\end{flushleft}





2	 1	 0	 3


3	 0	 0	 3


2	 0	 4	 4


2	 0	 0	 2


3	 0	0	 3


3	 0	0	 3


2	 0	 2	 3


3	 0	 0	 3


3	 0	 2	 4


3	 0	 0	 3


2	 0	 0	 2


2	 0	2	 3


3	 0	 0	 3





\begin{flushleft}
Contact h/week
\end{flushleft}


\begin{flushleft}
L
\end{flushleft}





\begin{flushleft}
T
\end{flushleft}





\begin{flushleft}
P
\end{flushleft}





\begin{flushleft}
Total
\end{flushleft}





\begin{flushleft}
Credits
\end{flushleft}





\begin{flushleft}
Basic Electronics	
\end{flushleft}


\begin{flushleft}
Basic Mathematics for Biologists	
\end{flushleft}


\begin{flushleft}
Basic Biology \& Physiology	
\end{flushleft}


\begin{flushleft}
Mechanics of Biomaterials	
\end{flushleft}





\begin{flushleft}
Lecture
\end{flushleft}


\begin{flushleft}
courses
\end{flushleft}





\begin{flushleft}
BMV701	
\end{flushleft}


\begin{flushleft}
BMV702	
\end{flushleft}


\begin{flushleft}
BMV703	
\end{flushleft}


\begin{flushleft}
BMV705	
\end{flushleft}





6





13-15





0





0





13-15





13-15!





5





11-13





0





4





15-17





13-15!





2





2-4





0





18





20-22





11-13





0





0





0





24





24





12





(0-0-4) 2





\begin{flushleft}
Summer
\end{flushleft}





\begin{flushleft}
BMD801
\end{flushleft}


\begin{flushleft}
III
\end{flushleft}





\begin{flushleft}
Major Project 1
\end{flushleft}





(0-0-18) 9





\begin{flushleft}
OE-1
\end{flushleft}


(2-4)





\begin{flushleft}
BMD802
\end{flushleft}


\begin{flushleft}
IV
\end{flushleft}





\begin{flushleft}
Major Project 2
\end{flushleft}





(0-0-24) 12





\begin{flushleft}
\#	 PE-1, 2 \& 3; OE-1: Minimum 2 to maximum 4 credits can be taken by students towards each program or open elective courses.
\end{flushleft}


\begin{flushleft}
Total credits for three program electives and one open electives should be a minimum of 12.
\end{flushleft}


\begin{flushleft}
!	 Total course credits for students in each semester should not exceed 15 for the first two semesters.
\end{flushleft}





135





\begin{flushleft}
Total = 53
\end{flushleft}





\begin{flushleft}
\newpage
Programme Code: JES
\end{flushleft}





\begin{flushleft}
Master of Technology in Energy Studies
\end{flushleft}


\begin{flushleft}
Interdisciplinary Programme
\end{flushleft}


\begin{flushleft}
The overall credits structure
\end{flushleft}


\begin{flushleft}
Category
\end{flushleft}





\begin{flushleft}
PC
\end{flushleft}





\begin{flushleft}
PE
\end{flushleft}





\begin{flushleft}
OE
\end{flushleft}





\begin{flushleft}
Total
\end{flushleft}





\begin{flushleft}
Credits
\end{flushleft}





30





18





06





54





\begin{flushleft}
Program Core
\end{flushleft}





\begin{flushleft}
ESL732	Bioconversion and Processing of Waste	
\end{flushleft}


\begin{flushleft}
ESL734	 Nuclear Energy	
\end{flushleft}


\begin{flushleft}
ESL737 	 Plasma Based Materials Processing	
\end{flushleft}


\begin{flushleft}
ESL746	 Hydrogen Energy	
\end{flushleft}


\begin{flushleft}
ESL755	 Solar Photovoltaic Devices and Systems 	
\end{flushleft}


\begin{flushleft}
ESL768 	 Wind Energy and Hydro Power Systems 	
\end{flushleft}


\begin{flushleft}
ESL770 	 Solar Energy Utilization 	
\end{flushleft}


\begin{flushleft}
ESL796	 Operation and Control of Electrical Energy	
\end{flushleft}


\begin{flushleft}
	Systems
\end{flushleft}


\begin{flushleft}
ESL810 	 MHD Power Generation 	
\end{flushleft}


\begin{flushleft}
ESL840	 Solar Architecture 	
\end{flushleft}


\begin{flushleft}
ESL850	 Solar Refrigeration and Air Conditioning	
\end{flushleft}


\begin{flushleft}
ESL860	 Electrical Power Systems Analysis	
\end{flushleft}


\begin{flushleft}
ESL870 	 Fusion Energy 	
\end{flushleft}


\begin{flushleft}
ESL871 	 Advanced Fusion Energy 	
\end{flushleft}


\begin{flushleft}
ESL880	 Solar Thermal Power Generation 	
\end{flushleft}


\begin{flushleft}
JSD799	 Minor Project (JES)	
\end{flushleft}


\begin{flushleft}
JSD802 	 Major Project Part -- 2 (JES) 	
\end{flushleft}


\begin{flushleft}
JSS801	 Independent Study (JES)	
\end{flushleft}





\begin{flushleft}
CLL723	
\end{flushleft}


\begin{flushleft}
ELL758	
\end{flushleft}


\begin{flushleft}
ELL871	
\end{flushleft}


\begin{flushleft}
ESL714	
\end{flushleft}


\begin{flushleft}
ESL718	
\end{flushleft}


\begin{flushleft}
ESL722	
\end{flushleft}





\begin{flushleft}
Hydrogen Energy and Fuel Cell Technology	 3	
\end{flushleft}


\begin{flushleft}
Power Quality 	
\end{flushleft}


3	


\begin{flushleft}
Distribution System Operation and planning 	 3	
\end{flushleft}


\begin{flushleft}
Power Plant Engineering	
\end{flushleft}


3	


\begin{flushleft}
Power Generation, Transmission and Distribution	 3	
\end{flushleft}


\begin{flushleft}
Integrated Energy Systems	
\end{flushleft}


3	





\begin{flushleft}
ESL740
\end{flushleft}





\begin{flushleft}
Non-conventional
\end{flushleft}


\begin{flushleft}
Sources of
\end{flushleft}


\begin{flushleft}
Energy
\end{flushleft}





\begin{flushleft}
ESL711
\end{flushleft}





\begin{flushleft}
ESL760
\end{flushleft}





(3-0-0) 3





(3-0-0) 3





\begin{flushleft}
Fuel Technology
\end{flushleft}





\begin{flushleft}
Heat Transfer
\end{flushleft}





(3-0-0) 3





\begin{flushleft}
Summer
\end{flushleft}





\begin{flushleft}
Energy
\end{flushleft}


\begin{flushleft}
Conservation
\end{flushleft}





(3-0-0) 3





\begin{flushleft}
JSD801
\end{flushleft}


\begin{flushleft}
JSD801
\end{flushleft}





\begin{flushleft}
III
\end{flushleft}





\begin{flushleft}
ESL710
\end{flushleft}





\begin{flushleft}
Energy,
\end{flushleft}


\begin{flushleft}
Ecology and
\end{flushleft}


\begin{flushleft}
Environment
\end{flushleft}





(3-0-0) 3





\begin{flushleft}
OE-1
\end{flushleft}


(3-0-0) 3





\begin{flushleft}
Economics and
\end{flushleft}


\begin{flushleft}
Planning of
\end{flushleft}


\begin{flushleft}
Energy Systems
\end{flushleft}





\begin{flushleft}
ESL713
\end{flushleft}





\begin{flushleft}
ESL730
\end{flushleft}





(0-0-6) 3





(3-0-0) 3





\begin{flushleft}
Energy
\end{flushleft}


\begin{flushleft}
Laboratory
\end{flushleft}





\begin{flushleft}
Direct Energy
\end{flushleft}


\begin{flushleft}
Conversion
\end{flushleft}





\begin{flushleft}
PE-1
\end{flushleft}


(3-0-0) 3





3	


3	


3	


3	


3	


3	


3	


3	


0	


0	





0	


0	


0	


0	


0	


0	


0	


0	


0	


3	





0	 3


0	 3


0	 3


0	 3


0	 3


0	 3


0	 3


0	 3


24	12


0	 3





\begin{flushleft}
Contact h/week
\end{flushleft}





3


3


3


3


3


3


3


3





\begin{flushleft}
L
\end{flushleft}





\begin{flushleft}
T
\end{flushleft}





\begin{flushleft}
P
\end{flushleft}





\begin{flushleft}
Total
\end{flushleft}





5





15





0





0





15





15





4





12





0





6





18





15





2





6





0





12





18





12





0





0





0





24





24





12





4





12





0





0





12





12





\begin{flushleft}
PE-2
\end{flushleft}


(3-0-0) 3





0





\begin{flushleft}
OE-2
\end{flushleft}


(3-0-0) 3





(0-0-12) 6


\begin{flushleft}
JSD802
\end{flushleft}





\begin{flushleft}
IV
\end{flushleft}





\begin{flushleft}
(Course
\end{flushleft}


\begin{flushleft}
based)
\end{flushleft}





\begin{flushleft}
ESL750
\end{flushleft}





\begin{flushleft}
Major Project Part I (JES)
\end{flushleft}





\begin{flushleft}
Major Project
\end{flushleft}


\begin{flushleft}
Part-I (JES)
\end{flushleft}





\begin{flushleft}
(Project
\end{flushleft}


\begin{flushleft}
based)
\end{flushleft}


\begin{flushleft}
OR
\end{flushleft}





\begin{flushleft}
IV
\end{flushleft}





3


3


3


3


3


3





0	


0	


0	


0	


0	


0	


0	


0	





(3-0-0) 3





\begin{flushleft}
ESL720
\end{flushleft}


\begin{flushleft}
II
\end{flushleft}





0	


0	


0	


0	


0	


0	





\begin{flushleft}
Courses
\end{flushleft}


\begin{flushleft}
(Number, Abbreviated Title, L-T-P, credits)
\end{flushleft}





\begin{flushleft}
Sem.
\end{flushleft}





\begin{flushleft}
I
\end{flushleft}





0	


0	


0	


0	


0	


0	





0	


0	


0	


0	


0	


0	


0	


0	





\begin{flushleft}
Credits
\end{flushleft}





\begin{flushleft}
Program Electives
\end{flushleft}





3	


3	


3	


3	


3	


3	


3	


3	





\begin{flushleft}
Lecture
\end{flushleft}


\begin{flushleft}
courses
\end{flushleft}





\begin{flushleft}
ESL710	 Energy, Ecology and Environment	
\end{flushleft}


3	 0	 0	 3


\begin{flushleft}
ESL711 	 Fuel Technology 	
\end{flushleft}


3	 0	 0	 3


\begin{flushleft}
ESL720 	 Energy Conservation 	
\end{flushleft}


3	 0	 0	 3


\begin{flushleft}
ESL730	 Direct Energy Conversion 	
\end{flushleft}


3	 0	 0	 3


\begin{flushleft}
ESL740	 Non-conventional Sources of Energy 	
\end{flushleft}


3	 0	 0	 3


\begin{flushleft}
ESL750	 Economics and Planning of Energy Systems	 3	 0	 0	 3
\end{flushleft}


\begin{flushleft}
ESL760	 Heat Transfer	
\end{flushleft}


3	 0	 0	 3


\begin{flushleft}
ESP713 	 Energy Laboratory 	
\end{flushleft}


0	 0	 6	 3


\begin{flushleft}
JSD801 	 Major Project Part -- 1 (JES) 	
\end{flushleft}


0	 0	 12	6


	


\begin{flushleft}
Total Credits				 30
\end{flushleft}





\begin{flushleft}
Major Project Part-II (JES)
\end{flushleft}





(0-0-24) 12


\begin{flushleft}
PE-3
\end{flushleft}


(3-0-0) 3





\begin{flushleft}
PE-4
\end{flushleft}


(3-0-0) 3





\begin{flushleft}
PE-5
\end{flushleft}


(3-0-0) 3





\begin{flushleft}
PE-6
\end{flushleft}


(3-0-0) 3





\begin{flushleft}
Total = 54
\end{flushleft}


136





\begin{flushleft}
\newpage
Programme Code: JIT
\end{flushleft}





\begin{flushleft}
Master of Technology in Industrial Tribology and Maintenance Engineering
\end{flushleft}


\begin{flushleft}
Interdisciplinary Programme
\end{flushleft}


\begin{flushleft}
The overall credits structure
\end{flushleft}


\begin{flushleft}
PE
\end{flushleft}





\begin{flushleft}
OE
\end{flushleft}





\begin{flushleft}
Total
\end{flushleft}





33





09





06





48





\begin{flushleft}
Program Core
\end{flushleft}


\begin{flushleft}
ITL702	 Diagnostic Maintenance and Condition	
\end{flushleft}


3	 0	 2	 4


\begin{flushleft}
	Monitoring
\end{flushleft}


\begin{flushleft}
ITL703		Fundamentals of Tribology
\end{flushleft}


	


3	 0	 2	 4


\begin{flushleft}
ITL705	 Materials for Tribological Applications 	
\end{flushleft}


3	 0	 0	 3


\begin{flushleft}
ITL714	 Failure Analysis and Repair
\end{flushleft}


	


3	 0	 2	 4


\begin{flushleft}
JIT801	 Major Project Part-I (JIT)
\end{flushleft}


	


0	 0	 12	6


\begin{flushleft}
JIT802	 Major Project Part-II (JIT)
\end{flushleft}


	


0	 0	 24	12


	


\begin{flushleft}
Total Credits				33
\end{flushleft}


\begin{flushleft}
Program Electives
\end{flushleft}


\begin{flushleft}
ITL709 	 Maintenance Planning and Control
\end{flushleft}





\begin{flushleft}
ITL703
\end{flushleft}





\begin{flushleft}
II
\end{flushleft}





3	 0	 0	 3





\begin{flushleft}
Fundamentals
\end{flushleft}


\begin{flushleft}
of Tribology
\end{flushleft}





\begin{flushleft}
ITL705
\end{flushleft}





(3-0-2) 4





\begin{flushleft}
Materials for
\end{flushleft}


\begin{flushleft}
Tribological
\end{flushleft}


\begin{flushleft}
Applications
\end{flushleft}





\begin{flushleft}
ITL702
\end{flushleft}





\begin{flushleft}
ITL714
\end{flushleft}





\begin{flushleft}
Diagnostic
\end{flushleft}


\begin{flushleft}
Maintenance \&
\end{flushleft}


\begin{flushleft}
Condition Monitoring
\end{flushleft}





(3-0-2) 4





\begin{flushleft}
PE-1
\end{flushleft}


(3-0-0) 3





\begin{flushleft}
OE-1
\end{flushleft}


(3-0-0) 3





\begin{flushleft}
PE-2
\end{flushleft}


(3-0-0) 3





\begin{flushleft}
PE-3
\end{flushleft}


(3-0-0) 3





3	 0	 0	 3


	 3	 0	 0	 3


3	


2	


2	


2	


2	


3	


0	


0	





0	


0	


1	


0	


0	


0	


3	


0	





0	


2	


0	


2	


2	


0	


0	


6	





3


3


3


3


3


3


3


3





\begin{flushleft}
L
\end{flushleft}





\begin{flushleft}
T
\end{flushleft}





\begin{flushleft}
P
\end{flushleft}





\begin{flushleft}
Total
\end{flushleft}





4





12





0





2





14





13





4





12





0





4





16





14





1





3





0





12





15





9





0





0





0





24





24





12





\begin{flushleft}
Contact h/week
\end{flushleft}





(3-0-0) 3


\begin{flushleft}
Failure Analysis
\end{flushleft}


\begin{flushleft}
\& Repair
\end{flushleft}





(3-0-2) 4





\begin{flushleft}
JID801
\end{flushleft}


\begin{flushleft}
III
\end{flushleft}





\begin{flushleft}
	 Design of Tribological Elements
\end{flushleft}


	


\begin{flushleft}
	 Reliability, Availability and Maintainability
\end{flushleft}


\begin{flushleft}
(RAM) Engineering
\end{flushleft}


\begin{flushleft}
	 Corrosion and its Control
\end{flushleft}


	


\begin{flushleft}
	 Lubricants
\end{flushleft}


	


\begin{flushleft}
	 Risk Analysis and Safety
\end{flushleft}


	


\begin{flushleft}
	 Bulk Materials Handling
\end{flushleft}


	


\begin{flushleft}
	 Noise Monitoring and Control
\end{flushleft}


	


\begin{flushleft}
	 Bearing Lubrication
\end{flushleft}


	


\begin{flushleft}
	 Independent Study
\end{flushleft}


	


\begin{flushleft}
	 Minor Project
\end{flushleft}


	





\begin{flushleft}
Courses
\end{flushleft}


\begin{flushleft}
(Number, Abbreviated Title, L-T-P, credits)
\end{flushleft}





\begin{flushleft}
Sem.
\end{flushleft}





\begin{flushleft}
I
\end{flushleft}





	





\begin{flushleft}
ITL710
\end{flushleft}


\begin{flushleft}
ITL711
\end{flushleft}


	


\begin{flushleft}
ITL717
\end{flushleft}


\begin{flushleft}
ITL730
\end{flushleft}


\begin{flushleft}
ITL740
\end{flushleft}


\begin{flushleft}
ITL752
\end{flushleft}


\begin{flushleft}
ITL760
\end{flushleft}


\begin{flushleft}
ITL810
\end{flushleft}


\begin{flushleft}
JIS800
\end{flushleft}


\begin{flushleft}
JID800
\end{flushleft}





\begin{flushleft}
Credits
\end{flushleft}





\begin{flushleft}
PC
\end{flushleft}





\begin{flushleft}
Credits
\end{flushleft}





\begin{flushleft}
Lecture
\end{flushleft}


\begin{flushleft}
courses
\end{flushleft}





\begin{flushleft}
Category
\end{flushleft}





\begin{flushleft}
OE-2
\end{flushleft}


(3-0-0) 3





\begin{flushleft}
Major Project Part-I
\end{flushleft}


\begin{flushleft}
(JIT)
\end{flushleft}





(0-0-12) 6


\begin{flushleft}
JID802
\end{flushleft}


\begin{flushleft}
IV
\end{flushleft}





\begin{flushleft}
Major Project Part-II
\end{flushleft}


\begin{flushleft}
(JIT)
\end{flushleft}





(0-0-24) 12





\begin{flushleft}
Total = 48
\end{flushleft}


137





\begin{flushleft}
\newpage
Master of Technology in Instrument Technology
\end{flushleft}





\begin{flushleft}
Programme Code: JID
\end{flushleft}





\begin{flushleft}
Interdisciplinary Programme
\end{flushleft}


\begin{flushleft}
The overall credits structure
\end{flushleft}


\begin{flushleft}
Category
\end{flushleft}





\begin{flushleft}
PC
\end{flushleft}





\begin{flushleft}
PE
\end{flushleft}





\begin{flushleft}
OE
\end{flushleft}





\begin{flushleft}
Total
\end{flushleft}





\begin{flushleft}
Credits
\end{flushleft}





39





15





0





54





\begin{flushleft}
Program Core
\end{flushleft}





\begin{flushleft}
PYL755	 Basic optics and optical instrumentation	
\end{flushleft}


\begin{flushleft}
PYL780	 Diffractive and micro optics	
\end{flushleft}


\begin{flushleft}
PYL790	 Integrated Optics 	
\end{flushleft}


\begin{flushleft}
PYL792	 Optical Electronics	
\end{flushleft}


\begin{flushleft}
PYL793	 Photonic Devices	
\end{flushleft}


\begin{flushleft}
PYP761	 Optical fabrication and metrology laboratory	
\end{flushleft}


\begin{flushleft}
CRL725	 Technology of RF and Microwave	
\end{flushleft}


	


\begin{flushleft}
Solid State Devices
\end{flushleft}


\begin{flushleft}
DSC812	 Term Paper and Seminar 	
\end{flushleft}


\begin{flushleft}
DSL601	 Electronic Components and Circuits	
\end{flushleft}


	


\begin{flushleft}
(for non-electrical students only)
\end{flushleft}


\begin{flushleft}
DSL603	 Material and Mechanical Design	
\end{flushleft}


	


\begin{flushleft}
(for electrical students only)
\end{flushleft}


\begin{flushleft}
DSL722	 Precision Measurement Systems	
\end{flushleft}


\begin{flushleft}
DSL733	 Optical Material and Optical Techniques 	
\end{flushleft}


	


\begin{flushleft}
in Instrumentation	
\end{flushleft}


\begin{flushleft}
DSL737	 Display Devices and Technology	
\end{flushleft}


\begin{flushleft}
DSL740	 Instrument Organization and Ergonomics	
\end{flushleft}


\begin{flushleft}
DSL811	 Selected Topics in Instrumentation-I	
\end{flushleft}


\begin{flushleft}
DSL814	 Selected Topics in Instrumentation-II	
\end{flushleft}


\begin{flushleft}
DSL815	 Special Topics in Instrumentation	
\end{flushleft}


\begin{flushleft}
DSP705	 Advanced Instrument Technology Lab	
\end{flushleft}


\begin{flushleft}
DSS720	Independent Study	
\end{flushleft}





\begin{flushleft}
ELL746	 Biomedical Electronics 	
\end{flushleft}


\begin{flushleft}
ELL783	 Operating System	
\end{flushleft}


\begin{flushleft}
ELL787	 Embedded Systems and Applications	
\end{flushleft}


\begin{flushleft}
ELL883	 Embedded Intelligence	
\end{flushleft}


\begin{flushleft}
MCL705	Experimental Methods	
\end{flushleft}


\begin{flushleft}
MCL749	 Mechatronic Product Design	
\end{flushleft}


\begin{flushleft}
MCL781	 Machining Processes and Analysis	
\end{flushleft}


\begin{flushleft}
MCL783	 Automation in Manufacturing	
\end{flushleft}





\begin{flushleft}
DSP703
\end{flushleft}





\begin{flushleft}
Instrument
\end{flushleft}


\begin{flushleft}
Technology
\end{flushleft}


\begin{flushleft}
Laboratory-I
\end{flushleft}





(0-0-6) 3


\begin{flushleft}
DSP704
\end{flushleft}


\begin{flushleft}
II
\end{flushleft}





0	 3


2	4


0	 3


0	3


2	4


2	 4


2	 4


2	 4





\begin{flushleft}
Courses
\end{flushleft}


\begin{flushleft}
(Number, Abbreviated Title, L-T-P, credits)
\end{flushleft}





\begin{flushleft}
Sem.
\end{flushleft}





\begin{flushleft}
I
\end{flushleft}





3	 0	


3	0	


3	 0	


3	0	


3	0	


3	 0	


3	 0	


3	 0	





\begin{flushleft}
Instrument
\end{flushleft}


\begin{flushleft}
Technology
\end{flushleft}


\begin{flushleft}
Laboratory-II
\end{flushleft}





(0-0-6) 3





\begin{flushleft}
DSL711
\end{flushleft}





\begin{flushleft}
Sensors and
\end{flushleft}


\begin{flushleft}
Transducers
\end{flushleft}





(3-0-0) 3





\begin{flushleft}
DSL731
\end{flushleft}





\begin{flushleft}
Optical
\end{flushleft}


\begin{flushleft}
Components
\end{flushleft}


\begin{flushleft}
and Basic
\end{flushleft}


\begin{flushleft}
Instruments
\end{flushleft}





(3-0-0) 3


\begin{flushleft}
DSL712
\end{flushleft}





\begin{flushleft}
Electronic
\end{flushleft}


\begin{flushleft}
Techniques
\end{flushleft}


\begin{flushleft}
for Signal
\end{flushleft}


\begin{flushleft}
Conditioning
\end{flushleft}


\begin{flushleft}
and Interfacing
\end{flushleft}





\begin{flushleft}
DSL714
\end{flushleft}





\begin{flushleft}
Instrument
\end{flushleft}


\begin{flushleft}
Design and
\end{flushleft}


\begin{flushleft}
Simulations
\end{flushleft}





(2-0-2) 3





\begin{flushleft}
DSL601*/
\end{flushleft}


\begin{flushleft}
DSL603**
\end{flushleft}





\begin{flushleft}
Electronic Components
\end{flushleft}


\begin{flushleft}
and Circuits/Material
\end{flushleft}


\begin{flushleft}
and Mechanical Design
\end{flushleft}





0	 3


0	 3


0	 3


0	3


0	3


6	 3


0	 3





0	 3	 0	 3


3	 0	 0	 3


3	 0	 0	 3


3	 0	 0	 3


2	 0	 2	 3


3	 0	


2	 0	


3	 0	


3	 0	


1	 0	


0	 0	


0	3	





\begin{flushleft}
Contact h/week
\end{flushleft}





0	 3


2	 3


0	 3


0	 3


0	 1


6	 3


0	3





\begin{flushleft}
L
\end{flushleft}





\begin{flushleft}
T
\end{flushleft}





\begin{flushleft}
P
\end{flushleft}





\begin{flushleft}
Total
\end{flushleft}





\begin{flushleft}
Credits
\end{flushleft}





\begin{flushleft}
Program Electives
\end{flushleft}





3	 0	


3	 0	


3	 0	


3	0	


3	0	


0	 0	


3	 0	





\begin{flushleft}
Lecture
\end{flushleft}


\begin{flushleft}
courses
\end{flushleft}





\begin{flushleft}
DSD801	 Major Project Part-I	
\end{flushleft}


0	 0	 12	6


\begin{flushleft}
DSD802	 Major Project Part-II 	
\end{flushleft}


0	 0	 24	12


\begin{flushleft}
DSL711	 Sensors and Transducers	
\end{flushleft}


3	0	 0	3


\begin{flushleft}
DSL712	 Electronic Techniques for Signal	
\end{flushleft}


3	 0	 0	 3


	


\begin{flushleft}
Conditioning and Interfacing
\end{flushleft}


\begin{flushleft}
DSL714	 Instrument Design and Simulations	
\end{flushleft}


2	 0	 2	 3


\begin{flushleft}
DSL731	 Optical Components and Basic Instruments	 3	 0	 0	 3
\end{flushleft}


\begin{flushleft}
DSL734	 Laser Based Instrumentation	
\end{flushleft}


3	 0	 0	 3


\begin{flushleft}
DSP703	Instrument Technology Laboratory-I	
\end{flushleft}


0	0	 6	3


\begin{flushleft}
DSP704	Instrument Technology Laboratory-II	
\end{flushleft}


0	0	 6	3


	


\begin{flushleft}
Total Credits				39
\end{flushleft}





4





12





0





6





18





15





4





11





0





8





19





15





2





6





0





12





18





12





0





0





0





24





24





12





\begin{flushleft}
PE-1
\end{flushleft}


(3-0-0) 3





(3-0-0) 3


\begin{flushleft}
DSL734
\end{flushleft}





\begin{flushleft}
Laser Based
\end{flushleft}


\begin{flushleft}
Instrumentation
\end{flushleft}





(3-0-0) 3





\begin{flushleft}
PE-2
\end{flushleft}


(3-0-0) 3





(3-0-0) 3


\begin{flushleft}
Summer
\end{flushleft}





\begin{flushleft}
III
\end{flushleft}





\begin{flushleft}
IV
\end{flushleft}





\begin{flushleft}
DSD801
\end{flushleft}


(0-0-12) 6





\begin{flushleft}
PE-3
\end{flushleft}


(3-0-0) 3





\begin{flushleft}
PE-4
\end{flushleft}


(3-0-0) 3





\begin{flushleft}
DSD802
\end{flushleft}


(0-0-24) 12





\begin{flushleft}
*For students with non-electrical Engineering background.
\end{flushleft}


\begin{flushleft}
**For students with Electrical Engineering background.
\end{flushleft}





\begin{flushleft}
Total = 54
\end{flushleft}


138





\begin{flushleft}
\newpage
Programme Code: JOP
\end{flushleft}





\begin{flushleft}
Master of Technology in Optoelectronics and Optical Communication
\end{flushleft}


\begin{flushleft}
Interdisciplinary Programme
\end{flushleft}


\begin{flushleft}
The overall credits structure
\end{flushleft}


\begin{flushleft}
Category
\end{flushleft}





\begin{flushleft}
PC
\end{flushleft}





\begin{flushleft}
PE
\end{flushleft}





\begin{flushleft}
OE
\end{flushleft}





\begin{flushleft}
Total
\end{flushleft}





\begin{flushleft}
Credits
\end{flushleft}





24





27





0





51





\begin{flushleft}
Program Core
\end{flushleft}





\begin{flushleft}
ELL716	
\end{flushleft}


\begin{flushleft}
ELL720	
\end{flushleft}


\begin{flushleft}
ELL723	
\end{flushleft}


\begin{flushleft}
ELL724	
\end{flushleft}


\begin{flushleft}
ELL726	
\end{flushleft}


\begin{flushleft}
ELL728	
\end{flushleft}


\begin{flushleft}
ELL785	
\end{flushleft}


\begin{flushleft}
ELL814	
\end{flushleft}





\begin{flushleft}
Telecommunication Switching and Transmission	 3	0	
\end{flushleft}


\begin{flushleft}
Advanced Digital Signal Processing	
\end{flushleft}


3	 0	


\begin{flushleft}
Broadband Communication Systems	
\end{flushleft}


3	 0	


\begin{flushleft}
Computational Electromagnetics	
\end{flushleft}


3	0	


\begin{flushleft}
Nano-Photonics and Plasmonics 	
\end{flushleft}


3	 0	


\begin{flushleft}
Optoelectronic Instrumentation	
\end{flushleft}


3	0	


\begin{flushleft}
Computer Communication Networks 	
\end{flushleft}


3	 0	


\begin{flushleft}
Wireless Optical Communications	
\end{flushleft}


3	 0	





\begin{flushleft}
Courses
\end{flushleft}


\begin{flushleft}
(Number, Abbreviated Title, L-T-P, credits)
\end{flushleft}





\begin{flushleft}
Sem.
\end{flushleft}





\begin{flushleft}
PYL791
\end{flushleft}





\begin{flushleft}
Fibre Optics
\end{flushleft}





\begin{flushleft}
I
\end{flushleft}





(3-0-0) 3





\begin{flushleft}
PYL792
\end{flushleft}


\begin{flushleft}
II
\end{flushleft}





0	3


0	 3


0	 3


0	3


0	 3


0	3


0	 3


0	 3





\begin{flushleft}
Optical
\end{flushleft}


\begin{flushleft}
Electronics
\end{flushleft}





\begin{flushleft}
ELL727
\end{flushleft}





\begin{flushleft}
Ditigal Comm.
\end{flushleft}


\begin{flushleft}
\& Information
\end{flushleft}


\begin{flushleft}
Systems
\end{flushleft}





\begin{flushleft}
JOP791
\end{flushleft}





\begin{flushleft}
Laboratory-I (Fibre
\end{flushleft}


\begin{flushleft}
Optics Lab/Opt.
\end{flushleft}


\begin{flushleft}
Comm. Lab)
\end{flushleft}





(3-0-0) 3





(0-0-6) 3





\begin{flushleft}
ELL717
\end{flushleft}





\begin{flushleft}
JOP792
\end{flushleft}





\begin{flushleft}
Optical Communication System
\end{flushleft}





(3-0-0) 3





(3-0-0) 3





\begin{flushleft}
PE/OC
\end{flushleft}





\begin{flushleft}
JOD801
\end{flushleft}





\begin{flushleft}
Laboratory-II
\end{flushleft}


\begin{flushleft}
(Fibre Optics Lab/
\end{flushleft}


\begin{flushleft}
Opt. Comm. Lab)
\end{flushleft}





\begin{flushleft}
PYL/ELL
\end{flushleft}





\begin{flushleft}
Programme
\end{flushleft}


\begin{flushleft}
Elective I
\end{flushleft}





(3-0-0) 3





\begin{flushleft}
L
\end{flushleft}





\begin{flushleft}
T
\end{flushleft}





\begin{flushleft}
P
\end{flushleft}





\begin{flushleft}
Total
\end{flushleft}





4





12





0





6





18





15





4





12





0





6





18





15





1





3





0





12





15





9





0





0





0





24





24





12





\begin{flushleft}
Contact h/week
\end{flushleft}





\begin{flushleft}
PYL/ELL
\end{flushleft}


\begin{flushleft}
PE-2
\end{flushleft}





(3-0-0) 3





\begin{flushleft}
PYL/ELL
\end{flushleft}





\begin{flushleft}
PYL/ELL
\end{flushleft}





(3-0-0) 3





(3-0-0) 3





\begin{flushleft}
PE-3
\end{flushleft}





0	 3


0	 3


24	12


0	3


0	3


0	3


0	 1


0	 3


0	 3


0	 3


0	 3


0	 3


0	 3


0	3


0	 3


0	 3


0	 3





\begin{flushleft}
Credits
\end{flushleft}





\begin{flushleft}
Program Electives
\end{flushleft}





3	 0	


3	 0	


0	 0	


3	0	


3	0	


0	3	


1	 0	


3	 0	


3	 0	


3	 0	


3	 0	


3	 0	


3	 0	


3	0	


3	 0	


3	 0	


3	 0	





\begin{flushleft}
Lecture
\end{flushleft}


\begin{flushleft}
courses
\end{flushleft}





\begin{flushleft}
ELL819	 Introduction to Plasmonics 	
\end{flushleft}


\begin{flushleft}
ELL820	 Photonic Switching and Networking	
\end{flushleft}


\begin{flushleft}
JOD802	 Major Project Part-II 	
\end{flushleft}


\begin{flushleft}
JOL793	 Selected Topics-I	
\end{flushleft}


\begin{flushleft}
JOL794	 Selected Topics-II	
\end{flushleft}


\begin{flushleft}
JOS795	 Independent Study	
\end{flushleft}


\begin{flushleft}
JOV796	 Selected Topics in Photonics	
\end{flushleft}


\begin{flushleft}
PYL756	 Fourier optics and holography	
\end{flushleft}


\begin{flushleft}
PYL757	 Statistical and Quantum optics 	
\end{flushleft}


\begin{flushleft}
PYL760	 Biomedical optics and Bio-photonics	
\end{flushleft}


\begin{flushleft}
PYL770	 Ultra-fast optics and applications	
\end{flushleft}


\begin{flushleft}
PYL771	 Green Photonics	
\end{flushleft}


\begin{flushleft}
PYL790	 Integrated Optics 	
\end{flushleft}


\begin{flushleft}
PYL793	 Photonic Devices	
\end{flushleft}


\begin{flushleft}
PYL795	 Optics and Lasers	
\end{flushleft}


\begin{flushleft}
PYL891	 Fiber Optic Components and Devices 	
\end{flushleft}


\begin{flushleft}
PYL892	 Guided Wave Photonic Sensors 	
\end{flushleft}





\begin{flushleft}
ELL717	 Optical Communication Systems	
\end{flushleft}


3	 0	 0	 3


\begin{flushleft}
ELL727	 Digital Communication \& Information Systems	3	0	 0	3
\end{flushleft}


\begin{flushleft}
JOD801	 Major Project Part-I	
\end{flushleft}


0	 0	 12	6


\begin{flushleft}
JOP791	 Laboratory-I (Fiber Optics Lab/ Opt. Comm. Lab)	 0	0	 6	3
\end{flushleft}


\begin{flushleft}
JOP792	 Laboratory-II (Fiber Optics Lab/ Opt. Comm. Lab)	 0	0	 6	3
\end{flushleft}


\begin{flushleft}
PYL791	 Fiber Optics	
\end{flushleft}


3	0	 0	3


\begin{flushleft}
PYL792	 Optical Electronics	
\end{flushleft}


3	0	 0	3


	


\begin{flushleft}
Total Credits				24
\end{flushleft}





\begin{flushleft}
PE-4
\end{flushleft}





(0-0-6) 3





\begin{flushleft}
Summer
\end{flushleft}


\begin{flushleft}
III
\end{flushleft}





\begin{flushleft}
PE-4
\end{flushleft}





(3-0-0) 3





\begin{flushleft}
Major Project
\end{flushleft}


\begin{flushleft}
Part-I
\end{flushleft}





(0-0-12) 6





\begin{flushleft}
JOD802
\end{flushleft}





\begin{flushleft}
IV
\end{flushleft}





\begin{flushleft}
Major Project
\end{flushleft}


\begin{flushleft}
Part-II
\end{flushleft}


\begin{flushleft}
Or
\end{flushleft}


\begin{flushleft}
12 Credits PE
\end{flushleft}


\begin{flushleft}
Courses in lieu
\end{flushleft}


\begin{flushleft}
of Major Project
\end{flushleft}


\begin{flushleft}
Part-II)
\end{flushleft}





(0-0-22) 12





\begin{flushleft}
Total = 51
\end{flushleft}


139





\begin{flushleft}
\newpage
Programme Code: JPT
\end{flushleft}





\begin{flushleft}
Master of Technology in Polymer Science and Technology
\end{flushleft}


\begin{flushleft}
Interdisciplinary Programme
\end{flushleft}


\begin{flushleft}
The overall credits structure
\end{flushleft}


\begin{flushleft}
Category
\end{flushleft}





\begin{flushleft}
PC
\end{flushleft}





\begin{flushleft}
PE
\end{flushleft}





\begin{flushleft}
OE
\end{flushleft}





\begin{flushleft}
Total
\end{flushleft}





\begin{flushleft}
Credits
\end{flushleft}





42





12





0





54


\begin{flushleft}
Program Electives
\end{flushleft}





\begin{flushleft}
Polymer Chemistry	
\end{flushleft}


\begin{flushleft}
Polymer Processing 	
\end{flushleft}


\begin{flushleft}
Polymer Physics	
\end{flushleft}


\begin{flushleft}
Polymer Technology	
\end{flushleft}


\begin{flushleft}
Polymer Characterization	
\end{flushleft}


\begin{flushleft}
Polymer Engineering and Rheology	
\end{flushleft}


\begin{flushleft}
Polymer Science Laboratory	
\end{flushleft}


\begin{flushleft}
Polymer Engineering Lab	
\end{flushleft}


\begin{flushleft}
Polymer Testing and Properties	
\end{flushleft}


\begin{flushleft}
Major Project Part-I	
\end{flushleft}


\begin{flushleft}
Major Project Part-II	
\end{flushleft}





	





\begin{flushleft}
Total Credits				42
\end{flushleft}





\begin{flushleft}
PTL701
\end{flushleft}





\begin{flushleft}
PTL703
\end{flushleft}





(3-0-0) 3





(3-0-0) 3





\begin{flushleft}
Polymer Chemistry
\end{flushleft}





\begin{flushleft}
PTL702
\end{flushleft}


\begin{flushleft}
II
\end{flushleft}





0	3


0	 3


0	3


0	3


0	3


0	 3


4	 2


2	 1


0	 3


12	6


24	12





\begin{flushleft}
PTL711	 Engineering Plastics and Speciality Polymers	3	0	
\end{flushleft}


\begin{flushleft}
PTL712	 Polymer Blends and Composites	
\end{flushleft}


3	 0	


\begin{flushleft}
PTL716	 Rubber Technology	
\end{flushleft}


3	0	


\begin{flushleft}
PTL718	 Polymer Reaction Engineering	
\end{flushleft}


3	 0	


\begin{flushleft}
PTL720	 Polymer Product and Mould Design	
\end{flushleft}


2	 0	


\begin{flushleft}
PTL722	 Polymer Degradation and Stabilization	
\end{flushleft}


3	 0	


\begin{flushleft}
PTL724	 Polymeric Coatings	
\end{flushleft}


3	0	


\begin{flushleft}
PTL726	 Polymeric Nanomaterials and Nanocomposites	3	0	
\end{flushleft}


\begin{flushleft}
PTL714	 Biodegradable Polymeric Materials	
\end{flushleft}


3	 0	


\begin{flushleft}
JPD799	 Minor Project	
\end{flushleft}


0	0	


\begin{flushleft}
JPS800	 Independent Study	
\end{flushleft}


0	3	


\begin{flushleft}
PTV700	 Special Lectures in Polymers	
\end{flushleft}


1	 0	





\begin{flushleft}
Courses
\end{flushleft}


\begin{flushleft}
(Number, Abbreviated Title, L-T-P, credits)
\end{flushleft}





\begin{flushleft}
Sem.
\end{flushleft}





\begin{flushleft}
I
\end{flushleft}





3	0	


3	 0	


3	0	


3	0	


3	0	


3	 0	


0	 0	


0	 0	


3	 0	


0	 0	


0	 0	





\begin{flushleft}
Polymer Processing
\end{flushleft}





\begin{flushleft}
Polymer Physics
\end{flushleft}





\begin{flushleft}
PTL704
\end{flushleft}





\begin{flushleft}
PTL705
\end{flushleft}





\begin{flushleft}
Polymer
\end{flushleft}


\begin{flushleft}
Characterization
\end{flushleft}





\begin{flushleft}
PTL707
\end{flushleft}





(3-0-0) 3





(3-0-0) 3





(0-0-4) 2





\begin{flushleft}
PTP710
\end{flushleft}





\begin{flushleft}
PTL713
\end{flushleft}





\begin{flushleft}
PE-1
\end{flushleft}


(3-0-0) 3





\begin{flushleft}
Polymer
\end{flushleft}


\begin{flushleft}
Engineering Lab
\end{flushleft}





\begin{flushleft}
Polymer Testing
\end{flushleft}


\begin{flushleft}
and Properties
\end{flushleft}





(3-0-0) 3





(3-0-0) 3





(0-0-2) 1





(3-0-0) 3





\begin{flushleft}
PE-2
\end{flushleft}


(3-0-0) 3





\begin{flushleft}
OE-1
\end{flushleft}


(3-0-0) 3





\begin{flushleft}
OE-2
\end{flushleft}


(3-0-0) 3





\begin{flushleft}
JPD801
\end{flushleft}





\begin{flushleft}
L
\end{flushleft}





\begin{flushleft}
T
\end{flushleft}





\begin{flushleft}
P
\end{flushleft}





\begin{flushleft}
Total
\end{flushleft}





4





12





0





4





16





14





4





12





0





2





14





13





3





9





0





12





21





15





0





0





0





24





24





12





\begin{flushleft}
PTP709
\end{flushleft}





\begin{flushleft}
Polymer
\end{flushleft}


\begin{flushleft}
Engineering and
\end{flushleft}


\begin{flushleft}
Rheology
\end{flushleft}





\begin{flushleft}
Polymer
\end{flushleft}


\begin{flushleft}
Technology
\end{flushleft}





\begin{flushleft}
Contact h/week
\end{flushleft}





0	3


0	 3


0	3


0	 3


2	 3


0	 3


0	3


0	3


0	 3


6	3


0	3


0	 1





\begin{flushleft}
Credits
\end{flushleft}





\begin{flushleft}
PTL701	
\end{flushleft}


\begin{flushleft}
PTL702	
\end{flushleft}


\begin{flushleft}
PTL703	
\end{flushleft}


\begin{flushleft}
PTL704	
\end{flushleft}


\begin{flushleft}
PTL705	
\end{flushleft}


\begin{flushleft}
PTL707	
\end{flushleft}


\begin{flushleft}
PTP709	
\end{flushleft}


\begin{flushleft}
PTP710	
\end{flushleft}


\begin{flushleft}
PTL713	
\end{flushleft}


\begin{flushleft}
JPD801	
\end{flushleft}


\begin{flushleft}
JPD802	
\end{flushleft}





\begin{flushleft}
Lecture
\end{flushleft}


\begin{flushleft}
courses
\end{flushleft}





\begin{flushleft}
Program Core
\end{flushleft}





\begin{flushleft}
Polymer
\end{flushleft}


\begin{flushleft}
Science
\end{flushleft}


\begin{flushleft}
Laboratory
\end{flushleft}





\begin{flushleft}
Summer
\end{flushleft}


\begin{flushleft}
III
\end{flushleft}





\begin{flushleft}
IV
\end{flushleft}





\begin{flushleft}
Major Project Part-I
\end{flushleft}





(0-0-12) 6





\begin{flushleft}
JPD801
\end{flushleft}





\begin{flushleft}
Major Project Part-II
\end{flushleft}





(0-0-24) 12





\begin{flushleft}
Total = 54
\end{flushleft}


140





\begin{flushleft}
\newpage
Programme Code: JTM
\end{flushleft}





\begin{flushleft}
Master of Technology in Telecommunication Technology \& Management
\end{flushleft}


\begin{flushleft}
Interdisciplinary Programme
\end{flushleft}


\begin{flushleft}
The overall credits structure
\end{flushleft}


\begin{flushleft}
PE
\end{flushleft}





\begin{flushleft}
OE
\end{flushleft}





\begin{flushleft}
Total
\end{flushleft}





33





21





0





54





\begin{flushleft}
Program Core
\end{flushleft}


\begin{flushleft}
ELL711	 Signal Theory	
\end{flushleft}


3	0	 0	3


\begin{flushleft}
ELL712	 Digital Communications	
\end{flushleft}


3	0	 0	3


\begin{flushleft}
ELL785	 Computer Communication Networks 	
\end{flushleft}


3	 0	 0	 3


\begin{flushleft}
ELL818	 Telecommunication Technologies	
\end{flushleft}


3	0	 0	3


\begin{flushleft}
ELP718	 Telecommunication Software Laboratory	
\end{flushleft}


0	 1	 4	 3


\begin{flushleft}
ELP720	 Telecommunication Networks Laboratory	
\end{flushleft}


0	 1	 4	 3


\begin{flushleft}
ELP725	 Wireless Communication Laboratory	
\end{flushleft}


0	 1	 4	 3


\begin{flushleft}
JTD792	 Minor Project	
\end{flushleft}


0	0	 6	3


\begin{flushleft}
JTD801	 Major Project-I	
\end{flushleft}


0	 0	 12	6


\begin{flushleft}
MSL723 	 Telecommunications Systems Management	 3	 0	 0	 3
\end{flushleft}


	


\begin{flushleft}
Total Credits				33
\end{flushleft}


\begin{flushleft}
Program Electives
\end{flushleft}


\begin{flushleft}
ELL723	 Broadband Communication Systems	
\end{flushleft}


3	 0	 0	 3


\begin{flushleft}
Streamed Electives (JTM) in (Signal and Information Processing)
\end{flushleft}


\begin{flushleft}
ELL715	 Digital Image Processing	
\end{flushleft}


3	 0	 2	 4


\begin{flushleft}
ELL718	 Statistical Signal Processing	
\end{flushleft}


3	 0	 0	 3


\begin{flushleft}
ELL720	 Advanced Digital Signal Processing	
\end{flushleft}


3	 0	 0	 3


\begin{flushleft}
ELL784	 Introduction to Machine Learning	
\end{flushleft}


3	 0	 0	 3


\begin{flushleft}
ELL786	 Multimedia Systems	
\end{flushleft}


3	0	 0	3


\begin{flushleft}
ELL792	 Computer Graphics	
\end{flushleft}


3	 0	 0	 3


\begin{flushleft}
ELL793	 Computer Vision	
\end{flushleft}


3	0	 0	3


\begin{flushleft}
CRL707	 Human \& Machine Speech Communication	 3	 0	 0	 3
\end{flushleft}


\begin{flushleft}
Streamed Electives (JTM) in (Communication Systems)
\end{flushleft}


\begin{flushleft}
ELL710	 Coding Theory	
\end{flushleft}


3	0	 0	3


\begin{flushleft}
ELL714	 Basic Information Theory	
\end{flushleft}


3	0	 0	3


\begin{flushleft}
ELL717	 Optical Communication Systems	
\end{flushleft}


3	 0	 0	 3


\begin{flushleft}
ELL718	 Statistical Signal Processing	
\end{flushleft}


3	 0	 0	 3


\begin{flushleft}
ELL719	 Detection and Estimation Theory	
\end{flushleft}


3	 0	 0	 3


\begin{flushleft}
ELL720	 Advanced Digital Signal Processing	
\end{flushleft}


3	 0	 0	 3


\begin{flushleft}
ELL813	 Advanced Information Theory	
\end{flushleft}


3	0	 0	3


\begin{flushleft}
ELL814	 Wireless Optical Communications	
\end{flushleft}


3	 0	 0	 3


\begin{flushleft}
ELL815	 MIMO Wireless Communications	
\end{flushleft}


3	 0	 0	 3


\begin{flushleft}
ELL816	 Satellite Communication	
\end{flushleft}


3	0	 0	3


\begin{flushleft}
Streamed Electives (JTM) in (Telecom Management)
\end{flushleft}


\begin{flushleft}
MSL700 	 Fundamentals of Management of Technology	3	0	 0	3
\end{flushleft}


\begin{flushleft}
MSL701 	 Strategic Technology Management	
\end{flushleft}


2	 0	 2	 3


\begin{flushleft}
MSL707 	 Management Accounting 	
\end{flushleft}


3	 0	 0	 3


\begin{flushleft}
MSL713 	 Information Systems Management 	
\end{flushleft}


2	 0	 2	 3


\begin{flushleft}
MSL726 	 Telecom Systems Analysis, Planning and Design	3	0	 0	3
\end{flushleft}


\begin{flushleft}
MSL728 	 International Telecommunication Management	 3	0	 0	3
\end{flushleft}


\begin{flushleft}
MSL760 	 Marketing Management 	
\end{flushleft}


2	 0	 2	 3


\begin{flushleft}
MSL815 	 Decision Support and Expert Systems 	
\end{flushleft}


2	 0	 2	 3





\begin{flushleft}
ELL711
\end{flushleft}





\begin{flushleft}
Signal Theory
\end{flushleft}





(3-0-0) 3


\begin{flushleft}
ELP725
\end{flushleft}





\begin{flushleft}
II
\end{flushleft}





\begin{flushleft}
Streamed Electives (JTM) in (Embedded Systems and Network
\end{flushleft}


\begin{flushleft}
Appliance Engineering)
\end{flushleft}


\begin{flushleft}
COL719	 Synthesis of Digital Systems	
\end{flushleft}


3	 0	 2	 4


\begin{flushleft}
COL740	Software Engineering	
\end{flushleft}


3	0	 2	4


\begin{flushleft}
ELL766	 Appliance Systems	
\end{flushleft}


3	0	 0	3


\begin{flushleft}
ELL787	 Embedded Systems and Applications	
\end{flushleft}


3	 0	 0	 3


\begin{flushleft}
ELL790	 Digital Hardware Design	
\end{flushleft}


3	 0	 0	 3


\begin{flushleft}
ELL887	 Cloud Computing	
\end{flushleft}


3	0	 0	3


\begin{flushleft}
ELL898	 Pervasive Computing	
\end{flushleft}


3	0	 0	3


\begin{flushleft}
ELL899	 Testing and Fault Tolerance	
\end{flushleft}


3	 0	 0	 3


\begin{flushleft}
ELP721	 Embedded Telecommunication Systems Laboratory	 0	1	 4	3
\end{flushleft}


\begin{flushleft}
ELP781	 Digital Systems Lab	
\end{flushleft}


0	 1	 4	 3


\begin{flushleft}
Streamed Electives (JTM) in (Computer and Communication
\end{flushleft}


\begin{flushleft}
Networks)
\end{flushleft}


\begin{flushleft}
COL724	 Advanced Computer Networks	
\end{flushleft}


3	 0	 2	 4


\begin{flushleft}
ELL716	 Telecommunication Switching and Transmission	3	 0	 0	 3
\end{flushleft}


\begin{flushleft}
ELL725	 Wireless Communications	
\end{flushleft}


3	0	 0	3


\begin{flushleft}
ELL816	 Satellite Communication	
\end{flushleft}


3	0	 0	3


\begin{flushleft}
ELL817	 Access Networks	
\end{flushleft}


3	0	 0	3


\begin{flushleft}
ELL820	 Photonic Switching and Networking	
\end{flushleft}


3	 0	 0	 3


\begin{flushleft}
ELL887	 Cloud Computing	
\end{flushleft}


3	0	 0	3


\begin{flushleft}
ELL889	 Protocol Engineering	
\end{flushleft}


3	0	 0	3


\begin{flushleft}
ELL892	 Internet Technologies	
\end{flushleft}


3	0	 0	3


\begin{flushleft}
ELL894	 Network Performance Modeling and Analysis	3	 0	 0	 3
\end{flushleft}


\begin{flushleft}
ELL895	 Network Security	
\end{flushleft}


3	0	 0	3


\begin{flushleft}
ELL896	 Mobile Computing	
\end{flushleft}


3	0	 0	3


\begin{flushleft}
ELL897	 Network Management	
\end{flushleft}


3	0	 0	3


\begin{flushleft}
ELL898	 Pervasive Computing	
\end{flushleft}


3	0	 0	3


\begin{flushleft}
ELP782	 Computer Networks Lab	
\end{flushleft}


0	 1	 4	 3


\begin{flushleft}
ELP821	 Advanced Telecommunication Networks Laboratory	0	1	 4	3
\end{flushleft}


\begin{flushleft}
ELP822	 Network Software Laboratory	
\end{flushleft}


0	 1	 4	 3





\begin{flushleft}
Wireless Comm. Lab
\end{flushleft}





(0-1-4) 3





\begin{flushleft}
ELL712
\end{flushleft}





\begin{flushleft}
Digital
\end{flushleft}


\begin{flushleft}
Communication
\end{flushleft}





(3-0-0) 3


\begin{flushleft}
ELP720
\end{flushleft}





\begin{flushleft}
ELL818
\end{flushleft}





\begin{flushleft}
Telecom.
\end{flushleft}


\begin{flushleft}
Technologies
\end{flushleft}





(3-0-0) 3


\begin{flushleft}
JTD792
\end{flushleft}





\begin{flushleft}
ELP718
\end{flushleft}


\begin{flushleft}
Telecom.
\end{flushleft}


\begin{flushleft}
Software
\end{flushleft}





(0-­1-­4) 3


\begin{flushleft}
MSL723
\end{flushleft}





\begin{flushleft}
Telecom. Network Minor Project (JTM) Telecom Syst.
\end{flushleft}


\begin{flushleft}
Lab
\end{flushleft}


\begin{flushleft}
Mgmt.
\end{flushleft}


(0-0-6) 3





(0-1-4) 3





\begin{flushleft}
Contact h/week
\end{flushleft}


\begin{flushleft}
L
\end{flushleft}





\begin{flushleft}
Computer
\end{flushleft}


\begin{flushleft}
Comm. Networks
\end{flushleft}





4





12





1





4





17





15





(3-0-0) 3


\begin{flushleft}
PE-1
\end{flushleft}


(3-0-0) 3





2





6





2





14





22





15





2





6





0





12





18





12





4





12





0





0





12





12





0





0





0





24





24





12





\begin{flushleft}
Courses
\end{flushleft}


\begin{flushleft}
(Number, Abbreviated Title, L-T-P, credits)
\end{flushleft}





\begin{flushleft}
Sem.
\end{flushleft}





\begin{flushleft}
I
\end{flushleft}





\begin{flushleft}
MSL850 	 Management of Information Technology 3	 0	 0	 3
\end{flushleft}


\begin{flushleft}
Streamed Electives (JTM) in (Telecom Analytics)
\end{flushleft}


\begin{flushleft}
COL762	Database Implementation	
\end{flushleft}


3	0	 2	4


\begin{flushleft}
ELL784	 Introduction to Machine Learning	
\end{flushleft}


3	 0	 0	 3


\begin{flushleft}
ELL791	 Neural Systems and Learning Machines	
\end{flushleft}


3	 0	 2	 4


\begin{flushleft}
ELL795	 Swarm Intelligence	
\end{flushleft}


3	0	 0	3


\begin{flushleft}
ELL798	 Agent Technologies	
\end{flushleft}


3	0	 0	3


\begin{flushleft}
ELL882	 Large-Scale Machine Learning	
\end{flushleft}


3	 0	 0	 3


\begin{flushleft}
ELL884	 Information Retrieval	
\end{flushleft}


3	0	 0	3


\begin{flushleft}
ELL886	 Big Data Systems	
\end{flushleft}


3	 0	 0	 3


\begin{flushleft}
ELL887	 Cloud Computing	
\end{flushleft}


3	0	 0	3


\begin{flushleft}
ELL888	 Advanced Machine Learning	
\end{flushleft}


3	 0	 0	 3


\begin{flushleft}
ELL892	 Internet Technologies	
\end{flushleft}


3	0	 0	3





\begin{flushleft}
T
\end{flushleft}





\begin{flushleft}
P
\end{flushleft}





\begin{flushleft}
Total
\end{flushleft}





\begin{flushleft}
Credits
\end{flushleft}





\begin{flushleft}
PC
\end{flushleft}





\begin{flushleft}
Credits
\end{flushleft}





\begin{flushleft}
Lecture
\end{flushleft}


\begin{flushleft}
courses
\end{flushleft}





\begin{flushleft}
Category
\end{flushleft}





\begin{flushleft}
ELL785
\end{flushleft}





(3-0-0) 3





\begin{flushleft}
Summer
\end{flushleft}


\begin{flushleft}
III
\end{flushleft}





\begin{flushleft}
JTD801
\end{flushleft}





\begin{flushleft}
Major Project Part-I (JTM)
\end{flushleft}





\begin{flushleft}
IV
\end{flushleft}





(0-0-12) 6


\begin{flushleft}
PE-3
\end{flushleft}


(3-0-0) 3





\begin{flushleft}
IV
\end{flushleft}





\begin{flushleft}
JTD802
\end{flushleft}





\begin{flushleft}
(Course based)
\end{flushleft}


\begin{flushleft}
OR
\end{flushleft}


\begin{flushleft}
(Project
\end{flushleft}


\begin{flushleft}
based)
\end{flushleft}





\begin{flushleft}
PE-2
\end{flushleft}


(3-0-0) 3





\begin{flushleft}
PE-3
\end{flushleft}


(3-0-0) 3





\begin{flushleft}
PE-3
\end{flushleft}


(3-0-0) 3





\begin{flushleft}
PE-4
\end{flushleft}


(3-0-0) 3





\begin{flushleft}
PE-5
\end{flushleft}


(3-0-0) 3





\begin{flushleft}
Major Project Part-II (JTM)
\end{flushleft}





(0-0-24) 12





\begin{flushleft}
Total = 54
\end{flushleft}


141





\begin{flushleft}
\newpage
Programme Code: JVL
\end{flushleft}





\begin{flushleft}
Master of Technology in VLSI Design Tools and Technology
\end{flushleft}


\begin{flushleft}
Interdisciplinary Programme
\end{flushleft}


\begin{flushleft}
The overall credits structure
\end{flushleft}


\begin{flushleft}
Category
\end{flushleft}





\begin{flushleft}
PC
\end{flushleft}





\begin{flushleft}
PE
\end{flushleft}





\begin{flushleft}
OC
\end{flushleft}





\begin{flushleft}
Total
\end{flushleft}





\begin{flushleft}
Credits
\end{flushleft}





18





30





0





48





\begin{flushleft}
Program Core
\end{flushleft}





\begin{flushleft}
CRL702	 Architectures and Algorithms for DSP Systems	2	0	 4	4
\end{flushleft}


\begin{flushleft}
CRL711	 CAD of RF and Microwave Circuits	
\end{flushleft}


3	 0	 2	 4


\begin{flushleft}
CRL712	 RF and Microwave Active Circuits	
\end{flushleft}


3	 0	 0	 3





\begin{flushleft}
MOS VLSI design	
\end{flushleft}


3	 0	 0	 3


\begin{flushleft}
Physical Design Laboratory	
\end{flushleft}


0	 0	 6	 3


\begin{flushleft}
Major Project-I	
\end{flushleft}


0	 0	 24	12


\begin{flushleft}
Total Credits				18
\end{flushleft}





\begin{flushleft}
Streamed Electives (JVL) in (ASIC and SoC Design)
\end{flushleft}


\begin{flushleft}
COL719	
\end{flushleft}


\begin{flushleft}
COL812	
\end{flushleft}


\begin{flushleft}
COP745	
\end{flushleft}


\begin{flushleft}
ELL731	
\end{flushleft}


\begin{flushleft}
ELL735	
\end{flushleft}


\begin{flushleft}
ELL749	
\end{flushleft}





\begin{flushleft}
Program Electives
\end{flushleft}





\begin{flushleft}
II
\end{flushleft}





\begin{flushleft}
ELP736
\end{flushleft}





\begin{flushleft}
ELL734
\end{flushleft}





(0-0-6) 3





(3-0-0) 3





\begin{flushleft}
PE-3
\end{flushleft}


(3-0-0) 3





\begin{flushleft}
PE-4
\end{flushleft}


(3-0-0) 3





\begin{flushleft}
Physical Design Laboratory
\end{flushleft}





\begin{flushleft}
MOS VLSI Design
\end{flushleft}





0	


0	


0	


0	


0	


0	





2	


0	


6	


0	


0	


0	





4


3


3


3


3


3





\begin{flushleft}
Streamed Electives (JVL) in (Micro and Nano Devices)
\end{flushleft}


\begin{flushleft}
ELL730	
\end{flushleft}


\begin{flushleft}
ELL732	
\end{flushleft}


\begin{flushleft}
ELL738	
\end{flushleft}


\begin{flushleft}
ELL739	
\end{flushleft}


\begin{flushleft}
ELL740	
\end{flushleft}


\begin{flushleft}
ELL744	
\end{flushleft}





\begin{flushleft}
I.C. Technology	
\end{flushleft}


3	0	


\begin{flushleft}
Micro and Nanoelectronics	
\end{flushleft}


3	 0	


\begin{flushleft}
Micro and Nano Photonics	
\end{flushleft}


3	 0	


\begin{flushleft}
Advanced Semiconductor Devices	
\end{flushleft}


3	 0	


\begin{flushleft}
Compact Modeling of Semiconductor Devices	3	0	
\end{flushleft}


\begin{flushleft}
Electronic and Photonic Nanomaterials	
\end{flushleft}


3	 0	





0	3


0	 3


0	 3


0	 3


0	3


0	 3





\begin{flushleft}
Streamed Electives (JVL) in (Embedded Intelligent Systems)
\end{flushleft}


\begin{flushleft}
COL788	
\end{flushleft}


\begin{flushleft}
COL821	
\end{flushleft}


\begin{flushleft}
ELL720	
\end{flushleft}


\begin{flushleft}
ELL741	
\end{flushleft}


\begin{flushleft}
ELL784	
\end{flushleft}


\begin{flushleft}
ELL797	
\end{flushleft}





\begin{flushleft}
Advanced Topics in Embedded Computing	
\end{flushleft}


\begin{flushleft}
Reconfigurable Computing	
\end{flushleft}


\begin{flushleft}
Advanced Digital Signal Processing	
\end{flushleft}


\begin{flushleft}
Neuromorphic Engineering	
\end{flushleft}


\begin{flushleft}
Introduction to Machine Learning	
\end{flushleft}


\begin{flushleft}
Energy-Efficient Computing	
\end{flushleft}





\begin{flushleft}
Courses
\end{flushleft}


\begin{flushleft}
(Number, Abbreviated Title, L-T-P, credits)
\end{flushleft}





\begin{flushleft}
Sem.
\end{flushleft}





\begin{flushleft}
I
\end{flushleft}





2	 4


2	4


0	3


0	 3


0	3


0	3


0	 3


0	 3


0	 3


0	 3


0	 3


0	 3


0	3


6	3


0	 1


0	 1


12	6


24	12


0	3


0	3





3	


3	


0	


3	


3	


3	





\begin{flushleft}
PE-1
\end{flushleft}


(3-0-0) 3





\begin{flushleft}
PE-2
\end{flushleft}


(3-0-0) 3





\begin{flushleft}
PE-5
\end{flushleft}


(3-0-0) 3





\begin{flushleft}
PE-6
\end{flushleft}


(3-0-0) 3





\begin{flushleft}
Contact h/week
\end{flushleft}


\begin{flushleft}
L
\end{flushleft}





\begin{flushleft}
T
\end{flushleft}





\begin{flushleft}
P
\end{flushleft}





\begin{flushleft}
Total
\end{flushleft}





3	 0	


3	 0	


3	 0	


3	0	


3	 0	


3	 0	





0	 3


0	 3


0	 3


0	3


0	 3


0	 3





\begin{flushleft}
Credits
\end{flushleft}





\begin{flushleft}
COL702	 Advanced Data Structures	
\end{flushleft}


3	 0	


\begin{flushleft}
COL718	 Architecture of High Performance Computers	3	0	
\end{flushleft}


\begin{flushleft}
ELL737	 Flexible Electronics	
\end{flushleft}


3	0	


\begin{flushleft}
ELL742	 Introduction to MEMS Design	
\end{flushleft}


3	 0	


\begin{flushleft}
ELL743	 Photovoltaics	
\end{flushleft}


3	0	


\begin{flushleft}
ELL745	 Quantum Electronics	
\end{flushleft}


3	0	


\begin{flushleft}
ELL746	 Biomedical Electronics 	
\end{flushleft}


3	 0	


\begin{flushleft}
ELL747	 Active and Passive Filter Design	
\end{flushleft}


3	 0	


\begin{flushleft}
ELL830	 Issues in Deep Submicron VLSI Design	
\end{flushleft}


3	 0	


\begin{flushleft}
ELL831	 CAD for VLSI, MEMS, and Nanoassembly	 3	 0	
\end{flushleft}


\begin{flushleft}
ELL832	 Selected Topics in IEC-I	
\end{flushleft}


3	 0	


\begin{flushleft}
ELL833	 CMOS RF IC Design	
\end{flushleft}


3	 0	


\begin{flushleft}
ELL883	 Embedded Intelligence	
\end{flushleft}


3	0	


\begin{flushleft}
ELP831	 IEC Laboratory-I	
\end{flushleft}


0	0	


\begin{flushleft}
ELV830	 Special Module in Low Power IC Design	
\end{flushleft}


1	 0	


\begin{flushleft}
ELV831	 Special Module in VLSI Testing	
\end{flushleft}


1	 0	


\begin{flushleft}
JVD799	 Minor Project	
\end{flushleft}


0	 0	


\begin{flushleft}
JVD812	 Major Project-II	
\end{flushleft}


0	 0	


\begin{flushleft}
JVS801	 Independent Study	
\end{flushleft}


0	3	


\begin{flushleft}
MTL704	 Numerical Optimization	
\end{flushleft}


3	0	





\begin{flushleft}
Synthesis of Digital Systems	
\end{flushleft}


\begin{flushleft}
System Level Design and Modelling	
\end{flushleft}


\begin{flushleft}
Digital System Design Laboratory	
\end{flushleft}


\begin{flushleft}
Mixed Signal Circuit Design	
\end{flushleft}


\begin{flushleft}
Analog Integrated Circuits	
\end{flushleft}


\begin{flushleft}
Semiconductor Memory Design	
\end{flushleft}





\begin{flushleft}
Lecture
\end{flushleft}


\begin{flushleft}
courses
\end{flushleft}





\begin{flushleft}
ELL734	
\end{flushleft}


\begin{flushleft}
ELP736	
\end{flushleft}


\begin{flushleft}
JVD811	
\end{flushleft}


	





12





12





\begin{flushleft}
Summer
\end{flushleft}





\begin{flushleft}
JVD811
\end{flushleft}





\begin{flushleft}
III
\end{flushleft}


\begin{flushleft}
(OR)
\end{flushleft}





\begin{flushleft}
Major Project-I
\end{flushleft}





\begin{flushleft}
IV
\end{flushleft}





\begin{flushleft}
Major Project-II
\end{flushleft}





\begin{flushleft}
III
\end{flushleft}


\begin{flushleft}
(OR)
\end{flushleft}





(0-0-24) 12


\begin{flushleft}
JVD812
\end{flushleft}





(0-0-24) 12


\begin{flushleft}
PE-7
\end{flushleft}


(3-0-0) 3





\begin{flushleft}
PE-8
\end{flushleft}


(3-0-0) 3





12





\begin{flushleft}
Minor Project
\end{flushleft}





(0-0-12) 6





\begin{flushleft}
JVD811
\end{flushleft}





\begin{flushleft}
IV
\end{flushleft}





\begin{flushleft}
Major Project-I
\end{flushleft}





\begin{flushleft}
III
\end{flushleft}


\begin{flushleft}
(OR)
\end{flushleft}





\begin{flushleft}
Major Project-I
\end{flushleft}





\begin{flushleft}
IV
\end{flushleft}





\begin{flushleft}
JVD799
\end{flushleft}





(0-0-24) 12


\begin{flushleft}
JVD811
\end{flushleft}





(0-0-24) 12


\begin{flushleft}
PE-7
\end{flushleft}


(3-0-0) 3





\begin{flushleft}
PE-8
\end{flushleft}


(3-0-0) 3





\begin{flushleft}
PE-9
\end{flushleft}


(3-0-0) 3





\begin{flushleft}
PE-10
\end{flushleft}


(3-0-0) 3





12





\begin{flushleft}
Total = 48
\end{flushleft}


142





\begin{flushleft}
\newpage
10. COURSE DESCRIPTIONS
\end{flushleft}


\begin{flushleft}
The details about every course are given in this section.
\end{flushleft}


\begin{flushleft}
Information about each course includes course number, credits,
\end{flushleft}


\begin{flushleft}
L-T-P structure, Pre-requisites, overlapped courses and course
\end{flushleft}


\begin{flushleft}
contents.
\end{flushleft}


\begin{flushleft}
For some 700 and 800 level courses, the Pre-requisites have
\end{flushleft}


\begin{flushleft}
been explicitly indicated. Where these are not mentioned, the
\end{flushleft}


\begin{flushleft}
default Pre-requisites shall be applicable for UG students (see
\end{flushleft}


\begin{flushleft}
sections 2.6 and 3.11).
\end{flushleft}


\begin{flushleft}
For additional information see the website or contact the
\end{flushleft}


\begin{flushleft}
concerned course coordinator or head of the department/centre/
\end{flushleft}


\begin{flushleft}
school or the programme coordinator.
\end{flushleft}





\begin{flushleft}
\newpage
Department of Applied Mechanics
\end{flushleft}


\begin{flushleft}
APL100 Engineering Mechanics
\end{flushleft}


\begin{flushleft}
4 Credits (3-1-0)
\end{flushleft}


\begin{flushleft}
Kinematics, Statics, Equations of Motion, Rigid body dynamics,
\end{flushleft}


\begin{flushleft}
Introduction to variational mechanics.
\end{flushleft}





\begin{flushleft}
APL102 Introduction to Materials Science and
\end{flushleft}


\begin{flushleft}
Engineering
\end{flushleft}


\begin{flushleft}
4 Credits (3-0-2)
\end{flushleft}


\begin{flushleft}
Structure of Solids: atomic and inter-atomic bonding, crystal structure
\end{flushleft}


\begin{flushleft}
and imperfection in solids.
\end{flushleft}


\begin{flushleft}
Properties of Materials: Mechanical, chemical, electrical and magnetic
\end{flushleft}


\begin{flushleft}
properties of metals, ceramics and polymers.
\end{flushleft}


\begin{flushleft}
Processing of Materials: Thermodynamics basics, Phase diagrams and
\end{flushleft}


\begin{flushleft}
phase transformation of metallic systems, fabrication and processing
\end{flushleft}


\begin{flushleft}
of metals, polymers and ceramics.
\end{flushleft}


\begin{flushleft}
Performance of Materials: Creep, fatigue, failure and corrosion of metals,
\end{flushleft}


\begin{flushleft}
ceramics (including cement and concrete), polymers, and composites
\end{flushleft}


\begin{flushleft}
(including fiber reinforced structure, sandwich panels, and wood).
\end{flushleft}


\begin{flushleft}
Selection of Materials: selection of materials for various applications,
\end{flushleft}


\begin{flushleft}
materials selection charts, CSE software, Example case studies such
\end{flushleft}


\begin{flushleft}
as materials for large astronomical telescopes, springs, flywheels, safe
\end{flushleft}


\begin{flushleft}
pressure vessels and reactors.
\end{flushleft}


\begin{flushleft}
Laboratory: The behavior of different types of materials (e.g. metals,
\end{flushleft}


\begin{flushleft}
ceramics, composites, polymers) will be studied through carefully
\end{flushleft}


\begin{flushleft}
designed experiments. The fundamentals of structure and properties of
\end{flushleft}


\begin{flushleft}
various materials will be communicated through hands on experiments
\end{flushleft}


\begin{flushleft}
and model demonstration.
\end{flushleft}





\begin{flushleft}
APL103 Experimental Methods
\end{flushleft}


\begin{flushleft}
4 Credits (3-0-2)
\end{flushleft}


\begin{flushleft}
Experimental Analysis: Types of measurements and errors, Relative
\end{flushleft}


\begin{flushleft}
frequency distribution, Histogram, True value, Precision of measurement,
\end{flushleft}


\begin{flushleft}
Method of least squares, the curve fitting, General linear regression,Theory
\end{flushleft}


\begin{flushleft}
of errors, Binomial and Gaussian distribution, Chi-square test.
\end{flushleft}


\begin{flushleft}
Experimental Methods: Principles of Measurement, Basic Elements
\end{flushleft}


\begin{flushleft}
of a Measuring Device.
\end{flushleft}


\begin{flushleft}
Displacement measurement,Force and Torque Measurement,
\end{flushleft}


\begin{flushleft}
Temperature Measurement, Pressure Measurement, Fluid Velocity
\end{flushleft}


\begin{flushleft}
Measurement, Miscellaneous measurements.
\end{flushleft}


\begin{flushleft}
Dynamics of Measurements: Dynamic Response of a Measuring
\end{flushleft}


\begin{flushleft}
Instrument, Response to Transient and Periodic Signals, First and
\end{flushleft}


\begin{flushleft}
Second order systems as well as their Dynamic Response Characteristics.
\end{flushleft}


\begin{flushleft}
Laboratory : The experiments have been designed to understand
\end{flushleft}


\begin{flushleft}
Experimental Analysis physically. Laboratory will enable the students to
\end{flushleft}


\begin{flushleft}
apply various statistical methodologies (viz. Mean, Median, Mode, Std
\end{flushleft}


\begin{flushleft}
Dev. etc) to get the optimum output from the day to day Engineering
\end{flushleft}


\begin{flushleft}
life experiment.
\end{flushleft}





\begin{flushleft}
APL104 Solid Mechanics
\end{flushleft}


\begin{flushleft}
4 Credits (3-1-0)
\end{flushleft}


\begin{flushleft}
Pre-requisites: APL100
\end{flushleft}


\begin{flushleft}
Overlaps with: APL105, APL108
\end{flushleft}


\begin{flushleft}
Introduction, State of stress at a point, equations of motion, principal
\end{flushleft}


\begin{flushleft}
stress, maximum shear stress. Concept of strain, strain displacement
\end{flushleft}


\begin{flushleft}
relations, compatibility conditions, principal strains, transformation of
\end{flushleft}


\begin{flushleft}
stress/strain tensor, state of plane stress/strain. Constitutive relations,
\end{flushleft}


\begin{flushleft}
uniaxial tension test, idealized stress-strain diagrams, isotropic linear
\end{flushleft}


\begin{flushleft}
elastic, viscoelastic and elasto-plastic materials. Energy Methods.
\end{flushleft}


\begin{flushleft}
Uniaxial stress and strain analysis of bars, thermal stresses, Torsion,
\end{flushleft}


\begin{flushleft}
Bending and shear stresses in beams, deflection of beams, stability
\end{flushleft}


\begin{flushleft}
of equilibrium configuration.
\end{flushleft}





\begin{flushleft}
APL105 Mechanics of Solids and Fluids
\end{flushleft}


\begin{flushleft}
4 Credits (3-1-0)
\end{flushleft}


\begin{flushleft}
Pre-requisites: APL100
\end{flushleft}


\begin{flushleft}
Overlaps with: APL104, APL106, APL107, APL108
\end{flushleft}





\begin{flushleft}
and Definitions, Solids and Fluids, Internal and external forces
\end{flushleft}


\begin{flushleft}
on a fluid element. PROPERTIES OF FLUID: Rheological Equation
\end{flushleft}


\begin{flushleft}
and Classification of fluids, Normal and Shear Stresses, Concept of
\end{flushleft}


\begin{flushleft}
Pressure, pressure gradient. STATICS OF FLUIDS: Types of Forces
\end{flushleft}


\begin{flushleft}
on Fluid Element, Mechanics of Fluid at Rest and in rigid body
\end{flushleft}


\begin{flushleft}
motion, Manometry, forces on fully and partially submerged bodies,
\end{flushleft}


\begin{flushleft}
stability of a floating body. KINEMATICS OF FLUID MOTION: Types
\end{flushleft}


\begin{flushleft}
of fluid motion, Stream lines, Streak and path lines, Acceleration and
\end{flushleft}


\begin{flushleft}
Rotation of a fluid particle, Vorticity and Circulation, Stream Function,
\end{flushleft}


\begin{flushleft}
Irrotational flow and Velocity Potential function. DYNAMICS OF AN
\end{flushleft}


\begin{flushleft}
IDEAL FLUID: Continuity and Euler's Equations of Motion, Bernoulli
\end{flushleft}


\begin{flushleft}
Equation, Applications to Flow Measurement and other real flow
\end{flushleft}


\begin{flushleft}
problems. MECHANICS OF VISCOUS FLOW: Navier Stokes equations,
\end{flushleft}


\begin{flushleft}
exact solutions, Laminar flow through a pipe, Turbulent flow through
\end{flushleft}


\begin{flushleft}
a pipe, Friction factor, Applications to Pipe Networks. DIMENSIONAL
\end{flushleft}


\begin{flushleft}
ANALYSIS: Similarity of motion, Dimensionless numbers, Modeling of
\end{flushleft}


\begin{flushleft}
fluid flows, Applications. INTEGRAL ANALYSIS: Reynolds Transport
\end{flushleft}


\begin{flushleft}
Theorem, Control Volume Analysis.
\end{flushleft}


\begin{flushleft}
Solid Mechanics: State of stress at a point, equations of motion,
\end{flushleft}


\begin{flushleft}
principal stress, maximum shear stress. Concept of strain, strain
\end{flushleft}


\begin{flushleft}
displacement relations, compatibility conditions, principal strains,
\end{flushleft}


\begin{flushleft}
transformation of stress/strain tensor, state of plane stress/strain.
\end{flushleft}


\begin{flushleft}
Constitutive relations, uniaxial tension test, idealized stress-strain
\end{flushleft}


\begin{flushleft}
diagrams, isotropic linear elastic and elasto-plastic materials. Energy
\end{flushleft}


\begin{flushleft}
Methods. Uniaxial stress and strain analysis of bars, thermal stresses,
\end{flushleft}


\begin{flushleft}
Torsion, Bending, Stability of Equilibrium.
\end{flushleft}





\begin{flushleft}
APL106 Fluid Mechanics
\end{flushleft}


\begin{flushleft}
4 Credits (3-1-0)
\end{flushleft}


\begin{flushleft}
Pre-requisites: APL100
\end{flushleft}


\begin{flushleft}
Overlaps with: APL107, APL105
\end{flushleft}


\begin{flushleft}
Introduction to Fluids and the concept of viscosity, Flow visualization,
\end{flushleft}


\begin{flushleft}
Fluid Statics, Physical laws for a control volume including continuity,
\end{flushleft}


\begin{flushleft}
momentum and energy equations, Bernoulli equation, Differential
\end{flushleft}


\begin{flushleft}
equations of fluid motion, Navier Stokes equations, vorticity and
\end{flushleft}


\begin{flushleft}
potential flows, dimensional analysis and similitude, Boundary layer
\end{flushleft}


\begin{flushleft}
theory, 1-D compressible flow.
\end{flushleft}





\begin{flushleft}
APL107 Mechanics of Fluids
\end{flushleft}


\begin{flushleft}
5 Credits (3-1-2)
\end{flushleft}


\begin{flushleft}
Pre-requisites: APL100
\end{flushleft}


\begin{flushleft}
Overlaps with: APL106, APL105
\end{flushleft}


\begin{flushleft}
Introduction to Fluids and the concept of viscosity, Flow visualization,
\end{flushleft}


\begin{flushleft}
Fluid Statics, Physical laws for a control volume including continuity,
\end{flushleft}


\begin{flushleft}
momentum and energy equations, Bernoulli equation, Differential
\end{flushleft}


\begin{flushleft}
equations of fluid motion, Navier Stokes equations, vorticity and
\end{flushleft}


\begin{flushleft}
potential flows, dimensional analysis and similitude, Boundary layer
\end{flushleft}


\begin{flushleft}
theory, viscous flow in ducts and applications to turbomachinery.
\end{flushleft}


\begin{flushleft}
Laboratory experiments will demonstrate the concepts learnt in the
\end{flushleft}


\begin{flushleft}
theory and appreciation of their limitations.
\end{flushleft}





\begin{flushleft}
APL108 Mechanics of Solids
\end{flushleft}


\begin{flushleft}
5 Credits (3-1-2)
\end{flushleft}


\begin{flushleft}
Pre-requisites: APL100
\end{flushleft}


\begin{flushleft}
Overlaps with: APL104, APL105
\end{flushleft}


\begin{flushleft}
Introduction, State of stress at a point, equations of motion, principal
\end{flushleft}


\begin{flushleft}
stress, maximum shear stress. Concept of strain, strain displacement
\end{flushleft}


\begin{flushleft}
relations, compatibility conditions, principal strains, transformation of
\end{flushleft}


\begin{flushleft}
stress/strain tensor, state of plane stress/strain. Constitutive relations,
\end{flushleft}


\begin{flushleft}
uniaxial tension test, idealized stress-strain diagrams, isotropic linear
\end{flushleft}


\begin{flushleft}
elastic, viscoelastic and elasto-plastic materials. Energy Methods.
\end{flushleft}


\begin{flushleft}
Uniaxial stress and strain analysis of bars, thermal stresses, Torsion,
\end{flushleft}


\begin{flushleft}
Bending and shear stresses in beams, deflection of beams, stability
\end{flushleft}


\begin{flushleft}
of equilibrium configuration.
\end{flushleft}





\begin{flushleft}
APL190 Design Engineering
\end{flushleft}


\begin{flushleft}
4 Credits (3-0-2)
\end{flushleft}





\begin{flushleft}
Fluid Mechanics MATHEMATICAL PRELIMINARIES: Cartesian Tensors,
\end{flushleft}


\begin{flushleft}
Index Notation, Integral Theorems. INTRODUCTION: Basic Concepts
\end{flushleft}





\begin{flushleft}
Modern Design Cycle, Craftsman versus Designer, Need Analysis
\end{flushleft}


\begin{flushleft}
and Broad Engineering Specifications, Concept Design, Feasibility
\end{flushleft}





144





\begin{flushleft}
\newpage
Applied Mechanics
\end{flushleft}





\begin{flushleft}
study and Evaluation of alternatives, Engineering Economics,
\end{flushleft}


\begin{flushleft}
Modelling Techniques-Mathematical, Graphical, iconic, solid, Analysis
\end{flushleft}


\begin{flushleft}
and Simulation (FEM, Monte Carlo, CFD, Dimensional analysis,
\end{flushleft}


\begin{flushleft}
Experimental Techniques), Material Selection (strength, stiffness,
\end{flushleft}


\begin{flushleft}
fatigue life consideration), Manufacturing Processes and Design for
\end{flushleft}


\begin{flushleft}
Manufacture, Reliability/Availability/Maintainability, Sustainability
\end{flushleft}


\begin{flushleft}
and Environment, Safety, Ergonomics and Human Factors, Detailed
\end{flushleft}


\begin{flushleft}
Drawings/Assembly Drawings/ Assembly Instructions /Maintenance
\end{flushleft}


\begin{flushleft}
Manuals, Case Studies.
\end{flushleft}





\begin{flushleft}
APL300 Computational Mechanics
\end{flushleft}


\begin{flushleft}
4 Credits (3-0-2)
\end{flushleft}


\begin{flushleft}
Pre-requisites: APL104/APL105/APL106/APL107/APL108/
\end{flushleft}


\begin{flushleft}
CHL231
\end{flushleft}


\begin{flushleft}
Concept of continuum; introduction to stress, strain and rate of
\end{flushleft}


\begin{flushleft}
strain tensors; Principal stress and strains; Equation of equilibrium/
\end{flushleft}


\begin{flushleft}
motion in solid and fluid mechanics; lagrangian and eulerian view
\end{flushleft}


\begin{flushleft}
point; constitutive equations in the context of both solids and fluids;
\end{flushleft}


\begin{flushleft}
System of simultaneous linear and non-linear equations: how they
\end{flushleft}


\begin{flushleft}
arise in mechanics; Determination of constitute curves; interpolation
\end{flushleft}


\begin{flushleft}
techniques; Application of numerical integration and differentiation to
\end{flushleft}


\begin{flushleft}
axial vibration of bars and beams; solution techniques for boundary
\end{flushleft}


\begin{flushleft}
value problems arising in bending of beams, one dimensional fluid
\end{flushleft}


\begin{flushleft}
flows and boundary layer equations; stability analysis -- computation
\end{flushleft}


\begin{flushleft}
of eigenvalues; Direct and indirect methods of solution of linear
\end{flushleft}


\begin{flushleft}
equations; Emphasis will be on using the finite difference method
\end{flushleft}


\begin{flushleft}
(FDM) to solve problems in solid and fluid mechanics.
\end{flushleft}





\begin{flushleft}
APD310 Mini Project
\end{flushleft}


\begin{flushleft}
3 Credits (0-0-6)
\end{flushleft}


\begin{flushleft}
Pre-requisites: EC 50
\end{flushleft}


\begin{flushleft}
APL310 Constitutive Modeling
\end{flushleft}


\begin{flushleft}
4 Credits (3-0-2)
\end{flushleft}


\begin{flushleft}
Pre-requisites: APL104/APL105/APL108 EC 50
\end{flushleft}


\begin{flushleft}
Mathematical Preliminaries (scalar, vector, tensor operation)
\end{flushleft}


\begin{flushleft}
Thermodynamics (thermodynamical framework for constitutive
\end{flushleft}


\begin{flushleft}
modeling), Kinematics of Deformation \& Motion, Stree-strain principles,
\end{flushleft}


\begin{flushleft}
Elasticity, Anisotropy, viscoelasticity, multi-physical coupling effect,
\end{flushleft}


\begin{flushleft}
plasticity, viscoplasticity. Experimental: Experimental characterization,
\end{flushleft}


\begin{flushleft}
data analysis, Model fitting.
\end{flushleft}





\begin{flushleft}
APD311 Project
\end{flushleft}


\begin{flushleft}
4 Credits (0-0-8)
\end{flushleft}


\begin{flushleft}
Pre-requisites: EC 50, 12 credits of Minor Area in Computational
\end{flushleft}


\begin{flushleft}
Mechanics
\end{flushleft}


\begin{flushleft}
APL340 Chaos
\end{flushleft}


\begin{flushleft}
4 Credits (3-1-0)
\end{flushleft}


\begin{flushleft}
Pre-requisites: APL100 and EC 50
\end{flushleft}





\begin{flushleft}
Introduction to Turbulent Flows: Reynolds decomposition; Closure
\end{flushleft}


\begin{flushleft}
problem. Scaling arguments; energy cascade and vorticity dynamics.
\end{flushleft}





\begin{flushleft}
APL380 Biomechanics
\end{flushleft}


\begin{flushleft}
4 Credits (3-0-2)
\end{flushleft}


\begin{flushleft}
Pre-requisites: APL100 and EC 50
\end{flushleft}


\begin{flushleft}
Basics of rigid body mechanics, solid mechanics, and fluid mechanics
\end{flushleft}


\begin{flushleft}
applied in biological system; Basic introduction to anatomy and
\end{flushleft}


\begin{flushleft}
physiology; Mechanics of Human Motion; Mechanics of response of tissues
\end{flushleft}


\begin{flushleft}
including bones; Mechanics of Blood flow, Biosolid-fluid interaction.
\end{flushleft}


\begin{flushleft}
Computer Lab contents: Matlab Programming basics, Image processing
\end{flushleft}


\begin{flushleft}
basics, Design of Joint: Rigid Body Mechanics based approach, Matlab
\end{flushleft}


\begin{flushleft}
programming for bone or Aortic Tissue; Matlab programming for blood
\end{flushleft}


\begin{flushleft}
flow analysis.
\end{flushleft}





\begin{flushleft}
APL440 Parallel Processing in Computational
\end{flushleft}


\begin{flushleft}
4 Credits (3-0-2)
\end{flushleft}


\begin{flushleft}
Pre-requisites: APL310
\end{flushleft}


\begin{flushleft}
Introduction to multi-processor, multi-core, multi-threaded processing
\end{flushleft}


\begin{flushleft}
and their clusters, GPUs and CUDA programing, Introduction to
\end{flushleft}


\begin{flushleft}
parallel processing hardware and software, Open MP, MPI, MPICH,
\end{flushleft}


\begin{flushleft}
HPC / Clustering tools and software suits.
\end{flushleft}


\begin{flushleft}
Exploring parallelism in solid/fluid mechanics problems and formulation
\end{flushleft}


\begin{flushleft}
of numerical methods, Partitioning and divide-and-conquer strategies,
\end{flushleft}


\begin{flushleft}
Parallel algorithms for solving dynamical and non-linear systems,
\end{flushleft}


\begin{flushleft}
Finite difference and Finite element analysis of plate and shells,
\end{flushleft}


\begin{flushleft}
Finite elements in fluids, Reduced integration patch test, Dynamic
\end{flushleft}


\begin{flushleft}
FE analysis, Geometrically nonlinear problems, Material nonlinearity,
\end{flushleft}


\begin{flushleft}
Automated mesh generation, Pre and post processing, Solid fluid
\end{flushleft}


\begin{flushleft}
interaction problems, Efficient solution technique-PCG, Domain
\end{flushleft}


\begin{flushleft}
decomposition, Point source method, Boundary element method, Aero
\end{flushleft}


\begin{flushleft}
elastic flutter, Other special topics.
\end{flushleft}





\begin{flushleft}
APL700 Experimental Methods for Solids and Fluids
\end{flushleft}


\begin{flushleft}
3 Credits (2-0-2)
\end{flushleft}


\begin{flushleft}
Types of Measurement and errors, Internal and external estimates of
\end{flushleft}


\begin{flushleft}
errors, Relative frequency distribution, Histogram, True Value, Precision
\end{flushleft}


\begin{flushleft}
of measurement, Best estimate of true value and standard deviation,
\end{flushleft}


\begin{flushleft}
Combination of measurements, accuracy of the mean, significant
\end{flushleft}


\begin{flushleft}
digits. Methods of least squares and its application to the calculation
\end{flushleft}


\begin{flushleft}
of best estimate of the true value, the curve fitting, general linear
\end{flushleft}


\begin{flushleft}
regression, comparison and combination of measurements. Extensions
\end{flushleft}


\begin{flushleft}
of least square method, Principle of maximum likelihood, and goodness
\end{flushleft}


\begin{flushleft}
of fit, chi-square test, Dynamic response of a measuring instrument,
\end{flushleft}


\begin{flushleft}
Response to transient and periodic signals, first and second order
\end{flushleft}


\begin{flushleft}
systems as well as their dynamics response characteristics.
\end{flushleft}





\begin{flushleft}
AML701 Engineering Mathematics \& Mechanics
\end{flushleft}


\begin{flushleft}
3 Credits (3-0-0)
\end{flushleft}





\begin{flushleft}
Introduction to linear systems and its classification, Fixed point and
\end{flushleft}


\begin{flushleft}
stability, linear stability analysis, Linearization of nonlinear systems,
\end{flushleft}


\begin{flushleft}
Types of bifurcation and examples, imperfect bifurcations and
\end{flushleft}


\begin{flushleft}
catastrophes, Coupled oscillators and quasiperiodicity, Poincare Maps,
\end{flushleft}


\begin{flushleft}
Introduction to Chaos, Lorenz equation, one-dimensional map, fractals.
\end{flushleft}





\begin{flushleft}
APL360 Engineering Fluid Flow
\end{flushleft}


\begin{flushleft}
4 Credits (3-0-2)
\end{flushleft}


\begin{flushleft}
Pre-requisites: APL100 and EC 50
\end{flushleft}


\begin{flushleft}
Introduction to Cartesian tensors.
\end{flushleft}


\begin{flushleft}
The Navier Stokes Equations: Derivation via continuum mechanics;
\end{flushleft}


\begin{flushleft}
Boundary conditions; surface tension; Exact Solutions; Steady and
\end{flushleft}


\begin{flushleft}
unsteady problems; Similarity solutions.
\end{flushleft}


\begin{flushleft}
Laminar Boundary-Layers: Order of magnitude analysis; Blasius
\end{flushleft}


\begin{flushleft}
solution; Von K\'{a}rm\'{a}n Momentum Integral; Free-shear flows.
\end{flushleft}


\begin{flushleft}
Low Reynolds Number flows: Stokes Flow; Oseen's Correction.
\end{flushleft}


\begin{flushleft}
Lubrication Approximation.
\end{flushleft}


\begin{flushleft}
Hydrodynamic Stability Theory: Capillary Instability; Orr-Sommerfeld
\end{flushleft}


\begin{flushleft}
Equation; Rayleigh Equation. Inflection Point Criterion. Rotating flows.
\end{flushleft}





\begin{flushleft}
Partial differential equations. Fourier Series and transforms. Calculus
\end{flushleft}


\begin{flushleft}
of variations. Newtonian and Lagrangian mechanics. Variational and
\end{flushleft}


\begin{flushleft}
Hamiltonian mechanics.
\end{flushleft}





\begin{flushleft}
APL701 Continuum Mechanics
\end{flushleft}


\begin{flushleft}
3 Credits (3-0-0)
\end{flushleft}


\begin{flushleft}
Concept of continuum, kinematics of deformation, concept of stress
\end{flushleft}


\begin{flushleft}
and strain tensor -- their transformation and decomposition, finite
\end{flushleft}


\begin{flushleft}
strain tensor and its linearization with examples, rate of deformation
\end{flushleft}


\begin{flushleft}
tensor -- velocity gradient and spin tensor, derivation of conservation
\end{flushleft}


\begin{flushleft}
laws -- mass continuity, linear and angular momentum conservation,
\end{flushleft}


\begin{flushleft}
derivation of linear equations of elasticity and Navier Stokes equations
\end{flushleft}


\begin{flushleft}
in both cartesian and polar co-ordinates, principle of minimum
\end{flushleft}


\begin{flushleft}
potential energy, virtual work theorem, uniqueness and reciprocal
\end{flushleft}


\begin{flushleft}
theorem, constitutive laws for linearly elastic solids and newtonian
\end{flushleft}


\begin{flushleft}
viscous fluids, incompressible case, applications in solid and fluid
\end{flushleft}


\begin{flushleft}
mechanics problems.
\end{flushleft}





\begin{flushleft}
AML702 Applied Computational Methods
\end{flushleft}


\begin{flushleft}
4 Credits (3-0-2)
\end{flushleft}


\begin{flushleft}
Algorithms. Methods of undetermined coefficients. Numerical
\end{flushleft}





145





\begin{flushleft}
\newpage
Applied Mechanics
\end{flushleft}





\begin{flushleft}
differentiation and integration. Solution of ordinary differential
\end{flushleft}


\begin{flushleft}
equations. Solution of linear and non-linear algebraic equations.
\end{flushleft}


\begin{flushleft}
Boundary value problems and initial value problems. Numerical solution
\end{flushleft}


\begin{flushleft}
of partial differential equations. Eigenvalue problems.
\end{flushleft}





\begin{flushleft}
APL703 Engineering Mathematics and Computation
\end{flushleft}


\begin{flushleft}
4 Credits (3-0-2)
\end{flushleft}


\begin{flushleft}
Tensors, Vector Calculus; Linear Algebra -- Solution of Linear
\end{flushleft}


\begin{flushleft}
Systems, Eigenvalue Problems; Variational calculus; Fourier Series
\end{flushleft}


\begin{flushleft}
and transform, Analytical and Numerical Solution methods of ODEs,
\end{flushleft}


\begin{flushleft}
Partial Differential Equations -- properties and solution techniques,
\end{flushleft}


\begin{flushleft}
Probability and Statistics.
\end{flushleft}





\begin{flushleft}
APL705 Finite Element Method
\end{flushleft}


\begin{flushleft}
4 Credits (3-0-2)
\end{flushleft}


\begin{flushleft}
Pre-requisites: EC 75
\end{flushleft}


\begin{flushleft}
Strong and weak forms of governing differential equations, and
\end{flushleft}


\begin{flushleft}
their equivalence, Weighted residual and variational approaches.
\end{flushleft}


\begin{flushleft}
Ritz method. Discretization of weak form and boundary conditions.
\end{flushleft}


\begin{flushleft}
Convergence. Bar and beam elements. Truss and frame problems,
\end{flushleft}


\begin{flushleft}
Isoparametric formulation. Plane strain, plane stress and axi-symmetric
\end{flushleft}


\begin{flushleft}
problems, 3D elasticity problems, one and two dimensional heat
\end{flushleft}


\begin{flushleft}
transfer. Formulation of dynamics problems. Laboratory work on solid
\end{flushleft}


\begin{flushleft}
mechanics and heat transfer problems.
\end{flushleft}





\begin{flushleft}
AML706 Finite Element Methods and its Applications
\end{flushleft}


\begin{flushleft}
to Marine Structures
\end{flushleft}


\begin{flushleft}
3 Credits (3-0-0)
\end{flushleft}


\begin{flushleft}
Introduction to FEM. Variational methods. Element types and properties.
\end{flushleft}


\begin{flushleft}
Boundary conditions. Stress-strain determination. Solution techniques.
\end{flushleft}


\begin{flushleft}
Mesh refinement. Convergence criterion. Frames, beams and axial
\end{flushleft}


\begin{flushleft}
element. Plane stress. Plane strain. Axisymmetric problems. Plate
\end{flushleft}


\begin{flushleft}
bending. Fluid mechanics and heat transfer. Modules modelling and elastic
\end{flushleft}


\begin{flushleft}
analysis. Super elements. Structural instability of frames and beams.
\end{flushleft}





\begin{flushleft}
APL710 Computer Aided Design
\end{flushleft}


\begin{flushleft}
4 Credits (3-0-2)
\end{flushleft}


\begin{flushleft}
Pre-requisites: EC 75
\end{flushleft}


\begin{flushleft}
Principles of computer aided design, Computer graphics fundamentals,
\end{flushleft}


\begin{flushleft}
2D and 3D Transformations and projections, Plane Curves, Space
\end{flushleft}


\begin{flushleft}
Curves, Synthetic curves. Analytical and parametric surfaces, Synthetic
\end{flushleft}


\begin{flushleft}
surfaces, Solid Modeling basics, Solid modeling techniques and
\end{flushleft}


\begin{flushleft}
schemes, Half-spaces, Boundary Representation (B-rep), Constructive
\end{flushleft}


\begin{flushleft}
Solid Geometry (CSG), Sweep Modeling, Analytical Solid Modeling,
\end{flushleft}


\begin{flushleft}
Visual Realism, hidden lines and surface.
\end{flushleft}





\begin{flushleft}
APL711 Advanced Fluid Mechanics
\end{flushleft}


\begin{flushleft}
3 Credits (3-0-0)
\end{flushleft}


\begin{flushleft}
Mathematical Preliminaries, Kinematics, Navier Stokes equations and
\end{flushleft}


\begin{flushleft}
some standard solutions, Low Reynolds number flows and Lubrication,
\end{flushleft}


\begin{flushleft}
Vorticity dynamics, Introduction to boundary layers, Hydrodynamic
\end{flushleft}


\begin{flushleft}
stability, 1-D compressible flows.
\end{flushleft}





\begin{flushleft}
AML713 Applied Fluid Mechanics
\end{flushleft}


\begin{flushleft}
4 Credits (3-1-0)
\end{flushleft}


\begin{flushleft}
Basic equations of fluid motion, Dynamics of ideal fluid motion,
\end{flushleft}


\begin{flushleft}
Generalised Bernoulli equation and special cases, Governing
\end{flushleft}


\begin{flushleft}
equations for viscous fluid flows, creeping fluid flows, Boundary layer
\end{flushleft}


\begin{flushleft}
approximation, Transition to turbulence, Fundamentals of turbulent
\end{flushleft}


\begin{flushleft}
flow, turbulent boundary layer over a flat plate.
\end{flushleft}





\begin{flushleft}
turbulent flows. Reynolds decomposition and the closure problem,
\end{flushleft}


\begin{flushleft}
estimates of the Reynolds stress, comparison with the kinetic theory
\end{flushleft}


\begin{flushleft}
of gases. Dynamics of turbulence, balance of kinetic energy, vorticity
\end{flushleft}


\begin{flushleft}
dynamics, scalar fluctuations. Free shear flows: jets, wakes and mixing
\end{flushleft}


\begin{flushleft}
layers. Wall bounded flows: channel and pipe flows, boundary layers.
\end{flushleft}


\begin{flushleft}
Kolmogorov hypotheses; probability density function, characteristic
\end{flushleft}


\begin{flushleft}
function and moments; structure and correlation functions; energy
\end{flushleft}


\begin{flushleft}
spectra, intermittency. Turbulent transport and dissipation.
\end{flushleft}





\begin{flushleft}
APL716 Fluid Transportation Systems
\end{flushleft}


\begin{flushleft}
3 Credits (3-0-0)
\end{flushleft}


\begin{flushleft}
Mechanism of transportation of materials by fluid flow, rheology and
\end{flushleft}


\begin{flushleft}
classification of complex mixtures, fundamentals of two-phase flow,
\end{flushleft}


\begin{flushleft}
Phase separation and settling behavior, Slurry Pipeline Transportation,
\end{flushleft}


\begin{flushleft}
Design methods, terminal facilities, pipe protection, pneumatic
\end{flushleft}


\begin{flushleft}
conveying, pneumocapsule and hydrocapsule pipelines, metrology
\end{flushleft}


\begin{flushleft}
associated with pipelines.
\end{flushleft}





\begin{flushleft}
APL720 Computational Fluid Dynamics
\end{flushleft}


\begin{flushleft}
4 Credits (3-0-2)
\end{flushleft}


\begin{flushleft}
Pre-requisites: EC 75
\end{flushleft}


\begin{flushleft}
Review of governing equations for fluid flow, finite volume method
\end{flushleft}


\begin{flushleft}
and its application to steady 1-D, 2-D and 3-D convection-diffusion
\end{flushleft}


\begin{flushleft}
problems, extension of FVM to unsteady 1-D, 2-D and 3-D convection
\end{flushleft}


\begin{flushleft}
diffusion problems, pressure-velocity coupling, staggered and
\end{flushleft}


\begin{flushleft}
colocated grids, solution of discretized equations, physical description
\end{flushleft}


\begin{flushleft}
of turbulence, Reynolds-Averaged Navier-Stokes equations, closure
\end{flushleft}


\begin{flushleft}
problem; RANS based turbulence models; DNS and LES.
\end{flushleft}





\begin{flushleft}
AML732 Ship Resistance \& Propulsion
\end{flushleft}


\begin{flushleft}
3 Credits (3-0-0)
\end{flushleft}


\begin{flushleft}
Elementary theory of elasticity and plasticity. Theory of plates.
\end{flushleft}


\begin{flushleft}
Instability of rectangular plates. Stiffened plates. Anisotropic plates.
\end{flushleft}





\begin{flushleft}
AML733 Dynamics
\end{flushleft}


\begin{flushleft}
3 Credits (3-0-0)
\end{flushleft}


\begin{flushleft}
Single degree freedom system. Multidegree freedom system. Numerical
\end{flushleft}


\begin{flushleft}
methods. Holzer-type problem geared and branched systems. Euler's
\end{flushleft}


\begin{flushleft}
equation for beams. Torsional vibrations. Continuous systems.
\end{flushleft}


\begin{flushleft}
Lagrange's equations. Balancing of shaft. Self excited vibration.
\end{flushleft}





\begin{flushleft}
APL734 Advanced Dynamics
\end{flushleft}


\begin{flushleft}
3 Credits (3-0-0)
\end{flushleft}


\begin{flushleft}
Single Degrees of Freedom systems, Multi-degree of freedom systems,
\end{flushleft}


\begin{flushleft}
Response spectrum, Time integration schemes, Lagrange's equations,
\end{flushleft}


\begin{flushleft}
Principle of virtual work, continuous system.
\end{flushleft}





\begin{flushleft}
APL736 Multiscale Modeling of Crystalline Materials
\end{flushleft}


\begin{flushleft}
4 Credits (3-0-2)
\end{flushleft}


\begin{flushleft}
Pre-requisites: EC 75
\end{flushleft}


\begin{flushleft}
Review of continuum mechanics, material symmetry, thermodynamics
\end{flushleft}


\begin{flushleft}
and constitutive relations, symmetry in crystals, empirical and semiempirical models of inter-atomic potential, molecular statics and
\end{flushleft}


\begin{flushleft}
dynamics, finite temperature effects in molecular systems, probabilistic
\end{flushleft}


\begin{flushleft}
behavior of material characteristics at macro scale, multiscale methods
\end{flushleft}


\begin{flushleft}
- Cauchy-Born rule and Quasi-continuum method, Mechanics of helical
\end{flushleft}


\begin{flushleft}
nanostructures (e.g., carbon nanotubes, DNA, polymers), Bending and
\end{flushleft}


\begin{flushleft}
twisting of nanotubes and nanorods.
\end{flushleft}





\begin{flushleft}
APL713 Turbulence and its Modeling
\end{flushleft}


\begin{flushleft}
3 Credits (3-0-0)
\end{flushleft}





\begin{flushleft}
Computer Lab contents: Programming molecular statics and molecular
\end{flushleft}


\begin{flushleft}
dynamics, molecular statics via conjugate gradient minimization and
\end{flushleft}


\begin{flushleft}
Newton-Raphson method, Monte Carlo simulation, Implementation of
\end{flushleft}


\begin{flushleft}
Cauchy-Born rule and Quasi-continuum method, Exposure to LAMMPS
\end{flushleft}


\begin{flushleft}
and AMBER.
\end{flushleft}





\begin{flushleft}
Nature of turbulence, Governing equations, Fourier, Lagrnagian
\end{flushleft}


\begin{flushleft}
and Eulerian description of turbulence, Statistical description
\end{flushleft}


\begin{flushleft}
of turbulence, Kolmogorov's hypotheses, turbulence processes,
\end{flushleft}


\begin{flushleft}
turbulence closure modelling.
\end{flushleft}





\begin{flushleft}
APL740 Mechanics of Biological Cells
\end{flushleft}


\begin{flushleft}
4 Credits (3-0-2)
\end{flushleft}


\begin{flushleft}
Pre-requisites: EC 75
\end{flushleft}





\begin{flushleft}
APL715 Physics of Turbulent Flows
\end{flushleft}


\begin{flushleft}
3 Credits (3-0-0)
\end{flushleft}





\begin{flushleft}
Theoretical Part: Basic Introduction to mechano-biology, Concept of
\end{flushleft}


\begin{flushleft}
Length Scale, Mechanical Forces, Mass, Stiffness and Damping of Proteins,
\end{flushleft}


\begin{flushleft}
Thermal Forces and Diffusion, Chemical Forces, Polymer Mechanics.
\end{flushleft}





\begin{flushleft}
Introduction, nature of turbulence, methods of analysis, scales of
\end{flushleft}





\begin{flushleft}
Intracellular Mechanics: Structures of Cytoskeleton Filaments, Dynamics
\end{flushleft}





146





\begin{flushleft}
\newpage
Applied Mechanics
\end{flushleft}





\begin{flushleft}
of Cell Filaments, Molecular motors, Introduction to Entropic Elasticity
\end{flushleft}


\begin{flushleft}
and Persistence Length, Force Generation by Cytoskeleton Filaments.
\end{flushleft}


\begin{flushleft}
Extracellular Mechanics: The Extracellular matrix (ECM), cell-ECM
\end{flushleft}


\begin{flushleft}
Interactions, Cell Migration, Forces and Adhesion.
\end{flushleft}


\begin{flushleft}
Tissue Mechanics: Cell-cell Assemblies,Tissue Material Behavior,
\end{flushleft}


\begin{flushleft}
Introduction to Linear Viscoelasticity, Concept of Constitutive Modeling.
\end{flushleft}


\begin{flushleft}
Experimental Part: Different Experimental Methods for Probing Cell
\end{flushleft}


\begin{flushleft}
Mechanical Properties. Intro to indentation, aspiration, twezeer,
\end{flushleft}


\begin{flushleft}
Nanopatterened platform based techniques etc.
\end{flushleft}





\begin{flushleft}
APL750 Modern Engineering Materials
\end{flushleft}


\begin{flushleft}
3 Credits (3-0-0)
\end{flushleft}


\begin{flushleft}
Pre-requisites: EC 75
\end{flushleft}


\begin{flushleft}
Elastic moduli, coefficient of thermal expansion: how properties
\end{flushleft}


\begin{flushleft}
are related with the bonding between the atoms, packing of atoms
\end{flushleft}


\begin{flushleft}
in solids, crystal structure, Plastic deformation of materials: yield
\end{flushleft}


\begin{flushleft}
strength, tensile strength, ductility and toughness of materials,
\end{flushleft}


\begin{flushleft}
perfect crystal, role of dislocations, strengthening methods, continuum
\end{flushleft}


\begin{flushleft}
aspects of plastic flow, Fatigue, fracture and creep of materials, ductile
\end{flushleft}


\begin{flushleft}
and brittle failure, micromechanism of failure, fatigue failure, Creep
\end{flushleft}


\begin{flushleft}
deformation and failure, mechanism of creep, Oxidation and corrosion
\end{flushleft}


\begin{flushleft}
of materials, carbon steels, alloy steels, TRIP steel, Dual phase steel,
\end{flushleft}


\begin{flushleft}
Bainitic steel, Martensitic steel, aluminum alloys, titanium alloys,
\end{flushleft}


\begin{flushleft}
carbon nanotubes, structure and properties of novel engineering
\end{flushleft}


\begin{flushleft}
materials: Composite materials, hybrid materials, metal foams,
\end{flushleft}


\begin{flushleft}
nanocrystalline materials, smart materials, case studies of materials
\end{flushleft}


\begin{flushleft}
applications in automotive, aerospace, power generation sectors etc.
\end{flushleft}





\begin{flushleft}
AML751 Materials for Marine Vehicles
\end{flushleft}


\begin{flushleft}
3 Credits (3-0-0)
\end{flushleft}


\begin{flushleft}
Corrosion. Selection of materials. Brittle fracture techniques.
\end{flushleft}


\begin{flushleft}
Introduction of fracture mechanics. Fatigue. Non-destructive testing.
\end{flushleft}





\begin{flushleft}
APL753 Properties and Selection of Engineering Materials
\end{flushleft}


\begin{flushleft}
3 Credits (3-0-0)
\end{flushleft}


\begin{flushleft}
Pre-requisites: EC 75
\end{flushleft}


\begin{flushleft}
Historical evolution of engineering materials, evolution of materials
\end{flushleft}


\begin{flushleft}
in products, Engineering materials and their properties: families of
\end{flushleft}


\begin{flushleft}
engineering materials, materials information for design, materials
\end{flushleft}


\begin{flushleft}
properties, Materials property chart: exploring materials properties,
\end{flushleft}


\begin{flushleft}
materials property charts e.g. the modulus-density chart, the
\end{flushleft}


\begin{flushleft}
strength-density chart, the fracture toughness-modulus chart, thermal
\end{flushleft}


\begin{flushleft}
conductivity-electrical resistivity chart, Materials selection-the basics:
\end{flushleft}


\begin{flushleft}
the selection strategy, materials indices, the selection procedure,
\end{flushleft}


\begin{flushleft}
Multiple constraints and conflicting objectives: selection with multiple
\end{flushleft}


\begin{flushleft}
constraints, conflicting objectives, Selection of materials and shape:
\end{flushleft}


\begin{flushleft}
shape factors, limits to shape efficiency, exploring the materials shape
\end{flushleft}


\begin{flushleft}
combinations, materials indices that include shape, architectured
\end{flushleft}


\begin{flushleft}
materials, Processes and process selection: classification of processes:
\end{flushleft}


\begin{flushleft}
shaping, joining and finishing, processing for properties, process
\end{flushleft}


\begin{flushleft}
selection, ranking process cost, Designing hybrid materials: holes in
\end{flushleft}


\begin{flushleft}
materials property space, composites, sandwich structures, cellular
\end{flushleft}


\begin{flushleft}
structures, segmented structures, case studies.
\end{flushleft}





\begin{flushleft}
APL756 Microstructural Characterization of Materials
\end{flushleft}


\begin{flushleft}
4 Credits (3-0-2)
\end{flushleft}


\begin{flushleft}
Pre-requisites: EC 75
\end{flushleft}


\begin{flushleft}
The concept of microstructure, diffraction analysis of crystal structure:
\end{flushleft}


\begin{flushleft}
X-ray and electron diffraction, optical microscopy, transmission electron
\end{flushleft}


\begin{flushleft}
microscopy, scanning electron microscopy, micro-analysis in electron
\end{flushleft}


\begin{flushleft}
microscopy, scanning probe microscopy and related techniques,
\end{flushleft}


\begin{flushleft}
chemical analysis of surface composition, quantitative and tomographic
\end{flushleft}


\begin{flushleft}
analysis of microstuctures.
\end{flushleft}





\begin{flushleft}
APL759 Phase Transformations
\end{flushleft}


\begin{flushleft}
3 Credits (3-0-0)
\end{flushleft}


\begin{flushleft}
Pre-requisites: EC 75
\end{flushleft}


\begin{flushleft}
Introduction and classification of phase transformations. Difussion in
\end{flushleft}


\begin{flushleft}
Solid: Fick's 1st and 2nd laws, The Kirkendall Effect, Darken's analysis,
\end{flushleft}


\begin{flushleft}
various diffusion mechanicsms. Thermodynamics of transformations:
\end{flushleft}





\begin{flushleft}
Free Energy of solid solutions, order of transformation, driving force
\end{flushleft}


\begin{flushleft}
for first order transformation with and without composition change,
\end{flushleft}


\begin{flushleft}
Second Order transformation, Spinodal decomposition. Nucleation
\end{flushleft}


\begin{flushleft}
kinetics: Homogeneous and heterogeneous nucleation, homogeneous
\end{flushleft}


\begin{flushleft}
nucleation with composition change, Heterogeneous nucleation, Strain
\end{flushleft}


\begin{flushleft}
energy effects. Growth Kinetics: Diffusion-controlled and Interfacecontrolled growth. Overall transformation kinetics, Johnson-MehlAvarami model. Particle coarsening. Recovery, Recrystallization and
\end{flushleft}


\begin{flushleft}
Grain Growth. Diffusionless transformation: Martensitic transformation.
\end{flushleft}


\begin{flushleft}
Solidification. Morphological instability of a Solid-liquid interface.
\end{flushleft}





\begin{flushleft}
APL763 Micro \& Nanoscale Mechanical Behaviour of
\end{flushleft}


\begin{flushleft}
Materials
\end{flushleft}


\begin{flushleft}
4 Credits (3-0-2)
\end{flushleft}


\begin{flushleft}
Pre-requisites: EC 75
\end{flushleft}


\begin{flushleft}
Elastic anisotropy of crystalline materials, defects in crystals: point
\end{flushleft}


\begin{flushleft}
defects and interfaces, dislocations and analysis of plasticity, geometric
\end{flushleft}


\begin{flushleft}
and energetic aspects of dislocations, microscale mechanisms of
\end{flushleft}


\begin{flushleft}
plastic deformation such as slip and twinning, single and polycrystal
\end{flushleft}


\begin{flushleft}
deformation, crystallographic textures, theory of work hardening in
\end{flushleft}


\begin{flushleft}
crystals, strengthening mechanisms in crystals, nanoscale testing of
\end{flushleft}


\begin{flushleft}
materials: in-situ SEM/TEM, nanoindentation, nano-wear, correlation
\end{flushleft}


\begin{flushleft}
of nanoscale measured response to macroscale response of materials.
\end{flushleft}





\begin{flushleft}
APL764 Biomaterials
\end{flushleft}


\begin{flushleft}
3 Credits (3-0-0)
\end{flushleft}


\begin{flushleft}
Pre-requisites: EC 75
\end{flushleft}


\begin{flushleft}
Introduction and history of biomaterials; Basic classes of engineering
\end{flushleft}


\begin{flushleft}
materials and structure property correlation; Structure and property
\end{flushleft}


\begin{flushleft}
of cells and tissues; Property requirement of biomaterials including
\end{flushleft}


\begin{flushleft}
biocompatibility, and biobegrability; Basic types of biomaterials;
\end{flushleft}


\begin{flushleft}
Mechanical testing of biomaterials; application of biomaterials
\end{flushleft}


\begin{flushleft}
(orthopedic, cardiovascular, dental) including detailed case study,
\end{flushleft}


\begin{flushleft}
Materials for biomedical devices and packaging (electronic interfacing etc.)
\end{flushleft}





\begin{flushleft}
APL765 Fracture Mechanics
\end{flushleft}


\begin{flushleft}
3 Credits (3-0-0)
\end{flushleft}


\begin{flushleft}
Pre-requisites: EC 75
\end{flushleft}


\begin{flushleft}
Fracture: an overview, theoretical cohesive strength, defect population
\end{flushleft}


\begin{flushleft}
in solids, stress concentration factor, notch strengthening, elements
\end{flushleft}


\begin{flushleft}
of fracture mechanics, Grifiths crack theory, stress analysis of crack,
\end{flushleft}


\begin{flushleft}
energy and stress field approaches, plane strain and plane stress
\end{flushleft}


\begin{flushleft}
fracture toughness testing, crack opening displacement, elastic-plastic
\end{flushleft}


\begin{flushleft}
analysis, J-integral, ductile-brittle transition, impact energy fracture
\end{flushleft}


\begin{flushleft}
toughness correlation, microstructural aspects of fracture toughness,
\end{flushleft}


\begin{flushleft}
environmental assisted cracking, cyclic stress and strain fatigue, fatigue
\end{flushleft}


\begin{flushleft}
crack propagation, analysis of engineering failures.
\end{flushleft}





\begin{flushleft}
APL767 Engineering Failure Analysis and Prevention
\end{flushleft}


\begin{flushleft}
3 Credits (3-0-0)
\end{flushleft}


\begin{flushleft}
Pre-requisites: EC 75
\end{flushleft}


\begin{flushleft}
Common causes of failure, principles of failure analysis, fracture
\end{flushleft}


\begin{flushleft}
mechanics approach to failure problems, techniques of failure analysis,
\end{flushleft}


\begin{flushleft}
service failure mechanisms, ductile and brittle fracture, fatigue failure,
\end{flushleft}


\begin{flushleft}
wear failure, hydrogen induced failure, environment induced failures,
\end{flushleft}


\begin{flushleft}
high temperature failure, faulty heat treatment and design failures,
\end{flushleft}


\begin{flushleft}
processing failure (forging, casting, machining etc.), failure problems
\end{flushleft}


\begin{flushleft}
in joints and weldments, case studies for failure analysis of structural
\end{flushleft}


\begin{flushleft}
components and mechanical system.
\end{flushleft}





\begin{flushleft}
APL771 Design Optimization and Decision Theory
\end{flushleft}


\begin{flushleft}
3 Credits (3-0-0)
\end{flushleft}


\begin{flushleft}
Introduction, classification of optimization problems, single variable
\end{flushleft}


\begin{flushleft}
and multi variable unconstrained optimization problems, constrained
\end{flushleft}


\begin{flushleft}
optimization, integer programming, genetic algorithms and simulated
\end{flushleft}


\begin{flushleft}
annealing, review of probability theory, decision theory.
\end{flushleft}





\begin{flushleft}
APL774 Modeling \& Analysis of Mechanical Systems
\end{flushleft}


\begin{flushleft}
3 Credits (3-0-0)
\end{flushleft}


\begin{flushleft}
Introduction, constitutive modeling of elastic orthotropic, elasto-plastic
\end{flushleft}


\begin{flushleft}
isotropic, and viscoelastic isotropic solids. Plane stress problems in polar
\end{flushleft}





147





\begin{flushleft}
\newpage
Applied Mechanics
\end{flushleft}





\begin{flushleft}
coordinate system, bending of rectangular plates-Navier and Levy's
\end{flushleft}


\begin{flushleft}
solution, bending of circular plates, membrane theory of shells, bending
\end{flushleft}


\begin{flushleft}
of cylindrical shells, vibration and buckling of rectangular plates. Flow
\end{flushleft}


\begin{flushleft}
measurement, velocity measurement, multi-hole probes and optical
\end{flushleft}


\begin{flushleft}
measurements. External flows, boundary layers (laminar and turbulent);
\end{flushleft}


\begin{flushleft}
estimation of lift and drag, internal flows, application to pipe lines.
\end{flushleft}





\begin{flushleft}
APL775 Design Methods
\end{flushleft}


\begin{flushleft}
3 Credits (3-0-0)
\end{flushleft}


\begin{flushleft}
Introduction, design cycle, need analysis, product specifications,
\end{flushleft}


\begin{flushleft}
quality function deployment (QFD), concept generation, concept
\end{flushleft}


\begin{flushleft}
selection, TRIZ, concept testing, preliminary design, architecture
\end{flushleft}


\begin{flushleft}
design. Modeling, sensitivity, compatibility, stability analyses. Design
\end{flushleft}


\begin{flushleft}
for manufacturing, material, maintenance and safety. Industrial design,
\end{flushleft}


\begin{flushleft}
detailed design, prototype/model testing. Axiomatic design. Detailed
\end{flushleft}


\begin{flushleft}
Drawings/Assembly Drawings/ Assembly Instructions /Maintenance
\end{flushleft}


\begin{flushleft}
Manuals, Case Studies.
\end{flushleft}





\begin{flushleft}
APL776 Product Design and Feasibility Study (Stream Core)
\end{flushleft}


\begin{flushleft}
4 Credits (2-0-4)
\end{flushleft}


\begin{flushleft}
Prefeasibility Study, Market Analysis-Development of Sales Plan. Technical
\end{flushleft}


\begin{flushleft}
Analysis- Development of Manufacturing Plan. Financial AnalysisDevelop General and Administrative Plan, Evaluate Project Feasibility,
\end{flushleft}


\begin{flushleft}
Preparation of project Proposal. Human Factors in Design, Human factors
\end{flushleft}


\begin{flushleft}
and systems, Information input, Human output and control, Workplace
\end{flushleft}


\begin{flushleft}
Design, environmental conditions, human factors applications.
\end{flushleft}





\begin{flushleft}
AML792 Structural Design of Ships
\end{flushleft}


\begin{flushleft}
3 Credits (3-0-0)
\end{flushleft}


\begin{flushleft}
Introduction, Ship as beam, long term loading of ship structure, periodic
\end{flushleft}


\begin{flushleft}
wave loading, longitudinal response \& dynamic behaviour, Criteria of
\end{flushleft}


\begin{flushleft}
failure, Analysis of plates and grillages, Buckling of plates and panels,
\end{flushleft}


\begin{flushleft}
Recent advances in load definition, transverse strength, torsional
\end{flushleft}


\begin{flushleft}
strength, bulkhead design, design of special structures, structural
\end{flushleft}


\begin{flushleft}
design of unconventional crafts like hydrofoils, hovercrafts, SES,
\end{flushleft}


\begin{flushleft}
SWATH, Catamarans, trimarans etc., design of submarine structures,
\end{flushleft}


\begin{flushleft}
pressure hull, design of cylindrical shells, cones, bulkheads etc.,
\end{flushleft}


\begin{flushleft}
Applications of computers to ship structures and structural optimization.
\end{flushleft}





\begin{flushleft}
AML793 Ship Dynamics
\end{flushleft}


\begin{flushleft}
3 Credits (3-0-0)
\end{flushleft}


\begin{flushleft}
Dynamics of oceans. Wave characteristics. Probabilistic theory of
\end{flushleft}


\begin{flushleft}
waves. Ship motions. Sea loads and bending moments. Limiting criteria
\end{flushleft}


\begin{flushleft}
stability and control of ships. Stabilization systems. Tests and trials.
\end{flushleft}





\begin{flushleft}
AML794 Warship Design
\end{flushleft}


\begin{flushleft}
3 Credits (3-0-0)
\end{flushleft}


\begin{flushleft}
Salient features of warships, merchantships, naval auxiliaries and
\end{flushleft}


\begin{flushleft}
yard-craft Principles and morphology of engineering design. Design
\end{flushleft}


\begin{flushleft}
spiralFeasibility studies. Preliminary design. Detailed design Warship
\end{flushleft}


\begin{flushleft}
design and production procedures. Staff requirements. Design activities.
\end{flushleft}


\begin{flushleft}
Drawing and specifications. Ship production Tests and trials. General
\end{flushleft}


\begin{flushleft}
arrangement drawings---Weapon layout. Mass and space analysis.
\end{flushleft}


\begin{flushleft}
Stability aspects, Resistance, propulsion. Seakeeping and manoeuvering
\end{flushleft}


\begin{flushleft}
considerations in design. Structural considerations. Survivability Cost
\end{flushleft}


\begin{flushleft}
aspects. Special types of hull forms. Computer aided ship design.
\end{flushleft}





\begin{flushleft}
AML795 Submarine Design
\end{flushleft}


\begin{flushleft}
3 Credits (3-0-0)
\end{flushleft}


\begin{flushleft}
Flotation and trim. Hydrostatics. Survivability. Surface unsinkability.
\end{flushleft}


\begin{flushleft}
Stability. Design of pressure proof structures. Design of school mounts
\end{flushleft}


\begin{flushleft}
of equipments. Resistance. Methods of drag reduction. Selection
\end{flushleft}


\begin{flushleft}
propulsion system. Endurance and indiscretion rates. Sea motions.
\end{flushleft}


\begin{flushleft}
Manoeuverability in vertical and horizontal planes and control surface
\end{flushleft}


\begin{flushleft}
design. Habitability. Ergonomics. Stealth systems. Submarine design
\end{flushleft}


\begin{flushleft}
procedures. System approach of submarine design and military
\end{flushleft}


\begin{flushleft}
economic analysis. Use of computers in submarine design. Outer hull
\end{flushleft}


\begin{flushleft}
lines development. Simulation of submarine in vertical plane.
\end{flushleft}





\begin{flushleft}
APL796 Advanced Solid Mechanics
\end{flushleft}


\begin{flushleft}
3 Credits (3-0-0)
\end{flushleft}


\begin{flushleft}
Large deformation kinematics, lagrangian stress and strain tensors,
\end{flushleft}





\begin{flushleft}
balance laws in lagrangian framework, nonlinear constitutive modeling,
\end{flushleft}


\begin{flushleft}
nonlinear theory of beams and buckling, wave propagation, theory of
\end{flushleft}


\begin{flushleft}
plasticity, solution of elasticity problems -- contact modeling, multiscale
\end{flushleft}


\begin{flushleft}
modeling etc.
\end{flushleft}





\begin{flushleft}
APL805 Advanced Finite Element Method
\end{flushleft}


\begin{flushleft}
3 Credits (3-0-0)
\end{flushleft}


\begin{flushleft}
Pre-requisites: APL705
\end{flushleft}


\begin{flushleft}
Variational calculus; Weak formulation of governing equations and its
\end{flushleft}


\begin{flushleft}
linearization; discretization of nonlinear weak form and its solution;
\end{flushleft}


\begin{flushleft}
convergence requirement of shape functions; systematic generation of
\end{flushleft}


\begin{flushleft}
higher order elements; mixed FEM/penalty method; non-uniform and
\end{flushleft}


\begin{flushleft}
adaptive discretization -- p and H convergence; solid-fluid interaction
\end{flushleft}


\begin{flushleft}
problems; Generalized and extended finite element methods.
\end{flushleft}





\begin{flushleft}
APD811 Major Project Part-I
\end{flushleft}


\begin{flushleft}
6 Credits (0-0-12)
\end{flushleft}


\begin{flushleft}
APD812 Major Project Part-II
\end{flushleft}


\begin{flushleft}
12 Credits (0-0-24)
\end{flushleft}


\begin{flushleft}
APL831 Theory of Plates and Shells
\end{flushleft}


\begin{flushleft}
3 Credits (3-0-0)
\end{flushleft}


\begin{flushleft}
Basic assumptions of two-dimensional theories, Theory of surfaces,
\end{flushleft}


\begin{flushleft}
Strain-displacement relations in shell coordinates, Stress-resultants,
\end{flushleft}


\begin{flushleft}
General governing equations of motion, Boundary conditions.
\end{flushleft}


\begin{flushleft}
Analytical solutions for bending and vibration of rectangular plates and
\end{flushleft}


\begin{flushleft}
circular plates. Approximate solution techniques. Membrane theory and
\end{flushleft}


\begin{flushleft}
its applicability, Membrane and general bending solutions cylindrical,
\end{flushleft}


\begin{flushleft}
conical and spherical shells, and pressure vessels. Selected problems
\end{flushleft}


\begin{flushleft}
on the stability. Design considerations.
\end{flushleft}





\begin{flushleft}
AML832 Applications of Theory of Plates and Shells
\end{flushleft}


\begin{flushleft}
2 Credits (2-0-0)
\end{flushleft}


\begin{flushleft}
Introduction. Recapitulation of classical plate theory. Orthotropic plate
\end{flushleft}


\begin{flushleft}
bending. Simplified 4th order theory. Panels and grillages. Navier's
\end{flushleft}


\begin{flushleft}
and Levy's solutions. Stability. Bending of circular cylindrical shells.
\end{flushleft}


\begin{flushleft}
Stability of semi-infinite and finite cylinders. Donnel equations. Shells
\end{flushleft}


\begin{flushleft}
of revolution. Applications.
\end{flushleft}





\begin{flushleft}
APL835 Mechanics of Composite Materials
\end{flushleft}


\begin{flushleft}
3 Credits (3-0-0)
\end{flushleft}


\begin{flushleft}
Composites, Various reinforcement and matrix materials, Strength
\end{flushleft}


\begin{flushleft}
and stiffness properties, Effective moduli, Spherical inclusions, Biocomposites, cylindrical and lamellar systems, Laminates: Laminated
\end{flushleft}


\begin{flushleft}
plates, Analysis and Design with composites, Fiber reinforced
\end{flushleft}


\begin{flushleft}
pressure vessels, dynamic, inelastic and non-linear effects, Fabrication
\end{flushleft}


\begin{flushleft}
of composites, Machining of composites, Strength evaluation,
\end{flushleft}


\begin{flushleft}
Technological applications.
\end{flushleft}





\begin{flushleft}
APL871 Product Reliability
\end{flushleft}


\begin{flushleft}
3 Credits (3-0-0)
\end{flushleft}


\begin{flushleft}
Reliability; basic concepts, Uncertainty in engineering systems;
\end{flushleft}


\begin{flushleft}
Modeling, Multiple random variables, product failure theories, Failure
\end{flushleft}


\begin{flushleft}
models, Limit state function, Probability distribution, PDF \& CDF,
\end{flushleft}


\begin{flushleft}
Evaluation of joint probability distribution, Markov Process, Stochastic
\end{flushleft}


\begin{flushleft}
Finite Element Analysis, Randomness in response variables, First
\end{flushleft}


\begin{flushleft}
and higher order methods for reliability assessment, Deterministic \&
\end{flushleft}


\begin{flushleft}
probabilistic approach, Risk based design, Maintainability, Central limit
\end{flushleft}


\begin{flushleft}
theorem, load and resistance approach, Fault tree approach, system
\end{flushleft}


\begin{flushleft}
reliability, stress strength interference method, Monte-Carlo and other
\end{flushleft}


\begin{flushleft}
simulation techniques, Regression analysis, Software based reliability
\end{flushleft}


\begin{flushleft}
analysis, Sensitivity analysis and reliability based design optimization,
\end{flushleft}


\begin{flushleft}
international standards, Applications \& case studies.
\end{flushleft}





\begin{flushleft}
APD895 MS Research Project
\end{flushleft}


\begin{flushleft}
36 Credits (0-0-72)
\end{flushleft}


\begin{flushleft}
AMD897 Minor Project
\end{flushleft}


\begin{flushleft}
3 Credits (0-0-6)
\end{flushleft}


\begin{flushleft}
AMD899 Design Project
\end{flushleft}


\begin{flushleft}
10 Credits (0-0-20)
\end{flushleft}





148





\begin{flushleft}
\newpage
Department of Biochemical Engineering and Biotechnology
\end{flushleft}


\begin{flushleft}
BBL110 Molecular Biotechnology
\end{flushleft}


\begin{flushleft}
3 Credits (3-0-0)
\end{flushleft}


\begin{flushleft}
Overlaps with: BBL131, BBL132
\end{flushleft}


\begin{flushleft}
The topics include introduction to cell, membrane structure and
\end{flushleft}


\begin{flushleft}
transport, enzyme technology, gene technology, Protein engineering
\end{flushleft}


\begin{flushleft}
and design, glycolysis and gluconeogenesis, citric acid cycle, ATP
\end{flushleft}


\begin{flushleft}
production, cell cycle, cell signalling, recombinant DNA technology
\end{flushleft}


\begin{flushleft}
including PCR, electrophoresis, cloning, and application of biological
\end{flushleft}


\begin{flushleft}
principles in Environmental Biotechnology.
\end{flushleft}





\begin{flushleft}
BBL131 Principles of Biochemistry
\end{flushleft}


\begin{flushleft}
4.5 Credits (3-0-3)
\end{flushleft}


\begin{flushleft}
Introduction-aims and Scope; Non-covalent interactions in biological
\end{flushleft}


\begin{flushleft}
systems, Carbohydrates-structure and function; Proteins-structure
\end{flushleft}


\begin{flushleft}
and function; Nucleic acids-structure and function Protein purification
\end{flushleft}


\begin{flushleft}
techniques; Introduction to enzymes; Vitamins and coenzymes;
\end{flushleft}


\begin{flushleft}
Lipids and biological membranes; Transport across cell membrane;
\end{flushleft}


\begin{flushleft}
Design of metabolism; Metabolic pathways for breakdown of
\end{flushleft}


\begin{flushleft}
carbohydrates-glycolysis, pentose phosphate pathway, citric acid
\end{flushleft}


\begin{flushleft}
cycle, electron transport chain, Photo-phosphorylation; Oxidation of
\end{flushleft}


\begin{flushleft}
fatty acids; Gluconeogenesis and control of glycogen metabolism,
\end{flushleft}


\begin{flushleft}
Signal transduction.
\end{flushleft}


\begin{flushleft}
Laboratory: Estimation of proteins and nucleic acids; Extraction
\end{flushleft}


\begin{flushleft}
of lipids; Separation of lipids using thin layer chromatography, Gel
\end{flushleft}


\begin{flushleft}
filtration and ion exchange chromatography; Gel electrophoresis,
\end{flushleft}


\begin{flushleft}
Determination of enzymatic activities and determination of Km, Vmax.
\end{flushleft}


\begin{flushleft}
Identification of intermediates of EMP pathway.
\end{flushleft}





\begin{flushleft}
BBL132 General Microbiology
\end{flushleft}


\begin{flushleft}
4.5 Credits (3-0-3)
\end{flushleft}


\begin{flushleft}
The topics include introduction to prokaryotic and eukaryotic cell
\end{flushleft}


\begin{flushleft}
structure; different groups of microorganisms; microbial nutrition and
\end{flushleft}


\begin{flushleft}
growth; metabolism including important pathways; reproduction and
\end{flushleft}


\begin{flushleft}
recombination; preservation and control of microbial cultures; viruses;
\end{flushleft}


\begin{flushleft}
microbial pathogenicity.
\end{flushleft}


\begin{flushleft}
Laboratory: Preparation and sterilisation of culture media, isolation
\end{flushleft}


\begin{flushleft}
of bacteria, Staining, Biochemical tests for identification of
\end{flushleft}


\begin{flushleft}
microorganisms, Antibiotic sensitivity, Bacterial growth curve, effect
\end{flushleft}


\begin{flushleft}
of environmental factors on bacterial growth, microbial diversity in
\end{flushleft}


\begin{flushleft}
environmental samples.
\end{flushleft}





\begin{flushleft}
BBL133 Mass and Energy Balances in Biochemical
\end{flushleft}


\begin{flushleft}
Engineering
\end{flushleft}


\begin{flushleft}
3 Credits (3-0-0)
\end{flushleft}


\begin{flushleft}
Overlaps with: CLL231, CHL251
\end{flushleft}


\begin{flushleft}
Stoichiometric relations and yield concepts, Maintenance coefficient,
\end{flushleft}


\begin{flushleft}
Mass balance based on available electron concept; Units and
\end{flushleft}


\begin{flushleft}
dimensions, Fundamentals of material balance, Balance on unit
\end{flushleft}


\begin{flushleft}
processes and reactive systems, Behaviour of ideal and real gases,
\end{flushleft}


\begin{flushleft}
vapour pressure, humidity and saturation. Energy balance, Heat
\end{flushleft}


\begin{flushleft}
capacity of gases, liquid and solids, Latent heat, Heat of reaction,
\end{flushleft}


\begin{flushleft}
formation and combustion, solution and dilution. Energy balance of
\end{flushleft}


\begin{flushleft}
reactive and non-reactive processes. Unsteady state material and
\end{flushleft}


\begin{flushleft}
energy balance in bioprocess. Case studies.
\end{flushleft}





\begin{flushleft}
BBL231 Molecular Biology and Genetics
\end{flushleft}


\begin{flushleft}
4.5 Credits (3-0-3)
\end{flushleft}


\begin{flushleft}
Pre-requisites: BBL131, BBL132
\end{flushleft}


\begin{flushleft}
Historical development and essentials of Mendelian genetics.
\end{flushleft}


\begin{flushleft}
Chromosomal theory of inheritance. Evolution and development of
\end{flushleft}


\begin{flushleft}
molecular biology. DNA model and classes. Organization of eukaryotic
\end{flushleft}


\begin{flushleft}
chromosome -- the chromatin structure. Gene structure and Genome.
\end{flushleft}


\begin{flushleft}
Transposon. Genetic Information and its perpetuation -- DNA
\end{flushleft}


\begin{flushleft}
replication, damage and repair. Telomere and Aging. Transcription,
\end{flushleft}


\begin{flushleft}
translation. Molecular biology of bacteriophage lamda. Gene exchange
\end{flushleft}


\begin{flushleft}
in bacteria. Gene regulation in prokaryotes. The operon model -- lac,
\end{flushleft}


\begin{flushleft}
ara, trp operons and gene regulation. Gene Regulation in Eukaryotes.
\end{flushleft}


\begin{flushleft}
DNA Methylation and Genomic Imprinting.
\end{flushleft}





\begin{flushleft}
Laboratory: Isolation of genomic and plasmid DNA, Agarose Gel
\end{flushleft}


\begin{flushleft}
Electrophoresis of DNA, Restriction digestion of DNA, RNA isolation,
\end{flushleft}


\begin{flushleft}
Primer design, PCR, RT-PCR, Competent cell preparation and
\end{flushleft}


\begin{flushleft}
Transformation, Gene Induction.
\end{flushleft}





\begin{flushleft}
BBL331 Bioprocess Engineering
\end{flushleft}


\begin{flushleft}
3 Credits (3-0-0)
\end{flushleft}


\begin{flushleft}
Pre-requisites: BBL132, BBL133
\end{flushleft}


\begin{flushleft}
Microbial growth, substrate utilisation, and product formation kinetics;
\end{flushleft}


\begin{flushleft}
simple structured models; batch, fed-batch, repeated fed-batch, CSTR,
\end{flushleft}


\begin{flushleft}
CSTR with recycle, multistage CSTRs, and PFR; aeration and agitation;
\end{flushleft}


\begin{flushleft}
rheology of fermentation fluids; mixing and scale-up; air sterilization;
\end{flushleft}


\begin{flushleft}
media sterilization; design of fermentation media; aseptic transfer.
\end{flushleft}





\begin{flushleft}
BBP332 Bioprocess Engineering Laboratory
\end{flushleft}


\begin{flushleft}
1.5 Credits (0-0-3)
\end{flushleft}


\begin{flushleft}
Pre-requisites: BBL131, BBL132
\end{flushleft}


\begin{flushleft}
Laboratory: Design and execution of simple laboratory scale
\end{flushleft}


\begin{flushleft}
experiments on the following topics: Estimation of cell mass; different
\end{flushleft}


\begin{flushleft}
phases of microbial growth; Mass and energy balance in a typical
\end{flushleft}


\begin{flushleft}
bioconversion process; Concept of limiting nutrient and its effect on
\end{flushleft}


\begin{flushleft}
cell growth; growth inhibition kinetics; product formation kinetics in a
\end{flushleft}


\begin{flushleft}
fermentation process; aerobic and anaerobic bioconversion process;
\end{flushleft}


\begin{flushleft}
power consumption in a fermentation process and its correlation with
\end{flushleft}


\begin{flushleft}
rheology of the fermentation fluid; different agitator types; mixing
\end{flushleft}


\begin{flushleft}
time in a bioreactor; quantification of KLa in a fermentation process;
\end{flushleft}


\begin{flushleft}
Heat balance across a batch sterilization process; Assembly and
\end{flushleft}


\begin{flushleft}
characterization of pH/DO electrodes.
\end{flushleft}





\begin{flushleft}
BBL341 Environmental Biotechnology
\end{flushleft}


\begin{flushleft}
3 Credits (3-0-0)
\end{flushleft}


\begin{flushleft}
Pre-requisites: CVL100 and EC 80
\end{flushleft}


\begin{flushleft}
Principles and concepts of ecosystem; Energy transfer in an
\end{flushleft}


\begin{flushleft}
ecosystem; Nutrient cycling; Basics of Environmental Microbiology,
\end{flushleft}


\begin{flushleft}
Environmental health: Ecotoxicology -- Heavy metals, pesticides and
\end{flushleft}


\begin{flushleft}
their effects, Indices of toxicity, Measurement of pollution (techniques
\end{flushleft}


\begin{flushleft}
and instrumentation), Dose--response relationship. Microbial
\end{flushleft}


\begin{flushleft}
biosensors in environmental monitoring, Environmental technologies:
\end{flushleft}


\begin{flushleft}
Microorganisms and renewable sources of energy, Biodegradation
\end{flushleft}


\begin{flushleft}
and bioremediation (phyto and microbial), Energy and nutrient
\end{flushleft}


\begin{flushleft}
recovery during waste treatment, Molecular tools in Environmental
\end{flushleft}


\begin{flushleft}
Biotechnology, Role of biotechnology in environmental protection.
\end{flushleft}





\begin{flushleft}
BBL342 Physical and Chemical Properties of
\end{flushleft}


\begin{flushleft}
Biomolecules
\end{flushleft}


\begin{flushleft}
3 Credits (2-1-0)
\end{flushleft}


\begin{flushleft}
Pre-requisites: BBL131
\end{flushleft}


\begin{flushleft}
Characteristic features of Biological system, Structure-function
\end{flushleft}


\begin{flushleft}
relationships. Characterization of biomolecules by molecular shape,
\end{flushleft}


\begin{flushleft}
size and molecular weight. Properties of biomolecules in solution:
\end{flushleft}


\begin{flushleft}
Diffusion, ultra-centrifugation and electrophoresis. Optical properties
\end{flushleft}


\begin{flushleft}
of biomolecules; Spectroscopic methods: IR, NMR, Optical rotary
\end{flushleft}


\begin{flushleft}
and circular dichroism \& imaging methods: Bright, darkfiled and
\end{flushleft}


\begin{flushleft}
fluorescence imaging.
\end{flushleft}





\begin{flushleft}
BBL343 Carbohydrates and Lipids in Biotechnology
\end{flushleft}


\begin{flushleft}
3 Credits (2-1-0)
\end{flushleft}


\begin{flushleft}
Pre-requisites: BBL131 and EC 60
\end{flushleft}


\begin{flushleft}
Introduction, Molecular structure of polysaccharides, Enzymes degrading
\end{flushleft}


\begin{flushleft}
polysaccharides, Physical properties of polysaccharides, Production
\end{flushleft}


\begin{flushleft}
of microbial Polysaccharides, Food usage of exopolysaccharides,
\end{flushleft}


\begin{flushleft}
Industrial Usage of exopolysaccharides, Medical applications of
\end{flushleft}


\begin{flushleft}
exopolysaccharides Molecular structure of lipids, Physical properties of
\end{flushleft}


\begin{flushleft}
lipids, Oleaginous microorganisms and their principal lipids, Production
\end{flushleft}


\begin{flushleft}
of microbial lipids, Modification of lipids for commercial applications,
\end{flushleft}


\begin{flushleft}
Extracellular microbial lipids and biosurfactants, Micelles and reverse
\end{flushleft}


\begin{flushleft}
micelles in biology, Liposomes in drug delivery.
\end{flushleft}





149





\begin{flushleft}
\newpage
Biochemical Engineering and Biotechnology
\end{flushleft}





\begin{flushleft}
BBV350 Special Module in Biochemical Engineering
\end{flushleft}


\begin{flushleft}
and Biotechnology
\end{flushleft}


\begin{flushleft}
1 Credit (1-0-0)
\end{flushleft}





\begin{flushleft}
BBL442 Immunology
\end{flushleft}


\begin{flushleft}
4 Credits (3-0-2)
\end{flushleft}


\begin{flushleft}
Pre-requisites: BBL131, BBL132, BBL231
\end{flushleft}





\begin{flushleft}
BBD351 Mini Project (BB)
\end{flushleft}


\begin{flushleft}
3 Credits (0-0-6)
\end{flushleft}


\begin{flushleft}
Pre-requisites: EC 60
\end{flushleft}





\begin{flushleft}
Historical background, Innate and acquired immunity. Cells and organs
\end{flushleft}


\begin{flushleft}
of immune system. Molecules of immune system -- immunoglobulins,
\end{flushleft}


\begin{flushleft}
MHCs, Cytokines, T cell receptors. Generation of antibody and
\end{flushleft}


\begin{flushleft}
T cell receptor diversity. Complement system. Humoral and Cell
\end{flushleft}


\begin{flushleft}
mediated immunity. Immune regulation. Vaccines. Hybridoma.
\end{flushleft}


\begin{flushleft}
Immunodeficiencies and AIDS. Transplantation immunity and cancer.
\end{flushleft}





\begin{flushleft}
BBL431 Bioprocess Technology
\end{flushleft}


\begin{flushleft}
2 Credits (2-0-0)
\end{flushleft}


\begin{flushleft}
Pre-requisites: EC25
\end{flushleft}


\begin{flushleft}
Chemical vs biochemical processing; Substrates for bioconversion
\end{flushleft}


\begin{flushleft}
processes; Process technology for production of primary and
\end{flushleft}


\begin{flushleft}
secondary metabolites such as ethanol, lactic acid, citric acid, amino
\end{flushleft}


\begin{flushleft}
acids, biopolymers, industrial enzymes, penicillin, recombinant
\end{flushleft}


\begin{flushleft}
glutathione and insulin.
\end{flushleft}





\begin{flushleft}
BBL432 Fluid Solid Systems
\end{flushleft}


\begin{flushleft}
2 Credits (2-0-0)
\end{flushleft}


\begin{flushleft}
Pre-requisites: CLL231
\end{flushleft}


\begin{flushleft}
Size reduction; crushing and grinding; equipment for size reduction;
\end{flushleft}


\begin{flushleft}
screening; design procedure; Flow of fluids past a stationary particle
\end{flushleft}


\begin{flushleft}
for low, medium and high Reynolds numbers; sedimentation and
\end{flushleft}


\begin{flushleft}
sedimentation theory; thickeners and classifiers; flow through packed
\end{flushleft}


\begin{flushleft}
beds; flow distribution, packing and pressure drop calculations;
\end{flushleft}


\begin{flushleft}
fluidization; filtration theory and its application in plate and frame and
\end{flushleft}


\begin{flushleft}
rotary vacuum filters; solid-liquid separation using centrifugation; {`}$\Delta$'
\end{flushleft}


\begin{flushleft}
concept in centrifugation for scale-up; different types of centrifuges
\end{flushleft}


\begin{flushleft}
and their design; application for biological suspensions.
\end{flushleft}





\begin{flushleft}
BBL433 Enzyme Science and Engineering
\end{flushleft}


\begin{flushleft}
4 Credits (3-0-2)
\end{flushleft}


\begin{flushleft}
Pre-requisites: BBL431
\end{flushleft}


\begin{flushleft}
Introduction and scope; Chemical and functional nature of enzymes;
\end{flushleft}


\begin{flushleft}
Application of enzymes in process industries and health care;
\end{flushleft}


\begin{flushleft}
microbial production and purification of industrial enzymes; kinetics of
\end{flushleft}


\begin{flushleft}
enzyme catalysed reactions; immobilization of enzymes; stabilization
\end{flushleft}


\begin{flushleft}
of enzymes. Bioreactors for soluble and immobilized enzymes,
\end{flushleft}


\begin{flushleft}
mass transfer and catalysis in immobilized reactors. Enzyme based
\end{flushleft}


\begin{flushleft}
biosensors; enzyme catalysed process with cofactor regeneration;
\end{flushleft}


\begin{flushleft}
Enzyme reactions in micro-aqueous medium and non-conventional
\end{flushleft}


\begin{flushleft}
medium. Laboratory: Assay of enzyme activity and specific activity;
\end{flushleft}


\begin{flushleft}
kinetic analysis of an enzyme catalysed reaction; Immobilization of
\end{flushleft}


\begin{flushleft}
enzymes by adsorption and covalent binding; salt precipitation of
\end{flushleft}


\begin{flushleft}
an enzyme; immobilization of microbial cells by entrapment; effect
\end{flushleft}


\begin{flushleft}
of water and solvent on the lipase catalysed esterification reaction.
\end{flushleft}





\begin{flushleft}
BBL434 Bioinformatics
\end{flushleft}


\begin{flushleft}
3 Credits (2-0-2)
\end{flushleft}


\begin{flushleft}
Pre-requisites: BBL131, BBL132
\end{flushleft}


\begin{flushleft}
The topics include introduction to bioinformatics - resources and
\end{flushleft}


\begin{flushleft}
applications, Biological sequence analysis, sequence alignment,
\end{flushleft}


\begin{flushleft}
molecular phylogenetic analysis, genome organization and analysis,
\end{flushleft}


\begin{flushleft}
protein analysis, molecular modeling and drug design.
\end{flushleft}





\begin{flushleft}
BBL441 Food Science and Engineering
\end{flushleft}


\begin{flushleft}
3 Credits (3-0-0)
\end{flushleft}


\begin{flushleft}
Chemical constituents of foods, their properties and functions;
\end{flushleft}


\begin{flushleft}
Characteristic features of natural and processed foods; Chemical/
\end{flushleft}


\begin{flushleft}
biochemical reactions in storage/handling of foods; Units operations
\end{flushleft}


\begin{flushleft}
in food processing- size reduction, evaporation, filtration etc.;
\end{flushleft}


\begin{flushleft}
Methods for food preservation; Rheology of food products; Flavour,
\end{flushleft}


\begin{flushleft}
aroma and other additives in processed foods; case studies of a few
\end{flushleft}


\begin{flushleft}
specific food processing sectors, cereals, protein foods, meat, fish
\end{flushleft}


\begin{flushleft}
and poultry, vegetable and fruit, milk products; legislation, safety
\end{flushleft}


\begin{flushleft}
and quality control.
\end{flushleft}





\begin{flushleft}
BBL443 Modelling and Simulation of Bioprocesses
\end{flushleft}


\begin{flushleft}
4 Credits (3-0-2)
\end{flushleft}


\begin{flushleft}
Pre-requisites: BBL331
\end{flushleft}


\begin{flushleft}
Types of kinetic models, Data smoothing and analysis, Mathematical
\end{flushleft}


\begin{flushleft}
representation of Bioprocesses, Parameter estimation, Numerical
\end{flushleft}


\begin{flushleft}
Integration techniques, Parameter Sensitivity analysis, Statistical
\end{flushleft}


\begin{flushleft}
validity, Discrimination between two models. Case studies Physiological
\end{flushleft}


\begin{flushleft}
state markers and its use in the formulation of a structured model,
\end{flushleft}


\begin{flushleft}
Development of compartment and metabolic pathway models
\end{flushleft}


\begin{flushleft}
(Software Probe) for intracellular state estimation. Dynamic Simulation
\end{flushleft}


\begin{flushleft}
of batch, fed-batch steady and transient culture metabolism, Numerical
\end{flushleft}


\begin{flushleft}
Optimization of Bioprocesses using Mathematical models.
\end{flushleft}





\begin{flushleft}
BBL444 Advanced Bioprocess Control
\end{flushleft}


\begin{flushleft}
3 Credits (3-0-0)
\end{flushleft}


\begin{flushleft}
Pre-requisites: CLL261
\end{flushleft}


\begin{flushleft}
The course begins with an overview of classical control theory leading
\end{flushleft}


\begin{flushleft}
to a detailed analysis of the stability of biological systems. Lyapunov
\end{flushleft}


\begin{flushleft}
stability is introduced followed by concepts of nonlinear control
\end{flushleft}


\begin{flushleft}
theory and applications to bioreactors. Control loops in metabolic
\end{flushleft}


\begin{flushleft}
and protein networks are discussed in the background evolution and
\end{flushleft}


\begin{flushleft}
motifs selected by natural systems. This leads to the introduction of
\end{flushleft}


\begin{flushleft}
large protein interaction networks and study of their architectures.
\end{flushleft}


\begin{flushleft}
Applications of these ideas in apriori analysis of synthetic circuits are
\end{flushleft}


\begin{flushleft}
examined. The course ends with case studies from the literature.
\end{flushleft}





\begin{flushleft}
BBL445 Membrane Applications in Bioprocessing
\end{flushleft}


\begin{flushleft}
3 Credits (3-0-0)
\end{flushleft}


\begin{flushleft}
Milk/cheese processing, Fruit/sugarcane juice processing,
\end{flushleft}


\begin{flushleft}
Pharmaceuticals/ Therapeutic drugs processing and membrane
\end{flushleft}


\begin{flushleft}
coupled separation of biomolecules; Membrane based bioreactor
\end{flushleft}


\begin{flushleft}
for cell/enzyme recycle; Mammalian/plant cell culture; Case studies.
\end{flushleft}





\begin{flushleft}
BBL446 Biophysics
\end{flushleft}


\begin{flushleft}
3 Credits (3-0-0)
\end{flushleft}


\begin{flushleft}
Pre-requisites: PYL100, BBL131
\end{flushleft}


\begin{flushleft}
Spectroscopic methods in biophysics, conformational changes in
\end{flushleft}


\begin{flushleft}
biological processes, biological energy conservation and transduction,
\end{flushleft}


\begin{flushleft}
photosynthesis, transport across biomembranes, the biophysics
\end{flushleft}


\begin{flushleft}
of motility, the biophysics of the nerve impulse, single molecule
\end{flushleft}


\begin{flushleft}
biophysical studies.
\end{flushleft}





\begin{flushleft}
BBL447 Enzyme Catalyzed Organic Synthesis
\end{flushleft}


\begin{flushleft}
3 Credits (2-0-2)
\end{flushleft}


\begin{flushleft}
Pre-requisites: BBL131 and EC 90
\end{flushleft}


\begin{flushleft}
Enzymes as biocatalysts. Various reaction media for enzyme catalyzed
\end{flushleft}


\begin{flushleft}
reaction [water, water poor media such as organic solvents, ionic
\end{flushleft}


\begin{flushleft}
liquids] and mixed solvents. Advantages of medium engineering.
\end{flushleft}


\begin{flushleft}
Fundamentals of non-aqueous enzymology [pH memory, molecular
\end{flushleft}


\begin{flushleft}
imprinting]. Improving biocatalysis in water and water poor media
\end{flushleft}


\begin{flushleft}
[chemical modification, immobilization, applications of protein
\end{flushleft}


\begin{flushleft}
engineering/directed evolution]. Enzyme promiscuity and its
\end{flushleft}


\begin{flushleft}
applications in organic synthesis. Biocatalytic applications in organic
\end{flushleft}


\begin{flushleft}
synthesis, hydrolytic reactions, oxidation reduction reactions, formation
\end{flushleft}


\begin{flushleft}
of C-C bond, addition and elimination reactions, glycosyl transfer
\end{flushleft}


\begin{flushleft}
reactions, isomerization, halogenation/dehalogenation reactions.
\end{flushleft}





150





\begin{flushleft}
\newpage
Biochemical Engineering and Biotechnology
\end{flushleft}





\begin{flushleft}
BBD451 Major Project Part 1 (BB1)
\end{flushleft}


\begin{flushleft}
3 Credits (0-0-6)
\end{flushleft}





\begin{flushleft}
and assembly; Next-generation sequencing; Studying gene expression
\end{flushleft}


\begin{flushleft}
and function; High throughput gene expression and analysis.
\end{flushleft}


\begin{flushleft}
PROTEOMICS - Sample preparation; Separation methods; Mass
\end{flushleft}


\begin{flushleft}
Spectroscopy and de novo sequencing; Comparative Proteomics;
\end{flushleft}


\begin{flushleft}
Protein-protein interactions.
\end{flushleft}





\begin{flushleft}
BBD452 Major Project Part 2 (BB1)
\end{flushleft}


\begin{flushleft}
8 Credits (0-0-16)
\end{flushleft}





\begin{flushleft}
BBL736 Dynamics of Microbial Systems
\end{flushleft}


\begin{flushleft}
3 Credits (3-0-0)
\end{flushleft}


\begin{flushleft}
Pre-requisites: BBL331, BBL432, BBL433
\end{flushleft}





\begin{flushleft}
BBL731 Bioseparation Engineering
\end{flushleft}


\begin{flushleft}
4.5 Credits (3-0-3)
\end{flushleft}


\begin{flushleft}
Pre-requisites: BBL331, BBL432, BBL433
\end{flushleft}


\begin{flushleft}
Characteristics of bio product, flocculation and conditioning of
\end{flushleft}


\begin{flushleft}
fermented medium, Revision of mechanical separation (filtration,
\end{flushleft}


\begin{flushleft}
Centrifugation etc.), cell disruption, Protein precipitation and
\end{flushleft}


\begin{flushleft}
its separation, Extraction, Adsorption Desorption processes,
\end{flushleft}


\begin{flushleft}
Chromatographic methods based on size, charge, shape, biological
\end{flushleft}


\begin{flushleft}
affinity etc., Membrane separations- ultrafiltration and electrodialysis,
\end{flushleft}


\begin{flushleft}
Electrophoresis, Crystallization, Drying, Case studies. Laboratory:
\end{flushleft}


\begin{flushleft}
Conventional filtration batch \& continuous, Centrifugation in batch
\end{flushleft}


\begin{flushleft}
and continuous centrifuge, Cell disruption, Protein precipitation and
\end{flushleft}


\begin{flushleft}
its recovery, Thin Layer Chromatography (TLC), Membrane based
\end{flushleft}


\begin{flushleft}
filtration- ultrafiltration in cross. Flow modules and microfiltration,
\end{flushleft}


\begin{flushleft}
electrodialysis, Adsorption Column Studies and Freeze Drying Studies.
\end{flushleft}





\begin{flushleft}
BBL732 Bioprocess Plant Design
\end{flushleft}


\begin{flushleft}
4 Credits (3-0-2)
\end{flushleft}


\begin{flushleft}
Pre-requisites: APL100 CLL251 CLL252 BBL331 BBL432
\end{flushleft}


\begin{flushleft}
Introduction; General design information; Mass and energy balance;
\end{flushleft}


\begin{flushleft}
Flow sheeting; Piping and instrumentation; Materials of construction
\end{flushleft}


\begin{flushleft}
for bioprocess plants; Mechanical design of process equipment; Vessels
\end{flushleft}


\begin{flushleft}
for biotechnology applications; Design considerations for maintaining
\end{flushleft}


\begin{flushleft}
sterility of process streams and processing equipment; Selection and
\end{flushleft}


\begin{flushleft}
specification of equipment for handling fluids and solids; Selection,
\end{flushleft}


\begin{flushleft}
specification and design of heat and mass transfer equipment used
\end{flushleft}


\begin{flushleft}
in bioprocess industries; Utilities for biotechnology production plants;
\end{flushleft}


\begin{flushleft}
Process economics; Bioprocess validation; Safety considerations;
\end{flushleft}


\begin{flushleft}
Case studies.
\end{flushleft}


\begin{flushleft}
Laboratory: Design of the complete process plant for an identified
\end{flushleft}


\begin{flushleft}
product or service. Each student to choose a separate product/industry
\end{flushleft}





\begin{flushleft}
BBL733 Recombinant DNA Technology
\end{flushleft}


\begin{flushleft}
3.5 Credits (2-0-3)
\end{flushleft}


\begin{flushleft}
Pre-requisites: BBL131, BBL132, BBL231 or Masters' degree
\end{flushleft}


\begin{flushleft}
in Bioscience
\end{flushleft}


\begin{flushleft}
Restriction and other modifying enzymes, Cloning vectors (plasmid,
\end{flushleft}


\begin{flushleft}
(-based, phagemids, high capacity) and expression vectors, Expression
\end{flushleft}


\begin{flushleft}
in bacterial, yeast and mammalian systems, Construction of genomic
\end{flushleft}


\begin{flushleft}
and cDNA libraries, DNA Sequencing, Polymerase chain reaction,
\end{flushleft}


\begin{flushleft}
Invitro mutagenesis, Genome mapping, Stability of recombinant cells
\end{flushleft}


\begin{flushleft}
in production of biochemicals.
\end{flushleft}





\begin{flushleft}
Stability analysis; analysis of multiple interacting microbial populations;
\end{flushleft}


\begin{flushleft}
stability of recombinant cells; Structured models of gene expression
\end{flushleft}


\begin{flushleft}
and growth, Cell cycle and age-dependent (segregated) models,
\end{flushleft}


\begin{flushleft}
Single-cell (stochastic) models of gene expression.
\end{flushleft}





\begin{flushleft}
BBL737 Instrumentation and Analytical Methods in
\end{flushleft}


\begin{flushleft}
Bioengineering
\end{flushleft}


\begin{flushleft}
3 Credits (2-0-2)
\end{flushleft}


\begin{flushleft}
Pre-requisites: BBL131
\end{flushleft}


\begin{flushleft}
Introduction to methods used in Analytical Bioengineering,
\end{flushleft}


\begin{flushleft}
Electrophoretic methods, Principles and applications of chromatography
\end{flushleft}


\begin{flushleft}
(GC, HPLC, FPLC, HPTLC), Spectrophotometry (UV-visible),
\end{flushleft}


\begin{flushleft}
Fluorescence methods, FTIR, Circular dichroism, Mass spectrometry
\end{flushleft}


\begin{flushleft}
(GC-MS, LC-MS, ICP-MS), Immunology based analytical methods
\end{flushleft}


\begin{flushleft}
(ELISA), qPCR, Advanced Microscopy techniques (Electron Microscopy,
\end{flushleft}


\begin{flushleft}
Confocal Microscopy).
\end{flushleft}





\begin{flushleft}
BBL740 Plant Cell Technology
\end{flushleft}


\begin{flushleft}
3 Credits (2-0-2)
\end{flushleft}


\begin{flushleft}
Pre-requisites: BBL331
\end{flushleft}


\begin{flushleft}
Special features and organization of Plant cells. Totipotency,
\end{flushleft}


\begin{flushleft}
regeneration of plants, Plant products of Industrial importance.
\end{flushleft}


\begin{flushleft}
Biochemistry of major metabolic pathways and products. Autotrophic
\end{flushleft}


\begin{flushleft}
and heterotrophic growth, Plant growth regulators and elicitors, Cell
\end{flushleft}


\begin{flushleft}
suspension culture development: methodology, kinetics of growth and
\end{flushleft}


\begin{flushleft}
production formation, nutrient optimization. Production of secondary
\end{flushleft}


\begin{flushleft}
metabolites by suspension cultures with a few case studies. Biological
\end{flushleft}


\begin{flushleft}
and technological barriers-hydrodynamic shear and its quantification,
\end{flushleft}


\begin{flushleft}
mixing and impeller design aspects. Plant cell reactors: comparison
\end{flushleft}


\begin{flushleft}
of reactor performances Immobilized plant cell and cell retention
\end{flushleft}


\begin{flushleft}
reactors. Hairy root induction and their mass propagation in different
\end{flushleft}


\begin{flushleft}
bioreactor configurations. Use of engineering optimization protocols
\end{flushleft}


\begin{flushleft}
for enhanced plant metabolite production in the bioreactor.
\end{flushleft}


\begin{flushleft}
Laboratory: Development of callus and suspension cultures of plant
\end{flushleft}


\begin{flushleft}
cells; shear sensitivity; growth and product formation kinetics in
\end{flushleft}


\begin{flushleft}
suspension cultures; development of hairy root cultures \& study of
\end{flushleft}


\begin{flushleft}
growth kinetics; production of secondary metabolites in bioreactors
\end{flushleft}


\begin{flushleft}
using suspension cultures/hairy roots/ immobilized cells
\end{flushleft}





\begin{flushleft}
BBL741 Protein Science and Engineering
\end{flushleft}


\begin{flushleft}
3 Credits (3-0-0)
\end{flushleft}


\begin{flushleft}
Pre-requisites: BBL131 BBL231 or Masters' degree in Bioscience
\end{flushleft}





\begin{flushleft}
BBL734 Metabolic Regulation and Engineering
\end{flushleft}


\begin{flushleft}
3 Credits (3-0-0)
\end{flushleft}


\begin{flushleft}
Pre-requisites: BBL331, BBL431
\end{flushleft}


\begin{flushleft}
Regulatory mechanisms for control of enzyme synthesis - an overview;
\end{flushleft}


\begin{flushleft}
Control of enzyme activity- proteolysis, covalent modification and
\end{flushleft}


\begin{flushleft}
ligand binding; Metabolic control theory and metabolic flux analysis;
\end{flushleft}


\begin{flushleft}
Metabolic regulation of a few major metabolic pathways especially
\end{flushleft}


\begin{flushleft}
those relevant to bioprocess industries; Pathway engineering;
\end{flushleft}


\begin{flushleft}
Application of gene cloning in re-directing cellular metabolism for
\end{flushleft}


\begin{flushleft}
over-production of a few industrial products; Strategies to overcome
\end{flushleft}


\begin{flushleft}
regulatory mechanisms for over-production of several industrially
\end{flushleft}


\begin{flushleft}
important primary and secondary metabolites such as alcohols, organic
\end{flushleft}


\begin{flushleft}
acids, amino acids, enzymes and therapeutic compounds.
\end{flushleft}





\begin{flushleft}
Introduction and aim; Basic structural principles of proteins-amino
\end{flushleft}


\begin{flushleft}
acids; Motifs of protein structure and their packing: alpha domain,
\end{flushleft}


\begin{flushleft}
alpha/Beta domain, Antiparallel B structures; Protein folding and
\end{flushleft}


\begin{flushleft}
assembly -- protein folding pathways for single and multiple domain
\end{flushleft}


\begin{flushleft}
proteins; Recovery of active proteins from inclusion bodies; Structure
\end{flushleft}


\begin{flushleft}
prediction-structural classes, secondary and tertiary protein structure
\end{flushleft}


\begin{flushleft}
prediction; Sequence homology searches; Strategies for protein
\end{flushleft}


\begin{flushleft}
engineering -- random, site-directed, case studies; Drug-protein
\end{flushleft}


\begin{flushleft}
interactions and design, Rational protein design.
\end{flushleft}





\begin{flushleft}
BBL735 Genomics and Proteomics
\end{flushleft}


\begin{flushleft}
3 Credits (2-0-2)
\end{flushleft}


\begin{flushleft}
Pre-requisites: BBL231, BBL733
\end{flushleft}





\begin{flushleft}
BBL742 Biological Waste Treatment
\end{flushleft}


\begin{flushleft}
4 Credits (3-0-2)
\end{flushleft}


\begin{flushleft}
Pre-requisites: BBL132, BBL331 or Bachelor's degree in
\end{flushleft}


\begin{flushleft}
Engineering or Masters' degree in Science
\end{flushleft}





\begin{flushleft}
Introduction to -omes and -omics; GENOMICS - Genome sequencing
\end{flushleft}





\begin{flushleft}
Qualitative and quantitative characterization of wastes; Waste
\end{flushleft}





151





\begin{flushleft}
\newpage
Biochemical Engineering and Biotechnology
\end{flushleft}





\begin{flushleft}
disposal norms and regulations; Indian regulations; Principles of
\end{flushleft}


\begin{flushleft}
biological treatment; Aerobic and anaerobic biological wastewater
\end{flushleft}


\begin{flushleft}
treatment systems; Suspended and attached cell biological wastewater
\end{flushleft}


\begin{flushleft}
treatment systems; Biological nutrient removal; Treatment plant design
\end{flushleft}


\begin{flushleft}
calculations; Treatment and disposal of sludge; biological means for
\end{flushleft}


\begin{flushleft}
stabilization and disposal of solid wastes; Treatment of hazardous and
\end{flushleft}


\begin{flushleft}
toxic wastes; Degradation of xenobiotic compounds; bioremediation.
\end{flushleft}


\begin{flushleft}
Laboratory: Characterization of wastes; Design calculations for various
\end{flushleft}


\begin{flushleft}
types of wastes using various types of biological processes.
\end{flushleft}





\begin{flushleft}
BBL743 High Resolution Methods in Biotechnology
\end{flushleft}


\begin{flushleft}
3 Credits (2-0-2)
\end{flushleft}


\begin{flushleft}
Pre-requisites: BBL131, BBL331 or Masters' degree in Bioscience
\end{flushleft}


\begin{flushleft}
Need for high resolution separation for biologicals; Difficulties with
\end{flushleft}


\begin{flushleft}
traditional methodologies; Affinity precipitation and partitioning; MF/
\end{flushleft}


\begin{flushleft}
UF/NF for high resolution separation; chromatography techniques;
\end{flushleft}


\begin{flushleft}
Affinity chromatography and electrophoresis, Separation by gene
\end{flushleft}


\begin{flushleft}
amplification (PCR), Molecular imprinting.
\end{flushleft}





\begin{flushleft}
Hypothesis driven experiments using focused microarrays, Biological
\end{flushleft}


\begin{flushleft}
interpretation, Commercial software available.
\end{flushleft}





\begin{flushleft}
BBL749 Cancer Cell Biology
\end{flushleft}


\begin{flushleft}
4.5 Credits (3-0-3)
\end{flushleft}


\begin{flushleft}
Pre-requisites: BBL131 BBL132 BBL231
\end{flushleft}


\begin{flushleft}
This course provides students with a deeper understanding of cancer
\end{flushleft}


\begin{flushleft}
biology and is heavily focused on experiments: Topics include: Cancer
\end{flushleft}


\begin{flushleft}
Biology Overview, Types of Cancer, Causes for cancer, Oncogenes and
\end{flushleft}


\begin{flushleft}
Tumor suppressors, Cell Cycle and Regulation, Cell Differentiation, Cell
\end{flushleft}


\begin{flushleft}
Death Pathways (Apoptosis, Autophagy), Necrosis, Cell Senescence,
\end{flushleft}


\begin{flushleft}
Cell Adhesion and Motility, Cancer Epigenetics and sRNAs, Cancer
\end{flushleft}


\begin{flushleft}
Genome instability, Tumor Immunity, Growth Signaling pathways,
\end{flushleft}


\begin{flushleft}
Tumor angiogenesis, Cancer Stem Cell, Diagnosis, prognosis and
\end{flushleft}


\begin{flushleft}
treatment of cancer.
\end{flushleft}


\begin{flushleft}
Laboratory: Experiments on Cell cycle, Differentiation, Necrosis and
\end{flushleft}


\begin{flushleft}
Apoptosis, Senescence, Anchorage Independence, Cell Migration and
\end{flushleft}


\begin{flushleft}
Invasion, MicroRNAs, Stem cell, Fluorescence Microscopy.
\end{flushleft}





\begin{flushleft}
BBL750 Genome Engineering
\end{flushleft}


\begin{flushleft}
3 Credits (2-0-2)
\end{flushleft}





\begin{flushleft}
BBL744 Animal Cell Technology
\end{flushleft}


\begin{flushleft}
4 Credits (3-0-2)
\end{flushleft}


\begin{flushleft}
Characteristic of animal cell, metabolism, regulation and nutritional
\end{flushleft}


\begin{flushleft}
requirements; Kinetics of cell growth and product formation and effect
\end{flushleft}


\begin{flushleft}
of shear force; Product and substrate transport; Perfusion bioreactors,
\end{flushleft}


\begin{flushleft}
Hollow fiber bioreactor, Operational strategies and integrated
\end{flushleft}


\begin{flushleft}
approach; Micro and macro carrier culture; Hybridoma technology;
\end{flushleft}


\begin{flushleft}
Genetic engineering in animal cell culture; Scale-up and large scale
\end{flushleft}


\begin{flushleft}
operation; Case studies.
\end{flushleft}


\begin{flushleft}
Laboratory: Cell culture in static phase (T-flask), quantification of cell
\end{flushleft}


\begin{flushleft}
growth, monolayer culture, determination of critical shear stress, micro
\end{flushleft}


\begin{flushleft}
carrier, Cell viability assay. Case studies to understand growth kinetics
\end{flushleft}


\begin{flushleft}
and product kinetics in actul cell culture system.
\end{flushleft}





\begin{flushleft}
BBL745 Combinatorial Biotechnology
\end{flushleft}


\begin{flushleft}
3 Credits (3-0-0)
\end{flushleft}


\begin{flushleft}
Solid phase synthesis, solution phase synthesis, encoding technologies,
\end{flushleft}


\begin{flushleft}
deconvolution methods, Tools for Combinatorial Biotechnology, Display
\end{flushleft}


\begin{flushleft}
libraries, Applications.
\end{flushleft}





\begin{flushleft}
BBL746 Current Topics in Biochemical Engineering and
\end{flushleft}


\begin{flushleft}
Biotechnology
\end{flushleft}


\begin{flushleft}
3 Credits (3-0-0)
\end{flushleft}


\begin{flushleft}
Pre-requisites: BBL131, BBL331
\end{flushleft}


\begin{flushleft}
BBL747 Bionanotechnology
\end{flushleft}


\begin{flushleft}
3 Credits (3-0-0)
\end{flushleft}


\begin{flushleft}
Pre-requisites: BBL131 or Masters' degree in Bioscience
\end{flushleft}


\begin{flushleft}
Introduction, Self-assembly of biomolecules in nanotechnology;
\end{flushleft}


\begin{flushleft}
Bacterial S-Layer, Biomimetic Ferritin, Biodegradable nanoparticles
\end{flushleft}


\begin{flushleft}
for drug delivery to cells and tissues, Polymer Nanocontainers, Ion
\end{flushleft}


\begin{flushleft}
channels as molecular switches, Patch clamp technique, Protein based
\end{flushleft}


\begin{flushleft}
nanoelectronics, Bacteriorhodopsin and its technical applications,
\end{flushleft}


\begin{flushleft}
Carbon Nanotubes: Towards next generation biosensors, Molecular
\end{flushleft}


\begin{flushleft}
Lego: Design for molecular actuators, Biological Membranes,
\end{flushleft}


\begin{flushleft}
Magnetosomes: Trapping nano-magnetite in biological membranes,
\end{flushleft}


\begin{flushleft}
Biomolecular Motors, Techniques used in Bionanotechnology
\end{flushleft}


\begin{flushleft}
Nanoanalytics: Fluorescent Quantum Dots for Biological Labelling,
\end{flushleft}


\begin{flushleft}
Nanoparticle Molecular Labels.
\end{flushleft}





\begin{flushleft}
BBL748 Data Analysis for DNA Microarrays
\end{flushleft}


\begin{flushleft}
4 Credits (3-0-2)
\end{flushleft}


\begin{flushleft}
Pre-requisites: BBL131, BBL231, BBL733
\end{flushleft}


\begin{flushleft}
Microarray technology, Basic digital imaging and image processing,
\end{flushleft}


\begin{flushleft}
Probabilities, common distributions, Bayes' theorem, Analyzing
\end{flushleft}


\begin{flushleft}
microarray data with classical hypothesis testing, Analysis of variance,
\end{flushleft}


\begin{flushleft}
Experimental Design, Analysis and visualization tools: Box plots,
\end{flushleft}


\begin{flushleft}
Scatter plots, Histograms, Cluster Analysis: one-way, two-way,
\end{flushleft}


\begin{flushleft}
Graphic, Methods for selection of differentially regulated genes,
\end{flushleft}





\begin{flushleft}
Genome engineering methods for bacteria, yeast, plants and
\end{flushleft}


\begin{flushleft}
mammalian cells, Newer gene delivery methods, Next generation
\end{flushleft}


\begin{flushleft}
cloning technologies.
\end{flushleft}





\begin{flushleft}
BBV750 Bioreaction Engineering
\end{flushleft}


\begin{flushleft}
1 Credit (1-0-0)
\end{flushleft}


\begin{flushleft}
Definitions of volumetric and specific rates. Yield and productivity
\end{flushleft}


\begin{flushleft}
of bioprocesses. Coupling of steady state black box stoichiometrics.
\end{flushleft}


\begin{flushleft}
Metabolic Flux Analysis. Transient operation, and the sensitivity of
\end{flushleft}


\begin{flushleft}
coupled enzymatic reactions to changes in the pathway flux. Design
\end{flushleft}


\begin{flushleft}
of bio-reactors with volumetric isotropy. The role of mixing in industrial
\end{flushleft}


\begin{flushleft}
bio-reactions. Large scale processes.
\end{flushleft}





\begin{flushleft}
BBL810 Enzyme and Microbial Technology
\end{flushleft}


\begin{flushleft}
3 Credits (3-0-0)
\end{flushleft}


\begin{flushleft}
Isolation, development and preservation of industrial microorganisms;
\end{flushleft}


\begin{flushleft}
Substrates for industrial microbial processes; Regulatory mechanisms
\end{flushleft}


\begin{flushleft}
of metabolic pathways in industrial strains; Analysis of various
\end{flushleft}


\begin{flushleft}
microbial processes used in production of biomass, primary and
\end{flushleft}


\begin{flushleft}
secondary metabolites; Microbial leaching of minerals; Microorganisms
\end{flushleft}


\begin{flushleft}
in degradation of xenobiotics and removal of heavy metals;
\end{flushleft}


\begin{flushleft}
Biotransformations. Enzymes as industrial biocatalysts; production;
\end{flushleft}


\begin{flushleft}
isolation; purification and application of industrial enzymes;
\end{flushleft}


\begin{flushleft}
immobilized enzymes; stabilization of enzymes; enzyme catalyzed
\end{flushleft}


\begin{flushleft}
organic synthesis; multienzyme systems.
\end{flushleft}





\begin{flushleft}
BBL820 Downstream Processing
\end{flushleft}


\begin{flushleft}
3 Credits (3-0-0)
\end{flushleft}


\begin{flushleft}
Characteristics of biological materials; Pre-treatment;, Microbial
\end{flushleft}


\begin{flushleft}
separation: Centrifugation and filtration, Cell disruption methods,
\end{flushleft}


\begin{flushleft}
Protein precipitation, Extraction, Adsorption, Electrophoresis,
\end{flushleft}


\begin{flushleft}
Chromatography, Ultrafiltration, Reverse osmosis, Isoelectric focusing,
\end{flushleft}


\begin{flushleft}
Affinity based separations, Case Studies.
\end{flushleft}





\begin{flushleft}
BBL830 Microbial Biochemisry
\end{flushleft}


\begin{flushleft}
3 Credits (3-0-0)
\end{flushleft}


\begin{flushleft}
Structure and function of biomolecules amino acids, proteins, lipids,
\end{flushleft}


\begin{flushleft}
nucleotides and nucleic acids: Enzymes-structure and kinetics,
\end{flushleft}


\begin{flushleft}
Vitamins and coenzymes, Metabolic pathways: Carbohydrate
\end{flushleft}


\begin{flushleft}
metabolism: glycolysis, pentose phosphate pathway, citric acid cycle;
\end{flushleft}


\begin{flushleft}
Bioenergetics oxidative phosphorylation and photo-synthesis: Fatty
\end{flushleft}


\begin{flushleft}
acid metabolism; Amino acid metabolism; Regulatory mechanismsfeedback inhibition, induction, catabolite repression; Nucleic acid and
\end{flushleft}


\begin{flushleft}
protein biosynthesis.
\end{flushleft}





\begin{flushleft}
BBP840 Laboratory Techniques in Microbial Biochemistry
\end{flushleft}


\begin{flushleft}
2 Credits (0-0-4)
\end{flushleft}


\begin{flushleft}
Estimation of carbohydrates/proteins/nucleic acids; separation of
\end{flushleft}


\begin{flushleft}
phospho-lipids by thin layer chromatography; chromatographic
\end{flushleft}





152





\begin{flushleft}
\newpage
Biochemical Engineering and Biotechnology
\end{flushleft}





\begin{flushleft}
separation of proteins; identification and estimation of intermediates
\end{flushleft}


\begin{flushleft}
of glycolytic pathway; oxidative phosphorylation; cell fractionation;
\end{flushleft}


\begin{flushleft}
aseptic techniques; microscopic examination of bacteria \& fungi;
\end{flushleft}


\begin{flushleft}
selected biochemical tests; plasmid DNA preparation; expression of
\end{flushleft}


\begin{flushleft}
cloned DNA in bacteria; isolation of auxotrophic mutants.
\end{flushleft}





\begin{flushleft}
BBL850 Advanced Biochemical Engineering
\end{flushleft}


\begin{flushleft}
5 Credits (3-0-4)
\end{flushleft}


\begin{flushleft}
Kinetics of cell growth; Mathematical models for substrate uptake
\end{flushleft}


\begin{flushleft}
and product formation; Plasmid stability in recombinant cell cultures;
\end{flushleft}


\begin{flushleft}
Kinetics of enzyme-catalyzed reactions; Media and air sterilization;
\end{flushleft}


\begin{flushleft}
Cell cultivation strategies; Novel bioreactor designs; Developments
\end{flushleft}


\begin{flushleft}
in aeration \& agitation in bioreactors; immobilized whole cell and
\end{flushleft}


\begin{flushleft}
immobilized enzyme reactors; RTD and mixing in bioreactors;
\end{flushleft}


\begin{flushleft}
Dynamics of mixed cultures; Scale-up and scale down of bioreactors.
\end{flushleft}


\begin{flushleft}
Laboratory: Microbial growth and product formation kinetics;
\end{flushleft}


\begin{flushleft}
enzyme kinetics; Effects of inhibitor on microbial growth; enzyme
\end{flushleft}


\begin{flushleft}
immobilization techniques; Bioconversion using immobilized enzyme
\end{flushleft}


\begin{flushleft}
preparation; Bioconversion in batch, fedbatch and continuous
\end{flushleft}


\begin{flushleft}
bioreactors; Oxygen transfer studies in fermentation; Mixing and
\end{flushleft}





\begin{flushleft}
agitation in fermenters; RTD studies; Mass transfer in immobilized
\end{flushleft}


\begin{flushleft}
cell/enzyme reactors.
\end{flushleft}





\begin{flushleft}
BBD851 Major Project Part 1 (BB5)
\end{flushleft}


\begin{flushleft}
6 Credits (0-0-12)
\end{flushleft}


\begin{flushleft}
BBD852 Major Project Part 2 (BB5)
\end{flushleft}


\begin{flushleft}
14 Credits (0-0-28)
\end{flushleft}


\begin{flushleft}
BBD853 Major Project Part 1 (BB5)
\end{flushleft}


\begin{flushleft}
4 Credits (0-0-8)
\end{flushleft}


\begin{flushleft}
BBD854 Major Project Part 2 (BB5)
\end{flushleft}


\begin{flushleft}
16 Credits (0-0-32)
\end{flushleft}


\begin{flushleft}
BBD895 MS Research
\end{flushleft}


\begin{flushleft}
36 Credits (0-0-72)
\end{flushleft}





153





\begin{flushleft}
\newpage
Department of Chemical Engineering
\end{flushleft}


\begin{flushleft}
CLL110 Transport Phenomena
\end{flushleft}


\begin{flushleft}
4 Credits (3-1-0)
\end{flushleft}





\begin{flushleft}
CLL133 Powder Processing and Technology
\end{flushleft}


\begin{flushleft}
3 Credits (3-0-0)
\end{flushleft}





\begin{flushleft}
Vector and tensor analysis. Euler/Lagrangian viewpoint of momentum
\end{flushleft}


\begin{flushleft}
transport, stress tensor and Newton's law of viscosity, shell momentum
\end{flushleft}


\begin{flushleft}
balances. Derivation of equations of change for isothermal, nonisothermal, and multicomponent systems. Solutions to 1D flow
\end{flushleft}


\begin{flushleft}
problems involving Newtonian or non-Newtonian fluids, friction
\end{flushleft}


\begin{flushleft}
factor. Mechanisms of energy transport, energy flux for conduction,
\end{flushleft}


\begin{flushleft}
convection and viscous dissipation, solutions to 1D conduction and
\end{flushleft}


\begin{flushleft}
convection problems. Mechanisms of mass transport, mass and molar
\end{flushleft}


\begin{flushleft}
diffusion fluxes, derivation and application of continuity equation to
\end{flushleft}


\begin{flushleft}
mass transfer in binary mixtures. Dimensional analysis of equations
\end{flushleft}


\begin{flushleft}
of change to solve higher dimensional transport problems. Unsteadystate momentum, heat, and mass transport.
\end{flushleft}





\begin{flushleft}
Powder characterization for size, shape, surface area and flowability
\end{flushleft}


\begin{flushleft}
and relation between them. Characterization techniques: light
\end{flushleft}


\begin{flushleft}
scattering, light extinction, sedimentation, ultrasonic methods. Powder
\end{flushleft}


\begin{flushleft}
storage: designing of silos, flow of powders, measurement of flow
\end{flushleft}


\begin{flushleft}
factors, analytical methods for flow problems in chutes, cyclones and
\end{flushleft}


\begin{flushleft}
silos, funnel and mass flow. Segregation of powder during flow through
\end{flushleft}


\begin{flushleft}
chutes and air-induced flows. Segregation during heap formation.
\end{flushleft}


\begin{flushleft}
Comminution equipment: selection and designing. Particle size control
\end{flushleft}


\begin{flushleft}
in grinding circuit analysis. Gas-solid separation equipment, application
\end{flushleft}


\begin{flushleft}
for pollution control.
\end{flushleft}





\begin{flushleft}
CLL111 Material and Energy Balances
\end{flushleft}


\begin{flushleft}
4 Credits (2-2-0)
\end{flushleft}


\begin{flushleft}
Mathematics and engineering calculations, dimensional groups and
\end{flushleft}


\begin{flushleft}
constants. Vapour pressure; Clausius-Clapeyron equation, Cox chart,
\end{flushleft}


\begin{flushleft}
Duhring's plot, Raoult's law. Humidity and saturation, humid heat,
\end{flushleft}


\begin{flushleft}
humid volume, dew point, humidity chart and its use. Crystallization,
\end{flushleft}


\begin{flushleft}
dissolution. Ideal gas behavior. Material balance: solving material
\end{flushleft}


\begin{flushleft}
balance problems with and without chemical reaction, recycle, bypass
\end{flushleft}


\begin{flushleft}
and purge calculations, aid of computers in solving material balance
\end{flushleft}


\begin{flushleft}
problems. Energy balance: closed and open systems, heat capacity,
\end{flushleft}


\begin{flushleft}
calculation of enthalpy changes, energy balances with chemical
\end{flushleft}


\begin{flushleft}
reaction, heat of vaporization, heat of formation, heat of combination,
\end{flushleft}


\begin{flushleft}
heat of reaction.
\end{flushleft}





\begin{flushleft}
CLL113 Numerical Methods in Chemical Engineering
\end{flushleft}


\begin{flushleft}
4 Credits (3-0-2)
\end{flushleft}


\begin{flushleft}
Overlaps with: MTL107, MTP290, MTL445, CVL734, COL726
\end{flushleft}


\begin{flushleft}
Estimation and round-off error calculations. Solution of linear algebraic
\end{flushleft}


\begin{flushleft}
equations via Gauss elimination, matrix inversion and LU decomposition,
\end{flushleft}


\begin{flushleft}
Gauss-Seidel method. Solving non-linear algebraic equations with the
\end{flushleft}


\begin{flushleft}
help of root finding. Numerical integration and differentiation. Solution
\end{flushleft}


\begin{flushleft}
of ordinary differential equations encountered in initial/boundary
\end{flushleft}


\begin{flushleft}
value problems via implicit and explicit methods, solution of partial
\end{flushleft}


\begin{flushleft}
differential equations, Chemical engineering problems where the
\end{flushleft}


\begin{flushleft}
above mentioned numerical schemes are involved will be illustrated.
\end{flushleft}





\begin{flushleft}
CLL121 Chemical Engineering Thermodynamics
\end{flushleft}


\begin{flushleft}
4 Credits (3-1-0)
\end{flushleft}


\begin{flushleft}
Overlaps with: MCL140, MCL141, MCL142
\end{flushleft}


\begin{flushleft}
Review of conservation of energy, mass and introduction to workenergy conversions, and the concept of entropy. Application to closed
\end{flushleft}


\begin{flushleft}
and open systems, application in analysis of energy and efficiency
\end{flushleft}


\begin{flushleft}
of equipment. State and properties of pure fluids under different
\end{flushleft}


\begin{flushleft}
conditions and in flow through equipment. Use of equations of
\end{flushleft}


\begin{flushleft}
states, graphs, correlations and tables to estimate fluid properties,
\end{flushleft}


\begin{flushleft}
understanding the relationships between fluid properties and changes
\end{flushleft}


\begin{flushleft}
in properties. Equilibrium properties of pure materials and mixtures.
\end{flushleft}


\begin{flushleft}
Understanding the phase behaviour and phase transitions of pure
\end{flushleft}


\begin{flushleft}
fluids. Thermodynamic analysis of fluids in standard fixtures and
\end{flushleft}


\begin{flushleft}
equipment (piping fixtures, power plants, engines, refrigerators).
\end{flushleft}


\begin{flushleft}
Equilibrium behaviour of mixtures of fluids, the nature of interactions
\end{flushleft}


\begin{flushleft}
between various fluids and how interactions affect their properties
\end{flushleft}


\begin{flushleft}
and phase transitions. Introduction to separation processes based
\end{flushleft}


\begin{flushleft}
on difference in equilibrium thermodynamic properties. Introduction
\end{flushleft}


\begin{flushleft}
to reaction equilibria.
\end{flushleft}





\begin{flushleft}
CLL122 Chemical Reaction Engineering I
\end{flushleft}


\begin{flushleft}
4 Credits (3-1-0)
\end{flushleft}


\begin{flushleft}
Introduction to rate equations, calculation of conversion in single
\end{flushleft}


\begin{flushleft}
reaction, kinetics of homogeneous reactions. Derivation of reactor
\end{flushleft}


\begin{flushleft}
design equations, analysis and sizing of reactors, data collection
\end{flushleft}


\begin{flushleft}
and plotting to determine rate constants. Reactor networks (series/
\end{flushleft}


\begin{flushleft}
parallel), concepts of selectivity and yield, reaction mechanisms.
\end{flushleft}


\begin{flushleft}
Temperature and pressure effects on reactions and reactor design,
\end{flushleft}


\begin{flushleft}
simultaneous material and energy balances, multiple steady-states.
\end{flushleft}


\begin{flushleft}
Residence time distributions in non ideal reactors.
\end{flushleft}





\begin{flushleft}
CLL141 Introduction to Materials for Chemical Engineers
\end{flushleft}


\begin{flushleft}
3 Credits (3-0-0)
\end{flushleft}


\begin{flushleft}
Brief introduction to crystalline solids - metals and semiconductors, types
\end{flushleft}


\begin{flushleft}
of atomic bonding and lattices. Semi-crystalline materials - ceramics,
\end{flushleft}


\begin{flushleft}
polymers, copolymers, liquid crystals and surfactants. Amorphous and
\end{flushleft}


\begin{flushleft}
composite systems such as glass, fibers, granular materials, matrices
\end{flushleft}


\begin{flushleft}
and alloys. Role of materials selection in design, structure-propertyprocessing-performance relationships. Materials characterization via
\end{flushleft}


\begin{flushleft}
experimental techniques. Special materials like biomaterials and zeolites.
\end{flushleft}





\begin{flushleft}
CLL222 Chemical Reaction Engineering II
\end{flushleft}


\begin{flushleft}
3 Credits (3-0-0)
\end{flushleft}


\begin{flushleft}
Pre-requisites: CLL122
\end{flushleft}


\begin{flushleft}
Definition of catalysis, homogeneous and heterogeneous catalysis.
\end{flushleft}


\begin{flushleft}
Adsorption on catalytic surfaces, kinetic models, catalyst preparation,
\end{flushleft}


\begin{flushleft}
physical characterization of catalysts, supported metal catalysts.
\end{flushleft}


\begin{flushleft}
Mass transfer and internal diffusion effects in catalytic reactions,
\end{flushleft}


\begin{flushleft}
Thiele modulus and effectiveness factor, falsification of kinetics,
\end{flushleft}


\begin{flushleft}
catalyst deactivation.
\end{flushleft}


\begin{flushleft}
Packed bed reactor design, introduction to other multiphase reactors, gasliquid reactors and enhancement factor. Gas-solid non-catalytic reactions.
\end{flushleft}





\begin{flushleft}
CLL231 Fluid Mechanics for Chemical Engineers
\end{flushleft}


\begin{flushleft}
4 Credits (3-1-0)
\end{flushleft}


\begin{flushleft}
Pre-requisites: CLL110
\end{flushleft}


\begin{flushleft}
Overlaps with: APL107, APL106, APL105
\end{flushleft}


\begin{flushleft}
Introduction to fluids, Forces on fluids, Fluid statics, Hydrostatic force
\end{flushleft}


\begin{flushleft}
on submerged bodies, Rigid body motion, Kinematics of flow - Eulerian
\end{flushleft}


\begin{flushleft}
and Lagrangian descriptions, Flow visualization, Integral analysis mass and momentum balances, Bernoulli equation, Flow through pipes
\end{flushleft}


\begin{flushleft}
and ducts, Flow measurement, Flow transportation - pumps, blowers
\end{flushleft}


\begin{flushleft}
and compressors, Differential analysis of flow, Conservation of mass,
\end{flushleft}


\begin{flushleft}
linear and angular momentum, Navier-Stokes equation, Unidirectional
\end{flushleft}


\begin{flushleft}
flows, Viscous flows, Skin friction and form friction, Lubrication
\end{flushleft}


\begin{flushleft}
approximation, Potential flows, Boundary layer theory, Blasius equation
\end{flushleft}


\begin{flushleft}
for flow over a flat plate, Boundary layer separation, Drag and lift force
\end{flushleft}


\begin{flushleft}
on immersed bodies, Similitude analysis, Turbulent flows.
\end{flushleft}





\begin{flushleft}
CLL251 Heat Transfer for Chemical Engineers
\end{flushleft}


\begin{flushleft}
4 Credits (3-1-0)
\end{flushleft}


\begin{flushleft}
Pre-requisites: CLL110
\end{flushleft}


\begin{flushleft}
Overlaps with: MCL242
\end{flushleft}


\begin{flushleft}
Modes of heat transfer - conduction, convection, radiation; Heat
\end{flushleft}


\begin{flushleft}
transfer coefficients in natural and forced convection; Basic
\end{flushleft}


\begin{flushleft}
conservation equations; Heat transfer with phase change; Design of
\end{flushleft}


\begin{flushleft}
double pipe heat exchangers, shell and tube heat exchangers and
\end{flushleft}


\begin{flushleft}
evaporators; Introduction to radiative heat transfer. Unsteady state
\end{flushleft}


\begin{flushleft}
heat transfer. Two-dimensional heat transfer problems.
\end{flushleft}





\begin{flushleft}
CLL252 Mass Transfer I
\end{flushleft}


\begin{flushleft}
3 Credits (3-0-0)
\end{flushleft}


\begin{flushleft}
Pre-requisites: CLL110
\end{flushleft}


\begin{flushleft}
Lattice, Fick's, Stefan-Maxwell, Stokes-Einstein and irreversible
\end{flushleft}


\begin{flushleft}
thermodynamic approaches to diffusivity of binary and multicomponent
\end{flushleft}


\begin{flushleft}
system. Film theory and other theories of mass transfer. Analogy and
\end{flushleft}





154





\begin{flushleft}
\newpage
Chemical Engineering
\end{flushleft}





\begin{flushleft}
correlation approaches to mass transfer coefficients in interphase
\end{flushleft}


\begin{flushleft}
mass transfer. Analysis of co-current, counter-current and cross flow
\end{flushleft}


\begin{flushleft}
stage cascades. Design and operating conditions of differential contact
\end{flushleft}


\begin{flushleft}
equipment such as packed towers for absorption, adsorption, drying
\end{flushleft}


\begin{flushleft}
and leaching.
\end{flushleft}





\begin{flushleft}
CLL261 Process Dynamics and Control
\end{flushleft}


\begin{flushleft}
4 Credits (3-1-0)
\end{flushleft}


\begin{flushleft}
Pre-requisites: MTL100, CLL111
\end{flushleft}


\begin{flushleft}
Overlaps with: MCL212, ELL225, ELL205
\end{flushleft}


\begin{flushleft}
Introduction to automation, block diagrams; revision of Laplace
\end{flushleft}


\begin{flushleft}
transform. Modeling based on transfer function approach, open-loop
\end{flushleft}


\begin{flushleft}
systems: dynamic response of first order systems, first order systems
\end{flushleft}


\begin{flushleft}
in series, second order systems, and transportation lag. Feedback
\end{flushleft}


\begin{flushleft}
control: P, PI, PID controllers. Dynamic response of closed loop
\end{flushleft}


\begin{flushleft}
systems Linear stability analysis: Routh stability criterion, root locus
\end{flushleft}


\begin{flushleft}
diagrams. Frequency response: Bode diagrams, Nyquist diagrams,
\end{flushleft}


\begin{flushleft}
Bode and Nyquist stability criterion. Controller tuning: ZeiglerNichols and Cohen-Coon methods. Introduction to advanced control:
\end{flushleft}


\begin{flushleft}
feedforward control, cascade control, dead time compensation, ratio
\end{flushleft}


\begin{flushleft}
control, internal model control.
\end{flushleft}





\begin{flushleft}
CLL271 Introduction to Industrial Biotechnology
\end{flushleft}


\begin{flushleft}
3 Credits (3-0-0)
\end{flushleft}


\begin{flushleft}
Pre-requisites: CLL110
\end{flushleft}


\begin{flushleft}
Overlaps with: BEN150, BBL431, BBL731
\end{flushleft}


\begin{flushleft}
Introduction to biopharmaceutical industry. Monod kinetics. Michaelis
\end{flushleft}


\begin{flushleft}
Menten kinetics. Introduction to the different bioprocessing unit
\end{flushleft}


\begin{flushleft}
operations utilized in production of biotech drugs - upstream,
\end{flushleft}


\begin{flushleft}
harvest, and downstream. Design, control and scale up of bioreactor.
\end{flushleft}


\begin{flushleft}
Introduction to analytical methods used for characterization of biotech
\end{flushleft}


\begin{flushleft}
products and processes (high performance liquid chromatography,
\end{flushleft}


\begin{flushleft}
mass spectrophotometry, capillary electrophoresis, near infrared
\end{flushleft}


\begin{flushleft}
spectroscopy, UV spectroscopy). Fundamentals and design of
\end{flushleft}


\begin{flushleft}
filtration and other membrane based separation techniques. Process
\end{flushleft}


\begin{flushleft}
chromatography - theory, practice, design and scale-up. Mixing,
\end{flushleft}


\begin{flushleft}
heat transfer and mass transfer in bioprocessing unit operations.
\end{flushleft}


\begin{flushleft}
Scale-up of filtration and chromatography unit operations utilized in
\end{flushleft}


\begin{flushleft}
bioprocessing: procedures, issues that frequently occur and possible
\end{flushleft}


\begin{flushleft}
solutions. Process design, control and optimization. Current topics in
\end{flushleft}


\begin{flushleft}
biopharmaceutical technology.
\end{flushleft}





\begin{flushleft}
CLL296 Nano-engineering of Soft Materials
\end{flushleft}


\begin{flushleft}
3 Credits (3-0-0)
\end{flushleft}


\begin{flushleft}
Pre-requisites: CLL121
\end{flushleft}


\begin{flushleft}
Overlaps with: PYL421
\end{flushleft}





\begin{flushleft}
CLL331 Fluid-Particle Mechanics
\end{flushleft}


\begin{flushleft}
4 Credits (3-1-0)
\end{flushleft}


\begin{flushleft}
Pre-requisites: CLL110, CLL231
\end{flushleft}


\begin{flushleft}
Introduction to industries dealing with the particles (solid, liquid, gas,
\end{flushleft}


\begin{flushleft}
soft-materials: colloids, polymer), solid particles: particle size, shape
\end{flushleft}


\begin{flushleft}
and their distribution, relationship among shape factors and particle
\end{flushleft}


\begin{flushleft}
dimensions, specific surface area, measurement of surface area and
\end{flushleft}


\begin{flushleft}
particle size distribution, drag coefficient, packed bed, fluidization.
\end{flushleft}


\begin{flushleft}
Sedimentation: settling, hindered settling, design of settling tank,
\end{flushleft}


\begin{flushleft}
filtration, centrifugal separation, cyclone separator, mixing (solid-solid,
\end{flushleft}


\begin{flushleft}
solid-liquid and liquid-liquid), segregation.
\end{flushleft}


\begin{flushleft}
Size reduction, size enlargement, flow properties of slurries, behaviour
\end{flushleft}


\begin{flushleft}
of colloidal particles in dispersed condition.
\end{flushleft}





\begin{flushleft}
CLL352 Mass Transfer II
\end{flushleft}


\begin{flushleft}
4 Credits (3-1-0)
\end{flushleft}


\begin{flushleft}
Pre-requisites: CLL252
\end{flushleft}


\begin{flushleft}
Review of VLE. Separation quantification: separation factor, relative
\end{flushleft}


\begin{flushleft}
volatility, key components, flash: graphical and algebraic (RichfordRice) method. Differential distillation, binary distillation: McCabe-Thiele
\end{flushleft}


\begin{flushleft}
method - minimum reflux, minimum number of stages, open steam,
\end{flushleft}


\begin{flushleft}
multiple feeds, side streams. Packed columns - HETP, HTU method.
\end{flushleft}


\begin{flushleft}
Column pressure. Tray efficiency. Column sizing, sieve tray design,
\end{flushleft}


\begin{flushleft}
packed column design.
\end{flushleft}


\begin{flushleft}
LLE - equilibrium diagram, selection of solvent, design calculations
\end{flushleft}


\begin{flushleft}
for single stage, cascade of stages using Hunter and Nash graphical
\end{flushleft}


\begin{flushleft}
method, McCabe-Thiele method, continuous contacting.
\end{flushleft}


\begin{flushleft}
Multicomponent system: selection of key components, approximate
\end{flushleft}


\begin{flushleft}
- FUG method, DOF for cascade of stages, MESH formulation,
\end{flushleft}


\begin{flushleft}
introduction to azeotropic and extractive distillation, adsoption
\end{flushleft}


\begin{flushleft}
equilibrium, breakthrough curve.
\end{flushleft}





\begin{flushleft}
CLL361 Instrumentation and Automation
\end{flushleft}


\begin{flushleft}
2.5 Credits (1-0-3)
\end{flushleft}


\begin{flushleft}
Pre-requisites: CLL261
\end{flushleft}


\begin{flushleft}
Signals and standards (pneumatic, voltage, current). Basics of
\end{flushleft}


\begin{flushleft}
control loop components: sensors, transmitters, transducers, control
\end{flushleft}


\begin{flushleft}
valves, and converters. Measurement devices for process variables:
\end{flushleft}


\begin{flushleft}
temperature, pressure, level, flow, pH, humidity, density, and viscosity.
\end{flushleft}


\begin{flushleft}
Control valves, actuators, positioners; computer-based control
\end{flushleft}


\begin{flushleft}
systems: PLC, DCS, SCADA.
\end{flushleft}





\begin{flushleft}
CLL371 Chemical Process Technology and Economics
\end{flushleft}


\begin{flushleft}
4 Credits (3-1-0)
\end{flushleft}


\begin{flushleft}
Pre-requisites: CLL252, CLL122
\end{flushleft}





\begin{flushleft}
Mathematical characterization of phase transitions in soft matter, e.g.
\end{flushleft}


\begin{flushleft}
thin films, polymers and colloidal suspensions. Universality in phase
\end{flushleft}


\begin{flushleft}
separation kinetics. Evolution of order parameter. Time dependent
\end{flushleft}


\begin{flushleft}
mean field theories (MFTs). Kinetically-driven morphological changes
\end{flushleft}


\begin{flushleft}
in nano-pattern formation in thin films. Colloidal crystallization and
\end{flushleft}


\begin{flushleft}
at liquid fronts. Field-induced effects on assembly in complex fluids.
\end{flushleft}





\begin{flushleft}
CLP301 Chemical Engineering Laboratory I
\end{flushleft}


\begin{flushleft}
1.5 Credits (0-0-3)
\end{flushleft}


\begin{flushleft}
Pre-requisites: CLL231, CLL251
\end{flushleft}


\begin{flushleft}
Practicals in fluid mechanics and heat transfer.
\end{flushleft}





\begin{flushleft}
Introduction to process flowsheet, equipment symbols and sections
\end{flushleft}


\begin{flushleft}
of a chemical plant. Use of flowsheeting software. Process synthesis
\end{flushleft}


\begin{flushleft}
and process flow diagrams of chemical plants (gas-liquid, liquidsolid, gas-liquid-solid handling plants). Fertilizer technology:
\end{flushleft}


\begin{flushleft}
manufacture of fertilizers including naphtha reforming, air separation,
\end{flushleft}


\begin{flushleft}
ammonia synthesis technology. Utilities and safety issues in fertilizer
\end{flushleft}


\begin{flushleft}
plants. Chlor-alkali and sulfuric acid manufacturing. Refining and
\end{flushleft}


\begin{flushleft}
petrochemical technology: Crude occurrence, properties, distillation,
\end{flushleft}


\begin{flushleft}
refinery processes and technology, petrochemical technologies. Semiconductor chip manufacturing. food technology. Safety and hazard
\end{flushleft}


\begin{flushleft}
analysis, and debottlenecking of chemical plants. Introduction to
\end{flushleft}


\begin{flushleft}
process engineering economics.
\end{flushleft}





\begin{flushleft}
CLL390 Process Utilities and Pipeline Design
\end{flushleft}


\begin{flushleft}
3 Credits (3-0-0)
\end{flushleft}


\begin{flushleft}
Pre-requisites: CLL231
\end{flushleft}





\begin{flushleft}
CLP302 Chemical Engineering Laboratory II
\end{flushleft}


\begin{flushleft}
1.5 Credits (0-0-3)
\end{flushleft}


\begin{flushleft}
Pre-requisites: CLL331, CLL252
\end{flushleft}


\begin{flushleft}
Practicals in unit operations, mechanical operations, fluid-particle
\end{flushleft}


\begin{flushleft}
mechanics and principles of mass transfer.
\end{flushleft}





\begin{flushleft}
CLP303 Chemical Engineering Laboratory III
\end{flushleft}


\begin{flushleft}
1.5 Credits (0-0-3)
\end{flushleft}


\begin{flushleft}
Pre-requisites: CLL121, CLL122
\end{flushleft}


\begin{flushleft}
Practicals in reaction engineering, thermodynamics and chemical
\end{flushleft}


\begin{flushleft}
processing.
\end{flushleft}





\begin{flushleft}
Transportation and measurement of utilities like air, water and steam.
\end{flushleft}


\begin{flushleft}
Handling of steam. Design of insulation for steam carrying pipes,
\end{flushleft}


\begin{flushleft}
water hammer. Design of flash tank. Water treatment and reduction of
\end{flushleft}


\begin{flushleft}
scaling. Storage tank analysis for water. Piping network design, fittings
\end{flushleft}


\begin{flushleft}
and valves. Air treatment: cleaning and Dehumidification, design of
\end{flushleft}


\begin{flushleft}
refrigeration and air-conditioning systems. Transportation of air: duct
\end{flushleft}


\begin{flushleft}
design, selection of blowers and compressors. Instrumentation and
\end{flushleft}


\begin{flushleft}
control for fluid transportation. Energy audit for industrial air and
\end{flushleft}


\begin{flushleft}
steam handling systems.
\end{flushleft}





155





\begin{flushleft}
\newpage
Chemical Engineering
\end{flushleft}





\begin{flushleft}
CLL402 Process Plant Design
\end{flushleft}


\begin{flushleft}
3 Credits (3-0-0)
\end{flushleft}


\begin{flushleft}
Pre-requisites: CLL371
\end{flushleft}


\begin{flushleft}
Overlaps with: CLL703
\end{flushleft}





\begin{flushleft}
reversibility; First and Second Laws of Thermodynamics; Equations
\end{flushleft}


\begin{flushleft}
of state; Equilibrium behaviour of mixtures of fluids; Phase equilibria
\end{flushleft}


\begin{flushleft}
and VLE; Reaction thermodynamics.
\end{flushleft}





\begin{flushleft}
Plant layout and flowsheeting. Issues related to materials
\end{flushleft}


\begin{flushleft}
handling, equipment selection and design (pumps, blowers and
\end{flushleft}


\begin{flushleft}
compressions, mixers, conveyors, seperation columns, reactors),
\end{flushleft}


\begin{flushleft}
utilities and auxiliaries, offsite facilities. Cost estimation. Selection
\end{flushleft}


\begin{flushleft}
and detailed design of equipment. Steam handling. Valves, piping
\end{flushleft}


\begin{flushleft}
and instrumentation. Environmental footprint assessment, pollution
\end{flushleft}


\begin{flushleft}
reduction, and life cycle analysis of process plant.
\end{flushleft}





\begin{flushleft}
CLD411 B. Tech. project
\end{flushleft}


\begin{flushleft}
4 Credits (0-0-8)
\end{flushleft}


\begin{flushleft}
Formulation of the problem. Literature search. Design and fabrication
\end{flushleft}


\begin{flushleft}
of the experimental setup. Study of experimental techniques in the
\end{flushleft}


\begin{flushleft}
case of experimental projects. Formulation of equations and analytical/
\end{flushleft}


\begin{flushleft}
numerical solution in case of modeling projects. Development of
\end{flushleft}


\begin{flushleft}
software. Analysis of results. Presentation of results and scientific
\end{flushleft}


\begin{flushleft}
reporting in form of thesis and presentation.
\end{flushleft}





\begin{flushleft}
CLD412 Major Project in Energy and Environment
\end{flushleft}


\begin{flushleft}
5 Credits (0-0-10)
\end{flushleft}





\begin{flushleft}
Reaction equilibria and chemical kinetics; Ideal reactors; Isothermal
\end{flushleft}


\begin{flushleft}
reactor design; Temperature and pressure effects in ideal reactors;
\end{flushleft}


\begin{flushleft}
Heterogeneous catalysis and effectiveness factors; Fluid-solid noncatalytic reactions.
\end{flushleft}





\begin{flushleft}
CLL703 Process Engineering
\end{flushleft}


\begin{flushleft}
3 Credits (3-0-0)
\end{flushleft}


\begin{flushleft}
Process synthesis, material balances and decision making in reactors
\end{flushleft}


\begin{flushleft}
with recycle streams, input-output structure of flowsheet for batch
\end{flushleft}


\begin{flushleft}
vs. continuous reactors, hierarchial approach for process engineering
\end{flushleft}


\begin{flushleft}
design, reactor and separation system selection guidelines, distillation
\end{flushleft}


\begin{flushleft}
column sequencing, heat exchanger network design, pinch technology,
\end{flushleft}


\begin{flushleft}
utility selection, grand composite curve, steam and cooling water
\end{flushleft}


\begin{flushleft}
circuits, integration of heat pumps and heat engines
\end{flushleft}


\begin{flushleft}
Process economics: Cost estimation, annuities, perpetuities and
\end{flushleft}


\begin{flushleft}
present value, tax and depreciation, profitability measures, comparison
\end{flushleft}


\begin{flushleft}
of equipments and projects, NPV, IRR, risk management.
\end{flushleft}


\begin{flushleft}
Process modeling tools: AspenPlus® or Promax that are used in
\end{flushleft}


\begin{flushleft}
industry for large scale problem solving to undertake problems of
\end{flushleft}


\begin{flushleft}
current interest.
\end{flushleft}





\begin{flushleft}
CLD413 Major Project in Complex Fluids
\end{flushleft}


\begin{flushleft}
5 Credits (0-0-10)
\end{flushleft}





\begin{flushleft}
CLP704 Technical Communication for Chemical
\end{flushleft}


\begin{flushleft}
Engineers
\end{flushleft}


\begin{flushleft}
1 Credit (0-0-2)
\end{flushleft}





\begin{flushleft}
CLD414 Major Project in Process Engineering, Modeling
\end{flushleft}


\begin{flushleft}
and Optimization
\end{flushleft}


\begin{flushleft}
5 Credits (0-0-10)
\end{flushleft}





\begin{flushleft}
Technical paper and report writing, Knowledge of leading Chemical
\end{flushleft}


\begin{flushleft}
Engineering journals and conferences, carrying out literature search,
\end{flushleft}


\begin{flushleft}
research methodology, paper referencing and critiquing, ethics and
\end{flushleft}


\begin{flushleft}
plagiarism, improving presentation and communication skills.
\end{flushleft}





\begin{flushleft}
CLD415 Major Project in Biopharmaceuticals and Fine
\end{flushleft}


\begin{flushleft}
Chemicals
\end{flushleft}


\begin{flushleft}
5 Credits (0-0-10)
\end{flushleft}





\begin{flushleft}
CLL705 Petroleum Reservoir Engineering
\end{flushleft}


\begin{flushleft}
3 Credits (3-0-0)
\end{flushleft}


\begin{flushleft}
Pre-requisites: CLL110, CLL121
\end{flushleft}





\begin{flushleft}
CLL475 Safety and Hazards in Process Industries
\end{flushleft}


\begin{flushleft}
3 Credits (3-0-0)
\end{flushleft}


\begin{flushleft}
Pre-requisites: CLL371
\end{flushleft}


\begin{flushleft}
Loss statistics and prevention. Fires and explosions. Hazards related
\end{flushleft}


\begin{flushleft}
to static electricity. Safety system designs for prevention of fire and
\end{flushleft}


\begin{flushleft}
explosions. Hazards due to toxicity. Industrial hygiene. Hazards
\end{flushleft}


\begin{flushleft}
identification and risk assessment methods. Event probability and
\end{flushleft}


\begin{flushleft}
failure frequency analysis. Case studies.
\end{flushleft}





\begin{flushleft}
Introduction of static model including porosity, permeability,
\end{flushleft}


\begin{flushleft}
compressibility and saturations. Crude oil phase behaviour and
\end{flushleft}


\begin{flushleft}
their measurement techniques for reservoir and laboratory settings.
\end{flushleft}


\begin{flushleft}
Meaning and calculation of {`}oil in place' numbers with respect to
\end{flushleft}


\begin{flushleft}
different recovery mechanisms. Material balance for hydrocarbon
\end{flushleft}


\begin{flushleft}
reservoirs. Pressure transient analysis. Primary, secondary and tertiary
\end{flushleft}


\begin{flushleft}
recovery mechanisms, Buckley- Leverett theory (fractional flow curves)
\end{flushleft}


\begin{flushleft}
for immiscible and miscible displacement. Production forecasting and
\end{flushleft}


\begin{flushleft}
introduction to reservoir simulation.
\end{flushleft}





\begin{flushleft}
CLL477 Materials of Construction
\end{flushleft}


\begin{flushleft}
3 Credits (3-0-0)
\end{flushleft}


\begin{flushleft}
Pre-requisites: CLL371
\end{flushleft}





\begin{flushleft}
CLL706 Petroleum Production Engineering
\end{flushleft}


\begin{flushleft}
3 Credits (3-0-0)
\end{flushleft}


\begin{flushleft}
Pre-requisites: CLL231, CLL121
\end{flushleft}





\begin{flushleft}
Types and mechanisms of corrosion, factors influencing corrosion.
\end{flushleft}


\begin{flushleft}
Corrosion testing methods. Combating corrosion in metals and nonmetals. High and low temperature materials. Selection of materials of
\end{flushleft}


\begin{flushleft}
construction for handling different chemicals. Industrial applications
\end{flushleft}


\begin{flushleft}
and case studies.
\end{flushleft}





\begin{flushleft}
CLL701 Modelling of Transport Processes
\end{flushleft}


\begin{flushleft}
2 Credits (2-0-0)
\end{flushleft}


\begin{flushleft}
Fundamentals of momentum transport, Mass and momentum
\end{flushleft}


\begin{flushleft}
conservation equations and their applications to solve 1-D problems,
\end{flushleft}


\begin{flushleft}
Fundamentals of heat transport, Equation of energy/temperature and
\end{flushleft}


\begin{flushleft}
its application to solve problems involving conduction, Fundamentals
\end{flushleft}


\begin{flushleft}
of mass transport, Equation of mass conservation and its application
\end{flushleft}


\begin{flushleft}
to solve problems involving binary diffusion.
\end{flushleft}


\begin{flushleft}
Introduction to methods for solution of algebraic equations, Methods
\end{flushleft}


\begin{flushleft}
for solution of ODEs, Functions, approximations and regression
\end{flushleft}


\begin{flushleft}
analysis, Introduction to Design of Experiments.
\end{flushleft}





\begin{flushleft}
CLL702 Principles of Thermodynamics, Reaction
\end{flushleft}


\begin{flushleft}
Kinetics and Reactors
\end{flushleft}


\begin{flushleft}
2 Credits (2-0-0)
\end{flushleft}


\begin{flushleft}
Introduction to thermodynamics; Notion of equilibrium, states and
\end{flushleft}





\begin{flushleft}
Basic concepts: well drilling, well completions, drive mechanisms
\end{flushleft}


\begin{flushleft}
for different reservoirs, Darcy's law. Movement of fluids in the well,
\end{flushleft}


\begin{flushleft}
different artificial lift mechanisms, VLP (vertical lift performance
\end{flushleft}


\begin{flushleft}
curves), IPR (inflow performance relationships). Well analysis
\end{flushleft}


\begin{flushleft}
tools (different well performance curves, well logging). Problem
\end{flushleft}


\begin{flushleft}
identification in wells (examples). Well stimulation techniques.
\end{flushleft}





\begin{flushleft}
CLL707 Population Balance Modeling
\end{flushleft}


\begin{flushleft}
3 Credits (3-0-0)
\end{flushleft}


\begin{flushleft}
Pre-requisites: MTL101, CLL331, CLL352
\end{flushleft}


\begin{flushleft}
Theory of crystallization. Particle size distribution, particle phase
\end{flushleft}


\begin{flushleft}
space. Population balance equation for convection in state space
\end{flushleft}


\begin{flushleft}
(pure growth). Solution of PBE using method of characteristics. PBE
\end{flushleft}


\begin{flushleft}
with breakage and coalescence/aggregation terms. Scaling theory
\end{flushleft}


\begin{flushleft}
and phenomenological models for rate of breakage and coalescence
\end{flushleft}


\begin{flushleft}
induced by turbulence. Solution of PBE for pure breakage and pure
\end{flushleft}


\begin{flushleft}
coalescence. Moment transformation of PBE. Numerical approaches
\end{flushleft}


\begin{flushleft}
to solve PBE. Integrating PBE with transport equations.
\end{flushleft}





\begin{flushleft}
CLL720 Principles of Electrochemical Engineering
\end{flushleft}


\begin{flushleft}
3 Credits (3-0-0)
\end{flushleft}


\begin{flushleft}
Volta and Galvani potentials, electrochemical potential, electrochemical
\end{flushleft}





156





\begin{flushleft}
\newpage
Chemical Engineering
\end{flushleft}





\begin{flushleft}
equilibrium, Nernst equation. Born-Haber cycle for enthalpy and Gibbs
\end{flushleft}


\begin{flushleft}
free energy calculation, conventions for ionic species, solvation energy,
\end{flushleft}


\begin{flushleft}
ionic equilibrium. Electrochemical cell, standard electrode potential,
\end{flushleft}


\begin{flushleft}
Pourbaix diagram, Donnan potential, reversible electrode. Born model
\end{flushleft}


\begin{flushleft}
for ion-solvation energy. Ion-ion interactions: Debye-Huckel theory,
\end{flushleft}


\begin{flushleft}
activity coefficient of ionic solution, ion pair, Bjerrum theory and Fuoss
\end{flushleft}


\begin{flushleft}
theory. Ionic transport: migration, extended Nernst-Planck equation,
\end{flushleft}


\begin{flushleft}
electrochemical mobility and its relation with diffusivity, Stokes-Einstein
\end{flushleft}


\begin{flushleft}
equation, ionic conductivity, transport number, Kohlrausch law.
\end{flushleft}


\begin{flushleft}
Charged interface: surface excess quantity, Lippmann equation, GouyChapman model, Stern layer, internal and external Helmholtz layer,
\end{flushleft}


\begin{flushleft}
zeta potential, energy of double layer. Electrokinetic phenomena: Nonequilibrium formulation, diffusion potential, junction potential, PlanckHenderson equation, pH electrode, electrosmosis, electrophoresis,
\end{flushleft}


\begin{flushleft}
streaming potential, sedimentation potential. Introduction to electrode
\end{flushleft}


\begin{flushleft}
kinetics: Butler-Volmer formulation, Tafel equation.
\end{flushleft}





\begin{flushleft}
of emission profile from IC engine. Effect of fuel type and quality
\end{flushleft}


\begin{flushleft}
and engine performance on air quality. Automotive catalysts and
\end{flushleft}


\begin{flushleft}
monoliths. Diesel particulate filters and their operation. Selective
\end{flushleft}


\begin{flushleft}
catalytic reduction. Stationary sources of air pollutants. Household
\end{flushleft}


\begin{flushleft}
pollutants and control of indoor air quality. Control of pollutants from
\end{flushleft}


\begin{flushleft}
power plants.
\end{flushleft}





\begin{flushleft}
CLL721 Electrochemical Methods
\end{flushleft}


\begin{flushleft}
3 Credits (3-0-0)
\end{flushleft}





\begin{flushleft}
Kinetics of elementary steps (adsorption, desorption and surface
\end{flushleft}


\begin{flushleft}
reactions). Reaction on uniform and non-uniform surfaces. Structuresensitive and non-sensitive reactions on metals.
\end{flushleft}





\begin{flushleft}
Galvani Potential, Butler-Volmer Equation, Tafel Equation. Potential
\end{flushleft}


\begin{flushleft}
Step voltammetry, pulse voltammetry, cyclic voltammetry. Controlled
\end{flushleft}


\begin{flushleft}
current methods, current-interrupt measurements. Conductivity
\end{flushleft}


\begin{flushleft}
relaxation, impedance spectroscopy. Coulometric methods, scanning
\end{flushleft}


\begin{flushleft}
probe techniques, spectro-electrochemistry.
\end{flushleft}





\begin{flushleft}
CLL722 Electrochemical Conversion and Storage Devices
\end{flushleft}


\begin{flushleft}
3 Credits (3-0-0)
\end{flushleft}





\begin{flushleft}
CLL726 Molecular Modeling of Catalytic Reactions
\end{flushleft}


\begin{flushleft}
3 Credits (3-0-0)
\end{flushleft}


\begin{flushleft}
Pre-requisites: CLL222
\end{flushleft}


\begin{flushleft}
Sabatier principle. Catalytic cycle, transition state theory. Ensemble
\end{flushleft}


\begin{flushleft}
effect, defect sites, cluster size effects, metal-support interactions,
\end{flushleft}


\begin{flushleft}
structural effects, quantum size effects, electron transfer effects.
\end{flushleft}


\begin{flushleft}
Br{\O}nsted-Evans-Polanyi relations. Reactivity of transition-metal
\end{flushleft}


\begin{flushleft}
surfaces, quantum chemistry of chemical bond, bonding to transition
\end{flushleft}


\begin{flushleft}
metals, chemisorption.
\end{flushleft}





\begin{flushleft}
Electronic structure methods, potential energy surface, Born--
\end{flushleft}


\begin{flushleft}
Oppenheimer approximation, Hartree-Fock theory, self-consistent
\end{flushleft}


\begin{flushleft}
field, Kohn-Sham Density Functional Theory, Bloch's theorem and
\end{flushleft}


\begin{flushleft}
plane wave basis set, exchange-correlation functionals, pseudopotential. Search for transition state, dimer method, nudged elastic
\end{flushleft}


\begin{flushleft}
band method, density of states.
\end{flushleft}





\begin{flushleft}
Electrochemical cell, fuel cells, proton exchange membrane fuel cells,
\end{flushleft}


\begin{flushleft}
solid oxide fuel cells. Batteries, lead acid battery, Nickel-metal hydride
\end{flushleft}


\begin{flushleft}
(Ni-MH) rechargeable batteries, lithium-ion rechargeable batteries,
\end{flushleft}


\begin{flushleft}
liquid-redox rechargeable batteries. Electrochemical supercapacitors.
\end{flushleft}


\begin{flushleft}
Solar cells. Electrodialysis and reverse electrodialysis. Electrochemical
\end{flushleft}


\begin{flushleft}
hydrogen production and storage.
\end{flushleft}





\begin{flushleft}
Catalysis by metals, oxides, sulfides and zeolites. Aqueous phase
\end{flushleft}


\begin{flushleft}
heterogeneous catalysis and electrocatalysis.
\end{flushleft}





\begin{flushleft}
CLL723 Hydrogen Energy and Fuel Cell Technology
\end{flushleft}


\begin{flushleft}
3 Credits (3-0-0)
\end{flushleft}





\begin{flushleft}
Overlaps with: Basic concepts in heterogeneous catalysis, catalyst
\end{flushleft}


\begin{flushleft}
preparation and characterization, poisoning and regeneration.
\end{flushleft}


\begin{flushleft}
Industrially important catalysts and processes such as oxidation,
\end{flushleft}


\begin{flushleft}
processing of petroleum and hydrocarbons, synthesis gas and related
\end{flushleft}


\begin{flushleft}
processes. Commercial reactors: adiabatic and multi-tubular packed
\end{flushleft}


\begin{flushleft}
beds, fluidized bed, trickle-bed, slurry reactors. Heat and mass transfer
\end{flushleft}


\begin{flushleft}
and its role in heterogeneous catalysis. Calculations of effective
\end{flushleft}


\begin{flushleft}
diffusivity and thermal conductivity of porous catalysts. Reactor
\end{flushleft}


\begin{flushleft}
modeling. Chemistry and engineering aspects of catalytic processes
\end{flushleft}


\begin{flushleft}
along with problems arising in industry. Catalyst deactivation kinetics
\end{flushleft}


\begin{flushleft}
and modeling.
\end{flushleft}





\begin{flushleft}
Overview of fuel cells: low and high temperature fuel cells. Fuel cell
\end{flushleft}


\begin{flushleft}
thermodynamics -- heat and work potentials, prediction of reversible
\end{flushleft}


\begin{flushleft}
voltage, fuel cell efficiency. Fuel cell reaction kinetics -- electrode
\end{flushleft}


\begin{flushleft}
kinetics, overvoltages, exchange currents. Electrocatalysis - design,
\end{flushleft}


\begin{flushleft}
activation kinetics. Fuel cell charge and mass transport - transport
\end{flushleft}


\begin{flushleft}
in flow field, electrode and electrolyte. Fuel cell characterization- insitu and ex-situ characterization techniques, I-V curve, application of
\end{flushleft}


\begin{flushleft}
voltammetry and frequency response analyses. Fuel cell modeling
\end{flushleft}


\begin{flushleft}
and system integration. Fuel cell diagnostics. Balance of plant.
\end{flushleft}


\begin{flushleft}
Different routes of hydrogen generation: electrolysis versus reforming
\end{flushleft}


\begin{flushleft}
for hydrogen production, solar hydrogen. Hydrogen storage and
\end{flushleft}


\begin{flushleft}
transportation, safety issues. Cost expectation and life cycle analysis.
\end{flushleft}





\begin{flushleft}
CLL724 Environmental Engineering and Waste
\end{flushleft}


\begin{flushleft}
Management
\end{flushleft}


\begin{flushleft}
3 Credits (3-0-0)
\end{flushleft}


\begin{flushleft}
Overlaps with: CVL100, CVL212, CVL311, CVL312, BBL742
\end{flushleft}


\begin{flushleft}
The course covers the concept of ecological balance and the
\end{flushleft}


\begin{flushleft}
contribution of industrial and human activities in the changes of the
\end{flushleft}


\begin{flushleft}
environmental quality. Ecological cycles. Concept of pollutants and
\end{flushleft}


\begin{flushleft}
regulatory measures for the maintenance of environmental quality.
\end{flushleft}


\begin{flushleft}
Air pollution sources and its dependence on the atmospheric factors,
\end{flushleft}


\begin{flushleft}
atmospheric stability and dispersion of pollutants. Control of emission
\end{flushleft}


\begin{flushleft}
of pollutants using multi-cyclone systems, electrostatic precipitators,
\end{flushleft}


\begin{flushleft}
bag filters, wet scrubbers for gas cleaning, adsorption by activated
\end{flushleft}


\begin{flushleft}
carbon etc. Water pollution, its causes and effects. Pollutants and its
\end{flushleft}


\begin{flushleft}
dispersion in water bodies to predict water quality through modeling.
\end{flushleft}


\begin{flushleft}
Concept of inorganic and organic wastes and definition of BOD and
\end{flushleft}


\begin{flushleft}
COD. Control of water pollution by primary treatment and biological
\end{flushleft}


\begin{flushleft}
treatment systems. Solid waste management systems. Hazardous
\end{flushleft}


\begin{flushleft}
waste treatment, disposal and storage in engineered landfill.
\end{flushleft}





\begin{flushleft}
CLL725 Air Pollution Control Engineering
\end{flushleft}


\begin{flushleft}
3 Credits (3-0-0)
\end{flushleft}


\begin{flushleft}
Pre-requisites: CLL222
\end{flushleft}


\begin{flushleft}
Overview of air pollution from mobile and stationary sources. Modeling
\end{flushleft}





\begin{flushleft}
CLL727 Heterogeneous Catalysis and Catalytic Reactors
\end{flushleft}


\begin{flushleft}
3 Credits (3-0-0)
\end{flushleft}


\begin{flushleft}
Pre-requisites: CLL222
\end{flushleft}





\begin{flushleft}
CLL728 Biomass Conversion and Utilization
\end{flushleft}


\begin{flushleft}
3 Credits (3-0-0)
\end{flushleft}


\begin{flushleft}
Pre-requisites: CLL122
\end{flushleft}


\begin{flushleft}
Critical analysis of issues associated with implementing large scale
\end{flushleft}


\begin{flushleft}
biofuel and biomass energy production. Processes for converting
\end{flushleft}


\begin{flushleft}
feedstocks to biofuels by thermochemical methods. Biomass
\end{flushleft}


\begin{flushleft}
conversion catalysis, kinetics and reaction mechanisms, reactor design
\end{flushleft}


\begin{flushleft}
and scale up issues.
\end{flushleft}





\begin{flushleft}
CLL730 Structure, Transport and Reactions in BioNano
\end{flushleft}


\begin{flushleft}
Systems
\end{flushleft}


\begin{flushleft}
3 Credits (3-0-0)
\end{flushleft}


\begin{flushleft}
Pre-requisites: CLL110
\end{flushleft}


\begin{flushleft}
Overlaps with: SBV882, MCL442
\end{flushleft}


\begin{flushleft}
Introduction to biology: protein structure, composition, pKa and
\end{flushleft}


\begin{flushleft}
isoelectric point. Governing equations applied to biological systems:
\end{flushleft}


\begin{flushleft}
conservation laws, flux equations, mathematical functions and
\end{flushleft}


\begin{flushleft}
solutions, scaling and order, laminar flow. Electromechanical transport:
\end{flushleft}


\begin{flushleft}
biomolecular migration through blood capillaries, Poisson-Boltzmann
\end{flushleft}


\begin{flushleft}
equation in heterogeneous media, electrical-shear stress balance in
\end{flushleft}


\begin{flushleft}
electrical double layers. Transport across membranes: structure and
\end{flushleft}


\begin{flushleft}
self-assembly of lipid bilayers, ligand-receptor interactions, membrane
\end{flushleft}


\begin{flushleft}
permeability, Nernst potential, adsorption isotherms and transport across
\end{flushleft}


\begin{flushleft}
membrane. Estimation of transport coefficients based on biomolecular
\end{flushleft}


\begin{flushleft}
interactions. Research-specific case studies incorporating coupled
\end{flushleft}


\begin{flushleft}
migration through reactive, electrical and heterogeneity considerations.
\end{flushleft}





157





\begin{flushleft}
\newpage
Chemical Engineering
\end{flushleft}





\begin{flushleft}
CLL731 Advanced Transport Phenomena
\end{flushleft}


\begin{flushleft}
3 Credits (3-0-0)
\end{flushleft}


\begin{flushleft}
Pre-requisites: CLL110
\end{flushleft}


\begin{flushleft}
Review of fluid kinematics, conservation laws and constitutive
\end{flushleft}


\begin{flushleft}
equations. Solution methods for equations of change (e.g., unsteady
\end{flushleft}


\begin{flushleft}
fluid flow in bounded/unbounded geometries). Creeping flow and
\end{flushleft}


\begin{flushleft}
lubrication approximation. Surface tension driven flows and multiphase
\end{flushleft}


\begin{flushleft}
flows. Boundary layer theory. Unsteady heat and mass transport.
\end{flushleft}


\begin{flushleft}
Coupled transport processes-- forced convection heat and mass
\end{flushleft}


\begin{flushleft}
transport in confined/unconfined flows. Multicomponent energy and
\end{flushleft}


\begin{flushleft}
mass transport. Turbulence modeling.
\end{flushleft}





\begin{flushleft}
CLL732 Advanced Chemical Engineering Thermodynamics
\end{flushleft}


\begin{flushleft}
3 Credits (3-0-0)
\end{flushleft}


\begin{flushleft}
Pre-requisites: CLL121
\end{flushleft}


\begin{flushleft}
First and second law of thermodynamics. Application in analysis of
\end{flushleft}


\begin{flushleft}
energy and efficiency of equipment, flow through equipment. State and
\end{flushleft}


\begin{flushleft}
behavior of materials, degree of freedom analysis. Material properties
\end{flushleft}


\begin{flushleft}
as a function of conditions. Relationships between material properties,
\end{flushleft}


\begin{flushleft}
and changes in material properties. Equilibrium properties of materials:
\end{flushleft}


\begin{flushleft}
pure materials, and mixtures. A-priori probability postulate, ergodic
\end{flushleft}


\begin{flushleft}
hypothesis, introduction to microcanonical, canonical and grand
\end{flushleft}


\begin{flushleft}
canonical ensembles, derivation of physical properties for pure
\end{flushleft}


\begin{flushleft}
components and mixtures, ideal gas and lattice gas, virial coefficient
\end{flushleft}


\begin{flushleft}
calculations. Crystal structures, solutions, modeling and analysis of
\end{flushleft}


\begin{flushleft}
adsorption phenomena, relating them to macroscopic thermodynamics.
\end{flushleft}





\begin{flushleft}
CLL733 Industrial Multiphase Reactors
\end{flushleft}


\begin{flushleft}
3 Credits (3-0-0)
\end{flushleft}


\begin{flushleft}
Pre-requisites: CLL122, CLL222
\end{flushleft}


\begin{flushleft}
Introduction to advanced reactor analysis tools: RTD theory, RTD
\end{flushleft}


\begin{flushleft}
based models, axial dispersion, tank-in-series, multizonal models.
\end{flushleft}


\begin{flushleft}
Hydrodynamics and flow regimes. Transport effects in multiphase
\end{flushleft}


\begin{flushleft}
reactors, interplay of length and time scales. Process parameters
\end{flushleft}


\begin{flushleft}
of interest. Effectiveness factors in G/S and L/S systems, including
\end{flushleft}


\begin{flushleft}
non-isothermal effects. Enhancement factor in G/L systems. Models
\end{flushleft}


\begin{flushleft}
for non-catalytic heterogeneous reactions. Introduction to multiphase
\end{flushleft}


\begin{flushleft}
reactors and their applications, classification of multiphase reactors,
\end{flushleft}


\begin{flushleft}
performance/operating characteristics. Mechanically agitated reactors,
\end{flushleft}


\begin{flushleft}
bubble column/slurry bubble column reactors, fluidized bed, packed
\end{flushleft}


\begin{flushleft}
bed, trickle bed reactor reactors. Limitations of models, applications
\end{flushleft}


\begin{flushleft}
to design of multiphase reactors for specific applications.
\end{flushleft}





\begin{flushleft}
CLL734 Process Intensification and Novel Reactors
\end{flushleft}


\begin{flushleft}
3 Credits (3-0-0)
\end{flushleft}


\begin{flushleft}
Pre-requisites: CLL122, CLL222
\end{flushleft}


\begin{flushleft}
Introduction to process intensification, possible ways of process
\end{flushleft}


\begin{flushleft}
intensification and their examples. Introduction to multifunctional
\end{flushleft}


\begin{flushleft}
reactors/process equipment: reactive distillation, reactor-heatexchangers. membrane reactors, micro-reactors, structured/monolithic
\end{flushleft}


\begin{flushleft}
reactors. Intensification of conventional reactors/process equipments,
\end{flushleft}


\begin{flushleft}
analysis of fluid dynamics and transport effects of intensified reactors.
\end{flushleft}


\begin{flushleft}
Order of magnitude analysis of reaction rates, heat/mass transfer rates.
\end{flushleft}


\begin{flushleft}
Flow patterns in intensified reactors. Design and scale of intensified
\end{flushleft}


\begin{flushleft}
reactors, fabrication issues. Examples of process intensification.
\end{flushleft}





\begin{flushleft}
CLL735 Design of Multicomponent Separation Processes
\end{flushleft}


\begin{flushleft}
3 Credits (3-0-0)
\end{flushleft}


\begin{flushleft}
Pre-requisites: CLL352
\end{flushleft}


\begin{flushleft}
Overview of multi-component separation. Non-ideal solution and
\end{flushleft}


\begin{flushleft}
properties, equation of state, vapour liquid equilibrium. Multi
\end{flushleft}


\begin{flushleft}
component separation: Short cut method, rigorous calculations sum rate, boiling point and Newton's methods, inside-out method for
\end{flushleft}


\begin{flushleft}
designing of multi-component distillation, absorption and extraction
\end{flushleft}


\begin{flushleft}
column / contacting devices. Choice of column: tray, random packing
\end{flushleft}


\begin{flushleft}
and structured packing. Design of adsorption and ion exchange column.
\end{flushleft}


\begin{flushleft}
Crystallization. Affinity separation and chromatographic separation.
\end{flushleft}


\begin{flushleft}
Optimal reflux ratio (recycle stream) - operating expenditure versus
\end{flushleft}


\begin{flushleft}
capital expenditure for all types of columns and contacting devices.
\end{flushleft}





\begin{flushleft}
CLL736 Experimental Characterization of Multiphase
\end{flushleft}


\begin{flushleft}
Reactors
\end{flushleft}


\begin{flushleft}
3 Credits (3-0-0)
\end{flushleft}


\begin{flushleft}
Pre-requisites: CLL122, CLL222
\end{flushleft}


\begin{flushleft}
Analytical techniques: Introduction to various analytical techniques
\end{flushleft}


\begin{flushleft}
e.g. GC, HPLC, UV Spectroscopy, TGA /DTA, FTIR, MS, GCMS, NMR,
\end{flushleft}


\begin{flushleft}
TOC, CHONS. Principle of measurement techniques, instruments and
\end{flushleft}


\begin{flushleft}
procedures. Calibration, data processing, analysis and interpretation.
\end{flushleft}


\begin{flushleft}
Few working demonstrations.
\end{flushleft}


\begin{flushleft}
Catalysis characterization: Introduction to various catalysis
\end{flushleft}


\begin{flushleft}
preparations and characterization techniques, e.g. porosity, surface
\end{flushleft}


\begin{flushleft}
area, pore volume and pore size distribution (using BET), XRD, SEM,
\end{flushleft}


\begin{flushleft}
TEM, NMR, AFM, ESCA. Mossabauer spectroscopy, chemisorption,
\end{flushleft}


\begin{flushleft}
TPD/TPR.
\end{flushleft}


\begin{flushleft}
Flow characterization: Introduction to single/multiphase flows/
\end{flushleft}


\begin{flushleft}
reactors, role of hydrodynamics. Process parameters of interest, length
\end{flushleft}


\begin{flushleft}
and time scales, instantaneous vs. time averaged characteristics.
\end{flushleft}


\begin{flushleft}
Introduction to various advanced intrusive and non-intrusive flow
\end{flushleft}


\begin{flushleft}
measurement technqiues, e.g. mininaturized pressure probes,
\end{flushleft}


\begin{flushleft}
gamma-ray tomography, densitometry, PIV, RPT, ECT/ERT, high speed
\end{flushleft}


\begin{flushleft}
photography, tracers and radiotracers.
\end{flushleft}





\begin{flushleft}
CLL742 Experimental Characterization of
\end{flushleft}


\begin{flushleft}
BioMacromolecules
\end{flushleft}


\begin{flushleft}
3 Credits (3-0-0)
\end{flushleft}


\begin{flushleft}
Pre-requisites: CLL141, CLL271
\end{flushleft}


\begin{flushleft}
Overlaps with: PTL705
\end{flushleft}


\begin{flushleft}
Theory and working principles of analytical instruments including high
\end{flushleft}


\begin{flushleft}
performance liquid chromatography (HPLC), ultra-high performance
\end{flushleft}


\begin{flushleft}
liquid chromatography (UPLC), capillary electrophoresis (CE), capillary
\end{flushleft}


\begin{flushleft}
isoelectric focusing (cIEF), gel electrophoresis, circular dichroism
\end{flushleft}


\begin{flushleft}
(CD) spectroscopy, Fourier transform infrared spectroscopy (FTIR),
\end{flushleft}


\begin{flushleft}
mass spectroscopy (MS), atomic force microscopy (AFM), scanning
\end{flushleft}


\begin{flushleft}
electron microscope (SEM), differential scanning calorimetry (DSC),
\end{flushleft}


\begin{flushleft}
ultraviolet (UV) spectroscopy, surface plasmon resonance (SPR), 2D gel
\end{flushleft}


\begin{flushleft}
electrophoresis, fluorescence spectroscopy, Zeta-meter, contact angle
\end{flushleft}


\begin{flushleft}
goniometer, oscillatory drop module (ODM) of goniometer, and quartz
\end{flushleft}


\begin{flushleft}
crystal microbalance (QCM). Hands-on experience on characterization
\end{flushleft}


\begin{flushleft}
of proteins. Case studies in biotech industry.
\end{flushleft}





\begin{flushleft}
CLL743 Petrochemicals Technology
\end{flushleft}


\begin{flushleft}
3 Credits (3-0-0)
\end{flushleft}


\begin{flushleft}
Pre-requisites: CLL222
\end{flushleft}


\begin{flushleft}
Composition of petroleum: laboratory tests, refinery products,
\end{flushleft}


\begin{flushleft}
characterization of crude oil. Review of petrochemicals sector
\end{flushleft}


\begin{flushleft}
and Indian petrochemical industries in particular. Feed stocks
\end{flushleft}


\begin{flushleft}
for petrochemical industries and their sources. Overview of
\end{flushleft}


\begin{flushleft}
refining processes: catalytic cracking, reforming, delayed coking,
\end{flushleft}


\begin{flushleft}
Hydrogenation and Hydrocracking, Isomerization, Alkylation and
\end{flushleft}


\begin{flushleft}
polymerization, purification of gases, separation of aromatics by
\end{flushleft}


\begin{flushleft}
various techniques. Petrochemicals from methane, ethane, ethylene,
\end{flushleft}


\begin{flushleft}
acetylene, C3/C4 and higher hydrocarbons. Synthesis gas chemicals.
\end{flushleft}


\begin{flushleft}
Polymers from Olefins. Synthetic fibers, rubber, plastics and synthetic
\end{flushleft}


\begin{flushleft}
detergents. Energy conservation in petrochemical Industries. Pollution
\end{flushleft}


\begin{flushleft}
control in petrochemical industries. New trends in petrochemical
\end{flushleft}


\begin{flushleft}
industry. Planning and commissioning of a petrochemicals complex.
\end{flushleft}





\begin{flushleft}
CLL761 Chemical Engineering Mathematics
\end{flushleft}


\begin{flushleft}
3 Credits (3-0-0)
\end{flushleft}


\begin{flushleft}
Pre-requisites: MTL101, CLL110
\end{flushleft}


\begin{flushleft}
Classification, estimation and propagation of errors. Presentation of
\end{flushleft}


\begin{flushleft}
data. Statistical methods: sample and population distributions, testing
\end{flushleft}


\begin{flushleft}
of hypothesis, analysis of variance.
\end{flushleft}


\begin{flushleft}
Vector spaces, basis, matrices and differential operators. Eigen values,
\end{flushleft}


\begin{flushleft}
vectors and functions. Solvability conditions for linear equations.
\end{flushleft}


\begin{flushleft}
Frobenius method for ordinary differential equations. Sturm-Louiville
\end{flushleft}


\begin{flushleft}
Theorem: Separation of variables and Fourier transform. Green's
\end{flushleft}


\begin{flushleft}
function and its applications.
\end{flushleft}





158





\begin{flushleft}
\newpage
Chemical Engineering
\end{flushleft}





\begin{flushleft}
CLL762 Advanced Computational Techniques in
\end{flushleft}


\begin{flushleft}
Chemical Engineering
\end{flushleft}


\begin{flushleft}
3 Credits (2-0-2)
\end{flushleft}


\begin{flushleft}
Pre-requisites: CLL113
\end{flushleft}


\begin{flushleft}
Overlaps with: APL703
\end{flushleft}


\begin{flushleft}
Introduction to models in Chemical Engineering. Formulation of
\end{flushleft}


\begin{flushleft}
problems leading to ODEs of initial value types. Stability and stiffness of
\end{flushleft}


\begin{flushleft}
matrices. Solution of stiff problems like Rober's problem in autocatalytic
\end{flushleft}


\begin{flushleft}
reactions by Gear's algorithm. Formulation of problems leading to
\end{flushleft}


\begin{flushleft}
steady state ODEs of boundary value types. Different weighted residual
\end{flushleft}


\begin{flushleft}
methods to solve BVPs. Orthogonal collocation and Galerkin finite
\end{flushleft}


\begin{flushleft}
element method. Application to reaction diffusion in porous catalysts
\end{flushleft}


\begin{flushleft}
pellets under non-isothermal conditions, calculation of effectiveness
\end{flushleft}


\begin{flushleft}
factor. Moving boundary problems. Transient problems leading to PDEs.
\end{flushleft}


\begin{flushleft}
Examples in heat and mass transfer and their numerical solution:
\end{flushleft}


\begin{flushleft}
orthogonal collocation. Monte Carlo method and its applications.
\end{flushleft}


\begin{flushleft}
Introduction to LBM method to solve fluid flow problems.
\end{flushleft}





\begin{flushleft}
CLL766 Interfacial Engineering
\end{flushleft}


\begin{flushleft}
3 Credits (3-0-0)
\end{flushleft}


\begin{flushleft}
Pre-requisites: CLL110, CLL121
\end{flushleft}


\begin{flushleft}
Overlaps with: CML103
\end{flushleft}


\begin{flushleft}
Concept and definition of interface. Physical surfaces. Surface
\end{flushleft}


\begin{flushleft}
chemistry and physics of colloids, thin films, dispersions, emulsions,
\end{flushleft}


\begin{flushleft}
foams, polyaphrons. Interfacial processes such as crystallization,
\end{flushleft}


\begin{flushleft}
epitaxy, froth flotation, adsorption, adsorptive bubble separation,
\end{flushleft}


\begin{flushleft}
catalysis, reaction-injection moulding, microencapsulation. Industrial
\end{flushleft}


\begin{flushleft}
aspects of interfacial engineering.
\end{flushleft}





\begin{flushleft}
CLL767 Structures and Properties of Polymers
\end{flushleft}


\begin{flushleft}
3 Credits (3-0-0)
\end{flushleft}


\begin{flushleft}
Pre-requisites: CLL141
\end{flushleft}


\begin{flushleft}
Overlaps with: PTL703, PTL701, TTL712
\end{flushleft}





\begin{flushleft}
Simulation (LES) and Direct Numerical Simulations (DNS). Numerical
\end{flushleft}


\begin{flushleft}
simulations of multiphase flows. Two-fluid and Multi-fluid Eulermethods; Discrete particle (Euler-Lagrange) methods; Interface
\end{flushleft}


\begin{flushleft}
tracking/capturing methods (Volumne of Fluid Methods) (Volume of
\end{flushleft}


\begin{flushleft}
Fluid Method). Applications of these Methods to simulate dispersed
\end{flushleft}


\begin{flushleft}
gas-liquid, gas-solid flows in bubble columns, fluidized beds, packed
\end{flushleft}


\begin{flushleft}
beds, etc. Numerical simulations of fluid flows with heat transfer.
\end{flushleft}


\begin{flushleft}
Numerical simulations of reactive flows.
\end{flushleft}





\begin{flushleft}
CLD771 Minor Project
\end{flushleft}


\begin{flushleft}
3 Credits (0-0-6)
\end{flushleft}


\begin{flushleft}
Literature survey, Writing technical report, Planning and execution of
\end{flushleft}


\begin{flushleft}
the project work within the stipulated time frame.
\end{flushleft}





\begin{flushleft}
CLL771 Introduction to Complex Fluids
\end{flushleft}


\begin{flushleft}
3 Credits (3-0-0)
\end{flushleft}


\begin{flushleft}
Overview of complex fluids: forces, energies, responses and timescales
\end{flushleft}


\begin{flushleft}
in complex fluids. Types of complex fluids: colloidal dispersions,
\end{flushleft}


\begin{flushleft}
polymers, gels, liquid crystals, polymer crystals, granular materials,
\end{flushleft}


\begin{flushleft}
biomolecules. Characterization of structure-property relationships in
\end{flushleft}


\begin{flushleft}
complex fluids.
\end{flushleft}





\begin{flushleft}
CLL772 Transport Phenomena in Complex Fluids
\end{flushleft}


\begin{flushleft}
3 Credits (3-0-0)
\end{flushleft}


\begin{flushleft}
Pre-requisites: CLL110
\end{flushleft}


\begin{flushleft}
Classification of fluids under time dependent, time independent
\end{flushleft}


\begin{flushleft}
and viscoelastic behaviors. Particle level responses: microstructural
\end{flushleft}


\begin{flushleft}
origins of deformation. Linear and non-linear viscoelasticity. Transport
\end{flushleft}


\begin{flushleft}
processes in a variety of self-assembling fluids, including surfactant
\end{flushleft}


\begin{flushleft}
micelles, nano-emulsions, gels, liquid crystalline polymers. Dynamics
\end{flushleft}


\begin{flushleft}
of rod-like polymers. Static and viscoelastic properties of interfaces.
\end{flushleft}


\begin{flushleft}
Rheometry and constitutive modeling. Heat transfer in complex fluids:
\end{flushleft}


\begin{flushleft}
boundary layers. Mixing equipment and its selection.
\end{flushleft}





\begin{flushleft}
Overview of polymer science and engineering with reference to
\end{flushleft}


\begin{flushleft}
polymer-solution. Chain dimension: variation of chain dimension
\end{flushleft}


\begin{flushleft}
with concentration, solvency etc., scaling theory. Molecular weight
\end{flushleft}


\begin{flushleft}
distribution and its effect on properties of polymer solution. Polymer
\end{flushleft}


\begin{flushleft}
solution thermodynamics: Flory-Huggins equation and its development,
\end{flushleft}


\begin{flushleft}
phase separation. Polymer in good, theta and poor solution. Colligative
\end{flushleft}


\begin{flushleft}
properties of polymer solution. Flow phenomena in polymeric liquids.
\end{flushleft}


\begin{flushleft}
Material functions for polymeric liquids. General linear viscoelastic
\end{flushleft}


\begin{flushleft}
fluid: Rouse dynamics, Zimm dynamics. Hyper branched polymer and
\end{flushleft}


\begin{flushleft}
its physical properties in various solutions. Dynamics of entangled
\end{flushleft}


\begin{flushleft}
polymers - polymer melt, chain reptation, tube model, chain length
\end{flushleft}


\begin{flushleft}
fluctuations. Convective constraint release.
\end{flushleft}





\begin{flushleft}
CLL773 Thermodynamics of Complex Fluids
\end{flushleft}


\begin{flushleft}
3 Credits (3-0-0)
\end{flushleft}


\begin{flushleft}
Pre-requisites: CLL121
\end{flushleft}


\begin{flushleft}
Overlaps with: PYL202
\end{flushleft}





\begin{flushleft}
CLL768 Fundamentals of Computational Fluid Dynamics
\end{flushleft}


\begin{flushleft}
3 Credits (2-0-2)
\end{flushleft}


\begin{flushleft}
Pre-requisites: CLL113,CLL110
\end{flushleft}


\begin{flushleft}
Overlaps with: AML410, MEL807
\end{flushleft}





\begin{flushleft}
CLL774 Simulation Techniques for Complex Fluids
\end{flushleft}


\begin{flushleft}
3 Credits (3-0-0)
\end{flushleft}


\begin{flushleft}
Pre-requisites: CLL113
\end{flushleft}


\begin{flushleft}
Overlaps with: MCL315
\end{flushleft}





\begin{flushleft}
Review of basic fluid mechanics and the governing (Navier-Stokes)
\end{flushleft}


\begin{flushleft}
equations. Techniques for solution of PDEs -- finite difference method,
\end{flushleft}


\begin{flushleft}
finite element method and finite volume method. Finite volume (FV)
\end{flushleft}


\begin{flushleft}
method in one-dimension. Differencing schemes. Steady and unsteady
\end{flushleft}


\begin{flushleft}
calculations. Boundary conditions. FV discretization in two and three
\end{flushleft}


\begin{flushleft}
dimensions. SIMPLE algorithm and flow field calculations, variants
\end{flushleft}


\begin{flushleft}
of SIMPLE. Turbulence and turbulence modeling: illustrative flow
\end{flushleft}


\begin{flushleft}
computations. Commercial software - grid generation, flow prediction
\end{flushleft}


\begin{flushleft}
and post-processing.
\end{flushleft}





\begin{flushleft}
Simulation techniques: Molecular Dynamics, Brownian Dynamics,
\end{flushleft}


\begin{flushleft}
Monte-Carlo, Discrete Element Method and Lattice Boltzmann
\end{flushleft}


\begin{flushleft}
Simulations. Force fields and interactions. Statistical measures and
\end{flushleft}


\begin{flushleft}
trajectory analysis to determine structure (e.g., radial distribution
\end{flushleft}


\begin{flushleft}
function) and properties (e.g., self-diffusivity, shear-dependent
\end{flushleft}


\begin{flushleft}
viscosity) of complex fluids.
\end{flushleft}





\begin{flushleft}
CLL769 Applications of Computational Fluid Dynamics
\end{flushleft}


\begin{flushleft}
3 Credits (2-0-2)
\end{flushleft}


\begin{flushleft}
Pre-requisites: CLL110
\end{flushleft}


\begin{flushleft}
Overlaps with: APL410, APL720, MCL813
\end{flushleft}


\begin{flushleft}
Introduction and review of fundamentals of CFD. Numerical simulations
\end{flushleft}


\begin{flushleft}
of turbulent flows: RANS approach; Introduction to Large Eddy
\end{flushleft}





\begin{flushleft}
Intermolecular forces. Statistical mechanical approach to
\end{flushleft}


\begin{flushleft}
thermodynamic potentials. Characterization of free energy curves.
\end{flushleft}


\begin{flushleft}
Entropically driven phase separation, nucleation and spontaneous
\end{flushleft}


\begin{flushleft}
phase separations in complex fluids. Characterization of structures:
\end{flushleft}


\begin{flushleft}
Minkowski functionals. Phase separation in confinement. Mean field
\end{flushleft}


\begin{flushleft}
theories for phase transition, their break-down, introduction to field
\end{flushleft}


\begin{flushleft}
theory. Thermodynamics of colloidal systems and polymers.
\end{flushleft}





\begin{flushleft}
CLL775 Polymerization Process Modeling
\end{flushleft}


\begin{flushleft}
3 Credits (3-0-0)
\end{flushleft}


\begin{flushleft}
Pre-requisites: CLL122
\end{flushleft}


\begin{flushleft}
Overlaps with: PTL701
\end{flushleft}


\begin{flushleft}
Modeling of step-growth, chain-growth and non-linear polymerization
\end{flushleft}


\begin{flushleft}
in homogeneous and heterogeneous conditions. Design of CSTR,
\end{flushleft}


\begin{flushleft}
plug flow, batch and multistep reactors for different polymerization
\end{flushleft}


\begin{flushleft}
reactions. Control and optimization of polymer reactors, Mathematical
\end{flushleft}


\begin{flushleft}
modeling and analysis of polymer processing units.
\end{flushleft}





159





\begin{flushleft}
\newpage
Chemical Engineering
\end{flushleft}





\begin{flushleft}
CLL776 Granular Materials
\end{flushleft}


\begin{flushleft}
3 Credits (3-0-0)
\end{flushleft}


\begin{flushleft}
Pre-requisites: CLL331
\end{flushleft}





\begin{flushleft}
CLD781 Major Project Part-I
\end{flushleft}


\begin{flushleft}
8 Credits (0-0-16)
\end{flushleft}





\begin{flushleft}
Continuum mechanics, statistical physics and rigid body dynamics
\end{flushleft}


\begin{flushleft}
approaches to understand microscopic and macroscopic behavior of
\end{flushleft}


\begin{flushleft}
granular materials. Constitutive modeling and rheology of granular
\end{flushleft}


\begin{flushleft}
materials. Advanced simulation techniques for particle dynamics.
\end{flushleft}


\begin{flushleft}
Design of flow and handling systems for granular materials.
\end{flushleft}





\begin{flushleft}
CLL777 Complex Fluids Technology
\end{flushleft}


\begin{flushleft}
3 Credits (3-0-0)
\end{flushleft}


\begin{flushleft}
Pre-requisites: CLL141
\end{flushleft}


\begin{flushleft}
An overview of various technologies based on complex fluids and relate
\end{flushleft}


\begin{flushleft}
them to fundamental principles of thermodynamics and transport
\end{flushleft}


\begin{flushleft}
phenomena in complex fluids, e.g., how to manipulate micro-structures
\end{flushleft}


\begin{flushleft}
and their environment to achieve new products with desired properties.
\end{flushleft}


\begin{flushleft}
Case studies involving assembly, stability and applications of colloids,
\end{flushleft}


\begin{flushleft}
emulsions, suspensions, polymer melts and granular materials.
\end{flushleft}





\begin{flushleft}
CLL778 Interfacial Behaviour and Transport of
\end{flushleft}


\begin{flushleft}
Biomolecules
\end{flushleft}


\begin{flushleft}
3 Credits (3-0-0)
\end{flushleft}


\begin{flushleft}
Pre-requisites: CLL110
\end{flushleft}


\begin{flushleft}
Overlaps with: CYL669, SBL705, SBV885
\end{flushleft}


\begin{flushleft}
Structure of biomacromolecules. Attributes of biomacromolecules:
\end{flushleft}


\begin{flushleft}
size, charge, hydrophobicity. Characteristics of surface and
\end{flushleft}


\begin{flushleft}
interfaces: roughness, charge, hydrophobicity. Interactions between
\end{flushleft}


\begin{flushleft}
biomacromolecules and interfaces: adsorption, specific binding.
\end{flushleft}


\begin{flushleft}
Aggregation of proteins, modeling of the underlying phenomena.
\end{flushleft}


\begin{flushleft}
Elasticity of adsorbed macro-molecules at interfaces. Equilibrium and
\end{flushleft}


\begin{flushleft}
transient description of transport of biomolecules through intra- and
\end{flushleft}


\begin{flushleft}
extracellular space. Governing equations applied to biological systems:
\end{flushleft}


\begin{flushleft}
conservation laws, flux equations, Fickian and non-Fickian diffusion,
\end{flushleft}


\begin{flushleft}
diffusion with reaction/ binding, electrochemical transport. Constitutive
\end{flushleft}


\begin{flushleft}
laws and solution methods applied to biological systems. Adsorption
\end{flushleft}


\begin{flushleft}
isotherms and transport across membrane.
\end{flushleft}





\begin{flushleft}
CLL779 Molecular Biotechnology and in-vitro Diagnostics
\end{flushleft}


\begin{flushleft}
3 Credits (3-0-0)
\end{flushleft}


\begin{flushleft}
Introduction to the cellular structure and function of biomolecules,
\end{flushleft}


\begin{flushleft}
theory and experimental characterization of commonly-used laboratory
\end{flushleft}


\begin{flushleft}
techniques in molecular diagnostic protocols. Identification of
\end{flushleft}


\begin{flushleft}
the important parameters such as sensitivity, specificity, LOD etc.
\end{flushleft}


\begin{flushleft}
in the design of a quality system for molecular analyses. Highly
\end{flushleft}


\begin{flushleft}
sensitive reporter technologies and applications, technologies
\end{flushleft}


\begin{flushleft}
providing highly dense and bioactive solid phases, novel bioaffinity
\end{flushleft}


\begin{flushleft}
binders, heterogeneous and homogenous assay concepts, and
\end{flushleft}


\begin{flushleft}
multiplexed bioassays.
\end{flushleft}





\begin{flushleft}
Literature survey, Writing technical report, Planning and execution of
\end{flushleft}


\begin{flushleft}
the project work within the stipulated time frame.
\end{flushleft}





\begin{flushleft}
CLL781 Process Operations Scheduling
\end{flushleft}


\begin{flushleft}
3 Credits (3-0-0)
\end{flushleft}


\begin{flushleft}
Pre-requisites: minimum earned credit of 90 for UG (B.Tech./DD)
\end{flushleft}


\begin{flushleft}
Introduction to enterprise-wide supply-chain optimization. Decision
\end{flushleft}


\begin{flushleft}
making for planning and scheduling. Classification of scheduling
\end{flushleft}


\begin{flushleft}
formulations: various storage policies, network representations, time
\end{flushleft}


\begin{flushleft}
representations. Short-term scheduling of batch processes: discretetime and continuous-time based models. Cyclic and short-term
\end{flushleft}


\begin{flushleft}
scheduling of continuous processes. Solution of resulting models with
\end{flushleft}


\begin{flushleft}
industrial applications using GAMS modeling language.
\end{flushleft}





\begin{flushleft}
CLD782 Major Project Part-II
\end{flushleft}


\begin{flushleft}
12 Credits (0-0-24)
\end{flushleft}


\begin{flushleft}
Literature survey, Writing technical report, Planning and execution
\end{flushleft}


\begin{flushleft}
of the project work within the stipulated time frame. Analysis and
\end{flushleft}


\begin{flushleft}
interpretation of the obtained data.
\end{flushleft}





\begin{flushleft}
CLL782 Process Optimization
\end{flushleft}


\begin{flushleft}
3 Credits (3-0-0)
\end{flushleft}


\begin{flushleft}
Pre-requisites: CLL222, CLL352
\end{flushleft}


\begin{flushleft}
Overlaps with: MTL103, MTL625, MTL704, APL771, MCL742
\end{flushleft}


\begin{flushleft}
Introduction to optimization and applications; classification (LP, NLP,
\end{flushleft}


\begin{flushleft}
MILP, MINLP), convexity, unimodal vs multimodal. Single variable
\end{flushleft}


\begin{flushleft}
and multivariable unconstrained optimization methods. Linear
\end{flushleft}


\begin{flushleft}
programming, branch and bound method for MILP. Constrained
\end{flushleft}


\begin{flushleft}
optimization: nonlinear programming. Necessary and sufficient
\end{flushleft}


\begin{flushleft}
conditions of optimality. Quadratic programming. Case studies from
\end{flushleft}


\begin{flushleft}
chemical industry.
\end{flushleft}





\begin{flushleft}
CLL783 Advanced Process Control
\end{flushleft}


\begin{flushleft}
3 Credits (3-0-0)
\end{flushleft}


\begin{flushleft}
Pre-requisites: CLL261
\end{flushleft}


\begin{flushleft}
Overlaps with: ELL325, ELL723, BBL444
\end{flushleft}


\begin{flushleft}
State-space models. Distributed parameter models. Feedforward
\end{flushleft}


\begin{flushleft}
control. Ratio control. Dead-time compensation. Relative gain array.
\end{flushleft}


\begin{flushleft}
Z-transforms and digital control. Internal model control. State
\end{flushleft}


\begin{flushleft}
estimation and process identification. Adaptive control. Non-linear
\end{flushleft}


\begin{flushleft}
control. Model-based control structures. Synthesis of control systems
\end{flushleft}


\begin{flushleft}
with case studies. Intelligent control, model predictive control.
\end{flushleft}





\begin{flushleft}
CLL784 Process Modeling and Simulation
\end{flushleft}


\begin{flushleft}
3 Credits (3-0-0)
\end{flushleft}


\begin{flushleft}
Pre-requisites: CLL222, CLL352
\end{flushleft}


\begin{flushleft}
Introduction to modeling, physical and mathematical models, modeling
\end{flushleft}


\begin{flushleft}
individual units vs. process. Role of simulation and simulators.
\end{flushleft}


\begin{flushleft}
Sequential and modular approaches to flowsheet simulation: equation
\end{flushleft}


\begin{flushleft}
solving approach. Decomposition of networks: tearing algorithms,
\end{flushleft}


\begin{flushleft}
convergence promotion.
\end{flushleft}





\begin{flushleft}
CLL780 Bioprocessing and Bioseparations
\end{flushleft}


\begin{flushleft}
3 Credits (3-0-0)
\end{flushleft}


\begin{flushleft}
Pre-requisites: CLL271
\end{flushleft}


\begin{flushleft}
Overlaps with: BEL703, BEL820
\end{flushleft}


\begin{flushleft}
Introduction to the different unit operations utilized in production
\end{flushleft}


\begin{flushleft}
of biotech drugs in the areas of upstream processing, harvest,
\end{flushleft}


\begin{flushleft}
and downstream processing. Introduction to analytical methods
\end{flushleft}


\begin{flushleft}
used for characterization of biotech products and processes (high
\end{flushleft}


\begin{flushleft}
performance liquid chromatography, mass spectrophotometry,
\end{flushleft}


\begin{flushleft}
capillary electrophoresis, near infrared spectroscopy,UV spectroscopy).
\end{flushleft}


\begin{flushleft}
Optimization of biotech processes - unit operation specific optimization
\end{flushleft}


\begin{flushleft}
vs. process optimization, process intensification, statistical data
\end{flushleft}


\begin{flushleft}
analysis. Scale-up of different unit operations utilized in bioprocessing:
\end{flushleft}


\begin{flushleft}
procedures, issues that frequently occur and possible solutions.
\end{flushleft}


\begin{flushleft}
Good Manufacturing Practices (GMP): need, principles and key practical
\end{flushleft}


\begin{flushleft}
issues. Process validation: basics, planning and implementation.
\end{flushleft}


\begin{flushleft}
Industrial case studies in bioprocessing. Current topics in bioprocessing
\end{flushleft}


\begin{flushleft}
and bioseparations: Quality by Design and Process Analytical Technology.
\end{flushleft}





\begin{flushleft}
Specific purpose simulation. Dynamic simulation. Case studies using
\end{flushleft}


\begin{flushleft}
commercial or open source simulation packages.
\end{flushleft}





\begin{flushleft}
CLL785 Evolutionary Optimization
\end{flushleft}


\begin{flushleft}
3 Credits (3-0-0)
\end{flushleft}


\begin{flushleft}
Pre-requisites: CLL222
\end{flushleft}


\begin{flushleft}
Overlaps with: MTL720
\end{flushleft}


\begin{flushleft}
Traditional vs. nontraditional optimization techniques. Population
\end{flushleft}


\begin{flushleft}
based search algorithms. Evolutionary strategies. Simulated annealing.
\end{flushleft}


\begin{flushleft}
Genetic algorithms. Differential evolution. Different strategies of
\end{flushleft}


\begin{flushleft}
differential evolution. Memetic algorithms. Scatter, Tabu search.
\end{flushleft}


\begin{flushleft}
Ant-colony optimization. Particle swarm optimization. Self-organizing
\end{flushleft}


\begin{flushleft}
migrating algorithm. Neural networks. Quantum computing. DNA
\end{flushleft}


\begin{flushleft}
computing. Multi-objective optimization. Industrial applications.
\end{flushleft}





160





\begin{flushleft}
\newpage
Chemical Engineering
\end{flushleft}





\begin{flushleft}
CLL786 Fine Chemicals Technology
\end{flushleft}


\begin{flushleft}
3 Credits (3-0-0)
\end{flushleft}


\begin{flushleft}
Pre-requisites: CLL222
\end{flushleft}





\begin{flushleft}
CLL794 Petroleum Refinery Engineering
\end{flushleft}


\begin{flushleft}
3 Credits (3-0-0)
\end{flushleft}


\begin{flushleft}
Pre-requisites: CLL222, CLL352
\end{flushleft}





\begin{flushleft}
Introduction to fine and high value chemicals. Historical perspectives.
\end{flushleft}


\begin{flushleft}
Synthesis methods from chemical (petrochemicals and natural
\end{flushleft}


\begin{flushleft}
products) and biotechnology routes (enzymatic methods, fermentation
\end{flushleft}


\begin{flushleft}
and cell culture technology). Extraction of fine chemicals from
\end{flushleft}


\begin{flushleft}
microorganisms, plant sources and animal sources. Chromatographic
\end{flushleft}


\begin{flushleft}
separations. Reactor technology for fine chemicals. Scale-up and scaleout of reactors. Microreactor technology and process intensification.
\end{flushleft}


\begin{flushleft}
Novel high value chemicals for adhesives, electronic materials, food
\end{flushleft}


\begin{flushleft}
additives, specialty polymers, flavours and fragrances.
\end{flushleft}





\begin{flushleft}
Composition of petroleum, laboratory tests, refinery products,
\end{flushleft}


\begin{flushleft}
characterization of crude oil. Design of crude oil distillation column.
\end{flushleft}


\begin{flushleft}
Catalytic cracking, catalytic reforming, delayed coking, furnace
\end{flushleft}


\begin{flushleft}
design, hydrogenation and hydrocracking, isomerization, alkylation
\end{flushleft}


\begin{flushleft}
and polymerization. Lube oil manufacturing. Energy conservation in
\end{flushleft}


\begin{flushleft}
petroleum refineries. New trends in petroleum refinery operations.
\end{flushleft}


\begin{flushleft}
Pyrolysis of naphtha and light hydrocarbons.
\end{flushleft}





\begin{flushleft}
CLL791 Chemical Product and Process Integration
\end{flushleft}


\begin{flushleft}
3 Credits (3-0-0)
\end{flushleft}


\begin{flushleft}
Pre-requisites: CLL371
\end{flushleft}





\begin{flushleft}
CLV796 Current Topics in Chemical Engineering
\end{flushleft}


\begin{flushleft}
1 Credit (1-0-0)
\end{flushleft}


\begin{flushleft}
As per declaration of instructor(s).
\end{flushleft}





\begin{flushleft}
The course will be a structured project based course with initial
\end{flushleft}


\begin{flushleft}
exposure to industrial processes of understanding Voice of Customers,
\end{flushleft}


\begin{flushleft}
identifying design specifications, scoping the technology and product
\end{flushleft}


\begin{flushleft}
landscape and deciding on the technology strategy. Technical and
\end{flushleft}


\begin{flushleft}
economic feasibility analysis as well as scale-up and manufacturing
\end{flushleft}


\begin{flushleft}
concerns will also be discussed. Each group will identify a specific
\end{flushleft}


\begin{flushleft}
product or process of interest and work through these considerations
\end{flushleft}


\begin{flushleft}
as well as integrate thermodynamics, transport principles, fluid
\end{flushleft}


\begin{flushleft}
mechanics and reactor design understanding to design the product
\end{flushleft}


\begin{flushleft}
or process chosen.
\end{flushleft}





\begin{flushleft}
CLV797 Recent Advances in Chemical Engineering
\end{flushleft}


\begin{flushleft}
2 Credits (2-0-0)
\end{flushleft}


\begin{flushleft}
Pre-requisites: To be declared by Instructor
\end{flushleft}


\begin{flushleft}
As per declaration of instructor(s).
\end{flushleft}





\begin{flushleft}
CLL798 Selected Topics in Chemical Engineering-I
\end{flushleft}


\begin{flushleft}
3 Credits (3-0-0)
\end{flushleft}


\begin{flushleft}
Pre-requisites: To be declared by Instructor
\end{flushleft}


\begin{flushleft}
As per declaration of instructor(s).
\end{flushleft}





\begin{flushleft}
CLL792 Chemical Product Development and
\end{flushleft}


\begin{flushleft}
Commercialization
\end{flushleft}


\begin{flushleft}
3 Credits (3-0-0)
\end{flushleft}


\begin{flushleft}
Pre-requisites: CLL110
\end{flushleft}


\begin{flushleft}
Overlaps with: MTL766, MAL719, SML802
\end{flushleft}





\begin{flushleft}
CLL799 Selected Topics in Chemical Engineering-II
\end{flushleft}


\begin{flushleft}
3 Credits (3-0-0)
\end{flushleft}


\begin{flushleft}
Pre-requisites: To be declared by Instructor
\end{flushleft}





\begin{flushleft}
Design of experiments - factors, responses, main effects, interactions,
\end{flushleft}


\begin{flushleft}
different kinds of designs - screening vs. high resolution. Statistical
\end{flushleft}


\begin{flushleft}
data analysis - applied probability, sampling, estimation, hypothesis
\end{flushleft}


\begin{flushleft}
testing, linear regression, analysis of variance, types of data plots.
\end{flushleft}


\begin{flushleft}
Technology transfer of processes - need of technology transfer, key
\end{flushleft}


\begin{flushleft}
attributes, key challenges, solutions to various issues. Intellectual
\end{flushleft}


\begin{flushleft}
property management - intellectual property rights, IPR laws,
\end{flushleft}


\begin{flushleft}
patents, trademarks, designs, copyrights, licensing, IP management.
\end{flushleft}


\begin{flushleft}
Commercialization of technologies - invention, product development,
\end{flushleft}


\begin{flushleft}
technical and market feasibility analysis, intellectual property acquisition.
\end{flushleft}





\begin{flushleft}
CLD880 Minor Project
\end{flushleft}


\begin{flushleft}
4 Credits (0-0-8)
\end{flushleft}





\begin{flushleft}
CLL793 Membrane Science and Engineering
\end{flushleft}


\begin{flushleft}
3 Credits (3-0-0)
\end{flushleft}


\begin{flushleft}
Pre-requisites: CLL110, CLL252
\end{flushleft}





\begin{flushleft}
As per declaration of instructor(s).
\end{flushleft}





\begin{flushleft}
CLD871 Major Project Part-I
\end{flushleft}


\begin{flushleft}
6 Credits (0-0-12)
\end{flushleft}


\begin{flushleft}
CLD872 Major Project Part-II
\end{flushleft}


\begin{flushleft}
14 Credits (0-0-28)
\end{flushleft}


\begin{flushleft}
CLD881 Major Project Part-I
\end{flushleft}


\begin{flushleft}
8 Credits (0-0-16)
\end{flushleft}





\begin{flushleft}
Introduction to membrane separation processes, their classification,
\end{flushleft}


\begin{flushleft}
and applications. General transport theories including theory of
\end{flushleft}


\begin{flushleft}
irreversible thermodynamics for multicomponent systems. Membrane
\end{flushleft}


\begin{flushleft}
preparation techniques. Design and analysis and industrial application
\end{flushleft}


\begin{flushleft}
of various membrane processes such as reverse osmosis, ultra
\end{flushleft}


\begin{flushleft}
filtration, electrodialysis, dialysis, liquid membrane separation, gas
\end{flushleft}


\begin{flushleft}
permeation and pervaporation.
\end{flushleft}





\begin{flushleft}
CLD882 Major Project Part-II
\end{flushleft}


\begin{flushleft}
12 Credits (0-0-24)
\end{flushleft}


\begin{flushleft}
CLD895 MS Research Project
\end{flushleft}


\begin{flushleft}
36 Credits (0-0-72)
\end{flushleft}





161





\begin{flushleft}
\newpage
Department of Chemistry
\end{flushleft}


\begin{flushleft}
CML100 Introduction to Chemistry
\end{flushleft}


\begin{flushleft}
3 Credits (3-0-0)
\end{flushleft}





\begin{flushleft}
CML513 Photochemistry \& Pericyclic Reactions
\end{flushleft}


\begin{flushleft}
3 Credits (3-0-0)
\end{flushleft}





\begin{flushleft}
Entropy and free energy changes in chemical processes, chemical
\end{flushleft}


\begin{flushleft}
equilibria, phase transformations, structure and dynamics of
\end{flushleft}


\begin{flushleft}
microscopic systems, physical basis of atomic and molecular structure,
\end{flushleft}


\begin{flushleft}
three-dimensional arrangement of atoms in molecules, structure and
\end{flushleft}


\begin{flushleft}
reactivity of organic, inorganic and organometallic compounds, basic
\end{flushleft}


\begin{flushleft}
strategies for synthesis of carbon and silicon containing compounds,
\end{flushleft}


\begin{flushleft}
coordination chemistry, role of inorganic chemistry in living systems
\end{flushleft}





\begin{flushleft}
Pericyclic reaction, Introduction and classification, Theory of
\end{flushleft}


\begin{flushleft}
pericyclic reactions: correlation diagrams, FMO, and PMO methods,
\end{flushleft}


\begin{flushleft}
Cycloadditions reactions, Molecular rearrangements (pericyclic and
\end{flushleft}


\begin{flushleft}
non-pericyclic), Photochemistry: basics and mechanistic principles,
\end{flushleft}


\begin{flushleft}
Photochemical rearrangements, Reactivity of simple chromophores.
\end{flushleft}





\begin{flushleft}
CMP100 Chemistry Laboratory
\end{flushleft}


\begin{flushleft}
2 Credits (0-0-4)
\end{flushleft}





\begin{flushleft}
General properties of p block elements, bonding, historical landmarks,
\end{flushleft}


\begin{flushleft}
and periodic properties, Introduction to group theory, Chemistry of
\end{flushleft}


\begin{flushleft}
alkali and alkaline earth metals, Chemistry of group 13, 14, 15, and
\end{flushleft}


\begin{flushleft}
16 elements, Halogen chemistry, Chemistry of rare gases.
\end{flushleft}





\begin{flushleft}
Experiments involve the following: Titrations, Surface Tension and
\end{flushleft}


\begin{flushleft}
Viscosity, Potentiometery, Conductometry, Preparation of metal
\end{flushleft}


\begin{flushleft}
complexes and important organic compounds, Kinetics, Chromatography,
\end{flushleft}


\begin{flushleft}
Qualitative and quantitative estimation of organic compounds.
\end{flushleft}





\begin{flushleft}
CML102 Chemical Synthesis of Functional Materials
\end{flushleft}


\begin{flushleft}
3 Credits (3-0-0)
\end{flushleft}


\begin{flushleft}
Chemical approaches to the synthesis of functional materials -- the
\end{flushleft}


\begin{flushleft}
design of materials targeting important properties by {`}bottom-up'
\end{flushleft}


\begin{flushleft}
processes that manipulate primary chemical bonds.
\end{flushleft}


\begin{flushleft}
Fundamental chemistry principles involved in materials design through
\end{flushleft}


\begin{flushleft}
synthesis -- process methodologies such as self-assembly, sol-gel
\end{flushleft}


\begin{flushleft}
reactions, synthesis of nanomaterials, etc.
\end{flushleft}





\begin{flushleft}
CML103 Applied Chemistry - Chemistry at Interfaces
\end{flushleft}


\begin{flushleft}
3 Credits (3-0-0)
\end{flushleft}


\begin{flushleft}
Unit processses in organic synthesis. Laboratory vs. industrial
\end{flushleft}


\begin{flushleft}
synthesis. Role of medium in directing synthetic outcomes, organized
\end{flushleft}


\begin{flushleft}
media. Natural and synthetic constrained systems (inorganic and
\end{flushleft}


\begin{flushleft}
organic) for control of reactivity in organic reactions. Phase transfer
\end{flushleft}


\begin{flushleft}
catalysts, polymer and supported reagents for control of reactions.
\end{flushleft}


\begin{flushleft}
Green Chemistry. Heterogeneous and homogeneous catalysis, surface
\end{flushleft}


\begin{flushleft}
chemistry, kinetics of catalyzed reactions. Industrial catalysis.
\end{flushleft}





\begin{flushleft}
CML514 Main Group Chemistry
\end{flushleft}


\begin{flushleft}
3 Credits (3-0-0)
\end{flushleft}





\begin{flushleft}
CML515 Instrumental Methods of Analysis
\end{flushleft}


\begin{flushleft}
3 Credits (3-0-0)
\end{flushleft}


\begin{flushleft}
Measurement basics and data analysis, Introduction to spectrometric
\end{flushleft}


\begin{flushleft}
methods and components of optical instruments, Atomic absorption,
\end{flushleft}


\begin{flushleft}
fluorescence, emission, mass, and X-ray spectrometry, Introduction
\end{flushleft}


\begin{flushleft}
to and applications of uv-vis molecular absorption, luminescence,
\end{flushleft}


\begin{flushleft}
infrared, Raman, nuclear magnetic resonance, and mass spectroscopy/
\end{flushleft}


\begin{flushleft}
spectrometry, Introduction to electroanalytical methods: potentiometry,
\end{flushleft}


\begin{flushleft}
coulometry, and voltammetry, Introduction to chromatographic
\end{flushleft}


\begin{flushleft}
separation: gas, high-performance liquid, supercritical fluid, and
\end{flushleft}


\begin{flushleft}
capillary electrophoresis chromatography, Introduction to thermal
\end{flushleft}


\begin{flushleft}
methods of analysis.
\end{flushleft}





\begin{flushleft}
CML521 Molecular Thermodynamics
\end{flushleft}


\begin{flushleft}
3 Credits (3-0-0)
\end{flushleft}


\begin{flushleft}
Basics concepts, Review of first, second, and third laws of
\end{flushleft}


\begin{flushleft}
thermodynamics, Gibb's free energy, Extra work, Chemical potential,
\end{flushleft}


\begin{flushleft}
Ideal and non ideal solution, Phase rule, Phase diagram, Solutions,
\end{flushleft}


\begin{flushleft}
Chemical equilibrium, Postulates of statistical thermodynamics,
\end{flushleft}


\begin{flushleft}
Ensembles, Monoatomic and polyatomic ideal gases, Molar heat
\end{flushleft}


\begin{flushleft}
capacities, Classical statistical mechanics.
\end{flushleft}





\begin{flushleft}
CMP521 Laboratory-III
\end{flushleft}


\begin{flushleft}
2 Credits (0-0-4)
\end{flushleft}





\begin{flushleft}
CML511 Quantum Chemistry
\end{flushleft}


\begin{flushleft}
3 Credits (3-0-0)
\end{flushleft}


\begin{flushleft}
Basic concepts and postulates of quantum mechanics, Hydrogen
\end{flushleft}


\begin{flushleft}
atom, Quantization of angular momentum, Many electron atoms,
\end{flushleft}


\begin{flushleft}
Variation theorem, Perturbation theory, Molecular orbital and valence
\end{flushleft}


\begin{flushleft}
bond theories, Introductory treatment of semi-empirical and ab initio
\end{flushleft}


\begin{flushleft}
calculations on molecular systems, Density functional theory.
\end{flushleft}





\begin{flushleft}
CMP511 Laboratory-I
\end{flushleft}


\begin{flushleft}
2 Credits (0-0-4)
\end{flushleft}


\begin{flushleft}
Experiments highlighting the principles of thermodynamics and
\end{flushleft}


\begin{flushleft}
chemical equilibrium, electrochemistry, chemical kinetics, spectroscopy,
\end{flushleft}


\begin{flushleft}
and computer simulations. Examples include: Thermodynamics
\end{flushleft}


\begin{flushleft}
of micellization, Synthesis, stabilization, and spectroscopy of
\end{flushleft}


\begin{flushleft}
nanoparticles, Steady-state and time resolved fluorescence, Cyclic
\end{flushleft}


\begin{flushleft}
and linear sweep voltammetry, Electronic structure calculations, etc.
\end{flushleft}





\begin{flushleft}
CMP512 Laboratory-II
\end{flushleft}


\begin{flushleft}
2 Credits (0-0-4)
\end{flushleft}


\begin{flushleft}
Selected experiments to develop the synthetic, purification, and
\end{flushleft}


\begin{flushleft}
analytical/characterization skills in different areas of inorganic
\end{flushleft}


\begin{flushleft}
chemistry, such as, coordination, organometallic, bioinorganic
\end{flushleft}


\begin{flushleft}
chemistry, and so forth.
\end{flushleft}





\begin{flushleft}
CML512 Stereochemistry \& Organic Reaction
\end{flushleft}


\begin{flushleft}
Mechanisms
\end{flushleft}


\begin{flushleft}
3 Credits (3-0-0)
\end{flushleft}


\begin{flushleft}
Stereochemistry of acyclic and cyclic compounds including chiral
\end{flushleft}


\begin{flushleft}
molecules without a chiral centre, Reaction mechanisms (polar and free
\end{flushleft}


\begin{flushleft}
radical) with stereochemical considerations, Reactive intermediates:
\end{flushleft}


\begin{flushleft}
generation, structure, and reactivity.
\end{flushleft}





\begin{flushleft}
Basic laboratory techniques to synthesize, purify, and characterize
\end{flushleft}


\begin{flushleft}
small organic molecules by analytical and spectroscopic methods.
\end{flushleft}





\begin{flushleft}
CMP522 Laboratory-IV
\end{flushleft}


\begin{flushleft}
2 Credits (0-0-4)
\end{flushleft}


\begin{flushleft}
Determination of enzyme activity in biological samples, Protein
\end{flushleft}


\begin{flushleft}
purification and characterization, Microbial growth experiments, DNA
\end{flushleft}


\begin{flushleft}
and RNA isolation, Gel electrophoresis.
\end{flushleft}





\begin{flushleft}
CML522 Chemical Dynamics \& Surface Chemistry
\end{flushleft}


\begin{flushleft}
3 Credits (3-0-0)
\end{flushleft}


\begin{flushleft}
Kinetics of simple and complex reactions, Transport properties,
\end{flushleft}


\begin{flushleft}
Theories of reaction rates and dynamics of gas and liquid phase
\end{flushleft}


\begin{flushleft}
reactions, Experimental techniques to study fast reactions,
\end{flushleft}


\begin{flushleft}
Photochemical reactions, Surface phenomena and physical methods
\end{flushleft}


\begin{flushleft}
for studying surfaces, Heterogeneous and homogeneous catalysis.
\end{flushleft}





\begin{flushleft}
CML523 Organic Synthesis
\end{flushleft}


\begin{flushleft}
3 Credits (3-0-0)
\end{flushleft}


\begin{flushleft}
Formation of carbon-carbon bonds including organometallic reactions,
\end{flushleft}


\begin{flushleft}
Synthetic applications of organoboranes and organosilanes, Reactions
\end{flushleft}


\begin{flushleft}
at unactivated C-H bonds, Oxidations, Reductions, Newer Reagents,
\end{flushleft}


\begin{flushleft}
Design of organic synthesis, Retrosynthetic analysis, Selectivity in
\end{flushleft}


\begin{flushleft}
organic synthesis, Protection and deprotection of functional groups,
\end{flushleft}


\begin{flushleft}
Multistep synthesis of some representative molecules.
\end{flushleft}





\begin{flushleft}
CML524 Transition and Inner Transition Metal Chemistry
\end{flushleft}


\begin{flushleft}
3 Credits (3-0-0)
\end{flushleft}


\begin{flushleft}
Introduction to coordination chemistry, Crystal field theory, Ligand
\end{flushleft}


\begin{flushleft}
field theory, Molecular orbital theory, Magnetic and spectral
\end{flushleft}





162





\begin{flushleft}
\newpage
Chemistry
\end{flushleft}





\begin{flushleft}
characteristics of inner transition metal complexes, Substitution,
\end{flushleft}


\begin{flushleft}
Electron transfer and photochemical reactions of transition
\end{flushleft}


\begin{flushleft}
metal complexes, Physical, spectroscopic, and electrochemical
\end{flushleft}


\begin{flushleft}
methods used in the study of transition metal complexes,
\end{flushleft}


\begin{flushleft}
Metal-metal bonded compounds and transition metal cluster
\end{flushleft}


\begin{flushleft}
compounds, Uses of lanthanide complexes: as shift reagents,
\end{flushleft}


\begin{flushleft}
as strong magnets, and in fluorescence, Bioinorganic chemistry:
\end{flushleft}


\begin{flushleft}
introduction, Bioinorganic chemistry of iron: hemoglobin, myoglobin,
\end{flushleft}


\begin{flushleft}
cytochromes, Bioinorganic chemistry of zinc, cobalt, and copper.
\end{flushleft}





\begin{flushleft}
CML664 Group Theory \& Spectroscopy
\end{flushleft}


\begin{flushleft}
3 Credits (3-0-0)
\end{flushleft}





\begin{flushleft}
CML525 Basic Organometalic Chemistry
\end{flushleft}


\begin{flushleft}
3 Credits (3-0-0)
\end{flushleft}





\begin{flushleft}
Structure and conformations of proteins, nucleic acids and other
\end{flushleft}


\begin{flushleft}
biological polymers, Techniques for the study of biological structure
\end{flushleft}


\begin{flushleft}
and function, Configurational statistics and conformational transitions,
\end{flushleft}


\begin{flushleft}
Thermodynamics and kinetics of ligand interactions, Regulation of
\end{flushleft}


\begin{flushleft}
biological activity, Bioinformatics: Genomics and proteomics.
\end{flushleft}





\begin{flushleft}
Organometallic chemistry of main group, transition, and inner
\end{flushleft}


\begin{flushleft}
transition metals. Synthesis and applications of BuLi, Grignard,
\end{flushleft}


\begin{flushleft}
organoaluminum, and organozinc reagents, 18 electron rule, Metal
\end{flushleft}


\begin{flushleft}
carbonyls: bonding and infrared spectra, phosphines and NHC's,
\end{flushleft}


\begin{flushleft}
Alkenes and alkynes, carbenes and carbynes (Fisher and Schrock),
\end{flushleft}


\begin{flushleft}
Hapto ligands with hapticity from 2-8, Oxidative addition and reductive
\end{flushleft}


\begin{flushleft}
elimination, 1,1 and 1,2-migratory insertions and beta hydrogen
\end{flushleft}


\begin{flushleft}
elimination, Mechanism of substitution reactions, Fluxionality and
\end{flushleft}


\begin{flushleft}
hapticity change, Organometallic clusters, C-H activation: agostic
\end{flushleft}


\begin{flushleft}
and anagostic interactions, Homogeneous catalysis: hydrogenation,
\end{flushleft}


\begin{flushleft}
hydroformylation, methanol to acetic acid processes, and Wacker
\end{flushleft}


\begin{flushleft}
oxidation, Introduction to cross coupling and olefin metathesis
\end{flushleft}


\begin{flushleft}
reactions, Olefin oligomerization and polymerization.
\end{flushleft}





\begin{flushleft}
CML526 Structure \& Function of Cellular Biomolecules
\end{flushleft}


\begin{flushleft}
3 Credits (3-0-0)
\end{flushleft}


\begin{flushleft}
Prokaryotic and eukaryotic cells, Structure and function of proteins,
\end{flushleft}


\begin{flushleft}
carbohydrates, nucleic acids, and lipids. Biological membranes,
\end{flushleft}


\begin{flushleft}
Enzymes: classification, kinetics, mechanism, and applications. Basic
\end{flushleft}


\begin{flushleft}
concepts of microbial culture, growth, and physiology.
\end{flushleft}





\begin{flushleft}
CMD611 Project Part-I
\end{flushleft}


\begin{flushleft}
6 Credits (0-0-12)
\end{flushleft}





\begin{flushleft}
Symmetry operations, Review of point and space groups, Applications
\end{flushleft}


\begin{flushleft}
of group theoretical techniques in spectroscopy, Chemical bonding,
\end{flushleft}


\begin{flushleft}
Crystallography, Theoretical treatment of rotational, vibrational, and
\end{flushleft}


\begin{flushleft}
electronic spectroscopy, Magnetic spectroscopy.
\end{flushleft}





\begin{flushleft}
CML665 Biophysical Chemistry
\end{flushleft}


\begin{flushleft}
3 Credits (3-0-0)
\end{flushleft}





\begin{flushleft}
CML671 Supramolecular Chemistry
\end{flushleft}


\begin{flushleft}
3 Credits (3-0-0)
\end{flushleft}


\begin{flushleft}
Non-covalent associations, Molecular recognition, Design and
\end{flushleft}


\begin{flushleft}
applications of molecular hosts: crown compounds, cyclophanes,
\end{flushleft}


\begin{flushleft}
cyclodextrins, etc., Nano technology, Molecular clefts, tweezers, and
\end{flushleft}


\begin{flushleft}
devices, Self assembly and replication.
\end{flushleft}





\begin{flushleft}
CML672 Recent Trends in Organic Chemistry
\end{flushleft}


\begin{flushleft}
3 Credits (3-0-0)
\end{flushleft}


\begin{flushleft}
Recent advances in organic synthesis, spectroscopy, and reaction
\end{flushleft}


\begin{flushleft}
mechanisms.
\end{flushleft}





\begin{flushleft}
CML673 Bio-organic and Medicinal Chemistry
\end{flushleft}


\begin{flushleft}
3 Credits (3-0-0)
\end{flushleft}


\begin{flushleft}
Bio-organic: Amino acids, polypeptides, and enzyme models, Medicinal:
\end{flushleft}


\begin{flushleft}
definitions and classifications, Pharmaceutical, pharmacokinetic, and
\end{flushleft}


\begin{flushleft}
pharmacodynamic phases, Drug-receptor interactions, Intra- and
\end{flushleft}


\begin{flushleft}
intermolecular forces, Solvent effects, Ligand binding, Docking and
\end{flushleft}


\begin{flushleft}
design, Drug metabolism.
\end{flushleft}





\begin{flushleft}
CML674 Physical Methods of Structure Determination
\end{flushleft}


\begin{flushleft}
of Organic Compounds		
\end{flushleft}


\begin{flushleft}
3 Credits (3-0-0)
\end{flushleft}





\begin{flushleft}
CMD621 Project Part-II
\end{flushleft}


\begin{flushleft}
10 Credits (0-0-20)
\end{flushleft}





\begin{flushleft}
Applications of UV, IR, NMR, and mass spectral methods in structure
\end{flushleft}


\begin{flushleft}
determination of organic compounds.
\end{flushleft}





\begin{flushleft}
CML631 Molecular Biochemistry
\end{flushleft}


\begin{flushleft}
3 Credits (3-0-0)
\end{flushleft}


\begin{flushleft}
Central dogma, DNA replication and repair, Transcription, Translation,
\end{flushleft}


\begin{flushleft}
Recombinant DNA technology, Basic concept of metabolism: glycolysis,
\end{flushleft}


\begin{flushleft}
TCA cycle, \ss{}-oxidation, Amino acid transamination and urea cycle.
\end{flushleft}





\begin{flushleft}
CML675/CML740 Chemistry of Heterocyclic Compounds
\end{flushleft}


\begin{flushleft}
3 Credits (3-0-0)
\end{flushleft}


\begin{flushleft}
Chemistry of heterocyclic compounds containing one, two, and three
\end{flushleft}


\begin{flushleft}
heteroatoms, Total synthesis of representative natural products.
\end{flushleft}





\begin{flushleft}
CML681 Physical Methods in Inorganic Chemistry
\end{flushleft}


\begin{flushleft}
3 Credits (3-0-0)
\end{flushleft}





\begin{flushleft}
CML661 Solid State Chemistry	
\end{flushleft}


\begin{flushleft}
3 Credits (3-0-0)
\end{flushleft}


\begin{flushleft}
Crystal chemistry, Bonding in solids, Defects and non-stoichiometry,
\end{flushleft}


\begin{flushleft}
A range of synthetic and analytical techniques to prepare and
\end{flushleft}


\begin{flushleft}
characterize solids, Electronic, magnetic, and superconducting
\end{flushleft}


\begin{flushleft}
properties, Optical properties which include: luminescence and lasers,
\end{flushleft}


\begin{flushleft}
nanostructures and low dimensional properties, etc.
\end{flushleft}





\begin{flushleft}
CML662 Statistical Mechanics \& Molecular Simulation
\end{flushleft}


\begin{flushleft}
Methods
\end{flushleft}


\begin{flushleft}
3 Credits (3-0-0)
\end{flushleft}


\begin{flushleft}
Micro- and macroscopic state of a classical system, Phase space,
\end{flushleft}


\begin{flushleft}
Ergodicity and mixing in phase space, Theory of ensembles, Classical
\end{flushleft}


\begin{flushleft}
fluids, Phase transitions and relaxation phenomena, Monte Carlo,
\end{flushleft}


\begin{flushleft}
molecular dynamics, and Brownian dynamics, Computer simulations,
\end{flushleft}


\begin{flushleft}
Brownian motion, Langevin equation, Elucidation of structural, dynamic,
\end{flushleft}


\begin{flushleft}
and thermodynamic properties of complex fluids and soft matter.
\end{flushleft}





\begin{flushleft}
CML663 Selected Topics in Spectroscopy
\end{flushleft}


\begin{flushleft}
3 Credits (3-0-0)
\end{flushleft}


\begin{flushleft}
Franck-Condon principle, Fermi Golden rule, Normal mode analysis,
\end{flushleft}


\begin{flushleft}
Multi-photon spectroscopy, Molecular beam techniques, Non-linear
\end{flushleft}


\begin{flushleft}
laser spectroscopy, Two-level systems, Precession, Rabi frequency,
\end{flushleft}


\begin{flushleft}
Nutation, Block equations, Multi-dimensional NMR techniques.
\end{flushleft}





\begin{flushleft}
Use of NMR spectroscopy for structural elucidation of simple inorganic
\end{flushleft}


\begin{flushleft}
and organometallic compounds using chemical shifts and heteronuclear coupling constants, Relaxation phenomena in inorganic
\end{flushleft}


\begin{flushleft}
compounds, Double resonance technique and its applications, EPR
\end{flushleft}


\begin{flushleft}
spectroscopy for the identification of inorganic radicals, Introduction
\end{flushleft}


\begin{flushleft}
to Mossbauer spectroscopy, Factors influencing chemical shifts and
\end{flushleft}


\begin{flushleft}
quadrupolar splitting, Structural information: X-ray diffraction methods
\end{flushleft}


\begin{flushleft}
(powder and single crystal), Finger printing of solids from powder
\end{flushleft}


\begin{flushleft}
data and determination of crystal structures by Rietveld analysis and
\end{flushleft}


\begin{flushleft}
single crystal studies.
\end{flushleft}





\begin{flushleft}
CML682 Inorganic Polymers
\end{flushleft}


\begin{flushleft}
3 Credits (3-0-0)
\end{flushleft}


\begin{flushleft}
Homo and heterocatenated inorganic polymers: general introduction,
\end{flushleft}


\begin{flushleft}
Polyphosphazenes: synthetic routes and bonding features,
\end{flushleft}


\begin{flushleft}
Polymerization of organo/organometallic substituted phosphazenes
\end{flushleft}


\begin{flushleft}
and their applications, Polysilanes: synthesis and characterization
\end{flushleft}


\begin{flushleft}
of polysilanes, unique electronic and optical properties and
\end{flushleft}


\begin{flushleft}
its applications, Polysiloxanes: precursors used in synthesis of
\end{flushleft}


\begin{flushleft}
polysiloxanes via anionic and cationic polymerization methods,
\end{flushleft}


\begin{flushleft}
properties and environmental aspects, Polysiloles and their comparison
\end{flushleft}


\begin{flushleft}
with polythiophenes, Introduction to organometallic polymers:
\end{flushleft}


\begin{flushleft}
synthesis of poly(ferrocenylsilane)s and their applications. Catalytic
\end{flushleft}





163





\begin{flushleft}
\newpage
Chemistry
\end{flushleft}





\begin{flushleft}
methods for homo and hetero-catenated polymers, Characterization
\end{flushleft}


\begin{flushleft}
methods (spectroscopy, gel permeation chromatography, differential
\end{flushleft}


\begin{flushleft}
scanning calorimetry)
\end{flushleft}





\begin{flushleft}
CML683 Applied Organometallic Chemistry
\end{flushleft}


\begin{flushleft}
3 Credits (3-0-0)
\end{flushleft}


\begin{flushleft}
Introduction to homogeneous catalysis, TON and TOF, Some aspects
\end{flushleft}


\begin{flushleft}
of commonly used ligands in homogeneous catalysis, such as, CO,
\end{flushleft}


\begin{flushleft}
amines, phosphines, NHC's, alkenes, alkynes, carbenes, carbynes,
\end{flushleft}


\begin{flushleft}
etc., Recent developments in hydrogenation and hydroformylation
\end{flushleft}


\begin{flushleft}
and their asymmetric variations using OM catalysts, Wacker oxidation,
\end{flushleft}


\begin{flushleft}
Monsanto and Cativa processes, Olefin and alkyne trimerization and
\end{flushleft}


\begin{flushleft}
oligomerization, Olefin polymerization using Ziegler-Natta, Titanium
\end{flushleft}


\begin{flushleft}
group metallocenes, Post metallocene late TM catalysts and FI
\end{flushleft}


\begin{flushleft}
catalysts, Olefin and alkyne metathesis, Grubbs I, II, and III, Schrock,
\end{flushleft}


\begin{flushleft}
and Schrock-Hoveyda catalysts, Types of metathesis such as RCM, ROM,
\end{flushleft}


\begin{flushleft}
ROMP, ADMET, and EM. Applications in industry, Palladium and nickel
\end{flushleft}


\begin{flushleft}
catalyzed cross coupling reactions such as Suzuki, Heck, Sonogashira,
\end{flushleft}


\begin{flushleft}
Stille, Negishi, Hiyama, Buchwald-Hartwig, decarboxylative cross
\end{flushleft}


\begin{flushleft}
coupling, and alpha arylation of carbonyls, Fischer Tropsch Process,
\end{flushleft}


\begin{flushleft}
C-H activation of alkyls and aryls using transition metal complexes,
\end{flushleft}


\begin{flushleft}
Organometallic polymers, Bio-organometallic chemistry: Vitamin B-12,
\end{flushleft}


\begin{flushleft}
Planar chirality of metal sandwich compounds and their applications in
\end{flushleft}


\begin{flushleft}
industry (e.g. Josiphos catalyst), Ferrocene based drugs, Sustainable
\end{flushleft}


\begin{flushleft}
catalysis for pharmaceuticals and industry using organometallics.
\end{flushleft}





\begin{flushleft}
CML684 Bio-Inorganic Chemistry
\end{flushleft}


\begin{flushleft}
3 Credits (3-0-0)
\end{flushleft}


\begin{flushleft}
Introduction of bio-inorganic chemistry, General properties of biological
\end{flushleft}


\begin{flushleft}
molecules, Physical methods in bio-inorganic chemistry, Binding of
\end{flushleft}


\begin{flushleft}
metal ions and complexes to biomolecule active centers, Synthesis and
\end{flushleft}


\begin{flushleft}
reactivity of active sites, Atom and group transfer chemistry, Electron
\end{flushleft}


\begin{flushleft}
transfer in proteins, Frontiers of bio-inorganic chemistry: some topics
\end{flushleft}


\begin{flushleft}
of current research interest.
\end{flushleft}





\begin{flushleft}
CML691 Microbial Biochemistry
\end{flushleft}


\begin{flushleft}
3 Credits (3-0-0)
\end{flushleft}


\begin{flushleft}
Microscopic examination of microorganisms, classification, morphology
\end{flushleft}


\begin{flushleft}
and fine structure of microbial cells, cultivation, reproduction and
\end{flushleft}


\begin{flushleft}
growth, pure culture techniques, Basic microbial metabolisms,
\end{flushleft}


\begin{flushleft}
Concepts of their genetics: transformation, transduction, and
\end{flushleft}


\begin{flushleft}
conjugation, Important microorganisms and enzymes.
\end{flushleft}





\begin{flushleft}
CML692 Food Chemistry and Biochemistry
\end{flushleft}


\begin{flushleft}
3 Credits (3-0-0)
\end{flushleft}


\begin{flushleft}
Carbohydrates: structure and functional properties of mono-oligopolysaccharides including starch, cellulose, pectic substances, and
\end{flushleft}


\begin{flushleft}
dietary fibers, Essential amino acids, proteins, and lipids in food and
\end{flushleft}


\begin{flushleft}
their impact on functional properties, vitamins and minerals, Food
\end{flushleft}


\begin{flushleft}
flavours: terpenes, esters, ketones, and quinines; Food additives,
\end{flushleft}


\begin{flushleft}
Bioactive constituents in food: isoflavones, phenol, and glycosides;
\end{flushleft}


\begin{flushleft}
Enzymes: enzymatic and non-enzymatic browning, enzymes in food
\end{flushleft}


\begin{flushleft}
processing, oxidative enzymes, Food biochemistry: balanced diet, PER,
\end{flushleft}


\begin{flushleft}
anti-nutrients and toxins, nutrition deficiency diseases.
\end{flushleft}





\begin{flushleft}
CML695/ CML739 Applied Biocatalysis
\end{flushleft}


\begin{flushleft}
3 Credits (3-0-0)
\end{flushleft}


\begin{flushleft}
Introduction to enzymes and enzyme catalysed reactions, Classification
\end{flushleft}


\begin{flushleft}
and mechanism of reaction, Purification and characterization of
\end{flushleft}


\begin{flushleft}
enzymes, Michelis Menten kinetics, Industrial enzymes, Applications
\end{flushleft}


\begin{flushleft}
of enzymes in diagnostics, analysis, biosensors, and other industrial
\end{flushleft}


\begin{flushleft}
processes and bio-transformations, Enzyme structure determination,
\end{flushleft}


\begin{flushleft}
stability, and stabilisation, Enzyme immobilization and concept of
\end{flushleft}


\begin{flushleft}
enzyme engineering, Nanobiocatalysis.
\end{flushleft}





\begin{flushleft}
CML721 Design and Synthesis of Organic Molecules
\end{flushleft}


\begin{flushleft}
3 Credits (3-0-0)
\end{flushleft}


\begin{flushleft}
Selectivity in organic synthesis: chemo-, regio-, stereo- and
\end{flushleft}


\begin{flushleft}
enantioselectivity. Target-oriented synthesis: Designing organic
\end{flushleft}


\begin{flushleft}
synthesis, Retrosynthetic analysis, disconnetion approach, linear
\end{flushleft}





\begin{flushleft}
and convergent synthesis. Diversity-oriented synthesis: concept of
\end{flushleft}


\begin{flushleft}
forward-synthetic analysis, appendage diversity, skeletal diversity,
\end{flushleft}


\begin{flushleft}
stereochemical diversity, complexity and diversity. Asymmetric
\end{flushleft}


\begin{flushleft}
Synthesis: Use of chiral catalysts, organocatalysis, chiron approach
\end{flushleft}


\begin{flushleft}
and N-heterocyclic carbenes. Principles and use of enzymes in the
\end{flushleft}


\begin{flushleft}
syntheis of industrially important sugar / fatty acid esters, sugar
\end{flushleft}


\begin{flushleft}
nucleotide derivatives ; enantiomeric pure compounds and biobased
\end{flushleft}


\begin{flushleft}
platform chemicals.
\end{flushleft}





\begin{flushleft}
CMP722 Synthesis of Organic and Inorganic
\end{flushleft}


\begin{flushleft}
Compounds
\end{flushleft}


\begin{flushleft}
3 Credits (0-0-6)
\end{flushleft}


\begin{flushleft}
Single, double and multi-stage preparation of organic, inorganic
\end{flushleft}


\begin{flushleft}
and organometallic compounds; experiments involving the concepts
\end{flushleft}


\begin{flushleft}
of protecting groups and selectivity; identification of compounds
\end{flushleft}


\begin{flushleft}
through thin-layer chromatography and their purification by column
\end{flushleft}


\begin{flushleft}
chromatography. Characterization of synthesized compounds using
\end{flushleft}


\begin{flushleft}
IR, UV, 1H-NMR and mass spectromteric techniques.
\end{flushleft}





\begin{flushleft}
CML723 Principles and practice of NMR and Optical
\end{flushleft}


\begin{flushleft}
Spectroscopy
\end{flushleft}


\begin{flushleft}
3 Credits (3-0-0)
\end{flushleft}


\begin{flushleft}
Fundamentals of FT NMR spectroscopy, relation between structure and
\end{flushleft}


\begin{flushleft}
NMR properties, one-dimensional spectroscopy (1H, 13C, DEPT, steady
\end{flushleft}


\begin{flushleft}
state NOE, saturation transfer) and an introduction to two-dimensional
\end{flushleft}


\begin{flushleft}
NMR (COSY, NOESY, and HSQC) and their use in structure elucidation.
\end{flushleft}


\begin{flushleft}
Principles and analytical applications of optical spectroscopic methods
\end{flushleft}


\begin{flushleft}
including atomic absorption and emission, UV-Visible, IR absorption,
\end{flushleft}


\begin{flushleft}
scattering, and luminescence.
\end{flushleft}





\begin{flushleft}
CML724 Synthesis of Industrially Important Inorganic
\end{flushleft}


\begin{flushleft}
Materials
\end{flushleft}


\begin{flushleft}
3 Credits (3-0-0)
\end{flushleft}


\begin{flushleft}
Modern methods applied in the synthesis of inorganic, organometallic
\end{flushleft}


\begin{flushleft}
and polymer materials. Handling of air and moisture sensitive
\end{flushleft}


\begin{flushleft}
compounds, dry box, glove bag, Schlenk line and vacuum line
\end{flushleft}


\begin{flushleft}
techniques. Methods of purification of and handling of reactive
\end{flushleft}


\begin{flushleft}
industrial gases. Methods of purification of inorganic compounds and
\end{flushleft}


\begin{flushleft}
crystallization of solids for X-ray analysis. General strategies, brief
\end{flushleft}


\begin{flushleft}
outline of theory and methodology used for the synthesis of inorganic/
\end{flushleft}


\begin{flushleft}
organometallic molecules to materials including macromolecules.
\end{flushleft}


\begin{flushleft}
Emphasis will be placed how to adopt appropriate synthetic routes to
\end{flushleft}


\begin{flushleft}
control shape and size of the final product, ranging from amorphous
\end{flushleft}


\begin{flushleft}
materials, porous solids, thin films, large single crystals, and special
\end{flushleft}


\begin{flushleft}
forms of nanomaterials. A few examples of detailed synthesis will be
\end{flushleft}


\begin{flushleft}
highlighted in each category of materials.
\end{flushleft}





\begin{flushleft}
CML726 Cheminformatics and Molecular Modelling
\end{flushleft}


\begin{flushleft}
3 Credits (3-0-0)
\end{flushleft}


\begin{flushleft}
Chemistry \& Information technology, chemical / biochemical data
\end{flushleft}


\begin{flushleft}
collation, retrieval, analysis \& interpretation, hypothesis generation
\end{flushleft}


\begin{flushleft}
\& validation, development of structure activity/property relationships,
\end{flushleft}


\begin{flushleft}
artificial intelligence techniques in chemistry. Building molecules on a
\end{flushleft}


\begin{flushleft}
computer, quantum and molecular mechanics methods for geometry
\end{flushleft}


\begin{flushleft}
optimization, Simulation methods for molecules and materials.
\end{flushleft}





\begin{flushleft}
CMP728 Instrumentation Laboratory
\end{flushleft}


\begin{flushleft}
3 Credits (0-0-6)
\end{flushleft}


\begin{flushleft}
Experiments based on Instrumental methods of chemical analysis
\end{flushleft}


\begin{flushleft}
involving spectroscopy, microscopy and thermal methods.
\end{flushleft}





\begin{flushleft}
CML729 Material Characterization
\end{flushleft}


\begin{flushleft}
3 Credits (3-0-0)
\end{flushleft}


\begin{flushleft}
Compositional analysis of solid materials by X-ray and electron
\end{flushleft}


\begin{flushleft}
microscopic techniques. Basic concepts of diffraction techniques
\end{flushleft}


\begin{flushleft}
(powder and single crystal) in elucidating the crystal structures
\end{flushleft}


\begin{flushleft}
organic, inorganic and hybrid materials. Applications of electron
\end{flushleft}


\begin{flushleft}
microscopic techniques (scanning and transmission) for morphological
\end{flushleft}


\begin{flushleft}
and nanostructural features. Thermal analytical methods for
\end{flushleft}


\begin{flushleft}
correlating structural information and monitoring phase transition.
\end{flushleft}





164





\begin{flushleft}
\newpage
Chemistry
\end{flushleft}





\begin{flushleft}
Emphasis will be placed on the above techniques for industrially
\end{flushleft}


\begin{flushleft}
important materials and the interpretation and evaluation of the
\end{flushleft}


\begin{flushleft}
results obtained by various methods.
\end{flushleft}





\begin{flushleft}
Frustrated Lewis acid bases as catalysts. Superacids and their uses.
\end{flushleft}


\begin{flushleft}
Sulphonamides, industrial applications of sulfur and selenium. Fluorine
\end{flushleft}


\begin{flushleft}
in pharmaceuticals, fluoropolymers.
\end{flushleft}





\begin{flushleft}
CML731 Chemical Separation and Electroanalytical
\end{flushleft}


\begin{flushleft}
Methods
\end{flushleft}


\begin{flushleft}
3 Credits (3-0-0)
\end{flushleft}





\begin{flushleft}
CML739/695 Applied Biocatalysis
\end{flushleft}


\begin{flushleft}
3 Credits (3-0-0)
\end{flushleft}





\begin{flushleft}
Theory and applications of equilibrium and nonequilibrium
\end{flushleft}


\begin{flushleft}
separation techniques. Extraction, countercurrent distribution, gas
\end{flushleft}


\begin{flushleft}
chromatography, column and plane chromatographic techniques,
\end{flushleft}


\begin{flushleft}
electrophoresis, ultracentrifugation, and other separation methods,
\end{flushleft}


\begin{flushleft}
Modern analytical and separation techniques used in biochemical
\end{flushleft}


\begin{flushleft}
analysis. Principles of electrochemical methods, electrochemical
\end{flushleft}


\begin{flushleft}
reactions, steady-state and potential step techniques; polarography,
\end{flushleft}


\begin{flushleft}
cyclic voltammetry, chrono methods, rotating disc and ring disc
\end{flushleft}


\begin{flushleft}
electrodes, concepts and applications of AC impedance techniques.
\end{flushleft}





\begin{flushleft}
Introduction to enzymes and enzyme catalysed reactions. Classification
\end{flushleft}


\begin{flushleft}
and mechanism of reaction. Purification and characterization of
\end{flushleft}


\begin{flushleft}
enzymes. Michelis Menten kinetics, Industrial enzymes. Applications
\end{flushleft}


\begin{flushleft}
of enzymes in diagnostics, analysis, biosensors and other industrial
\end{flushleft}


\begin{flushleft}
processes and bio-transformations. Enzyme structure determination,
\end{flushleft}


\begin{flushleft}
stability and stabilisation. Enzyme immobilization and concept of
\end{flushleft}


\begin{flushleft}
enzyme engineering. Nanobiocatalysis.
\end{flushleft}





\begin{flushleft}
CML740/675 Chemistry of Heterocyclic Compounds
\end{flushleft}


\begin{flushleft}
3 Credits (3-0-0)
\end{flushleft}


\begin{flushleft}
Chemistry of heterocyclic compounds containing one, two and three
\end{flushleft}


\begin{flushleft}
heteroatoms. Total synthesis of representative natural products.
\end{flushleft}





\begin{flushleft}
CML733 Chemistry of Industrial Catalysts
\end{flushleft}


\begin{flushleft}
3 Credits (3-0-0)
\end{flushleft}


\begin{flushleft}
Fundamental aspects of Catalysis - Homogeneous \& heterogeneous
\end{flushleft}


\begin{flushleft}
catalysis -The role of catalytic processes in modern chemical
\end{flushleft}


\begin{flushleft}
manufacturing -organometallic catalysts -catalysis in organic
\end{flushleft}


\begin{flushleft}
polymer chemistry -catalysis in petroleum industry - catalysis in
\end{flushleft}


\begin{flushleft}
environmental control.
\end{flushleft}





\begin{flushleft}
CML734 Chemistry of Nanostructured Materials
\end{flushleft}


\begin{flushleft}
3 Credits (3-0-0)
\end{flushleft}


\begin{flushleft}
Introduction; fundamentals of nanomaterials science, surface science
\end{flushleft}


\begin{flushleft}
for nanomaterials, colloidal chemistry; Synthesis, preparation and
\end{flushleft}


\begin{flushleft}
fabrication: chemical routes, self assembly methods, biomimetic and
\end{flushleft}


\begin{flushleft}
electrochemical approaches; Size controls properties (optical, electronic
\end{flushleft}


\begin{flushleft}
and magnetic properties of materials) - Applications (carbon nanotubes
\end{flushleft}


\begin{flushleft}
and nanoporous zeolites; Quantum Dots, basic ideas of nanodevices).
\end{flushleft}





\begin{flushleft}
CML737 Applied Spectroscopy
\end{flushleft}


\begin{flushleft}
3 Credits (3-0-0)
\end{flushleft}


\begin{flushleft}
Applications of advanced 1D-NMR techniques such as nOe, 1D
\end{flushleft}


\begin{flushleft}
13C-NMR (including APT and DEPT) techniques, multinuclear
\end{flushleft}


\begin{flushleft}
NMR spectroscopy, 2D NMR techniques (COSY, HETCOR, HSQC,
\end{flushleft}


\begin{flushleft}
HMBC, NOESY, ROESY etc.) for the structural and stereochemical
\end{flushleft}


\begin{flushleft}
determination of organic compounds. Introduction to various types
\end{flushleft}


\begin{flushleft}
of ionizations (such as EI, CI, MALDI, field ionization/desorption,
\end{flushleft}


\begin{flushleft}
electrospray ionization) and analyzers (such as quadrupole, time of
\end{flushleft}


\begin{flushleft}
flight, triple quadupole, QqTOF, ion-trap) in mass spectrometry for MS,
\end{flushleft}


\begin{flushleft}
MS/MS and MSn applications. Determination of peptide sequencing
\end{flushleft}


\begin{flushleft}
using mass spectrometric techniques.
\end{flushleft}





\begin{flushleft}
CML738 Applications of P-block Elements and their
\end{flushleft}


\begin{flushleft}
Compounds
\end{flushleft}


\begin{flushleft}
3 Credits (3-0-0)
\end{flushleft}


\begin{flushleft}
Introduction, Structure, bonding and recent discussions on d orbital
\end{flushleft}


\begin{flushleft}
participation. Boranes, carboranes and metallaboranes and their use
\end{flushleft}


\begin{flushleft}
in BNCT and as control rods in nuclear reactors, modern electron
\end{flushleft}


\begin{flushleft}
counting methods such as Jemmis rules, chemistry of B(0) and B(1).
\end{flushleft}


\begin{flushleft}
GaAs, GaN, InSnO3: Synthesis and applications in solar cells, LED and
\end{flushleft}


\begin{flushleft}
as transparent conducting materials. Fullerenes, nanotubes, graphene,
\end{flushleft}


\begin{flushleft}
silicates, aluminosilicates, zeolites and their applications. Silicones
\end{flushleft}


\begin{flushleft}
and their industrial applications. Si(II) and Ge(II) chemistry. NHC's
\end{flushleft}


\begin{flushleft}
and their use in stabilizing main group compounds. Nitrogen based
\end{flushleft}


\begin{flushleft}
fertilizers, Ammonia, Haber-Bosch Process, nitrogen based explosives,
\end{flushleft}


\begin{flushleft}
hydrazines as rockel fuels, applications of azides and pentazenium.
\end{flushleft}


\begin{flushleft}
Phosphorus based fertilizer processes, phosphorus based pesticides,
\end{flushleft}


\begin{flushleft}
phosphorus-nitrogen compounds as multidentate ligands, superbases,
\end{flushleft}


\begin{flushleft}
dendrimer cores and polymers. Phosphines and their industrial uses.
\end{flushleft}





\begin{flushleft}
CML741 Organo and organometallic catalysis
\end{flushleft}


\begin{flushleft}
3 Credits (3-0-0)
\end{flushleft}


\begin{flushleft}
Introduction. Enamine catalysis. Iminium catalysis. Asymmetric
\end{flushleft}


\begin{flushleft}
proton catalysis. Ammonium ions as chiral templates. Chiral Lewis
\end{flushleft}


\begin{flushleft}
bases as catalysts. Asymmetric acyl transfer reactions. Ylide based
\end{flushleft}


\begin{flushleft}
reactions. Transition metal catalyzed reactions. C-H activation.
\end{flushleft}


\begin{flushleft}
N-Heterocyclic carbenes.
\end{flushleft}





\begin{flushleft}
CML742 Reagents in Synthetic Transformations
\end{flushleft}


\begin{flushleft}
3 Credits (3-0-0)
\end{flushleft}


\begin{flushleft}
The course will cover the applications of various oxidation and
\end{flushleft}


\begin{flushleft}
reduction reactions in organic chemistry with special emphasis on
\end{flushleft}


\begin{flushleft}
special reagents that are used for selective transformations. Use of
\end{flushleft}


\begin{flushleft}
organolithium and organoboron compounds in organic synthesis and
\end{flushleft}


\begin{flushleft}
olefin metathesis will also serve a part of the course.
\end{flushleft}





\begin{flushleft}
CMD799 Minor project
\end{flushleft}


\begin{flushleft}
3 Credits (0-0-6)
\end{flushleft}


\begin{flushleft}
CML801 Molecular Modelling and Simulations:
\end{flushleft}


\begin{flushleft}
Concepts and Techniques
\end{flushleft}


\begin{flushleft}
3 Credits (3-0-0)
\end{flushleft}





\begin{flushleft}
Review of Basic Concepts: Length and Time Scales, Intermolecular
\end{flushleft}


\begin{flushleft}
Interactions and Potential Energy Surfaces, Evaluation of Long-range
\end{flushleft}


\begin{flushleft}
interactions Static and Dynamic Properties of Simple and Complex
\end{flushleft}


\begin{flushleft}
Liquids Molecular Dynamics: Microcanonical and other ensembles;
\end{flushleft}


\begin{flushleft}
Constrained simulations; non-equilibrium approaches Monte
\end{flushleft}


\begin{flushleft}
Carlo Methods: Random Numbers and Random Walk, Metropolis
\end{flushleft}


\begin{flushleft}
Algorithm in various ensembles, Biased Monte Carlo Schemes Free
\end{flushleft}


\begin{flushleft}
Energy Estimations: Mapping Phase Diagrams, Generating Free
\end{flushleft}


\begin{flushleft}
Energy Landscapes, Collective Variables Rare Event Simulations
\end{flushleft}


\begin{flushleft}
and Reaction Dynamics
\end{flushleft}


\begin{flushleft}
VII. Advanced Topics: First principles molecular dynamics, Quantum
\end{flushleft}


\begin{flushleft}
Monte Carlo methods, Coarse-Graining and Multiscale Simulations for
\end{flushleft}


\begin{flushleft}
Nanoscale Systems, Quantum mechanics/molecular mechanics (QM/
\end{flushleft}


\begin{flushleft}
MM) approaches. (To some extent, coverage of advanced topics will
\end{flushleft}


\begin{flushleft}
depend on research interests of students and faculty since this is a
\end{flushleft}


\begin{flushleft}
Pre-Ph.D. course).
\end{flushleft}





\begin{flushleft}
CMD806 Major Project Part-I
\end{flushleft}


\begin{flushleft}
6 Credits (0-0-12)
\end{flushleft}


\begin{flushleft}
CMD807 Major Project Part-II
\end{flushleft}


\begin{flushleft}
12 Credits (0-0-24)
\end{flushleft}





165





\begin{flushleft}
\newpage
Department of Civil Engineering
\end{flushleft}


\begin{flushleft}
CVL100 Environmental Science
\end{flushleft}


\begin{flushleft}
2 Credits (2-0-0)
\end{flushleft}


\begin{flushleft}
Pollutant sources and control in air and water, solid waste management,
\end{flushleft}


\begin{flushleft}
noise pollution and control, cleaner production and life cycle analysis,
\end{flushleft}


\begin{flushleft}
reuse, recovery, source reduction and raw material substitution, basics
\end{flushleft}


\begin{flushleft}
of environmental impact assessment, environmental risk assessment
\end{flushleft}


\begin{flushleft}
and environmental audit, emerging technologies for sustainable
\end{flushleft}


\begin{flushleft}
environmental management, identification and evaluation of emerging
\end{flushleft}


\begin{flushleft}
environmental issues with air, water, wastewater and solid wastes.
\end{flushleft}





\begin{flushleft}
CVL111 Elements of Surveying
\end{flushleft}


\begin{flushleft}
4 Credits (3-0-2)
\end{flushleft}


\begin{flushleft}
Introduction to Surveying, Levels, Theodolites, total station.
\end{flushleft}


\begin{flushleft}
Measurement of distances, directions and elevations. Traversing.
\end{flushleft}


\begin{flushleft}
Trigonometric levelling. Mapping and contouring. Measurement of
\end{flushleft}


\begin{flushleft}
areas, volumes. Quantity computations. Errors of measurements
\end{flushleft}


\begin{flushleft}
and their adjustments. Curve setting: simple, compound and reverse
\end{flushleft}


\begin{flushleft}
curves. Introduction to GPS, Differential GPS, Remote sensing
\end{flushleft}


\begin{flushleft}
techniques and application in land use change and mapping, arial
\end{flushleft}


\begin{flushleft}
surveying, photogrametery.
\end{flushleft}





\begin{flushleft}
CVL121 Engineering Geology
\end{flushleft}


\begin{flushleft}
3 Credits (3-0-0)
\end{flushleft}


\begin{flushleft}
Engineering Geology: Introduction; Dynamic Earth; Origin, Age,
\end{flushleft}


\begin{flushleft}
Interior, Materials of Earth; Silicate Structures and Symmetry Elements;
\end{flushleft}


\begin{flushleft}
Physical properties, Formation of Rocks ;Igneous, Sedimentary and
\end{flushleft}


\begin{flushleft}
Metamorphic processes and structures, Characterisation; Weathering
\end{flushleft}


\begin{flushleft}
Processes; Geological Work of Rivers, Glaciers, Wind and Sea/Oceans,
\end{flushleft}


\begin{flushleft}
Deposits and Landforms; Formation of Soils; Geological Time Scale;
\end{flushleft}


\begin{flushleft}
Structural Features, Attitude of beds, Folds, Joints, Faults, Plate
\end{flushleft}


\begin{flushleft}
tectonics; Stress Distribution; Geophysical methods,Earthquakes.
\end{flushleft}


\begin{flushleft}
Engineering Properties of Rocks; Rock as Construction Material;
\end{flushleft}


\begin{flushleft}
Geological Site Criteria for Tunnels and Underground Structures,
\end{flushleft}


\begin{flushleft}
Foundations, Dams, Rock Slopes and Landslides.
\end{flushleft}





\begin{flushleft}
CVP121 Engineering Geology Lab
\end{flushleft}


\begin{flushleft}
1 Credit (0-0-2)
\end{flushleft}


\begin{flushleft}
Pre-requisites: CVL121 or concurrent with CVL121
\end{flushleft}


\begin{flushleft}
Geological Maps, Geological Mapping -- contouring, topo sheets,
\end{flushleft}


\begin{flushleft}
outcrops, apparent and true dips, three point problems, depth and
\end{flushleft}


\begin{flushleft}
thickness problems, joints, faults; Megascopic and Microscopic
\end{flushleft}


\begin{flushleft}
identification of Minerals and Rocks, Engineering properties of rocks,
\end{flushleft}


\begin{flushleft}
refraction and resistivity methods, Guided tour through representative
\end{flushleft}


\begin{flushleft}
geological formations and structures.
\end{flushleft}





\begin{flushleft}
CVL141 Civil Engineering Materials
\end{flushleft}


\begin{flushleft}
3 Credits (3-0-0)
\end{flushleft}


\begin{flushleft}
Mechanical properties of engineered materials, Temperature
\end{flushleft}


\begin{flushleft}
and time effects. Failure and safety. Non-mechanical properties.
\end{flushleft}


\begin{flushleft}
Durability. Nature of materials, classes of materials based on
\end{flushleft}


\begin{flushleft}
bonding, inorganic and organic solids. Variability in materials and
\end{flushleft}


\begin{flushleft}
its implication on measurement. Cement based materials, concrete
\end{flushleft}


\begin{flushleft}
production and processes; properties. Steel and other metals used
\end{flushleft}


\begin{flushleft}
in construction. Bricks and Masonry; wood and engineered wood
\end{flushleft}


\begin{flushleft}
products; glass and heat transmission properties. Polymers for
\end{flushleft}


\begin{flushleft}
construction and maintenance of infrastructure. Composites: fiber
\end{flushleft}


\begin{flushleft}
reinforced composites, particle reinforced composites. Introduction
\end{flushleft}


\begin{flushleft}
to sustainable materials.
\end{flushleft}





\begin{flushleft}
CVL212 Environmental Engineering
\end{flushleft}


\begin{flushleft}
4 Credits (3-0-2)
\end{flushleft}


\begin{flushleft}
Pre-requisites: CVL100
\end{flushleft}


\begin{flushleft}
Water and wastewater treatment overview; Unit processes: systems of
\end{flushleft}


\begin{flushleft}
water purification, processes (sedimentation, coagulation-flocculation,
\end{flushleft}


\begin{flushleft}
softening, disinfection, adsorption, ion exchange, filtration) and
\end{flushleft}


\begin{flushleft}
kinetics in unit operation of water purification-theory and design
\end{flushleft}


\begin{flushleft}
aspects; distribution of water layout systems: design aspects;
\end{flushleft}


\begin{flushleft}
Wastewater engineering: systems of sanitation, wastewater collection
\end{flushleft}


\begin{flushleft}
systems design and flows,; Characteristics and microbiology of
\end{flushleft}


\begin{flushleft}
wastewater, BOD kinetics; Unit processes for wastewater treatment
\end{flushleft}





\begin{flushleft}
(screening, sedimentation; biological aerobic and anaerobic
\end{flushleft}


\begin{flushleft}
process)-theory and design aspects; Biological processes (Nutrient
\end{flushleft}


\begin{flushleft}
and phosphorous removal); advanced wastewater treatmenttheory and design aspects; Air pollution (health effects, regulatory
\end{flushleft}


\begin{flushleft}
standards, dispersion; stacks, control systems); Municipal solid waste
\end{flushleft}


\begin{flushleft}
management; Noise pollution.
\end{flushleft}





\begin{flushleft}
CVL222 Soil Mechanics
\end{flushleft}


\begin{flushleft}
3 Credits (3-0-0)
\end{flushleft}


\begin{flushleft}
Origin and Classification of Soils; Phase Relationships; Effective Stress
\end{flushleft}


\begin{flushleft}
Principle; Effective Stress Under Hydrostatic and 1D flow; Permeability;
\end{flushleft}


\begin{flushleft}
Flow Through Soils--Laplace equation, flownets, seepage; Contaminant
\end{flushleft}


\begin{flushleft}
Transport; Compressibility; Consolidation; Terzaghi's 1D Consolidation
\end{flushleft}


\begin{flushleft}
Theory; Shear Strength; Drainage Conditions; Pore Water Pressure;
\end{flushleft}


\begin{flushleft}
Mohr's Circle; Failure Envelope and Strength Parameters; Factors
\end{flushleft}


\begin{flushleft}
Affecting Shear Strength; Critical State frame work; Behaviour of
\end{flushleft}


\begin{flushleft}
soils under cyclic loading, Liquefaction,; Compaction; Engineering
\end{flushleft}


\begin{flushleft}
properties of Natural soils, Compacted Soils and modified soils; Site
\end{flushleft}


\begin{flushleft}
Investigations; Soil deposits of India.
\end{flushleft}





\begin{flushleft}
CVP222 Soil Mechanics Lab
\end{flushleft}


\begin{flushleft}
1 Credit (0-0-2)
\end{flushleft}


\begin{flushleft}
Pre-requisites: CVL222 or concurrent with CVL222
\end{flushleft}


\begin{flushleft}
Visual Soil Classification; Water Content; Atterberg Limits; Grain Size
\end{flushleft}


\begin{flushleft}
Analysis; Specific Gravity; Permeability; standard proctor compaction
\end{flushleft}


\begin{flushleft}
test, consolidation test, site investigations and introduction to
\end{flushleft}


\begin{flushleft}
triaxial testing.
\end{flushleft}





\begin{flushleft}
CVL242 Structural Analysis I
\end{flushleft}


\begin{flushleft}
3 Credits (3-0-0)
\end{flushleft}


\begin{flushleft}
Pre-requisites: APL108
\end{flushleft}


\begin{flushleft}
General Concept of Static Equilibrium of Structures, Concept of Free
\end{flushleft}


\begin{flushleft}
Body Diagram, Analysis of Statically Determinate Trusses, Energy
\end{flushleft}


\begin{flushleft}
Methods for Determination of Joint Displacements - Castiliagno
\end{flushleft}


\begin{flushleft}
Theorem, Unit Load Method etc., Introduction to Analysis of Statically
\end{flushleft}


\begin{flushleft}
Indeterminate Trusses using Energy Methods, Analysis Statically
\end{flushleft}


\begin{flushleft}
Determinate Beams - Moment Area Theorem, Conjugate Beam
\end{flushleft}


\begin{flushleft}
Method, Maxwell Betti Theorem, Method of Superposition, Application
\end{flushleft}


\begin{flushleft}
of Energy Methods to Statically Determinate Beams and Rigid Frames,
\end{flushleft}


\begin{flushleft}
Solving Simple Indeterminate Beams Structures using Energy Methods,
\end{flushleft}


\begin{flushleft}
Analysis of Rolling Loads and Influence Line Diagram, Analysis of
\end{flushleft}


\begin{flushleft}
Arches and cable structures.
\end{flushleft}





\begin{flushleft}
CVP242 Structural Analysis Lab
\end{flushleft}


\begin{flushleft}
1 Credit (0-0-2)
\end{flushleft}


\begin{flushleft}
Pre-requisites: CVL242 or Concurrent with CVL242
\end{flushleft}


\begin{flushleft}
Determination of forces and displacements in statically determinate
\end{flushleft}


\begin{flushleft}
and indeterminate trusses, Influence Line Diagram for Trusses,
\end{flushleft}


\begin{flushleft}
Measurement of bending moment and shear forces in beams,
\end{flushleft}


\begin{flushleft}
Determination of Elastic Properties of Beams, Verification of the
\end{flushleft}


\begin{flushleft}
Moment Area Theorem, Maxwell Betti Theorem, Influence Line
\end{flushleft}


\begin{flushleft}
Diagram for Displacement, Support Reaction, Shear Force at an
\end{flushleft}


\begin{flushleft}
Intermediate Section and Bending Moment, Determination of Carry
\end{flushleft}


\begin{flushleft}
over Factor, Verification of Carry Over Factor, Determination of
\end{flushleft}


\begin{flushleft}
displacements in curved members, Analysis of Elastically Coupled
\end{flushleft}


\begin{flushleft}
Beams, Determination of horizontal reactions in two and three hinged
\end{flushleft}


\begin{flushleft}
arches, experiment on cable structures.
\end{flushleft}





\begin{flushleft}
CVL243 Reinforced Concrete Design
\end{flushleft}


\begin{flushleft}
3 Credits (3-0-0)
\end{flushleft}


\begin{flushleft}
Pre-requisites: CVL141
\end{flushleft}


\begin{flushleft}
Design Philosophy: Working stress and limit state design concepts;
\end{flushleft}


\begin{flushleft}
Design of and detailing of RC beam sections in flexure, shear, torsion
\end{flushleft}


\begin{flushleft}
and bond; Design for serviceability; Design of RC beams, One way and
\end{flushleft}


\begin{flushleft}
two way RC slabs, RC short and long columns, RC footings.
\end{flushleft}





\begin{flushleft}
CVP243 Materials and Structures Laboratory Concrete
\end{flushleft}


\begin{flushleft}
1.5 Credits (0-0-3)
\end{flushleft}


\begin{flushleft}
Testing of cement, testing of aggregates, mixture design and testing,
\end{flushleft}


\begin{flushleft}
non-destructive tests, testing of reinforcement, behaviour of reinforced
\end{flushleft}





166





\begin{flushleft}
\newpage
Civil Engineering
\end{flushleft}





\begin{flushleft}
concrete beams under flexure and torsion, behaviour of reinforced
\end{flushleft}


\begin{flushleft}
concrete slabs under uniform and point loads, behaviour of reinforced
\end{flushleft}


\begin{flushleft}
concrete columns under concentric and eccentric loads.
\end{flushleft}





\begin{flushleft}
CVL244 Construction Practices
\end{flushleft}


\begin{flushleft}
2 Credits (2-0-0)
\end{flushleft}


\begin{flushleft}
Pre-requisites: EC35
\end{flushleft}





\begin{flushleft}
CVP281 Hydraulics Lab
\end{flushleft}


\begin{flushleft}
1 Credit (0-0-2)
\end{flushleft}


\begin{flushleft}
Pre-requisites: CVL281 or Concurrent with CVL281
\end{flushleft}


\begin{flushleft}
Experiments on Open Channel Flow Hydraulics, Boundary Layer
\end{flushleft}


\begin{flushleft}
Theory, Pipe flow, Sediment transport.
\end{flushleft}





\begin{flushleft}
Introduction and role of technologies, Construction technologies
\end{flushleft}


\begin{flushleft}
in RC Buildings for Reinforcement, Formwork, and concreting
\end{flushleft}


\begin{flushleft}
activities, Excavation and Concreting equipment, Formwork
\end{flushleft}


\begin{flushleft}
material and Design Concepts, Formwork system for Foundations,
\end{flushleft}


\begin{flushleft}
walls, columns, slab and beams and their design, Flying Formwork
\end{flushleft}


\begin{flushleft}
such as Table form, tunnel form. Slipform, temporary structures
\end{flushleft}


\begin{flushleft}
failure, Determining construction loads and ensuring safety of
\end{flushleft}


\begin{flushleft}
slabs during construction of high rise buildings- shoring, reshoring,
\end{flushleft}


\begin{flushleft}
preshroing and backshoring technology, Top down construction
\end{flushleft}


\begin{flushleft}
technology for high rise and underground construction, Bridge
\end{flushleft}


\begin{flushleft}
construction including segmental construction, incremental
\end{flushleft}


\begin{flushleft}
construction and push launching techniques, Prefab construction.
\end{flushleft}





\begin{flushleft}
CVL245 Construction Management
\end{flushleft}


\begin{flushleft}
2 Credits (2-0-0)
\end{flushleft}


\begin{flushleft}
Pre-requisites: EC35
\end{flushleft}


\begin{flushleft}
Introduction to construction projects, stakeholders, phases in a project,
\end{flushleft}


\begin{flushleft}
Cost estimation from clients perspective, Project selection using
\end{flushleft}


\begin{flushleft}
time value of money concept, construction contract, cost estimate --
\end{flushleft}


\begin{flushleft}
contractors perspective, Project planning and network analysis-PERT,
\end{flushleft}


\begin{flushleft}
CPM, and Precedence Network, Resource scheduling, Time Cost trade
\end{flushleft}


\begin{flushleft}
off, Time -cost monitoring and control using S-curve and earned
\end{flushleft}


\begin{flushleft}
value analysis, Construction claims and disputes, and introduction to
\end{flushleft}


\begin{flushleft}
construction quality and safety.
\end{flushleft}





\begin{flushleft}
CVL261 Introduction to Transportation Engineering
\end{flushleft}


\begin{flushleft}
3 Credits (3-0-0)
\end{flushleft}


\begin{flushleft}
Pre-requisites: CVL111
\end{flushleft}


\begin{flushleft}
Transportation systems and their classification; Role of transportation
\end{flushleft}


\begin{flushleft}
with respect to socio-economic conditions; Transportation planning
\end{flushleft}


\begin{flushleft}
process; Road user and the vehicle; Geometric design of roads:
\end{flushleft}


\begin{flushleft}
horizontal alignment, vertical alignment, cross-section elements;
\end{flushleft}


\begin{flushleft}
Relevant geometric design standards; Pavements: flexible and
\end{flushleft}


\begin{flushleft}
rigid; Characterization of pavement materials; Analysis and design
\end{flushleft}


\begin{flushleft}
of pavement systems; Pavement design specifications; Pavement
\end{flushleft}


\begin{flushleft}
construction process; Pavement performance; Traffic engineering: Traffic
\end{flushleft}


\begin{flushleft}
characteristics; Fundamental relationships; Theories of traffic flow;
\end{flushleft}


\begin{flushleft}
Intersection design; Design of traffic signs and signals; Highway capacity.
\end{flushleft}





\begin{flushleft}
CVP261 Transportation Engineering Lab
\end{flushleft}


\begin{flushleft}
1 Credit (0-0-2)
\end{flushleft}


\begin{flushleft}
Pre-requisites: CVL261 or Concurrent with CVL261
\end{flushleft}


\begin{flushleft}
Introduction to material behavior; Characterization of materials used
\end{flushleft}


\begin{flushleft}
in pavement construction: soil, aggregate, asphalt, asphalt concrete;
\end{flushleft}


\begin{flushleft}
Introduction to traffic survey methodologies; Traffic surveys: speed
\end{flushleft}


\begin{flushleft}
studies, intersection study.
\end{flushleft}





\begin{flushleft}
CVL282 Engineering Hydrology
\end{flushleft}


\begin{flushleft}
4 Credits (3-0-2)
\end{flushleft}


\begin{flushleft}
Pre-requisites: APL107
\end{flushleft}


\begin{flushleft}
Hydrologic Cycle, Processes and Applied Methodologies. Rainfall;
\end{flushleft}


\begin{flushleft}
Evapotranspiration; Infiltration;
\end{flushleft}


\begin{flushleft}
Groundwater: Occurrence, Movement, Governing equations, Well
\end{flushleft}


\begin{flushleft}
hydraulics.
\end{flushleft}


\begin{flushleft}
Runoff: Hydrograph, Unit Hydrographs; Streamflow measurement.
\end{flushleft}


\begin{flushleft}
Flood Routing: Hydrological routing for reservoirs and channels.
\end{flushleft}


\begin{flushleft}
Frequency Analysis.
\end{flushleft}





\begin{flushleft}
CVL284 Fundamentals of Geographic Information
\end{flushleft}


\begin{flushleft}
Systems
\end{flushleft}


\begin{flushleft}
3 Credits (2-0-2)
\end{flushleft}


\begin{flushleft}
Pre-requisites: COL100
\end{flushleft}


\begin{flushleft}
What is GIS. Geographic concepts for GIS. Spatial relationships,
\end{flushleft}


\begin{flushleft}
topology, spatial patterns, spatial interpolation. Data storage, data
\end{flushleft}


\begin{flushleft}
structure, non-spatial database models. Populating GIS, digitizing data
\end{flushleft}


\begin{flushleft}
exchange, data conversion. Spatial data models, Raster and Vector
\end{flushleft}


\begin{flushleft}
data structures and algorithms. Digital Elevation Models (DEM) and
\end{flushleft}


\begin{flushleft}
their application. Triangulated Irregular Network (TIN) model. GIS
\end{flushleft}


\begin{flushleft}
application areas, Spatial analysis, quantifying relationships, spatial
\end{flushleft}


\begin{flushleft}
statistics, spatial search. Decision making in GIS context.
\end{flushleft}





\begin{flushleft}
CVL311 Industrial Waste Management
\end{flushleft}


\begin{flushleft}
3 Credits (3-0-0)
\end{flushleft}


\begin{flushleft}
Pre-requisites: CVL212
\end{flushleft}


\begin{flushleft}
Industrial waste types and characteristics; levels of environmental
\end{flushleft}


\begin{flushleft}
pollution due to industrial wastes; health issues due to industrial
\end{flushleft}


\begin{flushleft}
wastes; ecological and human health risk assessment due to
\end{flushleft}


\begin{flushleft}
industrial wastes; waste characterization methods; treatment
\end{flushleft}


\begin{flushleft}
methods-conventional and recent trends (for air, water, soil media);
\end{flushleft}


\begin{flushleft}
Prevention versus control of industrial pollution; hierarchy of priorities
\end{flushleft}


\begin{flushleft}
for industrial waste management; comparison of real-life industrial
\end{flushleft}


\begin{flushleft}
waste management practices (ex: superfund remedial sites, etc.);
\end{flushleft}


\begin{flushleft}
economics of industrial waste management and sustainability issues;
\end{flushleft}


\begin{flushleft}
environmental rules and regulations; clean up goals;disposal/reuse of
\end{flushleft}


\begin{flushleft}
treated wastes; Source reduction and control of industrial water and
\end{flushleft}


\begin{flushleft}
air pollution; Minimization of industrial solid and hazardous waste;
\end{flushleft}


\begin{flushleft}
Waste management case studies from various industries.
\end{flushleft}





\begin{flushleft}
CVL312 Environmental Assessment Methodologies
\end{flushleft}


\begin{flushleft}
3 Credits (3-0-0)
\end{flushleft}


\begin{flushleft}
Pre-requisites: CVL212
\end{flushleft}


\begin{flushleft}
Environmental issues related to developmental activities: Nature
\end{flushleft}


\begin{flushleft}
and characteristics of environmental impacts of urban and industrial
\end{flushleft}


\begin{flushleft}
developments.
\end{flushleft}





\begin{flushleft}
CVL281 Hydraulics
\end{flushleft}


\begin{flushleft}
4 Credits (3-1-0)
\end{flushleft}


\begin{flushleft}
Pre-requisites: APL107
\end{flushleft}


\begin{flushleft}
Open Channel Flow: Channel Characteristics and parameters, Uniform
\end{flushleft}


\begin{flushleft}
flow, Critical flow, Specific Energy concepts, Gradually Varied Flows,
\end{flushleft}


\begin{flushleft}
Rapidly Varied flow with special reference to hydraulic jump, Unsteady
\end{flushleft}


\begin{flushleft}
flow in open channels.
\end{flushleft}


\begin{flushleft}
Boundary Layer Theory: Navier Stokes Equation, Boundary Layer
\end{flushleft}


\begin{flushleft}
Equation in 2-dimension, Boundary layer characteristics, Integral
\end{flushleft}


\begin{flushleft}
Momentum equation, onset of turbulence, properties of turbulent flow,
\end{flushleft}


\begin{flushleft}
skin friction,application of drag, lift and circulation to hydraulic problems.
\end{flushleft}


\begin{flushleft}
Pipe Flow: Laminar and Turbulent flow in Smooth and Rough pipes,
\end{flushleft}


\begin{flushleft}
pipe network analysis, Losses in pipes
\end{flushleft}


\begin{flushleft}
Fluvial Hydraulics: Settling velocity, Incipient motion, Resistance to
\end{flushleft}


\begin{flushleft}
flow and bed forms, Sediment load and transport.
\end{flushleft}





\begin{flushleft}
Linkages between technology, environmental quality, economic gain,
\end{flushleft}


\begin{flushleft}
and societal goals.
\end{flushleft}


\begin{flushleft}
Environmental indices and indicators for describing affected
\end{flushleft}


\begin{flushleft}
environment. Methodologies and environmental systems modeling
\end{flushleft}


\begin{flushleft}
tools for prediction and assessment of impacts on environmental
\end{flushleft}


\begin{flushleft}
quality (surface water, ground water, air, soil).
\end{flushleft}


\begin{flushleft}
Monitoring and control of undesirable environmental implications.
\end{flushleft}


\begin{flushleft}
Environmental cost benefit analysis. Decision methods for evaluation
\end{flushleft}


\begin{flushleft}
of environmentally sound alternatives.
\end{flushleft}


\begin{flushleft}
Environmental health and safety: Basic concepts of environmental
\end{flushleft}


\begin{flushleft}
risk and definitions; Hazard identification procedures; Consequence
\end{flushleft}


\begin{flushleft}
analysis and modeling (discharge models, dispersion models, fire and
\end{flushleft}


\begin{flushleft}
explosion models, effect models etc.).
\end{flushleft}





167





\begin{flushleft}
\newpage
Civil Engineering
\end{flushleft}





\begin{flushleft}
Emerging tools for environmental management: Environmental
\end{flushleft}


\begin{flushleft}
Management Systems, Environmentally sound technology transfer,
\end{flushleft}


\begin{flushleft}
emission trading, international resource sharing issues, climate change,
\end{flushleft}


\begin{flushleft}
international environmental treaties and protocols. Case studies.
\end{flushleft}





\begin{flushleft}
CVL313 Air and Noise Pollution
\end{flushleft}


\begin{flushleft}
3 Credits (3-0-0)
\end{flushleft}


\begin{flushleft}
Pre-requisites: CVL212
\end{flushleft}


\begin{flushleft}
Definitions, source and types of air and noise pollution, physical and
\end{flushleft}


\begin{flushleft}
chemical properties of air pollutants, secondary pollutants formation,
\end{flushleft}


\begin{flushleft}
instrument design and industrial application, gas phase adsorption and
\end{flushleft}


\begin{flushleft}
biofiltration, carbon Credit, global warming potential, case studies,
\end{flushleft}


\begin{flushleft}
data analysis, interpretation.
\end{flushleft}





\begin{flushleft}
CVL321 Geotechnical Engineering
\end{flushleft}


\begin{flushleft}
4 Credits (3-1-0)
\end{flushleft}


\begin{flushleft}
Pre-requisites: CVL222
\end{flushleft}


\begin{flushleft}
Foundations: types, selection and design considerations; Bearing
\end{flushleft}


\begin{flushleft}
capacity of shallow foundations: Terzaghi theory, factors affecting;
\end{flushleft}


\begin{flushleft}
Bearing capacity of deep foundations: single pile analysis, pile tests,
\end{flushleft}


\begin{flushleft}
pile driving formula, group capacity, introduction to laterally loaded
\end{flushleft}


\begin{flushleft}
piles; Settlement of shallow and deep foundations: stress distribution,
\end{flushleft}


\begin{flushleft}
immediate and consolidation settlements; Slope stability analysis:
\end{flushleft}


\begin{flushleft}
infinite slopes, method of slices, Swedish circle method; Earth dams:
\end{flushleft}


\begin{flushleft}
types and design aspects; Earth pressure analysis: Rankine and
\end{flushleft}


\begin{flushleft}
Coulomb methods; Earth retaining structures: types, design aspects,
\end{flushleft}


\begin{flushleft}
underground structures; Earthquake geotechnics: evaluation of
\end{flushleft}


\begin{flushleft}
liquefaction potential, seismic slope stability, seismic bearing capacity;
\end{flushleft}


\begin{flushleft}
Machine foundations: types, analysis, design procedure; Ground
\end{flushleft}


\begin{flushleft}
improvement techniques: types, deep stabilization, anchorage,
\end{flushleft}


\begin{flushleft}
grouting; Geosynthetics: types, functions, properties; reinforced soil
\end{flushleft}


\begin{flushleft}
walls; Geoenvironment: Landfills - types, liner, cover, stability; Ash
\end{flushleft}


\begin{flushleft}
ponds - stage raising, design aspects.
\end{flushleft}





\begin{flushleft}
CVL344 Construction Project Management
\end{flushleft}


\begin{flushleft}
3 Credits (3-0-0)
\end{flushleft}


\begin{flushleft}
Pre-requisites: CVL245
\end{flushleft}


\begin{flushleft}
Additional network analysis- Ladder Network, LoB,etc., Time constrained
\end{flushleft}


\begin{flushleft}
Resource allocation and resource constrained problems, Time Cost
\end{flushleft}


\begin{flushleft}
trade off, project updating and control using EVM, Construction
\end{flushleft}


\begin{flushleft}
contracts and its types, tendering procedure, estimation and fixing of
\end{flushleft}


\begin{flushleft}
markup, bidding models, claims compensation and disputes, dispute
\end{flushleft}


\begin{flushleft}
resolution models, FIDIC contracts, Linear programming, Problems in
\end{flushleft}


\begin{flushleft}
construction, Formulation, Graphical solution, Simplex method, Dual
\end{flushleft}


\begin{flushleft}
problem, sensitivity analysis and their application to Civil engineering,
\end{flushleft}


\begin{flushleft}
Transportation Assignment problems and their applications
\end{flushleft}





\begin{flushleft}
CVL361 Introduction to Railway Engineering
\end{flushleft}


\begin{flushleft}
3 Credits (3-0-0)
\end{flushleft}


\begin{flushleft}
Pre-requisites: CVL261
\end{flushleft}


\begin{flushleft}
History: Indian railways, international perspective; Railway track
\end{flushleft}


\begin{flushleft}
gauge: factors affecting gauge choice, multi gauge; New project
\end{flushleft}


\begin{flushleft}
planning and surveys; Alignment of railway track; Structure of railway
\end{flushleft}


\begin{flushleft}
track: rails, sleepers, ballast, subgrade, track fittings; Structural
\end{flushleft}


\begin{flushleft}
design of railway track: stresses, creep; Geometric design of rail
\end{flushleft}


\begin{flushleft}
track: gradients, curves, superelevation; Locomotives and rolling
\end{flushleft}


\begin{flushleft}
stock: resistance and tractive power; Points and crossings; Railway
\end{flushleft}


\begin{flushleft}
stations and yards; Traffic control; Signalling and interlocking; Public
\end{flushleft}


\begin{flushleft}
rail transportation in metros.
\end{flushleft}





\begin{flushleft}
CVL381 Design of Hydraulic Structures
\end{flushleft}


\begin{flushleft}
4 Credits (3-0-2)
\end{flushleft}


\begin{flushleft}
Pre-requisites: CVL281 and CVL282
\end{flushleft}


\begin{flushleft}
Input studies.
\end{flushleft}





\begin{flushleft}
Vane shear test, Direct shear test, Specimen preparation, Unconfined
\end{flushleft}


\begin{flushleft}
compression test, Unconsolidated undrained test, Consolidated drained
\end{flushleft}


\begin{flushleft}
test, Consolidated undrained test with pore water pressure measurement.
\end{flushleft}





\begin{flushleft}
Storage structures: Dams and reservoirs, Different types of dams and
\end{flushleft}


\begin{flushleft}
selection of suitable type and dam site, Gravity dam, Embankment
\end{flushleft}


\begin{flushleft}
dams. Diversion works: Design concepts for irrigation structures on
\end{flushleft}


\begin{flushleft}
permeable foundations, Design of Weirs and barrages. Design of
\end{flushleft}


\begin{flushleft}
energy dissipation devices. Canals: canal layout, Regime canal design,
\end{flushleft}


\begin{flushleft}
Rigid boundary canal design. Design of canal falls. Design of cross
\end{flushleft}


\begin{flushleft}
drainage works. Design of head regulator, cross regulator and canal
\end{flushleft}


\begin{flushleft}
outlet structures.
\end{flushleft}





\begin{flushleft}
CVL341 Structural Analysis II
\end{flushleft}


\begin{flushleft}
3 Credits (3-0-0)
\end{flushleft}


\begin{flushleft}
Pre-requisites: CVL242
\end{flushleft}





\begin{flushleft}
CVL382 Groundwater
\end{flushleft}


\begin{flushleft}
2 Credits (2-0-0)
\end{flushleft}


\begin{flushleft}
Pre-requisites: CVL282 or EC 75
\end{flushleft}





\begin{flushleft}
Determinacy and stability; Method of consistent deformations-Matrix
\end{flushleft}


\begin{flushleft}
formulation, Application to beams, trusses and frames; Slope-deflection
\end{flushleft}


\begin{flushleft}
method and Moment-distribution method- Beams and frames with
\end{flushleft}


\begin{flushleft}
uneven loading, support settlements, dealing with symmetry and
\end{flushleft}


\begin{flushleft}
anti-symmetry, Non-sway and sway frames; Matrix stiffness method;
\end{flushleft}


\begin{flushleft}
Matrix flexibility method; Energy methods; Approximate methods of
\end{flushleft}


\begin{flushleft}
analysis; Direct stiffness method for computer applications including
\end{flushleft}


\begin{flushleft}
computational aspects and MATLAB Assignments.
\end{flushleft}





\begin{flushleft}
Introduction, importance and occurrence of groundwater; Aquifers and
\end{flushleft}


\begin{flushleft}
groundwater scenario in India; Surface and subsurface investigation of
\end{flushleft}


\begin{flushleft}
groundwater; Construction, development and maintenance of wells;
\end{flushleft}


\begin{flushleft}
Flow through porous media, Darcy's law, regional flow; Well hydraulics;
\end{flushleft}


\begin{flushleft}
Groundwater management.
\end{flushleft}





\begin{flushleft}
CVP321 Geotechnical Engineering Lab
\end{flushleft}


\begin{flushleft}
1 Credit (0-0-2)
\end{flushleft}


\begin{flushleft}
Pre-requisites: CVL321 or Concurrent with CVL321
\end{flushleft}





\begin{flushleft}
CVL342 Design of Steel Structures
\end{flushleft}


\begin{flushleft}
3 Credits (3-0-0)
\end{flushleft}


\begin{flushleft}
Pre-requisites: CVL242
\end{flushleft}


\begin{flushleft}
Structural steel and properties, Design pholisophy-Working stress and
\end{flushleft}


\begin{flushleft}
limit state; Connection types- Riveted, bolted and welded; Design
\end{flushleft}


\begin{flushleft}
of tension, compression and flexural members; Design of members
\end{flushleft}


\begin{flushleft}
subjected to combined loadings-Axial and bending, Torsion, Biaxial
\end{flushleft}


\begin{flushleft}
bending; Column bases, Gantry and plate girders; Roof trusses;
\end{flushleft}


\begin{flushleft}
Plastic design; Introduction to stability concepts, Design of shedtype structures.
\end{flushleft}





\begin{flushleft}
CVP342 Materials and Structures Laboratory-Steel
\end{flushleft}


\begin{flushleft}
1 Credit (0-0-2)
\end{flushleft}


\begin{flushleft}
Basic properties of structural steel; Tensile stress-strain behaviour;
\end{flushleft}


\begin{flushleft}
Buckling of slender columns, Flexural testing of beams; Torsional
\end{flushleft}


\begin{flushleft}
behaviour of beams, Unsymmetrical bending; Lateral-torsional
\end{flushleft}


\begin{flushleft}
buckling; Flexural-torsional buckling; Connection behaviour; Tensionfield action in plate girders.
\end{flushleft}





\begin{flushleft}
CVL383 Water Resources Systems
\end{flushleft}


\begin{flushleft}
2 Credits (2-0-0)
\end{flushleft}


\begin{flushleft}
Pre-requisites: CVL282
\end{flushleft}


\begin{flushleft}
Water Resources Planning Purposes and Objectives; Multi-component,
\end{flushleft}


\begin{flushleft}
multi-user, multi-objective and multi-purpose attributes of an
\end{flushleft}


\begin{flushleft}
Integrated Water Resources System; Economic basis for selection of a
\end{flushleft}


\begin{flushleft}
Plan Alternative; Introduction to Linear Programming and applications
\end{flushleft}


\begin{flushleft}
in Water Resources Engineering; Linear, Deterministic Integrated Water
\end{flushleft}


\begin{flushleft}
Resources Management Model on River Basin Scale.
\end{flushleft}





\begin{flushleft}
CVL384 Urban Hydrology
\end{flushleft}


\begin{flushleft}
2 Credits (2-0-0)
\end{flushleft}


\begin{flushleft}
Pre-requisites: CVL282 or EC 75
\end{flushleft}


\begin{flushleft}
Distinctive characteristics of natural and urban watersheds; Urban
\end{flushleft}


\begin{flushleft}
Heat Island; Changes in rainfall, infiltration and runoff characteristics
\end{flushleft}


\begin{flushleft}
in urban watershed; IDF relationship and its adaptation for urban
\end{flushleft}


\begin{flushleft}
settings; Adjusting runoff record for urbanization; Stormwater
\end{flushleft}


\begin{flushleft}
Management and rainwater harvesting; Urban drainage: layout,
\end{flushleft}


\begin{flushleft}
structures, flooding and control, combined sewer overflows,
\end{flushleft}


\begin{flushleft}
sedimentation; Management of stormwater.
\end{flushleft}





168





\begin{flushleft}
\newpage
Civil Engineering
\end{flushleft}





\begin{flushleft}
CVL385 Frequency Analysis in Hydrology
\end{flushleft}


\begin{flushleft}
2 Credits (2-0-0)
\end{flushleft}


\begin{flushleft}
Pre-requisites: CVL282
\end{flushleft}


\begin{flushleft}
Concepts of probability in Hydrology, Random events, Random
\end{flushleft}


\begin{flushleft}
variables; moments and expectations; Common probabilistic
\end{flushleft}


\begin{flushleft}
distributions; goodness of fit tests; Stochastic processes.
\end{flushleft}





\begin{flushleft}
CVL386 Fundamentals of Remote Sensing
\end{flushleft}


\begin{flushleft}
3 Credits (2-0-2)
\end{flushleft}


\begin{flushleft}
Pre-requisites: EC 75
\end{flushleft}


\begin{flushleft}
What is Remote Sensing? Historical development of remote sensing,
\end{flushleft}


\begin{flushleft}
Remote sensing components, Data collection and transmission,
\end{flushleft}


\begin{flushleft}
Sensors and satellite imageries, Electromagnetic energy and spectrum,
\end{flushleft}


\begin{flushleft}
Wavebands, Interactions of electromagnetic energy with atmosphere
\end{flushleft}


\begin{flushleft}
and earth's surface, radiometric quantities, Photogrammetry and aerial
\end{flushleft}


\begin{flushleft}
photography, Vertical and tilted photographs, Photographic materials,
\end{flushleft}


\begin{flushleft}
Photo-processes, Stereoscopic viewing, fly view, Aerial mosaics,
\end{flushleft}


\begin{flushleft}
Various satellite systems and monitoring programs, Data Products,
\end{flushleft}


\begin{flushleft}
Satellite data, Data formats, Data acquisition for natural resources
\end{flushleft}


\begin{flushleft}
management and weather forecast, Random errors and least square
\end{flushleft}


\begin{flushleft}
adjustment, Coordinate transformation, Photographic interpretation,
\end{flushleft}


\begin{flushleft}
Image processing, Potential applications of remote sensing in diverse
\end{flushleft}


\begin{flushleft}
areas and decision making, Integrated use of remote sensing and
\end{flushleft}


\begin{flushleft}
GIS, Case studies.
\end{flushleft}





\begin{flushleft}
CVD411 B.Tech. Project Part-I
\end{flushleft}


\begin{flushleft}
4 Credits (0-0-8)
\end{flushleft}





\begin{flushleft}
foundations; Elastic homogeneous half space and lumped parameter
\end{flushleft}


\begin{flushleft}
solutions; Vibration isolation; Codal provisions; Causes of Earthquakes;
\end{flushleft}


\begin{flushleft}
Strong Ground Motion: Measurement, characterization and estimation;
\end{flushleft}


\begin{flushleft}
Amplification theory and ground response analysis; Liquefaction of soil
\end{flushleft}


\begin{flushleft}
and its remediation; Seismic slope stability; Seismic bearing capacity
\end{flushleft}


\begin{flushleft}
and earth pressures
\end{flushleft}





\begin{flushleft}
CVL424 Environmental Geotechniques \& Geosynthetics
\end{flushleft}


\begin{flushleft}
3 Credits (3-0-0)
\end{flushleft}


\begin{flushleft}
Pre-requisites: CVL321
\end{flushleft}


\begin{flushleft}
Causes and effects of subsurface contamination; Waste disposal on
\end{flushleft}


\begin{flushleft}
land; Characteristics of solid wastes; Waste Containment Principles;
\end{flushleft}


\begin{flushleft}
Types of landfills; Planning of landfills; Design of liners and covers for
\end{flushleft}


\begin{flushleft}
landfills; Environmental Monitoring around landfills; Detection, control
\end{flushleft}


\begin{flushleft}
and remediation of subsurface contamination; Geotechnical re-use of
\end{flushleft}


\begin{flushleft}
solid waste materials.
\end{flushleft}


\begin{flushleft}
Types of geosynthetics;Manufacturing; Functions; Testing and evaluation;
\end{flushleft}


\begin{flushleft}
Designing with geotextiles, geogrids, geonets and geomembranes.
\end{flushleft}





\begin{flushleft}
CVL431 Design of Foundations \& Retaining Structures
\end{flushleft}


\begin{flushleft}
3 Credits (3-0-0)
\end{flushleft}


\begin{flushleft}
Pre-requisites: CVL321
\end{flushleft}


\begin{flushleft}
Shallow Foundations: Bearing Capacity, Generalized bearing capacity
\end{flushleft}


\begin{flushleft}
theory, Empirical methods, Layered soil, Foundations on or near slopes,
\end{flushleft}


\begin{flushleft}
Settlement of foundations, codal provisions.
\end{flushleft}


\begin{flushleft}
Pile Foundations: Types and their selection, Ultimate load of individual
\end{flushleft}


\begin{flushleft}
piles in compressive, uplift, and lateral loading, Pile load tests,
\end{flushleft}


\begin{flushleft}
Downdrag, Pile groups. Caissons. Codal provisions.
\end{flushleft}





\begin{flushleft}
CVD412 B.Tech. Project Part-II
\end{flushleft}


\begin{flushleft}
6 Credits (0-0-12)
\end{flushleft}





\begin{flushleft}
Earth Retaining Structures: Types, Earth pressures, Design of rigid,
\end{flushleft}


\begin{flushleft}
flexible and reinforced soil retaining walls, braced excavations, and
\end{flushleft}


\begin{flushleft}
ground anchors for retaning walls.
\end{flushleft}





\begin{flushleft}
CVL421 Ground Engineering
\end{flushleft}


\begin{flushleft}
3 Credits (3-0-0)
\end{flushleft}


\begin{flushleft}
Pre-requisites: CVL321
\end{flushleft}





\begin{flushleft}
Introduction to design of foundation for dynamic loads.
\end{flushleft}





\begin{flushleft}
Planning of investigation programmes, Geophysical methods. Methods
\end{flushleft}


\begin{flushleft}
of site investigations: Direct methods, semi-direct methods and
\end{flushleft}


\begin{flushleft}
indirect methods, Drilling methods. Boring in soils and rocks, Methods
\end{flushleft}


\begin{flushleft}
of stabilizing the bore holes, measurement of water table, field
\end{flushleft}


\begin{flushleft}
record. Principles of compaction, Laboratory compaction,Engineering
\end{flushleft}


\begin{flushleft}
behaviour of compacted clays, Field compaction techniques- static,
\end{flushleft}


\begin{flushleft}
vibratory, impact, Compaction control. Shallow stabilization with
\end{flushleft}


\begin{flushleft}
additives: Lime, fly ash, cement and other chemicals and bitumen; Deep
\end{flushleft}


\begin{flushleft}
Stabilization: sand column, stone column, sand drains, prefabricated
\end{flushleft}


\begin{flushleft}
drains, electroosmosis, lime column. soil-lime column, blasting.
\end{flushleft}


\begin{flushleft}
Grouting : permeation, compaction and jet. Vibro-floatation, dynamic
\end{flushleft}


\begin{flushleft}
compaction, thermal freezing. Dewatering systems. Functions and
\end{flushleft}


\begin{flushleft}
applications of geosynthetics -- geotextiles, geogrids, geomembranes;
\end{flushleft}


\begin{flushleft}
soil reinforcement using strips, bars and geosynthetics; soil nailing
\end{flushleft}


\begin{flushleft}
and ground anchors, Earthmoving machines and earthwork principles,
\end{flushleft}


\begin{flushleft}
Piling and diaphragm wall construction, Tunneling methods in soils,
\end{flushleft}


\begin{flushleft}
Hydraulic barriers and containment systems for waste disposal in soil,
\end{flushleft}


\begin{flushleft}
Control and remediation of soil contamination.
\end{flushleft}





\begin{flushleft}
CVL422 Rock Engineering
\end{flushleft}


\begin{flushleft}
3 Credits (3-0-0)
\end{flushleft}


\begin{flushleft}
Pre-requisites: CVL321
\end{flushleft}


\begin{flushleft}
Geological classification, rock and rock mass classification, strength
\end{flushleft}


\begin{flushleft}
and deformation behaviour of rocks, pore pressures, failure criteria,
\end{flushleft}


\begin{flushleft}
laboratory and field testing, measurement of in-situ stresses and
\end{flushleft}


\begin{flushleft}
strains, stability of rock slopes and foundations, design of underground
\end{flushleft}


\begin{flushleft}
structures, improvement of in situ properties of rock masses and
\end{flushleft}


\begin{flushleft}
support measures.
\end{flushleft}





\begin{flushleft}
CVL423 Soil Dynamics
\end{flushleft}


\begin{flushleft}
3 Credits (3-0-0)
\end{flushleft}


\begin{flushleft}
Pre-requisites: CVL321
\end{flushleft}


\begin{flushleft}
Engineering problems involving soil dynamics; Role of inertia; Theory of
\end{flushleft}


\begin{flushleft}
Vibrations: Single and two-degree freedom systems; Wave propagation
\end{flushleft}


\begin{flushleft}
in elastic media; Soil behaviour under cyclic/dynamic loading; Small
\end{flushleft}


\begin{flushleft}
and large strain dynamic properties of soils; Design criteria for machine
\end{flushleft}





\begin{flushleft}
CVL432 Stability of Slopes
\end{flushleft}


\begin{flushleft}
2 Credits (2-0-0)
\end{flushleft}


\begin{flushleft}
Pre-requisites: CVL321
\end{flushleft}


\begin{flushleft}
Slope Stability: Short term and long term stabilities; Limit equilibrium
\end{flushleft}


\begin{flushleft}
methods; Infinite slopes; Finite height slopes - Swedish method,
\end{flushleft}


\begin{flushleft}
Bishop's simplified method, Stability charts; Conditions of analysis
\end{flushleft}


\begin{flushleft}
- steady state, end of construction, sudden draw down conditions;
\end{flushleft}


\begin{flushleft}
Factor of safety; Codal provisions; Earthquake effects. Seepage
\end{flushleft}


\begin{flushleft}
Analysis: Types of flow; Laplace equation; Flownet in isotropic,
\end{flushleft}


\begin{flushleft}
anisotropic and layered media; Entrance-exit conditions; Theoretical
\end{flushleft}


\begin{flushleft}
solutions; Determination of phreatic line. Earth Dams: Introduction;
\end{flushleft}


\begin{flushleft}
Factors influencing design; Design of components; Instrumentation.
\end{flushleft}


\begin{flushleft}
Reinforced Slopes: Steep slopes; Embankments on soft soils;
\end{flushleft}


\begin{flushleft}
Reinforcement design.
\end{flushleft}





\begin{flushleft}
CVL433 FEM in Geotechnical Engineering
\end{flushleft}


\begin{flushleft}
3 Credits (3-0-0)
\end{flushleft}


\begin{flushleft}
Pre-requisites: CVL321
\end{flushleft}


\begin{flushleft}
Steps in FEM. Stress-deformation analysis: One dimensional, Two
\end{flushleft}


\begin{flushleft}
dimensional and Three-dimensional formulations. Discretization of
\end{flushleft}


\begin{flushleft}
a Continuum, Elements, Strains, Stresses, Constitutive, Relations,
\end{flushleft}


\begin{flushleft}
Hooke's Law, Formulation of Stiffness Matrix, Boundary Conditions,
\end{flushleft}


\begin{flushleft}
Solution Algorithms.
\end{flushleft}


\begin{flushleft}
Settlement Analysis, 2-D elastic solutions for homogeneous, isotropic
\end{flushleft}


\begin{flushleft}
medium, Steady Seepage Analysis: Finite element solutions of
\end{flushleft}


\begin{flushleft}
Laplace's equation, Consolidation Analysis: Terzaghi consolidation
\end{flushleft}


\begin{flushleft}
problem, Choice of Soil Properties for Finite Element Analysis,
\end{flushleft}


\begin{flushleft}
Introduction to PHASE2.
\end{flushleft}





\begin{flushleft}
CVP434 Geotechnical Design Studio
\end{flushleft}


\begin{flushleft}
2 Credits (0-0-4)
\end{flushleft}


\begin{flushleft}
Pre-requisites: CVL321
\end{flushleft}


\begin{flushleft}
Seepage analysis through an earth dam. Slope stability analysis of a
\end{flushleft}


\begin{flushleft}
dam. Settlement analysis of shallow and deep foundations; Analysis
\end{flushleft}


\begin{flushleft}
and design of retaining structures; Analysing the structural forces in
\end{flushleft}


\begin{flushleft}
a tunnel lining.
\end{flushleft}





169





\begin{flushleft}
\newpage
Civil Engineering
\end{flushleft}





\begin{flushleft}
CVL435 Underground Structures
\end{flushleft}


\begin{flushleft}
2 Credits (2-0-0)
\end{flushleft}


\begin{flushleft}
Pre-requisites: CVL321
\end{flushleft}


\begin{flushleft}
Overlaps with: CVL713
\end{flushleft}





\begin{flushleft}
CVL461 Logistics and Freight Transport
\end{flushleft}


\begin{flushleft}
3 Credits (3-0-0)
\end{flushleft}


\begin{flushleft}
Overlaps with: SML843
\end{flushleft}


\begin{flushleft}
Pre-requisites: CVL261 or Instructor's permission
\end{flushleft}





\begin{flushleft}
Types and classification of underground structures, Functional aspects,
\end{flushleft}


\begin{flushleft}
Sizes and shapes, Support systems, Design methodology.
\end{flushleft}





\begin{flushleft}
Evolution of freight and logistics; Interrelationships between
\end{flushleft}


\begin{flushleft}
society, environment and freight transport; Survey methodologies
\end{flushleft}


\begin{flushleft}
to understand freight movement; Cost measurement: Production,
\end{flushleft}


\begin{flushleft}
Holding, Transportation, Handling; Effect of internal and external
\end{flushleft}


\begin{flushleft}
variables on cost; Demand forecasting; Inventory planning and
\end{flushleft}


\begin{flushleft}
management; Transportation and distribution network: Design,
\end{flushleft}


\begin{flushleft}
Development, Management; Ware house operations; Pricing:
\end{flushleft}


\begin{flushleft}
Perishable, seasonal demand, uncertainty issues; Vehicle routing: Oneto-one distribution, One-to-many distribution, Shortest path algorithm,
\end{flushleft}


\begin{flushleft}
Quickest time algorithm; Logistics information system; Designing and
\end{flushleft}


\begin{flushleft}
planning transportation networks; Multi-modal transportation issues.
\end{flushleft}





\begin{flushleft}
Stresses- deformation analysis of openings (circular, elliptical,
\end{flushleft}


\begin{flushleft}
spherical, ellipsoidal) using analytical and numerical methods
\end{flushleft}


\begin{flushleft}
Design of underground structures using analytical methods, empirical
\end{flushleft}


\begin{flushleft}
methods and observational methods, Rock support interaction analysis,
\end{flushleft}


\begin{flushleft}
NATM Hydraulic tunnels, Shafts, Tunnel portals, Metro tunnels.
\end{flushleft}





\begin{flushleft}
CVL441 Structural Design
\end{flushleft}


\begin{flushleft}
3 Credits (3-0-0)
\end{flushleft}


\begin{flushleft}
Pre-requisites: CVL241, CVL243, CVL342
\end{flushleft}


\begin{flushleft}
Design of Reinforced Cement concrete (RCC) Structures -- Building
\end{flushleft}


\begin{flushleft}
frames Liquid retaining structures, Earth Retaining walls,
\end{flushleft}


\begin{flushleft}
Design of Steel Structures -- Plate girders, gantry girders and steel
\end{flushleft}


\begin{flushleft}
bridge components
\end{flushleft}





\begin{flushleft}
CVP441 Structural Design \& Detailing
\end{flushleft}


\begin{flushleft}
1.5 Credits (0-0-3)
\end{flushleft}


\begin{flushleft}
Pre-requisites: CVL243, CVL342
\end{flushleft}


\begin{flushleft}
Part-I Concrete Structures
\end{flushleft}


\begin{flushleft}
Computer-aided analysis and design of real-life reinforced concrete
\end{flushleft}


\begin{flushleft}
(RC) structure. Dimensioning of concrete elements based on modular
\end{flushleft}


\begin{flushleft}
formworks available in construction industry. Detailing of concrete
\end{flushleft}


\begin{flushleft}
elements in terms of reinforcement, curtailment, lapping, splicing
\end{flushleft}


\begin{flushleft}
of reinforcements and connection with adjoining elements in the
\end{flushleft}


\begin{flushleft}
structure; member drawings. Joint detailing from ductility view point,
\end{flushleft}


\begin{flushleft}
Indian standard (IS) code recommendations and practical intricacies
\end{flushleft}


\begin{flushleft}
involved in casting and handling of the RC members, its sequence of
\end{flushleft}


\begin{flushleft}
construction and constructability.
\end{flushleft}


\begin{flushleft}
Part-II Steel Structures
\end{flushleft}


\begin{flushleft}
Computer-aided analysis and design of real-life steel structure. Steel
\end{flushleft}


\begin{flushleft}
member details as per shop/ field activities for welding/ bolting; i.e.
\end{flushleft}


\begin{flushleft}
fabrication (shop) drawings. Connection details, gusset plate design
\end{flushleft}


\begin{flushleft}
and detailing from ductility view point, Indian standard (IS) code
\end{flushleft}


\begin{flushleft}
recommendations and practical intricacies involved in fabrication
\end{flushleft}


\begin{flushleft}
and handling of the steel members, its sequence of erection and
\end{flushleft}


\begin{flushleft}
constructability.
\end{flushleft}





\begin{flushleft}
CVL442 Structural Analysis-III
\end{flushleft}


\begin{flushleft}
3 Credits (3-0-0)
\end{flushleft}


\begin{flushleft}
Pre-requisites: CVL341
\end{flushleft}


\begin{flushleft}
Introduction to FEM for structural analysis with review of energy
\end{flushleft}


\begin{flushleft}
methods-2D plane stress and plane strain elements, beam element, 2D
\end{flushleft}


\begin{flushleft}
bending element, example problems, elements of structural dynamicsfree and forced vibration of SDOF system, treatment of impact and
\end{flushleft}


\begin{flushleft}
arbitrary loading, frequency and time domain analysis; free vibration
\end{flushleft}


\begin{flushleft}
mode shapes and frequencies of MDOF systems; normal mode theory
\end{flushleft}


\begin{flushleft}
for forced vibration analysis of MODF system; example problems.
\end{flushleft}


\begin{flushleft}
Elements of plastic analysis; upper and lower bound theorems;
\end{flushleft}


\begin{flushleft}
methods of collapse mechanism; application to beams and multistory
\end{flushleft}


\begin{flushleft}
frames; example problems.
\end{flushleft}





\begin{flushleft}
CVL443 Prestressed Concrete \& Industrial Structures
\end{flushleft}


\begin{flushleft}
3 Credits (3-0-0)
\end{flushleft}


\begin{flushleft}
Pre-requisites: CVL241, CVL243, CVL341
\end{flushleft}


\begin{flushleft}
Prestressed Concrete Structures-Fundamentals of presenting,
\end{flushleft}


\begin{flushleft}
Prestressing technology, Analysis of prestressed losses, Design for
\end{flushleft}


\begin{flushleft}
Flexure, Design for shear and torsion, Design of anchorage Zones in
\end{flushleft}


\begin{flushleft}
Post-tensioned members.
\end{flushleft}


\begin{flushleft}
Industrial Structures-Analysis and design of Cylindrical shell structures,
\end{flushleft}


\begin{flushleft}
Folded plates, Chimneys, Silos, Bunkers.
\end{flushleft}





\begin{flushleft}
CVL462 Introduction to Intelligent Transportation
\end{flushleft}


\begin{flushleft}
Systems
\end{flushleft}


\begin{flushleft}
3 Credits (3-0-0)
\end{flushleft}


\begin{flushleft}
Pre-requisites: CVL261
\end{flushleft}


\begin{flushleft}
Introduction to Intelligent Transportation Systems (ITS); ITS
\end{flushleft}


\begin{flushleft}
Organizational Issues, the fundamental concepts of Intelligent
\end{flushleft}


\begin{flushleft}
Transportation Systems (ITS) to students with interest in engineering,
\end{flushleft}


\begin{flushleft}
transportation systems, communication systems, vehicle technologies,
\end{flushleft}


\begin{flushleft}
transportation planning, transportation policy, and urban planning.
\end{flushleft}


\begin{flushleft}
ITS in transportation infrastructure and vehicles, that improve
\end{flushleft}


\begin{flushleft}
transportation safety, productivity, environment, and travel reliability.
\end{flushleft}


\begin{flushleft}
Mobile device applications of ITS such as trip planners.
\end{flushleft}





\begin{flushleft}
CVL481 Water Resources Management
\end{flushleft}


\begin{flushleft}
3 Credits (3-0-0)
\end{flushleft}


\begin{flushleft}
Pre-requisites: CVL282 and EC 100
\end{flushleft}


\begin{flushleft}
Scope of water resources management, Global trends in water
\end{flushleft}


\begin{flushleft}
utilization, Crop water requirements and irrigation, Planning and
\end{flushleft}


\begin{flushleft}
desing of various irrigation methods, Soil salinity and water logging,
\end{flushleft}


\begin{flushleft}
Hydropower systems management, Strom water system management,
\end{flushleft}


\begin{flushleft}
Economic analysis of water resources projects, Flood Control studies.
\end{flushleft}





\begin{flushleft}
CVL482 Water Power Engineering
\end{flushleft}


\begin{flushleft}
3 Credits (2-0-2)
\end{flushleft}


\begin{flushleft}
Pre-requisites: CVL281 and EC 100
\end{flushleft}


\begin{flushleft}
Basic principle of hydropower generation, Hydropower Project
\end{flushleft}


\begin{flushleft}
Planning, Site selection, Hydropower development schemes, Reservoir
\end{flushleft}


\begin{flushleft}
storage, Assessment of power potential, Hydrologic analysis: Flow
\end{flushleft}


\begin{flushleft}
duration and load duration curves, Dependable flow, Design flood,
\end{flushleft}


\begin{flushleft}
Reservoir operation; Hydraulic design of various components of
\end{flushleft}


\begin{flushleft}
hydropower plants: intakes, hydraulic turbines, conduits and water
\end{flushleft}


\begin{flushleft}
conveyance, penstock; Performance characteristics of turbines, Specific
\end{flushleft}


\begin{flushleft}
and unit quantities, Electrical load on hydro-turbines, Power house
\end{flushleft}


\begin{flushleft}
dimension and planning, Water hammer and surge analysis, Surge
\end{flushleft}


\begin{flushleft}
tanks, Small and micro hydro power development, tidal plants, Current
\end{flushleft}


\begin{flushleft}
scenarios in hydropower development, Project feasibility, Impact of
\end{flushleft}


\begin{flushleft}
hydropower development on water sources systems, environment,
\end{flushleft}


\begin{flushleft}
socioeconomic conditions and national economy.
\end{flushleft}





\begin{flushleft}
CVL483 Groundwater \& Surface-water Pollution
\end{flushleft}


\begin{flushleft}
2 Credits (2-0-0)
\end{flushleft}


\begin{flushleft}
Pre-requisites: CVL282 and EC 100
\end{flushleft}


\begin{flushleft}
Groundwater contamination; River and Lake pollution; Pollution
\end{flushleft}


\begin{flushleft}
sources, Geogenic and anthropogenic pollution; Soil Pollution;
\end{flushleft}


\begin{flushleft}
Contaminant transport mechanisms; Pollution control, remediation
\end{flushleft}


\begin{flushleft}
technologies and role of wetlands. Environmental impact assessments,
\end{flushleft}


\begin{flushleft}
Hydrological impacts, Vulnerability, Case studies.
\end{flushleft}





\begin{flushleft}
CVP484 Computational Aspects in Water Resources
\end{flushleft}


\begin{flushleft}
3 Credits (1-0-4)
\end{flushleft}


\begin{flushleft}
Pre-requisites: CVL281 and EC 100
\end{flushleft}





170





\begin{flushleft}
\newpage
Civil Engineering
\end{flushleft}





\begin{flushleft}
Numerical Interpolation and Integration and application to water
\end{flushleft}


\begin{flushleft}
resources problems; Numerical solution of differential equations
\end{flushleft}


\begin{flushleft}
in Water Resources such as groundwater flow, pipe flows, open
\end{flushleft}


\begin{flushleft}
channel flows.
\end{flushleft}





\begin{flushleft}
CVL485 River Mechanics
\end{flushleft}


\begin{flushleft}
3 Credits (2-0-2)
\end{flushleft}


\begin{flushleft}
Pre-requisites: CVL281 and EC 100
\end{flushleft}


\begin{flushleft}
Introduction, river morpohology, drainage patterns, stream order.
\end{flushleft}


\begin{flushleft}
Properties of mixture of sediment and water, Incipient motion and
\end{flushleft}


\begin{flushleft}
quantitative approach to incipient motion, channel degradation and
\end{flushleft}


\begin{flushleft}
armoring. Bed forms and resistance to flow, various approaches
\end{flushleft}


\begin{flushleft}
for bed load transport, suspended load profile and suspended load
\end{flushleft}


\begin{flushleft}
equations, total load transport including total load transport equations.
\end{flushleft}


\begin{flushleft}
Comparison and evaluation of sediment transport equations. Stable
\end{flushleft}


\begin{flushleft}
channel design with critical tractive force theory.
\end{flushleft}





\begin{flushleft}
CVL486 Geo-informatics
\end{flushleft}


\begin{flushleft}
3 Credits (2-0-2)
\end{flushleft}


\begin{flushleft}
Pre-requisites: EC 100
\end{flushleft}


\begin{flushleft}
Geospatial and temporal data, Data acquisition, Global positioning
\end{flushleft}


\begin{flushleft}
system, Global Navigational Satellite System, GPS survey, Aerial
\end{flushleft}


\begin{flushleft}
and laser scanning surveys, Data acquisition using remote sensing
\end{flushleft}


\begin{flushleft}
techniques, Sensors and satellite imageries, Stereoscopic 3D viewing,
\end{flushleft}


\begin{flushleft}
Fly view, Satellite data formats and specifications, Data acquisition
\end{flushleft}


\begin{flushleft}
for natural resources management and weather forecast, Image
\end{flushleft}


\begin{flushleft}
processing and interpretation, GIS concepts and Spatial data models,
\end{flushleft}


\begin{flushleft}
Introduction to microwave remote sensing \& LiDAR, Geospatial
\end{flushleft}


\begin{flushleft}
analysis, DEM/DTM generation \& 3D modelling, Inferential statistics,
\end{flushleft}


\begin{flushleft}
Spatial interpolation, Integrated use of geospatial technologies,
\end{flushleft}


\begin{flushleft}
Applications and case studies.
\end{flushleft}





\begin{flushleft}
CVD700 Minor Project
\end{flushleft}


\begin{flushleft}
3 Credits (0-0-6)
\end{flushleft}


\begin{flushleft}
CVL700 Engineering Behaviour of Soils
\end{flushleft}


\begin{flushleft}
3 Credits (3-0-0)
\end{flushleft}


\begin{flushleft}
Origin, nature and distribution of soils. Description of individual
\end{flushleft}


\begin{flushleft}
particle. Clay mineralogy, clay-water-electrolytes. Soil fabric and
\end{flushleft}


\begin{flushleft}
structure. Effective stress principle. Steady state flow in soils. Effect
\end{flushleft}


\begin{flushleft}
of flow on effective stress. Determination of coefficient of permeability.
\end{flushleft}


\begin{flushleft}
Consolidation: one, two, three dimensional and radial consolidation.
\end{flushleft}


\begin{flushleft}
Various consolidation tests and determination of parameters.
\end{flushleft}


\begin{flushleft}
Stress-path. Triaxial and direct shear tests. Shear behaviour of soils
\end{flushleft}


\begin{flushleft}
under static and dynamic loads. Factors affecting shear beahviour.
\end{flushleft}


\begin{flushleft}
Determination of parameters. Shear behavior of fine grained soils.
\end{flushleft}


\begin{flushleft}
Pore-pressure parameters. UU, CU, CD tests. Total and effective
\end{flushleft}


\begin{flushleft}
stress-strength parameters. Total and effective stress-paths. Water
\end{flushleft}


\begin{flushleft}
content contours. Factors affecting strength : stress history, rate of
\end{flushleft}


\begin{flushleft}
testing, structure and temperature. Anisotropy of strength, thixotropy,
\end{flushleft}


\begin{flushleft}
creep. Determination of in-situ undrained strength. Stress-strain
\end{flushleft}


\begin{flushleft}
characteristics of soils. Determination of modulus values. Critical
\end{flushleft}


\begin{flushleft}
state model, Engineering behaviour of soils of India: Black cotton
\end{flushleft}


\begin{flushleft}
soils, alluvial silts and sands, laterites, collapsible and sensitive soils.
\end{flushleft}





\begin{flushleft}
CVP700 Soil Engineering Lab
\end{flushleft}


\begin{flushleft}
3 Credits (0-0-6)
\end{flushleft}


\begin{flushleft}
Laboratory Tests: Preparation of samples - Sand and Clay, Consolidation
\end{flushleft}


\begin{flushleft}
test, Direct shear test, Vane shear test, Unconfined compression test,
\end{flushleft}


\begin{flushleft}
Unconsolidated undrained triaxial test, Consolidated drained triaxial
\end{flushleft}


\begin{flushleft}
test, Consolidated undrained triaxial test with pore water pressure
\end{flushleft}


\begin{flushleft}
measurement, Free swell index test, Swelling pressure test.
\end{flushleft}


\begin{flushleft}
Field Investigations and field tests: Drilling of bore hole, standard
\end{flushleft}


\begin{flushleft}
penetration test. undisturbed and representative sampling. SCP Test,
\end{flushleft}


\begin{flushleft}
Electrical resistivity, Plate load test, Pile load test.
\end{flushleft}





\begin{flushleft}
CVL701 Site Investigation and Foundation Design
\end{flushleft}


\begin{flushleft}
3 Credits (3-0-0)
\end{flushleft}


\begin{flushleft}
Site Investigation: Geophysical methods-Seismic, electrical; Drilling
\end{flushleft}


\begin{flushleft}
methods; Boring in soils and rocks. Field tests: In-situ tests, SPT,
\end{flushleft}


\begin{flushleft}
DCPT, SCPT, in-situ vane shear test, pressure meter test, plate load
\end{flushleft}





\begin{flushleft}
test. Sampling techniques and disturbances. Shallow Foundations:
\end{flushleft}


\begin{flushleft}
Design considerations, codal provisions. Bearing capacity theories,
\end{flushleft}


\begin{flushleft}
Layered soils, Choice of shear strength parameters. Bearing capacity
\end{flushleft}


\begin{flushleft}
from field tests. Total and differential settlements. Deep foundations:
\end{flushleft}


\begin{flushleft}
Types of piles. Construction methods. Axial capacity of single piles.
\end{flushleft}


\begin{flushleft}
Axial capacity of groups. Settlement of single piles and groups. Uplift
\end{flushleft}


\begin{flushleft}
capacity (including under-reamed piles) . Negative skin friction. Pile
\end{flushleft}


\begin{flushleft}
load tests. Pile integrity tests. Codal provisions. Caissons.
\end{flushleft}


\begin{flushleft}
Laterally Loaded Piles: Analysis and Design; Foundations in Difficult
\end{flushleft}


\begin{flushleft}
soil conditions.
\end{flushleft}





\begin{flushleft}
CVL702 Ground Improvement and Geosynthetics
\end{flushleft}


\begin{flushleft}
3 Credits (3-0-0)
\end{flushleft}


\begin{flushleft}
Principles of compaction, Engineering behaviour of compacted clays.
\end{flushleft}


\begin{flushleft}
Shallow stabilization with additives: lime, fly ash and cement. Deep
\end{flushleft}


\begin{flushleft}
stabilization: stone column, sand drains, prefabricated drains, lime
\end{flushleft}


\begin{flushleft}
column, soil-lime column, vibro-floatation, dynamic compaction,
\end{flushleft}


\begin{flushleft}
electro-osmosis. Grouting : permeation, compaction and jet;
\end{flushleft}


\begin{flushleft}
Dewatering systems. Geosynthetics: types and functions, materials
\end{flushleft}


\begin{flushleft}
and manufacturing processes, testing and evaluation; Reinforced soil
\end{flushleft}


\begin{flushleft}
structures: principles of soil reinforcement, application of geotextiles
\end{flushleft}


\begin{flushleft}
and geogrids in roads, walls, and embankments. Application of
\end{flushleft}


\begin{flushleft}
geotextiles, geonets and geocomposites as drains and filters.
\end{flushleft}


\begin{flushleft}
Multiple functions: railways and overlay design. Geosynthetics in
\end{flushleft}


\begin{flushleft}
environmental control: covers and liners for landfills -- material aspects
\end{flushleft}


\begin{flushleft}
and stability considerations.
\end{flushleft}





\begin{flushleft}
CVL703 Geoenvironmental Engineering
\end{flushleft}


\begin{flushleft}
3 Credits (3-0-0)
\end{flushleft}


\begin{flushleft}
Subsurface Contamination and Contaminant Transport; Waste disposal
\end{flushleft}


\begin{flushleft}
on Land and containment, Landfills and Slurry ponds, Monitoring of
\end{flushleft}


\begin{flushleft}
subsurface contamination, Control and Remediation. Engineering
\end{flushleft}


\begin{flushleft}
Properties of waste and geotechnical reuse, erosoin control,
\end{flushleft}


\begin{flushleft}
sustainability, energy geotechnics.
\end{flushleft}





\begin{flushleft}
CVL704 Finite Element Method in Geotechnical
\end{flushleft}


\begin{flushleft}
Engineering
\end{flushleft}


\begin{flushleft}
3 Credits (3-0-0)
\end{flushleft}


\begin{flushleft}
Introduction. Steps in FEM. Variational Methods, Stress-deformation
\end{flushleft}


\begin{flushleft}
analysis: One-Two dimensional formulations; Three-dimensional
\end{flushleft}


\begin{flushleft}
formulations; Boundary conditions; Solution algorithms; Discretization;
\end{flushleft}


\begin{flushleft}
use of FEM2D Program and Commercial packages. Analysis of
\end{flushleft}


\begin{flushleft}
foundations, dams, underground structures and earth retaining
\end{flushleft}


\begin{flushleft}
structures. Analysis of flow (seepage) through dams and foundations.
\end{flushleft}


\begin{flushleft}
Consolidation Analysis, Linear and non-linear analysis. Insitu stresses.
\end{flushleft}


\begin{flushleft}
Sequence construction and excavation. Joint/interface elements.
\end{flushleft}


\begin{flushleft}
Infinite elements. Dynamic analysis. Evaluation of material parameters
\end{flushleft}


\begin{flushleft}
for linear and non-linear analysis, Recent developments.
\end{flushleft}





\begin{flushleft}
CVL705 Slopes and Retaining Structures
\end{flushleft}


\begin{flushleft}
3 Credits (3-0-0)
\end{flushleft}


\begin{flushleft}
Slope stability: infinite slopes; finite height slopes -- Swedish method,
\end{flushleft}


\begin{flushleft}
Bishop's simplified method and other limit equilibrium methods;
\end{flushleft}


\begin{flushleft}
Stability charts; conditions of analysis -- steady state, end of
\end{flushleft}


\begin{flushleft}
construction and sudden draw down; earthquake effects. Seepage:
\end{flushleft}


\begin{flushleft}
flownet in isotropic, anisotropic and layered media; entrance-exit
\end{flushleft}


\begin{flushleft}
conditions; determination of phreatic line. Earth Dams: Introduction,
\end{flushleft}


\begin{flushleft}
factors influencing design, design of components, construction,
\end{flushleft}


\begin{flushleft}
instrumentation. Road and rail embankments. Reinforced slopes. Soil
\end{flushleft}


\begin{flushleft}
nailing; Gabions. Earth Pressure: Types; Rankine's theory and Coulomb's
\end{flushleft}


\begin{flushleft}
theory; Effects due to wall friction; Graphical methods; Earthquake
\end{flushleft}


\begin{flushleft}
effects. Rigid retaining structures: Types; stability analysis. Flexible
\end{flushleft}


\begin{flushleft}
retaining structures: Types; material; cantilever sheet piles; anchored
\end{flushleft}


\begin{flushleft}
bulkheads--methods of analysis, moment reduction factors; anchorage.
\end{flushleft}


\begin{flushleft}
Reinforced soil walls: Elements and stability. Soil arching. Braced
\end{flushleft}


\begin{flushleft}
excavation: Pressure distribution in sands and clays; bottom heave.
\end{flushleft}


\begin{flushleft}
Underground structures in soils: Pipes; tunnels. Tunneling techniques.
\end{flushleft}





\begin{flushleft}
CVL706 Soil Dynamics and Earthquake Geotechnical
\end{flushleft}


\begin{flushleft}
Engineering
\end{flushleft}


\begin{flushleft}
3 Credits (3-0-0)
\end{flushleft}


\begin{flushleft}
Engineering problems involving soil dynamics; Role of inertia; Theory
\end{flushleft}





171





\begin{flushleft}
\newpage
Civil Engineering
\end{flushleft}





\begin{flushleft}
of Vibrations: Single and two-degrees of freedom systems, vibration
\end{flushleft}


\begin{flushleft}
measuring instruments, Vibration absorption and isolation techniques.
\end{flushleft}


\begin{flushleft}
Wave propagation: elastic continuum medium and semi-infinite elastic
\end{flushleft}


\begin{flushleft}
continuum medium. Measurement of small strain and large strain
\end{flushleft}


\begin{flushleft}
dynamic soil properties: Field and Laboratory tests. Selection of design
\end{flushleft}


\begin{flushleft}
values. Design criteria for machine foundations, elastic homogeneous
\end{flushleft}


\begin{flushleft}
half space solutions, lumped parameter solutions. Codal provisions;
\end{flushleft}


\begin{flushleft}
Design of Pile-supported machine foundations. Strong Ground Motion:
\end{flushleft}


\begin{flushleft}
Measurement, characterization and estimation; Amplification theory
\end{flushleft}


\begin{flushleft}
and ground response analysis. Liquefaction of soils: evaluation using
\end{flushleft}


\begin{flushleft}
simple methods and mitigation measures. Seismic slope stability
\end{flushleft}


\begin{flushleft}
analysis, Seismic bearing capacity and earth pressures. Codal provisions.
\end{flushleft}





\begin{flushleft}
CVL707 Soil-Structure Interaction Analysis
\end{flushleft}


\begin{flushleft}
3 Credits (3-0-0)
\end{flushleft}


\begin{flushleft}
Basic Soil Models: Single parameter model - Winkler; Two parameter
\end{flushleft}


\begin{flushleft}
models - Bilonenko-Borodick, Pasternak; Elastic Continuum - plane
\end{flushleft}


\begin{flushleft}
strain, plane stress, Boussinesq's problem, line load, strip load; Special
\end{flushleft}


\begin{flushleft}
models starting with elastic continuum - Vlazov, Reissner; Three
\end{flushleft}


\begin{flushleft}
parameter model - Kerr model; Evaluation of model parameters for
\end{flushleft}


\begin{flushleft}
different conditions. Beam on Winkler foundation: solutions for infinite
\end{flushleft}


\begin{flushleft}
and semi-infinite beams; Finite beams: method of initial parameters,
\end{flushleft}


\begin{flushleft}
method of superposition. Beams on Elastic continuum: Use of finite
\end{flushleft}


\begin{flushleft}
difference method, rigid and flexible beams, lift-off, non-homogeneous
\end{flushleft}


\begin{flushleft}
soil, non-linear soil, plastic yielding of soil. Raft or Mat foundations:
\end{flushleft}


\begin{flushleft}
thin rectangular plates, approximate theory of plates, circular plates.
\end{flushleft}


\begin{flushleft}
Pile on Winkler foundation: Vertically loaded pile - rigid pile, evaluation
\end{flushleft}


\begin{flushleft}
of spring stiffness, non-homogeneous soil, compressible pile; Laterally
\end{flushleft}


\begin{flushleft}
loaded pile - rigid pile, Elastic pile, standard solutions for different end
\end{flushleft}


\begin{flushleft}
conditions; Pile on elastic continuum : vertically loaded piles - rigid pile.
\end{flushleft}





\begin{flushleft}
CVL708 Geotechnology of Waste Disposal Facilities
\end{flushleft}


\begin{flushleft}
3 Credits (3-0-0)
\end{flushleft}


\begin{flushleft}
Integrated waste management, Detailed design of MSW Landfills and
\end{flushleft}


\begin{flushleft}
HW Landfills including individual components, Closure of Old landfills,
\end{flushleft}


\begin{flushleft}
Expansion of old landfills, Ashponds and Tailings Ponds, Seismic
\end{flushleft}


\begin{flushleft}
Stability; Disposal of Nuclear Waste.
\end{flushleft}





\begin{flushleft}
CVL709 Offshore Geotechnical Engineering
\end{flushleft}


\begin{flushleft}
3 Credits (3-0-0)
\end{flushleft}


\begin{flushleft}
Submarine soils: Origin, nature and distribution. Terrigenic and pelagic
\end{flushleft}


\begin{flushleft}
soils. Submarine soils of India. Engineering behaviour of submarine
\end{flushleft}


\begin{flushleft}
soils: under-consolidated soils, calcareous soils, cemented soils, corals;
\end{flushleft}


\begin{flushleft}
Offshore site investigations: sampling and sampling disturbance,
\end{flushleft}


\begin{flushleft}
insitu testing, wireline technology. Offshore pile foundations for jacket
\end{flushleft}


\begin{flushleft}
type structures. Foundations of gravity structures; Foundations for
\end{flushleft}


\begin{flushleft}
jackup rigs. Anchors and breakout forces; anchor systems for floating
\end{flushleft}


\begin{flushleft}
structures. Stability of submarine slopes. Installation and stability of
\end{flushleft}


\begin{flushleft}
submarine pipelines.
\end{flushleft}





\begin{flushleft}
CVD710 Minor Project (CEU)
\end{flushleft}


\begin{flushleft}
3 Credits (0-0-6)
\end{flushleft}


\begin{flushleft}
CVL710 Engineering Properties of Rocks and Rock Masses
\end{flushleft}


\begin{flushleft}
3 Credits (3-0-0)
\end{flushleft}


\begin{flushleft}
Introduction. Rock materials, Physical properties, Strength behaviour
\end{flushleft}


\begin{flushleft}
in uniaxial compression, tension and triaxial state. Laboratory testing
\end{flushleft}


\begin{flushleft}
methods. Stress-strain relationships. Factors influencing strength.
\end{flushleft}


\begin{flushleft}
Failure mechanism. Anisotropy. Failure criteria, Coulomb, Mohr's,
\end{flushleft}


\begin{flushleft}
Griffiths and Modified Griffiths criteria and Empirical criteria. Brittle
\end{flushleft}


\begin{flushleft}
-- ductile transition, Post failure behaviour. Strength and deformation
\end{flushleft}


\begin{flushleft}
behaviour of discontinuities. Rockmass behaviour, Shear strength
\end{flushleft}


\begin{flushleft}
of jointed rocks, roughness, peak and residual strengths. Strength
\end{flushleft}


\begin{flushleft}
criteria for rockmass. Intact and rockmass classifications, Terzaghi,
\end{flushleft}


\begin{flushleft}
RQD, RSR, RMR and Q classifications, Rating, Applications. Creep
\end{flushleft}


\begin{flushleft}
and cyclic loading. Weathered rocks. Flow through intact and fissured
\end{flushleft}


\begin{flushleft}
rocks. Dynamic properties.
\end{flushleft}





\begin{flushleft}
CVP710 Rock Mechanics Laboratory 1
\end{flushleft}


\begin{flushleft}
3 Credits (0-0-6)
\end{flushleft}


\begin{flushleft}
Tests and test procedures, Rock samples,Specimen preparation,
\end{flushleft}


\begin{flushleft}
Coring, cutting and lapping. Tolerance limits.
\end{flushleft}





\begin{flushleft}
Physical Properties: Water absorption, density, specific gravity,
\end{flushleft}


\begin{flushleft}
porosity, void index, electrical resistivity and sonic wave velocity
\end{flushleft}


\begin{flushleft}
tests. Mechanical Properties: Uniaxial compression, Point load index
\end{flushleft}


\begin{flushleft}
and Brazilian strength tests, Elastic properties. Effect of L/D ratio
\end{flushleft}


\begin{flushleft}
and saturation. Strength anisotropy. Shear tests: Single, double,
\end{flushleft}


\begin{flushleft}
oblique tests, Punch shear,Triaxial compression tests, Direct shear
\end{flushleft}


\begin{flushleft}
test. Slake durability and Permeability tests. Compilation of test data.
\end{flushleft}


\begin{flushleft}
Classification. Codal provisions.
\end{flushleft}





\begin{flushleft}
CVL711 Structural Geology
\end{flushleft}


\begin{flushleft}
3 Credits (3-0-0)
\end{flushleft}


\begin{flushleft}
Origin, interior and composition of the earth. Rock cycle, Igneous,
\end{flushleft}


\begin{flushleft}
Metamorphic and Sedimentary rocks. Rock structures. Plate
\end{flushleft}


\begin{flushleft}
tectonics, Continental drift and sea floor spreading. Geological
\end{flushleft}


\begin{flushleft}
time scale. Layered formations, Attitude, true and apparent
\end{flushleft}


\begin{flushleft}
dips, topographic maps, outcrops. Measurement of attitude of
\end{flushleft}


\begin{flushleft}
formations. Folds, types of folds, classification, field study of folds,
\end{flushleft}


\begin{flushleft}
mechanics of folds, causes of folding. Joints, rock mass concept,
\end{flushleft}


\begin{flushleft}
Joint description and classification. Three point problems, Depth
\end{flushleft}


\begin{flushleft}
and thickness problems. Faults, mechanics of faulting, normal,
\end{flushleft}


\begin{flushleft}
reverse and thrusts, faults. Lineations. Foliations, Schistocity. Fault
\end{flushleft}


\begin{flushleft}
problems. Stereographic projection methods, Use of DIPS software,
\end{flushleft}


\begin{flushleft}
presentation of geological data and analysis, Applications,Scan line
\end{flushleft}


\begin{flushleft}
survey of rock joints in the visit.
\end{flushleft}





\begin{flushleft}
CVL712 Slopes and Foundations
\end{flushleft}


\begin{flushleft}
3 Credits (3-0-0)
\end{flushleft}


\begin{flushleft}
Introduction, Short-term and long-term stability. Influence of ground
\end{flushleft}


\begin{flushleft}
water, Seismic effects. Types of rock slope failures. Infinite slopes,
\end{flushleft}


\begin{flushleft}
Circular and non-circular slip surface analysis, Stability charts. Plane
\end{flushleft}


\begin{flushleft}
failure analysis. Wedge failure analysis analytical, Stereographic
\end{flushleft}


\begin{flushleft}
methods. Buckling and toppling failures, Rock falls, Landslides.
\end{flushleft}


\begin{flushleft}
Foundations: Bearing capacity, settlement and stress distribution in
\end{flushleft}


\begin{flushleft}
intact and layered rocks. Foundations of dams. Deep foundations.
\end{flushleft}


\begin{flushleft}
Tension foundations, Codal provisions. Foundation improvement. Use
\end{flushleft}


\begin{flushleft}
of appropriate software packages.
\end{flushleft}





\begin{flushleft}
CVL713 Analysis and Design of Underground Structures
\end{flushleft}


\begin{flushleft}
3 Credits (3-0-0)
\end{flushleft}


\begin{flushleft}
Introduction. Types and classification of underground openings. Factors
\end{flushleft}


\begin{flushleft}
affecting design. Design methodology. Functional aspects. Size and
\end{flushleft}


\begin{flushleft}
shapes. Support systems. Codal provisions. Analysis: Stresses and
\end{flushleft}


\begin{flushleft}
deformations around openings, Stresses and deformations around
\end{flushleft}


\begin{flushleft}
tunnels and galleries with composite lining due to internal pressure,
\end{flushleft}


\begin{flushleft}
Closed form solutions, BEM, FEM. Design: Design based on analytical
\end{flushleft}


\begin{flushleft}
methods; Empirical methods based on RSR, RMR, Q systems; Design
\end{flushleft}


\begin{flushleft}
based on Rock support interaction analysis; Observational methodNATM, Convergence-confinement method. Design based on Wedge
\end{flushleft}


\begin{flushleft}
failure and key block analysis. Design of Shafts and hydraulic tunnels.
\end{flushleft}


\begin{flushleft}
Stability of excavation face and Tunnel portals. Use of appropriate
\end{flushleft}


\begin{flushleft}
software packages.
\end{flushleft}





\begin{flushleft}
CVL714 Field Exploration and Geotechnical Processes
\end{flushleft}


\begin{flushleft}
3 Credits (3-0-0)
\end{flushleft}


\begin{flushleft}
Surface and sub surface exploration methods. Aerial and remote
\end{flushleft}


\begin{flushleft}
sensing techniques, Geophysical methods, electrical resistivity, seismic
\end{flushleft}


\begin{flushleft}
refraction, applications. Rock drilling, Core samplers, Core boxes,
\end{flushleft}


\begin{flushleft}
Core orientations.
\end{flushleft}


\begin{flushleft}
Logging, stratigraphic profile, scan line survey. Laboratory tests, report.
\end{flushleft}


\begin{flushleft}
Stresses in rocks. Stress anisotropy and stress ratio. Stress relief
\end{flushleft}


\begin{flushleft}
and compensation techniques, USBM, door stopper cells, flat jack,
\end{flushleft}


\begin{flushleft}
hydrofrac, strain rossette and dilatometers. Deformability, plate load,
\end{flushleft}


\begin{flushleft}
pressure tunnel and bore hole tests. Strength tests, insitu compression,
\end{flushleft}


\begin{flushleft}
tension and direct shear tests. Pull out tests. Borehole extensometers,
\end{flushleft}


\begin{flushleft}
piezometers, embedment gauges, inclinometers, Slope indicators,
\end{flushleft}


\begin{flushleft}
packer tests for insitu permeability, Codal provisions.
\end{flushleft}


\begin{flushleft}
Ground improvement techniques. Compaction, Grouting, Types of
\end{flushleft}


\begin{flushleft}
grouts, technique, Rheological models. Viscous and viscoplastic flows.
\end{flushleft}


\begin{flushleft}
Spherical and radial flows, Shotcrete, Ground anchors, Rock bolts.
\end{flushleft}





172





\begin{flushleft}
\newpage
Civil Engineering
\end{flushleft}





\begin{flushleft}
CVL715 Excavation Methods and Underground Space
\end{flushleft}


\begin{flushleft}
Technology
\end{flushleft}


\begin{flushleft}
3 Credits (3-0-0)
\end{flushleft}


\begin{flushleft}
Principles of rock breakage, explosive energy, energy balance, blasting
\end{flushleft}


\begin{flushleft}
mechanism. Types of explosives, initiators, delay devices, primer and
\end{flushleft}


\begin{flushleft}
booster selection. Blast hole design. Drilling methods and machines
\end{flushleft}


\begin{flushleft}
Blast hole timing. Pattern design, open pit and underground blasting,
\end{flushleft}


\begin{flushleft}
production, estimation and damage criteria of ground vibrations. TBM
\end{flushleft}


\begin{flushleft}
tunnelling. Factors influencing and evaluation, Excavation mechanics,
\end{flushleft}


\begin{flushleft}
Boom machines, transverse boom tunnelling machines and Robins
\end{flushleft}


\begin{flushleft}
mobile miner. Drag pick cutting, cutting tool materials and wear, disc
\end{flushleft}


\begin{flushleft}
cutters. Case studies.
\end{flushleft}


\begin{flushleft}
Tunnels, energy storage caverns, nuclear waste disposal repositories,
\end{flushleft}


\begin{flushleft}
metros, underground chambers and defence installations. Geological
\end{flushleft}


\begin{flushleft}
considerations, layout, survey and alignment. Analysis and design
\end{flushleft}


\begin{flushleft}
methods. Construction methods. Ventilation, provisions, equipment.
\end{flushleft}


\begin{flushleft}
Control and monitoring system, services, operations and maintenance.
\end{flushleft}


\begin{flushleft}
Lighting, specifications, maintenance, emergency lighting. Power supply
\end{flushleft}


\begin{flushleft}
and distribution, Water supply and distribution. Safety provisions,
\end{flushleft}


\begin{flushleft}
localized hazards, fire hazards in highway tunnels, rapid transit tunnels.
\end{flushleft}


\begin{flushleft}
Surveillance and control system for highway tunnels. Tunnel finish.
\end{flushleft}





\begin{flushleft}
CVL716 Environmental Rock Engineering
\end{flushleft}


\begin{flushleft}
3 Credits (3-0-0)
\end{flushleft}


\begin{flushleft}
Theory: Stress-strain behaviour of rocks and rock masses: Elastic,
\end{flushleft}


\begin{flushleft}
elasto-plastic, and brittle, Crack phenomena and mechanisms of
\end{flushleft}


\begin{flushleft}
rock fracture.
\end{flushleft}


\begin{flushleft}
Temperature, pressure and water related, problems, Effect of temperature
\end{flushleft}


\begin{flushleft}
on rock behaviour. Fluid flow through intact and fissured rocks.
\end{flushleft}


\begin{flushleft}
Time dependent behaviour of rocks: Creep, Viscoelasticity and
\end{flushleft}


\begin{flushleft}
Viscoplasticity
\end{flushleft}


\begin{flushleft}
Continuum and discontinuum theories: Equivalent material, Block
\end{flushleft}


\begin{flushleft}
and Distinct element.
\end{flushleft}


\begin{flushleft}
Application: Waste disposal, Radioactive and hazardous wastes,
\end{flushleft}


\begin{flushleft}
repositories, location and design, VLH, VDH and KBS3 concepts. Waste
\end{flushleft}


\begin{flushleft}
container, barriers, rock structure, embedment, buffers and seals.
\end{flushleft}


\begin{flushleft}
Performance assessment, quality control and monitoring. Case histories.
\end{flushleft}


\begin{flushleft}
Hazardous Earth processes, high ground stresses, rock bursts,
\end{flushleft}


\begin{flushleft}
subsidence. Karst formations. Landslides and rock falls, slopes
\end{flushleft}


\begin{flushleft}
stabilization, mitigation, Case studies.
\end{flushleft}


\begin{flushleft}
Earthquakes, tectonic stresses, creep, ground motions, damage,
\end{flushleft}


\begin{flushleft}
prediction. Volcanic activity and hazard. Tsunamis. Case studies.
\end{flushleft}


\begin{flushleft}
Thermal analysis, Thermo-mechanical analysis, thermo-hydromechanical analysis. Rock dynamics. Physical modelling.
\end{flushleft}





\begin{flushleft}
CVD720 Major thesis part1
\end{flushleft}


\begin{flushleft}
6 Credits (0-0-12)
\end{flushleft}


\begin{flushleft}
CVS720 Independent Study
\end{flushleft}


\begin{flushleft}
3 Credits (0-3-0)
\end{flushleft}





\begin{flushleft}
Management of: Municipal, Biomedical, Nuclear, Electronic and
\end{flushleft}


\begin{flushleft}
Industrial Solid Wastes and the rules and regulations.
\end{flushleft}





\begin{flushleft}
CVL722 Water Engineering
\end{flushleft}


\begin{flushleft}
3 Credits (3-0-0)
\end{flushleft}


\begin{flushleft}
Water quality parameters-conventional contaminants and emerging
\end{flushleft}


\begin{flushleft}
contaminants; Sedimentation; Coagulation and flocculation; Filtrationmechanisms and interpretations; Ion exchange and adsorption;
\end{flushleft}


\begin{flushleft}
Disinfection; Reverse osmosis, electrodialysis, desalination.
\end{flushleft}


\begin{flushleft}
Water treatment : Source selection process, selection of treatment
\end{flushleft}


\begin{flushleft}
chain, plant siting, Treatability studies. Design of physico-chemical
\end{flushleft}


\begin{flushleft}
unit operations.
\end{flushleft}





\begin{flushleft}
CVL723 Wastewater Engineering
\end{flushleft}


\begin{flushleft}
3 Credits (3-0-0)
\end{flushleft}


\begin{flushleft}
Wastewater quality parameters, Biological processes; Microbial
\end{flushleft}


\begin{flushleft}
growth kinetics; Modeling of suspended growth systems; concepts
\end{flushleft}


\begin{flushleft}
and principles of carbon oxidation, nitrification, denitrification,
\end{flushleft}


\begin{flushleft}
methanogenasis. Biological nutrient removal; Anaerobic treatment;
\end{flushleft}


\begin{flushleft}
Attached growth reactors; decentralised wastewater treatment
\end{flushleft}


\begin{flushleft}
systems; constructed wetlands; Design of pretreatment, secondary
\end{flushleft}


\begin{flushleft}
treatment, and tertiary disposal systems. Sludge stabilization,
\end{flushleft}


\begin{flushleft}
treatment, sludge thickening, sludge drying, aerobic and anaerobic
\end{flushleft}


\begin{flushleft}
digestion of sludges; reliability and cost effectiveness of wastewater
\end{flushleft}


\begin{flushleft}
systems; Emerging contaminants in wastewater-treatment issues.
\end{flushleft}





\begin{flushleft}
CVL724 Environmental Systems Analysis
\end{flushleft}


\begin{flushleft}
3 Credits (3-0-0)
\end{flushleft}


\begin{flushleft}
Introduction to natural and man-made systems. Systems modeling as
\end{flushleft}


\begin{flushleft}
applied to environmental systems. Nature of environmental systems,
\end{flushleft}


\begin{flushleft}
the model building process addressing to specific environmental
\end{flushleft}


\begin{flushleft}
problems. Strategies for analyzing and using environmental
\end{flushleft}


\begin{flushleft}
systems models. Fate and transport models for contaminants in air,
\end{flushleft}


\begin{flushleft}
water, and soil. Optimization methods (search techniques, linear
\end{flushleft}


\begin{flushleft}
programming, non-linear programming, dynamic programming) to
\end{flushleft}


\begin{flushleft}
evaluate alternatives for solid-waste management and water and air
\end{flushleft}


\begin{flushleft}
pollution control. Optimization over time. Integrated environmental
\end{flushleft}


\begin{flushleft}
management strategies addressing multi-objective and multistakeholder planning.
\end{flushleft}





\begin{flushleft}
CVL725 Environmental Chemistry and Microbiology
\end{flushleft}


\begin{flushleft}
3 Credits (1-0-4)
\end{flushleft}


\begin{flushleft}
Chemical equilibria and kinetics fundamentals; Acids and bases;
\end{flushleft}


\begin{flushleft}
Titrations; Acidity; Alkalinity; Buffers and buffer intensity; Chemical
\end{flushleft}


\begin{flushleft}
equilibrium calculations; pC-pH diagram; Langelier index; Solubility
\end{flushleft}


\begin{flushleft}
diagram; Oxidation and reduction reactions; Cell structure; Types
\end{flushleft}


\begin{flushleft}
of microorganisms in environment;metabolic classification of
\end{flushleft}


\begin{flushleft}
organisms; laboratory procedure for determining chemical and
\end{flushleft}


\begin{flushleft}
microbial parameters, Introduction to advanced instruments.
\end{flushleft}





\begin{flushleft}
CVD726 Minor Project
\end{flushleft}


\begin{flushleft}
3 Credits (0-0-6)
\end{flushleft}





\begin{flushleft}
Specific to the context of the problem decided by the supervisor.
\end{flushleft}





\begin{flushleft}
CVL727 Environmental Risk Assessment
\end{flushleft}


\begin{flushleft}
3 Credits (3-0-0)
\end{flushleft}





\begin{flushleft}
CVL720 Air Pollution and Control
\end{flushleft}


\begin{flushleft}
3 Credits (3-0-0)
\end{flushleft}





\begin{flushleft}
CVD721 Major Thesis Part-II
\end{flushleft}


\begin{flushleft}
12 Credits (0-0-24)
\end{flushleft}





\begin{flushleft}
Basic concepts of environmental risk and definitions; Human health
\end{flushleft}


\begin{flushleft}
risk and ecological risk assessment framework;Hazard identification
\end{flushleft}


\begin{flushleft}
procedures and hazard prioritization; Environmental risk zonation;
\end{flushleft}


\begin{flushleft}
Consequence analysis and modelling (discharge models, dispersion
\end{flushleft}


\begin{flushleft}
models, fire and explosion models, effect models etc). Estimation
\end{flushleft}


\begin{flushleft}
of incident frequencies from historical data, frequency modelling
\end{flushleft}


\begin{flushleft}
techniques e.g., Fault tree analysis (FTA) and Event tree analysis
\end{flushleft}


\begin{flushleft}
(ETA), Reliability block diagram. Human factors in risk analysis; Risk
\end{flushleft}


\begin{flushleft}
management \& communication. Rules, regulations and conventions.
\end{flushleft}





\begin{flushleft}
CVL721 Solid Waste Engineering
\end{flushleft}


\begin{flushleft}
3 Credits (3-0-0)
\end{flushleft}





\begin{flushleft}
CVL728 Environmental Quality Modeling
\end{flushleft}


\begin{flushleft}
3 Credits (3-0-0)
\end{flushleft}





\begin{flushleft}
Solid Wastes: Origin, Analysis, Composition and Characteristics.
\end{flushleft}


\begin{flushleft}
Integrated Solid Waste Management System: Collection, Storage,
\end{flushleft}


\begin{flushleft}
Segregation, Reuse and Recycling possibilities, Transportation,
\end{flushleft}


\begin{flushleft}
Treatment / Processing and Transformation Techniques, Final Disposal.
\end{flushleft}





\begin{flushleft}
Plume Rise Models; Introduction to Air Quality Modelling;Turbulence
\end{flushleft}


\begin{flushleft}
fundamentals;Basic diffusion equation; ficks law; deterministic;
\end{flushleft}


\begin{flushleft}
numerical and statistical modeling approach; Fundamentals of
\end{flushleft}


\begin{flushleft}
Receptor modelling; Dispersion and receptor models; Fundamentals
\end{flushleft}





\begin{flushleft}
Air-pollution; Air Pollution Effect on Plants; Air Pollution effect on
\end{flushleft}


\begin{flushleft}
Human health; Air quality monitoring; Air Pollution Meteorology;
\end{flushleft}


\begin{flushleft}
Gaussian Plume model; Urban Air Pollution; Air Pollution from
\end{flushleft}


\begin{flushleft}
Industries; Air Pollution control; Air pollution indices; standards;
\end{flushleft}


\begin{flushleft}
norms; rules and regulations; Indoor Air Pollution.
\end{flushleft}





173





\begin{flushleft}
\newpage
Civil Engineering
\end{flushleft}





\begin{flushleft}
of Indoor air quality modelling techniques; Fundamentals of Water
\end{flushleft}


\begin{flushleft}
quality modeling: surface water and ground water models; Fate and
\end{flushleft}


\begin{flushleft}
transport of Conservative and non-conservative pollutants. Modelling
\end{flushleft}


\begin{flushleft}
as a tool for strategising pollution prevention and control.
\end{flushleft}





\begin{flushleft}
CVL729 Environmental Statistics and Experimental
\end{flushleft}


\begin{flushleft}
Design
\end{flushleft}


\begin{flushleft}
3 Credits (2-0-2)
\end{flushleft}


\begin{flushleft}
Introduction on environmental data, environmental statistics
\end{flushleft}


\begin{flushleft}
estimation (concentration, frequency of detection,minimum detection
\end{flushleft}


\begin{flushleft}
limit, sample size), frequency and probability distributions, inferences
\end{flushleft}


\begin{flushleft}
concerning mean and variance, confidence Interval estimation,
\end{flushleft}


\begin{flushleft}
hypotheses test, ANOVA, regression, goodness of fit, factoral
\end{flushleft}


\begin{flushleft}
experimentation, exceedance factor, intervention model, Case studies.
\end{flushleft}





\begin{flushleft}
CVL730 Hydrologic Processes and Modeling
\end{flushleft}


\begin{flushleft}
3 Credits (3-0-0)
\end{flushleft}


\begin{flushleft}
Hydrologic Cycle and its individual component processes. River Basin
\end{flushleft}


\begin{flushleft}
as a Linear Hydrologic System. Linear Theory of Hydrologic Systems.
\end{flushleft}


\begin{flushleft}
Lumped Integral and Distributed Differential modelling approaches.
\end{flushleft}


\begin{flushleft}
Transform methods of Linear Systems Analysis. Morphological
\end{flushleft}


\begin{flushleft}
attributes of watersheds and its role in runoff dynamics. Flood
\end{flushleft}


\begin{flushleft}
Routing by Lumped Hydrologic and Distributed Hydraulic approaches.
\end{flushleft}


\begin{flushleft}
Unsaturated zone Hydrology and physics of the Soil-Plant-Atmosphere
\end{flushleft}


\begin{flushleft}
Continuum. Calibration and Validation of Rainfall-Runoff models.
\end{flushleft}





\begin{flushleft}
CVP730 Simulation Laboratory-I
\end{flushleft}


\begin{flushleft}
1.5 Credits (0-0-3)
\end{flushleft}


\begin{flushleft}
Basic of Fortran 90, Fortran 95 and computing, Numerical solution of
\end{flushleft}


\begin{flushleft}
different types of partial differential equations: parabolic equation,
\end{flushleft}


\begin{flushleft}
elliptical equation, hyperbolic equation, Backwater curve analysis;
\end{flushleft}


\begin{flushleft}
Groundwater flow problems, Pipe network analysis, Unsteady channel flow.
\end{flushleft}





\begin{flushleft}
CVS730 Minor Project (CEW)
\end{flushleft}


\begin{flushleft}
3 Credits (0-0-6)
\end{flushleft}


\begin{flushleft}
CVL731 Optimization Techniques in Water Resources
\end{flushleft}


\begin{flushleft}
3 Credits (3-0-0)
\end{flushleft}


\begin{flushleft}
Optimization techniques commonly used in water resources planning
\end{flushleft}


\begin{flushleft}
\& management, water infrastructures, and irrigation and hydropower
\end{flushleft}


\begin{flushleft}
projects; Linear programming and duality, Network flow algorithms,
\end{flushleft}


\begin{flushleft}
Dynamic programming, Nonlinear programming, Geometric and Goal
\end{flushleft}


\begin{flushleft}
programming, Introduction to modern heuristic methods like genetic
\end{flushleft}


\begin{flushleft}
algorithm and simulated annealing, Multiobjective optimization,
\end{flushleft}


\begin{flushleft}
Applications and case studies in water resources, agriculture,
\end{flushleft}


\begin{flushleft}
environment and other areas of science \& engineering.
\end{flushleft}





\begin{flushleft}
CVP731 Simulation Laboratory-II
\end{flushleft}


\begin{flushleft}
1.5 Credits (0-0-3)
\end{flushleft}


\begin{flushleft}
Simulate hydraulic, hydrologic, pipe flow, water hammer using various
\end{flushleft}


\begin{flushleft}
softwares such as Visual Mod Flow, SWAT, HYDRUS, Hytran, MIKE,
\end{flushleft}


\begin{flushleft}
Bentley Software, Fluent, HMS, SAMS.
\end{flushleft}





\begin{flushleft}
CVL732 Groundwater Hydrology
\end{flushleft}


\begin{flushleft}
3 Credits (3-0-0)
\end{flushleft}


\begin{flushleft}
Occurrence and movement of groundwater including subsurface
\end{flushleft}


\begin{flushleft}
investigations of groundwater. Flow through saturated and unsaturated
\end{flushleft}


\begin{flushleft}
media. Well Hydraulics and aquifer parameters. Pumping wells and
\end{flushleft}


\begin{flushleft}
their design, construction, monitoring and rehabilation of wells.
\end{flushleft}


\begin{flushleft}
Recharge of groundwater by various means. Salt water intrusion
\end{flushleft}


\begin{flushleft}
and coastal aquifer hydraulics. Analog and numerical models and
\end{flushleft}


\begin{flushleft}
application of Finite Difference method to groundwater, case studies.
\end{flushleft}





\begin{flushleft}
CVL733 Stochastic Hydrology
\end{flushleft}


\begin{flushleft}
3 Credits (2-0-2)
\end{flushleft}


\begin{flushleft}
Concepts of probability and Random variables; moments and
\end{flushleft}


\begin{flushleft}
expectations; Common probabilistic distributions and estimation of
\end{flushleft}


\begin{flushleft}
parameters; goodness of fit tests; Modelling of Hydrologic High and
\end{flushleft}


\begin{flushleft}
Low Extremes, Regional Frequency Analysis, Stochastic processes and
\end{flushleft}


\begin{flushleft}
modelling of stochastic time series; Markov Chains and Probabilistic
\end{flushleft}


\begin{flushleft}
Theory of Reservoir Storages.
\end{flushleft}





\begin{flushleft}
CVL734 Advanced Hydraulics
\end{flushleft}


\begin{flushleft}
3 Credits (3-0-0)
\end{flushleft}


\begin{flushleft}
Energy and Monument principles in open channel, Curvilinear Flows,
\end{flushleft}


\begin{flushleft}
Backwater computations, Controls, Rapidly varied flows, Spatially
\end{flushleft}


\begin{flushleft}
varied flows, Unsteady flow, Surges, Flood wave passage, Roll waves,
\end{flushleft}


\begin{flushleft}
Sediments transport, Incipient motion criteria, Resistance to flow
\end{flushleft}


\begin{flushleft}
and bed forms, Bed load theory, Stratified flows, Fluvial Systems,
\end{flushleft}


\begin{flushleft}
Industrial Hydraulics.
\end{flushleft}





\begin{flushleft}
CVL735 Finite Element in Water Resources
\end{flushleft}


\begin{flushleft}
3 Credits (3-0-0)
\end{flushleft}


\begin{flushleft}
Introduction to finite element method, Mathematical concepts and
\end{flushleft}


\begin{flushleft}
weighted residual techniques, Spatical discretization, Shape functions,
\end{flushleft}


\begin{flushleft}
Isoparametric elements, Explicit and implicit time marching schemes,
\end{flushleft}


\begin{flushleft}
Equation assembly and solution techniques, Application: Navier-Stokes
\end{flushleft}


\begin{flushleft}
equations, dispersion of pollutants into ground and surface water,
\end{flushleft}


\begin{flushleft}
Flow through earthen dams, seepage beneath a hydraulic structure,
\end{flushleft}


\begin{flushleft}
Groundwater flow in confined and unconfined aquifers.
\end{flushleft}





\begin{flushleft}
CVL736 Soft Computing Techniques in Water Resources
\end{flushleft}


\begin{flushleft}
3 Credits (2-0-2)
\end{flushleft}





\begin{flushleft}
Artificial Intelligence; Expert Systems; Artificial Neural Networks:
\end{flushleft}


\begin{flushleft}
Introduction, Training, Applications in Hydrology; Genetic Algorithms;
\end{flushleft}


\begin{flushleft}
Fuzzy Logic Systems, Fuzzy Set Theory, Predictive and Descriptive Data
\end{flushleft}


\begin{flushleft}
Mining; Classification Methods: Decision trees, NN, Bayesian, ANN,
\end{flushleft}


\begin{flushleft}
SVM, Applications; Association Analysis; Cluster Analysis - K-means,
\end{flushleft}


\begin{flushleft}
Fuzzy, Self-Organising maps; Anomaly detection; Applications in Water
\end{flushleft}


\begin{flushleft}
Resources - Forecasting, Regionalization.
\end{flushleft}





\begin{flushleft}
CVL737 Environmental Dynamics and Management
\end{flushleft}


\begin{flushleft}
3 Credits (3-0-0)
\end{flushleft}


\begin{flushleft}
Environmental property and processes, Environmental simulation
\end{flushleft}


\begin{flushleft}
models, Elements of environmental impact analysis, Impact
\end{flushleft}


\begin{flushleft}
assessment methodologies, Framework of environmental assessment,
\end{flushleft}


\begin{flushleft}
Environmental impact of water resources projects, Assessment of
\end{flushleft}


\begin{flushleft}
hydrological hazards, Environmental management, Case studies.
\end{flushleft}





\begin{flushleft}
CVL738 Economic Aspects of Water Resources
\end{flushleft}


\begin{flushleft}
Development
\end{flushleft}


\begin{flushleft}
3 Credits (3-0-0)
\end{flushleft}


\begin{flushleft}
Economics of water and development, Basic economic concepts,
\end{flushleft}


\begin{flushleft}
Financial analysis of a project, Pricing concepts, Benefit-cost-sensitivity
\end{flushleft}


\begin{flushleft}
analysis, Capital budgeting and cost allocation, Economics of natural
\end{flushleft}


\begin{flushleft}
resources management, Hydro economic model, Hydro-economic
\end{flushleft}


\begin{flushleft}
risk assessment, Economics of river restoration, Economics of transboundary water resources management.
\end{flushleft}





\begin{flushleft}
CVL740 Pavement Materials and Design of Pavements
\end{flushleft}


\begin{flushleft}
4 Credits (3-0-2)
\end{flushleft}


\begin{flushleft}
Pre-requisites: M.Tech: Nil; B.Tech: Instructor's permission
\end{flushleft}


\begin{flushleft}
Components of pavement structure and its requirements; Materials
\end{flushleft}


\begin{flushleft}
used in pavement construction: aggregate, Portland cement, asphalt,
\end{flushleft}


\begin{flushleft}
Portland cement concrete, asphalt concrete; Aggregates: production,
\end{flushleft}


\begin{flushleft}
properties, testing procedures, gradation and blending; Portland
\end{flushleft}


\begin{flushleft}
cement based materials: mixture design, production, properties,
\end{flushleft}


\begin{flushleft}
testing, construction; Asphalt binder: refining process, properties,
\end{flushleft}


\begin{flushleft}
testing procedures, grading systems; Asphalt concrete mixture
\end{flushleft}


\begin{flushleft}
design: fundamentals of mix design procedure, mixture volumetrics,
\end{flushleft}


\begin{flushleft}
current mix design procedures; Production and construction practices;
\end{flushleft}


\begin{flushleft}
Stresses and strains in pavement system: traffic, environment
\end{flushleft}


\begin{flushleft}
considerations; Design of pavements: new, overlay; Pavement
\end{flushleft}


\begin{flushleft}
performance; Drainage consideration.
\end{flushleft}





\begin{flushleft}
CVL741 Urban and Regional Transportation Planning
\end{flushleft}


\begin{flushleft}
4 Credits (3-0-2)
\end{flushleft}


\begin{flushleft}
Pre-requisites: M.Tech: Nil; B.Tech: Instructor's permission
\end{flushleft}


\begin{flushleft}
Fundamentals of transportation planning. Components of transportation
\end{flushleft}


\begin{flushleft}
system and their interaction. Historical development and current status
\end{flushleft}


\begin{flushleft}
of techniques used in travel demand forecasting; Economic Theory
\end{flushleft}


\begin{flushleft}
of travel demand forecasting; trip generation, trip distribution, mode
\end{flushleft}


\begin{flushleft}
choice, traffic assignment models. Integration of landuse transport
\end{flushleft}





174





\begin{flushleft}
\newpage
Civil Engineering
\end{flushleft}





\begin{flushleft}
models. Comparison and evaluation of various models. Simultaneous
\end{flushleft}


\begin{flushleft}
travel demand models: Parameter Estimation and Validation. Travel
\end{flushleft}


\begin{flushleft}
Data collection and use of surveys. The role of transportation planning
\end{flushleft}


\begin{flushleft}
in the overall regional system. Methodology and models for regional
\end{flushleft}


\begin{flushleft}
transportation system, planning, implementation framework and case
\end{flushleft}


\begin{flushleft}
studies. Applications to passenger and freight movement in urban
\end{flushleft}


\begin{flushleft}
area. Implications for policy formulations and analysis.
\end{flushleft}





\begin{flushleft}
CVL742 Traffic Engineering
\end{flushleft}


\begin{flushleft}
4 Credits (3-0-2)
\end{flushleft}


\begin{flushleft}
Pre-requisites: M.Tech: Nil; B.Tech: Instructor's permission
\end{flushleft}


\begin{flushleft}
Introductory concepts of traffic engineering, road user and vehicle
\end{flushleft}


\begin{flushleft}
characteristics, Road way geometric characteristics, traffic stream
\end{flushleft}


\begin{flushleft}
characteristics, and traffic flow theory basics. Statistical applications
\end{flushleft}


\begin{flushleft}
in traffic engineering. Traffic data collection methods - speed, volume,
\end{flushleft}


\begin{flushleft}
travel time and delay studies. Parking studies. Highway safety and
\end{flushleft}


\begin{flushleft}
statistics. Capacity analysis of freeway and multilane highways fundamental concepts, freeway segment analysis, two-way highways.
\end{flushleft}


\begin{flushleft}
Intersections concepts of intersection control, intersection layout,
\end{flushleft}


\begin{flushleft}
signalization basics, signal timing. Analysis of signals and coordination
\end{flushleft}


\begin{flushleft}
under undersaturated and oversaturated conditions.
\end{flushleft}





\begin{flushleft}
CVL743 Airport Planning and Design
\end{flushleft}


\begin{flushleft}
3 Credits (3-0-0)
\end{flushleft}


\begin{flushleft}
Pre-requisites: M.Tech: Nil; B.Tech: Instructor's permission
\end{flushleft}


\begin{flushleft}
Overview of air transport; Forecasting demand-passenger, freight;
\end{flushleft}


\begin{flushleft}
Aircraft characteristics; Airport planning-requirements site selection,
\end{flushleft}


\begin{flushleft}
layout plan; Geometric design of runway, taxiway and aprons; Airport
\end{flushleft}


\begin{flushleft}
capacity-airside, landside; Passenger terminal-functions, passenger
\end{flushleft}


\begin{flushleft}
and baggage flow; Airport pavement design and drainage; Parking
\end{flushleft}


\begin{flushleft}
and apron design; Air cargo facilities; Air traffic control lighting and
\end{flushleft}


\begin{flushleft}
signing; Airport safety; Environmental impact of airports; Airport
\end{flushleft}


\begin{flushleft}
financing and economic analysis.
\end{flushleft}





\begin{flushleft}
CVL744 Transportation Infrastructure Design
\end{flushleft}


\begin{flushleft}
3 Credits (2-0-2)
\end{flushleft}


\begin{flushleft}
Pre-requisites: M.Tech: Nil; B.Tech: Instructor's permission
\end{flushleft}


\begin{flushleft}
Transportation infrastructure: components, structural and functional
\end{flushleft}


\begin{flushleft}
requirements, capacity, level of service; Highway infrastructure: grade
\end{flushleft}


\begin{flushleft}
intersections, rotaries, interchanges; Railway infrastructure: trackbed
\end{flushleft}


\begin{flushleft}
design, grade-crossing design, embankment, retaining walls; Drainage
\end{flushleft}


\begin{flushleft}
infrastructure: culverts, bridges; Pedestrian infrastructure: pedestrian
\end{flushleft}


\begin{flushleft}
sideways, foot bridges; Miscellaneous: bus and truck terminals, parking
\end{flushleft}


\begin{flushleft}
facilities, guard rails, tunnels, underpasses;.
\end{flushleft}





\begin{flushleft}
Transit Demand; Route planning techniques; Bus Scheduling; Transit
\end{flushleft}


\begin{flushleft}
Corridor identification and planning; Mass Transport Management
\end{flushleft}


\begin{flushleft}
Measures; Integration of Public Transportation Modes. Public transport
\end{flushleft}


\begin{flushleft}
Infrastructure; Case Studies. Multimodal Transportation Systems.
\end{flushleft}





\begin{flushleft}
CVL747 Transportation Safety and Environment
\end{flushleft}


\begin{flushleft}
3 Credits (3-0-0)
\end{flushleft}


\begin{flushleft}
Pre-requisites: M.Tech: Nil; B.Tech: Instructor's permission
\end{flushleft}


\begin{flushleft}
Scientific management techniques in planning, implementing, and
\end{flushleft}


\begin{flushleft}
evaluating highway safety programs, strategies to integrate and
\end{flushleft}


\begin{flushleft}
amplify safety in transportation planning processes., multidisciplinary
\end{flushleft}


\begin{flushleft}
relationships necessary to support effective traffic safety initiatives.
\end{flushleft}


\begin{flushleft}
Traffic Safety as public health problem, Injury indices and costing ,
\end{flushleft}


\begin{flushleft}
emergency care, pollution inventory in urban areas, environment and
\end{flushleft}


\begin{flushleft}
safety standards.
\end{flushleft}





\begin{flushleft}
CVL750 Intelligent Transportation Systems
\end{flushleft}


\begin{flushleft}
3 Credits (3-0-0)
\end{flushleft}


\begin{flushleft}
Pre-requisites: M.Tech: Nil; B.Tech: Instructor's permission
\end{flushleft}


\begin{flushleft}
Introduction to Intelligent Transportation Systems (ITS); ITS
\end{flushleft}


\begin{flushleft}
Organizational Issues, the fundamental concepts of Intelligent
\end{flushleft}


\begin{flushleft}
Transportation Systems (ITS) to students with interest in engineering,
\end{flushleft}


\begin{flushleft}
transportation systems, communication systems, vehicle technologies,
\end{flushleft}


\begin{flushleft}
transportation planning, transportation policy, and urban planning.
\end{flushleft}


\begin{flushleft}
ITS in transportation infrastructure and vehicles, that improve
\end{flushleft}


\begin{flushleft}
transportation safety, productivity, environment, and travel reliability.
\end{flushleft}


\begin{flushleft}
Mobile device applications of ITS such as trip planners, ETA s of public
\end{flushleft}


\begin{flushleft}
transit vehicles.
\end{flushleft}





\begin{flushleft}
CVD753 Minor Project in Transportation Engineering
\end{flushleft}


\begin{flushleft}
3 Credits (0-0-6)
\end{flushleft}


\begin{flushleft}
Pre-requisites: M.Tech.: Nil
\end{flushleft}


\begin{flushleft}
CVS754 Independent Study
\end{flushleft}


\begin{flushleft}
3 Credits (0-0-6)
\end{flushleft}


\begin{flushleft}
Pre-requisites: Instructor's permission
\end{flushleft}


\begin{flushleft}
CVD756 Minor Project in Structural Engineering
\end{flushleft}


\begin{flushleft}
3 Credits (0-0-6)
\end{flushleft}


\begin{flushleft}
The course content will be decided by the concerned faculty member
\end{flushleft}


\begin{flushleft}
(supervisor) who will be assigning the research project to the students
\end{flushleft}


\begin{flushleft}
registered for this course.
\end{flushleft}





\begin{flushleft}
CVL756 Advanced Structural Analysis
\end{flushleft}


\begin{flushleft}
3 Credits (3-0-0)
\end{flushleft}





\begin{flushleft}
CVL745 Modeling of Pavement Materials
\end{flushleft}


\begin{flushleft}
3 Credits (2-0-2)
\end{flushleft}


\begin{flushleft}
Pre-requisites: M.Tech: Nil; B.Tech: Instructor's permission
\end{flushleft}


\begin{flushleft}
Role of constitutive modeling; Laboratory testing in relation to
\end{flushleft}


\begin{flushleft}
constitutive modeling: elastic modulus, resilient modulus, complex
\end{flushleft}


\begin{flushleft}
modulus, creep, rheological tests; Introduction to continuum
\end{flushleft}


\begin{flushleft}
mechanics: strain tensor, stress tensor, isotropy, anisotropy,
\end{flushleft}


\begin{flushleft}
constitutive relationships; Factors affecting material behavior:
\end{flushleft}


\begin{flushleft}
temperature, rate, time, confining pressure; Unbound materials:
\end{flushleft}


\begin{flushleft}
soil, aggregate; Bound materials: binding using asphalt, water, lime,
\end{flushleft}


\begin{flushleft}
polymer, fly ash, cement; Constitutive models: unbound materials,
\end{flushleft}


\begin{flushleft}
bound materials; Field performance of pavement materials: fatigue,
\end{flushleft}


\begin{flushleft}
rutting, temperature issues, moisture damage, permeability; Transfer
\end{flushleft}


\begin{flushleft}
functions to relate laboratory performance with field performance.
\end{flushleft}





\begin{flushleft}
CVL746 Public Transportation Systems
\end{flushleft}


\begin{flushleft}
3 Credits (3-0-0)
\end{flushleft}


\begin{flushleft}
Pre-requisites: M.Tech: Nil; B.Tech: Instructor's permission
\end{flushleft}


\begin{flushleft}
This course discusses the role of urban public transportation modes,
\end{flushleft}


\begin{flushleft}
focusing on bus and rail systems. Operational and Technological
\end{flushleft}


\begin{flushleft}
characteristics are described, along with their impacts on capacity,
\end{flushleft}


\begin{flushleft}
service quality, and cost. Current practice and methods for data
\end{flushleft}


\begin{flushleft}
collection and analysis, performance evaluation, route and network
\end{flushleft}


\begin{flushleft}
design, frequency determination, and vehicle and crew scheduling
\end{flushleft}


\begin{flushleft}
are covered. Main topics include: Transit System; Estimation of
\end{flushleft}





\begin{flushleft}
Matrix methods for 3-D skeletal structures: force and displacement
\end{flushleft}


\begin{flushleft}
methods including analysis using substructures, static condensation.
\end{flushleft}


\begin{flushleft}
Computational aspects including in plane rigidity of slab, non-prismatic
\end{flushleft}


\begin{flushleft}
members, and shear deformation effects. Non-linear analysis: second
\end{flushleft}


\begin{flushleft}
order and elastoplastic analysis. Energy approaches. Analysis of plates
\end{flushleft}


\begin{flushleft}
and singly curved shells.
\end{flushleft}





\begin{flushleft}
CVP756 Structural Engineering Laboratory
\end{flushleft}


\begin{flushleft}
3 Credits (0-0-6)
\end{flushleft}


\begin{flushleft}
Concrete: Concrete mix-design Evaluation of stress-strain response
\end{flushleft}


\begin{flushleft}
of plain, self-compacting and high-performance concrete; Behaviour
\end{flushleft}


\begin{flushleft}
of RC members under axial, flexure, shear torsion, and interaction;
\end{flushleft}


\begin{flushleft}
Behavior of slabs, Non-destructing testing. Response of structures
\end{flushleft}


\begin{flushleft}
and its elements against extreme loading events. Model testing:
\end{flushleft}


\begin{flushleft}
Models of plates, shells, and frames; Free and forced vibrations;
\end{flushleft}


\begin{flushleft}
Evaluation of dynamic modulus; Beam vibrations; Vibration isolation;
\end{flushleft}


\begin{flushleft}
Shear wall building model; Time and frequency-domain study. Smart
\end{flushleft}


\begin{flushleft}
materials; Photogrammetry for Displacement Measurement; Vibration
\end{flushleft}


\begin{flushleft}
Characteristics of RC Beams using Piezoelectric Sensors etc.
\end{flushleft}





\begin{flushleft}
CVS756 Independent Study (CES)
\end{flushleft}


\begin{flushleft}
3 Credits (0-0-6)
\end{flushleft}


\begin{flushleft}
Course content will be decided by the concerned faculty member of
\end{flushleft}


\begin{flushleft}
structural engineering.
\end{flushleft}





175





\begin{flushleft}
\newpage
Civil Engineering
\end{flushleft}





\begin{flushleft}
CVD757 Major Project Part-I (CES)
\end{flushleft}


\begin{flushleft}
9 Credits (0-0-18)
\end{flushleft}


\begin{flushleft}
Pre-requisites: programme core Credits+minimum 24 credits
\end{flushleft}


\begin{flushleft}
CVL757 Finite Element Methods in Structural
\end{flushleft}


\begin{flushleft}
Engineering
\end{flushleft}


\begin{flushleft}
3 Credits (2-0-2)
\end{flushleft}


\begin{flushleft}
Review of principles of virtual work and minimum potential energy.
\end{flushleft}


\begin{flushleft}
Elements of theory of elasticity. Finite element (FE) techniques
\end{flushleft}


\begin{flushleft}
for linear and static problems. Developing various types of finite
\end{flushleft}


\begin{flushleft}
elements: 1-D, 2-D, and 3-D. Formulating displacement and shape
\end{flushleft}


\begin{flushleft}
functions. Variational and weighted residual techniques. Higher order/
\end{flushleft}


\begin{flushleft}
isoparametric formulation for truss, beam, frame, plate, and shell
\end{flushleft}


\begin{flushleft}
elements. Numerical solution procedures and computational aspects.
\end{flushleft}


\begin{flushleft}
Applications to structures such as dams, frames, shear walls, grid floors,
\end{flushleft}


\begin{flushleft}
rafts etc. Algorithms for FE problem solving and commercial software
\end{flushleft}


\begin{flushleft}
modeling issues. Application of FE methods to solve thermal problems.
\end{flushleft}





\begin{flushleft}
CVD758 Major Project Part-II (CES)
\end{flushleft}


\begin{flushleft}
9 Credits (0-0-18)
\end{flushleft}


\begin{flushleft}
Pre-requisites: CVD757 must be passed
\end{flushleft}





\begin{flushleft}
Plastic design: Plate instabilities, Local buckling, Section classifications;
\end{flushleft}


\begin{flushleft}
Structural stability: Global buckling, Member and frames under
\end{flushleft}


\begin{flushleft}
axial and combined loading; Sway and non-sway frames; Design of
\end{flushleft}


\begin{flushleft}
members under combined bending, shear and torsion; Connections:
\end{flushleft}


\begin{flushleft}
Simple, Semi-rigid, Rigid; Plates girders: Simple post-critical theory,
\end{flushleft}


\begin{flushleft}
Tension-field theory, Section design, Stiffener requirements; Gantry
\end{flushleft}


\begin{flushleft}
girder; Grillage foundation; Earthquake-resistant design and detailing;
\end{flushleft}


\begin{flushleft}
Fire-resistant design; Fatigue-resistant design.
\end{flushleft}





\begin{flushleft}
CVL762 Earthquake Analysis and Design
\end{flushleft}


\begin{flushleft}
3 Credits (3-0-0)
\end{flushleft}


\begin{flushleft}
Seismology; Seismic Risk and Hazard; Soil Dynamics and Seismic Inputs
\end{flushleft}


\begin{flushleft}
to Structures; Response Spectrum Analysis (RSA); Special Analysis;
\end{flushleft}


\begin{flushleft}
Nonlinear and Push-Over Analysis; Dynamic Soil-Structure Interaction
\end{flushleft}


\begin{flushleft}
(SSI); Earthquake Resistant Design Philosophy; Performance Based
\end{flushleft}


\begin{flushleft}
Earthquake Engineering; Code Provisions for Seismic Design of
\end{flushleft}


\begin{flushleft}
Structures; Retrofitting and Strengthening of Structures; Concept
\end{flushleft}


\begin{flushleft}
of Base Isolation Design and Structural Vibration Control; Advanced
\end{flushleft}


\begin{flushleft}
Topics in Earthquake Engineering.
\end{flushleft}





\begin{flushleft}
CVL763 Analytical and Numerical Methods for Structural
\end{flushleft}


\begin{flushleft}
Engineering
\end{flushleft}


\begin{flushleft}
3 Credits (3-0-0)
\end{flushleft}





\begin{flushleft}
CVL758 Solid Mechanics in Structural Engineering
\end{flushleft}


\begin{flushleft}
3 Credits (3-0-0)
\end{flushleft}


\begin{flushleft}
Pre-requisites: UG/Dual- 120 credits
\end{flushleft}


\begin{flushleft}
Introduction; Historical developments; Theory of stress; Kinematics;
\end{flushleft}


\begin{flushleft}
Isotropic/ anisotropic linear elastic solids; Axioms of constitutive
\end{flushleft}


\begin{flushleft}
equations; Finite isotropic elasticity; Hypo/ hyperelasticity; Hardening
\end{flushleft}


\begin{flushleft}
plasticity; Viscoelasticity; Boundary Value Problems (BVPs); Plane
\end{flushleft}


\begin{flushleft}
elasticity; Polar coordinates; Torsion and bending of prismatic bars
\end{flushleft}


\begin{flushleft}
with general section; Elastic wave propagation; Current trends.
\end{flushleft}





\begin{flushleft}
CVL759 Structural Dynamics
\end{flushleft}


\begin{flushleft}
3 Credits (3-0-0)
\end{flushleft}


\begin{flushleft}
Theory of structural dynamics and vibration analysis. Free and forced
\end{flushleft}


\begin{flushleft}
vibration of single degree of freedom (SDOF) systems, load regimes
\end{flushleft}


\begin{flushleft}
and response to harmonic, periodic, impulsive, and general dynamic
\end{flushleft}


\begin{flushleft}
loading. Response of SDOF to earthquake and response spectrum
\end{flushleft}


\begin{flushleft}
concept. Damping in structures and its evaluation. Free and forced
\end{flushleft}


\begin{flushleft}
vibration of lumped multi degree of freedom (MDOF) structures.
\end{flushleft}


\begin{flushleft}
Methods for obtaining natural frequencies and mode shapes. Normal
\end{flushleft}


\begin{flushleft}
mode theory; mode combination rules; dynamic response evaluation.
\end{flushleft}


\begin{flushleft}
Force excited and base excited dynamical systems. Time domain
\end{flushleft}


\begin{flushleft}
analysis using numerical integration scheme. Free and forced vibration
\end{flushleft}


\begin{flushleft}
of continuous systems. Frequency domain analysis of dynamical
\end{flushleft}


\begin{flushleft}
systems. Introduction to advanced topics in structural dynamics.
\end{flushleft}





\begin{flushleft}
CVL760 Theory of Concrete Structures
\end{flushleft}


\begin{flushleft}
3 Credits (3-0-0)
\end{flushleft}


\begin{flushleft}
Introduction: Historical developments, Material properties;
\end{flushleft}


\begin{flushleft}
Cracked concrete members under flexural moment and axial force;
\end{flushleft}


\begin{flushleft}
Deformations and collapse; M-P interaction. Beams without stirrups
\end{flushleft}


\begin{flushleft}
under flexural and torsional shear: Morsch and Regan theories; Skewbending theory. Beams with stirrups under flexural and torsional
\end{flushleft}


\begin{flushleft}
shear: Plane and space truss analogies, Modified compression field
\end{flushleft}


\begin{flushleft}
theory, Unified theory, P-M-V-T interaction; Strut and tie model;
\end{flushleft}


\begin{flushleft}
Cracking: Bond slip, Development length, Tension stiffening, Durability
\end{flushleft}


\begin{flushleft}
detailing; Serviceability: Elastic, creep and shrinkage deformations;
\end{flushleft}


\begin{flushleft}
Elastic analysis: Redistribution of moments; Plastic analysis: Inelastic
\end{flushleft}


\begin{flushleft}
and hysteretic behaviour, Limit design, Confined concrete: Ductility
\end{flushleft}


\begin{flushleft}
detailing requirements; Buckling of columns; Concrete slabs: Yield
\end{flushleft}


\begin{flushleft}
line theory, Strip Theory; Reliability and safety: Limit state design
\end{flushleft}


\begin{flushleft}
method, Target reliability; Current trends: Constitutive modelling,
\end{flushleft}


\begin{flushleft}
Capacity design, Finite element analysis.
\end{flushleft}





\begin{flushleft}
CVL761 Theory of Steel Structures
\end{flushleft}


\begin{flushleft}
3 Credits (3-0-0)
\end{flushleft}


\begin{flushleft}
Structural steel: Classifications, Grades, Behavioural characteristics,
\end{flushleft}


\begin{flushleft}
Plasticity and hardening; Material models: Simple, Rigid, Power
\end{flushleft}


\begin{flushleft}
function, Smooth hysteretic; Design methodology: Allowable, Limit
\end{flushleft}


\begin{flushleft}
state, Ultimate; Methods of analysis including second-order effects;
\end{flushleft}





\begin{flushleft}
Introduction: Mathematical foundations of structural theory. Linear
\end{flushleft}


\begin{flushleft}
algebra: vector spaces and linear transformations. Linear differential
\end{flushleft}


\begin{flushleft}
equations and function spaces. Partial differential equations; Elliptic,
\end{flushleft}


\begin{flushleft}
parabolic and hyperbolic PDEs. Nonlinear differential equations.
\end{flushleft}


\begin{flushleft}
Gaussian Elimination; Factorization Techniques - LU, Cholesky; Iterative
\end{flushleft}


\begin{flushleft}
Methods of Solution of Linear Simultaneous Equations. Properties of
\end{flushleft}


\begin{flushleft}
Eigenvalues and Eigenvectors; Similarity Transforms; Diagonalization
\end{flushleft}


\begin{flushleft}
and Numerical Techniques to Compute Eigenvalues - Vector Iteration,
\end{flushleft}


\begin{flushleft}
QR algorithm, Jacobi Method. Time Marching Schemes (Step by Step
\end{flushleft}


\begin{flushleft}
Solutions); Euler's Method; Runge Kutta Method; Newmark Beta
\end{flushleft}


\begin{flushleft}
Method. Numerical Solution of Boundary Value Problems - Finite
\end{flushleft}


\begin{flushleft}
Difference Method, Explicit and Implicit Approaches; Method of
\end{flushleft}


\begin{flushleft}
Weighted Residuals, Galerkin's Method. Numerical Integration: GaussLegendre Method, Newton-Cotes Method. Regression Analysis and
\end{flushleft}


\begin{flushleft}
Curve Fitting. Applications of mathematical and numerical methods to
\end{flushleft}


\begin{flushleft}
static, dynamic and stability analysis of elastic structures and cables.
\end{flushleft}





\begin{flushleft}
CVL764 Blast Resistant Design of Structures
\end{flushleft}


\begin{flushleft}
3 Credits (2-0-2)
\end{flushleft}


\begin{flushleft}
Blast Engineering: Explosion Phenomena, Shock Front, Fragmentation,
\end{flushleft}


\begin{flushleft}
Waves, Ground Shock, and Interaction with Structures; Structural
\end{flushleft}


\begin{flushleft}
Analysis for Impulsive Loading; Pressure-Impulse (PI) Diagrams;
\end{flushleft}


\begin{flushleft}
Material Behaviour under High Strain-Rate of Loadings; Blast Resistant
\end{flushleft}


\begin{flushleft}
Design of Structures; Performance-Based Blast Design; Progressive
\end{flushleft}


\begin{flushleft}
Collapse; Anti-Terrorism Planning and Design of Facilities; Blast
\end{flushleft}


\begin{flushleft}
Retrofitting; Indian/ International Standards and Codes of Practice;
\end{flushleft}


\begin{flushleft}
Numerical Analysis Tools for Blast Analysis using Finite Element (FE)
\end{flushleft}


\begin{flushleft}
Software and Hydrocodes.
\end{flushleft}





\begin{flushleft}
CVL765 Concrete Mechanics
\end{flushleft}


\begin{flushleft}
3 Credits (3-0-0)
\end{flushleft}


\begin{flushleft}
Introduction; Rheological modelling of fresh concrete; Flowing
\end{flushleft}


\begin{flushleft}
concrete; Mechanics of hardened concrete: Failure criteria; Constitutive
\end{flushleft}


\begin{flushleft}
equations; Elasto-plasticity, visco-elasticity, fatigue, damage mechanics
\end{flushleft}


\begin{flushleft}
and fracture; Mechanics of hydrating concretes, Durability Mechanics,
\end{flushleft}


\begin{flushleft}
Transport processes; Drying shrinkage; Micromechanics , Numerical
\end{flushleft}


\begin{flushleft}
and analytical homogenisation, poromechanics , Crystalline growths
\end{flushleft}


\begin{flushleft}
and internal microstresses.
\end{flushleft}





\begin{flushleft}
CVL766 Design of Bridge Structures
\end{flushleft}


\begin{flushleft}
3 Credits (3-0-0)
\end{flushleft}


\begin{flushleft}
Introduction, historical/ magnificent bridges; Site Selection, Planning,
\end{flushleft}


\begin{flushleft}
and Type of Bridges, Loads and Forces; Code Provisions for Design
\end{flushleft}


\begin{flushleft}
of Steel and Concrete Bridges; Analysis Methods, Grillage Analogy;
\end{flushleft}


\begin{flushleft}
Theories of Lateral Load Distribution and Design of Superstructure:
\end{flushleft}


\begin{flushleft}
Slab Type, Beam-Slab, and Box Type; Distribution of Externally
\end{flushleft}


\begin{flushleft}
Applied and Self-Induced Horizontal Forces among Bridge Supports in
\end{flushleft}


\begin{flushleft}
Straight, Curved, and Skewed Decks; Continuous Type and Balanced
\end{flushleft}


\begin{flushleft}
Cantilever Type Superstructure; Temperature Stresses in Concrete
\end{flushleft}





176





\begin{flushleft}
\newpage
Civil Engineering
\end{flushleft}





\begin{flushleft}
Bridge Deck; Different Types of Foundations: Open, Pile, and Well
\end{flushleft}


\begin{flushleft}
Foundations; Choice of Foundation for Abutments and Piers; Design
\end{flushleft}


\begin{flushleft}
of Abutments, Piers, Pile/ Pier Caps; Effect of Differential Settlement
\end{flushleft}


\begin{flushleft}
of Supports; Bridge Bearings; Expansion Joints for Bridge Decks;
\end{flushleft}


\begin{flushleft}
Vibration of Bridge Decks; Parapet and Railings for Highway Bridges;
\end{flushleft}


\begin{flushleft}
Construction Methods; Segmental Construction of Bridges; Inspection
\end{flushleft}


\begin{flushleft}
and Maintenance of Bridges; Health Monitoring and Evaluation of
\end{flushleft}


\begin{flushleft}
Existing Bridges; Bridge Failure: Case Studies.
\end{flushleft}





\begin{flushleft}
CVL767 Design of Fiber Reinforced Composite Structures
\end{flushleft}


\begin{flushleft}
3 Credits (3-0-0)
\end{flushleft}


\begin{flushleft}
Introduction; Types of structural fibers: matrix, fiber and interface;
\end{flushleft}


\begin{flushleft}
Fiber reinforced concrete (FRC); High-performance concrete;
\end{flushleft}


\begin{flushleft}
Stress transfer, Bond, Pull-out, Toughening mechanism; Fracture
\end{flushleft}


\begin{flushleft}
mechanics; Modeling of tensile and flexural behaviours; Behaviour
\end{flushleft}


\begin{flushleft}
under compression; Shear failure theory; Behaviour under seismic
\end{flushleft}


\begin{flushleft}
loading; Composite structural design: Design spirals, Selection Criteria
\end{flushleft}


\begin{flushleft}
configurations; Laminate design; Mathematical analysis of laminates;
\end{flushleft}


\begin{flushleft}
Design of single skin panels, Design of composite stiffeners.
\end{flushleft}





\begin{flushleft}
CVL768 Design of Masonry Structures
\end{flushleft}


\begin{flushleft}
3 Credits (3-0-0)
\end{flushleft}


\begin{flushleft}
Introduction and Historical Perspective; Masonry Materials; Masonry
\end{flushleft}


\begin{flushleft}
Design Approaches; Overview of Load Conditions; Compression
\end{flushleft}


\begin{flushleft}
Behavior of Masonry; Masonry Wall Configurations; Distribution of
\end{flushleft}


\begin{flushleft}
Lateral Forces; Flexural Strength of Reinforced Masonry Members: Inplane and Out-of-plane Loading, Interactions; Structural Wall; Columns
\end{flushleft}


\begin{flushleft}
and Pilasters; Retaining Wall; Pier and Foundation; Shear Strength
\end{flushleft}


\begin{flushleft}
and Ductility of Reinforced Masonry Members; Prestressed Masonry;
\end{flushleft}


\begin{flushleft}
Stability of Walls; Coupling of Masonry Walls, Openings, Columns,
\end{flushleft}


\begin{flushleft}
Beams; Elastic and inelastic analysis; Modelling Techniques; Static
\end{flushleft}


\begin{flushleft}
Push-Over Analysis and use of Capacity Design Spectra.
\end{flushleft}





\begin{flushleft}
CVL769 Design of Tall Buildings
\end{flushleft}


\begin{flushleft}
3 Credits (3-0-0)
\end{flushleft}


\begin{flushleft}
Structural systems and general concepts of tall buildings; Various
\end{flushleft}


\begin{flushleft}
methods of structural analysis; Gravity systems for steel, concrete, and
\end{flushleft}


\begin{flushleft}
composite buildings; Lateral systems for steel, concrete, and composite
\end{flushleft}


\begin{flushleft}
buildings; Interaction of frames and shear walls; Simultaneous and
\end{flushleft}


\begin{flushleft}
sequential loading; Differential shortening of columns; P-$\Delta$ effects;
\end{flushleft}


\begin{flushleft}
Effect of openings; Foundations and foundation-superstructure
\end{flushleft}


\begin{flushleft}
interaction; Wind/ earthquake effects and design for ductility; Damping
\end{flushleft}


\begin{flushleft}
systems; Asymmetric structures and twisting of frames.
\end{flushleft}





\begin{flushleft}
CVL770 Prestressed and Composite Structures
\end{flushleft}


\begin{flushleft}
3 Credits (2-0-2)
\end{flushleft}


\begin{flushleft}
Introduction; Need, Advantages, and Disadvantages; High Strength
\end{flushleft}


\begin{flushleft}
Materials; Pretensioning and Post-Tensioning Methods; Prestressing
\end{flushleft}


\begin{flushleft}
Methods; Prestressing Systems and Devices; Camber, Deflections,
\end{flushleft}


\begin{flushleft}
and Cable Profiles/ Layouts; Load-Balancing; Codes and Standards;
\end{flushleft}


\begin{flushleft}
Prestressed Concrete Members - Flexure, Shear, Torsion Behaviors;
\end{flushleft}


\begin{flushleft}
Design Methods and Code Provisions; Strain Compatibility Method;
\end{flushleft}


\begin{flushleft}
Pressure/ Thrust Line; Pre-Tensioning; Grouted/ Bonded and
\end{flushleft}


\begin{flushleft}
Ungrouted/ Unbonded Post-Tensioning; Partial Prestressing; Bursting
\end{flushleft}


\begin{flushleft}
Stresses; Anchorage Zone (End Block Design); Transmission and
\end{flushleft}


\begin{flushleft}
Transfer Length; De-Bonding and Draping of Prestressing Tendons;
\end{flushleft}


\begin{flushleft}
Camber, Deflection, and Ductility; External Prestressing; DeCompression; Losses in Prestress; Bearing and Bond Stresses; Case
\end{flushleft}


\begin{flushleft}
Studies of Prestressed Concrete Bridge Design and Practices.
\end{flushleft}


\begin{flushleft}
Need of Composite Construction; Analysis of Indeterminate and
\end{flushleft}


\begin{flushleft}
Composite Structures; Design Methods for Composite Beams, Slabs,
\end{flushleft}


\begin{flushleft}
Columns, Box-girders, Shear Studs etc.
\end{flushleft}





\begin{flushleft}
CVC771 Seminar in Construction Technology and
\end{flushleft}


\begin{flushleft}
Management-I
\end{flushleft}


\begin{flushleft}
0 Credits (0-0-2)/Compulsory Audit
\end{flushleft}


\begin{flushleft}
CVD771 Minor Project (CEC)
\end{flushleft}


\begin{flushleft}
3 Credits (0-0-6)
\end{flushleft}


\begin{flushleft}
CVL771 Advanced Concrete Technology
\end{flushleft}


\begin{flushleft}
3 Credits (3-0-0)
\end{flushleft}





\begin{flushleft}
Hydration of cements and microstructural development, Mineral
\end{flushleft}


\begin{flushleft}
additives, Chemical admixtures, Rheology of concrete, Creep and
\end{flushleft}


\begin{flushleft}
relaxation, Shrinkage, cracking and volume stability, deterioration
\end{flushleft}


\begin{flushleft}
processes, special concretes, Advanced characterisation techniques,
\end{flushleft}


\begin{flushleft}
sustainability issues in concreting, Modelling properties of concrete.
\end{flushleft}





\begin{flushleft}
CVP771 Construction Technology Laboratory
\end{flushleft}


\begin{flushleft}
1.5 Credits (0-0-3)
\end{flushleft}


\begin{flushleft}
Tests related to quality control at site, in-situ tests, tests related to damage
\end{flushleft}


\begin{flushleft}
and deterioration assessment, performance monitoring of structures.
\end{flushleft}





\begin{flushleft}
CVS771 Independent Study (CEC)
\end{flushleft}


\begin{flushleft}
3 Credits (0-0-6)
\end{flushleft}


\begin{flushleft}
CVC772 Seminar in Construction Technology and
\end{flushleft}


\begin{flushleft}
Management-II
\end{flushleft}


\begin{flushleft}
0 Credits (0-0-2)/Compulsory Audit
\end{flushleft}


\begin{flushleft}
CVD772 Major Project Part-I (CEC)
\end{flushleft}


\begin{flushleft}
9 Credits (0-0-18)	
\end{flushleft}


\begin{flushleft}
CVL772 Construction Project Management
\end{flushleft}


\begin{flushleft}
3 Credits (3-0-0)
\end{flushleft}


\begin{flushleft}
Introduction to construction project management - CPM, PERT, PDM,
\end{flushleft}


\begin{flushleft}
LOB. Scope management, WBS, PDRI. Time and cost management,
\end{flushleft}


\begin{flushleft}
material related management - purchase \& inventory control, timecost-resource optimization, quality, safety - planning \& control.
\end{flushleft}


\begin{flushleft}
Labor productivity variations, productivity improvement - work study.
\end{flushleft}


\begin{flushleft}
Measuring project progress \& performance - EVA \& ES. Identification
\end{flushleft}


\begin{flushleft}
of risks and impact. Management Information systems.
\end{flushleft}





\begin{flushleft}
CVP772 Computational Laboratory for Construction
\end{flushleft}


\begin{flushleft}
Management
\end{flushleft}


\begin{flushleft}
1.5 Credits (0-0-3)
\end{flushleft}


\begin{flushleft}
Introduction to construction project models - analytical and numerical.
\end{flushleft}


\begin{flushleft}
Application software for project planning, scheduling \& control.
\end{flushleft}


\begin{flushleft}
Programming exercises for estimation, network planning and control,
\end{flushleft}


\begin{flushleft}
LP in construction.
\end{flushleft}


\begin{flushleft}
MATLAB Programming in linear and non-linear programming.
\end{flushleft}





\begin{flushleft}
CVD773 Major Project Part-II (CEC)
\end{flushleft}


\begin{flushleft}
12 Credits (0-0-24)
\end{flushleft}


\begin{flushleft}
CVL773 Quantitative Methods in Construction
\end{flushleft}


\begin{flushleft}
Management
\end{flushleft}


\begin{flushleft}
3 Credits (3-0-0)
\end{flushleft}


\begin{flushleft}
Introduction and concepts of probability and statistics, Linear
\end{flushleft}


\begin{flushleft}
programming, Transportation and assignment problems. Dynamic
\end{flushleft}


\begin{flushleft}
programming, Queuing theory, Decision theory, Games theory.
\end{flushleft}


\begin{flushleft}
Simulations applied to construction, Modifications and improvement
\end{flushleft}


\begin{flushleft}
on CPM/PERT techniques.
\end{flushleft}





\begin{flushleft}
CVL774 Construction Contract Management
\end{flushleft}


\begin{flushleft}
3 Credits (3-0-0)
\end{flushleft}


\begin{flushleft}
Professional Ethics, Duties and Responsibilities of Parties. Owner's
\end{flushleft}


\begin{flushleft}
and contractor's estimate, Bidding Models and Bidding Strategies,
\end{flushleft}


\begin{flushleft}
Qualification of Bidders. Tendering and Contractual procedures,
\end{flushleft}


\begin{flushleft}
Indian Contract Act 1872, Definition of Contract and its Applicability,
\end{flushleft}


\begin{flushleft}
Types of Contracts, Clauses in Domestic and International Contracts
\end{flushleft}


\begin{flushleft}
- CPWD, MES, FIDIC, AIA, NEC, JCT, etc. Contract Administration,
\end{flushleft}


\begin{flushleft}
Delay Protocol, Change Orders Analysis, Claim Management and
\end{flushleft}


\begin{flushleft}
Compensation, Disputes and Resolution Techniques, Arbitration and
\end{flushleft}


\begin{flushleft}
Conciliation Act 1996, Arbitration Case Studies.
\end{flushleft}





\begin{flushleft}
CVL775 Construction Economics and Finance
\end{flushleft}


\begin{flushleft}
3 Credits (3-0-0)
\end{flushleft}


\begin{flushleft}
Engineering economics, Time value of money, discounted cash flow,
\end{flushleft}


\begin{flushleft}
NPV, ROR, PI. Basis of comparison, Incremental rate of return, Benefitcost analysis, Replacement analysis, Break even analysis. Depreciation
\end{flushleft}


\begin{flushleft}
and amortization. Taxation and inflation, Evaluation of profit before and
\end{flushleft}


\begin{flushleft}
after tax. Risks and uncertainties and management decision in capital
\end{flushleft}


\begin{flushleft}
budgeting. Working capital management, financial plan and multiple
\end{flushleft}





177





\begin{flushleft}
\newpage
Civil Engineering
\end{flushleft}





\begin{flushleft}
source of finance. Budgeting and budgetary control, Performance
\end{flushleft}


\begin{flushleft}
budgeting. Profit \& Loss, Balance Sheet, Income statement, Ratio
\end{flushleft}


\begin{flushleft}
analysis, Appraisal through financial statements, International finance,
\end{flushleft}


\begin{flushleft}
forward, futures and swap. Practical problems and case studies.
\end{flushleft}





\begin{flushleft}
A course which will vary from year to year to study new and
\end{flushleft}


\begin{flushleft}
exciting developments in the broad spectrum of Geotechnical and
\end{flushleft}


\begin{flushleft}
Geoenvironmental Engineering. The course will also focus on new
\end{flushleft}


\begin{flushleft}
offshoots of Geotechnical and Geoenvironmental Engineering.
\end{flushleft}





\begin{flushleft}
CVD776 Minor Project (CET)
\end{flushleft}


\begin{flushleft}
3 Credits (0-0-6)
\end{flushleft}





\begin{flushleft}
CVP800 Geoenvironmental and Geotechnical Engg. Lab
\end{flushleft}


\begin{flushleft}
3 Credits (0-0-6)
\end{flushleft}





\begin{flushleft}
CVL776 Construction Practices and Equipment
\end{flushleft}


\begin{flushleft}
3 Credits (3-0-0)
\end{flushleft}


\begin{flushleft}
Form work design and scaffolding, slipform and other moving forms, Shoring,
\end{flushleft}


\begin{flushleft}
Reshoring, and Backshoring in multistoreyed Building construction.
\end{flushleft}


\begin{flushleft}
Prestressing, Steel and composites construction methods: Fabrication
\end{flushleft}


\begin{flushleft}
and erection of structures including heavy structures, Prefab construction,
\end{flushleft}


\begin{flushleft}
Industrialized construction, Modular coordination. Special construction
\end{flushleft}


\begin{flushleft}
methods: High rise construction, Bridge construction including segmental
\end{flushleft}


\begin{flushleft}
construction, incremental construction and push launching techniques.
\end{flushleft}


\begin{flushleft}
Factors affecting selection of equipment - technical and economic, Analysis
\end{flushleft}


\begin{flushleft}
of production outputs and costs, Characteristics and performances of
\end{flushleft}


\begin{flushleft}
equipment for major civil engineering activities such as Earth moving,
\end{flushleft}


\begin{flushleft}
erection, material transport, pile driving, Dewatering, and Concreting.
\end{flushleft}





\begin{flushleft}
Engineering properties and compaction characteristics of waste - coal
\end{flushleft}


\begin{flushleft}
ash, mine tailings. Permeability of clays and bentonite amended soils.
\end{flushleft}


\begin{flushleft}
Physical, Mechanical and Hydraulic Testing of Geosynthetics Landfill
\end{flushleft}


\begin{flushleft}
liner and cover: Evaluation of shear strength parameters of various
\end{flushleft}


\begin{flushleft}
Interfaces and design. Project based laboratory for evaluation of
\end{flushleft}


\begin{flushleft}
engineering properties of soils for design of embankments.
\end{flushleft}





\begin{flushleft}
CVS800 Independent Study
\end{flushleft}


\begin{flushleft}
3 Credits (0-0-6)
\end{flushleft}


\begin{flushleft}
CVD801 Major Project Part-II
\end{flushleft}


\begin{flushleft}
12 Credits (0-0-24)
\end{flushleft}


\begin{flushleft}
CVL801 Constitutive Modelling in Geotechnics
\end{flushleft}


\begin{flushleft}
3 Credits (3-0-0)
\end{flushleft}


\begin{flushleft}
Introduction: fundamental relations, models and soil mechanics.
\end{flushleft}


\begin{flushleft}
Elasticity: Isotropic, anisotropic, soil elasticity. Plasticity and yielding:
\end{flushleft}


\begin{flushleft}
yielding of clays, yielding of sands, slip line fields, introduction to upper
\end{flushleft}


\begin{flushleft}
and lower bounds, selected boundary value problems. Elasto-plastic
\end{flushleft}


\begin{flushleft}
model for soils: elastic volumetric strains, plastic volumetric strains,
\end{flushleft}


\begin{flushleft}
plastic hardening, plastic shear strains, plastic potentials, flow rule.
\end{flushleft}


\begin{flushleft}
Cam clay model: critical state line, shear strength, stress-dilatancy,
\end{flushleft}


\begin{flushleft}
index properties, prediction of conventional soil tests. Applications.
\end{flushleft}





\begin{flushleft}
CVS776 Independent Study (CET)
\end{flushleft}


\begin{flushleft}
3 Credits (0-0-6)
\end{flushleft}


\begin{flushleft}
CVD777 Major Project Part-I (CET)
\end{flushleft}


\begin{flushleft}
9 Credits (0-0-18)
\end{flushleft}


\begin{flushleft}
CVL777 Building Science
\end{flushleft}


\begin{flushleft}
3 Credits (3-0-0)
\end{flushleft}


\begin{flushleft}
Introduction to environmental features relevant to functional design.
\end{flushleft}


\begin{flushleft}
Their measures description and quantification. Periodic nature of
\end{flushleft}


\begin{flushleft}
variation of environmental descriptors. Heat exchange of building
\end{flushleft}


\begin{flushleft}
with environment under diurnal periodic variation temperature and
\end{flushleft}


\begin{flushleft}
modelling. Estimation of hourly internal temperature through CIBS
\end{flushleft}


\begin{flushleft}
method. Thermal Design philosophy and optimization for decision
\end{flushleft}


\begin{flushleft}
variables such as shape, orientation, envelope properties etc. Purpose
\end{flushleft}


\begin{flushleft}
of ventilation, wind and stack effect as driving force. Design for desired
\end{flushleft}


\begin{flushleft}
flow and indoor velocity. Fundamentals of acoustics, Sound ion free field
\end{flushleft}


\begin{flushleft}
and enclosure. External and Internal air borne noise control. Protection
\end{flushleft}


\begin{flushleft}
against structure borne noise. Lighting principles and daylighting. Day
\end{flushleft}


\begin{flushleft}
light factor, and design for desired illumination and glare free lighting.
\end{flushleft}





\begin{flushleft}
CVD810 Major Project Part-I (CEU)
\end{flushleft}


\begin{flushleft}
6 Credits (0-0-12)
\end{flushleft}





\begin{flushleft}
CVD778 Major Project Part-II (CET)
\end{flushleft}


\begin{flushleft}
12 Credits (0-0-24)
\end{flushleft}





\begin{flushleft}
Project planning,Schedule and cost assessment,DPR and GD for
\end{flushleft}


\begin{flushleft}
Major projects,Field visit, Sample collection, Scanline survey and
\end{flushleft}


\begin{flushleft}
seismic survey, Rock characterization, Determination of physical and
\end{flushleft}


\begin{flushleft}
mechanical properties of rocks, Analysis of slopes using GEOSLOPE and
\end{flushleft}


\begin{flushleft}
Analysis of tunnels using Phase2, both using the material properties
\end{flushleft}


\begin{flushleft}
determined through laboratory tests. Design of slopes and tunnels.
\end{flushleft}





\begin{flushleft}
CVL778 Building Services and Maintenance Management
\end{flushleft}


\begin{flushleft}
3 Credits (3-0-0)
\end{flushleft}





\begin{flushleft}
Concepts of functional design of building for fire protection, design of lift
\end{flushleft}


\begin{flushleft}
systems for optimum service. Building service system design. Control
\end{flushleft}


\begin{flushleft}
and intelligent buildings, HVAC, hot and cold water services, waste water
\end{flushleft}


\begin{flushleft}
handling system, electrical services, building maintenance management.
\end{flushleft}





\begin{flushleft}
CVL779 Formwork for Concrete Structures
\end{flushleft}


\begin{flushleft}
3 Credits (3-0-0)
\end{flushleft}


\begin{flushleft}
Requirements and selection for Formwork , Formwork Materials,
\end{flushleft}


\begin{flushleft}
such as Timber, Plywood, Steel, Aluminum Form, Plastic Forms, and
\end{flushleft}


\begin{flushleft}
Accessories, Horizontal and Vertical Formwork Supports; Formwork
\end{flushleft}


\begin{flushleft}
Design Concepts, Illustration of Formwork system for Foundations,
\end{flushleft}


\begin{flushleft}
walls, columns, slab and beams and their design, Formwork for Shells,
\end{flushleft}


\begin{flushleft}
Domes, Folded Plates, Overhead Water Tanks, Natural Draft Cooling
\end{flushleft}


\begin{flushleft}
Tower. Formwork for Bridge Structures, Flying Formwork such as
\end{flushleft}


\begin{flushleft}
Table form, Tunnel form. Slipform, Formwork for Precast Concrete,
\end{flushleft}


\begin{flushleft}
Formwork Management Issues pre award and post award, Formwork
\end{flushleft}


\begin{flushleft}
failures-causes and Case Studies in Formwork Failure, Formwork issues
\end{flushleft}


\begin{flushleft}
in multi-story building construction.
\end{flushleft}





\begin{flushleft}
CVD800 Major Project Part-I
\end{flushleft}


\begin{flushleft}
6 Credits (0-0-12)
\end{flushleft}


\begin{flushleft}
CVL800 Emerging Topics in Geotechnical Engineering
\end{flushleft}


\begin{flushleft}
3 Credits (3-0-0)
\end{flushleft}





\begin{flushleft}
CVL810 Emerging Topics in Rock Engineering and
\end{flushleft}


\begin{flushleft}
Underground Structures
\end{flushleft}


\begin{flushleft}
3 Credits (3-0-0)
\end{flushleft}


\begin{flushleft}
Advanced and state-of-the-art rock engineering topics.
\end{flushleft}





\begin{flushleft}
CVP810 Rock Mechanics Laboratory-II
\end{flushleft}


\begin{flushleft}
3 Credits (0-0-6)
\end{flushleft}


\begin{flushleft}
Pre-requisites: Rock Mechanics Lab-I CVP710
\end{flushleft}





\begin{flushleft}
CVS810 Independent Study (CEU)
\end{flushleft}


\begin{flushleft}
3 Credits (0-0-6)
\end{flushleft}


\begin{flushleft}
CVD811 Major Project Part-II (CEU)
\end{flushleft}


\begin{flushleft}
12 Credits (0-0-24)
\end{flushleft}


\begin{flushleft}
CVL811 Numerical and Computer Methods in
\end{flushleft}


\begin{flushleft}
Geomechanics
\end{flushleft}


\begin{flushleft}
3 Credits (2-0-2)
\end{flushleft}


\begin{flushleft}
Pre-requisites: CVL704 or Equivalent
\end{flushleft}


\begin{flushleft}
Introduction to Numerical Methods, ODEs, PDEs, Equation solution
\end{flushleft}


\begin{flushleft}
techniques, Root finding techniques, Fourier Series, Types of
\end{flushleft}


\begin{flushleft}
geotechnical boundary value problems, Numerical modeling, Numerical
\end{flushleft}


\begin{flushleft}
solution schemes, pros and cons, Programming tools- FORTRAN,
\end{flushleft}


\begin{flushleft}
MATLAB, MATHCAD, Development of programming flowchart.
\end{flushleft}


\begin{flushleft}
Simplified and advanced constitutive models and their calibration:
\end{flushleft}


\begin{flushleft}
Elastic Models, Elasto-plastic Models, Formulation of Elasto-Plastic
\end{flushleft}


\begin{flushleft}
Stiffness Matrix, Governing equations of elastoplasticity, Rock and
\end{flushleft}


\begin{flushleft}
Soil constitutive models.
\end{flushleft}


\begin{flushleft}
Integration of stress-strain equations, Concepts of verification and
\end{flushleft}


\begin{flushleft}
validation, Selection of model input parameters, Integration of loaddisplacement relations, Integration of seepage, consolidation and heat
\end{flushleft}





178





\begin{flushleft}
\newpage
Civil Engineering
\end{flushleft}





\begin{flushleft}
conduction equations, Sturm--Liouville problem, Solution of seepage,
\end{flushleft}


\begin{flushleft}
consolidation, heat conduction and Sturm-Liouville equations using
\end{flushleft}


\begin{flushleft}
finite difference and finite element programming methods, Comparison
\end{flushleft}


\begin{flushleft}
with commercially available software results.
\end{flushleft}





\begin{flushleft}
and operation of treatability studies and microbial growth kinetics,
\end{flushleft}


\begin{flushleft}
microbial toxicity and bioaccumulation studies. Micropollutants
\end{flushleft}


\begin{flushleft}
detection;Package programmes for water and wastewater conveyance,
\end{flushleft}


\begin{flushleft}
treatment and disposal.
\end{flushleft}





\begin{flushleft}
CVL817 Structural Safety and Reliability (PG)
\end{flushleft}


\begin{flushleft}
3 Credits (3-0-0)
\end{flushleft}





\begin{flushleft}
CVL822 Emerging Technologies for Environmental
\end{flushleft}


\begin{flushleft}
Management
\end{flushleft}


\begin{flushleft}
3 Credits (3-0-0)
\end{flushleft}





\begin{flushleft}
Fundamentals of set theory and probability, probability distribution,
\end{flushleft}


\begin{flushleft}
regression analysis, hypothesis testing. Stochastic process and
\end{flushleft}


\begin{flushleft}
its moments and distributions, concepts of safety factors, Safety,
\end{flushleft}


\begin{flushleft}
reliability and risk analysis, first order and second order reliability
\end{flushleft}


\begin{flushleft}
methods, simulation based methods, confidence limits and bayesian
\end{flushleft}


\begin{flushleft}
revision of reliability, reliability based design, examples of reliability
\end{flushleft}


\begin{flushleft}
analysis of structures.
\end{flushleft}





\begin{flushleft}
CVL818 Design of Plates and Shells (PG)
\end{flushleft}


\begin{flushleft}
3 Credits (2-1-0)
\end{flushleft}


\begin{flushleft}
Prismatic folded plate systems. Shell equations. Approximate solutions.
\end{flushleft}


\begin{flushleft}
Analysis and design of cylindrical shells. Approximate design methods
\end{flushleft}


\begin{flushleft}
for doubly curved shells.
\end{flushleft}





\begin{flushleft}
CVL819 Concrete Mechanics (PG)
\end{flushleft}


\begin{flushleft}
3 Credits (3-0-0)
\end{flushleft}


\begin{flushleft}
Introduction; Rheological modelling of fresh concrete; Constitutive
\end{flushleft}


\begin{flushleft}
equations; Nonlinear elasticity, plasticity, visco-elasticity and fracture
\end{flushleft}


\begin{flushleft}
mechanics of hardened concrete; Confinement and ductility; Moisture
\end{flushleft}


\begin{flushleft}
diffusion; Drying shrinkage; Solid and structural mechanics of
\end{flushleft}


\begin{flushleft}
reinforced concrete, Skew bending, modified compression field and
\end{flushleft}


\begin{flushleft}
unified theories of R.C. beams under bending, shear and torsion;
\end{flushleft}


\begin{flushleft}
Bond-slip and phenomenon of cracking in reinforced concrete; Statical
\end{flushleft}


\begin{flushleft}
and dynamical analysis of R.C. Structures; Trends.
\end{flushleft}





\begin{flushleft}
CVL820 Environmental impact assessment
\end{flushleft}


\begin{flushleft}
3 Credits (3-0-0)
\end{flushleft}


\begin{flushleft}
Planning and Management of Environmental Impact Studies. Impact
\end{flushleft}


\begin{flushleft}
indentation methodologies: base line studies, screening, scoping,
\end{flushleft}


\begin{flushleft}
checklist, networks, overlays. Prediction and assessment of impacts
\end{flushleft}


\begin{flushleft}
on the socio-economic environment. Environmental cost benefit
\end{flushleft}


\begin{flushleft}
analysis. Decision methods for evaluation of alternatives. Case
\end{flushleft}


\begin{flushleft}
Studies. Environmental impact assessment at project level, regional
\end{flushleft}


\begin{flushleft}
level, sectoral level, and policy level. Sustainable development;
\end{flushleft}


\begin{flushleft}
Environmental policy in planned, mixed and market economies.
\end{flushleft}


\begin{flushleft}
Preventive environmental management.
\end{flushleft}





\begin{flushleft}
CVP820 Advanced Air Pollution Laboratory
\end{flushleft}


\begin{flushleft}
3 Credits (1-0-4)
\end{flushleft}


\begin{flushleft}
Monitoring of TSP using HVS, Monitoring of PM2.5 using cyclone
\end{flushleft}


\begin{flushleft}
based sampler, Size segregated particle collection and data analysis
\end{flushleft}


\begin{flushleft}
using histogram, inversion program, Personal exposure assessment,
\end{flushleft}


\begin{flushleft}
determination of count and geometric mean diameter, determination
\end{flushleft}


\begin{flushleft}
of chemical species in air samples, Determination of emission factors
\end{flushleft}


\begin{flushleft}
of particle and gases for combustion sources, Determination of TVOC;
\end{flushleft}


\begin{flushleft}
Determination of indoor air quality parameters, determination of
\end{flushleft}


\begin{flushleft}
Bioaerosol; Monitoring and analysis of meteorological parameters.
\end{flushleft}





\begin{flushleft}
CVL821 Industrial Waste Management and Audit
\end{flushleft}


\begin{flushleft}
3 Credits (3-0-0)
\end{flushleft}


\begin{flushleft}
Industrial Wastes: Nature and characteristics, Prevention and Control,
\end{flushleft}


\begin{flushleft}
Tools for clean processes: reuse, recycle, recovery, source reduction,
\end{flushleft}


\begin{flushleft}
raw material substitution, process modification, Flow sheet analysis,
\end{flushleft}


\begin{flushleft}
Energy and resources audit, Waste audit, emission inventory and
\end{flushleft}


\begin{flushleft}
waste management hierarchy for process industries, Zero discharge,
\end{flushleft}


\begin{flushleft}
Environmental indicators, Industrial ecology and ecoparks, rules and
\end{flushleft}


\begin{flushleft}
regulations, Case studies: Dairy, Fertilizer, Distillery, Pulp and Paper,
\end{flushleft}


\begin{flushleft}
Iron and steel, Metal plating, Refineries, Thermal power plants, etc.
\end{flushleft}





\begin{flushleft}
CVP821 Advanced Water and Wastewater Laboratory
\end{flushleft}


\begin{flushleft}
3 Credits (1-0-4)
\end{flushleft}


\begin{flushleft}
Principles of instrumentation and application for water quality
\end{flushleft}


\begin{flushleft}
parameters measurements, Operation of batch scale models for
\end{flushleft}


\begin{flushleft}
various processes: Activated sludge process, Disinfection, Settlers,
\end{flushleft}


\begin{flushleft}
Coagulation,Filtration, Anaerobic digestion, Adsorption. Design
\end{flushleft}





\begin{flushleft}
Contemporary micro and macro environmental issues of importance,
\end{flushleft}


\begin{flushleft}
global environmental and resource sharing issues, international treaties
\end{flushleft}


\begin{flushleft}
and protocols. Emerging contaminants and emerging technologies
\end{flushleft}


\begin{flushleft}
for waste management, Case studies of environmental pollution
\end{flushleft}


\begin{flushleft}
and innovative management strategies. Environmental technology
\end{flushleft}


\begin{flushleft}
transfer, Non-conventional Energy, Emission trading, Adaptation to
\end{flushleft}


\begin{flushleft}
climate change.
\end{flushleft}





\begin{flushleft}
CVL823 Thermal Techniques for Waste Management
\end{flushleft}


\begin{flushleft}
3 Credits (3-0-0)
\end{flushleft}


\begin{flushleft}
Fundamentals of Thermodynamics, Heat Transfer and Combustion as
\end{flushleft}


\begin{flushleft}
applied to Waste Incineration. Introduction to fuels, reactor design,
\end{flushleft}


\begin{flushleft}
fluidization engineering and furnace technology. Combustion of
\end{flushleft}


\begin{flushleft}
gaseous, liquid and solid fuels. Wastes as fuels. Low, medium and
\end{flushleft}


\begin{flushleft}
high temperature thermal treatment techniques, Energy recovery,
\end{flushleft}


\begin{flushleft}
pollution control techniques for thermal facilities, Design of thermal
\end{flushleft}


\begin{flushleft}
treatment facilities with pollution control devices.
\end{flushleft}





\begin{flushleft}
CVL824 Life Cycle Analysis and Design for Environment
\end{flushleft}


\begin{flushleft}
3 Credits (3-0-0)
\end{flushleft}


\begin{flushleft}
Engineering products and processes : Environmental health and safety,
\end{flushleft}


\begin{flushleft}
Product life cycle stages, Material toxicity, pollution, and degradation,
\end{flushleft}


\begin{flushleft}
Environmentally conscious design and manufacturing approaches,
\end{flushleft}


\begin{flushleft}
Sustainable development and industrial ecology. System life-cycles
\end{flushleft}


\begin{flushleft}
from cradle to reincarnation, Product life-extension, Organizational
\end{flushleft}


\begin{flushleft}
issues. Pollution prevention practices, Manufacturing process selection
\end{flushleft}


\begin{flushleft}
and trade-offs. Design for Environment : Motivation, concerns,
\end{flushleft}


\begin{flushleft}
definitions, examples, guidelines, methods, and tools. Recyclability
\end{flushleft}


\begin{flushleft}
assessments, Design for recycling practices. Re- manufacturability
\end{flushleft}


\begin{flushleft}
assessments, Design for Remanufacture / Reuse practices. Industrial
\end{flushleft}


\begin{flushleft}
ecology and Eco-industrial parks. Eco-Labels and Life-Cycle analysis
\end{flushleft}


\begin{flushleft}
(LCA): LCA methodology, steps, tools and problems, Life-Cycle
\end{flushleft}


\begin{flushleft}
Accounting and Costing. ISO 14000 Environmental Management
\end{flushleft}


\begin{flushleft}
Standards. New business paradigms and associated design practices.
\end{flushleft}





\begin{flushleft}
CVL825 Fundamental of Aerosol: Health and Climate
\end{flushleft}


\begin{flushleft}
Change
\end{flushleft}


\begin{flushleft}
3 Credits (3-0-0)
\end{flushleft}


\begin{flushleft}
This course will introduce the students with fundamentals of aerosols,
\end{flushleft}


\begin{flushleft}
Difference in gas and particle motion in the air, physio-chemical
\end{flushleft}


\begin{flushleft}
and optical properties of individual and mixed particles, behaviour
\end{flushleft}


\begin{flushleft}
of non-spherical particles, thermodynamic properties of aerosol,
\end{flushleft}


\begin{flushleft}
particle formation, application of aerosol fundamentals and properties
\end{flushleft}


\begin{flushleft}
in research and industries, impact of aerosol properties on indoor/
\end{flushleft}


\begin{flushleft}
outdoor air quality, health and climate.
\end{flushleft}





\begin{flushleft}
CVL826 Quantitative microbial risk assessment
\end{flushleft}


\begin{flushleft}
1 Credit (1-0-0)
\end{flushleft}


\begin{flushleft}
Pathogens, Occurrence and fate in environment, Human exposure
\end{flushleft}


\begin{flushleft}
pathways, Microbial exposure dose estimation, Infection and doseresponse modeling, Risk of infection estimation,Uncertainty estimation.
\end{flushleft}





\begin{flushleft}
CVL827 Environmental Implications of Engineered
\end{flushleft}


\begin{flushleft}
Nanomaterials
\end{flushleft}


\begin{flushleft}
2 Credits (2-0-0)
\end{flushleft}


\begin{flushleft}
Engineered nanomateials, Occurrence of nanomaterials in environment,
\end{flushleft}


\begin{flushleft}
Fate of nanomaterials in environment, Exposure pathways-model
\end{flushleft}


\begin{flushleft}
development and parameter estimation,Dose-response effects
\end{flushleft}


\begin{flushleft}
of nanomaterials to humans and aquatic species; dose-response
\end{flushleft}


\begin{flushleft}
modeling and risk estimation of nanomaterials exposures; Risk
\end{flushleft}


\begin{flushleft}
management of nanomaterials pollution; Prioritization of nanomaterials
\end{flushleft}


\begin{flushleft}
for monitoring; Regulatory guidelines for implications assessment and
\end{flushleft}


\begin{flushleft}
pollution regulations; Emerging challenges for long-term management
\end{flushleft}


\begin{flushleft}
of nanomaterials exposure.
\end{flushleft}





179





\begin{flushleft}
\newpage
Civil Engineering
\end{flushleft}





\begin{flushleft}
CVL828 Water Distribution and Sewerage Network Design
\end{flushleft}


\begin{flushleft}
3 Credits (3-0-0)
\end{flushleft}


\begin{flushleft}
Planning for water supply sources and demand assessment. Water
\end{flushleft}


\begin{flushleft}
demand forecasting. Types of water distribution systems. Intermittent
\end{flushleft}


\begin{flushleft}
and continuous water supply systems. Design and analysis of Water
\end{flushleft}


\begin{flushleft}
mains. Design and analysis of water distribution system. Analysis of
\end{flushleft}


\begin{flushleft}
water deficient systems. Optimal design of water distribution systems.
\end{flushleft}


\begin{flushleft}
On-line monitoring of water quality parameters. Retrofitting of the
\end{flushleft}


\begin{flushleft}
existing water supply systems.
\end{flushleft}


\begin{flushleft}
Planning for wastewater conveyance system in urban areas. Combined
\end{flushleft}


\begin{flushleft}
and separate systems for storm and sewage. Design and analysis
\end{flushleft}


\begin{flushleft}
of wastewater conveyance system. Optimal design of wastewater
\end{flushleft}


\begin{flushleft}
conveyance systems. Operation and maintenance issues. Retrofitting
\end{flushleft}


\begin{flushleft}
of the sewerage system.
\end{flushleft}





\begin{flushleft}
CVL830 Groundwater Flow and Pollution Modeling
\end{flushleft}


\begin{flushleft}
3 Credits (3-0-0)
\end{flushleft}


\begin{flushleft}
Subsurface processes and concepts for groundwater resources
\end{flushleft}


\begin{flushleft}
evaluation, Unsaturated zone properties: Soil moisture levels,
\end{flushleft}


\begin{flushleft}
Retention curves, Flow through unsaturated porous media, Multiphase
\end{flushleft}


\begin{flushleft}
flows, infiltration and Wetting front, Groundwater contamination,
\end{flushleft}


\begin{flushleft}
Sources and causes of groundwater pollution, Pollution dynamics,
\end{flushleft}


\begin{flushleft}
Hydrodynamics dispersion, Adsorption, Biodegradation, Radioactive
\end{flushleft}


\begin{flushleft}
decay, Reactive processes, Multiphase contamination, NAPLs, VOCs,
\end{flushleft}


\begin{flushleft}
Site specific groundwater quality problems in Indian context, Numerical
\end{flushleft}


\begin{flushleft}
models, Finite difference methods, Numerical modeling of steady and
\end{flushleft}


\begin{flushleft}
transient flows in saturated and unsaturated domain, Contaminant
\end{flushleft}


\begin{flushleft}
transport modeling, Application of FEM and BIEM in groundwater
\end{flushleft}


\begin{flushleft}
modeling, Regional aquifer simulation, Contaminated groundwater
\end{flushleft}


\begin{flushleft}
systems and their rehabilitation, Development and optimization based
\end{flushleft}


\begin{flushleft}
management of aquifer systems, Stochastic models, Random field
\end{flushleft}


\begin{flushleft}
concepts in groundwater models; Application of emerging techniques
\end{flushleft}


\begin{flushleft}
to groundwater management.
\end{flushleft}





\begin{flushleft}
CVS830 Independent Study (CEW)
\end{flushleft}


\begin{flushleft}
3 Credits (0-3-0)
\end{flushleft}





\begin{flushleft}
Integrated Water Resources Management Model on River Basin Scale,
\end{flushleft}


\begin{flushleft}
River Basin Scale Integrated Stochastic Water Resources Planning and
\end{flushleft}


\begin{flushleft}
Management Models.
\end{flushleft}





\begin{flushleft}
CVL834 Urban Water Infrastructure
\end{flushleft}


\begin{flushleft}
3 Credits (3-0-0)
\end{flushleft}


\begin{flushleft}
Urban water cycle, Urban water infrastructures - water supply, storm
\end{flushleft}


\begin{flushleft}
water drainage, sanitation, sewerage and wastewater conveyance
\end{flushleft}


\begin{flushleft}
infrastructures, Water supply and sewerage network hydraulics,
\end{flushleft}


\begin{flushleft}
SCADA systems, Sustainable urban designs, Methodologies for
\end{flushleft}


\begin{flushleft}
assessing sustainability of urban water infrastructures, Emerging
\end{flushleft}


\begin{flushleft}
sustainable materials and design procedures for water supply and
\end{flushleft}


\begin{flushleft}
sewerage pipelines, Hydraulic performance and structural strength,
\end{flushleft}


\begin{flushleft}
chemical resistance and resilience characteristics of emerging materials
\end{flushleft}


\begin{flushleft}
based water and sewer pipelines, Rehabilitation and augmentation
\end{flushleft}


\begin{flushleft}
technologies for water supply and sewerage networks, Analytic
\end{flushleft}


\begin{flushleft}
hierarchy process and optimization techniques for arriving at the best
\end{flushleft}


\begin{flushleft}
appropriate rehabilitation / augmentation technology, Urban water
\end{flushleft}


\begin{flushleft}
management, Rain water harvesting, Managed aquifer recharge,
\end{flushleft}


\begin{flushleft}
Constructed/engineered wetlands, Sprinkler and drip irrigation, Water
\end{flushleft}


\begin{flushleft}
use efficiencies, Effect of water management practices on urban
\end{flushleft}


\begin{flushleft}
water infrastructure, hydrology and groundwater regime, Surface and
\end{flushleft}


\begin{flushleft}
subsurface mapping of water supply and sewerage networks, Structural
\end{flushleft}


\begin{flushleft}
safety and mitigating plans against natural and human caused threats.
\end{flushleft}





\begin{flushleft}
CVL835 Eco-hydraulics and Hydrology
\end{flushleft}


\begin{flushleft}
3 Credits (3-0-0)
\end{flushleft}


\begin{flushleft}
Classification of Hydro environmental systems, governing equations
\end{flushleft}


\begin{flushleft}
for open surface flow domains, pollutant transport equations in hydroenvironmental flow systems, computational methods and solution
\end{flushleft}


\begin{flushleft}
techniques. Study of ecological descriptors, numerical ecology, multiobjective definitions of environmental flows, Hydrologic indices for
\end{flushleft}


\begin{flushleft}
e-flows and river health assessment. Riverine habitat characterization
\end{flushleft}


\begin{flushleft}
and habitat simulation models. Anthropogenic triggers for changes
\end{flushleft}


\begin{flushleft}
in riverine habitat.
\end{flushleft}





\begin{flushleft}
CVL836 Advanced Hydrologic Land Surface Processes
\end{flushleft}


\begin{flushleft}
3 Credits (3-0-0)
\end{flushleft}





\begin{flushleft}
CVD831 Major Project Part-I
\end{flushleft}


\begin{flushleft}
6 Credits (0-0-12)
\end{flushleft}


\begin{flushleft}
CVL831 Surface Water Quality Modeling and Control
\end{flushleft}


\begin{flushleft}
3 Credits (3-0-0)
\end{flushleft}


\begin{flushleft}
River hydrology and derivation of Stream Equation, Derivation of
\end{flushleft}


\begin{flushleft}
Estaury equation, Distribution of water quality in rivers and estuaries.
\end{flushleft}


\begin{flushleft}
Physical and Chemical characteristics of Lakes, Finite Difference steady
\end{flushleft}


\begin{flushleft}
state river, estaury and Lake models., Dissolved Oxygen models in
\end{flushleft}


\begin{flushleft}
rivers, estuaries and Lakes, Fate of Indicator Bacteria and pathogens in
\end{flushleft}


\begin{flushleft}
water bodies. Basic Mechanism of Eutrophication, Lake phytoplankton
\end{flushleft}


\begin{flushleft}
models, eutophication in rivers and estuaries. Elements of Toxic
\end{flushleft}


\begin{flushleft}
substance analysis.
\end{flushleft}





\begin{flushleft}
CVD832 Major Project Part-II
\end{flushleft}


\begin{flushleft}
12 Credits (0-0-24)
\end{flushleft}





\begin{flushleft}
CVL832 Hydroelectric Engineering
\end{flushleft}


\begin{flushleft}
3 Credits (3-0-0)
\end{flushleft}


\begin{flushleft}
Hydropower development schemes and their various configurations,
\end{flushleft}


\begin{flushleft}
Planning for firm Capacities, Peak Load and Base Load configurations,
\end{flushleft}


\begin{flushleft}
Role of and Regulation of Hydropower development in a mixed hydrosteam system, Governing of Hydropower systems; study of hydraulic
\end{flushleft}


\begin{flushleft}
transients in Penstocks. Surge analysis and dynamics of Surge tanks.
\end{flushleft}


\begin{flushleft}
Micro hydro power developments.
\end{flushleft}





\begin{flushleft}
CVL833 Water Resources Systems
\end{flushleft}


\begin{flushleft}
3 Credits (3-0-0)
\end{flushleft}


\begin{flushleft}
Water Resources Planning Purposes and Objectives, Multi-component,
\end{flushleft}


\begin{flushleft}
multi-user, multi-objective and multi-purpose attributes of an
\end{flushleft}


\begin{flushleft}
Integrated Water Resources System, Economic basis for selection of
\end{flushleft}


\begin{flushleft}
a Plan Alternative.
\end{flushleft}


\begin{flushleft}
Introduction to Linear Programming and applications in Water
\end{flushleft}


\begin{flushleft}
Resources Engineering.
\end{flushleft}


\begin{flushleft}
Irrigation Planning and Operation Models, Linear, Deterministic
\end{flushleft}





\begin{flushleft}
Introduction: Eco-hydro-climatology; Climate System; Climate, weather
\end{flushleft}


\begin{flushleft}
and Climate Change; Water, Energy and Carbon Cycle; Overview of
\end{flushleft}


\begin{flushleft}
Earth's Atmosphere: Heat-Balance of Earth Atmosphere System;
\end{flushleft}


\begin{flushleft}
Temporal Variation of Air temperature; Introduction to Atmospheric
\end{flushleft}


\begin{flushleft}
Thermodynamics: First and second law of thermodynamics, Adiabatic
\end{flushleft}


\begin{flushleft}
process and adiabatic lapse rate, Entropy, Clausius-Clapeyron Theory,
\end{flushleft}


\begin{flushleft}
Introduction to cloud microphysics and cloud droplet formation
\end{flushleft}


\begin{flushleft}
process, Cloud liquid water content, entrainment, warm and cold
\end{flushleft}


\begin{flushleft}
cloud. Hydrologic Cycle: Global water balance; Precipitation and
\end{flushleft}


\begin{flushleft}
Weather, Forms of Precipitation; Atmospheric Stability; Monsoon;
\end{flushleft}


\begin{flushleft}
Global Wind Circulation; Indian Summer Monsoon Rainfall. Climate
\end{flushleft}


\begin{flushleft}
Variability: Floods, Droughts, Climate Extremes. Climate Change:
\end{flushleft}


\begin{flushleft}
Introduction; Causes and Modeling of Climate Change, Climate Models,
\end{flushleft}


\begin{flushleft}
Downscaling; IPCC Scenarios; Commonly used Statistical Methods in
\end{flushleft}


\begin{flushleft}
Hydro-climatology: Trend Analysis; EOF, PCA; Canonical Correlation;
\end{flushleft}


\begin{flushleft}
Statistical Downscaling; Ecological Climatology: Leaf energy fluxes and
\end{flushleft}


\begin{flushleft}
leaf photosynthesis; Ecosystem and vegetation dynamics; Coupled
\end{flushleft}


\begin{flushleft}
climate vegetation dynamics, Carbon cycle climate feedbacks.
\end{flushleft}





\begin{flushleft}
CVL837 Mechanics of Sediment Transport
\end{flushleft}


\begin{flushleft}
3 Credits (2-0-2)
\end{flushleft}


\begin{flushleft}
Introduction; Equations of Particle Motion particle in a moving fluid,
\end{flushleft}


\begin{flushleft}
collision with the bed, diffusion of turbulence; Macroscopic View of
\end{flushleft}


\begin{flushleft}
Sediment Transport -- bedload, suspended load; Threshold Condition
\end{flushleft}


\begin{flushleft}
for Sediment Motion -- Critical stress for flow over a granular bed,
\end{flushleft}


\begin{flushleft}
Shields diagram; Mechanics of Bedload Transport: Bagnold hypothesis
\end{flushleft}


\begin{flushleft}
of bedload transport, bedload transport relations; Mechanics of
\end{flushleft}


\begin{flushleft}
Suspended Sediment Transport; Total load transport; Descriptive
\end{flushleft}


\begin{flushleft}
Analysis of Bedforms -- introduction of bedform mechanics, dunes,
\end{flushleft}


\begin{flushleft}
antidunes, ripples, bars; Stability Analysis of Bedforms; Mechanism of
\end{flushleft}


\begin{flushleft}
transportation of materials by fluid flow through pipeline; Rheology and
\end{flushleft}


\begin{flushleft}
classification of complex mixtures; Fundamentals of two-phase flow;
\end{flushleft}


\begin{flushleft}
Phase separation and settling behaviour; Flow of non-Newtonian fluids
\end{flushleft}


\begin{flushleft}
through pipes: Turbulent flows of Complex mixtures, Slurry pipeline
\end{flushleft}


\begin{flushleft}
transportation, Design methods.
\end{flushleft}





180





\begin{flushleft}
\newpage
Civil Engineering
\end{flushleft}





\begin{flushleft}
CVL838 Geographic Information Systems
\end{flushleft}


\begin{flushleft}
3 Credits (2-0-2)
\end{flushleft}





\begin{flushleft}
Pre-requisites: M.Tech: CVL740; B.Tech: Instructor's permission
\end{flushleft}





\begin{flushleft}
What is GIS. Geographic concepts for GIS. Spatial relationships,
\end{flushleft}


\begin{flushleft}
topology, spatial patterns, spatial interpolation. Data storage, data
\end{flushleft}


\begin{flushleft}
structure, non-spatial database models. Populating GIS, digitizing, data
\end{flushleft}


\begin{flushleft}
conversion. Spatial data models, Raster and Vector data structures
\end{flushleft}


\begin{flushleft}
and algorithms. Digital Elevation Models (DEM) and their application.
\end{flushleft}


\begin{flushleft}
Georeferencing and projection systems, GIS application areas, Spatial
\end{flushleft}


\begin{flushleft}
analysis, quantifying relationships, spatial statistics, spatial search.
\end{flushleft}





\begin{flushleft}
CVL839 Hydrologic Applications of Remote Sensing
\end{flushleft}


\begin{flushleft}
3 Credits (2-0-2)
\end{flushleft}


\begin{flushleft}
Principles of remote sensing, Remote sensing platforms and data
\end{flushleft}


\begin{flushleft}
acquisition systems, Wavebands, Radiometric quantities, Spectral
\end{flushleft}


\begin{flushleft}
reflectance and spectral signature, Interaction of electromagnetic
\end{flushleft}


\begin{flushleft}
radiation with land surface features, hydrosphere and atmosphere,
\end{flushleft}


\begin{flushleft}
Data capture for simulation of land surface processes, Photographic
\end{flushleft}


\begin{flushleft}
and image interpretation, Satellite image processing, Earth surface
\end{flushleft}


\begin{flushleft}
features inventory, Geomorphology, Landuse classification, Landuse
\end{flushleft}


\begin{flushleft}
planning and landcover mapping, Flood plain mapping and flood plain
\end{flushleft}


\begin{flushleft}
zoning, Remote sensing applications in water resources, agriculture,
\end{flushleft}


\begin{flushleft}
geology and environmental monitoring, Applications in snow and
\end{flushleft}


\begin{flushleft}
glacier studies, Snow line, Ice cover, Snow-pack properties, Integrated
\end{flushleft}


\begin{flushleft}
use of remote sensing and GIS, Database preparation and Decision
\end{flushleft}


\begin{flushleft}
support analysis, Estimation of damages due to hydrologic extremes
\end{flushleft}


\begin{flushleft}
and preparation of contingency plans, Case studies.
\end{flushleft}





\begin{flushleft}
CVL840 Planning and Design of Sustainable Transport
\end{flushleft}


\begin{flushleft}
Systems
\end{flushleft}


\begin{flushleft}
3 Credits (3-0-0)
\end{flushleft}


\begin{flushleft}
Pre-requisites: M.Tech: CVL741; B.Tech: Instructor's permission
\end{flushleft}


\begin{flushleft}
Sustainable Transportation Planning and Design including:
\end{flushleft}


\begin{flushleft}
Consideration of bicycles, pedestrian, mass transit modes, and
\end{flushleft}


\begin{flushleft}
private vehicles like cars and two wheelers as well as how these
\end{flushleft}


\begin{flushleft}
modes interrelate. Applicability at varying scales, from a downtown
\end{flushleft}


\begin{flushleft}
street to a neighborhood to a regional network Case studies are
\end{flushleft}


\begin{flushleft}
discussed from different parts of the world. Various indicators for
\end{flushleft}


\begin{flushleft}
measuring sustainability index of transport system including public
\end{flushleft}


\begin{flushleft}
health, resource consumption, local and global pollution and equity
\end{flushleft}


\begin{flushleft}
considerations are discussed.
\end{flushleft}





\begin{flushleft}
CVL841 Advanced Transportation Modelling
\end{flushleft}


\begin{flushleft}
3 Credits (2-0-2)
\end{flushleft}


\begin{flushleft}
Pre-requisites: M.Tech: CVL741; B.Tech: Instructor's permission
\end{flushleft}


\begin{flushleft}
Systems Approach to Travel demand models, Trip generation Models
\end{flushleft}


\begin{flushleft}
Using Different Statistical techniques, Trip distribution,Discrete Choice
\end{flushleft}


\begin{flushleft}
Logit, Nested Logit and other Models,Network Assignment,Traffic
\end{flushleft}


\begin{flushleft}
Assignment Using User Equilibrium and Systems Optimization
\end{flushleft}


\begin{flushleft}
Techniques, Revealed preference and Stated Preference surveys,
\end{flushleft}


\begin{flushleft}
Analysis of Ranked and Rated data, Demand models for Nonmotorised
\end{flushleft}


\begin{flushleft}
transport and Public Transport systems.
\end{flushleft}





\begin{flushleft}
CVL842 Geometric Design of Roads
\end{flushleft}


\begin{flushleft}
3 Credits (2-0-2)
\end{flushleft}


\begin{flushleft}
Pre-requisites: M.Tech: CVL741, CVL742; B.Tech: CVL261 and
\end{flushleft}


\begin{flushleft}
one TE elective
\end{flushleft}


\begin{flushleft}
Introduction to basic road geoemetric design elements and
\end{flushleft}


\begin{flushleft}
methodology - design philosophy and design techniques; Design
\end{flushleft}


\begin{flushleft}
controls - human, vehicle and speed related factors. Road vehicle
\end{flushleft}


\begin{flushleft}
performance - road vehicle dynamics - tractive and resisting forces.
\end{flushleft}


\begin{flushleft}
Braking forces. Theoretical and practical stopping distances. Elements
\end{flushleft}


\begin{flushleft}
of geometric design - cross section elements; Horizontal Alignment
\end{flushleft}


\begin{flushleft}
- tangents, curves, transitions, superelevation; Vertical Alignment grades and curves; Coordination of Horizontal and Vertical Alignment.
\end{flushleft}


\begin{flushleft}
Design of Intersections at-grade- design principles, channelization,
\end{flushleft}


\begin{flushleft}
roundabouts, Interchanges- types, warrants, lane balancing; Road
\end{flushleft}


\begin{flushleft}
side safety- hazards and clear zone concept, traffic safety barriers,
\end{flushleft}


\begin{flushleft}
impact attenuation.
\end{flushleft}





\begin{flushleft}
CVL844 Transportation Infrastructure Management
\end{flushleft}


\begin{flushleft}
3 Credits (3-0-0)
\end{flushleft}





\begin{flushleft}
Transportation infrastructure components; Deterioration phenomena;
\end{flushleft}


\begin{flushleft}
Effect of external factors like environment, traffic loading,material
\end{flushleft}


\begin{flushleft}
properties on deterioration mechanisms; Evaluation techniques
\end{flushleft}


\begin{flushleft}
to evaluate damage: destructive, nondestructive; Performance
\end{flushleft}


\begin{flushleft}
models: development, calibration; Infrastructure management
\end{flushleft}


\begin{flushleft}
systems; Serviceability of condition and safety; Decision making and
\end{flushleft}


\begin{flushleft}
optimization techniques applied to infrastructure management; Life
\end{flushleft}


\begin{flushleft}
cycle cost analysis techniques.
\end{flushleft}





\begin{flushleft}
CVL845 Viscoelastic Behavior of Bituminous Materials
\end{flushleft}


\begin{flushleft}
3 Credits (3-0-0)
\end{flushleft}


\begin{flushleft}
Pre-requisites: M.Tech: CVL740; B.Tech: Instructor's permission
\end{flushleft}


\begin{flushleft}
Overview of material behavior-elastic, plastic, viscoelastic,
\end{flushleft}


\begin{flushleft}
Viscoplastic response; Aging; Issues in representative volume
\end{flushleft}


\begin{flushleft}
element; Mechanical analogs for viscoelastic response; Fundamental
\end{flushleft}


\begin{flushleft}
viscoelastic response-creep compliance, relaxation, complex modulus;
\end{flushleft}


\begin{flushleft}
Interconversion techniques to obtain fundamental viscoelastic
\end{flushleft}


\begin{flushleft}
responses; Time-temperature superposition; linear viscoelastic
\end{flushleft}


\begin{flushleft}
constitutive equations; Elastic-viscoelastic correspondence principle;
\end{flushleft}


\begin{flushleft}
Predicting material behavior-undamaged, damaged state conditions,
\end{flushleft}


\begin{flushleft}
Introduction to nonlinear viscoelasticity, Viscoelastoplastic behavoir,
\end{flushleft}


\begin{flushleft}
fracture mechanics.
\end{flushleft}





\begin{flushleft}
CVL846 Transportation System Management
\end{flushleft}


\begin{flushleft}
3 Credits (3-0-0)
\end{flushleft}


\begin{flushleft}
Pre-requisites: M.Tech: CVL741 and CVL742; B.Tech:
\end{flushleft}


\begin{flushleft}
Instructor's permission
\end{flushleft}


\begin{flushleft}
Transportation systems - resource management, approaches to
\end{flushleft}


\begin{flushleft}
funding. Asset and demand management - Integrated network
\end{flushleft}


\begin{flushleft}
design, changing travel behaviour, optimising asset management, role
\end{flushleft}


\begin{flushleft}
of technology; Optimizing the investment outcomes - movement of
\end{flushleft}


\begin{flushleft}
freight and passenger, traffic. Land use planning and urban growth
\end{flushleft}


\begin{flushleft}
management - land use and its effect on infrastructure and efficient
\end{flushleft}


\begin{flushleft}
network operations. Congestion, systemic congestion improvement
\end{flushleft}


\begin{flushleft}
and system-wide efficiency, Transit oriented development, safety
\end{flushleft}


\begin{flushleft}
considerations; evaluation of strategies; case studies.
\end{flushleft}





\begin{flushleft}
CVL847 Transportation Economics
\end{flushleft}


\begin{flushleft}
3 Credits (3-0-0)
\end{flushleft}


\begin{flushleft}
Pre-requisites: M.Tech: CVL741; B.Tech: Instructor's permission
\end{flushleft}


\begin{flushleft}
Overview of Transportation Economics; Transportation Investments
\end{flushleft}


\begin{flushleft}
and economic Development. Basics of Engineering economics,
\end{flushleft}


\begin{flushleft}
marginal analysis, opportunity cost, shadow price, money value
\end{flushleft}


\begin{flushleft}
of time, discounted cash flow, NPV, ROR, benefit-cost analysis.
\end{flushleft}


\begin{flushleft}
Road User Costs; Public transportation economics; Social Cost
\end{flushleft}


\begin{flushleft}
of Transportation;Cost of congestion, pollution, traffic accidents.
\end{flushleft}


\begin{flushleft}
Taxation, regulations, financing Transport Systems; Legal framework
\end{flushleft}


\begin{flushleft}
for transportation sector, case studies.
\end{flushleft}





\begin{flushleft}
CVL849 Traffic Flow Modelling
\end{flushleft}


\begin{flushleft}
3 Credits (3-0-0)
\end{flushleft}


\begin{flushleft}
Pre-requisites: CVL742
\end{flushleft}


\begin{flushleft}
Descriptors of traffic flow: Macroscopic and Microscopic, time, space
\end{flushleft}


\begin{flushleft}
and generalized measurement regions. Cumulative plots. Traffic Flow
\end{flushleft}


\begin{flushleft}
models - General classification and typology. Macroscopic Flow Models
\end{flushleft}


\begin{flushleft}
- continuity equation, LWR model, higher order models, numerical
\end{flushleft}


\begin{flushleft}
schema, Mesoscopic Flow Models - gas kinetic theory, Microscopic
\end{flushleft}


\begin{flushleft}
and Submicroscopic Flow Models - car following and lane changing;
\end{flushleft}


\begin{flushleft}
Pipes and forbes models; General motors-Gazis-Herman-Rothery
\end{flushleft}


\begin{flushleft}
(GHR) models, Stability analysis, macro-micro bridge. Modelling at
\end{flushleft}


\begin{flushleft}
Junctions/Intersections; Un-signalized and Signalized; Roundabouts;
\end{flushleft}


\begin{flushleft}
Pedestrian Modelling - normal and panic movements; variations across
\end{flushleft}


\begin{flushleft}
infrastructure; Simulation - simple and complex traffic conditions.
\end{flushleft}





\begin{flushleft}
CVL850 Transportation Logistics
\end{flushleft}


\begin{flushleft}
3 Credits (3-0-0)
\end{flushleft}


\begin{flushleft}
Pre-requisites: M.Tech: CVL742 else Instructor's permission
\end{flushleft}


\begin{flushleft}
(including B.Tech)
\end{flushleft}


\begin{flushleft}
Evolution of freight and logistics; Interrelationships between
\end{flushleft}





181





\begin{flushleft}
\newpage
Civil Engineering
\end{flushleft}





\begin{flushleft}
society, environment and freight transport; Survey methodologies
\end{flushleft}


\begin{flushleft}
to understand freight movement; Cost measurement: Production,
\end{flushleft}


\begin{flushleft}
Holding, Transportation, Handling; Effect of internal and external
\end{flushleft}


\begin{flushleft}
variables on cost; Demand forecasting; Inventory planning and
\end{flushleft}


\begin{flushleft}
management; Transportation and distribution network: Design,
\end{flushleft}


\begin{flushleft}
Reverse Logistics. Development, Management; Ware house operations;
\end{flushleft}


\begin{flushleft}
Pricing: Perishable, seasonal demand, uncertainty issues; Vehicle
\end{flushleft}


\begin{flushleft}
routing: One-to-one distribution, One-to-many distribution, Shortest
\end{flushleft}


\begin{flushleft}
path algorithm, Quickest time algorithm; Logistics information system;
\end{flushleft}


\begin{flushleft}
Designing and planning transportation networks; City logistics.
\end{flushleft}





\begin{flushleft}
CVL859 Theory of Structural Stability
\end{flushleft}


\begin{flushleft}
3 Credits (3-0-0)
\end{flushleft}





\begin{flushleft}
CVL851 Special Topics in Transportation Engineering
\end{flushleft}


\begin{flushleft}
3 Credits (3-0-0)
\end{flushleft}


\begin{flushleft}
Pre-requisites: CVL740 or CVL741 or CVL742 or Instructor's
\end{flushleft}


\begin{flushleft}
permission
\end{flushleft}





\begin{flushleft}
CVL860 Advanced Finite Element Method and
\end{flushleft}


\begin{flushleft}
Programming
\end{flushleft}


\begin{flushleft}
3 Credits (2-0-2)
\end{flushleft}


\begin{flushleft}
Pre-requisites: CVL757
\end{flushleft}





\begin{flushleft}
Course details shall be announced by the instructor at the time
\end{flushleft}


\begin{flushleft}
of offering of the course. The lectures will be supplemented by
\end{flushleft}


\begin{flushleft}
reading materials. The assessment will be based on a combination of
\end{flushleft}


\begin{flushleft}
assignments, quizzes, and term papers and tests.
\end{flushleft}





\begin{flushleft}
CVS852 Advanced Topics in Transportation Engineering
\end{flushleft}


\begin{flushleft}
3 Credits (0-0-6)
\end{flushleft}


\begin{flushleft}
Pre-requisites: CVL740, CVL741, CVL742
\end{flushleft}


\begin{flushleft}
This is an advanced course for M.Tech. Transportation engineering
\end{flushleft}


\begin{flushleft}
program where students will study a specialized topic within
\end{flushleft}


\begin{flushleft}
transportation engineering (including but not limited to transportation
\end{flushleft}


\begin{flushleft}
planning, traffic engineering and pavement engineering). The topic
\end{flushleft}


\begin{flushleft}
shall be announced by instructor at the beginning. The performance
\end{flushleft}


\begin{flushleft}
of student in this course will be evaluated through presentation(s) and
\end{flushleft}


\begin{flushleft}
report(s) made by student during the registered term.
\end{flushleft}





\begin{flushleft}
CVD853 Major Project Part-I
\end{flushleft}


\begin{flushleft}
9 Credits (0-0-18)
\end{flushleft}


\begin{flushleft}
Pre-requisites: CVL740, CVL741, CVL742
\end{flushleft}


\begin{flushleft}
CVD854 Major Project Part-II
\end{flushleft}


\begin{flushleft}
12 Credits (0-0-24)
\end{flushleft}


\begin{flushleft}
Pre-requisites: CVL740, CVL741, CVL742
\end{flushleft}


\begin{flushleft}
CVL856 Strengthening and Retrofitting of Structures
\end{flushleft}


\begin{flushleft}
3 Credits (3-0-0)
\end{flushleft}


\begin{flushleft}
Structural assessment, damage under accidental and cyclic
\end{flushleft}


\begin{flushleft}
loads, cracking in structures, evaluation of damage, analysis of
\end{flushleft}


\begin{flushleft}
existing structures, compression, flexural and shear strengthening,
\end{flushleft}


\begin{flushleft}
strengthening using laminates, strengthening using prestressing,
\end{flushleft}


\begin{flushleft}
bracing and stiffening of structures, maintenance of retrofitting, design
\end{flushleft}


\begin{flushleft}
codes for retrofitting of structures, retrofitting of steel structures,
\end{flushleft}


\begin{flushleft}
retrofitting of masonry structures.
\end{flushleft}





\begin{flushleft}
CVL857 Structural Safety and Reliability
\end{flushleft}


\begin{flushleft}
3 Credits (3-0-0)
\end{flushleft}


\begin{flushleft}
Fundamentals of Set Theory and Probability; Probability Distribution,
\end{flushleft}


\begin{flushleft}
Regression Analysis, Hypothesis Testing. Stochastic Process and Its
\end{flushleft}


\begin{flushleft}
Moments; Probability Distributions; Concepts of Safety Factors, Safety,
\end{flushleft}


\begin{flushleft}
Reliability and Risk Analysis; First Order and Second Order Reliability
\end{flushleft}


\begin{flushleft}
Methods; Simulation Based Methods; Confidence Limits and Baysean
\end{flushleft}


\begin{flushleft}
Revision of Reliability; Reliability Based Design; System Reliability;
\end{flushleft}


\begin{flushleft}
Examples of Reliability Analysis of Structures.
\end{flushleft}





\begin{flushleft}
CVL858 Theory of Plates and Shells
\end{flushleft}


\begin{flushleft}
3 Credits (3-0-0)
\end{flushleft}


\begin{flushleft}
Thin and thick plate theories. Bending of long rectangular plate to a
\end{flushleft}


\begin{flushleft}
cylindrical surface. Prismatic folded plate systems. Pure and symmetric
\end{flushleft}


\begin{flushleft}
bending of plates. Small and large deflections of plates. Special and
\end{flushleft}


\begin{flushleft}
approximate methods in theory of plates. General theory of cylindrical
\end{flushleft}


\begin{flushleft}
shells. Shell equations. Approximate solutions of plates and shells
\end{flushleft}


\begin{flushleft}
equations. Analysis and design of cylindrical shells. Approximate design
\end{flushleft}


\begin{flushleft}
methods for doubly curved shells. Stress analysis methods in sperical
\end{flushleft}


\begin{flushleft}
shells. Spherical shell of constant thickness. Symmetrical bending of
\end{flushleft}


\begin{flushleft}
shallow sperical shells. Conical shells.
\end{flushleft}





\begin{flushleft}
Introduction: Buckling of steel and concrete structures; Conservative
\end{flushleft}


\begin{flushleft}
and non-conservative loads. Elastic buckling of columns and beamcolumns: Static, dynamical and energy-based approaches. Viscoelastic
\end{flushleft}


\begin{flushleft}
and elastoplastic buckling. Torsional buckling. Flexural-torsional and
\end{flushleft}


\begin{flushleft}
lateral buckling. Plate and frame buckling. Imperfection sensitivity;
\end{flushleft}


\begin{flushleft}
Post-buckling theory. Snap-through. Dynamic stability: Divergence,
\end{flushleft}


\begin{flushleft}
flutter and parametric resonance. Nonlinear dynamical systems theory;
\end{flushleft}


\begin{flushleft}
Bifurcations. Recent trends.
\end{flushleft}





\begin{flushleft}
Finite element method (FEM) to solve complex structural engineering
\end{flushleft}


\begin{flushleft}
problems. Various types of finite elements (FE) considering nonlinear
\end{flushleft}


\begin{flushleft}
material models; constitutive laws; hybrid elements. Strong and
\end{flushleft}


\begin{flushleft}
weak form representation and solutions. FEM for dynamic problems:
\end{flushleft}


\begin{flushleft}
consistent mass matrix, vibration of bars, beams, and plate elements.
\end{flushleft}


\begin{flushleft}
FEM for buckling problems: geometric matrix, buckling of struts, and
\end{flushleft}


\begin{flushleft}
plate elements. FE modeling and analysis of complex structures:
\end{flushleft}


\begin{flushleft}
3-D frames, shear walls, bridges, cooling towers, continuua etc.
\end{flushleft}


\begin{flushleft}
Computational aspects: meshing, convergence, singularity, etc.
\end{flushleft}


\begin{flushleft}
Interpretation of results. Comparison with other methods.
\end{flushleft}





\begin{flushleft}
CVL861 Analysis and Design of Machine Foundations
\end{flushleft}


\begin{flushleft}
3 Credits (2-0-2)
\end{flushleft}


\begin{flushleft}
General design requirements, general dynamics of machine
\end{flushleft}


\begin{flushleft}
foundations for rotating and reciprocating machines, determination
\end{flushleft}


\begin{flushleft}
of soil properties, modelling, analysis and design of block/frame type
\end{flushleft}


\begin{flushleft}
foundations, specific details for machines applying impulsive loads,
\end{flushleft}


\begin{flushleft}
compressors and turbo-generators, detailed dynamic analysis and
\end{flushleft}


\begin{flushleft}
modes of vibration for frame type foundations, techniques for vibration
\end{flushleft}


\begin{flushleft}
isolation, practical case studies, codal requirements, construction
\end{flushleft}


\begin{flushleft}
aspects of machine foundations.
\end{flushleft}


\begin{flushleft}
Laboratory : Instrumentation aspects in terms of sensors and data
\end{flushleft}


\begin{flushleft}
acquisition systems, measurement of dynamic soil parameters,
\end{flushleft}


\begin{flushleft}
measurement of vibration related parameters, vibration isolation,
\end{flushleft}


\begin{flushleft}
computational aspects related to frame type foundations including
\end{flushleft}


\begin{flushleft}
dynamic analysis.
\end{flushleft}





\begin{flushleft}
CVL862 Design of Offshore Structures
\end{flushleft}


\begin{flushleft}
3 Credits (3-0-0)
\end{flushleft}


\begin{flushleft}
Rudiments of offshore engineering; sea spectra; wave theories; wavestructure interaction. Design of offshore platforms: introduction, fixed
\end{flushleft}


\begin{flushleft}
and floating platforms. Buoyed structures/ articulated towers; tensionleg platform (TLP); Marine risers; compliant and non-compliant
\end{flushleft}


\begin{flushleft}
structures; offshore pipelines and risers; Steel, concrete, and hybrid
\end{flushleft}


\begin{flushleft}
platforms. Buoys and mooring system design; Design criteria and
\end{flushleft}


\begin{flushleft}
code provisions. Environmental loading. Wind, wave and current
\end{flushleft}


\begin{flushleft}
loads. Loads and stability during handling and towing. Introduction
\end{flushleft}


\begin{flushleft}
to stochastic dynamics of ocean structures considering different sea
\end{flushleft}


\begin{flushleft}
spectra. Soil-structure interaction (SSI): beam on Winkler foundation
\end{flushleft}


\begin{flushleft}
(p-y curve approach). Dynamic analysis of SPAR platforms. Fatigue
\end{flushleft}


\begin{flushleft}
analysis of fixed and floating offshore structure: stress concentration,
\end{flushleft}


\begin{flushleft}
S-N curves. Foundations: site investigations, gravity, jacket platforms,
\end{flushleft}


\begin{flushleft}
hybrid platforms. Piled foundation and behavior under dynamic
\end{flushleft}


\begin{flushleft}
loading. Static and dynamic analysis of platforms and components.
\end{flushleft}


\begin{flushleft}
Dynamic analysis using software: response of fixed type offshore
\end{flushleft}


\begin{flushleft}
structures, articulated towers, single leg and multi-legged towers.
\end{flushleft}





\begin{flushleft}
CVL863 General Continuum Mechanics
\end{flushleft}


\begin{flushleft}
3 Credits (3-0-0)
\end{flushleft}


\begin{flushleft}
Introduction: Field and particle theories in physics. Historical
\end{flushleft}


\begin{flushleft}
development of continuum mechanics-A basic engineering science.
\end{flushleft}


\begin{flushleft}
Classical theories: Stress and kinematics. Elasticity, Viscoelasticity and
\end{flushleft}


\begin{flushleft}
Elastoplasticity; Newtonian fluids.
\end{flushleft}


\begin{flushleft}
Continuum thermomechanics; Classius-Duhem Inequality;
\end{flushleft}


\begin{flushleft}
Thermodynamics with internal variables. Constitutive equations;
\end{flushleft}


\begin{flushleft}
Axioms for simple materials; Frame indifference; Finite elasticity;
\end{flushleft}


\begin{flushleft}
Hyper/hypoelasticity; Non- Newtonian fluids.
\end{flushleft}





182





\begin{flushleft}
\newpage
Civil Engineering
\end{flushleft}





\begin{flushleft}
Polar and nonlocal materials; Materials of differential/gradient
\end{flushleft}


\begin{flushleft}
type; Configurational mechanics; Biomechanics; Nanomechanics.
\end{flushleft}


\begin{flushleft}
Theories of conduction and diffusion; Electromagnetism. Coupled
\end{flushleft}


\begin{flushleft}
fields: Thermoelasticity and electromagnetoelasticity; MHD;
\end{flushleft}


\begin{flushleft}
Chemomechanics. Intermediate problems; Statistical continuum
\end{flushleft}


\begin{flushleft}
theories; Relativistic continuum mechanics; Materials models for
\end{flushleft}


\begin{flushleft}
luminiferous Aether.
\end{flushleft}


\begin{flushleft}
Rational methodology and realism; Current trends.
\end{flushleft}





\begin{flushleft}
CVL864 Structural Health Monitoring
\end{flushleft}


\begin{flushleft}
3 Credits (2-0-2)
\end{flushleft}


\begin{flushleft}
Concept of structural health monitoring, sensor systems and hardware
\end{flushleft}


\begin{flushleft}
requirements, global and local techniques, computational aspects
\end{flushleft}


\begin{flushleft}
of global dynamic techniques, experimental mode shapes, damage
\end{flushleft}


\begin{flushleft}
localization and quantification, piezo--electric materials and other
\end{flushleft}


\begin{flushleft}
smart materials, electro--mechanical impedance (EMI) technique,
\end{flushleft}


\begin{flushleft}
adaptations of EMI technique.
\end{flushleft}


\begin{flushleft}
Laboratory: Sensor installation and diagnostics, mode shape
\end{flushleft}


\begin{flushleft}
extraction, location and quantification of damage using global dynamic
\end{flushleft}


\begin{flushleft}
techniques, damage detection using electro -- mechanical impedance
\end{flushleft}


\begin{flushleft}
technique, remote monitoring.
\end{flushleft}





\begin{flushleft}
CVL865 Structural Vibration Control
\end{flushleft}


\begin{flushleft}
3 Credits (3-0-0)
\end{flushleft}


\begin{flushleft}
Pre-requisites: CVL759
\end{flushleft}


\begin{flushleft}
Introduction; Types and classifications; Control theories; Optimal
\end{flushleft}


\begin{flushleft}
stiffness distributions for building type structures; Role of damping
\end{flushleft}


\begin{flushleft}
in controlling motion; Active and semi-active systems; Tuned mass
\end{flushleft}


\begin{flushleft}
dampers - single/ multiple; Quasi-static active control; Passive control:
\end{flushleft}


\begin{flushleft}
viscous, visco-elastic, friction, hysteretic dampers, base isolation;
\end{flushleft}


\begin{flushleft}
Nonlinear modeling; Dynamic feedback control; Neural network based
\end{flushleft}


\begin{flushleft}
control systems; Design for buildings, bridges, power plants, and other
\end{flushleft}


\begin{flushleft}
structures; Current trends and performance-based design.
\end{flushleft}





\begin{flushleft}
CVL866 Wind Resistant Design of Structures
\end{flushleft}


\begin{flushleft}
3 Credits (3-0-0)
\end{flushleft}


\begin{flushleft}
Pre-requisites: CVL759
\end{flushleft}


\begin{flushleft}
Causes and types of wind. Atmospheric boundary layer and
\end{flushleft}


\begin{flushleft}
turbulence. Wind velocity measurements and distribution. Bluffbody
\end{flushleft}


\begin{flushleft}
aerodynamics, random vibrations, and spectral analysis. Along
\end{flushleft}


\begin{flushleft}
wind and across wind response considering vortex shedding of
\end{flushleft}


\begin{flushleft}
tall buildings, towers, and slender structures. Vortex induced
\end{flushleft}


\begin{flushleft}
vibrations of slender structures. Wind-Induced lock-in excitation of
\end{flushleft}


\begin{flushleft}
tall structures. Buffeting response prediction subjected to random
\end{flushleft}


\begin{flushleft}
load. Aeroelastic phenomena. Turbulence modeling. Gust buffeting
\end{flushleft}


\begin{flushleft}
and fluttering effect on structures. Vibration of cable supported
\end{flushleft}


\begin{flushleft}
bridges and power lines due to wind effects. Wind pressure on
\end{flushleft}


\begin{flushleft}
cooling towers. Design of cladding and wind damping devices. Wind
\end{flushleft}


\begin{flushleft}
tunnel simulations and tornado effects.
\end{flushleft}





\begin{flushleft}
CVL871 Durability and Repair of Concrete Structures
\end{flushleft}


\begin{flushleft}
3 Credits (3-0-0)
\end{flushleft}


\begin{flushleft}
Chemical composition of concrete, permeability and transport
\end{flushleft}


\begin{flushleft}
processes, corrosion of reinforcement and prestressing steel in
\end{flushleft}


\begin{flushleft}
concrete, carbonation, chloride attack, alkali-silica reaction, freezethaw attack, sulphate attack, acid attack, effect of fire and high
\end{flushleft}


\begin{flushleft}
temperatures and seawater attack, cracking, weathering, biological
\end{flushleft}


\begin{flushleft}
processes, non-destructive testing, repairs, protection and retrofitting,
\end{flushleft}


\begin{flushleft}
durability based design of structures.
\end{flushleft}





\begin{flushleft}
CVL872 Infrastructure Development and Management
\end{flushleft}


\begin{flushleft}
3 Credits (3-0-0)
\end{flushleft}


\begin{flushleft}
Introduction to Indian Infrastructure. Govt. initiatives through various
\end{flushleft}


\begin{flushleft}
five year plans.
\end{flushleft}


\begin{flushleft}
Overview of various sectors of infrastructure and SEZ.
\end{flushleft}





\begin{flushleft}
Infrastructure procurement through Public-Private-Partnership.
\end{flushleft}


\begin{flushleft}
Sector-wise differences in policies, Concession agreement,
\end{flushleft}


\begin{flushleft}
Selection procedure of concessionaires, Issues in financial closure,
\end{flushleft}


\begin{flushleft}
Stakeholder management.
\end{flushleft}


\begin{flushleft}
Financial Models, Risk management, Environmental Impact
\end{flushleft}


\begin{flushleft}
Assessment, Case studies.
\end{flushleft}





\begin{flushleft}
CVL873 Fire Engineering and Design
\end{flushleft}


\begin{flushleft}
3 Credits (3-0-0)
\end{flushleft}


\begin{flushleft}
(A) Fire engineering: fundamentals of fire science, fire dynamics,
\end{flushleft}


\begin{flushleft}
hazard mitigation, and safety; codes, standards, rules and fire safety
\end{flushleft}


\begin{flushleft}
regulations; thermodynamics, thermofluids, heat and mass transfer;
\end{flushleft}


\begin{flushleft}
human behavior in fire and urban planning; fire testing methods for
\end{flushleft}


\begin{flushleft}
materials; large-scale fire testing. {``}Fire protection'' - current methods
\end{flushleft}


\begin{flushleft}
in fire safety engineering; mechanics of repair; mitigation of fire
\end{flushleft}


\begin{flushleft}
damage by due design, and construction; industrial fire safety. Passive
\end{flushleft}


\begin{flushleft}
fire protection: analyzing the thermal effects of fires on buildings and
\end{flushleft}


\begin{flushleft}
designing structural members. Introduction to active fire protection.
\end{flushleft}


\begin{flushleft}
(B) Structural fire engineering: fire behavior and scenarios, heat
\end{flushleft}


\begin{flushleft}
transfer to the structure, structural response and stability under
\end{flushleft}


\begin{flushleft}
thermo-mechanical loads; fire safety design; mechanical properties of
\end{flushleft}


\begin{flushleft}
structural materials at elevated temperatures; fire response of steel,
\end{flushleft}


\begin{flushleft}
concrete, fiber reinforced polymers, high-performance materials etc.;
\end{flushleft}


\begin{flushleft}
computational procedures to predict structural behavior under fire
\end{flushleft}


\begin{flushleft}
conditions; structural fire resistance based on theoretical/ empirical
\end{flushleft}


\begin{flushleft}
relationships; performance-based fire engineering; strengthening/
\end{flushleft}


\begin{flushleft}
repair of structures.
\end{flushleft}





\begin{flushleft}
CVL874 Quality and Safety in Construction
\end{flushleft}


\begin{flushleft}
3 Credits (3-0-0)
\end{flushleft}


\begin{flushleft}
Introduction to safety. Types of injuries, Factors affecting safety,
\end{flushleft}


\begin{flushleft}
Strategic Planning for safety provisions. Personal \& Structural safety Safety consideration during construction, demolition and during use of
\end{flushleft}


\begin{flushleft}
equipment. Recording injuries and accident indices. Method statement,
\end{flushleft}


\begin{flushleft}
SOPs, PPE, Inspections, Investigations. Site safety programmes - JSA,
\end{flushleft}


\begin{flushleft}
JHA, Root cause analysis, meetings, safety policy, manuals, training
\end{flushleft}


\begin{flushleft}
\& orientation. Safety legislation regard to violation.
\end{flushleft}


\begin{flushleft}
Introduction to quality, assurance, control and audit. Regulatory
\end{flushleft}


\begin{flushleft}
agent - owner, designer, contractor. Strategic Planning and control of
\end{flushleft}


\begin{flushleft}
quality during design and construction, Quality tools in construction
\end{flushleft}


\begin{flushleft}
projects, Customer satisfaction and QFD, Quantitative techniques in
\end{flushleft}


\begin{flushleft}
quality control, Quality assurance during construction, Inspection of
\end{flushleft}


\begin{flushleft}
materials and machinery. Assessing quality. Teachings/findings of
\end{flushleft}


\begin{flushleft}
Gurus - Concept and philosophy of TQM, 6Sigma, ISO Certification.
\end{flushleft}


\begin{flushleft}
IS codes and standards regard to quality \& safety.
\end{flushleft}





\begin{flushleft}
CVL875 Sustainable Materials and Green Buildings
\end{flushleft}


\begin{flushleft}
3 Credits (3-0-0)
\end{flushleft}


\begin{flushleft}
Introduction and definition of Sustainability. Carbon cycle and role of
\end{flushleft}


\begin{flushleft}
construction material such as concrete and steel, etc. CO2 contribution
\end{flushleft}


\begin{flushleft}
from cement and other construction materials. Construction
\end{flushleft}


\begin{flushleft}
materials and indoor air quality. No/Low cement concrete. Recycled
\end{flushleft}


\begin{flushleft}
and manufactured aggregate. Role of QC and durability. Life cycle
\end{flushleft}


\begin{flushleft}
and sustainability. Components of embodied energy, calculation of
\end{flushleft}


\begin{flushleft}
embodied energy for construction materials. Exergy concept and
\end{flushleft}


\begin{flushleft}
primary energy. Embodied energy via-a-vis operational energy in
\end{flushleft}


\begin{flushleft}
conditioned building. Life Cycle energy use. Control of energy use in
\end{flushleft}


\begin{flushleft}
building, ECBC code, codes in neighboring tropical countries, OTTV
\end{flushleft}


\begin{flushleft}
concepts and calculations, features of LEED and TERI Griha ratings.
\end{flushleft}


\begin{flushleft}
Role of insulation and thermal properties of construction materials,
\end{flushleft}


\begin{flushleft}
influence of moisture content and modeling. Performance ratings of
\end{flushleft}


\begin{flushleft}
green buildings. Zero energy building.
\end{flushleft}





\begin{flushleft}
CVD895 MS Research Project
\end{flushleft}


\begin{flushleft}
36 Credits (0-0-72)
\end{flushleft}





183





\begin{flushleft}
\newpage
Department of Computer Science and Engineering
\end{flushleft}


\begin{flushleft}
COL100 Introduction to Computer Science
\end{flushleft}


\begin{flushleft}
4 Credits (3-0-2)
\end{flushleft}


\begin{flushleft}
Organization of Computing Systems. Concept of an algorithm;
\end{flushleft}


\begin{flushleft}
termination and correctness. Algorithms to programs: specification,
\end{flushleft}


\begin{flushleft}
top-down development and stepwise refinement. Problem solving
\end{flushleft}


\begin{flushleft}
using a functional style; Correctness issues in programming; Efficiency
\end{flushleft}


\begin{flushleft}
issues in programming; Time and space measures. Procedures,
\end{flushleft}


\begin{flushleft}
functions. Data types, representational invariants. Encapsulation,
\end{flushleft}


\begin{flushleft}
abstractions, interaction and modularity. Identifying and exploiting
\end{flushleft}


\begin{flushleft}
inherent concurrency. Structured style of imperative programming.
\end{flushleft}


\begin{flushleft}
Introduction to numerical methods. At least one example of large
\end{flushleft}


\begin{flushleft}
program development.
\end{flushleft}





\begin{flushleft}
COL106 Data Structures \& Algorithms
\end{flushleft}


\begin{flushleft}
5 Credits (3-0-4)
\end{flushleft}


\begin{flushleft}
Pre-requisites: COL100
\end{flushleft}


\begin{flushleft}
Introduction to object-oriented programming through stacks queues
\end{flushleft}


\begin{flushleft}
and linked lists. Dictionaries; skip-lists, hashing, analysis of collision
\end{flushleft}


\begin{flushleft}
resolution techniques. Trees, traversals, binary search trees, optimal
\end{flushleft}


\begin{flushleft}
and average BSTs. Balanced BST: AVL Trees, 2-4 trees, red-black
\end{flushleft}


\begin{flushleft}
trees, B-trees. Tries and suffix trees. Priority queues and binary heaps.
\end{flushleft}


\begin{flushleft}
Sorting: merge, quick, radix, selection and heap sort, Graphs: Breadth
\end{flushleft}


\begin{flushleft}
first search and connected components. Depth first search in directed
\end{flushleft}


\begin{flushleft}
and undirected graphs.
\end{flushleft}


\begin{flushleft}
Disjkra's algorithm, directed acyclic graphs and topological sort. Some
\end{flushleft}


\begin{flushleft}
geometric data-structures.
\end{flushleft}





\begin{flushleft}
COL202 Discrete Mathematical Structures
\end{flushleft}


\begin{flushleft}
4 Credits (3-1-0)
\end{flushleft}


\begin{flushleft}
Overlaps with: MTL180
\end{flushleft}


\begin{flushleft}
Propositional logic, Predicate Calculus and Quantifiers; Proof Methods;
\end{flushleft}


\begin{flushleft}
Sets, functions, relations, Cardinality, Infinity and Diagonalization;
\end{flushleft}


\begin{flushleft}
Induction and Recursion; Modular Arithmetic, Euclid's Algorithm,
\end{flushleft}


\begin{flushleft}
primes, Public Key Cryptography; Polynomials, finite fields and Secret
\end{flushleft}


\begin{flushleft}
Sharing; Coding Theory: Error correcting codes, Hamming codes,
\end{flushleft}


\begin{flushleft}
Hamming bound; Basic Counting - Pigeon hole principle; Advanced
\end{flushleft}


\begin{flushleft}
Counting - recurrence relations, generating functions, inclusionexclusion; basic information theory, entropy, Kraft's inequality,
\end{flushleft}


\begin{flushleft}
mutual information, lower bounds; Probability - sample space,
\end{flushleft}


\begin{flushleft}
conditional probability, expectation, linearity of expectation, variance,
\end{flushleft}


\begin{flushleft}
Markov, Chebychev, probabilistic methods; Graph Theory - Eulerian,
\end{flushleft}


\begin{flushleft}
Hamiltonian \& planar graphs, edge and vertex coloring.
\end{flushleft}





\begin{flushleft}
COL226 Programming Languages
\end{flushleft}


\begin{flushleft}
5 Credits (3-0-4)
\end{flushleft}


\begin{flushleft}
Pre-requisites: COL106
\end{flushleft}


\begin{flushleft}
Value and state oriented paradigms. Translation. Notions of syntax
\end{flushleft}


\begin{flushleft}
and semantics of programming languages; introduction to operational/
\end{flushleft}


\begin{flushleft}
natural semantics of functional and imperative languages. Data
\end{flushleft}


\begin{flushleft}
abstractions and control constructs; block-structure and scope,
\end{flushleft}


\begin{flushleft}
principles of abstraction, qualification and correspondence; parameter
\end{flushleft}


\begin{flushleft}
passing mechanisms; runtime structure and operating environment;
\end{flushleft}


\begin{flushleft}
practical and implementation issues in run-time systems and
\end{flushleft}


\begin{flushleft}
environment; abstract machines; features of functional and imperative
\end{flushleft}


\begin{flushleft}
languages; the un-typed and simply-typed Lambda calculus, type
\end{flushleft}


\begin{flushleft}
systems for programming languages including simple types and
\end{flushleft}


\begin{flushleft}
polymorphism; objects; classes and inheritance in object-oriented
\end{flushleft}


\begin{flushleft}
languages. Interactive programming and interfaces. The laboratory
\end{flushleft}


\begin{flushleft}
activities will involve building a variety of small interpreters for core
\end{flushleft}


\begin{flushleft}
languages in various paradigms. Tools such as lex and yacc will be
\end{flushleft}


\begin{flushleft}
introduced for front-end analysis.
\end{flushleft}





\begin{flushleft}
COP290 Design Practices
\end{flushleft}


\begin{flushleft}
3 Credits (0-0-6)
\end{flushleft}


\begin{flushleft}
Pre-requisites: COL 106
\end{flushleft}


\begin{flushleft}
The contents may differ each year depending on the instructor.
\end{flushleft}


\begin{flushleft}
The course should involve 2-3 large programming projects done in
\end{flushleft}


\begin{flushleft}
groups of 2-4.
\end{flushleft}





\begin{flushleft}
COD300 Design Project
\end{flushleft}


\begin{flushleft}
2 Credits (0-0-4)
\end{flushleft}


\begin{flushleft}
Basic design methodology -- introduction to the steps involved,
\end{flushleft}


\begin{flushleft}
Familiarization with software practices, tools and techniques, software
\end{flushleft}


\begin{flushleft}
project involving conceptualization, design analysis, implementation
\end{flushleft}


\begin{flushleft}
and testing using the tools and techniques learnt.
\end{flushleft}





\begin{flushleft}
COD310 Mini Project
\end{flushleft}


\begin{flushleft}
3 Credits (0-0-6)
\end{flushleft}


\begin{flushleft}
Design/fabrication/implementation work under the guidance of a
\end{flushleft}


\begin{flushleft}
faculty member. Prior to registration, a detailed plan of work should be
\end{flushleft}


\begin{flushleft}
submitted by the student to the Head of the Department for approval.
\end{flushleft}





\begin{flushleft}
COR310 Professional Practices (CS)
\end{flushleft}


\begin{flushleft}
2 Credits (1-0-2)
\end{flushleft}


\begin{flushleft}
EC - 60
\end{flushleft}


\begin{flushleft}
The course would consist of talks by working professionals from
\end{flushleft}


\begin{flushleft}
industry, government and research organizations. It may also include
\end{flushleft}


\begin{flushleft}
site visits to various organizations.
\end{flushleft}





\begin{flushleft}
COL215 Digital Logic \& System Design
\end{flushleft}


\begin{flushleft}
5 Credits (3-0-4)
\end{flushleft}


\begin{flushleft}
Pre-requisites: COL100, ELL100
\end{flushleft}


\begin{flushleft}
Overlaps with: ELL201
\end{flushleft}


\begin{flushleft}
The course contents can be broadly divided into two parts. First part
\end{flushleft}


\begin{flushleft}
deals with the basics of circuit design and includes topics like circuit
\end{flushleft}


\begin{flushleft}
minimization, sequential circuit design and design of and using RTL
\end{flushleft}


\begin{flushleft}
building blocks. The second part is focused on ASIC style system
\end{flushleft}


\begin{flushleft}
design and introduces VHDL, FPGA as implementation technology,
\end{flushleft}


\begin{flushleft}
synthesis steps as well as testing techniques. Course ends with
\end{flushleft}


\begin{flushleft}
introducing the challenges of embedded design where software is
\end{flushleft}


\begin{flushleft}
becoming integral to all devices. The laboratory assignments are a
\end{flushleft}


\begin{flushleft}
key component of this course and requires students to design and
\end{flushleft}


\begin{flushleft}
implement circuits and sub-systems on FPGA kits covering almost all
\end{flushleft}


\begin{flushleft}
the topics covered in the lectures.
\end{flushleft}





\begin{flushleft}
COL216 Computer Architecture
\end{flushleft}


\begin{flushleft}
4 Credits (3-0-2)
\end{flushleft}


\begin{flushleft}
Pre-requisites: ELL201
\end{flushleft}


\begin{flushleft}
Overlaps with: ELL305
\end{flushleft}


\begin{flushleft}
History of computers, Boolean logic and number systems, Assembly
\end{flushleft}


\begin{flushleft}
language programming, ARM assembly language, Computer arithmetic,
\end{flushleft}


\begin{flushleft}
Design of a basic processor, Microprogramming, Pipelining, Memory
\end{flushleft}


\begin{flushleft}
system, Virtual memory, I/O protocols and devices, Multiprocessors.
\end{flushleft}





\begin{flushleft}
COS310 Independent Study (CS)
\end{flushleft}


\begin{flushleft}
3 Credits (0-3-0)
\end{flushleft}


\begin{flushleft}
EC - 60
\end{flushleft}


\begin{flushleft}
Research oriented activities or study of subjects outside regular course
\end{flushleft}


\begin{flushleft}
offerings under the guidance of a faculty member. Prior to registration,
\end{flushleft}


\begin{flushleft}
a detailed plan of work should be submitted by the student to the
\end{flushleft}


\begin{flushleft}
Head of the Department for approval.
\end{flushleft}





\begin{flushleft}
COP315 Embedded System Design Project
\end{flushleft}


\begin{flushleft}
4 Credits (0-1-6)
\end{flushleft}


\begin{flushleft}
Pre-requisites: COL215, COL216 or equivalent courses
\end{flushleft}


\begin{flushleft}
Students working in small groups of four to six are expected to
\end{flushleft}


\begin{flushleft}
deliver in one semester on an innovative solution for problems/
\end{flushleft}


\begin{flushleft}
challenges that are typical to India and perhaps other developing
\end{flushleft}


\begin{flushleft}
countries. The students would have to go through the full cycle of
\end{flushleft}


\begin{flushleft}
specification, design and prototyping/building a concept demonstrator.
\end{flushleft}


\begin{flushleft}
Key component of the assessment would be through a public
\end{flushleft}


\begin{flushleft}
demonstration of their solution.
\end{flushleft}


\begin{flushleft}
Learning to work in groups as well as planning and delivering a large
\end{flushleft}


\begin{flushleft}
task are other expected learnings.
\end{flushleft}





184





\begin{flushleft}
\newpage
Computer Science
\end{flushleft}





\begin{flushleft}
COL331 Operating Systems
\end{flushleft}


\begin{flushleft}
5 Credits (3-0-4)
\end{flushleft}


\begin{flushleft}
Pre-requisites: COL106 COP290
\end{flushleft}


\begin{flushleft}
Overlaps with: ELL405
\end{flushleft}





\begin{flushleft}
Intractability, NP-completeness, Polynomial time reductions. String
\end{flushleft}


\begin{flushleft}
matching, KMP and Rabin-Karp. Universal hashing and applications.
\end{flushleft}


\begin{flushleft}
Geometric algorithms like convex hulls, multidimensional data
\end{flushleft}


\begin{flushleft}
structures, plane sweep paradigm.
\end{flushleft}





\begin{flushleft}
Primary UNIX abstractions: threads, address spaces, file system,
\end{flushleft}


\begin{flushleft}
devices, inter process communication; Introduction to hardware
\end{flushleft}


\begin{flushleft}
support for OS (e.g., discuss x86 architecture); Processes and Memory;
\end{flushleft}


\begin{flushleft}
Address Translation; Interrupts and Exceptions; Context Switching;
\end{flushleft}


\begin{flushleft}
Scheduling; Multiprocessors and Locking; Condition Variables,
\end{flushleft}


\begin{flushleft}
Semaphores, Barriers, Message Passing, etc.; File system semantics,
\end{flushleft}


\begin{flushleft}
design and implementation; File system Durability and Crash recovery;
\end{flushleft}


\begin{flushleft}
Security and Access Control.
\end{flushleft}





\begin{flushleft}
COL333 Principles of Artificial Intelligence
\end{flushleft}


\begin{flushleft}
4 Credits (3-0-2)
\end{flushleft}


\begin{flushleft}
Pre-requisites: COL106
\end{flushleft}


\begin{flushleft}
Overlaps with: COL671, COL770, ELL789
\end{flushleft}


\begin{flushleft}
Philosophy of artificial intelligence, problem solving, search techniques,
\end{flushleft}


\begin{flushleft}
constraint satisfaction, game playing (minimax, expectiminimax),
\end{flushleft}


\begin{flushleft}
automated planning, knowledge representation and reasoning through
\end{flushleft}


\begin{flushleft}
logic, knowledge representation and reasoning through fuzzy logic
\end{flushleft}


\begin{flushleft}
and Bayesian networks, Markov decision processes, machine learning,
\end{flushleft}


\begin{flushleft}
neural networks, reinforcement learning, soft computing, introduction
\end{flushleft}


\begin{flushleft}
to natural language processing.
\end{flushleft}





\begin{flushleft}
COL334 Computer Networks
\end{flushleft}


\begin{flushleft}
4 Credits (3-0-2)
\end{flushleft}


\begin{flushleft}
Pre-requisites: COL106, COL216
\end{flushleft}


\begin{flushleft}
Overlaps with: ELL402
\end{flushleft}


\begin{flushleft}
Students will be exposed to common network algorithms and protocols,
\end{flushleft}


\begin{flushleft}
including physical layer modulation (analog AM/FM, digital ASK/FSK/
\end{flushleft}


\begin{flushleft}
PSK), encoding (NRZ, Manchester, 4B/5B), link layer framing, error
\end{flushleft}


\begin{flushleft}
control, medium access control (TDMA, FDMA, CSMA/CA, CSMA/
\end{flushleft}


\begin{flushleft}
CD), bridging, SDN, addressing (IPv4/v6), name resolution (DNS),
\end{flushleft}


\begin{flushleft}
routing (DV, LS, protocols RIP, OSPF, BGP), transport protocols
\end{flushleft}


\begin{flushleft}
(TCP), congestion avoidance (window based AIMD), and application
\end{flushleft}


\begin{flushleft}
design models (client-server, P2P, functioning of HTTP, SMTP,
\end{flushleft}


\begin{flushleft}
IMAP). Programming assignments will be designed to test network
\end{flushleft}


\begin{flushleft}
application design concepts, protocol design towards developing error
\end{flushleft}


\begin{flushleft}
detection and correction methods, efficient network utilization, and
\end{flushleft}


\begin{flushleft}
familiarization with basic tools such as ping, trace route, wires hark.
\end{flushleft}





\begin{flushleft}
COL341 Fundamentals of Machine Learning
\end{flushleft}


\begin{flushleft}
4 Credits (3-0-2)
\end{flushleft}


\begin{flushleft}
Pre-requisites: COL106, MTL106
\end{flushleft}


\begin{flushleft}
Overlaps with: ELL409, ELL784
\end{flushleft}


\begin{flushleft}
Supervised Learning Algorithms: 1. Logistic Regression 2.Neural
\end{flushleft}


\begin{flushleft}
Networks 3.Decision Trees 4.Nearest Neighbour 5. Support Vector
\end{flushleft}


\begin{flushleft}
Machines 6. Naive Bayes. ML and MAP estimates. Bayes' Optimal
\end{flushleft}


\begin{flushleft}
Classifier. Introduction to Graphical Models. Generative Vs.
\end{flushleft}


\begin{flushleft}
Discriminative Models. Unsupervised learning algorithms: K-Means
\end{flushleft}


\begin{flushleft}
clustering, Expectation Maximization, Gaussian Mixture Models. PCA
\end{flushleft}


\begin{flushleft}
and Feature Selection, PAC Learnability, Reinforcement Learning.
\end{flushleft}


\begin{flushleft}
Some application areas of machine learning e.g. Natural Language
\end{flushleft}


\begin{flushleft}
Processing, Computer Vision, applications on the web. Introduction
\end{flushleft}


\begin{flushleft}
to advanced topics such as Statistical Relational Learning.
\end{flushleft}





\begin{flushleft}
COL351 Analysis and Design of Algorithms
\end{flushleft}


\begin{flushleft}
4 Credits (3-1-0)
\end{flushleft}


\begin{flushleft}
Pre-requisites: COL106
\end{flushleft}


\begin{flushleft}
Overlaps with: MTL342, COL702
\end{flushleft}


\begin{flushleft}
Checking 2-edge, 2-node and strong connectivity using DFS, Strongly
\end{flushleft}


\begin{flushleft}
connected components. Greedy algorithms, minimum spanning
\end{flushleft}


\begin{flushleft}
trees (Prim/Kruskal), Union-find data structure. Matroids. Divide and
\end{flushleft}


\begin{flushleft}
conquer algorithms. Polynomial multiplication, DFT and FFT. Dynamic
\end{flushleft}


\begin{flushleft}
Programming, All pairs shortest paths (Bellman-Ford, Floyd Warshall).
\end{flushleft}


\begin{flushleft}
s-t flows, Ford-Fulkerson, Edmonds-Karp, applications of maxflow
\end{flushleft}





\begin{flushleft}
COL352 Introduction to Automata \& Theory of
\end{flushleft}


\begin{flushleft}
Computation
\end{flushleft}


\begin{flushleft}
3 Credits (3-0-0)
\end{flushleft}


\begin{flushleft}
Pre-requisites: COL202
\end{flushleft}


\begin{flushleft}
Overlaps with: MTL383
\end{flushleft}


\begin{flushleft}
Regular Languages, Finite Automata, equivalence, minimization,
\end{flushleft}


\begin{flushleft}
Myhill-Nerode Theorem, introduction to non-determinism, Context
\end{flushleft}


\begin{flushleft}
free grammars, Pushdown automata, equivalence and applications.
\end{flushleft}


\begin{flushleft}
Turing machines, Recursive and Recursively enumerable sets, nondeterminism, RAMs and equivalence, Universal Turing Machines,
\end{flushleft}


\begin{flushleft}
undecidability, Rice's theorems for RE sets, Post machines, Basics of
\end{flushleft}


\begin{flushleft}
Recursive function theory. Equivalence, Church's thesis, computational
\end{flushleft}


\begin{flushleft}
complexity, space and time complexity of Turing Machines,
\end{flushleft}


\begin{flushleft}
Relationships, Savage's theorem, Complexity classes, Complete
\end{flushleft}


\begin{flushleft}
problems, NP-completeness, Cook-Levin theorem.
\end{flushleft}





\begin{flushleft}
COL362 Introduction to Database Management Systems
\end{flushleft}


\begin{flushleft}
4 Credits (3-0-2)
\end{flushleft}


\begin{flushleft}
Pre-requisites: COL106
\end{flushleft}


\begin{flushleft}
Overlaps with: MTL710
\end{flushleft}


\begin{flushleft}
Data models (ER, relational models, constraints, normalization),
\end{flushleft}


\begin{flushleft}
declarative querying (relational algebra, datalog, SQL), query
\end{flushleft}


\begin{flushleft}
processing/optimization (basics of indexes, logical/physical query plans,
\end{flushleft}


\begin{flushleft}
views) and transaction management (introduction to concurrency
\end{flushleft}


\begin{flushleft}
control and recovery). Overview of XML data management, text
\end{flushleft}


\begin{flushleft}
management, distributed data management. Course project to build
\end{flushleft}


\begin{flushleft}
a web-based database application.
\end{flushleft}





\begin{flushleft}
COL380 Introduction to Parallel \& Distributed
\end{flushleft}


\begin{flushleft}
Programming
\end{flushleft}


\begin{flushleft}
3 Credits (2-0-2)
\end{flushleft}


\begin{flushleft}
Pre-requisites: COL106, COL351, COL331
\end{flushleft}


\begin{flushleft}
Overlaps with: COL730, MTL765
\end{flushleft}


\begin{flushleft}
Concurrency, Consistency of state and memory, Parallel architecture,
\end{flushleft}


\begin{flushleft}
Latency and throughput, Models of parallel computation, performance
\end{flushleft}


\begin{flushleft}
metrics and speedup, Message-passing and Shared-memory
\end{flushleft}


\begin{flushleft}
programming paradigms, Communication networks and primitives,
\end{flushleft}


\begin{flushleft}
Concepts of Atomicity, Consensus, Conditions and Synchronization,
\end{flushleft}


\begin{flushleft}
Security, Fault tolerance, Replication of state and memory.
\end{flushleft}





\begin{flushleft}
COD490 B.Tech. Project
\end{flushleft}


\begin{flushleft}
6 Credits (0-0-12)
\end{flushleft}


\begin{flushleft}
Pre-requisites: EC 100
\end{flushleft}


\begin{flushleft}
Overlaps with: COD492
\end{flushleft}


\begin{flushleft}
This course is designed for CSE B.Tech. students who do not seek
\end{flushleft}


\begin{flushleft}
departmental specialization. The course is done, usually in groups,
\end{flushleft}


\begin{flushleft}
under the supervision of one or more faculty members of the computer
\end{flushleft}


\begin{flushleft}
science department.
\end{flushleft}





\begin{flushleft}
COD492 B.Tech. Project Part-I
\end{flushleft}


\begin{flushleft}
6 Credits (0-0-12)
\end{flushleft}


\begin{flushleft}
Pre-requisites: EC 100
\end{flushleft}


\begin{flushleft}
Overlaps with: COD490
\end{flushleft}


\begin{flushleft}
This course is part-1 of a large project and is designed for CSE B.Tech.
\end{flushleft}


\begin{flushleft}
students seeking department specialization. This project is done
\end{flushleft}


\begin{flushleft}
individually, or sometimes in small groups, under the supervision of
\end{flushleft}


\begin{flushleft}
one or more faculty member of the computer science department.
\end{flushleft}


\begin{flushleft}
This project spans also the course COD494. Hence it is expected
\end{flushleft}


\begin{flushleft}
that the problem specification and the milestones to be achieved
\end{flushleft}


\begin{flushleft}
in solving the problem are clearly specified. Students not seeking
\end{flushleft}


\begin{flushleft}
specialization may takes this course of if they are interested in the
\end{flushleft}


\begin{flushleft}
COD490-COD492 sequence.
\end{flushleft}





185





\begin{flushleft}
\newpage
Computer Science
\end{flushleft}





\begin{flushleft}
COD494 B.Tech. Project Part-II
\end{flushleft}


\begin{flushleft}
8 Credits (0-0-16)
\end{flushleft}


\begin{flushleft}
Pre-requisites: COD492, EC 100
\end{flushleft}





\begin{flushleft}
COP701 Software Systems Laboratory
\end{flushleft}


\begin{flushleft}
3 Credits (0-0-6)
\end{flushleft}





\begin{flushleft}
The student(s) who work on a project are expected to work towards
\end{flushleft}


\begin{flushleft}
the goals and milestones set in COD492. At the end there would be a
\end{flushleft}


\begin{flushleft}
demonstration of the solution and possible future work on the same
\end{flushleft}


\begin{flushleft}
problem. A dissertation outlining the entire problem, including a survey
\end{flushleft}


\begin{flushleft}
of literature and the various results obtained along with their solutions
\end{flushleft}


\begin{flushleft}
is expected to be produced.
\end{flushleft}





\begin{flushleft}
COL632: Introduction to Database Systems
\end{flushleft}


\begin{flushleft}
4 Credits (3-0-2)
\end{flushleft}


\begin{flushleft}
Pre-requisites: COL106 OR Equivalent
\end{flushleft}


\begin{flushleft}
Overlap with: COL362, MTL710
\end{flushleft}


\begin{flushleft}
Data models (ER, relational models, constraints, normalization),
\end{flushleft}


\begin{flushleft}
declarative querying (relational algebra, datalog, SQL), query
\end{flushleft}


\begin{flushleft}
processing/optimization (basics of indexes, logical/physical query plans,
\end{flushleft}


\begin{flushleft}
views) and transaction management (introduction to concurrency
\end{flushleft}


\begin{flushleft}
control and recovery). Overview of XML data management, text
\end{flushleft}


\begin{flushleft}
management, distributed data management. Course project to build
\end{flushleft}


\begin{flushleft}
a web-based database application.
\end{flushleft}





\begin{flushleft}
COL633: Resource Management in Computer Systems
\end{flushleft}


\begin{flushleft}
4 Credits (3-0-2)
\end{flushleft}


\begin{flushleft}
Pre-requisites: COL106 OR Equivalent
\end{flushleft}


\begin{flushleft}
Overlap with: COL331 EEL405, MTL358
\end{flushleft}


\begin{flushleft}
Primary UNIX abstractions: threads, address spaces, filesystem,
\end{flushleft}


\begin{flushleft}
devices, interprocess communication; Introduction to hardware
\end{flushleft}


\begin{flushleft}
support for OS (e.g., discuss x86 architecture); Processes and Memory;
\end{flushleft}


\begin{flushleft}
Address Translation; Interrupts and Exceptions; Context Switching;
\end{flushleft}


\begin{flushleft}
Scheduling; Multiprocessors and Locking; Condition Variables,
\end{flushleft}


\begin{flushleft}
Semaphores, Barriers, Message Passing, etc.; Filesystem semantics,
\end{flushleft}


\begin{flushleft}
design and implementation; Filesystem Durability and Crash recovery;
\end{flushleft}


\begin{flushleft}
Security and Access Control
\end{flushleft}





\begin{flushleft}
COL 671: Principles of Artificial Intelligence:
\end{flushleft}


\begin{flushleft}
4 Credits (3-0-2)
\end{flushleft}


\begin{flushleft}
Pre-requisites: COL106 OR Equivalent
\end{flushleft}


\begin{flushleft}
Overlap with: COL333, COL770, ELL789
\end{flushleft}


\begin{flushleft}
Problem solving, search techniques, control strategies, game playing
\end{flushleft}


\begin{flushleft}
(minimax), reasoning, knowledge representation through predicate
\end{flushleft}


\begin{flushleft}
logic, rule based systems, semantics nets, frames, conceptual
\end{flushleft}


\begin{flushleft}
dependency. Planning. Handling uncertainty: probability theory,
\end{flushleft}


\begin{flushleft}
Bayesian Networks, Dempster-Shafer theory, Fuzzy logic, Learning
\end{flushleft}


\begin{flushleft}
through Neural nets - Back propagation, radial basis functions, Neural
\end{flushleft}


\begin{flushleft}
computational models - Hopfield Nets, Boltzman machines. PROLOG
\end{flushleft}


\begin{flushleft}
programming. Expert Systems, Soft computing, introduction to natural
\end{flushleft}


\begin{flushleft}
language processing.
\end{flushleft}





\begin{flushleft}
COL 672: Computer Networks
\end{flushleft}


\begin{flushleft}
4 Credits (3-0-2)
\end{flushleft}


\begin{flushleft}
Pre-requisites: COL106 OR Equivalent
\end{flushleft}


\begin{flushleft}
Overlap with: COL334, ELL789
\end{flushleft}


\begin{flushleft}
Students will be exposed to common network algorithms and protocols,
\end{flushleft}


\begin{flushleft}
including physical layer modulation (analog AM/FM, digital ASK/FSK/
\end{flushleft}


\begin{flushleft}
PSK), encoding (NRZ, Manchester, 4B/5B), link layer framing, error
\end{flushleft}


\begin{flushleft}
control, medium access control (TDMA, FDMA, CSMA/CA, CSMA/
\end{flushleft}


\begin{flushleft}
CD), bridging, SDN, addressing (IPv4/v6), name resolution (DNS),
\end{flushleft}


\begin{flushleft}
routing (DV, LS, protocols RIP, OSPF, BGP), transport protocols
\end{flushleft}


\begin{flushleft}
(TCP), congestion avoidance (window based AIMD), and application
\end{flushleft}


\begin{flushleft}
design models (clientserver,P2P, functioning of HTTP, SMTP,
\end{flushleft}


\begin{flushleft}
IMAP). Programming assignments will be designed to test network
\end{flushleft}


\begin{flushleft}
application design concepts, protocol design towards developing error
\end{flushleft}


\begin{flushleft}
detection and correction methods, efficient network utilization, and
\end{flushleft}


\begin{flushleft}
familiarization with basic tools such as ping, traceroute, wireshark.
\end{flushleft}





\begin{flushleft}
The contents may differ each year depending on the instructor.
\end{flushleft}


\begin{flushleft}
The course should involve 2-3 large programming projects done in
\end{flushleft}


\begin{flushleft}
groups of 2-4. A set of three project oriented assignments which will
\end{flushleft}


\begin{flushleft}
be announced at the start of each semester with definite submission
\end{flushleft}


\begin{flushleft}
deadlines. The set of assignments will be designed to develop skills
\end{flushleft}


\begin{flushleft}
and familarity with a majority of the following: make, configuration
\end{flushleft}


\begin{flushleft}
management tools, installation of software, archiving and creation
\end{flushleft}


\begin{flushleft}
of libraries, version control systems, documentation and literate
\end{flushleft}


\begin{flushleft}
programming systems, GUI creation, distributed state maintenance
\end{flushleft}


\begin{flushleft}
over a network, programming in different environments like desktop
\end{flushleft}


\begin{flushleft}
and handhelds, program parsing and compilation including usage of
\end{flushleft}


\begin{flushleft}
standard libraries like pthreads, numerical packages, XML and semistructured data, simulation environments, testing and validation tools.
\end{flushleft}





\begin{flushleft}
COL702 Advanced Data Structures and Algorithms
\end{flushleft}


\begin{flushleft}
4 Credits (3-0-2)
\end{flushleft}


\begin{flushleft}
Pre-requisites: COL106 OR Equivalent
\end{flushleft}


\begin{flushleft}
Overlaps with: COL351
\end{flushleft}


\begin{flushleft}
Review of basic data structures and their realization in object oriented
\end{flushleft}


\begin{flushleft}
Environments. Dynamic Data structures: 2-3 trees, Redblack trees,
\end{flushleft}


\begin{flushleft}
binary heaps, binomial and Fibonacci heaps, Skip lists, Universal
\end{flushleft}


\begin{flushleft}
Hashing. Data structures for maintaining ranges, intervals and disjoint
\end{flushleft}


\begin{flushleft}
sets with applications. B-trees. Tries and suffix trees. Priority queues
\end{flushleft}


\begin{flushleft}
and binary heaps. Sorting: merge, quick, radix, selection and heap
\end{flushleft}


\begin{flushleft}
sort, Graphs: Breadth first search and connected components. Depth
\end{flushleft}


\begin{flushleft}
first search in directed and undirected graphs. Disjkra's algorithm,
\end{flushleft}


\begin{flushleft}
Directed acyclic graphs and topological sort. Some geometric datastructures. Basic algorithmic techniques like dynamic programming and
\end{flushleft}


\begin{flushleft}
divide-and-conquer. Sorting algorithms with analysis, integer sorting.
\end{flushleft}


\begin{flushleft}
Graph algorithms like DFS with applications, MSTs and shortest paths.
\end{flushleft}





\begin{flushleft}
COL703 Logic for Computer Science
\end{flushleft}


\begin{flushleft}
4 Credits (3-0-2)
\end{flushleft}


\begin{flushleft}
Pre-requisites: COL106 OR Equivalent
\end{flushleft}


\begin{flushleft}
Overlaps with: MTL747
\end{flushleft}


\begin{flushleft}
Review of the principle of mathematical induction; the principle
\end{flushleft}


\begin{flushleft}
of structural induction; review of Boolean algebras; Syntax of
\end{flushleft}


\begin{flushleft}
propositional formulas; Truth and the semantics of propositional logic;
\end{flushleft}


\begin{flushleft}
Notions of satisfiability, validity, inconsistency; Deduction systems for
\end{flushleft}


\begin{flushleft}
propositional logic; Completeness of deduction systems; First order
\end{flushleft}


\begin{flushleft}
logic (FOL); Proof theory for FOL; introduction to model theory;
\end{flushleft}


\begin{flushleft}
completeness and compactness theorems; First order theories.
\end{flushleft}


\begin{flushleft}
Programming exercises will include representation and evaluation;
\end{flushleft}


\begin{flushleft}
conversion to normal-forms; tautology checking; proof normalization;
\end{flushleft}


\begin{flushleft}
resolution; unification; Skolemization, conversion to Horn-clauses;
\end{flushleft}


\begin{flushleft}
binary-decision diagrams. Decidability, undecidability and complexity
\end{flushleft}


\begin{flushleft}
results. Introduction to formal methods, temporal/modal logics.
\end{flushleft}





\begin{flushleft}
COL718 Architecture of High Performance Computers
\end{flushleft}


\begin{flushleft}
4 Credits (3-0-2)
\end{flushleft}


\begin{flushleft}
Pre-requisites: COL216 OR Equivalent
\end{flushleft}


\begin{flushleft}
Classification of parallel computing structures; Instruction level
\end{flushleft}


\begin{flushleft}
parallelism - static and dynamic pipelining, improving branch
\end{flushleft}


\begin{flushleft}
performance, superscalar and VLIW processors; High performance
\end{flushleft}


\begin{flushleft}
memory system; Shared memory multiprocessors and cache
\end{flushleft}


\begin{flushleft}
coherence; Multiprocessor interconnection networks; Performance
\end{flushleft}


\begin{flushleft}
modelling; Issues in programming multiprocessors; Data parallel
\end{flushleft}


\begin{flushleft}
architectures
\end{flushleft}





\begin{flushleft}
COL719 Synthesis of Digital Systems
\end{flushleft}


\begin{flushleft}
4 Credits (3-0-2)
\end{flushleft}


\begin{flushleft}
Pre-requisites: COL215 OR Equivalent
\end{flushleft}


\begin{flushleft}
After a basic overview of the VLSI design flow, hardware modelling
\end{flushleft}


\begin{flushleft}
principles and hardware description using the VHDL language are
\end{flushleft}


\begin{flushleft}
covered. This is followed by a study of the major steps involved in
\end{flushleft}


\begin{flushleft}
behavioural synthesis: scheduling, allocation, and binding. This is
\end{flushleft}


\begin{flushleft}
followed by register-transfer level synthesis, which includes retiming
\end{flushleft}





186





\begin{flushleft}
\newpage
Computer Science
\end{flushleft}





\begin{flushleft}
and Finite State Machine encoding. Logic synthesis, consisting
\end{flushleft}


\begin{flushleft}
of combinational logic optimisation and technology mapping, is
\end{flushleft}


\begin{flushleft}
covered next. Popular chip architectures - standard cells and FPGA
\end{flushleft}


\begin{flushleft}
are introduced. The course concludes with a brief overview of layout
\end{flushleft}


\begin{flushleft}
synthesis topics: placement and routing.
\end{flushleft}





\begin{flushleft}
unrolling, loop tiling; Function inlining and tail recursion; Dependence
\end{flushleft}


\begin{flushleft}
analysis; Just-in-time compilation; Garbage collection. Laboratory
\end{flushleft}


\begin{flushleft}
component would involve getting familiar with internal representations
\end{flushleft}


\begin{flushleft}
of compilers; profiling and performance evaluation; and the design
\end{flushleft}


\begin{flushleft}
and implementation of novel compiler optimizations.
\end{flushleft}





\begin{flushleft}
COL722 Introduction to Compressed Sensing
\end{flushleft}


\begin{flushleft}
3 Credits (3-0-0)
\end{flushleft}


\begin{flushleft}
Pre-requisites: COL106 OR Equivalent
\end{flushleft}





\begin{flushleft}
COL730 Parallel Programming
\end{flushleft}


\begin{flushleft}
4 Credits (3-0-2)
\end{flushleft}


\begin{flushleft}
Pre-requisites: COL106, COL331
\end{flushleft}





\begin{flushleft}
Sparsity, L1 minimization, Sparse regression, deterministic and
\end{flushleft}


\begin{flushleft}
probabilistic approaches to compressed sensing, restricted
\end{flushleft}


\begin{flushleft}
isometry property and its application in sparse recovery,
\end{flushleft}


\begin{flushleft}
robustness in the presence of noise, algorithms for compressed
\end{flushleft}


\begin{flushleft}
sensing. Applications in magnetic resonance imaging (MRI),
\end{flushleft}


\begin{flushleft}
applications in analog-to-digital conversion, low-rank matrix
\end{flushleft}


\begin{flushleft}
recovery, applications in image reconstruction.
\end{flushleft}





\begin{flushleft}
COL724 Advanced Computer Networks
\end{flushleft}


\begin{flushleft}
4 Credits (3-0-2)
\end{flushleft}


\begin{flushleft}
Pre-requisites: COL334 OR Equivalent
\end{flushleft}


\begin{flushleft}
Review of the Internet architecture, layering; wired and wireless
\end{flushleft}


\begin{flushleft}
MAC; intra- and inter-domain Internet routing, BGP, MPLS, MANETs;
\end{flushleft}


\begin{flushleft}
error control and reliable delivery, ARQ, FEC, TCP; congestion and
\end{flushleft}


\begin{flushleft}
flow control; QoS, scheduling; mobility, mobile IP, TCP and MAC
\end{flushleft}


\begin{flushleft}
interactions, session persistence; multicast; Internet topology,
\end{flushleft}


\begin{flushleft}
economic models of ISPs/CDNs/content providers; future directions.
\end{flushleft}





\begin{flushleft}
COL726 Numerical Algorithms
\end{flushleft}


\begin{flushleft}
4 Credits (3-0-2)
\end{flushleft}


\begin{flushleft}
Pre-requisites: COL106 OR Equivalent
\end{flushleft}


\begin{flushleft}
Overlaps with: MTL704
\end{flushleft}





\begin{flushleft}
Parallel computer organization, Parallel performance analysis,
\end{flushleft}


\begin{flushleft}
Scalability, High level Parallel programming models and framework,
\end{flushleft}


\begin{flushleft}
Load distribution and scheduling, Throughput, Latency, Memory and
\end{flushleft}


\begin{flushleft}
Data Organizations, Inter-process communication and synchronization,
\end{flushleft}


\begin{flushleft}
Shared memory architecture, Memory consistency, Interconnection
\end{flushleft}


\begin{flushleft}
network and routing, Distributed memory architecture, Distributed
\end{flushleft}


\begin{flushleft}
shared memory, Parallel IO, Parallel graph algorithms, Parallel
\end{flushleft}


\begin{flushleft}
Algorithm techniques: Searching, Sorting, Prefix operations, Pointer
\end{flushleft}


\begin{flushleft}
Jumping, Divide-and-Conquer, Partitioning, Pipelining, Accelerated
\end{flushleft}


\begin{flushleft}
Cascading, Symmetry Breaking, Synchronization (Locked/Lock-free).
\end{flushleft}





\begin{flushleft}
COL732 Virtualization and Cloud Computing
\end{flushleft}


\begin{flushleft}
4 Credits (3-0-2)
\end{flushleft}


\begin{flushleft}
Pre-requisites: COL331
\end{flushleft}


\begin{flushleft}
Introduction to Virtualization and Cloud Computing; Binary Translation;
\end{flushleft}


\begin{flushleft}
Hardware Virtualization; Memory Resource Management in Virtual
\end{flushleft}


\begin{flushleft}
Machine Monitor; Application of Virtualization; Cloud-scale Data
\end{flushleft}


\begin{flushleft}
Management and Processing; I/O Virtualization.
\end{flushleft}





\begin{flushleft}
COL733 Cloud Computing Technology Fundamentals
\end{flushleft}


\begin{flushleft}
4 Credits (3-0-2)
\end{flushleft}


\begin{flushleft}
Pre-requisites: COL331
\end{flushleft}





\begin{flushleft}
Number representation, fundamentals of error analysis, conditioning,
\end{flushleft}


\begin{flushleft}
stability, polynomials and root finding, interpolation, singular value
\end{flushleft}


\begin{flushleft}
decomposition and its applications, QR factorization, condition number,
\end{flushleft}


\begin{flushleft}
least squares and regression, Gaussian elimination, eigenvalue
\end{flushleft}


\begin{flushleft}
computations and applications, iterative methods, linear programming,
\end{flushleft}


\begin{flushleft}
elements of convex optimization including steepest descent, conjugate
\end{flushleft}


\begin{flushleft}
gradient, Newton's method.
\end{flushleft}





\begin{flushleft}
COL728 Compiler Design
\end{flushleft}


\begin{flushleft}
4.5 Credits (3-0-3)
\end{flushleft}


\begin{flushleft}
Pre-requisites: COL 216, COL 226 OR Equivalent
\end{flushleft}


\begin{flushleft}
Compilers and translators; lexical and syntactic analysis, top-down
\end{flushleft}


\begin{flushleft}
and bottom up parsing techniques; internal form of source programs;
\end{flushleft}


\begin{flushleft}
semantic analysis, symbol tables, error detection and recovery, code
\end{flushleft}


\begin{flushleft}
generation and optimization. Type checking and static analysis. Static
\end{flushleft}


\begin{flushleft}
analysis formulated as fixpoint of simultaneous semantic equations.
\end{flushleft}


\begin{flushleft}
Data flow. Abstract interpretation. Correctness issues in code
\end{flushleft}


\begin{flushleft}
optimizations. Algorithms and implementation techniques for typechecking, code generation and optimization. Students will design and
\end{flushleft}


\begin{flushleft}
implement translators, static analysis, type-checking and optimization.
\end{flushleft}


\begin{flushleft}
This is a praxis-based course. Students will use a variety of software
\end{flushleft}


\begin{flushleft}
tools and techniques in implementing a complete compiler.
\end{flushleft}





\begin{flushleft}
COL729 Compiler Optimization
\end{flushleft}


\begin{flushleft}
4.5 Credits (3-0-3)
\end{flushleft}


\begin{flushleft}
Pre-requisites: COL 216, COL 226 OR Equivalent
\end{flushleft}


\begin{flushleft}
Overlaps with: COL728
\end{flushleft}


\begin{flushleft}
Program representation -- symbol table, abstract syntax tree; Control
\end{flushleft}


\begin{flushleft}
flow analysis; Data flow analysis; Static single assignment; Def-use
\end{flushleft}


\begin{flushleft}
and Use-def chains; Early optimizations -- constant folding, algebraic
\end{flushleft}


\begin{flushleft}
simplifications, value numbering, copy propagation, constant
\end{flushleft}


\begin{flushleft}
propagation; Redundancy Elimination -- dead code elimination, loop
\end{flushleft}


\begin{flushleft}
invariant code motion, common sub-expression elimination; Register
\end{flushleft}


\begin{flushleft}
Allocation; Scheduling -- branch delay slot scheduling, list scheduling,
\end{flushleft}


\begin{flushleft}
trace scheduling, software pipelining; Optimizing for memory hierarchy
\end{flushleft}


\begin{flushleft}
-- code placement, scalar replacement of arrays, register pipelining;
\end{flushleft}


\begin{flushleft}
Loop transformations -- loop fission, loop fusion, loop permutation, loop
\end{flushleft}





\begin{flushleft}
Overview of Cloud Computing, Virtualisation of CPU, Memory and I/O
\end{flushleft}


\begin{flushleft}
Devices; Storage Virtualisation and Software Defined Storage (SDS),
\end{flushleft}


\begin{flushleft}
Software Defined Networks (SDN) and Network Virtualisation, Data
\end{flushleft}


\begin{flushleft}
Centre Design and interconnection Networks, Cloud Architectures,
\end{flushleft}


\begin{flushleft}
Public Cloud Platforms (Google App Engine, AWS, Azure), Cloud
\end{flushleft}


\begin{flushleft}
Security and Trust Management, Open Source Clouds (Baadal, Open
\end{flushleft}


\begin{flushleft}
Stack, Cloud Stack), Cloud Programming and Software Environments
\end{flushleft}


\begin{flushleft}
(Hadoop, GFS, Map Reduce, NoSQL systems, Big Table, HBase, Libvirt,
\end{flushleft}


\begin{flushleft}
OpenVswitch), Amazon (Iaas), Azure(PaaS), GAE (PaaS)
\end{flushleft}





\begin{flushleft}
COL740 Software Engineering
\end{flushleft}


\begin{flushleft}
4 Credits (3-0-2)
\end{flushleft}


\begin{flushleft}
Pre-requisites: COL106, COL226
\end{flushleft}


\begin{flushleft}
Introduction to Software Engineering, Software Life Cycle models and
\end{flushleft}


\begin{flushleft}
Processes, Requirement Engineering, System Models, Architectural
\end{flushleft}


\begin{flushleft}
Design, Abstraction \& Modularity, Structured Programming, Objectoriented techniques, Design Patterns, Service Oriented Architecture,
\end{flushleft}


\begin{flushleft}
User Interface Design, Verification and Validation, Reliability, Software
\end{flushleft}


\begin{flushleft}
Evolution, Project Management \& Risk Analysis, Software Quality
\end{flushleft}


\begin{flushleft}
Management, Configuration Management,Software Metrics, Cost
\end{flushleft}


\begin{flushleft}
Analysis and Estimation, Manpower Management, Organization and
\end{flushleft}


\begin{flushleft}
Management of large Software Projects.
\end{flushleft}





\begin{flushleft}
COD745 Minor Project
\end{flushleft}


\begin{flushleft}
3 Credits (0-0-6)
\end{flushleft}


\begin{flushleft}
Pre-requisites: EC 75
\end{flushleft}


\begin{flushleft}
Research and development projects based on problems of practical and
\end{flushleft}


\begin{flushleft}
theoretical interest. Evaluation will be based on periodic presentations,
\end{flushleft}


\begin{flushleft}
student seminars, written reports, and evaluation of the developed
\end{flushleft}


\begin{flushleft}
system (if applicable).
\end{flushleft}





\begin{flushleft}
COP745 Digital System Design Laboratory
\end{flushleft}


\begin{flushleft}
3 Credits (0-0-6)
\end{flushleft}


\begin{flushleft}
Pre-requisites: COL215 OR Equivalent
\end{flushleft}


\begin{flushleft}
Being primarily a laboratory course, it would consist of a series
\end{flushleft}


\begin{flushleft}
of assignments that would increase in complexity in terms of
\end{flushleft}


\begin{flushleft}
designs to be carried out. Each assignment would involve learning
\end{flushleft}


\begin{flushleft}
to translate starting from natural language specifications to HDL
\end{flushleft}





187





\begin{flushleft}
\newpage
Computer Science
\end{flushleft}





\begin{flushleft}
design representation. The students would use modern synthesis
\end{flushleft}


\begin{flushleft}
techniques to realize these designs on FPGA boards before testing
\end{flushleft}


\begin{flushleft}
them for functionality as well as performance. Students would also
\end{flushleft}


\begin{flushleft}
be required to specify and implement a project (small system design)
\end{flushleft}


\begin{flushleft}
as part of the course.
\end{flushleft}





\begin{flushleft}
Geometric sampling: random sampling and $\epsilon$-nets, $\epsilon$-approximation
\end{flushleft}


\begin{flushleft}
and discrepancy, cuttings, coresetsGeometric optimization: linear
\end{flushleft}


\begin{flushleft}
programming, LP-type problems, parametric searching, approximation
\end{flushleft}


\begin{flushleft}
techniques. Implementation Issues: robust computing, perturbation
\end{flushleft}


\begin{flushleft}
techniques, floating-point filters, rounding techniques.
\end{flushleft}





\begin{flushleft}
COL750 Foundations of Automatic Verification
\end{flushleft}


\begin{flushleft}
4 Credits (3-0-2)
\end{flushleft}


\begin{flushleft}
Pre-requisites: COL226, COL352 OR Equivalent
\end{flushleft}





\begin{flushleft}
COL753 Complexity Theory
\end{flushleft}


\begin{flushleft}
3 Credits (3-0-0)
\end{flushleft}


\begin{flushleft}
Pre-requisites: COL352, COL705 OR Equivalent
\end{flushleft}





\begin{flushleft}
A selection from the following topics, and experiments with the
\end{flushleft}


\begin{flushleft}
mentioned tools: Review of first-order logic, syntax and semantics.
\end{flushleft}


\begin{flushleft}
Resolution theorem proving. Binary Decision Diagrams (BDDs) and
\end{flushleft}


\begin{flushleft}
their use in representing systems. (Programming exercises coding and
\end{flushleft}


\begin{flushleft}
using logic programming frameworks). Transition systems, automata
\end{flushleft}


\begin{flushleft}
and transducers. Buechi and other automata on infinite words; Linear
\end{flushleft}


\begin{flushleft}
Time Temporal Logic (LTL), and specifying properties of systems in
\end{flushleft}


\begin{flushleft}
LTL; the relationship between temporal logic and automata on infinite
\end{flushleft}


\begin{flushleft}
words, LTL Model checking (exercises using Spin or similar tools);
\end{flushleft}


\begin{flushleft}
Computational Tree Logic (CTL and CTL*); CTL model checking
\end{flushleft}


\begin{flushleft}
(exercises); Process calculi such as CSP and CCS. Notions of program
\end{flushleft}


\begin{flushleft}
equivalence -- traces, bisimulation and other notions. Hennessy-Milner
\end{flushleft}


\begin{flushleft}
Logic (HML) and Mu calculus (exercises using tools such as CWB
\end{flushleft}


\begin{flushleft}
-- Concurrency Work Bench). Symbolic model checking, exercises
\end{flushleft}


\begin{flushleft}
using tools such as SMV. Sat-based model checking and Davis-Putnam
\end{flushleft}


\begin{flushleft}
procedure; (exercises using tools such as nuSMV). Possible additional
\end{flushleft}


\begin{flushleft}
topics include: equational logic frameworks, real-time frameworks,
\end{flushleft}


\begin{flushleft}
reactive frameworks, pi-calculus (exercises using tools such as the
\end{flushleft}


\begin{flushleft}
Mobility Workbench), Tree automata and Weak Second-order Logic
\end{flushleft}


\begin{flushleft}
with k successors (WSkS), (exercises using Mona or similar tools).
\end{flushleft}





\begin{flushleft}
COL751 Algorithmic Graph Theory
\end{flushleft}


\begin{flushleft}
3 Credits (3-0-0)
\end{flushleft}


\begin{flushleft}
Pre-requisites: COL351 OR Equivalent
\end{flushleft}


\begin{flushleft}
Overlaps with: MTL468
\end{flushleft}


\begin{flushleft}
Algorithms for computing maximum s-t flows in graphs: augmenting
\end{flushleft}


\begin{flushleft}
path, blocking flow, push-relabel, capacity scaling etc. Max-flow
\end{flushleft}


\begin{flushleft}
min-cut theorem and its applications. Algorithms for computing
\end{flushleft}


\begin{flushleft}
min-cuts in graphs, structure of min-cuts. Min-cost flows and
\end{flushleft}


\begin{flushleft}
applications: cycle cancelling algorithms, successive shortest paths,
\end{flushleft}


\begin{flushleft}
strongly polynomial algorithms. Maximum matching in bipartite and
\end{flushleft}


\begin{flushleft}
general graphs: Hall's theorem, Tutte's theorem, Gallai-Edmonds
\end{flushleft}


\begin{flushleft}
decomposition. Weighted bipartite matching, Edmonds Algorithm for
\end{flushleft}


\begin{flushleft}
Weighted Non-Bipartite Matching,T-joins and applications. FactorCritical Graphs, Ear Decompositions, Graph orientations, Splitting
\end{flushleft}


\begin{flushleft}
Off, k-Connectivity Orientations and Augmentations, Arborescences
\end{flushleft}


\begin{flushleft}
and Branchings, Edmonds theorem for disjoint arborescences. Planar
\end{flushleft}


\begin{flushleft}
graphs, algorithms for checking planarity, planar-separator theorem
\end{flushleft}


\begin{flushleft}
and its applications. Intersection graphs, perfect graphs: polyhedral
\end{flushleft}


\begin{flushleft}
characterization, the strong perfect graph theorem, kinds of perfect
\end{flushleft}


\begin{flushleft}
graphs and algorithms on them. Treewidth, algorithm for computing
\end{flushleft}


\begin{flushleft}
tree width, algorithms on graphs with bounded tree width.
\end{flushleft}





\begin{flushleft}
COL752 Geometric Algorithms
\end{flushleft}


\begin{flushleft}
4 Credits (3-0-2)
\end{flushleft}


\begin{flushleft}
Pre-requisites: COL351 OR Equivalent
\end{flushleft}


\begin{flushleft}
Geometric Fundamentals: Models of computation, lower bound
\end{flushleft}


\begin{flushleft}
techniques, geometric primitives, geometric transforms Convex hulls:
\end{flushleft}


\begin{flushleft}
Planar convex hulls, higher dimensional convex hulls, randomized,
\end{flushleft}


\begin{flushleft}
output-sensitive, and dynamic algorithms, applications of convex hull,
\end{flushleft}


\begin{flushleft}
Intersection detection: segment intersection, line sweep, map overlay,
\end{flushleft}


\begin{flushleft}
halfspace intersection, polyhedra intersection, Geometric searching:
\end{flushleft}


\begin{flushleft}
segment, interval, and priority-search trees, point location, persistent
\end{flushleft}


\begin{flushleft}
data structure, fractional cascading, range searching, nearestneighbor searching Proximity problems: closest pair, Voronoi diagram,
\end{flushleft}


\begin{flushleft}
Delaunay triangulation and their subgraphs, spanners, well separated
\end{flushleft}


\begin{flushleft}
pair decomposition Arrangements: Arrangements of lines and
\end{flushleft}


\begin{flushleft}
hyperplanes, sweep-line and incremental algorithms, lower envelopes,
\end{flushleft}


\begin{flushleft}
levels, and zones, applications of arrangements Triangulations:
\end{flushleft}


\begin{flushleft}
monotone and simple polygon triangulations, point-set triangulations,
\end{flushleft}


\begin{flushleft}
optimization criteria, Steiner triangulation, Delaunay refinement
\end{flushleft}





\begin{flushleft}
Modeling computation (Finite state machines, Non-determinism, Turing
\end{flushleft}


\begin{flushleft}
machines, class P etc.), NP and NP-completeness, Diagonalization
\end{flushleft}


\begin{flushleft}
(Time hierarchy and Ladner's theorem), Space complexity (PSPACE,
\end{flushleft}


\begin{flushleft}
NL, Savitch's theorem, Immerman-Szelepcs\'{e}nyi theorem etc.),
\end{flushleft}


\begin{flushleft}
Polynomial hierarchy, Boolean circuits (P/poly), Randomized classes
\end{flushleft}


\begin{flushleft}
(RP, BPP, ZPP, Adleman's Theorem, G\'{a}cs-Sipser-Lautemann Theorem),
\end{flushleft}


\begin{flushleft}
Interactive proofs (Arthur-Merlin, IP=PSPACE), Cryptography (one-way
\end{flushleft}


\begin{flushleft}
functions, pseudorandom generators, zero knowledge), PCP theorem
\end{flushleft}


\begin{flushleft}
and hardness of approximation, Circuit lower bounds (Hastad's
\end{flushleft}


\begin{flushleft}
switching lemma), Other topics (\#P, Toda's theorem, Average-case
\end{flushleft}


\begin{flushleft}
complexity, derandomization, pseudorandom construction)
\end{flushleft}





\begin{flushleft}
COL754 Approximation Algorithms
\end{flushleft}


\begin{flushleft}
3 Credits (3-0-0)
\end{flushleft}


\begin{flushleft}
Pre-requisites: COL351 OR Equivalent
\end{flushleft}


\begin{flushleft}
NP-hardness and approximation algorithms. Different kinds of
\end{flushleft}


\begin{flushleft}
approximability. Greedy algorithm and local search with applications
\end{flushleft}


\begin{flushleft}
in facility location, TSP and scheduling. Dynamic programming
\end{flushleft}


\begin{flushleft}
with applications in knapsack, Euclidean TSP, bin packing. Linear
\end{flushleft}


\begin{flushleft}
programming, duality and rounding. Applications in facility location,
\end{flushleft}


\begin{flushleft}
Steiner tree and bin packing. Randomized rounding with applications.
\end{flushleft}


\begin{flushleft}
Primal-dual algorithms and applications in facility location and network
\end{flushleft}


\begin{flushleft}
design. Cuts and metrics with applications to multi-commodity flow.
\end{flushleft}


\begin{flushleft}
Semi-definite programming and applications: max-cut, graph coloring.
\end{flushleft}


\begin{flushleft}
Hardness of approximation.
\end{flushleft}





\begin{flushleft}
COL756 Mathematical Programming
\end{flushleft}


\begin{flushleft}
3 Credits (3-0-0)
\end{flushleft}


\begin{flushleft}
Pre-requisites: COL351 OR Equivalent
\end{flushleft}


\begin{flushleft}
Overlaps with: MTL103, MTL704
\end{flushleft}


\begin{flushleft}
Linear programming: introduction, geometry, duality, sensitivity
\end{flushleft}


\begin{flushleft}
analysis. Simplex method, Large scale optimization, network simplex.
\end{flushleft}


\begin{flushleft}
Ellipsoid method, problems with exponentially many constraints,
\end{flushleft}


\begin{flushleft}
equivalence of optimization and separation. Convex sets and functions
\end{flushleft}


\begin{flushleft}
-- cones, hyperplanes, norm balls, generalized inequalities and
\end{flushleft}


\begin{flushleft}
convexity, perspective and conjugate functions. Convex optimization
\end{flushleft}


\begin{flushleft}
problems -- quasi-convex, linear, quadratic, geometric, vector, semidefinite. Duality -- Lagrange, geometric interpretation, optimality
\end{flushleft}


\begin{flushleft}
conditions, sensitivity analysis. Applications to approximation, fitting,
\end{flushleft}


\begin{flushleft}
statistical estimation, classification. Unconstrained minimization,
\end{flushleft}


\begin{flushleft}
equality constrained minimization and interior point methods.
\end{flushleft}


\begin{flushleft}
Integer Programming: formulations, complexity, duality. Lattices,
\end{flushleft}


\begin{flushleft}
geometry, cutting plane and branch and bound methods. Mixed
\end{flushleft}


\begin{flushleft}
integer optimization.
\end{flushleft}





\begin{flushleft}
COL757 Model Centric Algorithm Design
\end{flushleft}


\begin{flushleft}
4 Credits (3-0-2)
\end{flushleft}


\begin{flushleft}
Pre-requisites: COL351 OR Equivalent
\end{flushleft}


\begin{flushleft}
The RAM model and its limitations, Introduction to alternate
\end{flushleft}


\begin{flushleft}
algorithmic models Parallel models like PRAM and Interconnection
\end{flushleft}


\begin{flushleft}
networks; Basic problems like Sorting, Merging, Routing, Parallel Prefix
\end{flushleft}


\begin{flushleft}
and applications, graph algorithms like BFS, Matching
\end{flushleft}


\begin{flushleft}
Memory hierarchy models; Caching, Sorting, Merging, FFT, Permutation,
\end{flushleft}


\begin{flushleft}
Lower bounds Data Structures - searching, Priority queues
\end{flushleft}


\begin{flushleft}
Streaming Data models: Distinct items, frequent items, frequency
\end{flushleft}


\begin{flushleft}
moments, estimating norms, clustering
\end{flushleft}


\begin{flushleft}
On line algorithms: competitive ratio, list accessing, paging, k-server,
\end{flushleft}


\begin{flushleft}
load-balancing, lower-bounds.
\end{flushleft}





188





\begin{flushleft}
\newpage
Computer Science
\end{flushleft}





\begin{flushleft}
COL758 Advanced Algorithms
\end{flushleft}


\begin{flushleft}
4 Credits (3-0-2)
\end{flushleft}


\begin{flushleft}
Pre-requisites: COL351 OR Equivalent
\end{flushleft}


\begin{flushleft}
Advanced data structures: self-adjustment, persistence and
\end{flushleft}


\begin{flushleft}
multidimensional trees. Randomized algorithms: Use of probabilistic
\end{flushleft}


\begin{flushleft}
inequalities in analysis, Geometric algorithms: Point location, Convex
\end{flushleft}


\begin{flushleft}
hulls and Voronoi diagrams, Arrangements applications using
\end{flushleft}


\begin{flushleft}
examples. Graph algorithms: Matching and Flows. Approximation
\end{flushleft}


\begin{flushleft}
algorithms: Use of Linear programming and primal dual, Local search
\end{flushleft}


\begin{flushleft}
heuristics. Parallel algorithms: Basic techniques for sorting, searching,
\end{flushleft}


\begin{flushleft}
merging, list ranking in PRAMs and Interconnection networks.
\end{flushleft}





\begin{flushleft}
COL759 Cryptography \& Computer Security
\end{flushleft}


\begin{flushleft}
3 Credits (3-0-0)
\end{flushleft}


\begin{flushleft}
Pre-requisites: COL351, MTL106 OR Equivalent
\end{flushleft}


\begin{flushleft}
Overlaps with: MTL730
\end{flushleft}


\begin{flushleft}
Part 1: Foundations: Perfect secrecy and its limitations, computational
\end{flushleft}


\begin{flushleft}
security, pseudorandom generators and one time encryption,
\end{flushleft}


\begin{flushleft}
pseudorandom functions, one way permutations, message
\end{flushleft}


\begin{flushleft}
authentication and cryptographic hash functions.
\end{flushleft}


\begin{flushleft}
Part 2: Basic Constructions and proofs: Some number theory,
\end{flushleft}


\begin{flushleft}
symmetric key encryption, public key encryption, CPA and CCA security,
\end{flushleft}


\begin{flushleft}
digital signatures, oblivious transfer, secure multiparty computation.
\end{flushleft}


\begin{flushleft}
Part 3: Advanced Topics: Zero knowledge proofs, identity based
\end{flushleft}


\begin{flushleft}
encryption, broadcast encryption, homomorphic encryption, lattice
\end{flushleft}


\begin{flushleft}
based cryptography.
\end{flushleft}





\begin{flushleft}
COL760 Advanced Data Management
\end{flushleft}


\begin{flushleft}
4 Credits (3-0-2)
\end{flushleft}


\begin{flushleft}
Pre-requisites: COL362 OR Equivalent
\end{flushleft}


\begin{flushleft}
Storage and file structures, advanced query processing and
\end{flushleft}


\begin{flushleft}
optimization for single server databases, distributed data management
\end{flushleft}


\begin{flushleft}
(including distributed data storage, query processing and transaction
\end{flushleft}


\begin{flushleft}
management), web-data management (including managing the webgraph and implementation of web-search), big data systems.
\end{flushleft}





\begin{flushleft}
COL762 Database Implementation
\end{flushleft}


\begin{flushleft}
4 Credits (3-0-2)
\end{flushleft}


\begin{flushleft}
Pre-requisites: COL362 OR Equivalent
\end{flushleft}


\begin{flushleft}
Review of Relational Model, Algebra and SQL, File structures,
\end{flushleft}


\begin{flushleft}
Constraints and Triggers, System Aspects of SQL, Data Storage,
\end{flushleft}


\begin{flushleft}
Representing Data Elements, Index, Multi dimensional and Bit-map
\end{flushleft}


\begin{flushleft}
Indexes, Hashing, Query Execution, Query Compiler.
\end{flushleft}





\begin{flushleft}
COL765 Intro. To Logic and Functional Programming
\end{flushleft}


\begin{flushleft}
4 Credits (3-0-2)
\end{flushleft}


\begin{flushleft}
Pre-requisites: COL106 OR Equivalent
\end{flushleft}


\begin{flushleft}
Introduction to declarative programming paradigms. The functional
\end{flushleft}


\begin{flushleft}
style of programming, paradigms of developments of functional
\end{flushleft}


\begin{flushleft}
programs, use of higher order functionals and pattern-matching.
\end{flushleft}


\begin{flushleft}
Introduction to lambda calculus. Interpreters for functional languages
\end{flushleft}


\begin{flushleft}
and abstract machines for lazy and eager lambda calculi, Types, typechecking and their relationship to logic. Logic as a system for declarative
\end{flushleft}


\begin{flushleft}
programming. The use of pattern-matching and programming of higher
\end{flushleft}


\begin{flushleft}
order functions within a logic programming framework. Introduction
\end{flushleft}


\begin{flushleft}
to symbolic processing. The use of resolution and theorem-proving
\end{flushleft}


\begin{flushleft}
techniques in logic programming. The relationship between logic
\end{flushleft}


\begin{flushleft}
programming and functional programming.
\end{flushleft}





\begin{flushleft}
COL770 Advanced Artificial Intelligence
\end{flushleft}


\begin{flushleft}
4 Credits (3-0-2)
\end{flushleft}


\begin{flushleft}
Pre-requisites: COL106 OR Equivalent
\end{flushleft}


\begin{flushleft}
Overlap with: COL333, COL770, ELL789
\end{flushleft}


\begin{flushleft}
Philosophy of artificial intelligence, fundamental and advanced search
\end{flushleft}


\begin{flushleft}
techniques (A*, local search, suboptimal heuristic search, search
\end{flushleft}


\begin{flushleft}
in AND/OR graphs), constraint optimization, temporal reasoning,
\end{flushleft}


\begin{flushleft}
knowledge representation and reasoning through propositional
\end{flushleft}


\begin{flushleft}
and first order logic, modern game playing (UCT), planning under
\end{flushleft}


\begin{flushleft}
uncertainty (Topological value iteration, LAO*, LRTDP), reinforcement
\end{flushleft}


\begin{flushleft}
learning, introduction to robotics, introduction to probabilistic graphical
\end{flushleft}


\begin{flushleft}
models (Bayesian networks, Hidden Markov models, Conditional
\end{flushleft}


\begin{flushleft}
random fields), machine learning, introduction to information systems
\end{flushleft}


\begin{flushleft}
(information retrieval, information extraction).
\end{flushleft}





\begin{flushleft}
COL772 Natural Language Processing
\end{flushleft}


\begin{flushleft}
4 Credits (3-0-2)
\end{flushleft}


\begin{flushleft}
Pre-requisites: COL106 OR Equivalent
\end{flushleft}


\begin{flushleft}
Overlaps with: MTL785
\end{flushleft}


\begin{flushleft}
NLP concepts: Tokenization, lemmatization, part of speech tagging,
\end{flushleft}


\begin{flushleft}
noun phrase chunking, named entity recognition, co-reference
\end{flushleft}


\begin{flushleft}
resolution, parsing, information extraction, sentiment analysis,
\end{flushleft}


\begin{flushleft}
question answering, text classification, document clustering, document
\end{flushleft}


\begin{flushleft}
summarization, discourse, machine translation.
\end{flushleft}


\begin{flushleft}
Machine learning concepts: Naïve Bayes, Hidden Markov Models, EM,
\end{flushleft}


\begin{flushleft}
Conditional Random Fields, MaxEnt Classifiers, Probabilistic Context
\end{flushleft}


\begin{flushleft}
Free Grammars.
\end{flushleft}





\begin{flushleft}
COL774 Machine Learning
\end{flushleft}


\begin{flushleft}
4 Credits (3-0-2)
\end{flushleft}


\begin{flushleft}
Pre-requisites: MTL106 OR Equivalent
\end{flushleft}


\begin{flushleft}
Overlaps with: COL341 ELL784, ELL888
\end{flushleft}


\begin{flushleft}
Supervised learning algorithms: Linear and Logistic Regression,
\end{flushleft}


\begin{flushleft}
Gradient Descent, Support Vector Machines, Kernels, Artificial Neural
\end{flushleft}


\begin{flushleft}
Networks, Decision Trees, ML and MAP Estimates, K-Nearest Neighbor,
\end{flushleft}


\begin{flushleft}
Naive Bayes, Introduction to Bayesian Networks. Unsupervised
\end{flushleft}


\begin{flushleft}
learning algorithms: K-Means clustering, Gaussian Mixture Models,
\end{flushleft}


\begin{flushleft}
Learning with Partially Observable Data (EM). Dimensionality Reduction
\end{flushleft}


\begin{flushleft}
and Principal Component Analysis. Bias Variance Trade-off. Model
\end{flushleft}


\begin{flushleft}
Selection and Feature Selection. Regularization. Learning Theory.
\end{flushleft}


\begin{flushleft}
Introduction to Markov Decision Processes. Application to Information
\end{flushleft}


\begin{flushleft}
Retrieval, NLP, Biology and Computer Vision. Advanced Topics.
\end{flushleft}





\begin{flushleft}
COL776 Learning Probabilistic Graphical Models
\end{flushleft}


\begin{flushleft}
4 Credits (3-0-2)
\end{flushleft}


\begin{flushleft}
Pre-requisites: MTL106 OR Equivalent
\end{flushleft}


\begin{flushleft}
Basics: Introduction. Undirected and Directed Graphical Models.
\end{flushleft}


\begin{flushleft}
Bayesian Networks. Markov Networks. Exponential Family Models.
\end{flushleft}


\begin{flushleft}
Factor Graph Representation. Hidden Markov Models. Conditional
\end{flushleft}


\begin{flushleft}
Random Fields. Triangulation and Chordal Graphs. Other Special
\end{flushleft}


\begin{flushleft}
Cases: Chains, Trees. Inference: Variable Elimination (Sum Product and
\end{flushleft}


\begin{flushleft}
Max-Product). Junction Tree Algorithm. Forward Backward Algorithm
\end{flushleft}


\begin{flushleft}
(for HMMs). Loopy Belief Propagation. Markov Chain Monte Carlo.
\end{flushleft}


\begin{flushleft}
Metropolis Hastings. Importance Sampling. Gibbs Sampling. Variational
\end{flushleft}


\begin{flushleft}
Inference. Learning: Discriminative Vs. Generative Learning. Parameter
\end{flushleft}


\begin{flushleft}
Estimation in Bayesian and Markov Networks. Structure Learning. EM:
\end{flushleft}


\begin{flushleft}
Handling Missing Data. Applications in Vision, Web/IR, NLP and Biology.
\end{flushleft}


\begin{flushleft}
Advanced Topics: Statistical Relational Learning, Markov Logic Networks.
\end{flushleft}





\begin{flushleft}
COL768 Wireless Networks
\end{flushleft}


\begin{flushleft}
4 Credits (3-0-2)
\end{flushleft}


\begin{flushleft}
Pre-requisites: COL334 OR Equivalent
\end{flushleft}





\begin{flushleft}
COL780 Computer Vision
\end{flushleft}


\begin{flushleft}
4 Credits (3-0-2)
\end{flushleft}


\begin{flushleft}
Pre-requisites: EC 80
\end{flushleft}


\begin{flushleft}
Overlaps with: ELL793
\end{flushleft}





\begin{flushleft}
Radio signal propagation, advanced modulation and coding, medium
\end{flushleft}


\begin{flushleft}
access techniques, self-configurable networks, mesh networks,
\end{flushleft}


\begin{flushleft}
cognitive radio and dynamic spectrum access networks, TCP over
\end{flushleft}


\begin{flushleft}
wireless, wireless security, emerging applications.
\end{flushleft}





\begin{flushleft}
Camera models. Calibration, multi-views projective geometry and
\end{flushleft}


\begin{flushleft}
invariants. Feature detection, correspondence and tracking. 3D structure/
\end{flushleft}


\begin{flushleft}
motion estimation. Application of machine learning in object detection
\end{flushleft}


\begin{flushleft}
and recognition, category discovery, scene and activity interpretation.
\end{flushleft}





189





\begin{flushleft}
\newpage
Computer Science
\end{flushleft}





\begin{flushleft}
COL781 Computer Graphics
\end{flushleft}


\begin{flushleft}
4.5 Credits (3-0-3)
\end{flushleft}


\begin{flushleft}
Pre-requisites: COL106 OR Equivalent
\end{flushleft}


\begin{flushleft}
Overlaps with: ELL792
\end{flushleft}





\begin{flushleft}
COL818 Principles of Multiprocessor Systems
\end{flushleft}


\begin{flushleft}
4 Credits (3-0-2)
\end{flushleft}


\begin{flushleft}
Pre-requisites: COL216, COL351, COL331 OR Equivalent
\end{flushleft}





\begin{flushleft}
Graphics pipeline; Graphics hardware: Display devices, Input devices;
\end{flushleft}


\begin{flushleft}
Raster Graphics: line and circle drawing algorithms; Windowing and
\end{flushleft}


\begin{flushleft}
2D/3D clipping: Cohen and Sutherland line clipping, Cyrus Beck
\end{flushleft}


\begin{flushleft}
clipping method; 2D and 3D Geometrical Transformations: scaling,
\end{flushleft}


\begin{flushleft}
translation, rotation, reflection; Viewing Transformations: parallel and
\end{flushleft}


\begin{flushleft}
perspective projection; Curves and Surfaces: cubic splines, Bezier
\end{flushleft}


\begin{flushleft}
curves, B-splines, Parametric surfaces, Surface of revolution, Sweep
\end{flushleft}


\begin{flushleft}
surfaces, Fractal curves and surfaces; Hidden line/surface removal
\end{flushleft}


\begin{flushleft}
methods; illuminations model; shading: Gouraud, Phong; Introduction
\end{flushleft}


\begin{flushleft}
to Ray-tracing; Animation; Programming practices with standard
\end{flushleft}


\begin{flushleft}
graphics libraries like openGL.
\end{flushleft}





\begin{flushleft}
COL783 Digital Image Analysis
\end{flushleft}


\begin{flushleft}
4.5 Credits (3-0-3)
\end{flushleft}


\begin{flushleft}
Pre-requisites: COL106, ELL205 OR Equivalent
\end{flushleft}


\begin{flushleft}
Overlap with: ELL715
\end{flushleft}





\begin{flushleft}
Mutual Exclusion, Coherence and Consistency, Register Constructions ,
\end{flushleft}


\begin{flushleft}
Power of Synchronization Operations , Locks and Monitors, Concurrent
\end{flushleft}


\begin{flushleft}
queues, Futures and Work-Stealing, Barriers, Basics of Transactional
\end{flushleft}


\begin{flushleft}
Memory (TM), Regular Hardware TMs, Unbounded HadwareTMs,
\end{flushleft}


\begin{flushleft}
Software TMs.
\end{flushleft}





\begin{flushleft}
COL819 Advanced Distributed Systems
\end{flushleft}


\begin{flushleft}
4 Credits (3-0-2)
\end{flushleft}


\begin{flushleft}
Pre-requisites: COL331, COL334, COL380 OR Equivalent
\end{flushleft}


\begin{flushleft}
Epidemic/Gossip based algorithms, Peer to peer networks, Distributed
\end{flushleft}


\begin{flushleft}
hash tables, Synchronization, Mutual exclusion, Leader election,
\end{flushleft}


\begin{flushleft}
Distributed fault tolerance, Large scale storage systems, Distributed
\end{flushleft}


\begin{flushleft}
file systems, Design of social networking systems.
\end{flushleft}





\begin{flushleft}
COL821 Reconfigurable Computing
\end{flushleft}


\begin{flushleft}
3 Credits (3-0-0)
\end{flushleft}


\begin{flushleft}
Pre-requisites: COL719
\end{flushleft}





\begin{flushleft}
Digital Image Fundamentals; Image Enhancement in Spatial Domain:
\end{flushleft}


\begin{flushleft}
Gray Level Transformation, Histogram Processing, Spatial Filters;
\end{flushleft}


\begin{flushleft}
Image Transforms: Fourier Transform and their properties, Fast Fourier
\end{flushleft}


\begin{flushleft}
Transform, Other Transforms; Image Enhancement in Frequency
\end{flushleft}


\begin{flushleft}
Domain; Color Image Processing; Image Warping and Restoration;
\end{flushleft}


\begin{flushleft}
Image Compression; Image Segmentation: edge detection, Hough
\end{flushleft}


\begin{flushleft}
transform, region based segmentation; Morphological operators;
\end{flushleft}


\begin{flushleft}
Representation and Description; Features based matching and Bayes
\end{flushleft}


\begin{flushleft}
classification; Introduction to some computer vision techniques:
\end{flushleft}


\begin{flushleft}
Imaging geometry, shape from shading, optical flow; Laboratory
\end{flushleft}


\begin{flushleft}
exercises will emphasize development and evaluation of image
\end{flushleft}


\begin{flushleft}
processing methods.
\end{flushleft}





\begin{flushleft}
COL786 Advanced Functional Brain Imaging
\end{flushleft}


\begin{flushleft}
4 Credits (3-0-2)
\end{flushleft}


\begin{flushleft}
Introduction to human Neuro-anatomy, Hodgkin Huxley model,
\end{flushleft}


\begin{flushleft}
overview of brain imaging methods, introduction to magnetic
\end{flushleft}


\begin{flushleft}
resonance imaging, detailed fMRI, fMRI data analysis methods,
\end{flushleft}


\begin{flushleft}
general linear model, network analysis, machine learning based
\end{flushleft}


\begin{flushleft}
methods of analysis.
\end{flushleft}





\begin{flushleft}
COL788 Advanced Topics in Embedded Computing
\end{flushleft}


\begin{flushleft}
3 Credits (3-0-0)
\end{flushleft}


\begin{flushleft}
Pre-requisites: COL216, COL331 OR Equivalent
\end{flushleft}


\begin{flushleft}
Overlaps with: ELL782
\end{flushleft}





\begin{flushleft}
FPGA architectures, CAD for FPGAs: overview, LUT mapping, timing
\end{flushleft}


\begin{flushleft}
analysis, placement and routing, Reconfigurable devices - from finegrained to coarse-grained devices, Reconfiguration modes and multicontext devices, Dynamic reconfiguration, Compilation from high level
\end{flushleft}


\begin{flushleft}
languages, System level design for reconfigurable systems: heuristic
\end{flushleft}


\begin{flushleft}
temporal partitioning and ILP-based temporal partitioning, Behavioral
\end{flushleft}


\begin{flushleft}
synthesis, Reconfigurable example systems' tool chains.
\end{flushleft}





\begin{flushleft}
COL829 Advanced Computer Graphics
\end{flushleft}


\begin{flushleft}
4 Credits (3-0-2)
\end{flushleft}


\begin{flushleft}
Pre-requisites: COL781
\end{flushleft}


\begin{flushleft}
Rendering: Ray tracing, Radiosity methods, Global illumination models,
\end{flushleft}


\begin{flushleft}
Shadow generation, Mapping, Anti-aliasing, Volume rendering,
\end{flushleft}


\begin{flushleft}
Geometrical Modeling: Parametric surfaces, Implicit surfaces, Meshes,
\end{flushleft}


\begin{flushleft}
Animation: spline driven, quarternions, articulated structures (forward
\end{flushleft}


\begin{flushleft}
and inverse kinematics), deformation- purely geometric, physicallybased, Other advanced topics selected from research papers.
\end{flushleft}





\begin{flushleft}
COL830 Distributed Computing
\end{flushleft}


\begin{flushleft}
3 Credits (3-0-0)
\end{flushleft}


\begin{flushleft}
Pre-requisites: COL226 OR Equivalent
\end{flushleft}


\begin{flushleft}
Models of Distributed Computing; Basic Issues: Causality, Exclusion,
\end{flushleft}


\begin{flushleft}
Fairness, Independence, Consistency; Specification of Distributed
\end{flushleft}


\begin{flushleft}
Systems: Transition systems, petri nets, process algebra properties:
\end{flushleft}


\begin{flushleft}
Safety, Liveness, stability.
\end{flushleft}





\begin{flushleft}
Embedded Platforms , Embedded processor architectures, System
\end{flushleft}


\begin{flushleft}
initialization, Embedded operating systems (linux) , DSP and graphics
\end{flushleft}


\begin{flushleft}
acceleration, Interfaces, Device Drivers, Network, Security, Debug
\end{flushleft}


\begin{flushleft}
support, Performance tuning.
\end{flushleft}


\begin{flushleft}
The course would involve substantial programming assignments on
\end{flushleft}


\begin{flushleft}
embedded platforms.
\end{flushleft}





\begin{flushleft}
COS799 Independent Study
\end{flushleft}


\begin{flushleft}
3 Credits (0-3-0)
\end{flushleft}


\begin{flushleft}
The student will be tasked with certain reading assignments and
\end{flushleft}


\begin{flushleft}
related problem solving in a appropriate area of research in Computer
\end{flushleft}


\begin{flushleft}
Science under the overall guidance of a CSE Faculty member. The work
\end{flushleft}


\begin{flushleft}
will be evaluated through term paper.
\end{flushleft}





\begin{flushleft}
COL812 System Level Design and Modelling
\end{flushleft}


\begin{flushleft}
3 Credits (3-0-0)
\end{flushleft}


\begin{flushleft}
Pre-requisites: COL719
\end{flushleft}


\begin{flushleft}
E m b e d d e d s y s t e m s a n d s y s t e m - l e ve l d e s i g n , m o d e l s o f
\end{flushleft}


\begin{flushleft}
computation, specification languages, hardware/software codesign, system partitioning, application specific processors and
\end{flushleft}


\begin{flushleft}
memory, low power design.
\end{flushleft}





\begin{flushleft}
COL831 Semantics of Programming Languages
\end{flushleft}


\begin{flushleft}
3 Credits (3-0-0)
\end{flushleft}


\begin{flushleft}
Pre-requisites: COL226, COL352
\end{flushleft}


\begin{flushleft}
Study of operational, axiomatic and denotational semantics of
\end{flushleft}


\begin{flushleft}
procedural languages; semantics issues in the design of functional and
\end{flushleft}


\begin{flushleft}
logic programming languages, study of abstract data types.
\end{flushleft}





\begin{flushleft}
COL832 Proofs and Types
\end{flushleft}


\begin{flushleft}
3 Credits (3-0-0)
\end{flushleft}


\begin{flushleft}
Pre-requisites: COL226, COL352
\end{flushleft}


\begin{flushleft}
Syntax and semantic foundations: Ranked algebras, homomorphisms,
\end{flushleft}


\begin{flushleft}
initial algebras, congruences. First-order logic review: Soundness,
\end{flushleft}


\begin{flushleft}
completeness, compactness. Herbrand models and Herbrand's
\end{flushleft}


\begin{flushleft}
theorem, Horn-clauses and resolution. Natural deduction and the
\end{flushleft}


\begin{flushleft}
Sequent calculus. Normalization and cut elimination. Lambda-calculus
\end{flushleft}


\begin{flushleft}
and Combinatory Logic: syntax and operational semantics (beta-eta
\end{flushleft}


\begin{flushleft}
equivalence), confluence and Church-Rosser property. Introduction
\end{flushleft}


\begin{flushleft}
to Type theory: The simply-typed lambda-calculus, Intuitionistic type
\end{flushleft}


\begin{flushleft}
theory. Curry-Howard correspondence. Polymorphism, algorithms
\end{flushleft}


\begin{flushleft}
for polymorphic type inference, Girard and Reynolds' System F.
\end{flushleft}


\begin{flushleft}
Applications: type-systems for programming languages; modules and
\end{flushleft}


\begin{flushleft}
functors; theorem proving, executable specifications.
\end{flushleft}





190





\begin{flushleft}
\newpage
Computer Science
\end{flushleft}





\begin{flushleft}
COL851 Special Topics in Operating Systems
\end{flushleft}


\begin{flushleft}
3 Credits (3-0-0)
\end{flushleft}


\begin{flushleft}
Pre-requisites: COL331 Or Equivalent
\end{flushleft}





\begin{flushleft}
Special topic that focuses on special topics and research problems of
\end{flushleft}


\begin{flushleft}
importance in this area.
\end{flushleft}





\begin{flushleft}
To provide insight into current research problems in the area of
\end{flushleft}


\begin{flushleft}
operating systems. Topics may include, but are not limited to,
\end{flushleft}


\begin{flushleft}
OS design, web servers, Networking stack, Virtualization, Cloud
\end{flushleft}


\begin{flushleft}
Computing, Distributed Computing, Parallel Computing, Heterogeneous
\end{flushleft}


\begin{flushleft}
Computing, etc.
\end{flushleft}





\begin{flushleft}
COL852 Special Topics in COMPILER DESIGN
\end{flushleft}


\begin{flushleft}
3 Credits (3-0-0)
\end{flushleft}


\begin{flushleft}
Pre-requisites: COL728/COL729
\end{flushleft}


\begin{flushleft}
Special topic that focuses on state of the art and research problems
\end{flushleft}


\begin{flushleft}
of importance in this area.
\end{flushleft}





\begin{flushleft}
COL859 Advanced Computer Graphics
\end{flushleft}


\begin{flushleft}
4 Credits (3-0-2)
\end{flushleft}


\begin{flushleft}
Rendering: Ray tracing, Radiosity methods, Global illumination models,
\end{flushleft}


\begin{flushleft}
Shadow generation, Mapping, Anti-aliasing, Volume rendering,
\end{flushleft}


\begin{flushleft}
Geometrical Modeling: Parametric surfaces, Implicit surfaces, Meshes,
\end{flushleft}


\begin{flushleft}
Animation: spline driven, quarternions, articulated structures (forward
\end{flushleft}


\begin{flushleft}
and inverse kinematics), deformation --- purely geometric, physically
\end{flushleft}


\begin{flushleft}
based, Other advanced topics selected from research papers.
\end{flushleft}





\begin{flushleft}
COL860 Special Topics in Parallel Computation
\end{flushleft}


\begin{flushleft}
3 Credits (3-0-0)
\end{flushleft}


\begin{flushleft}
The course will focus on research issues in areas like parallel
\end{flushleft}


\begin{flushleft}
computation models, parallel algorithms, Parallel Computer
\end{flushleft}


\begin{flushleft}
architectures and interconnection networks, Shared memory parallel
\end{flushleft}


\begin{flushleft}
architectures and programming with OpenMP and Ptheards, Distributed
\end{flushleft}


\begin{flushleft}
memory message-passing parallel architectures and programming,
\end{flushleft}


\begin{flushleft}
portable parallel message-passing programming using MPI. This
\end{flushleft}


\begin{flushleft}
will also include design and implementation of parallel numerical
\end{flushleft}


\begin{flushleft}
and non-numerical algorithms for scientific and engineering, and
\end{flushleft}


\begin{flushleft}
commercial applications. Performance evaluation and benchmarking
\end{flushleft}


\begin{flushleft}
high-performance computers.
\end{flushleft}





\begin{flushleft}
COL861 Special Topics in Hardware Systems
\end{flushleft}


\begin{flushleft}
3 Credits (3-0-0)
\end{flushleft}


\begin{flushleft}
Under this topic one of the following areas will be covered: Fault
\end{flushleft}


\begin{flushleft}
Detection and Diagnosability. Special Architectures. Design Automation
\end{flushleft}


\begin{flushleft}
Issues. Computer Arithmetic, VLSI.
\end{flushleft}





\begin{flushleft}
COL862 Special Topics in Software Systems
\end{flushleft}


\begin{flushleft}
3 Credits (3-0-0)
\end{flushleft}


\begin{flushleft}
Special topic that focuses on state of the art and research problems
\end{flushleft}


\begin{flushleft}
of importance in this area.
\end{flushleft}





\begin{flushleft}
COL863 Special Topics in Theoretical Computer Science
\end{flushleft}


\begin{flushleft}
3 Credits (3-0-0)
\end{flushleft}


\begin{flushleft}
Pre-requisites: COL351
\end{flushleft}


\begin{flushleft}
Under this topic one of the following areas will be covered: Design
\end{flushleft}


\begin{flushleft}
and Analysis of Sequential and Parallel Algorithms. Complexity
\end{flushleft}


\begin{flushleft}
issues, Trends in Computer Science Logic, Quantum Computing and
\end{flushleft}


\begin{flushleft}
Bioinformatics, Theory of computability. Formal Languages. Semantics
\end{flushleft}


\begin{flushleft}
and Verification issues.
\end{flushleft}





\begin{flushleft}
COL864 Special Topics in Artificial Intelligence
\end{flushleft}


\begin{flushleft}
3 Credits (3-0-0)
\end{flushleft}


\begin{flushleft}
Pre-requisites: COL333 / COL671 / Equivalent
\end{flushleft}


\begin{flushleft}
Potential topics or themes which may be covered (one topic per
\end{flushleft}


\begin{flushleft}
offering) include: information extraction, industrial applications of
\end{flushleft}


\begin{flushleft}
AI, advanced logic-based AI, Markov Decision Processes, statistical
\end{flushleft}


\begin{flushleft}
relational learning, etc.
\end{flushleft}





\begin{flushleft}
COL865 Special Topics in Computer Applications
\end{flushleft}


\begin{flushleft}
3 Credits (3-0-0)
\end{flushleft}


\begin{flushleft}
Pre-requisites: Permission of the Instructor
\end{flushleft}





\begin{flushleft}
COL866 Special Topics in Algorithms
\end{flushleft}


\begin{flushleft}
3 Credits (3-0-0)
\end{flushleft}


\begin{flushleft}
Pre-requisites: COL 351 OR Equivalent
\end{flushleft}


\begin{flushleft}
The course will focus on specialized topics in areas like Computational
\end{flushleft}


\begin{flushleft}
Topology, Manufacturing processes, Quantum Computing,
\end{flushleft}


\begin{flushleft}
Computational Biology, Randomized algorithms and other research
\end{flushleft}


\begin{flushleft}
intensive topics.
\end{flushleft}





\begin{flushleft}
COL867 Special Topics in High Speed Networks
\end{flushleft}


\begin{flushleft}
3 Credits (3-0-0)
\end{flushleft}


\begin{flushleft}
Pre-requisites: COL334 OR COL672
\end{flushleft}


\begin{flushleft}
The course will be delivered through a mix of lectures and paper
\end{flushleft}


\begin{flushleft}
reading seminars on advanced topics in Computer Networks. Handson projects will be conceptualized to challenge students to take
\end{flushleft}


\begin{flushleft}
up current research problems in areas such as software defined
\end{flushleft}


\begin{flushleft}
networking, content distribution, advanced TCP methodologies,
\end{flushleft}


\begin{flushleft}
delay tolerant networking, data center networking, home networking,
\end{flushleft}


\begin{flushleft}
green networking, clean state architecture for the Internet, Internet
\end{flushleft}


\begin{flushleft}
of things, etc.
\end{flushleft}





\begin{flushleft}
COL868 Special topics in Database Systems
\end{flushleft}


\begin{flushleft}
3 Credits (3-0-0)
\end{flushleft}


\begin{flushleft}
Pre-requisites: COL334 / COL672 / Equivalent
\end{flushleft}


\begin{flushleft}
The contents would include specific advanced topics in Database
\end{flushleft}


\begin{flushleft}
Management Systems in which research is currently going on in
\end{flushleft}


\begin{flushleft}
the department. These would be announced every time the course
\end{flushleft}


\begin{flushleft}
is offered.
\end{flushleft}





\begin{flushleft}
COL869 Special topics in Concurrency
\end{flushleft}


\begin{flushleft}
3 Credits (3-0-0)
\end{flushleft}


\begin{flushleft}
The course will focus on research issues in concurrent, distributed
\end{flushleft}


\begin{flushleft}
and mobile computations. Models of Concurrent, Distributed and
\end{flushleft}


\begin{flushleft}
Mobile computation. Process calculi, Event Structures, Petri Nets
\end{flushleft}


\begin{flushleft}
an labeled transition systems. Implementations of concurrent and
\end{flushleft}


\begin{flushleft}
mobile, distributed programming languages. Logics and specification
\end{flushleft}


\begin{flushleft}
models for concurrent and mobile systems. Verification techniques
\end{flushleft}


\begin{flushleft}
and algorithms for model checking. Type systems for concurrent/
\end{flushleft}


\begin{flushleft}
mobile programming languages. Applications of the above models
\end{flushleft}


\begin{flushleft}
and techniques.
\end{flushleft}





\begin{flushleft}
COL870 Special Topics in Machine Learning
\end{flushleft}


\begin{flushleft}
3 Credits (3-0-0)
\end{flushleft}


\begin{flushleft}
Pre-requisites: COL341 OR Equivalent
\end{flushleft}


\begin{flushleft}
Contents may vary based on the instructor's expertise and interests
\end{flushleft}


\begin{flushleft}
within the broader area of Machine Learning. Example topics include
\end{flushleft}


\begin{flushleft}
(but not limiting to) Statistical Relational Learning, Markov Logic,
\end{flushleft}


\begin{flushleft}
Multiple Kernel Learning, Multi-agent Systems, Multi-Class Multi-label
\end{flushleft}


\begin{flushleft}
Learning, Deep Learning, Sum-Product Networks, Active and Semisupervised Learning, Reinforcement Learning, Dealing with Very
\end{flushleft}


\begin{flushleft}
High-Dimensional Data, Learning with Streaming Data, Learning under
\end{flushleft}


\begin{flushleft}
Distributed Architecture.
\end{flushleft}





\begin{flushleft}
COL871 Special Topics in programming languages \&
\end{flushleft}


\begin{flushleft}
Compilers
\end{flushleft}


\begin{flushleft}
3 Credits (3-0-0)
\end{flushleft}


\begin{flushleft}
Pre-requisites: COL728 / COL729 / Equivalent
\end{flushleft}


\begin{flushleft}
Contents may vary based on the instructor's interests within the
\end{flushleft}


\begin{flushleft}
broader area of Programming Languages and Compilers.
\end{flushleft}





\begin{flushleft}
COL872 Special Topics in Cryptography
\end{flushleft}


\begin{flushleft}
3 Credits (3-0-0)
\end{flushleft}


\begin{flushleft}
Pre-requisites: COL759 OR Equivalent
\end{flushleft}


\begin{flushleft}
Contents may vary based on the instructor's interests within
\end{flushleft}


\begin{flushleft}
the broader area of Cryptography. Examples include CCA secure
\end{flushleft}


\begin{flushleft}
encryption, multiparty computation, leakage resilient cryptography,
\end{flushleft}





191





\begin{flushleft}
\newpage
Computer Science
\end{flushleft}





\begin{flushleft}
broadcast encryption, fully homomorphic encryption, obfuscation,
\end{flushleft}


\begin{flushleft}
functional encryption, zero knowledge, private information retrieval,
\end{flushleft}


\begin{flushleft}
byzantine agreement, cryptography against extreme attacks etc.
\end{flushleft}





\begin{flushleft}
COV876 Special Module on Automated Reasoning
\end{flushleft}


\begin{flushleft}
Methods for Program Analysis
\end{flushleft}


\begin{flushleft}
Course Categories: DE for CSI. PE for CS5. PE(SS) for MCS.
\end{flushleft}


\begin{flushleft}
1 Credit (1-0-0)
\end{flushleft}


\begin{flushleft}
Pre-requisites: EC100 for UG
\end{flushleft}





\begin{flushleft}
COV884 Special Module in Artificial Intelligence
\end{flushleft}


\begin{flushleft}
1 Credits (1-0-0)
\end{flushleft}


\begin{flushleft}
Pre-requisites: COL333 / COL671 / Equivalent
\end{flushleft}


\begin{flushleft}
Special module that focuses on special topics and research problems
\end{flushleft}


\begin{flushleft}
of importance in this area.
\end{flushleft}





\begin{flushleft}
COV885 Special Module in Computer Applications
\end{flushleft}


\begin{flushleft}
1 Credits (1-0-0)
\end{flushleft}





\begin{flushleft}
Through the course students will (1) get exposure to fundamental
\end{flushleft}


\begin{flushleft}
concepts in building automated reasoning tools to support deployment
\end{flushleft}


\begin{flushleft}
of formal methods for software and cyber physical systems, (2) get
\end{flushleft}


\begin{flushleft}
an overview of the advanced state of the art approaches towards
\end{flushleft}


\begin{flushleft}
building automated reasoning tools, (3) learn about foundational
\end{flushleft}


\begin{flushleft}
aspects so as to prepare them to pursue these topics and related
\end{flushleft}


\begin{flushleft}
literature independently for research and use in system design and
\end{flushleft}


\begin{flushleft}
other applications and (4) become aware of exciting new directions
\end{flushleft}


\begin{flushleft}
in research on software and system analysis, particularly techniques
\end{flushleft}


\begin{flushleft}
for automatically generating invariant properties.
\end{flushleft}





\begin{flushleft}
COV877 Special Module on Visual Computing
\end{flushleft}


\begin{flushleft}
1 Credit (1-0-0)
\end{flushleft}


\begin{flushleft}
The course will be a seminar-based course where the instructor would
\end{flushleft}


\begin{flushleft}
present topics in a selected theme in the area of visual computing
\end{flushleft}


\begin{flushleft}
through research papers. Students will also be expected to participate
\end{flushleft}


\begin{flushleft}
in the seminar.
\end{flushleft}





\begin{flushleft}
COV878 Special Module in Machine Learning
\end{flushleft}


\begin{flushleft}
1 Credit (1-0-0)
\end{flushleft}


\begin{flushleft}
Contents may vary based on the instructor's expertise and interests
\end{flushleft}


\begin{flushleft}
within the broader area of Machine Learning. Example topics include
\end{flushleft}


\begin{flushleft}
(but not limiting to) Statistical Relational Learning, Markov Logic,
\end{flushleft}


\begin{flushleft}
Multiple Kernel Learning, Multi-agent Systems, Multi-Class Multi-label
\end{flushleft}


\begin{flushleft}
Learning, Deep Learning, Sum-Product Networks, Active and Semisupervised Learning, Reinforcement Learning, Dealing with Very
\end{flushleft}


\begin{flushleft}
High-Dimensional Data, Learning with Streaming Data, Learning under
\end{flushleft}


\begin{flushleft}
Distributed Architecture.
\end{flushleft}





\begin{flushleft}
COV879 Special Module in Financial Algorithms
\end{flushleft}


\begin{flushleft}
1 Credit (1-0-0)
\end{flushleft}


\begin{flushleft}
Pre-requisites: MTL106 OR Equivalent
\end{flushleft}


\begin{flushleft}
Overlap with: MTL 732 \& MTL 733
\end{flushleft}





\begin{flushleft}
Special module that focuses on special topics and research problems
\end{flushleft}


\begin{flushleft}
of importance in this area.
\end{flushleft}





\begin{flushleft}
COV886 Special Module in Algorithms
\end{flushleft}


\begin{flushleft}
1 Credits (1-0-0)
\end{flushleft}


\begin{flushleft}
Pre-requisites: COL351 OR Equivalent
\end{flushleft}


\begin{flushleft}
Special module that focuses on special topics and research problems
\end{flushleft}


\begin{flushleft}
of importance in this area.
\end{flushleft}





\begin{flushleft}
COV887 Special Module in High Speed Networks
\end{flushleft}


\begin{flushleft}
1 Credits (1-0-0)
\end{flushleft}


\begin{flushleft}
Pre-requisites: COL 334 OR COl 672
\end{flushleft}


\begin{flushleft}
The course will be delivered through a mix of lectures and paper
\end{flushleft}


\begin{flushleft}
reading seminars on advanced topics in Computer Networks.
\end{flushleft}


\begin{flushleft}
Students will be introduced to topics such as software defined
\end{flushleft}


\begin{flushleft}
networking, content distribution, advanced TCP methodologies,
\end{flushleft}


\begin{flushleft}
delay tolerant networking, data center networking, home
\end{flushleft}


\begin{flushleft}
networking, green networking, clean state architecture for the
\end{flushleft}


\begin{flushleft}
Internet, Internet of things, etc.
\end{flushleft}





\begin{flushleft}
COV888 Special Module in Database Systems
\end{flushleft}


\begin{flushleft}
1 Credits (1-0-0)
\end{flushleft}


\begin{flushleft}
Pre-requisites: COL362 OR COL632 OR Equivalent
\end{flushleft}


\begin{flushleft}
Potential topics or themes which may be covered (one topic per
\end{flushleft}


\begin{flushleft}
offering) include: data mining, big data management, information
\end{flushleft}


\begin{flushleft}
retrieval and database systems, semantic web data management, etc.
\end{flushleft}





\begin{flushleft}
COV889 Special Module in Concurrency
\end{flushleft}


\begin{flushleft}
1 Credits (1-0-0)
\end{flushleft}


\begin{flushleft}
Pre-requisites: MTL106 OR Equivalent
\end{flushleft}


\begin{flushleft}
Special module that focuses on special topics and research problems
\end{flushleft}


\begin{flushleft}
of importance in this area.
\end{flushleft}





\begin{flushleft}
Special module that focuses on special topics and research problems
\end{flushleft}


\begin{flushleft}
of importance in this area.
\end{flushleft}





\begin{flushleft}
COD891 M.Tech. Minor Project
\end{flushleft}


\begin{flushleft}
3 Credits (0-0-6)
\end{flushleft}





\begin{flushleft}
COV880 Special Module in Parallel Computation
\end{flushleft}


\begin{flushleft}
1 Credits (1-0-0)
\end{flushleft}


\begin{flushleft}
Pre-requisites: Permission of Instructor
\end{flushleft}





\begin{flushleft}
Research and development oriented projects based on problems of
\end{flushleft}


\begin{flushleft}
practical and theoretical interest. Evaluation done based on periodic
\end{flushleft}


\begin{flushleft}
presentations, student seminars, written reports, and evaluation of the
\end{flushleft}


\begin{flushleft}
developed system (if applicable). Students are generally expected to
\end{flushleft}


\begin{flushleft}
work towards the goals and mile stones set for Minor Project COP 891.
\end{flushleft}





\begin{flushleft}
Special module that focuses on special topics and research problems
\end{flushleft}


\begin{flushleft}
of importance in this area.
\end{flushleft}





\begin{flushleft}
COV881 Special Module in Hardware Systems
\end{flushleft}


\begin{flushleft}
1 Credits (1-0-0)
\end{flushleft}


\begin{flushleft}
Pre-requisites: Permission of Instructor
\end{flushleft}


\begin{flushleft}
Special module that focuses on special topics and research problems
\end{flushleft}


\begin{flushleft}
of importance in this area.
\end{flushleft}





\begin{flushleft}
COV882 Special Module in Software Systems
\end{flushleft}


\begin{flushleft}
1 Credits (1-0-0)
\end{flushleft}


\begin{flushleft}
Special module that focuses on special topics and research problems
\end{flushleft}


\begin{flushleft}
of importance in this area.
\end{flushleft}





\begin{flushleft}
COV883 Special Module in Theoretical Computer Science
\end{flushleft}


\begin{flushleft}
1 Credits (1-0-0)
\end{flushleft}


\begin{flushleft}
Pre-requisites: COL 351 OR equivalent
\end{flushleft}


\begin{flushleft}
Special module that focuses on special topics and research problems
\end{flushleft}


\begin{flushleft}
of importance in this area.
\end{flushleft}





\begin{flushleft}
COD892 M.Tech. Project Part-I
\end{flushleft}


\begin{flushleft}
7 Credits (0-0-14)
\end{flushleft}


\begin{flushleft}
It is expected that the problem specification and milestones to be
\end{flushleft}


\begin{flushleft}
achieved in solving the problem are clearly specified. Survey of the
\end{flushleft}


\begin{flushleft}
related area should be completed. This project spans also the course
\end{flushleft}


\begin{flushleft}
COP892. Hence it is expected that the problem specification and the
\end{flushleft}


\begin{flushleft}
milestones to be achieved in solving the problem are clearly specified.
\end{flushleft}





\begin{flushleft}
COD893 M.Tech. Project Part-II
\end{flushleft}


\begin{flushleft}
14 Credits (0-0-28)
\end{flushleft}


\begin{flushleft}
Pre-requisites: COD 892
\end{flushleft}


\begin{flushleft}
The student(s) who work on a project are expected to work towards
\end{flushleft}


\begin{flushleft}
the goals and milstones set in COP893. At the end there would be a
\end{flushleft}


\begin{flushleft}
demonstration of the solution and possible future work on the same
\end{flushleft}


\begin{flushleft}
problem. A dissertation outlining the entire problem, including a survey
\end{flushleft}


\begin{flushleft}
of literature and the various results obtained along with their solutions
\end{flushleft}


\begin{flushleft}
is expected to be produced by each student.
\end{flushleft}





\begin{flushleft}
COD895 MS Research Project
\end{flushleft}


\begin{flushleft}
36 Credits (0-0-72)
\end{flushleft}





192





\begin{flushleft}
\newpage
Department of Electrical Engineering
\end{flushleft}


\begin{flushleft}
ELL100 Introduction to Electrical Engineering
\end{flushleft}


\begin{flushleft}
4 Credits (3-0-2)
\end{flushleft}


\begin{flushleft}
Elements in an Electrical circuit: R, L, C, Diode, Voltage and current
\end{flushleft}


\begin{flushleft}
sources (independent and dependent/controlled sources with
\end{flushleft}


\begin{flushleft}
examples). DC circuits, KCL, KVL, Network theorems, Mesh and nodal
\end{flushleft}


\begin{flushleft}
analysis. Step response in RL, RC, RLC circuits. Phasor analysis of AC
\end{flushleft}


\begin{flushleft}
circuits. Single phase and 3-phase circuits. Two port network, BJT: CE
\end{flushleft}


\begin{flushleft}
and small signal model, Operational amplifiers: Model and applications
\end{flushleft}


\begin{flushleft}
Introduction to Digital circuits. Magnetic circuits, Transformers:
\end{flushleft}


\begin{flushleft}
Modeling and analysis; parameter determination. Energy in magnetic
\end{flushleft}


\begin{flushleft}
field. Electromechanical energy conversion principles with examples.
\end{flushleft}


\begin{flushleft}
Principles of measurement of voltage, current and power.
\end{flushleft}


\begin{flushleft}
Laboratory component and the List of experiments.
\end{flushleft}


\begin{flushleft}
CRO (mechanism and usage). KCL, KVL, Network theorem verification.
\end{flushleft}


\begin{flushleft}
Step/ transient response of RL, RC, RLC, circuits. Steady state response
\end{flushleft}


\begin{flushleft}
of Circuits of sinusoidal excitation. Diode experiment (clipping,
\end{flushleft}


\begin{flushleft}
clamping and rectification). Basic circuits using opamp. Transformers
\end{flushleft}


\begin{flushleft}
OC and SC tests. BH loop in an iron core, DC and AC motor -- for
\end{flushleft}


\begin{flushleft}
observation only. A small mini-project.
\end{flushleft}





\begin{flushleft}
ELL201 Digital Electronics
\end{flushleft}


\begin{flushleft}
4.5 Credits (3-0-3)
\end{flushleft}


\begin{flushleft}
Pre-requisites: EEL 100
\end{flushleft}


\begin{flushleft}
Gates, binary number systems, arithmetic operations. Minimization
\end{flushleft}


\begin{flushleft}
using K-maps, reduced K-maps, tabular methods; design using
\end{flushleft}


\begin{flushleft}
multiplexers, decoders, and ROMs. Latches, flip-flops, registers and
\end{flushleft}


\begin{flushleft}
counters. Asynchronous, synchronous counters. Finite state machines,
\end{flushleft}


\begin{flushleft}
implementations thereof. Mealy, Moore machines. Clock period
\end{flushleft}


\begin{flushleft}
computation. Memories. Partitioning and pipelining. VHDL/Verilog,
\end{flushleft}


\begin{flushleft}
the register-transfer-level description style. Switch level introduction
\end{flushleft}


\begin{flushleft}
to logic families, CMOS logic, static, pre-charge and clocked logic.
\end{flushleft}


\begin{flushleft}
Asynchronous circuits and design styles.
\end{flushleft}





\begin{flushleft}
ELL202 Circuit Theory
\end{flushleft}


\begin{flushleft}
4 Credits (3-1-0)
\end{flushleft}


\begin{flushleft}
Pre-requisites: ELL100
\end{flushleft}


\begin{flushleft}
Overview of network analysis techniques, network theorems,
\end{flushleft}


\begin{flushleft}
transient and steady-state sinusoidal response. Network graphs and
\end{flushleft}


\begin{flushleft}
their applications in network analysis. Tellegen's theorem, two-port
\end{flushleft}


\begin{flushleft}
networks, Z, Y, h, g, and transmission matrices. Combining two ports
\end{flushleft}


\begin{flushleft}
in various configurations. Analysis of transmission lines to motivate
\end{flushleft}


\begin{flushleft}
the scattering matrix. Scattering matrix and its applications in network
\end{flushleft}


\begin{flushleft}
analysis. Network functions, positive real functions, and network
\end{flushleft}


\begin{flushleft}
synthesis. Butterworth and Chebyshev approximations. Synthesis of
\end{flushleft}


\begin{flushleft}
lossless two-port networks. Synthesis of lattice all-pass filters.
\end{flushleft}





\begin{flushleft}
ELL203 Electromechanics
\end{flushleft}


\begin{flushleft}
4 Credits (3-1-0)
\end{flushleft}


\begin{flushleft}
Pre-requisites: EEL 100
\end{flushleft}


\begin{flushleft}
Review: AC Circuits, Complex representation and Power Measurement.
\end{flushleft}


\begin{flushleft}
Magnetic Circuits: Simple magnetic circuit, analogy between magnetic
\end{flushleft}


\begin{flushleft}
circuits and electrical circuits, linear and nonlinear magnetic circuits,
\end{flushleft}


\begin{flushleft}
hysteresis and eddy current losses, permanent magnet materials.
\end{flushleft}


\begin{flushleft}
Transformers: Single-phase and three-phase, analysis, equivalent
\end{flushleft}


\begin{flushleft}
circuit, Tests on transformers, phasor diagram regulation and efficiency,
\end{flushleft}


\begin{flushleft}
auto-transformer and instrument transformers (PT/CT).
\end{flushleft}


\begin{flushleft}
Electro-mechanical energy conversion principles: Force and EMF
\end{flushleft}


\begin{flushleft}
production in a rotating machine.
\end{flushleft}


\begin{flushleft}
DC machines: Types, construction, working principle, characteristics
\end{flushleft}


\begin{flushleft}
and applications.
\end{flushleft}


\begin{flushleft}
3-phase induction machines: Types, construction, Introduction to
\end{flushleft}


\begin{flushleft}
windings and winding factor, production of revolving magnetic field,
\end{flushleft}


\begin{flushleft}
working principle on 3-phase induction machine, equivalent circuit,
\end{flushleft}


\begin{flushleft}
characteristics, phasor diagram and applications.
\end{flushleft}


\begin{flushleft}
3-phase synchronous machines: Types, construction, working principle,
\end{flushleft}


\begin{flushleft}
equivalent circuit, characteristics, phasor diagram and applications.
\end{flushleft}


\begin{flushleft}
Fractional-HP and Special Machines.
\end{flushleft}





\begin{flushleft}
ELP203 Electromechanics Laboratory
\end{flushleft}


\begin{flushleft}
1.5 Credits (0-0-3)
\end{flushleft}


\begin{flushleft}
Pre-requisites: ELL100
\end{flushleft}


\begin{flushleft}
ELL205 Signals and Systems
\end{flushleft}


\begin{flushleft}
4 Credits (3-1-0)
\end{flushleft}


\begin{flushleft}
Motivation \& orientation, Classifications of signals \& systems, Dynamic
\end{flushleft}


\begin{flushleft}
representation of LTI systems (discrete \& continuous-time systems),
\end{flushleft}


\begin{flushleft}
Fourier analysis of continuous-time signals \& systems, Fourier analysis
\end{flushleft}


\begin{flushleft}
of discrete-time signals \& systems, Nyquist sampling theorem, Laplace
\end{flushleft}


\begin{flushleft}
transform, The z-transform, Introduction to probability, random
\end{flushleft}


\begin{flushleft}
variables and stochastic processes.
\end{flushleft}





\begin{flushleft}
ELL211 Physical Electronics
\end{flushleft}


\begin{flushleft}
3 Credits (3-0-0)
\end{flushleft}


\begin{flushleft}
Pre-requisites: ELL100 and PYL100
\end{flushleft}


\begin{flushleft}
Overlaps with: EEL732, ELL231, EPL336, EPL439 , PHL653,
\end{flushleft}


\begin{flushleft}
PHL704, PHL705, PHL727, PHL793
\end{flushleft}


\begin{flushleft}
Semiconductor materials , crystal structure, carriers in semiconductors,
\end{flushleft}


\begin{flushleft}
band structure, density of states, excitons, doping and carrier
\end{flushleft}


\begin{flushleft}
statistics, carrier transport, recombination and generation, p-n junction
\end{flushleft}


\begin{flushleft}
physics: built-in potential, forward and reverse bias, capacitance,
\end{flushleft}


\begin{flushleft}
diode currents, breakdown, tunnel effects; metal-semiconductor
\end{flushleft}


\begin{flushleft}
junctions; BJTs: current gain/Gummel plots, transistor models,
\end{flushleft}


\begin{flushleft}
breakdown;MOSFET physics: MOS capacitors, inversion, depletion,
\end{flushleft}


\begin{flushleft}
accumulation, flatband, threshold voltage, long-channel model,
\end{flushleft}


\begin{flushleft}
saturation, short-channel models, sub-threshold conduction, SPICE
\end{flushleft}


\begin{flushleft}
models for MOSFETs; optoelectronic device physics, LEDs/OLEDs,
\end{flushleft}


\begin{flushleft}
lasers, photodetectors, solar cells.
\end{flushleft}





\begin{flushleft}
ELL212 Electromagnetics
\end{flushleft}


\begin{flushleft}
4 Credits (3-1-0)
\end{flushleft}


\begin{flushleft}
Pre-requisites: PYL100
\end{flushleft}


\begin{flushleft}
Review of Maxwell's equations, wave propagations in unbounded
\end{flushleft}


\begin{flushleft}
medium. Boundary conditions, reflection and refraction of plane waves.
\end{flushleft}


\begin{flushleft}
Evanescent waves and surface plasmons. Waveguides: parallel-plane
\end{flushleft}


\begin{flushleft}
guide, TE, TM and TEM waves, rectangular and cylindrical waveguides,
\end{flushleft}


\begin{flushleft}
resonators. Dielectric guides and optical fibres. Transmission Lines:
\end{flushleft}


\begin{flushleft}
distributed parameter circuits, traveling and standing waves,
\end{flushleft}


\begin{flushleft}
impedance matching, Smith chart, analogy with plane waves. Planar
\end{flushleft}


\begin{flushleft}
transmission lines: stripline, micro stripline. Radiation: retarded
\end{flushleft}


\begin{flushleft}
potentials, Hertzian dipole, short loop, antenna parameters. Numerical
\end{flushleft}


\begin{flushleft}
techniques in electromagnetics.
\end{flushleft}





\begin{flushleft}
ELP212 Electromagnetics Laboratory
\end{flushleft}


\begin{flushleft}
1.5 Credits (0-0-3)
\end{flushleft}


\begin{flushleft}
Pre-requisites: ELL212
\end{flushleft}


\begin{flushleft}
ELL225 Control Engineering
\end{flushleft}


\begin{flushleft}
4 Credits (3-1-0)
\end{flushleft}


\begin{flushleft}
Pre-requisites: ELL205
\end{flushleft}


\begin{flushleft}
Overlaps with: MCL212, CLL261
\end{flushleft}


\begin{flushleft}
Introduction to the control problem, Control System Components:
\end{flushleft}


\begin{flushleft}
Sensors, Actuators, Computational blocks. Mathematical representation
\end{flushleft}


\begin{flushleft}
of systems, state variable model, linearization, transfer function model.
\end{flushleft}


\begin{flushleft}
Transfer function and state variable models of suitable mechanical,
\end{flushleft}


\begin{flushleft}
electrical, thermal and pneumatic systems. Closed loop systems, Block
\end{flushleft}


\begin{flushleft}
diagram and signal flow analysis, Basic Characteristics of feedback
\end{flushleft}


\begin{flushleft}
control systems: stability, steady-state accuracy, transient accuracy,
\end{flushleft}


\begin{flushleft}
disturbance rejection, sensitivity analysis and robustness. Basic modes
\end{flushleft}


\begin{flushleft}
of feedback control: Proportional, Integral, Derivative. Concept of
\end{flushleft}


\begin{flushleft}
stability, Stability criteria: Routh stability criterion, Mikhailov's criterion,
\end{flushleft}


\begin{flushleft}
Kharitonov theorem. Time response of 2nd order system, steady state
\end{flushleft}


\begin{flushleft}
error analysis. Performance specifications in the time domain. Root
\end{flushleft}


\begin{flushleft}
locus method of design. Nyquist stability criterion. Frequency response
\end{flushleft}


\begin{flushleft}
analysis: Nyquist plots, Bode plots, Nichols Charts, Performance
\end{flushleft}


\begin{flushleft}
specifications in frequency domain, Frequency domain methods of
\end{flushleft}


\begin{flushleft}
design. Lead lag compensation.
\end{flushleft}





193





\begin{flushleft}
\newpage
Electrical Engineering
\end{flushleft}





\begin{flushleft}
ELP225 Control Engineering Laboratory
\end{flushleft}


\begin{flushleft}
1.5 Credits (0-0-3)
\end{flushleft}


\begin{flushleft}
Pre-requisites: ELL225
\end{flushleft}


\begin{flushleft}
Basics of Sensors and Actuators, Study of AC and DC Motors, Linear
\end{flushleft}


\begin{flushleft}
Systems, Analog and Digital Motors, Synchros, Temperature Control.
\end{flushleft}





\begin{flushleft}
Power flow analysis. Fault analysis in power systems. Power system
\end{flushleft}


\begin{flushleft}
stability studies. Transients in power system and travelling waves.
\end{flushleft}


\begin{flushleft}
Introduction to power system relaying and brief idea of over current,
\end{flushleft}


\begin{flushleft}
differentia and impedance based protection. Basic concepts of Power
\end{flushleft}


\begin{flushleft}
system operation and control. Introduction to HVDC and FACTS.
\end{flushleft}





\begin{flushleft}
ELP303 Power Engineering Laboratory
\end{flushleft}


\begin{flushleft}
1.5 Credits (0-0-3)
\end{flushleft}


\begin{flushleft}
Pre-requisites: ELP303
\end{flushleft}





\begin{flushleft}
ELL231 Power Electronics and Energy Devices
\end{flushleft}


\begin{flushleft}
3 Credits (3-0-0)
\end{flushleft}


\begin{flushleft}
Pre-requisites: ELL100
\end{flushleft}


\begin{flushleft}
Introduction to semiconductor basics and PN Junctions. Short
\end{flushleft}


\begin{flushleft}
introduction to power device technology, PIN diodes, Schottky diodes,
\end{flushleft}


\begin{flushleft}
Power BJTs, Power MOSFETs, IGBTs, Thyristors, Wide bandgap power
\end{flushleft}


\begin{flushleft}
semiconductor devices, Packaging and Reliability of Power devices,
\end{flushleft}


\begin{flushleft}
Destructive mechanisms in power devices, Power device induced
\end{flushleft}


\begin{flushleft}
oscillations and Electromagnetic disturbances, Selection of power
\end{flushleft}


\begin{flushleft}
devices in power electronic systems, Smart power integrated circuits.
\end{flushleft}





\begin{flushleft}
ELL301 Electrical and Electronics Instrumentation
\end{flushleft}


\begin{flushleft}
3 Credits (3-0-0)
\end{flushleft}


\begin{flushleft}
Pre-requisites: ELL100
\end{flushleft}


\begin{flushleft}
Basics of Measurement and Instrumentation, Instrument Examples:
\end{flushleft}


\begin{flushleft}
Galvanometer, Accelerometer etc; calibration methods, Voltage
\end{flushleft}


\begin{flushleft}
and Current Measurements; Theory, calibration, application,
\end{flushleft}


\begin{flushleft}
Errors and compensation. Power and Energy Measurement and its
\end{flushleft}


\begin{flushleft}
errors, Methods of correction, LPF wattmeter, Phantom loading,
\end{flushleft}


\begin{flushleft}
Induction type KWH meter; Calibration of wattmeter, energy
\end{flushleft}


\begin{flushleft}
meter. Potentiometer and Instrument Transformer :DC and AC
\end{flushleft}


\begin{flushleft}
potentiometer, C.T. and V.T. construction, theory, operation,
\end{flushleft}


\begin{flushleft}
characteristics. Digital Instrumentation.
\end{flushleft}





\begin{flushleft}
Experiments will be conducted on 3-phase alternators and transformers
\end{flushleft}


\begin{flushleft}
for measuring their sequence impedance. Directional, overcurrent and
\end{flushleft}


\begin{flushleft}
differential protection relays will be studied. Computer simulation for
\end{flushleft}


\begin{flushleft}
power flow, short circuit and stability studies of interconnected power
\end{flushleft}


\begin{flushleft}
systems. Numerical relays and synchrophasors will be introduced.
\end{flushleft}


\begin{flushleft}
FACTS devices will be experimented.
\end{flushleft}





\begin{flushleft}
ELL304 Analog Electronics
\end{flushleft}


\begin{flushleft}
5.5 Credits (3-1-3)
\end{flushleft}


\begin{flushleft}
Pre-requisites: ELL100, ELL202, ELL211, ELL231
\end{flushleft}


\begin{flushleft}
Review of working of BJT and MOSFET, large signal and small signal
\end{flushleft}


\begin{flushleft}
models, biasing schemes, analysis and design of various single stage
\end{flushleft}


\begin{flushleft}
amplifier configuration, low and high frequency analysis of single
\end{flushleft}


\begin{flushleft}
stage amplifiers, frequency compensation, current mirrors, multistage
\end{flushleft}


\begin{flushleft}
amplifiers; differential and operational amplifiers, negative and positive
\end{flushleft}


\begin{flushleft}
feedback, oscillators and power amplifiers.
\end{flushleft}





\begin{flushleft}
ELL305 Computer Architecture
\end{flushleft}


\begin{flushleft}
3 Credits (3-0-0)
\end{flushleft}


\begin{flushleft}
Pre-requisites: ELL201
\end{flushleft}


\begin{flushleft}
Overlaps with: CSL211
\end{flushleft}





\begin{flushleft}
ELL302 Power Electronics
\end{flushleft}


\begin{flushleft}
3 Credits (3-0-0)
\end{flushleft}


\begin{flushleft}
Pre-requisites: ELL231 (EE3) / ELL211 (EE1)
\end{flushleft}





\begin{flushleft}
Introduction: Performance measurement, Instruction Set Architecture,
\end{flushleft}


\begin{flushleft}
Computer Arithmetic, Processor: ALU design, Control design,
\end{flushleft}


\begin{flushleft}
Pipelining, Memory Hierarchy, I/O management, Multicores,
\end{flushleft}


\begin{flushleft}
Multiprocessors, Clusters, GPU.
\end{flushleft}





\begin{flushleft}
Introduction to Power Electronics devices and protection: Thyristor
\end{flushleft}


\begin{flushleft}
family devices, principle of operation, IGBT operation, principles and
\end{flushleft}


\begin{flushleft}
ratings. Snubber designs, selection and protection, Firing circuits.
\end{flushleft}





\begin{flushleft}
ELP305 Design and System Laboratory
\end{flushleft}


\begin{flushleft}
1.5 Credits (0-0-3)
\end{flushleft}





\begin{flushleft}
AC-DC converters: uncontrolled, semi-controlled, fully controlled
\end{flushleft}


\begin{flushleft}
and dual converters in single-phase and three-phase configurations,
\end{flushleft}


\begin{flushleft}
design, phase control, harmonic analysis, firing circuits and their
\end{flushleft}


\begin{flushleft}
designs. Improved power quality AC-DC converters.
\end{flushleft}


\begin{flushleft}
Choppers: Introduction to dc-dc conversion, various topologies,
\end{flushleft}


\begin{flushleft}
buck, boost, buck-boost converters, High frequency isolated dc-dc
\end{flushleft}


\begin{flushleft}
converters: design problems, PWM control and operation.
\end{flushleft}


\begin{flushleft}
Inverters: Basics of dc to ac conversion, inverter circuit configurations
\end{flushleft}


\begin{flushleft}
and principle of operation, VSI and CSI, single and three-phase
\end{flushleft}


\begin{flushleft}
configurations, Square wave and sinusoidal PWM control methods
\end{flushleft}


\begin{flushleft}
and harmonic control. Design problems.
\end{flushleft}


\begin{flushleft}
AC voltage controllers: Introduction to ac to ac conversion, singlephase and three-phase ac voltage controller circuit configurations,
\end{flushleft}


\begin{flushleft}
applications, advantages, harmonic analysis, control, design problems.
\end{flushleft}


\begin{flushleft}
Cyclo-converters: single-phase to single-phase, three-phase to singlephase, three-phase to three-phase and single-phase to three-phase
\end{flushleft}


\begin{flushleft}
circuit configurations thyristors and triacs.
\end{flushleft}





\begin{flushleft}
ELP302 Power Electronics Laboratory
\end{flushleft}


\begin{flushleft}
1.5 Credits (0-0-3)
\end{flushleft}


\begin{flushleft}
Pre-requisites: ELL302
\end{flushleft}





\begin{flushleft}
ELL311 Communication Engineering
\end{flushleft}


\begin{flushleft}
4 Credits (3-1-0)
\end{flushleft}


\begin{flushleft}
Pre-requisites: ELL205
\end{flushleft}


\begin{flushleft}
Review of Fourier Series and Transforms. Hilbert Transforms,
\end{flushleft}


\begin{flushleft}
BandpassSignal and System Representation. Random Processes,
\end{flushleft}


\begin{flushleft}
Stationarity, Power Spectral Density, Gaussian Process, Noise. Amplitude
\end{flushleft}


\begin{flushleft}
Modulation, DSBSC, SSB, VSB: Signal Representation, Generation
\end{flushleft}


\begin{flushleft}
and Demodulation. Frequency Modulation: Signal Representation,
\end{flushleft}


\begin{flushleft}
Generation and Demodulation. Mixing, Superheterodyne Receiver,
\end{flushleft}


\begin{flushleft}
Phase Recovery with PLLs. Noise: in AM Receivers using Coherent
\end{flushleft}


\begin{flushleft}
Detection, in AM Receiversusing Envelope Detection, in FM Receivers.
\end{flushleft}


\begin{flushleft}
Sampling, Pulse-AmplitudeModulation. Quantization, Pulse-Code
\end{flushleft}


\begin{flushleft}
Modulation. Noise considerations in PCM, Time Division Multiplexing,
\end{flushleft}


\begin{flushleft}
Delta Modulation. Intersymbol Interference, Introduction to
\end{flushleft}


\begin{flushleft}
Information Theory: concepts of Entropy and Source-Coding
\end{flushleft}





\begin{flushleft}
ELP311 Communication Engineering Laboratory
\end{flushleft}


\begin{flushleft}
1 Credit (0-0-2)
\end{flushleft}


\begin{flushleft}
Pre-requisites: ELL311
\end{flushleft}





\begin{flushleft}
Laboratory experiments on analog, pulse, and basic digital modulation
\end{flushleft}


\begin{flushleft}
and demodulation techniques.
\end{flushleft}





\begin{flushleft}
ELL303 Power Engineering-I
\end{flushleft}


\begin{flushleft}
4 Credits (3-1-0)
\end{flushleft}


\begin{flushleft}
Pre-requisites: ELL100, ELL203
\end{flushleft}


\begin{flushleft}
Introduction to the basic structure of power system along with various
\end{flushleft}


\begin{flushleft}
power generation technologies. Modeling of generators, transformers
\end{flushleft}


\begin{flushleft}
and transmission line for power system analysis. per unit system.
\end{flushleft}





\begin{flushleft}
ELL312 Semiconductor process technology
\end{flushleft}


\begin{flushleft}
3 Credits (3-0-0)
\end{flushleft}


\begin{flushleft}
Pre-requisites: ELL211
\end{flushleft}


\begin{flushleft}
Overlaps with: ELL784
\end{flushleft}


\begin{flushleft}
Semiconductor materials (inorganic and organic), history of
\end{flushleft}


\begin{flushleft}
semiconductor IC devices, crystal structure, defects, vacancies
\end{flushleft}





194





\begin{flushleft}
\newpage
Electrical Engineering
\end{flushleft}





\begin{flushleft}
and interstitials, semiconductor crystal growth, bulk doping
\end{flushleft}


\begin{flushleft}
methods, purification methods, wafer manufacture, diffusion,
\end{flushleft}


\begin{flushleft}
surface doping, oxidation, dopant redistribution, ion implantation
\end{flushleft}


\begin{flushleft}
and annealing, rapid thermal processes, photolithography, masks,
\end{flushleft}


\begin{flushleft}
photoresists, exposure, e-beam lithography, vacuum systems,
\end{flushleft}


\begin{flushleft}
gas flow, plasma processes, pumping theory, leaks, vacuum
\end{flushleft}


\begin{flushleft}
gauges, wet etching, plasma etching, process gas chemistry and
\end{flushleft}


\begin{flushleft}
polymerization, ion milling, reactive ion etching, lift-off, vapor
\end{flushleft}


\begin{flushleft}
pressure of materials, evaporation, sputtering, deposition rate
\end{flushleft}


\begin{flushleft}
and step coverage, codepositions, film growth mechanisms and
\end{flushleft}


\begin{flushleft}
stress, chemical vapor deposition, metal-organic chemical vapor
\end{flushleft}


\begin{flushleft}
deposition, atomic layer deposition, molecular beam epitaxy,
\end{flushleft}


\begin{flushleft}
planarization processes, interconnects, yield and device integration.
\end{flushleft}





\begin{flushleft}
ELL313 Antennas and Propagation
\end{flushleft}


\begin{flushleft}
3 Credits (3-0-0)
\end{flushleft}


\begin{flushleft}
Prerequisites: ELL212
\end{flushleft}


\begin{flushleft}
Starting from the principle of radiation different types of antenna;
\end{flushleft}


\begin{flushleft}
wire, slot, planar and their arrays with feeds. Antenna synthesis
\end{flushleft}


\begin{flushleft}
and design and measurements. Characteristics of propagation
\end{flushleft}


\begin{flushleft}
of radio waves in different atmospheric layers and study of the
\end{flushleft}


\begin{flushleft}
losses, fading and scattering of microwave and millimeter waves
\end{flushleft}


\begin{flushleft}
in the atmosphere.
\end{flushleft}





\begin{flushleft}
ELL315 Introduction to Analog Integrated Circuits
\end{flushleft}


\begin{flushleft}
3 Credits (3-0-0)
\end{flushleft}


\begin{flushleft}
Pre-requisites: ELL204, ELL202
\end{flushleft}


\begin{flushleft}
Review of basic amplifiers. Current Mirrors, Reference Current and
\end{flushleft}


\begin{flushleft}
Voltage Sources. CMOS Operational Amplifier: Structure, Analysis
\end{flushleft}


\begin{flushleft}
and Design, Frequency Response and Compensation Techniques.
\end{flushleft}


\begin{flushleft}
Switched Capacitor Circuits: Principles of operation, Filter and non
\end{flushleft}


\begin{flushleft}
filter applications. Sample and Hold Circuits, Comparators. ADC:
\end{flushleft}


\begin{flushleft}
Characterization, Types of ADC and their relative merits and demerits,
\end{flushleft}


\begin{flushleft}
Design issues. DAC: Characterization, Types of DAC and their relative
\end{flushleft}


\begin{flushleft}
merits and demerits, Design issues.
\end{flushleft}





\begin{flushleft}
ELL316 Introduction to VLSI Design
\end{flushleft}


\begin{flushleft}
3 Credits (3-0-0)
\end{flushleft}


\begin{flushleft}
Pre-requisites: ELL211
\end{flushleft}


\begin{flushleft}
Basic MOS characteristics; Deep sub-micron; velocity saturation;
\end{flushleft}


\begin{flushleft}
Dynamic MOS characteristics; parasitics; leakage; sizing; propagation
\end{flushleft}


\begin{flushleft}
delay; Logical effort, path delay, optimization; Ratio-ed logic, Pass
\end{flushleft}


\begin{flushleft}
transistor logic and parasitics; Dynamic logic, pulsed sequential
\end{flushleft}


\begin{flushleft}
logic; Logical synthesis, physical design, layout; Introduction to
\end{flushleft}


\begin{flushleft}
design of VLSI memories.
\end{flushleft}





\begin{flushleft}
Basic Concepts: Characteristics and operating modes of drive motors.
\end{flushleft}


\begin{flushleft}
Starting, braking and speed control of motors. 4 quadrant drives. Types
\end{flushleft}


\begin{flushleft}
of loads. Torque and associated controls used in process industries.
\end{flushleft}


\begin{flushleft}
DC Motor Drives: Characteristics, Starting Methods, Braking Methods,
\end{flushleft}


\begin{flushleft}
Speed Control Using Converters and Choppers.
\end{flushleft}


\begin{flushleft}
Three phase Induction Motor Drives: Characteristics and Equivalent
\end{flushleft}


\begin{flushleft}
Circuits, Starting Methods, Braking Methods, Speed Control of Cage
\end{flushleft}


\begin{flushleft}
Rotor Induction Machines using as AC voltage controllers, VoltageSource and Current-Source Inverters. V-by-F Control and other Control
\end{flushleft}


\begin{flushleft}
Techniques. Speed Control of Wound-Rotor Induction Machines using
\end{flushleft}


\begin{flushleft}
Rotor Resistance Variation; Slip-Power Recovery Scheme.
\end{flushleft}


\begin{flushleft}
Three phase Synchronous Motor Drives: Characteristics and Equivalent
\end{flushleft}


\begin{flushleft}
Circuits, Starting Methods, Braking Methods, Speed Control in True
\end{flushleft}


\begin{flushleft}
Synchronous and Self Control Modes.
\end{flushleft}


\begin{flushleft}
Special Machines: Permanent Magnet Brush-Less Motor Drives,
\end{flushleft}


\begin{flushleft}
Permanent Magnet Synchronous Motor Drives, Stepper and Reluctance
\end{flushleft}


\begin{flushleft}
Motor Drives.
\end{flushleft}





\begin{flushleft}
ELP332 Electric Drives Laboratory
\end{flushleft}


\begin{flushleft}
1.5 Credits (0-0-3)
\end{flushleft}


\begin{flushleft}
Pre-requisites: ELL332
\end{flushleft}


\begin{flushleft}
ELL333 Multivariable Control
\end{flushleft}


\begin{flushleft}
3 Credits (3-0-0)
\end{flushleft}


\begin{flushleft}
Pre-requisites: ELL225
\end{flushleft}


\begin{flushleft}
Overlaps with: ELL721
\end{flushleft}


\begin{flushleft}
Review of control system fundamentals and basic linear algebra.
\end{flushleft}


\begin{flushleft}
Introduction to linear dynamical systems and properties. State-space
\end{flushleft}


\begin{flushleft}
representation and canonical realizations. Relation between statespace and transfer function representations. Similarity transformation.
\end{flushleft}


\begin{flushleft}
Diagonalization. Jordan canonical form. Matrix exponential and its
\end{flushleft}


\begin{flushleft}
properties. Solution of state equations. Cayley-Hamilton Theorem,
\end{flushleft}


\begin{flushleft}
Stability: BIBO and internal. Linearization of nonlinear systems.
\end{flushleft}


\begin{flushleft}
Controllability and Observability. Minimal realization. State feedback
\end{flushleft}


\begin{flushleft}
and observer design. Linear Quadratic Regulator.
\end{flushleft}





\begin{flushleft}
ELL334 DSP based Control of Drives
\end{flushleft}


\begin{flushleft}
4 Credits (3-0-2)
\end{flushleft}


\begin{flushleft}
Pre-requisites: ELL203, ELL332
\end{flushleft}





\begin{flushleft}
ELL318 Digital Hardware Design
\end{flushleft}


\begin{flushleft}
3 Credits (3-0-0)
\end{flushleft}


\begin{flushleft}
Pre-requisites: ELL305
\end{flushleft}


\begin{flushleft}
Overlaps with: CSL316
\end{flushleft}


\begin{flushleft}
Technology basics and digital logic families such as static CMOS, pass
\end{flushleft}


\begin{flushleft}
transistor, transmission gate, dynamic and domino logic. Advanced
\end{flushleft}


\begin{flushleft}
sequential logic elements with latch-based design and timing and
\end{flushleft}


\begin{flushleft}
clocking concepts. Design flows and paradigms. Data path, control
\end{flushleft}


\begin{flushleft}
and advanced pipeline implementations. Advanced digital arithmetic.
\end{flushleft}


\begin{flushleft}
Performance evaluation.
\end{flushleft}





\begin{flushleft}
ELL319 Digital Signal Processing
\end{flushleft}


\begin{flushleft}
4 Credits (3-0-2)
\end{flushleft}


\begin{flushleft}
Pre-requisites: ELL205
\end{flushleft}


\begin{flushleft}
Review of Signals and Systems, Sampling and data reconstruction
\end{flushleft}


\begin{flushleft}
processes. Z transforms. Discrete linear systems. Frequency domain
\end{flushleft}


\begin{flushleft}
design of digital filters. Quantization effects in digital filters. Discrete
\end{flushleft}


\begin{flushleft}
Fourier transform and FFT algorithms. High speed convolution and
\end{flushleft}


\begin{flushleft}
its application to digital filtering.
\end{flushleft}





\begin{flushleft}
ELS330 Independent Study (EE3)
\end{flushleft}


\begin{flushleft}
3 Credits (0-3-0)
\end{flushleft}





\begin{flushleft}
ELL332 Electric Drives
\end{flushleft}


\begin{flushleft}
3 Credits (3-0-0)
\end{flushleft}


\begin{flushleft}
Pre-requisites: ELL203
\end{flushleft}





\begin{flushleft}
Introduction and Application of DSP in the power electronic converter
\end{flushleft}


\begin{flushleft}
controlled drives, Types of processors used for power control and their
\end{flushleft}


\begin{flushleft}
comparison, computational advantages, Limitations. Introduction to
\end{flushleft}


\begin{flushleft}
peripherals ADC, DAC, PWM, Encoders and their interface. Interfacing
\end{flushleft}


\begin{flushleft}
issues, Sampling process, Harmonic analysis in real-time using
\end{flushleft}


\begin{flushleft}
a DSP, Assembly language programming of a DSP, Motor control
\end{flushleft}


\begin{flushleft}
applications. Pulse-Width Modulation and Pulse-Frequency Modulation
\end{flushleft}


\begin{flushleft}
schemes, lookup tables and real-time computation. Interfacing and
\end{flushleft}


\begin{flushleft}
signal conditioning circuits for DSP based schemes. Realization of
\end{flushleft}


\begin{flushleft}
computationally intensive algorithms like variable structure, adaptive
\end{flushleft}


\begin{flushleft}
and neural network schemes for Drives systems.
\end{flushleft}





\begin{flushleft}
ELL335 CAD of Electric Machines
\end{flushleft}


\begin{flushleft}
4 Credits (3-0-2)
\end{flushleft}


\begin{flushleft}
Pre-requisites: ELL103
\end{flushleft}


\begin{flushleft}
1. Basic Considerations, 2. Design of Main Dimensions, 3. Transformer
\end{flushleft}


\begin{flushleft}
Design, 4. Design of rotating machines, 5. Computer Aided Design of
\end{flushleft}


\begin{flushleft}
Transformers, 6. Computer Aided Design of DC machines, 7. Computer
\end{flushleft}


\begin{flushleft}
Aided Design of Synchronous Machines, 8. Computer Aided Design of
\end{flushleft}


\begin{flushleft}
Induction Machines, 9. Computer Aided Design of Special Machines.
\end{flushleft}





195





\begin{flushleft}
\newpage
Electrical Engineering
\end{flushleft}





\begin{flushleft}
ELL363 Power Engineering-II
\end{flushleft}


\begin{flushleft}
3 Credits (3-0-0)
\end{flushleft}


\begin{flushleft}
Pre-requisites: ELL303
\end{flushleft}





\begin{flushleft}
vector routing, link state routing, RIP, OSPF; (viii) Cross-layer protocol
\end{flushleft}


\begin{flushleft}
optimization concepts: Distributed control, cost and energy efficiencies.
\end{flushleft}





\begin{flushleft}
Advanced concepts in power flow analysis, security analysis and state
\end{flushleft}


\begin{flushleft}
estimation. Economic load dispatch and unit commitment problem.
\end{flushleft}


\begin{flushleft}
Voltage and frequency control in power systems. Advanced concepts
\end{flushleft}


\begin{flushleft}
in multi-machine dynamics and stability. Electrical transients in power
\end{flushleft}


\begin{flushleft}
systems. Wind and solar generation technologies and their integration
\end{flushleft}


\begin{flushleft}
into the grid. Issues in restructured power systems. Modern numerical
\end{flushleft}


\begin{flushleft}
protection.
\end{flushleft}





\begin{flushleft}
ELL365 Embedded Systems
\end{flushleft}


\begin{flushleft}
3 Credits (3-0-0)
\end{flushleft}


\begin{flushleft}
Overview of Embedded Systems; Embedded System Architecture:
\end{flushleft}


\begin{flushleft}
processor example­A RM, PIC, etc.; features of digital signal
\end{flushleft}


\begin{flushleft}
processor; SOC, memory sub­system, busstructure (PC­104, I2C, SPI
\end{flushleft}


\begin{flushleft}
etc.), interfacing protocols (USB, IrDA etc), testing and debugging,
\end{flushleft}


\begin{flushleft}
power management; Embedded System Software: Program
\end{flushleft}


\begin{flushleft}
Optimization,Concurrent Programming, Real­time Scheduling and I/O
\end{flushleft}


\begin{flushleft}
management; Networked Embedded Systems: special networking
\end{flushleft}


\begin{flushleft}
protocols (CAN, Bluetooth); Applications.
\end{flushleft}





\begin{flushleft}
ELL400 Power Systems Protection
\end{flushleft}


\begin{flushleft}
3 Credits (3-0-0)
\end{flushleft}


\begin{flushleft}
Pre-requisites: ELL303
\end{flushleft}


\begin{flushleft}
Fundamentals of Power system protection, philosophy of protective
\end{flushleft}


\begin{flushleft}
relays, Different types of relays, Introduction to protection elements
\end{flushleft}


\begin{flushleft}
like CT, PT, CB, Isolator etc, (includes CT and PT class, CB transients,
\end{flushleft}


\begin{flushleft}
CB rating and testing, Arc extinction in CB), Over current relays:
\end{flushleft}


\begin{flushleft}
Principle, operation and setting, Directional relays : needs and
\end{flushleft}


\begin{flushleft}
operating principle, Power system components protected using over
\end{flushleft}


\begin{flushleft}
current relays, Differential relays: Principle, operation and setting,
\end{flushleft}


\begin{flushleft}
Protection of three phase transformer, bus bar and generator
\end{flushleft}


\begin{flushleft}
using differential relays, Distance relays : Principle, operation and
\end{flushleft}


\begin{flushleft}
setting, Simple impedance relay, reactance relay, Mho relay and
\end{flushleft}


\begin{flushleft}
angle impedance relays, Quadrilateral relays, Transmission line
\end{flushleft}


\begin{flushleft}
protection using distance relays, Static relays: principle, amplitude
\end{flushleft}


\begin{flushleft}
comparator and phase comparator, Phase comparator realization
\end{flushleft}


\begin{flushleft}
using positive coincidence period, Distance relay realization using
\end{flushleft}


\begin{flushleft}
comparators, Generator protection, Overview of Numerical relaying
\end{flushleft}


\begin{flushleft}
and few algorithms, Phasor extraction , Introduction to PMU and
\end{flushleft}


\begin{flushleft}
its use, Fault location.
\end{flushleft}





\begin{flushleft}
ELL401 Advanced Electromechanics
\end{flushleft}


\begin{flushleft}
3 Credits (3-0-0)
\end{flushleft}


\begin{flushleft}
Pre-requisites: ELL203
\end{flushleft}


\begin{flushleft}
Introduction to Advancement in Electromechanics, Permanent Magnet
\end{flushleft}


\begin{flushleft}
Brushless DC Machines, Permanent Magnet Synchronous Motors,
\end{flushleft}


\begin{flushleft}
Switched Reluctance Motors, Single-Phase Machines, Axial Field
\end{flushleft}


\begin{flushleft}
Machines and other Advanced Electrical Machines, Introduction to
\end{flushleft}


\begin{flushleft}
Control of Advanced Electrical Machines, Applications in Industry,
\end{flushleft}


\begin{flushleft}
Domestic Appliances, Electric Mobility, etc., Computer Aided Simulation
\end{flushleft}


\begin{flushleft}
and Design of Advanced Electrical Machines, Case Studies.
\end{flushleft}





\begin{flushleft}
ELL402 Computer Communication
\end{flushleft}


\begin{flushleft}
3 Credits (3-0-0)
\end{flushleft}


\begin{flushleft}
Pre-requisites: MEL250
\end{flushleft}


\begin{flushleft}
Overlaps with: ELL785, CSL374, ELL473
\end{flushleft}


\begin{flushleft}
(i) Introduction, network structure: Basic networking concepts,
\end{flushleft}


\begin{flushleft}
Motivations for layered network concepts, Network examples; (ii)
\end{flushleft}


\begin{flushleft}
OSI reference model: Layering concepts, Overview of different layer
\end{flushleft}


\begin{flushleft}
functionalities; (iii) TCP/IP: Layering concepts, Layered functionalities,
\end{flushleft}


\begin{flushleft}
packet formats, fragmentation, Different layer protocols and examples:
\end{flushleft}


\begin{flushleft}
ARP, ICMP, etc., Congestion and error control; (iv) Network examples
\end{flushleft}


\begin{flushleft}
and functionalities: Ethernet, hub, bridge, switch, WANs, MANs, LANs,
\end{flushleft}


\begin{flushleft}
PANs, BANs; (v) Basic network protocol analysis: Performance metrics,
\end{flushleft}


\begin{flushleft}
Queueing models; (vi) Multiaccess protocols: Need for multiaccess
\end{flushleft}


\begin{flushleft}
protocols, Contention-free access schemes, Contention-based
\end{flushleft}


\begin{flushleft}
protocols: ALOHA, CSMA; (vii) Routing in data networks: Basic graph
\end{flushleft}


\begin{flushleft}
theoretic concepts, spanning tree, Shortest path routing, distance
\end{flushleft}





\begin{flushleft}
ELL405 Operating Systems
\end{flushleft}


\begin{flushleft}
3 Credits (3-0-0)
\end{flushleft}


\begin{flushleft}
Pre-requisites: ELL305
\end{flushleft}


\begin{flushleft}
Overlaps with: ELL602, CSL373, MAL358, ELL358
\end{flushleft}


\begin{flushleft}
Introduction to OS; Process and Thread management; Scheduling;
\end{flushleft}


\begin{flushleft}
Concurren threads and processes: mutual exclusion, synchronization,
\end{flushleft}


\begin{flushleft}
inter-process communication; Memory management: Cache and
\end{flushleft}


\begin{flushleft}
Virtual Memorymanagement; Resource management: deadlock and
\end{flushleft}


\begin{flushleft}
its prevention; File management; I/O management; Introduction to
\end{flushleft}


\begin{flushleft}
real time systems; Elements distributed operating systems.
\end{flushleft}





\begin{flushleft}
ELL406 Robotics and Automation
\end{flushleft}


\begin{flushleft}
3 Credits (3-0-0)
\end{flushleft}


\begin{flushleft}
Pre-requisites: ELL225
\end{flushleft}


\begin{flushleft}
Introduction to robotics. Basic components of robotic systems.
\end{flushleft}


\begin{flushleft}
Coordinate Transformation, D-H parameters. Forward and inverse
\end{flushleft}


\begin{flushleft}
kinematics. Velocity kinematics and Jacobian, Singularity analysis,
\end{flushleft}


\begin{flushleft}
Robot Dynamics : Holonomic and Non-Holonomic Systems. Trajectory
\end{flushleft}


\begin{flushleft}
planning. Robot control: linear and nonlinear. Actuators and Sensors.
\end{flushleft}


\begin{flushleft}
Vision based Robotic Control. Mobile Robots : Modeling and Control.
\end{flushleft}





\begin{flushleft}
ELL408 Low power circuit design
\end{flushleft}


\begin{flushleft}
3 Credits (3-0-0)
\end{flushleft}


\begin{flushleft}
Pre-requisites: ELL211
\end{flushleft}


\begin{flushleft}
MOS Transistors, MOS Inverters, Static CMOS Circuits, MOS Dynamic
\end{flushleft}


\begin{flushleft}
Circuits, Pass Transistor Logic Circuits, MOS Memories, Finite State
\end{flushleft}


\begin{flushleft}
Machines, Switching Power Dissipation, Dynamic Power Dissipation,
\end{flushleft}


\begin{flushleft}
Leakage Power Dissipation, Supply Voltage Scaling, Minimizing
\end{flushleft}


\begin{flushleft}
Switched Capacitance Minimizing Leakage Power, Variation Tolerant
\end{flushleft}


\begin{flushleft}
Design, Battery-Driven System Design.
\end{flushleft}





\begin{flushleft}
ELL409 Machine Intelligence and Learning
\end{flushleft}


\begin{flushleft}
4 Credits (3-0-2)
\end{flushleft}


\begin{flushleft}
Pre-requisites: MTL106, COL106
\end{flushleft}


\begin{flushleft}
Overlaps with: ELL784, ELL789, CSL333/CSL671, CSL341/
\end{flushleft}


\begin{flushleft}
COL774, MAL803
\end{flushleft}


\begin{flushleft}
Introduction to machine intelligence and intelligent agents; problem
\end{flushleft}


\begin{flushleft}
solving; knowledge representation and reasoning (logical and
\end{flushleft}


\begin{flushleft}
probabilistic); need for learning; basics of machine learning; Decision
\end{flushleft}


\begin{flushleft}
Trees; Rule-based models; linear learning models; Support Vector
\end{flushleft}


\begin{flushleft}
Machines; Artificial Neural Networks; Deep Learning; Probabilistic
\end{flushleft}


\begin{flushleft}
Modelling; Naive Bayes; Reinforcement Learning; Clustering;
\end{flushleft}


\begin{flushleft}
Feature Selection; Principal Component Analysis; Combining models;
\end{flushleft}


\begin{flushleft}
Philosophical issues in intelligence and learning. Substantive
\end{flushleft}


\begin{flushleft}
implementation assignments or a term project involving design of an
\end{flushleft}


\begin{flushleft}
intelligent learning-based system.
\end{flushleft}





\begin{flushleft}
ELL410 Multicore Systems
\end{flushleft}


\begin{flushleft}
3 Credits (3-0-0)
\end{flushleft}


\begin{flushleft}
Motivation for muticores; Multithreading; Flynn's taxonomy; Stream
\end{flushleft}


\begin{flushleft}
processing (vectVLIW, GPU).Message passing; Shared memory;
\end{flushleft}


\begin{flushleft}
Cache coherence in multiprocessor Synchronisation; Interconnection
\end{flushleft}


\begin{flushleft}
networks; Benchmarks and advanced topics; Project.
\end{flushleft}





\begin{flushleft}
ELD411 B.Tech. Project-I
\end{flushleft}


\begin{flushleft}
3 Credits (0-0-6)
\end{flushleft}


\begin{flushleft}
ELL411 Digital Communications
\end{flushleft}


\begin{flushleft}
4 Credits (3-0-2)
\end{flushleft}


\begin{flushleft}
Pre-requisites: ELL311
\end{flushleft}


\begin{flushleft}
Overlaps with: ELL762
\end{flushleft}


\begin{flushleft}
Matched Filter, Error Rate due to Noise. Intersymbol Interference,
\end{flushleft}


\begin{flushleft}
Nyquist's Criterion, Duobinary Signaling. Optimum Linear Receiver,
\end{flushleft}


\begin{flushleft}
Geometric Representation of Signals. Coherent Detection of Signals
\end{flushleft}


\begin{flushleft}
in Noise, Probability of Error. Coherent Digital Modulation Schemes:
\end{flushleft}


\begin{flushleft}
MPSK, MFSK, MQAM; Error Analysis. Noncoherent FSK, Differential
\end{flushleft}





196





\begin{flushleft}
\newpage
Electrical Engineering
\end{flushleft}





\begin{flushleft}
PSK. Comparison of Digital Modulation Schemes, Bandwidth Efficiency.
\end{flushleft}


\begin{flushleft}
Pseudo-Noise Sequences and Spread Spectrum, Trellis coded
\end{flushleft}


\begin{flushleft}
modulation, Digital signaling over fading multipath channels, OFDM
\end{flushleft}


\begin{flushleft}
communications systems.
\end{flushleft}





\begin{flushleft}
ELP411 Digital Communications Laboratory
\end{flushleft}


\begin{flushleft}
1 Credit (0-0-2)
\end{flushleft}


\begin{flushleft}
Pre-requisites: ELP311
\end{flushleft}





\begin{flushleft}
Differential relays: Principle, operation and setting, Protection of three
\end{flushleft}


\begin{flushleft}
phase transformer, bus bar and generator using differential relays.
\end{flushleft}


\begin{flushleft}
Distance relays: Principle, operation and setting, Simple impedance
\end{flushleft}


\begin{flushleft}
relay, reactance relay, Mho relay and angle impedance relays,
\end{flushleft}


\begin{flushleft}
Quadrilateral relays, Transmission line protection using distance relays.
\end{flushleft}


\begin{flushleft}
Static relays: principle, amplitude comparator and phase comparator,
\end{flushleft}


\begin{flushleft}
Phase comparator realization using positive coincidence period,
\end{flushleft}


\begin{flushleft}
Distance relay realization using comparators.
\end{flushleft}


\begin{flushleft}
Generator protection.
\end{flushleft}





\begin{flushleft}
ELL417 Renewable Energy Systems
\end{flushleft}


\begin{flushleft}
3 Credits (3-0-0)
\end{flushleft}


\begin{flushleft}
Pre-requisites: ELL203
\end{flushleft}





\begin{flushleft}
Overview of Numerical relaying and few algorithms, Phasor extraction,
\end{flushleft}


\begin{flushleft}
Introduction to PMU and its use, Fault location.
\end{flushleft}





\begin{flushleft}
Modeling of wind resource, aerodynamic characteristics, wind energy
\end{flushleft}


\begin{flushleft}
generators -- steady-state and dynamic modeling, electrical and pitch
\end{flushleft}


\begin{flushleft}
controller design, effect of induction generators on grid operation, solar
\end{flushleft}


\begin{flushleft}
Photovoltaic systems -- steady state and dynamic modeling, MPPT
\end{flushleft}


\begin{flushleft}
operation, power electronic systems for solar PV, fuel cells.
\end{flushleft}





\begin{flushleft}
ELD431 B.Tech. Project-I
\end{flushleft}


\begin{flushleft}
3 Credits (0-0-6)
\end{flushleft}





\begin{flushleft}
ELD450 BTP Part-II
\end{flushleft}


\begin{flushleft}
8 Credits (0-0-16)
\end{flushleft}


\begin{flushleft}
ELL450 Selected Topics in AE--I
\end{flushleft}


\begin{flushleft}
3 Credits (3-0-0)
\end{flushleft}


\begin{flushleft}
ELD451 BTP Part-II
\end{flushleft}


\begin{flushleft}
8 Credits (0-0-16)
\end{flushleft}





\begin{flushleft}
ELL431 Power System Optimization
\end{flushleft}


\begin{flushleft}
3 Credits (3-0-0)
\end{flushleft}


\begin{flushleft}
Pre-requisites: ELL303
\end{flushleft}


\begin{flushleft}
Characteristic of Generation units, Economic dispatch of thermal
\end{flushleft}


\begin{flushleft}
plants, Unit commitment, Hydrothermal coordination, Maintenance
\end{flushleft}


\begin{flushleft}
scheduling, Emission minimization, Optimal Power flow, Security
\end{flushleft}


\begin{flushleft}
constrained optimization, Optimization of distribution networks,
\end{flushleft}


\begin{flushleft}
Optimization in Power Markets.
\end{flushleft}





\begin{flushleft}
ELV451 Special Modules in SG\&RE--I
\end{flushleft}


\begin{flushleft}
1 Credit (1-0-0)
\end{flushleft}


\begin{flushleft}
Pre-requisites: to be decided by the instructor
\end{flushleft}


\begin{flushleft}
ELD452 BTP Part-II
\end{flushleft}


\begin{flushleft}
8 Credits (0-0-16)
\end{flushleft}


\begin{flushleft}
ELL452 Special Topics in EET-I
\end{flushleft}


\begin{flushleft}
3 Credits (3-0-0)
\end{flushleft}





\begin{flushleft}
ELL433 CAD of Power Electronics Systems
\end{flushleft}


\begin{flushleft}
4 Credits (3-0-2)
\end{flushleft}


\begin{flushleft}
Pre-requisites: ELP302
\end{flushleft}


\begin{flushleft}
Introduction to Power Electronic systems, Mathematical modeling
\end{flushleft}


\begin{flushleft}
of power electronic systems, State-space modeling, Average model,
\end{flushleft}


\begin{flushleft}
Circuit averaging model, Canonical circuit model, small-signal models
\end{flushleft}


\begin{flushleft}
and circuit transfer functions. Introduction to power electronics
\end{flushleft}


\begin{flushleft}
simulators, system oriented simulators, circuit simulators, merits
\end{flushleft}


\begin{flushleft}
and limitations. Introduction to magnetic design, high frequency
\end{flushleft}


\begin{flushleft}
inductor and transformer design. Hands-on exercise problems on
\end{flushleft}


\begin{flushleft}
power electronic circuits simulation using PSPICE/ SIMULINK/PSIM
\end{flushleft}


\begin{flushleft}
simulators.
\end{flushleft}





\begin{flushleft}
ELL436 Digital control
\end{flushleft}


\begin{flushleft}
3 Credits (3-0-0)
\end{flushleft}


\begin{flushleft}
Pre-requisites: ELL225
\end{flushleft}





\begin{flushleft}
ELD453 BTP Part-II
\end{flushleft}


\begin{flushleft}
8 Credits (0-0-16)
\end{flushleft}


\begin{flushleft}
ELL453 Power System Dynamics and Control
\end{flushleft}


\begin{flushleft}
3 Credits (3-0-0)
\end{flushleft}


\begin{flushleft}
Characteristic of Generation units, Economic dispatch of thermal
\end{flushleft}


\begin{flushleft}
plants, Unit commitment, Hydrothermal coordination, Maintenance
\end{flushleft}


\begin{flushleft}
scheduling, Emission minimization, Optimal Power flow, Security
\end{flushleft}


\begin{flushleft}
constrained optimization, Optimization of distribution networks,
\end{flushleft}


\begin{flushleft}
Optimization in Power Markets.
\end{flushleft}





\begin{flushleft}
ELD454 BTP Part-II
\end{flushleft}


\begin{flushleft}
8 Credits (0-0-16)
\end{flushleft}





\begin{flushleft}
ELL437 Switched Mode Power Conversion
\end{flushleft}


\begin{flushleft}
3 Credits (3-0-0)
\end{flushleft}


\begin{flushleft}
Pre-requisites: ELL231
\end{flushleft}





\begin{flushleft}
ELL454 Special Topics in ET--I
\end{flushleft}


\begin{flushleft}
3 Credits (3-0-0)
\end{flushleft}





\begin{flushleft}
To give an introduction about the power switching devices such as
\end{flushleft}


\begin{flushleft}
thyristors, GTO, MOSFETS, BJT, IGBT and MCTS. Basic concept of gate
\end{flushleft}


\begin{flushleft}
drivers (Trigger techniques, optical isolators, protection circuits, and
\end{flushleft}


\begin{flushleft}
isolation transformers), snubber design and protection schemes of
\end{flushleft}


\begin{flushleft}
power devices are to be discussed. Basic circuit configurations, design
\end{flushleft}


\begin{flushleft}
and analysis of choppers (step-up, step-down, step-up/down and
\end{flushleft}


\begin{flushleft}
multi-phase choppers), DC-DC converters (non-isolated and isolated),
\end{flushleft}


\begin{flushleft}
inverters (voltage and current source configurations) are discussed.
\end{flushleft}


\begin{flushleft}
This is followed by improved power quality converters (non-isolated
\end{flushleft}


\begin{flushleft}
and isolated) for reduction of harmonics at AC mains.
\end{flushleft}





\begin{flushleft}
ELL440 Power Systems Protection
\end{flushleft}


\begin{flushleft}
Fundamentals of Power system protection, philosophy of protective
\end{flushleft}


\begin{flushleft}
relays, Different types of relays, Introduction to protection elements
\end{flushleft}


\begin{flushleft}
like CT, PT, CB, Isolator etc, (includes CT and PT class, CB transients,
\end{flushleft}


\begin{flushleft}
CB rating and testing, Arc extinction in CB).
\end{flushleft}


\begin{flushleft}
Over current relays: Principle, operation and setting, Directional relays:
\end{flushleft}


\begin{flushleft}
needs and operating principle, Power system components protected
\end{flushleft}


\begin{flushleft}
using over current relays.
\end{flushleft}





\begin{flushleft}
ELD455 BTP Part-II
\end{flushleft}


\begin{flushleft}
8 Credits (0-0-16)
\end{flushleft}


\begin{flushleft}
ELL455 Special Topics in V\&ES--I
\end{flushleft}


\begin{flushleft}
3 Credits (3-0-0)
\end{flushleft}


\begin{flushleft}
Pre-requisites: to be decided by instructor
\end{flushleft}


\begin{flushleft}
ELD456 BTP Part-II
\end{flushleft}


\begin{flushleft}
8 Credits (0-0-16)
\end{flushleft}


\begin{flushleft}
ELL456 Special Topics in NE\&PS--I
\end{flushleft}


\begin{flushleft}
3 Credits (3-0-0)
\end{flushleft}


\begin{flushleft}
Pre-requisites: to be decided by the instructor
\end{flushleft}


\begin{flushleft}
ELD457 BTP Part II
\end{flushleft}


\begin{flushleft}
8 Credits (0-0-16)
\end{flushleft}





197





\begin{flushleft}
\newpage
Electrical Engineering
\end{flushleft}





\begin{flushleft}
ELL457 Special Topics in C\&IS--I
\end{flushleft}


\begin{flushleft}
3 Credits (3-0-0)
\end{flushleft}





\begin{flushleft}
ELL458 Special Topics in CS\&N--I
\end{flushleft}


\begin{flushleft}
3 Credits (3-0-0)
\end{flushleft}





\begin{flushleft}
Necessary conditions for optimal control, Pontryagin's minimum
\end{flushleft}


\begin{flushleft}
principle and state inequality constraints, Minimum time problems,
\end{flushleft}


\begin{flushleft}
Minimum control effort problems, Linear quadratic regulator problems,
\end{flushleft}


\begin{flushleft}
Riccati Equation, Singular intervals in optimal control problems, The
\end{flushleft}


\begin{flushleft}
principle of optimality, Application of the principle of optimality to
\end{flushleft}


\begin{flushleft}
decision making, Dynamic programming applied to routing problems,
\end{flushleft}


\begin{flushleft}
Solving optimal control problems using dynamic programming,
\end{flushleft}


\begin{flushleft}
Discrete linear regulator problem, Hamilton -Jacobi -Bellman Equation,
\end{flushleft}


\begin{flushleft}
Numerical Techniques to determine optimal trajectories.
\end{flushleft}





\begin{flushleft}
ELD459 BTP Part-II
\end{flushleft}


\begin{flushleft}
8 Credits (0-0-16)
\end{flushleft}


\begin{flushleft}
Pre-requisites: ELD411, ELD431
\end{flushleft}





\begin{flushleft}
ELL703 Optimal Control Theory
\end{flushleft}


\begin{flushleft}
3 credits (3-0-0)
\end{flushleft}


\begin{flushleft}
Pre-requisites: ELL700 or ELL333
\end{flushleft}





\begin{flushleft}
ELD458 BTP Part-II
\end{flushleft}


\begin{flushleft}
8 Credits (0-0-16)
\end{flushleft}





\begin{flushleft}
Maximization of functionals of a single and several functions using
\end{flushleft}


\begin{flushleft}
calculus of variations, Constrained extremals, Euler-Lagrange Equation,
\end{flushleft}


\begin{flushleft}
Necessary conditions for optimal control, Pontryagin's minimum
\end{flushleft}


\begin{flushleft}
principle and state inequality constraints, Minimum time problems,
\end{flushleft}


\begin{flushleft}
Minimum control effort problems, Linear quadratic regulator problems,
\end{flushleft}


\begin{flushleft}
Riccati Equation, Singular intervals in optimal control problems, The
\end{flushleft}


\begin{flushleft}
principle of optimality, Application of the principle of optimality to
\end{flushleft}


\begin{flushleft}
decision making, Dynamic programming applied to routing problems,
\end{flushleft}


\begin{flushleft}
Solving optimal control problems using dynamic programming,
\end{flushleft}


\begin{flushleft}
Discrete linear regulator problem, Hamilton -Jacobi -Bellman Equation,
\end{flushleft}


\begin{flushleft}
Numerical Techniques to determine optimal trajectories.
\end{flushleft}





\begin{flushleft}
ELL459 Special Topics in IP--I
\end{flushleft}


\begin{flushleft}
3 Credits (3-0-0)
\end{flushleft}


\begin{flushleft}
Pre-requisites: to be decided by the instructor
\end{flushleft}


\begin{flushleft}
ELL460 Special Topics in IP--II
\end{flushleft}


\begin{flushleft}
3 Credits (3-0-0)
\end{flushleft}


\begin{flushleft}
Pre-requisites: to be decided by the instructor
\end{flushleft}


\begin{flushleft}
ELL700 Linear Systems Theory
\end{flushleft}


\begin{flushleft}
3 Credits (3-0-0)
\end{flushleft}


\begin{flushleft}
Pre-requisites: ELL225 or equivalent
\end{flushleft}


\begin{flushleft}
Review of matrix algebra, state variable modelling of continuous
\end{flushleft}


\begin{flushleft}
and discrete time systems, linearization of state equations, solution
\end{flushleft}


\begin{flushleft}
of state equations of linear time-invariant and timevarying systems,
\end{flushleft}


\begin{flushleft}
Controllability and observability of dynamical systems, Minimal
\end{flushleft}


\begin{flushleft}
realization of linear systems and canonical forms, Liapunov's stability
\end{flushleft}


\begin{flushleft}
theory for linear dynamical systems, State Feedback controllers,
\end{flushleft}


\begin{flushleft}
Observer and Controller design.
\end{flushleft}





\begin{flushleft}
ELV700 Special Module in Systems and Control
\end{flushleft}


\begin{flushleft}
1 Credit (1-0-0)
\end{flushleft}


\begin{flushleft}
Pre-requisites: to be decided by the instructor
\end{flushleft}





\begin{flushleft}
ELL704 Advanced Robotics
\end{flushleft}


\begin{flushleft}
3 Credits (3-0-0)
\end{flushleft}


\begin{flushleft}
Pre-requisites: ELL225
\end{flushleft}


\begin{flushleft}
Review of Coordinate Transformations, D-H parameters and kinematics.
\end{flushleft}


\begin{flushleft}
Velocity kinematics and Jacobian, Singularity analysis, Robot Dynamics.
\end{flushleft}


\begin{flushleft}
Motion planning, Robot control: linear methods -- feedforward control,
\end{flushleft}


\begin{flushleft}
state feedback, observers; Nonlinear Control methods -- Computed
\end{flushleft}


\begin{flushleft}
Torque Control, Feedback linearization, Sliding Mode control; Vision
\end{flushleft}


\begin{flushleft}
based Robotic Control. Holonomic and Non-Holonomic Systems,
\end{flushleft}


\begin{flushleft}
Mobile Robots : Modeling and Control, Odometry Analysis, Navigation
\end{flushleft}


\begin{flushleft}
problems with obstacle avoidance, motion capturing systems.
\end{flushleft}





\begin{flushleft}
ELL705 Stochastic Filtering and Identification
\end{flushleft}


\begin{flushleft}
3 Credits (3-0-0)
\end{flushleft}


\begin{flushleft}
Pre-requisites: ELL701 or ELL333
\end{flushleft}





\begin{flushleft}
To provide exposure in specialized topics in systems and control.
\end{flushleft}





\begin{flushleft}
ELL701 Mathematical Methods in Control
\end{flushleft}


\begin{flushleft}
3 Credits (3-0-0)
\end{flushleft}


\begin{flushleft}
Linear Spaces -- Vectors and Matrices, Transformations, Norms - Vector
\end{flushleft}


\begin{flushleft}
and Matrix norms, Matrix factorization, Eigenvalues and Eigenvectors
\end{flushleft}


\begin{flushleft}
and Applications, Singular Value Decomposition and its Applications,
\end{flushleft}


\begin{flushleft}
Projections, Least Square Solutions. Probability, Random Variables,
\end{flushleft}


\begin{flushleft}
Probability distribution and density functions, Joint density and
\end{flushleft}


\begin{flushleft}
Conditional distribution, Functions of random variables, Moments,
\end{flushleft}


\begin{flushleft}
characteristic functions, sequence of random variables, Correlation
\end{flushleft}


\begin{flushleft}
matrices and their properties, Random processes and their properties,
\end{flushleft}


\begin{flushleft}
Response of Linear systems to stochastic inputs, PSD theorem.
\end{flushleft}





\begin{flushleft}
ELL702 Nonlinear Systems
\end{flushleft}


\begin{flushleft}
3 Credits (3-0-0)
\end{flushleft}


\begin{flushleft}
Pre-requisites: ELL225 or equivalent
\end{flushleft}


\begin{flushleft}
Introduction to nonlinear systems: Examples of phenomena, models
\end{flushleft}


\begin{flushleft}
\& derivation of system equations. Fundamental properties: Existence
\end{flushleft}


\begin{flushleft}
\& uniqueness, Dependence on initial conditions \& parameters.
\end{flushleft}


\begin{flushleft}
Phase plane analysis. Limit cycles \& oscillations. Describing function
\end{flushleft}


\begin{flushleft}
method and applications. Circle criterion. Lyapunov stability of
\end{flushleft}


\begin{flushleft}
autonomous systems. Perturbation theory \& Averaging. Singular
\end{flushleft}


\begin{flushleft}
perturbation model and stability analysis. Basic results on Lie algebra.
\end{flushleft}


\begin{flushleft}
Controllability and Observability of nonlinear systems. Bifurcations.
\end{flushleft}


\begin{flushleft}
Chaos. Synchronization.
\end{flushleft}





\begin{flushleft}
ELL703 Optimal Control Theory
\end{flushleft}


\begin{flushleft}
3 Credits (3-0-0)
\end{flushleft}


\begin{flushleft}
Pre-requisites: ELL700 or ELL333
\end{flushleft}


\begin{flushleft}
Maximization of functionals of a single and several functions using
\end{flushleft}


\begin{flushleft}
calculus of variations, Constrained extremals, Euler-Lagrange Equation,
\end{flushleft}





\begin{flushleft}
MMSE estimation including LMS, Gaussian case. Wiener filtering \&
\end{flushleft}


\begin{flushleft}
prediction. Kalman filtering \& prediction. Extended Kalman filtering.
\end{flushleft}


\begin{flushleft}
Predictors for difference equation based models including ARMA, Box
\end{flushleft}


\begin{flushleft}
Jenkins \& others. Statistical properties of Least Squares estimation
\end{flushleft}


\begin{flushleft}
and its relationship with Bayes estimation (ML, MAP), convergence
\end{flushleft}


\begin{flushleft}
analysis, CR bound. Recursive Least Squares, Iterative methods for
\end{flushleft}


\begin{flushleft}
nonlinear Least Squares. Identification problem: Different approaches
\end{flushleft}


\begin{flushleft}
for linear dynamical systems. Offline identification methods including
\end{flushleft}


\begin{flushleft}
Least Squares, Prediction error framework, Pseudo-linear regression
\end{flushleft}


\begin{flushleft}
(PLR) \& Instrument variable methods. Recursive Identification of
\end{flushleft}


\begin{flushleft}
linear dynamical system: RLS, PLR, Prediction error framework \&
\end{flushleft}


\begin{flushleft}
its application to ARMA \& Innovations representation. Convergence
\end{flushleft}


\begin{flushleft}
Analysis of Recursive Identification methods: Associated ODE,
\end{flushleft}


\begin{flushleft}
Martingale. Nonlinear system identification. Subspace based method of
\end{flushleft}


\begin{flushleft}
system identification. Applications including LQG and adaptive control.
\end{flushleft}





\begin{flushleft}
ELL707 Systems Biology
\end{flushleft}


\begin{flushleft}
3 Credits (3-0-0)
\end{flushleft}


\begin{flushleft}
Pre-requisites: ELL225
\end{flushleft}


\begin{flushleft}
MODELS : Variables and parameters, Law of mass action,
\end{flushleft}


\begin{flushleft}
Representations : Deterministic vs stochastic, Spatial aspects,
\end{flushleft}


\begin{flushleft}
Examples of core processes: Gene expression, Protein degradation,
\end{flushleft}


\begin{flushleft}
Phosphorylation.
\end{flushleft}


\begin{flushleft}
DYNAMICS : Equilibrium solutions, Bifurcations, Switches, Bistability,
\end{flushleft}


\begin{flushleft}
Pulses and Oscillations, Circadian Rhythms and Clocks, Spatial
\end{flushleft}


\begin{flushleft}
patterns. Morphogenesis and Development.
\end{flushleft}


\begin{flushleft}
CONTROL MECHANISMS : Performance Goals, Integral Feedback
\end{flushleft}


\begin{flushleft}
Control, Homeostasis and Perfect Adaptation, Bacterial Chemotaxis,
\end{flushleft}


\begin{flushleft}
Feedforward Loops, Fold Change Detection, Robustness to
\end{flushleft}


\begin{flushleft}
Perturbations, Tradeoffs, Internal Model Principle.
\end{flushleft}





198





\begin{flushleft}
\newpage
Electrical Engineering
\end{flushleft}





\begin{flushleft}
ELL708 Selected Topics in Systems and Control
\end{flushleft}


\begin{flushleft}
3 Credits (3-0-0)
\end{flushleft}


\begin{flushleft}
Pre-requisites: to be decided by the instructor
\end{flushleft}


\begin{flushleft}
To be decided by the Instructor when floating this course: It can be
\end{flushleft}


\begin{flushleft}
anything that is related to systems and control engineering, but is
\end{flushleft}


\begin{flushleft}
not covered in any of the established courses.
\end{flushleft}





\begin{flushleft}
ELL709 Design Aspects in Control
\end{flushleft}


\begin{flushleft}
3 Credits (3-0-0)
\end{flushleft}


\begin{flushleft}
System Modeling -- model structures (Process model, ARX model),
\end{flushleft}


\begin{flushleft}
Review of concepts of stability, feedback and feedforward control.
\end{flushleft}


\begin{flushleft}
Classical control -- First-Order Plus Dead-Time model (FOPDT), process
\end{flushleft}


\begin{flushleft}
reaction curves, Second-Order Plus Dead-Time model (SOPDT), relay
\end{flushleft}


\begin{flushleft}
feedback process identification; Smith Predictor and its variations, PID
\end{flushleft}


\begin{flushleft}
controllers and their tuning, Ziegler-Nichols and Cohen-Coon techniques.
\end{flushleft}


\begin{flushleft}
Reliable State Feedback design -- pole placement, eigenstructure
\end{flushleft}


\begin{flushleft}
assignment, region based eigenvalue assignment, eigenstructure-time
\end{flushleft}


\begin{flushleft}
response relationships. Controller gain selection -- noise sensitivity.
\end{flushleft}


\begin{flushleft}
Controller robustness. Disturbance rejection. Frequency Domain Loop
\end{flushleft}


\begin{flushleft}
Shaping. Output feedback control -- compensator design, review of
\end{flushleft}


\begin{flushleft}
Lead, Lag and Lag-Lead compensators, Zero dynamics -- significance
\end{flushleft}


\begin{flushleft}
in servo control design, design for unstable zero dynamics. Observers
\end{flushleft}


\begin{flushleft}
-- concept and design philosophy. Applications in practical controller
\end{flushleft}


\begin{flushleft}
design scenarios.
\end{flushleft}





\begin{flushleft}
ELL710 Coding Theory
\end{flushleft}


\begin{flushleft}
3 Credits (3-0-0)
\end{flushleft}


\begin{flushleft}
Measure of information, Source coding, Communication channel
\end{flushleft}


\begin{flushleft}
models, Channel Capacity and coding, Linear Block codes, Low Density
\end{flushleft}


\begin{flushleft}
Parity Check (LDPC) Codes, Bounds on minimum distance, Cyclic codes,
\end{flushleft}


\begin{flushleft}
BCH codes, Reed Solomon Codes, Convolutional codes, Trellis coded
\end{flushleft}


\begin{flushleft}
Modulation, Viterbi decoding, Turbo codes, Introduction to Space-Time
\end{flushleft}


\begin{flushleft}
Codes and Introduction to Cryptography. If time permits, LDPC/Turbo
\end{flushleft}


\begin{flushleft}
codes in the wireless standards. There are no laboratory or design
\end{flushleft}


\begin{flushleft}
activities involved with this course.
\end{flushleft}





\begin{flushleft}
ELV710 Special Module in Cyber Security
\end{flushleft}


\begin{flushleft}
1 Credit (1-0-0)
\end{flushleft}


\begin{flushleft}
Overview of cyber security, computer security and the associated
\end{flushleft}


\begin{flushleft}
threat, attack, adversary models, access control, intrusion detection,
\end{flushleft}


\begin{flushleft}
basic network security, security of cyber physical systems and a brief
\end{flushleft}


\begin{flushleft}
introduction to cryptography.
\end{flushleft}





\begin{flushleft}
ELL711 Signal Theory
\end{flushleft}


\begin{flushleft}
3 Credits (3-0-0)
\end{flushleft}


\begin{flushleft}
Pre-requisites: ELL105, ELL311
\end{flushleft}





\begin{flushleft}
ELL713 Microwave Theory and Techniques
\end{flushleft}


\begin{flushleft}
3 Credits (3-0-0)
\end{flushleft}


\begin{flushleft}
Pre-requisites: ELL212
\end{flushleft}


\begin{flushleft}
Overlaps with: CRL711
\end{flushleft}


\begin{flushleft}
Review of EM theory: Maxwell's equations, plane waves in dielectric
\end{flushleft}


\begin{flushleft}
and conducting media, energy and power. Transmission lines and
\end{flushleft}


\begin{flushleft}
waveguides: closed and dielectric guides, planar transmission lines and
\end{flushleft}


\begin{flushleft}
optical fibre. Network analysis: scattering matrix other parameters,
\end{flushleft}


\begin{flushleft}
signal flow graphs and network representation. Impedance matching
\end{flushleft}


\begin{flushleft}
and tuning. Analysis of planar transmission lines. Analysis of design
\end{flushleft}


\begin{flushleft}
of passive components.
\end{flushleft}





\begin{flushleft}
ELL714 Basic Information Theory
\end{flushleft}


\begin{flushleft}
3 Credits (3-0-0)
\end{flushleft}


\begin{flushleft}
Pre-requisites: ELL105
\end{flushleft}


\begin{flushleft}
Introduction to entropy, relative entropy, mutual information,
\end{flushleft}


\begin{flushleft}
fundamental inequalities like Jensen's inequality and log sum inequality.
\end{flushleft}


\begin{flushleft}
Proof of asymptotic equipartition property and its usage in data
\end{flushleft}


\begin{flushleft}
compression. Study of entropy rates of the stochastic process following
\end{flushleft}


\begin{flushleft}
Markov chains. Study of data compression: Kraft inequality and
\end{flushleft}


\begin{flushleft}
optimal source coding. Channel capacity: symmetric channels, channel
\end{flushleft}


\begin{flushleft}
coding theorem, Fano's inequality, feedback capacity. Differential
\end{flushleft}


\begin{flushleft}
entropy. The Gaussian channel: bandlimited channels, channels with
\end{flushleft}


\begin{flushleft}
colored noise, Gaussian channels with feedback. Detailed study of
\end{flushleft}


\begin{flushleft}
the rate-distortion theory: rate distortion function, strongly typical
\end{flushleft}


\begin{flushleft}
sequences, computation of channel capacity. Joint source channel
\end{flushleft}


\begin{flushleft}
coding/separation theorem. There are no laboratory or design activities
\end{flushleft}


\begin{flushleft}
involved with this course.
\end{flushleft}





\begin{flushleft}
ELL715 Digital Image Processing
\end{flushleft}


\begin{flushleft}
4 Credits (3-0-2)
\end{flushleft}


\begin{flushleft}
Introduction to 2-D Signals and Systems. Image Digitization. Image
\end{flushleft}


\begin{flushleft}
Transforms. Image Enhancement: Image Restoration: Inverse
\end{flushleft}


\begin{flushleft}
Filtering, Algebraic Approach to Restoration, Wiener (LMS) approach,
\end{flushleft}


\begin{flushleft}
Constrained Least Squares Restoration, Adaptive methods for
\end{flushleft}


\begin{flushleft}
restoration. Image Reconstruction: The Filtered Back- Projection
\end{flushleft}


\begin{flushleft}
Algorithm, Algebraic Reconstruction Method. Image Segmentation:
\end{flushleft}


\begin{flushleft}
Detection of Discontinuities, Edge Linking and Boundary Detection,
\end{flushleft}


\begin{flushleft}
Thresholding, Region-Oriented Segmentation. Object representation
\end{flushleft}


\begin{flushleft}
and description: Boundary descriptors, region descriptors, HOG and
\end{flushleft}


\begin{flushleft}
SIFT based features. Colour Image processing: colour models, colour
\end{flushleft}


\begin{flushleft}
transformations, and processing techniques.
\end{flushleft}





\begin{flushleft}
ELL716 Telecommunication Switching and Transmission
\end{flushleft}


\begin{flushleft}
3 Credits (3-0-0)
\end{flushleft}





\begin{flushleft}
Discrete random variables (Bernoulli, binomial, Poisson, geometric,
\end{flushleft}


\begin{flushleft}
negative binomial, etc.) and their properties like PDF, CDF, MGF.
\end{flushleft}


\begin{flushleft}
Continuous random variables: Gaussian, multivariate Gaussian;
\end{flushleft}


\begin{flushleft}
whitening of the Gaussian random vector; complex Gaussian random
\end{flushleft}


\begin{flushleft}
vector, circularity; Rayleigh and Rician; exponential; chi-squared; gamma.
\end{flushleft}


\begin{flushleft}
Signal spaces: convergence and continuity; linear spaces, inner product
\end{flushleft}


\begin{flushleft}
spaces; basis, Gram-Scmidt orthogonalization.
\end{flushleft}


\begin{flushleft}
Stochastic convergence, law of large numbers, central limit theorem.
\end{flushleft}


\begin{flushleft}
Random processes: stationarity; mean, correlation, and covariance
\end{flushleft}


\begin{flushleft}
functions, WSS random process; autocorrelation and cross-correlation
\end{flushleft}


\begin{flushleft}
functions; transmission of a random process through a linear filter;
\end{flushleft}


\begin{flushleft}
power spectral density; white random process; Gaussian process;
\end{flushleft}


\begin{flushleft}
Poisson process.
\end{flushleft}





\begin{flushleft}
ELL712 Digital Communications
\end{flushleft}


\begin{flushleft}
3 Credits (3-0-0)
\end{flushleft}


\begin{flushleft}
Review of random variables and random process, signal space
\end{flushleft}


\begin{flushleft}
concepts, Common modulated signals and their power spectral
\end{flushleft}


\begin{flushleft}
densities, Optimum receivers for Gaussian channels, Coherent and
\end{flushleft}


\begin{flushleft}
non-cohrerent receivers and their performance (evaluating BER
\end{flushleft}


\begin{flushleft}
performance through software tools), Basics of Information theory,
\end{flushleft}


\begin{flushleft}
source and channel coding, capacity of channels, band-limited
\end{flushleft}


\begin{flushleft}
channels and ISI, multicarrier and spread-spectrum signaling, multiple
\end{flushleft}


\begin{flushleft}
access techniques.
\end{flushleft}





\begin{flushleft}
Wireline access circuits, long haul circuits, signaling, switching exchanges,
\end{flushleft}


\begin{flushleft}
analysis of telecom switching networks, teletraffic engineering,
\end{flushleft}


\begin{flushleft}
management protocols, multi-service telecom protocols and networks.
\end{flushleft}





\begin{flushleft}
ELL717 Optical Communication Systems
\end{flushleft}


\begin{flushleft}
3 Credits (3-0-0)
\end{flushleft}


\begin{flushleft}
Pre-requisites: ELL769
\end{flushleft}


\begin{flushleft}
The fiber channel with its linear and nonlinear characteristics, LED
\end{flushleft}


\begin{flushleft}
and Laser diode transmitter design, PIN and APD receiver design,
\end{flushleft}


\begin{flushleft}
Modulation schemes, Source and line coding in optical systems. Optical
\end{flushleft}


\begin{flushleft}
Link design with dispersion and power budgeting. Design of digital
\end{flushleft}


\begin{flushleft}
and analog communication systems. Optical amplifiers, WDM system
\end{flushleft}


\begin{flushleft}
design. Hybrid fiber co-axial/microwave links.
\end{flushleft}





\begin{flushleft}
ELL718 Statistical Signal Processing
\end{flushleft}


\begin{flushleft}
3 Credits (3-0-0)
\end{flushleft}


\begin{flushleft}
Pre-requisites: ELL711
\end{flushleft}


\begin{flushleft}
Review of random variables, GS orthogonalization, geometric
\end{flushleft}


\begin{flushleft}
concepts, notions of projection, random processes, WSS processes,
\end{flushleft}


\begin{flushleft}
properties of autocorrelation and power spectral densities, properties
\end{flushleft}


\begin{flushleft}
of autocorrelation matrices, Cholesky decomposition, eigen-analysis,
\end{flushleft}


\begin{flushleft}
optimum Linear filtering, LMS and its performance, variants, Leastsquares, QR decomposition and SVD, RLS and its performance,
\end{flushleft}


\begin{flushleft}
square-root RLS, Kalman Filters, spectrum modelling.
\end{flushleft}





199





\begin{flushleft}
\newpage
Electrical Engineering
\end{flushleft}





\begin{flushleft}
ELP718 Telecommunication Software Laboratory
\end{flushleft}


\begin{flushleft}
3 Credits (0-1-4)
\end{flushleft}


\begin{flushleft}
Contents: CASE tools, object-oriented program development, use
\end{flushleft}


\begin{flushleft}
of telecom network simulator, implementation using C/C++/Java,
\end{flushleft}


\begin{flushleft}
network management software design, V.5 test and simulation.
\end{flushleft}





\begin{flushleft}
ELL719 Detection and Estimation Theory
\end{flushleft}


\begin{flushleft}
3 Credits (3-0-0)
\end{flushleft}


\begin{flushleft}
Pre-requisites: ELL711
\end{flushleft}


\begin{flushleft}
Overview of the course, Classical Decision Theory: Binary hypothesis
\end{flushleft}


\begin{flushleft}
testing: Bayes criterion, Neyman-Pearson criterion, min-max test,
\end{flushleft}


\begin{flushleft}
M-ary hypothesis testing: General rule, minimum probability of error
\end{flushleft}


\begin{flushleft}
decision rule, Gaussian case and associated geometric concepts,
\end{flushleft}


\begin{flushleft}
Erasure decision problem, Random parameter estimation. Non --
\end{flushleft}


\begin{flushleft}
random parameter estimation: CRLB for nonrandom parameters,
\end{flushleft}


\begin{flushleft}
ML estimation rule, asymptotic properties of ML estimates. Linear
\end{flushleft}


\begin{flushleft}
minimum variance estimation, Least squares methods CRLB for random
\end{flushleft}


\begin{flushleft}
parameter estimation, condition for statistical efficiency, Multiple
\end{flushleft}


\begin{flushleft}
parameter estimation, Composite and non-parametric hypothesis
\end{flushleft}


\begin{flushleft}
testing, Applications, Detection of signals.
\end{flushleft}


\begin{flushleft}
Mathematical preliminaries: K-L expansion and its application to
\end{flushleft}


\begin{flushleft}
Detection of known and un-known (i.e. with unknown, parameters)
\end{flushleft}


\begin{flushleft}
signals in AWGN., Detection of signals in colored noise. Linear
\end{flushleft}


\begin{flushleft}
estimation, Wiener filters and solution of Wiener HopfEquations,KalmanBucyfilters, Miscellaneous estimation techniques.
\end{flushleft}





\begin{flushleft}
ELP719 Microwave Laboratory
\end{flushleft}


\begin{flushleft}
3 Credits (0-1-4)
\end{flushleft}


\begin{flushleft}
Design, fabrication and testing of simple linear microwave circuits
\end{flushleft}


\begin{flushleft}
using microstrip technology.
\end{flushleft}





\begin{flushleft}
ELL720 Advanced Digital Signal Processing
\end{flushleft}


\begin{flushleft}
3 Credits (3-0-0)
\end{flushleft}


\begin{flushleft}
Pre-requisites: ELL205
\end{flushleft}


\begin{flushleft}
Review of Signals and Systems, Sampling and data reconstruction
\end{flushleft}


\begin{flushleft}
processes.
\end{flushleft}


\begin{flushleft}
Z transforms.
\end{flushleft}


\begin{flushleft}
Discrete linear systems.
\end{flushleft}


\begin{flushleft}
Frequency domain design of digital filters.
\end{flushleft}


\begin{flushleft}
Quantization effects in digital filters.
\end{flushleft}


\begin{flushleft}
Discrete Fourier transform and FFT algorithms.
\end{flushleft}


\begin{flushleft}
High speed convolution and its application to digital filtering.
\end{flushleft}


\begin{flushleft}
Introduction to Multirate signal processing, Multirate filtering
\end{flushleft}


\begin{flushleft}
and Filterbanks: including Polyphase decomposition and perfect
\end{flushleft}


\begin{flushleft}
reconstruction, Cyclostationarity and LPTV filters, Introduction to
\end{flushleft}


\begin{flushleft}
Wavelet Transform.
\end{flushleft}


\begin{flushleft}
The self-study component will consist of design problems in the above
\end{flushleft}


\begin{flushleft}
to be implemented on MATLAB.
\end{flushleft}





\begin{flushleft}
ELP720 Telecommunication Networks Laboratory
\end{flushleft}


\begin{flushleft}
3 Credits (0-1-4)
\end{flushleft}


\begin{flushleft}
Contents: Development of network elements such as routers, SNMP
\end{flushleft}


\begin{flushleft}
nodes. Use of laboratory and telecom field test instruments such
\end{flushleft}


\begin{flushleft}
as: oscilloscopes, oscillators, RMS meters, transmission impairment
\end{flushleft}


\begin{flushleft}
measuring systems, return loss meters, etc. Enables students to study
\end{flushleft}


\begin{flushleft}
voice and data switching functions and to measure transmission and
\end{flushleft}


\begin{flushleft}
traffic characteristics on models of the major business communication
\end{flushleft}


\begin{flushleft}
systems and carrier transmission facilities (controlled LAN
\end{flushleft}


\begin{flushleft}
environments, Ethernet, E1, T1/T3lines). Experimental procedures
\end{flushleft}


\begin{flushleft}
include the use of frequency and time division multiplex systems and
\end{flushleft}


\begin{flushleft}
the modulation techniques employed by in such systems and the
\end{flushleft}


\begin{flushleft}
observation of noise and distortion effects.
\end{flushleft}





\begin{flushleft}
ELV720 Special Module in Communication Systems and
\end{flushleft}


\begin{flushleft}
Networking-I
\end{flushleft}


\begin{flushleft}
1 Credit (1-0-0)
\end{flushleft}





\begin{flushleft}
ELL721 Introduction to Telecommunication Systems
\end{flushleft}


\begin{flushleft}
3 Credits (3-0-0)
\end{flushleft}


\begin{flushleft}
Pre-requisites: Only for MBA students of Bharti School, Audit for
\end{flushleft}


\begin{flushleft}
others
\end{flushleft}


\begin{flushleft}
Fundamentals of signals, signal transmission and media, modulation
\end{flushleft}


\begin{flushleft}
techniques, equalization, amplification, crosstalk, attenuation,
\end{flushleft}


\begin{flushleft}
switching principles, telephony, signaling, transmission systems-DSL,
\end{flushleft}


\begin{flushleft}
optical, radio.
\end{flushleft}





\begin{flushleft}
ELP721 Embedded Telecommunication Systems
\end{flushleft}


\begin{flushleft}
Laboratory
\end{flushleft}


\begin{flushleft}
3 Credits (0-1-4)
\end{flushleft}


\begin{flushleft}
ELL722 Antenna Theory and Techniques
\end{flushleft}


\begin{flushleft}
3 Credits (3-0-0)
\end{flushleft}


\begin{flushleft}
Review of electromagnetism and vector calculus, history and context
\end{flushleft}


\begin{flushleft}
of antenna theory, operation of various antenna types, such as
\end{flushleft}


\begin{flushleft}
dipole, linear, loop, and resonant type, characterization of antenna
\end{flushleft}


\begin{flushleft}
performance metrics, and introduction to numerical techniques for
\end{flushleft}


\begin{flushleft}
visualizing antenna radiation patterns.
\end{flushleft}





\begin{flushleft}
ELL723 Broadband Communication Systems
\end{flushleft}


\begin{flushleft}
3 Credits (3-0-0)
\end{flushleft}


\begin{flushleft}
Multiple Access Techniques -- CSMA, Spread Spectrum (SS), Direct
\end{flushleft}


\begin{flushleft}
Spread SS, Frequency Hopping SS and CDMA, Timing Synchronization,
\end{flushleft}


\begin{flushleft}
Delay Lock Loop, ISDN Physical Layer, ISDN Data Link Layer, Signaling
\end{flushleft}


\begin{flushleft}
System Number 7, Broadband ISDN Protocols, ATM Switch and
\end{flushleft}


\begin{flushleft}
Protocols, CLOS Network Switch, OFDM Concept, OFDMA System,
\end{flushleft}


\begin{flushleft}
Multi-Carrier CDMA, WiMAX.
\end{flushleft}





\begin{flushleft}
ELL724 Computational Electromagnetics
\end{flushleft}


\begin{flushleft}
3 Credits (3-0-0)
\end{flushleft}


\begin{flushleft}
Capacity of single-user Gaussian multi-antenna deterministic channels
\end{flushleft}


\begin{flushleft}
and optimal strategies. Reliable transmission in single user state
\end{flushleft}


\begin{flushleft}
dependent channels. Capacity of Gaussian single-antenna fading
\end{flushleft}


\begin{flushleft}
channels with state (RX CSI, Full CSI). Capacity of single-antenna
\end{flushleft}


\begin{flushleft}
frequency-selective fading channels (OFDM modulation, waterfilling
\end{flushleft}


\begin{flushleft}
across frequency). Capacity of Gaussian multi-antenna single user
\end{flushleft}


\begin{flushleft}
fading channels (RX CSI only, Full CSI). Spatial multiplexing gain, array
\end{flushleft}


\begin{flushleft}
gain. Transmitter and receiver architectures, V-BLAST transmission,
\end{flushleft}


\begin{flushleft}
Zero-Forcing receiver, MMSE receiver, MMSE-SIC receiver. Optimality
\end{flushleft}


\begin{flushleft}
of MMSE-SIC.
\end{flushleft}


\begin{flushleft}
Capacity region of the multi-user Gaussian MAC channel. Capacity
\end{flushleft}


\begin{flushleft}
region of the multi-user Gaussian Broadcast channel (BC) with singleantenna terminals. Capacity of state dependent channels with noncausal side information (Gelfand-Pinsker coding). Dirty paper coding to
\end{flushleft}


\begin{flushleft}
pre-cancel known interference. MAC-BC duality. Capacity region of the
\end{flushleft}


\begin{flushleft}
multi-user Gaussian Broadcast channel with multi-antenna terminals
\end{flushleft}


\begin{flushleft}
(Dirty paper coding achieves the capacity region). Capacity region of
\end{flushleft}


\begin{flushleft}
the Interference channel. There are no laboratory or design activities
\end{flushleft}


\begin{flushleft}
involved in this course.
\end{flushleft}





\begin{flushleft}
ELL725 Wireless Communications
\end{flushleft}


\begin{flushleft}
3 Credits (3-0-0)
\end{flushleft}


\begin{flushleft}
Pre-requisites: ELL712
\end{flushleft}


\begin{flushleft}
The wireless channel (physical modeling, linear time-varying system,
\end{flushleft}


\begin{flushleft}
discrete-time baseband model, time and frequency coherence), pointto-point communication (detection, diversity, spatial multiplexing),
\end{flushleft}


\begin{flushleft}
cellular systems (multiple access and interference management),
\end{flushleft}


\begin{flushleft}
capacity of point-to-point wireless channels (single and multi-antenna),
\end{flushleft}


\begin{flushleft}
capacity of single-antenna multiuser channels, point-to-point multiantenna (MIMO) channels and spatial multiplexing, point-to-point
\end{flushleft}


\begin{flushleft}
MIMO capacity and multiplexing architectures.
\end{flushleft}





\begin{flushleft}
ELP725 Wireless Communication Laboratory
\end{flushleft}


\begin{flushleft}
3 Credits (0-1-4)
\end{flushleft}


\begin{flushleft}
ELL726 Nano-Photonics and Plasmonics
\end{flushleft}


\begin{flushleft}
3 Credits (3-0-0)
\end{flushleft}


\begin{flushleft}
EM Waves, Maxwell's Equations, Boundary Conditions, Drude, Debye,
\end{flushleft}





200





\begin{flushleft}
\newpage
Electrical Engineering
\end{flushleft}





\begin{flushleft}
Lorentz-Drude Dispersion Relation Models, Introduction to Surface
\end{flushleft}


\begin{flushleft}
Plasmons, Surface Plasmon Excitation Mechanisms, Plasmonic
\end{flushleft}


\begin{flushleft}
Nanogratings, Localized Surface Plasmon based Devices, Optical and
\end{flushleft}


\begin{flushleft}
Plasmonic Interconnects, Sensors based on Surface Plasmons, SERS
\end{flushleft}


\begin{flushleft}
based sensing, Photonic Crystals, Optical Metamaterials, Fabrication
\end{flushleft}


\begin{flushleft}
of Nanomaterials and Plasmonic Devices.
\end{flushleft}





\begin{flushleft}
ELV731 Special Modules in NE\&PS -- I
\end{flushleft}


\begin{flushleft}
1 Credit (1-0-0)
\end{flushleft}


\begin{flushleft}
Pre-requisites: to be decided by the instructor
\end{flushleft}





\begin{flushleft}
ELL727 Digital Communication and Information
\end{flushleft}


\begin{flushleft}
Systems
\end{flushleft}


\begin{flushleft}
3 Credits (3-0-0)
\end{flushleft}





\begin{flushleft}
Technology basics and digital logic families such as static CMOS, pass
\end{flushleft}


\begin{flushleft}
transistor, transmission gate, dynamic and domino logic. Advanced
\end{flushleft}


\begin{flushleft}
sequential logic elements with latch-based design and timing and
\end{flushleft}


\begin{flushleft}
clocking concepts. Power and delay of digital circuits. Physical and
\end{flushleft}


\begin{flushleft}
logical synthesis for ASICs and FPGAs. Verilog and VHDL with design
\end{flushleft}


\begin{flushleft}
examples. Design for testability with fault models.
\end{flushleft}





\begin{flushleft}
Review of Fourier Transforms, Sampling Theorem, Quantization,
\end{flushleft}


\begin{flushleft}
Pulse Code Modulation, Digital Modulation Schemes -- BPSK, QPSK,
\end{flushleft}


\begin{flushleft}
BFSK, QASK, MPSK,Random Processes, Probability density function,
\end{flushleft}


\begin{flushleft}
Gaussian density function, Frequency domain representation of
\end{flushleft}


\begin{flushleft}
noise, Spectral components of noise, Noise bandwidth, Properties
\end{flushleft}


\begin{flushleft}
of noise, Noise Performance Analysis of digital modulation schemes.
\end{flushleft}


\begin{flushleft}
Information Theory, Concept of information, Coding to increase
\end{flushleft}


\begin{flushleft}
average information per bit, Shannon's theorem, Capacity of Gaussian
\end{flushleft}


\begin{flushleft}
Channel, Bandwidth-S/N tradeoff. Discrete memory-less channel
\end{flushleft}


\begin{flushleft}
capacity. Error correcting codes, Block codes, Cyclic redundancy check,
\end{flushleft}


\begin{flushleft}
Coding gain, Bit error rate calculations.
\end{flushleft}





\begin{flushleft}
ELL728 Optoelectronic Instrumentation
\end{flushleft}


\begin{flushleft}
3 Credits (3-0-0)
\end{flushleft}


\begin{flushleft}
Introduction to test and measuring instruments, instrumentation
\end{flushleft}


\begin{flushleft}
amplifier, chopper stabilized amplifier, analog signal processing:
\end{flushleft}


\begin{flushleft}
active filter, A/D, D/A converters, integrated, transimpedance and low
\end{flushleft}


\begin{flushleft}
impedance pre-amplifiers design, sample \& hold circuits, multiplexer,
\end{flushleft}


\begin{flushleft}
peak detector, zero crossing detector etc., digital design: PALs, FPGA,
\end{flushleft}


\begin{flushleft}
signal analyzer: superheterodyne spectrum analyzer, DFT and FFT
\end{flushleft}


\begin{flushleft}
analyzer, digital filters and computer interface, microcontrollers:
\end{flushleft}


\begin{flushleft}
introduction to microcontroller and applications such as 8031, optical
\end{flushleft}


\begin{flushleft}
post, in-line and pre-amplifiers, noise figure, optoelectronic circuits:
\end{flushleft}


\begin{flushleft}
transmitter and receiver design, OTDR, optical spectrum analyzer,
\end{flushleft}


\begin{flushleft}
sensors: fiber optic and radiation types, distributed sensors, fiber
\end{flushleft}


\begin{flushleft}
optic smart structure, display devices.
\end{flushleft}





\begin{flushleft}
ELL730 I.C. Technology
\end{flushleft}


\begin{flushleft}
3 Credits (3-0-0)
\end{flushleft}


\begin{flushleft}
Course Introduction, Modern Semiconductor IC fabrication Industrial/
\end{flushleft}


\begin{flushleft}
Academic Landscape; Overview of modern CMOS process flow --
\end{flushleft}


\begin{flushleft}
basic steps; Crystal growth and wafer basics; Cleanroom basics --
\end{flushleft}


\begin{flushleft}
environment, infrastructure, advanced MOS cleaning, getering etc.
\end{flushleft}


\begin{flushleft}
Lithography; Oxidation; Diffusion; Ion-Implantation; Thin-Film
\end{flushleft}


\begin{flushleft}
Deposition; Etching; Backend processes; Process Simulation- tools,
\end{flushleft}


\begin{flushleft}
techniques and methods; Advanced device fabrication concepts -- I
\end{flushleft}


\begin{flushleft}
(SOI, FDSOI, etc); Advanced device fabrication concepts -- II (organic,
\end{flushleft}


\begin{flushleft}
PV, hetero); Advanced device fabrication concepts -- III (CNTs, Selfassembly etc).
\end{flushleft}





\begin{flushleft}
ELV730 Special Modules in V\&ES -- I
\end{flushleft}


\begin{flushleft}
1 Credit (1-0-0)
\end{flushleft}


\begin{flushleft}
Pre-requisites: to be decide by instructor
\end{flushleft}





\begin{flushleft}
ELL732 Micro and Nanoelectronics
\end{flushleft}


\begin{flushleft}
3 Credits (3-0-0)
\end{flushleft}





\begin{flushleft}
ELL733 Digital ASIC Design
\end{flushleft}


\begin{flushleft}
4 Credits (3-0-2)
\end{flushleft}


\begin{flushleft}
Pre-requisites: ELL308
\end{flushleft}


\begin{flushleft}
Overlaps with: CSL316
\end{flushleft}


\begin{flushleft}
Review of working of MOSFET, large signal and small signal models,
\end{flushleft}


\begin{flushleft}
biasing schemes, analysis and design of various single stage amplifier
\end{flushleft}


\begin{flushleft}
configuration, Noise and distortion analysis, Mismatch and nonlinearity, low and high frequency analysis of single stage amplifiers,
\end{flushleft}


\begin{flushleft}
frequency compensation, current mirrors and reference circuits,
\end{flushleft}


\begin{flushleft}
multistage amplifiers; differential and operational amplifiers, negative
\end{flushleft}


\begin{flushleft}
and positive feedback, oscillators and power amplifiers.
\end{flushleft}





\begin{flushleft}
ELL734 MOS VLSI design
\end{flushleft}


\begin{flushleft}
3 Credits (3-0-0)
\end{flushleft}


\begin{flushleft}
Overlaps with: ELL329,ELL324
\end{flushleft}


\begin{flushleft}
Digital integrated circuit design perspective. Basic static and dynamic
\end{flushleft}


\begin{flushleft}
MOS logic families. Sequential Circuits. Power dissipation and delay in
\end{flushleft}


\begin{flushleft}
circuits. Arithmetic Building blocks, ALU. Timing Issues in synchronous
\end{flushleft}


\begin{flushleft}
design. Interconnect Parasitics.
\end{flushleft}





\begin{flushleft}
ELV734 Special Module in Scientific Writing for
\end{flushleft}


\begin{flushleft}
Research
\end{flushleft}


\begin{flushleft}
1 Credit (1-0-0)
\end{flushleft}


\begin{flushleft}
Tools needed for scientific writing, ethics of publication, plagiarism,
\end{flushleft}


\begin{flushleft}
attribution, copyrights, writing impactful papers, writing theses, writing
\end{flushleft}


\begin{flushleft}
a technical disclosure or patent.
\end{flushleft}





\begin{flushleft}
ELL735 Analog Integrated Circuits
\end{flushleft}


\begin{flushleft}
3 Credits (3-0-0)
\end{flushleft}


\begin{flushleft}
Pre-requisites: ELL204
\end{flushleft}


\begin{flushleft}
Introduction to MOSFETs, Single stage amplifiers, Biasing circuits,
\end{flushleft}


\begin{flushleft}
Voltage and Current reference circuits, Feedback analysis, Multistage
\end{flushleft}


\begin{flushleft}
amplifiers, Mismatch and noise analysis, Differential amplifiers, High
\end{flushleft}


\begin{flushleft}
speed and low noise amplifiers, Output stage amplifiers, Oscillators.
\end{flushleft}





\begin{flushleft}
ELL736 Solid State Imaging Sensors
\end{flushleft}


\begin{flushleft}
3 Credits (3-0-0)
\end{flushleft}


\begin{flushleft}
Pre-requisites: ELL204, ELL782
\end{flushleft}





\begin{flushleft}
ELL731 Mixed Signal Circuit Design
\end{flushleft}


\begin{flushleft}
3 Credits (3-0-0)
\end{flushleft}


\begin{flushleft}
Pre-requisites: ELL782
\end{flushleft}


\begin{flushleft}
Switched capacitor circuit principles and applications in filter design;
\end{flushleft}


\begin{flushleft}
issues of clock feed through, charge injection and other non-idealities;
\end{flushleft}


\begin{flushleft}
design of switches; data converters: characteristics, static and
\end{flushleft}


\begin{flushleft}
dynamic; types of ADCs; track and hold, and sample and hold circuits;
\end{flushleft}


\begin{flushleft}
comparators; flash ADCs; pipelined ADCs; successive approximation
\end{flushleft}


\begin{flushleft}
register type ADCs; discrete-time and continuous time delta-sigma
\end{flushleft}


\begin{flushleft}
ADCs; higher order delta-sigma design; MASH structure; multi-bit
\end{flushleft}


\begin{flushleft}
delta-sigmas; decimation filtering -- sinc and comb filters; digital to
\end{flushleft}


\begin{flushleft}
analog conversion; voltage-based DACs; charge-based DACs; currentbased DACs -- binary and thermometer currents; linearizing techniques
\end{flushleft}


\begin{flushleft}
for DACs; delta-sigma DACs; interpolation filtering; phase-locked loop
\end{flushleft}


\begin{flushleft}
basics; PLL dynamics; frequency synthesis; all-digital PLLs.
\end{flushleft}





\begin{flushleft}
Radiometry and Photometry (Light radiation, photometry, light source,
\end{flushleft}


\begin{flushleft}
light units), Introduction to properties of silicon and photon absorption,
\end{flushleft}


\begin{flushleft}
Imager formats, Basics of image sensors (fundamental definition of
\end{flushleft}


\begin{flushleft}
image sensors, pixels, photo-conversion principles, Charge coupled
\end{flushleft}


\begin{flushleft}
devices (operational principles, types and performance metrics),
\end{flushleft}


\begin{flushleft}
CMOS image sensors (operational principles, types and performance
\end{flushleft}


\begin{flushleft}
metrics), Noise, quantum efficiency, dynamic range and modulation
\end{flushleft}


\begin{flushleft}
transfer function analysis in image sensors, High speed image sensors,
\end{flushleft}


\begin{flushleft}
Back side illumination, Electron multiplication CCDs and CMOS, Colour
\end{flushleft}


\begin{flushleft}
detection in silicon, 3D imaging, machine vision cameras, polarization
\end{flushleft}


\begin{flushleft}
detection and scientific applications.
\end{flushleft}





\begin{flushleft}
ELP736 Physical Design Laboratory
\end{flushleft}


\begin{flushleft}
3 Credits (0-0-6)
\end{flushleft}





201





\begin{flushleft}
\newpage
Electrical Engineering
\end{flushleft}





\begin{flushleft}
ELL737 Flexible Electronics
\end{flushleft}


\begin{flushleft}
3 Credits (3-0-0)
\end{flushleft}


\begin{flushleft}
Pre-requisites: ELL218, ELL111, or ELL732 or equivalent
\end{flushleft}


\begin{flushleft}
Introduction to displays and lighting technologies, solar cells, and
\end{flushleft}


\begin{flushleft}
sensors. Flexible substrates. Low cost materials. Solution-processed
\end{flushleft}


\begin{flushleft}
fabrication methods. Printing methods. Flexible displays. Flat panel
\end{flushleft}


\begin{flushleft}
lighting. Flexible solar cells. Low-cost sensors.
\end{flushleft}





\begin{flushleft}
ELL738 Micro and Nano Photonics
\end{flushleft}


\begin{flushleft}
3 Credits (3-0-0)
\end{flushleft}


\begin{flushleft}
Pre-requisites: PYL100, ELL207
\end{flushleft}


\begin{flushleft}
Overlaps with: PYL795
\end{flushleft}


\begin{flushleft}
Ray Optics; Wave Optics: Plane Waves, Spherical Waves, Interference,
\end{flushleft}


\begin{flushleft}
Diffraction; Paraxial Waves; Beam Optics; Fabry Perot Cavity;
\end{flushleft}


\begin{flushleft}
Microresonators - Ring Resonators, Disc Resonators; Review of
\end{flushleft}


\begin{flushleft}
Electromagnetic (EM) Theory; Boundary Conditions; and some relevant
\end{flushleft}


\begin{flushleft}
EM problems; FDTD and FEM modeling; Fundamentals of Plasmonics
\end{flushleft}


\begin{flushleft}
- Surface Plasmon Resonance, Dispersion relation, Plasmon coupling
\end{flushleft}


\begin{flushleft}
conditions, Plasmonic gratings, Models describing the refractive
\end{flushleft}


\begin{flushleft}
index of metals; Localized Surface Plasmon Resonance; Plasmonic
\end{flushleft}


\begin{flushleft}
Sensors and Devices; Surface-enhanced Raman Scattering; Plasmonic
\end{flushleft}


\begin{flushleft}
waveguides and Interconnects; Photonic Crystals and Devices.
\end{flushleft}





\begin{flushleft}
ELL739 Advanced Semiconductor Devices
\end{flushleft}


\begin{flushleft}
3 Credits (3-0-0)
\end{flushleft}


\begin{flushleft}
Pre-requisites: ELL218, ELL111(UG), ELL732(PG)
\end{flushleft}


\begin{flushleft}
Solid state device physics, generation and recombination processes,
\end{flushleft}


\begin{flushleft}
radiation basics, density of states, gain and absorption, LEDs, OLEDs,
\end{flushleft}


\begin{flushleft}
heterojunction LEDs, lasers, population inversion, photodetectors,
\end{flushleft}


\begin{flushleft}
CCDs, image sensors, photocurrent, solar cells, efficiency measures,
\end{flushleft}


\begin{flushleft}
multijunction PVs, organic solar cells, economics, memory devices,
\end{flushleft}


\begin{flushleft}
sensors, MEMS devices.
\end{flushleft}





\begin{flushleft}
ELL740 Compact Modeling of Semiconductor Devices
\end{flushleft}


\begin{flushleft}
3 Credits (3-0-0)
\end{flushleft}


\begin{flushleft}
Pre-requisites: Any course on MOS devices or Microelectronics
\end{flushleft}


\begin{flushleft}
or Physical Electronics or VLSI technology
\end{flushleft}


\begin{flushleft}
Introduction to AMS enablement and PDK elements, Basics of
\end{flushleft}


\begin{flushleft}
semiconductor devices, Device modeling tools-TCAD and SPICE, Diode
\end{flushleft}


\begin{flushleft}
modeling, Resistor modeling, FEOL capacitor modeling, Advanced
\end{flushleft}


\begin{flushleft}
CMOS Technology, MOS transistor modeling, modeling of process
\end{flushleft}


\begin{flushleft}
variations, Mismatch and corners.
\end{flushleft}





\begin{flushleft}
ELL741 Neuromorphic Engineering
\end{flushleft}


\begin{flushleft}
3 Credits (3-0-0)
\end{flushleft}


\begin{flushleft}
Motivation and field Introduction, Emerging computing trends and
\end{flushleft}


\begin{flushleft}
roadmap, non-von Neumann computing approach; Basic Biology --
\end{flushleft}


\begin{flushleft}
1: Neuron, Synapse, Synaptic Plasticity; Basic Biology -2 : Learning
\end{flushleft}


\begin{flushleft}
rules, Retina, Cochlea, STDP; Mathematical/Electrical modeling
\end{flushleft}


\begin{flushleft}
of Neurons - LIF, IF, HH; Hardware Implementation of Neuron
\end{flushleft}


\begin{flushleft}
circuits -- VLSI Digital/Analog; Advanced Nanodevices for Neuron
\end{flushleft}


\begin{flushleft}
Implementation; Hardware Implementation of Synaptic and Learning
\end{flushleft}


\begin{flushleft}
circuits -- VLSI Digital/Analog; Advanced Nanodevices for Synaptic
\end{flushleft}


\begin{flushleft}
emulation -- 1 (NVM, Flash etc); Advanced Nanodevices for Synaptic
\end{flushleft}


\begin{flushleft}
emulation -- 2 (RRAM, memristors, CNT etc); Synaptic programming
\end{flushleft}


\begin{flushleft}
methodology optimization; Nanodevice specific bio-inspired learning
\end{flushleft}


\begin{flushleft}
rule optimization; Full Network design example -1: Visual Application;
\end{flushleft}


\begin{flushleft}
Full Network design example -2: Auditory Application; Full system
\end{flushleft}


\begin{flushleft}
level power/energy dissipation considerations and course conclusion.
\end{flushleft}





\begin{flushleft}
ELL742 Introduction to MEMS Design
\end{flushleft}


\begin{flushleft}
3 Credits (3-0-0)
\end{flushleft}


\begin{flushleft}
Overlaps with: CRL726
\end{flushleft}


\begin{flushleft}
This course is an introduction to the multi-disciplinary and rapidly
\end{flushleft}


\begin{flushleft}
growing area of MEMS. A MEMS design engineer requires knowledge
\end{flushleft}


\begin{flushleft}
of several domains --namely mechanical, electrical, fluidic and thermal,
\end{flushleft}


\begin{flushleft}
as well as knowledge of circuits and microfabrication techniques. This
\end{flushleft}


\begin{flushleft}
course will cover the fundamentals as applicable to MEMS, as well as
\end{flushleft}


\begin{flushleft}
several case studies to understand the design process.
\end{flushleft}





\begin{flushleft}
ELL743 Photovoltaics
\end{flushleft}


\begin{flushleft}
3 Credits (3-0-0)
\end{flushleft}


\begin{flushleft}
Pre-requisites: ELL218, ELL111(UG), ELL732(PG)
\end{flushleft}


\begin{flushleft}
Overlaps with: ELL739
\end{flushleft}


\begin{flushleft}
Solid state device physics, p-n and p-i-n junctions. Homojunctions and
\end{flushleft}


\begin{flushleft}
heterojunctions. Generation and recombination processes. Radiation
\end{flushleft}


\begin{flushleft}
basics. Photon absorption. Photovoltaic efficiency. Thin film fabrication
\end{flushleft}


\begin{flushleft}
processes. Silicon-based solar cells. III-V and chalcogenide-based solar
\end{flushleft}


\begin{flushleft}
cells. Multijunction architectures. Dye-sensitized solar cells. Organic
\end{flushleft}


\begin{flushleft}
solar cells. Plasmonic structures. Solar cell economics and policy.
\end{flushleft}





\begin{flushleft}
ELL744 Electronic and Photonic Nanomaterials
\end{flushleft}


\begin{flushleft}
3 Credits (3-0-0)
\end{flushleft}


\begin{flushleft}
Pre-requisites: PHL100
\end{flushleft}


\begin{flushleft}
Overlaps with: EPL444, PHL726
\end{flushleft}


\begin{flushleft}
1D, 2D and 3D confinement; Density of states; Excitons; Coulomb
\end{flushleft}


\begin{flushleft}
blockade; Optical properties of semiconducting nanoparticles:
\end{flushleft}


\begin{flushleft}
Fluorescence of semiconductor nanocrystals, core-shell nanocrystals,
\end{flushleft}


\begin{flushleft}
effect of nanocrystal size; Optical properties of metallic nanoparticles:
\end{flushleft}


\begin{flushleft}
Surface Plasmons, Localized Surface Plasmons, Surface-enhanced
\end{flushleft}


\begin{flushleft}
Raman scattering; Electronic Applications of Nanomaterials: Nanowire
\end{flushleft}


\begin{flushleft}
transistors, Memory Devices, Single electron devices, Biosensors;
\end{flushleft}


\begin{flushleft}
Optical Applications of Nanomaterials - Quantum well, wire, and
\end{flushleft}


\begin{flushleft}
dot Diodes, Lasers and Detectors, Chemical sensors, Gas sensors,
\end{flushleft}


\begin{flushleft}
Biosensors; Development of Electronic and Optical Nanomaterials:
\end{flushleft}


\begin{flushleft}
Epitaxial Growth, Deposition of Nanomaterials, Self-Assembly
\end{flushleft}


\begin{flushleft}
of Nanomaterials, Nanofabrication techniques; Characterization
\end{flushleft}


\begin{flushleft}
of Nanomaterials: Electron microscopic techniques (scanning
\end{flushleft}


\begin{flushleft}
and transmission), Atomic Force Microscopy, X-Ray Diffraction,
\end{flushleft}


\begin{flushleft}
Characterization of optical and electronic properties of nanomaterials.
\end{flushleft}





\begin{flushleft}
ELL745 Quantum Electronics
\end{flushleft}


\begin{flushleft}
3 Credits (3-0-0)
\end{flushleft}


\begin{flushleft}
Pre-requisites: ELL218, ELL111(UG), ELL732(PG)
\end{flushleft}


\begin{flushleft}
Overlaps with: ELL739
\end{flushleft}


\begin{flushleft}
Newtonian mechanics, wavepackets, brief history of quantum
\end{flushleft}


\begin{flushleft}
mechanics, blackbody radiation, photoelectric effect, wave-particle
\end{flushleft}


\begin{flushleft}
duality, second quantization, Semiconductor materials, crystal
\end{flushleft}


\begin{flushleft}
structure and defects, Bravais lattices, Brillouin zones, Miller
\end{flushleft}


\begin{flushleft}
indices, periodic potentials, Kronig-Penney model, bandstructure in
\end{flushleft}


\begin{flushleft}
bulk semiconductors, Bloch theorem, direct and indirect bandgap
\end{flushleft}


\begin{flushleft}
semiconductors, effective mass, effect of alloying, carrier statistics,
\end{flushleft}


\begin{flushleft}
superlattices and quantum wells, density of states in 0,1,2 and
\end{flushleft}


\begin{flushleft}
3 dimensions, bandstructure in lower dimensional systems,
\end{flushleft}


\begin{flushleft}
heterojunctions, effect of strain on bandstructure, excitonic effects
\end{flushleft}


\begin{flushleft}
in semiconductors, tunneling, perturbation theory, scattering and
\end{flushleft}


\begin{flushleft}
collisions, phonons, high-field transport, Boltzmann transport theory,
\end{flushleft}


\begin{flushleft}
spin transport, excitons, optical processes in semiconductors and
\end{flushleft}


\begin{flushleft}
quantum wells, absorption, gain, spontaneous and stimulated
\end{flushleft}


\begin{flushleft}
emission, fluorescence and phosphorescence, photophysics of organic
\end{flushleft}


\begin{flushleft}
molecules and polymers.
\end{flushleft}





\begin{flushleft}
ELL746 Biomedical Electronics
\end{flushleft}


\begin{flushleft}
3 Credits (3-0-0)
\end{flushleft}


\begin{flushleft}
Introduction to Biomedical Instrumentation: Constraints, Regulations
\end{flushleft}


\begin{flushleft}
and health economics, Basic sensors, amplifiers and signal processing,
\end{flushleft}


\begin{flushleft}
Origin of bio potentials and electrode systems, Bio potential amplifiers,
\end{flushleft}


\begin{flushleft}
sources of noise and their Remedies, Blood pressure and heart sound
\end{flushleft}


\begin{flushleft}
systems, Measurement of flow and volume of blood Measurement of
\end{flushleft}


\begin{flushleft}
respiratory system, Ultrasonography, CAT, PET and MRI overview,
\end{flushleft}


\begin{flushleft}
Fuzzy Logic and its application medical instruments, Embedded system
\end{flushleft}


\begin{flushleft}
in medical electronics with selection of one microprocessor and then
\end{flushleft}


\begin{flushleft}
design tips, Overview of pace maker, defibrllator, hemodialysis and
\end{flushleft}


\begin{flushleft}
infant incubators. Safety codes and standards, Electro-chemical sensor,
\end{flushleft}


\begin{flushleft}
Ion Selective FET, Immunologically sensitive FET, Spectrophotometry,
\end{flushleft}


\begin{flushleft}
Optical biosensors, Fibre-optic sensors, blood glucose sensor, smell
\end{flushleft}


\begin{flushleft}
sensor, SAW devices, Sensor neural network, Expert systems and case
\end{flushleft}


\begin{flushleft}
studies of design examples.
\end{flushleft}





202





\begin{flushleft}
\newpage
Electrical Engineering
\end{flushleft}





\begin{flushleft}
ELL747 Active and Passive Filter Design
\end{flushleft}


\begin{flushleft}
3 Credits (3-0-0)
\end{flushleft}


\begin{flushleft}
Pre-requisites: ELL112 or circuit theory
\end{flushleft}


\begin{flushleft}
Review of network theorems such as reciprocity, Tellegen's theorem,
\end{flushleft}


\begin{flushleft}
scattering parameters, properties of lossless passive networks;
\end{flushleft}


\begin{flushleft}
Butterworth approximation; Chebyshev approximation; synthesis
\end{flushleft}


\begin{flushleft}
of Butterworth and Chebyshev filters; odd versus even order filters;
\end{flushleft}


\begin{flushleft}
sensitivity of lossless LC ladder filters; frequency transformations;
\end{flushleft}


\begin{flushleft}
inverse Chebyshev and elliptic approximations; synthesis of
\end{flushleft}


\begin{flushleft}
inverse Chebyshev and elliptic filters; review of properties of p.r.
\end{flushleft}


\begin{flushleft}
functions; Darlington synthesis; signal flow graphs of ladder filters;
\end{flushleft}


\begin{flushleft}
opamp-RC implementation; Gm-C implementation; switchedcapacitor implementation; minimum required performance of active
\end{flushleft}


\begin{flushleft}
components; tuning of filters; transmission line based filters: using
\end{flushleft}


\begin{flushleft}
high-Z low-Z technique, using Kuroda's identities; bi-quad based design
\end{flushleft}


\begin{flushleft}
approaches and drawbacks; Tow-Thomas biquad, Sallen-Key biquad.
\end{flushleft}





\begin{flushleft}
ELL748 System-on-Chip Design and Test
\end{flushleft}


\begin{flushleft}
3 Credits (3-0-0)
\end{flushleft}


\begin{flushleft}
Pre-requisites: ELL201
\end{flushleft}


\begin{flushleft}
Overview and definition of power quality (PQ), Sources of pollution,
\end{flushleft}


\begin{flushleft}
International power quality standards, and regulations. Power quality
\end{flushleft}


\begin{flushleft}
monitoring
\end{flushleft}


\begin{flushleft}
Power quality problems. Loads which causes power quality problems.
\end{flushleft}


\begin{flushleft}
Power factor correction, zero voltage regulation, reactive power
\end{flushleft}


\begin{flushleft}
compensation, load balancing using load compensation techniques:
\end{flushleft}


\begin{flushleft}
passive shunt and series compensation, DSTATCOM (Distribution Static
\end{flushleft}


\begin{flushleft}
Compensators), DVR (Dynamic Voltage Restorers), UPQC (Universal
\end{flushleft}


\begin{flushleft}
Power Quality Conditioners).
\end{flushleft}


\begin{flushleft}
Harmonic effects-within the power system, interference with
\end{flushleft}


\begin{flushleft}
communication Harmonic measurements. Harmonic elimination-using
\end{flushleft}


\begin{flushleft}
active (shunt, series and hybrid) and passive (shunt and series) filters.
\end{flushleft}


\begin{flushleft}
Improved power quality converters: single ac-dc converters, bridgeless
\end{flushleft}


\begin{flushleft}
isolated converter, bridgeless non-isolated converters, multi-pulse
\end{flushleft}


\begin{flushleft}
converters, multilevel converters, line commutated converters, power
\end{flushleft}


\begin{flushleft}
quality improvement in SMPS, UPS, drives, welding systems, lighting
\end{flushleft}


\begin{flushleft}
systems, and renewable energy systems.
\end{flushleft}





\begin{flushleft}
ELL749 Semiconductor Memory Design
\end{flushleft}


\begin{flushleft}
3 Credits (3-0-0)
\end{flushleft}


\begin{flushleft}
Pre-requisites: ELL734
\end{flushleft}





\begin{flushleft}
Refrigerators, Air Conditioners, Mixer-Grinders/Food Processors,
\end{flushleft}


\begin{flushleft}
Ceiling and other types of Fans, Introduction to Industrial Appliances,
\end{flushleft}


\begin{flushleft}
Drives and Control of Industrial Appliances, Computer Aided Simulation
\end{flushleft}


\begin{flushleft}
and Design of Drives and Control of Appliances, Smart Appliances.
\end{flushleft}





\begin{flushleft}
ELL751 Power Electronic Converters
\end{flushleft}


\begin{flushleft}
3 Credits (3-0-0)
\end{flushleft}


\begin{flushleft}
Introduction to various power switching devices and their control,
\end{flushleft}


\begin{flushleft}
introducing various power electronic circuits for realization of ACDC, AC-AC, DC-AC, DC-DC conversion, principle of operation, and
\end{flushleft}


\begin{flushleft}
analysis, pulse-width modulation and pulse frequency control of
\end{flushleft}


\begin{flushleft}
power electronic converters, design problems on power electronic
\end{flushleft}


\begin{flushleft}
converter systems.
\end{flushleft}





\begin{flushleft}
ELL752 Electric Drive System
\end{flushleft}


\begin{flushleft}
3 Credits (3-0-0)
\end{flushleft}


\begin{flushleft}
Components of electric drive system- electrical machines, power
\end{flushleft}


\begin{flushleft}
converters and control system. Different types of loads encountered in
\end{flushleft}


\begin{flushleft}
modern drive applications. dynamics of drive systems, starting, braking,
\end{flushleft}


\begin{flushleft}
speed-control, steady-state and dynamic operation of motors, load
\end{flushleft}


\begin{flushleft}
variations, closed loop control of drives, phase controlled and chopper
\end{flushleft}


\begin{flushleft}
controlled dc drives, induction motor drives,synchronous motor drives,
\end{flushleft}


\begin{flushleft}
space phasor model, v/f control, direct and indirect vector control,
\end{flushleft}


\begin{flushleft}
direct torque control, PMSM drives, BLDC drive, drive controller design.
\end{flushleft}





\begin{flushleft}
ELV752 Special Modules in EET--I
\end{flushleft}


\begin{flushleft}
1 Credit (1-0-0)
\end{flushleft}


\begin{flushleft}
ELL753 Physical Phenomena in Electrical Machines
\end{flushleft}


\begin{flushleft}
3 Credits (3-0-0)
\end{flushleft}


\begin{flushleft}
Engineering and physical aspects of rotating machines. Modern
\end{flushleft}


\begin{flushleft}
machine windings. Winding analysis and mmf waveforms. Space
\end{flushleft}


\begin{flushleft}
and time harmonics. Saturation. Unbalanced magnetic pull and
\end{flushleft}


\begin{flushleft}
magnetic noise in industrial machines. Heating/Cooling. Unbalanced
\end{flushleft}


\begin{flushleft}
and asymmetrical operation of induction motors. Special phenomena
\end{flushleft}


\begin{flushleft}
in electrical machines such as capacitor self excitation of induction
\end{flushleft}


\begin{flushleft}
machines and its applications. Use of electromagnetic field theory,
\end{flushleft}


\begin{flushleft}
performance of permanent magnet machines. Magnetic levitation
\end{flushleft}


\begin{flushleft}
Superconductors and applications. Permanent magnet and Switched
\end{flushleft}


\begin{flushleft}
Reluctance Motors.
\end{flushleft}





\begin{flushleft}
ELV753 Special Modules in ET--I
\end{flushleft}


\begin{flushleft}
1 Credit (1-0-0)
\end{flushleft}





\begin{flushleft}
Introduction to Special Electrical Machines and Magnetic Devices,
\end{flushleft}


\begin{flushleft}
Permanent Magnet Machines, Permanent Magnet Brushless DC
\end{flushleft}


\begin{flushleft}
Machines, Stepper Motors, Hysteresis Motors, Switched Reluctance
\end{flushleft}


\begin{flushleft}
Motors, Hybrid Motors, Linear Machines, Magnetic Devices,
\end{flushleft}


\begin{flushleft}
Applications in Robotics, Industry Automation, Electric Vehicles,
\end{flushleft}


\begin{flushleft}
Aerospace and Defense Systems etc. Super conducting Machines and
\end{flushleft}


\begin{flushleft}
Other Advanced machines, Case Studies, Computer Aided Simulation
\end{flushleft}


\begin{flushleft}
and Design of Special Electrical Machines.
\end{flushleft}





\begin{flushleft}
ELL750 Modelling of Electrical Machines
\end{flushleft}


\begin{flushleft}
3 Credits (3-0-0)
\end{flushleft}


\begin{flushleft}
Pre-requisites: ELL203
\end{flushleft}


\begin{flushleft}
Review of dynamic Modeling of systems, Basic concepts of
\end{flushleft}


\begin{flushleft}
electromechanical energy conversion, Modeling of Transformer,
\end{flushleft}


\begin{flushleft}
Generalized Theory of Electrical machines, Modeling of DC Machine,
\end{flushleft}


\begin{flushleft}
Induction Machine, Wound Field Synchronous machine, and special
\end{flushleft}


\begin{flushleft}
machines such as BLDC, PMSM etc.
\end{flushleft}





\begin{flushleft}
ELV750 Special Modules in AE--I
\end{flushleft}


\begin{flushleft}
1 Credit (1-0-0)
\end{flushleft}


\begin{flushleft}
ELL751 Appliance Systems
\end{flushleft}


\begin{flushleft}
3 Credits (3-0-0)
\end{flushleft}


\begin{flushleft}
Pre-requisites: ELL203, ELL332 ELL365
\end{flushleft}


\begin{flushleft}
Introduction to Domestic Appliances, Embedded System Design issues,
\end{flushleft}


\begin{flushleft}
Ergonomic Design aspects, Review of Electrical Machines and Drives,
\end{flushleft}


\begin{flushleft}
Review of Embedded Systems, Drive and Control of Washing Machines,
\end{flushleft}





\begin{flushleft}
ELL754 Permanent Magnet Machines
\end{flushleft}


\begin{flushleft}
3 Credits (3-0-0)
\end{flushleft}


\begin{flushleft}
Introduction to Permanent Magnet Machines, Permanent Magnet DC
\end{flushleft}


\begin{flushleft}
Commutator Machines, Permanent Magnet Synchronous Machines,
\end{flushleft}


\begin{flushleft}
Permanent Magnet Brushless DC machines, Hysteresis motors, Stepper
\end{flushleft}


\begin{flushleft}
Motors. Moreover various applications of permanent magnet machines
\end{flushleft}


\begin{flushleft}
are also integral part of syllabus. Various upcoming applications in field
\end{flushleft}


\begin{flushleft}
of robotics, solar pumping, wind energy generation system and many
\end{flushleft}


\begin{flushleft}
more are covered in the syllabus. Computer aided simulation studies
\end{flushleft}


\begin{flushleft}
for modeling and performance analysis are also part of this course.
\end{flushleft}





\begin{flushleft}
ELL755 Variable Reluctance Machines
\end{flushleft}


\begin{flushleft}
3 Credits (3-0-0)
\end{flushleft}


\begin{flushleft}
The objective of this course is to enhance the knowledge of students
\end{flushleft}


\begin{flushleft}
with the design, modeling, construction, operation and control of
\end{flushleft}


\begin{flushleft}
variable reluctance machines including the hybrid motors and their
\end{flushleft}


\begin{flushleft}
existing applications. The contents distinguish the variable reluctance
\end{flushleft}


\begin{flushleft}
machines from various conventional machines. Study of possible
\end{flushleft}


\begin{flushleft}
replacement of conventional machines by the variable reluctance
\end{flushleft}


\begin{flushleft}
machines for specific applications are the secondary objective of
\end{flushleft}


\begin{flushleft}
offering this course.
\end{flushleft}





\begin{flushleft}
ELL756 Special Electrical Machines
\end{flushleft}


\begin{flushleft}
3 Credits (3-0-0)
\end{flushleft}


\begin{flushleft}
Pre-requisites: ELL103
\end{flushleft}


\begin{flushleft}
Introduction to Special Electrical Machines and Magnetic Devices,
\end{flushleft}


\begin{flushleft}
Permanent Magnet Machines, Permanent Magnet Brushless DC
\end{flushleft}





203





\begin{flushleft}
\newpage
Electrical Engineering
\end{flushleft}





\begin{flushleft}
Machines, Permanent Magnet Brushless Synchronous Machines,
\end{flushleft}


\begin{flushleft}
Stepper Motors, Hysteresis Motors, Switched Reluctance Motors,
\end{flushleft}


\begin{flushleft}
Hybrid Motors, Linear Machines, Magnetic Devices, Applications in
\end{flushleft}


\begin{flushleft}
Robotics, Industry Automation, Electric Vehicles, Aerospace and
\end{flushleft}


\begin{flushleft}
Defense Systems, etc., Super conducting Machines, Written Pole
\end{flushleft}


\begin{flushleft}
Machines, Micro-motors, PCB motors, Case Studies, Computer Aided
\end{flushleft}


\begin{flushleft}
Simulation and Design of Special Electrical Machines.
\end{flushleft}





\begin{flushleft}
ELL757 Energy Efficient Motors
\end{flushleft}


\begin{flushleft}
3 Credits (3-0-0)
\end{flushleft}


\begin{flushleft}
Introduction to energy efficiency and its impacts on social life. Energyefficient motors, fundamentals of electric motor drives, power factor
\end{flushleft}


\begin{flushleft}
under non sinusoidal conditions, energy efficient induction motor
\end{flushleft}


\begin{flushleft}
under different input parameters and applications, adjustable-speed
\end{flushleft}


\begin{flushleft}
drives their advantages and benefits from efficiency point of view,
\end{flushleft}


\begin{flushleft}
case studies related to induction motor variable seed drive system,
\end{flushleft}


\begin{flushleft}
brushless dc motor drive, switched reluctance motor drives, permanent
\end{flushleft}


\begin{flushleft}
magnet synchronous motor drive etc.
\end{flushleft}





\begin{flushleft}
ELL758 Power Quality
\end{flushleft}


\begin{flushleft}
3 Credits (3-0-0)
\end{flushleft}


\begin{flushleft}
Overview and definition of power quality (PQ), Sources of pollution,
\end{flushleft}


\begin{flushleft}
International power quality standards, and regulations. Power quality
\end{flushleft}


\begin{flushleft}
monitoring. Power quality problems. Loads which causes power quality
\end{flushleft}


\begin{flushleft}
problems. Power factor correction, zero voltage regulation, reactive power
\end{flushleft}


\begin{flushleft}
compensation, load balancing using load compensation techniques:
\end{flushleft}


\begin{flushleft}
passive shunt and series compensation, DSTATCOM (Distribution Static
\end{flushleft}


\begin{flushleft}
Compensators), DVR (Dynamic Voltage Restorers), UPQC (Universal
\end{flushleft}


\begin{flushleft}
Power Quality Conditioners). Harmonic effects-within the power system,
\end{flushleft}


\begin{flushleft}
interference with communication Harmonic measurements. Harmonic
\end{flushleft}


\begin{flushleft}
elimination-using active (shunt, series and hybrid) and passive (shunt and
\end{flushleft}


\begin{flushleft}
series) filters. Improved power quality converters: single ac-dc converters,
\end{flushleft}


\begin{flushleft}
bridgeless isolated converter, bridgeless non-isolated converters, multipulse converters, multilevel converters, line commutated converters,
\end{flushleft}


\begin{flushleft}
power quality improvement in SMPS, UPS, drives, welding systems,
\end{flushleft}


\begin{flushleft}
lighting systems, and renewable energy systems.
\end{flushleft}





\begin{flushleft}
ELL759 Power Electronic Converters for Renewable
\end{flushleft}


\begin{flushleft}
Energy Systems
\end{flushleft}


\begin{flushleft}
3 Credits (3-0-0)
\end{flushleft}


\begin{flushleft}
Current status and future developments in renewable energy.
\end{flushleft}


\begin{flushleft}
Requirements for solar and wind power generation from the grid.
\end{flushleft}


\begin{flushleft}
Solar Power -- PV system configurations, Solar cell technologies,
\end{flushleft}


\begin{flushleft}
Maximum power point tracking, Photovoltaic Inverters different
\end{flushleft}


\begin{flushleft}
types of topologies and control strategies. Wind power -- Wind
\end{flushleft}


\begin{flushleft}
power energy system, types of wind turbines- fixed speed and
\end{flushleft}


\begin{flushleft}
variable speed, different types of converters -- AC-DC-AC converters,
\end{flushleft}


\begin{flushleft}
matrix converters, multilevel converter, control of converters. Fuel
\end{flushleft}


\begin{flushleft}
cells and battery energy storage systems. Grid synchronization and
\end{flushleft}


\begin{flushleft}
PLL, Grid regulations. Islanding operation. Control of converters for
\end{flushleft}


\begin{flushleft}
fault operation. Filter design. Relevant IEEE and IEC standards for
\end{flushleft}


\begin{flushleft}
renewable energy systems.
\end{flushleft}





\begin{flushleft}
ELL760 Switched Mode Power Conversion
\end{flushleft}


\begin{flushleft}
3 Credits (3-0-0)
\end{flushleft}


\begin{flushleft}
To give an introduction about the power switching devices such as
\end{flushleft}


\begin{flushleft}
Thyristors, GTO, MOSFETS, BJT, IGBT and MCTS. Basic concept of
\end{flushleft}


\begin{flushleft}
gate drivers (Trigger techniques, optical isolators, protection circuits,
\end{flushleft}


\begin{flushleft}
and isolation transformers), snubber design and protection schemes of
\end{flushleft}


\begin{flushleft}
power devices are to be discussed. Basic circuit configurations, design
\end{flushleft}


\begin{flushleft}
and analysis of choppers (step-up, step-down, step-up/down and
\end{flushleft}


\begin{flushleft}
multi-phase choppers), DC-DC converters (non-isolated and isolated),
\end{flushleft}


\begin{flushleft}
inverters (voltage and current source and multi-level configurations)
\end{flushleft}


\begin{flushleft}
are discussed. This is followed by improved power quality converters
\end{flushleft}


\begin{flushleft}
(non-isolated and isolated) for reduction of harmonics at AC mains.
\end{flushleft}





\begin{flushleft}
ELL761 Power Electronics for Utility Interface
\end{flushleft}


\begin{flushleft}
3 Credits (3-0-0)
\end{flushleft}


\begin{flushleft}
Overview of power electronic converters for utility applications,
\end{flushleft}


\begin{flushleft}
Converter requirements for Grid-interface, Harmonic compensation,
\end{flushleft}


\begin{flushleft}
Instantaneous power theory, STATCOM and active filtering and Control
\end{flushleft}


\begin{flushleft}
of converters under grid-faults.
\end{flushleft}





\begin{flushleft}
ELL762 Intelligent Motor Controllers
\end{flushleft}


\begin{flushleft}
3 Credits (3-0-0)
\end{flushleft}


\begin{flushleft}
Pre-requisites: ELL305
\end{flushleft}


\begin{flushleft}
Fundamental concepts in control of electric drive systems. Intelligent
\end{flushleft}


\begin{flushleft}
Control algorithms used for electric drive systems. Application of
\end{flushleft}


\begin{flushleft}
Fuzzy Logic, Neural Networks, Genetic Algorithm, Hybrid Fuzzy and
\end{flushleft}


\begin{flushleft}
Nonlinear Control of Power Converters and Drives. Other recent topics
\end{flushleft}


\begin{flushleft}
on Intelligent Control of Drives.
\end{flushleft}





\begin{flushleft}
ELL763 Advanced Electric Drives
\end{flushleft}


\begin{flushleft}
3 Credits (3-0-0)
\end{flushleft}


\begin{flushleft}
Pre-requisites: ELL305
\end{flushleft}


\begin{flushleft}
Types of Controllers: Proportional-Integral Control, Hysteresis
\end{flushleft}


\begin{flushleft}
Control etc, Advanced DC Drives: Cascaded Control Loop Structure,
\end{flushleft}


\begin{flushleft}
Control Loop Design etc, Control of BLDC drive: Modeling and
\end{flushleft}


\begin{flushleft}
Control of BLDC Drive, Review of Power Converter and Modulation
\end{flushleft}


\begin{flushleft}
Techniques: Modeling of Power Converters, Sinusoidal Pulse-Width
\end{flushleft}


\begin{flushleft}
Modulation, Space Vector Pulse-Width Modulation, Field Oriented
\end{flushleft}


\begin{flushleft}
Control (FOC) of AC Machines: Generalized Space-Phasor Model
\end{flushleft}


\begin{flushleft}
of AC Machines in different Flux Frames of References, Control
\end{flushleft}


\begin{flushleft}
Principle, FOC of Permanent Magnet Synchronous Machine (PMSM),
\end{flushleft}


\begin{flushleft}
FOC of Squirrel Cage Induction Machine (SQIM), Direct Torque
\end{flushleft}


\begin{flushleft}
Control (DTC) of AC Machines: Control Principle, DTC of Squirrel
\end{flushleft}


\begin{flushleft}
Cage Induction Machine (SQIM).
\end{flushleft}





\begin{flushleft}
ELL764 Electric Vehicles
\end{flushleft}


\begin{flushleft}
3 Credits (3-0-0)
\end{flushleft}


\begin{flushleft}
The objective of this course is to familiarize students with fundamental
\end{flushleft}


\begin{flushleft}
issues related to electric vehicles (EVs) and hybrid electric vehicles
\end{flushleft}


\begin{flushleft}
(HEVs). The various brushless motors such as PMSM, PMBLDCM,
\end{flushleft}


\begin{flushleft}
SRM, synchronous reluctance motor, induction motor for EVs are to be
\end{flushleft}


\begin{flushleft}
covered. Moreover, several types of chargers and energy management
\end{flushleft}


\begin{flushleft}
strategies are to be discussed. The objective also involves analyzing
\end{flushleft}


\begin{flushleft}
and disseminating information relating to electric vehicle. Various
\end{flushleft}


\begin{flushleft}
design and control aspects of electric drives and chargers for EVs and
\end{flushleft}


\begin{flushleft}
HEVs are to be discussed.
\end{flushleft}





\begin{flushleft}
ELL765 Smart Grid Technology
\end{flushleft}


\begin{flushleft}
3 Credits (3-0-0)
\end{flushleft}


\begin{flushleft}
Introduction:- Smart Grid an Overview; Components of Smart Grid;
\end{flushleft}


\begin{flushleft}
Intelligent Appliances; Smart Substations; Smart DistributionsGenerations; Smart Power meters; Universal Access (wind, solar, hydro
\end{flushleft}


\begin{flushleft}
etc.) Smart Grid Technologies: Integrated Communications; Sensing
\end{flushleft}


\begin{flushleft}
and Measurement; Advance Control Methods; Advance components
\end{flushleft}


\begin{flushleft}
and Improved Interfaces and Decision Support. Benefits of Smart
\end{flushleft}


\begin{flushleft}
Grid: Self-Healing; Power Quality Improvement; Utilization of all
\end{flushleft}


\begin{flushleft}
generation and storage options; Optimized use of assets and efficient
\end{flushleft}


\begin{flushleft}
Operation. Miscellaneous: Smart Grid Challenges; Smart Grid Projects;
\end{flushleft}


\begin{flushleft}
Contribution of Microgrid in development of Smart Grid.
\end{flushleft}





\begin{flushleft}
ELL766 Appliance Systems
\end{flushleft}


\begin{flushleft}
3 Credits (3-0-0)
\end{flushleft}


\begin{flushleft}
Pre-requisites: ELL203, ELL332 ELL365
\end{flushleft}


\begin{flushleft}
Overview of appliance systems, international standards and
\end{flushleft}


\begin{flushleft}
regulations, energy efficient appliances, energy efficiency in motor
\end{flushleft}


\begin{flushleft}
driven appliances, classification based on power rating: low, medium,
\end{flushleft}


\begin{flushleft}
and high power appliances, classification based on supply power:
\end{flushleft}


\begin{flushleft}
single-phase and three-phase, classification based on drives system,
\end{flushleft}


\begin{flushleft}
heating systems and renewable system.
\end{flushleft}


\begin{flushleft}
To understand the various types of appliance systems used in domestic
\end{flushleft}


\begin{flushleft}
and office or commercial scenarios.
\end{flushleft}


\begin{flushleft}
Low power appliances (working, types, power quality problems,
\end{flushleft}


\begin{flushleft}
numerical examples): laptops, mobile, fans, lighting system(CFL, LED,
\end{flushleft}


\begin{flushleft}
solar), water pumps, TV (LCD, LED Plasma), UPS, SMPS, computer,
\end{flushleft}


\begin{flushleft}
printer, scanner, hair drier, trimmer, electric rice cooker, induction
\end{flushleft}


\begin{flushleft}
heater, solar cooker, electric iron, micro-oven, driller etc.
\end{flushleft}


\begin{flushleft}
Medium power appliances (working, types, power quality problems,
\end{flushleft}


\begin{flushleft}
numerical examples): Air conditioner, electrical vehicle, centralized
\end{flushleft}


\begin{flushleft}
heating system, washing machines, refrigerators, welding system,
\end{flushleft}


\begin{flushleft}
solar boiler, water pumps etc.
\end{flushleft}





204





\begin{flushleft}
\newpage
Electrical Engineering
\end{flushleft}





\begin{flushleft}
High power appliances (working, types, power quality problems,
\end{flushleft}


\begin{flushleft}
numerical examples): welding machines, hammers, centralized AC
\end{flushleft}


\begin{flushleft}
system, etc.
\end{flushleft}


\begin{flushleft}
Power quality techniques used in appliances systems.
\end{flushleft}





\begin{flushleft}
ELL767 Mechatronics
\end{flushleft}


\begin{flushleft}
3 Credits (3-0-0)
\end{flushleft}


\begin{flushleft}
The course articulates the essence of mechatronics and provide
\end{flushleft}


\begin{flushleft}
examples of mechatronic systems. Moreover, it explains analog-todigital-conversion (A/D) and its implementation using a microcontroller
\end{flushleft}


\begin{flushleft}
and DSPs. Study of the underlying operational principles and
\end{flushleft}


\begin{flushleft}
construction of electromagnetic actuators such as DC, AC, and
\end{flushleft}


\begin{flushleft}
stepping motors. Study of various transducers their working principles
\end{flushleft}


\begin{flushleft}
etc. Selection of best electrical machines for a given motion control
\end{flushleft}


\begin{flushleft}
application considering system inertia, external forces or torques, and
\end{flushleft}


\begin{flushleft}
motion profiles and select an appropriate motor. Design and analysis
\end{flushleft}


\begin{flushleft}
for basic power controllers for various applications.
\end{flushleft}





\begin{flushleft}
ELL768 Computer Aided Design of Power Electronic
\end{flushleft}


\begin{flushleft}
Systems
\end{flushleft}


\begin{flushleft}
3 Credits (3-0-0)
\end{flushleft}


\begin{flushleft}
Introduction to modern simulation tools used for the power electronic
\end{flushleft}


\begin{flushleft}
systems analysis such as PSPICE, MATLAB, PSIM, SABER etc, Modeling
\end{flushleft}


\begin{flushleft}
of power electronic systems, filters designs. Introducing to advanced
\end{flushleft}


\begin{flushleft}
modeling techniques and their transformation into software platform,
\end{flushleft}


\begin{flushleft}
Closed-loop power electronic systems modeling and their simulation.
\end{flushleft}





\begin{flushleft}
ELL769 Electrical Systems for Construction Industries
\end{flushleft}


\begin{flushleft}
4 Credits (3-0-2)
\end{flushleft}


\begin{flushleft}
Elements of Distribution System: Distribution transformer circuit
\end{flushleft}


\begin{flushleft}
breakers, Cables, Fuses and protection schemes, Rectifiers, Battery
\end{flushleft}


\begin{flushleft}
chargers and inverters. Machines and Drives: D.C. Motors, 3-phase
\end{flushleft}


\begin{flushleft}
induction motors and FKW motors starting, speed control and braking,
\end{flushleft}


\begin{flushleft}
Application to air conditioning, lifts, cranes, water pumps. Illumination:
\end{flushleft}


\begin{flushleft}
Types of illumination, illumination laws, lamps \& fixtures. Electrical
\end{flushleft}


\begin{flushleft}
Energy Conservation: Modern compact fluorescent lamps, energy audit
\end{flushleft}


\begin{flushleft}
methods of saving electricity in drives, lighting, air conditioning, pumps
\end{flushleft}


\begin{flushleft}
and distributions systems metering, KW, KWh and KVAR meters stand
\end{flushleft}


\begin{flushleft}
by power generation: DG sets, UPS, maintenance and protection of
\end{flushleft}


\begin{flushleft}
D.G. sets and UPS.
\end{flushleft}





\begin{flushleft}
ELL770 Power System Analysis
\end{flushleft}


\begin{flushleft}
3 Credits (3-0-0)
\end{flushleft}


\begin{flushleft}
Pre-requisites: ELL303
\end{flushleft}


\begin{flushleft}
Revision of Basic Concepts in pu and modeling, Admittance model
\end{flushleft}


\begin{flushleft}
of transmission network, Power Flow solutions (GS, NR, DLF, FDLF,
\end{flushleft}


\begin{flushleft}
DCLF), Symmetrical components and sequence networks, Faults Symmetrical and unsymmetrical, Z Bus building algorithms, State
\end{flushleft}


\begin{flushleft}
Estimation, Voltage Stability, Continuation Power Flow, Power System
\end{flushleft}


\begin{flushleft}
Security (Overload, Voltage), Introduction to WAMS and PMUs, Linear
\end{flushleft}


\begin{flushleft}
State Estimation.
\end{flushleft}





\begin{flushleft}
ELL771 Advanced Power System Protection
\end{flushleft}


\begin{flushleft}
3 Credits (3-0-0)
\end{flushleft}


\begin{flushleft}
Fundamentals of protection, generator protection, transformer
\end{flushleft}


\begin{flushleft}
protection, bus bar protection, over current and differential protection.
\end{flushleft}


\begin{flushleft}
Out of step protection, blinder design. Static relays, Numerical relay.
\end{flushleft}


\begin{flushleft}
Wide area protection.
\end{flushleft}





\begin{flushleft}
ELL772 Planning and Operation of a Smart Grid
\end{flushleft}


\begin{flushleft}
3 Credits (3-0-0)
\end{flushleft}


\begin{flushleft}
Pre-requisites: ELL303
\end{flushleft}


\begin{flushleft}
Smart grids key characteristics, demand side management, load
\end{flushleft}


\begin{flushleft}
characteristics, hybrid electric vehicles, energy markets, deregulation,
\end{flushleft}


\begin{flushleft}
wide area monitoring, protection and control, smart metering,
\end{flushleft}


\begin{flushleft}
adaptive relaying, power line carrier communication and networking,
\end{flushleft}


\begin{flushleft}
architectures and standards, renewable energy, distributed generation,
\end{flushleft}


\begin{flushleft}
smart grids policies.
\end{flushleft}





\begin{flushleft}
ELL773 High Voltage DC Transmission
\end{flushleft}


\begin{flushleft}
3 Credits (3-0-0)
\end{flushleft}


\begin{flushleft}
General aspects and comparison with AC transmission system.
\end{flushleft}


\begin{flushleft}
Thyristor based HVDC Converter and inverter operation. Control
\end{flushleft}


\begin{flushleft}
of HVDC link. Interaction between AC and DC system. Harmonic
\end{flushleft}


\begin{flushleft}
generation and their elimination. Protections for HVDC system.
\end{flushleft}


\begin{flushleft}
Modeling of HVDC link for AC-DC power flow. AC-DC system power
\end{flushleft}


\begin{flushleft}
flow solution techniques. HVDC light.
\end{flushleft}





\begin{flushleft}
ELL774 Flexible AC Transmission System
\end{flushleft}


\begin{flushleft}
3 Credits (3-0-0)
\end{flushleft}


\begin{flushleft}
The phenomenon of voltage collapse; the basic theory of line
\end{flushleft}


\begin{flushleft}
compensation. Static VAR compensators; static phase shifters;
\end{flushleft}


\begin{flushleft}
thyristors controlled series capacitors. Co-ordination of FACTS devices
\end{flushleft}


\begin{flushleft}
with HVDC links. The FACTS optimization problem. Transient and
\end{flushleft}


\begin{flushleft}
dynamic stability enhancement using FACTS components.
\end{flushleft}





\begin{flushleft}
ELL775 Power System Dynamics
\end{flushleft}


\begin{flushleft}
3 Credits (3-0-0)
\end{flushleft}


\begin{flushleft}
Pre-requisites: ELL303
\end{flushleft}


\begin{flushleft}
Dynamic models of synchronous machines, excitation system,
\end{flushleft}


\begin{flushleft}
turbines, governors, loads. Modelling of single-machine-infinite
\end{flushleft}


\begin{flushleft}
bus system. Mathematical modelling of multimachine system.
\end{flushleft}


\begin{flushleft}
Dynamic and transient stability analysis of single machine and multimachine systems. Power system stabilizer design for multimachine
\end{flushleft}


\begin{flushleft}
systems. Dynamic equivalencing. Voltage stability Techniques for the
\end{flushleft}


\begin{flushleft}
improvement of stability. Direct method of transient stability analysis:
\end{flushleft}


\begin{flushleft}
Transient energy function approach.
\end{flushleft}





\begin{flushleft}
ELL776 Advanced Power System Optimization
\end{flushleft}


\begin{flushleft}
3 Credits (3-0-0)
\end{flushleft}


\begin{flushleft}
Introduction to power system optimization problems and linkages.
\end{flushleft}


\begin{flushleft}
Optimization basics and solution techniques for convex and non convex
\end{flushleft}


\begin{flushleft}
optimization problems. Basic Optimal power flow. Preventive and
\end{flushleft}


\begin{flushleft}
corrective security constrained optimal power flow, Unit commitment,
\end{flushleft}


\begin{flushleft}
hydrothermal scheduling, generation, transmission and reactive
\end{flushleft}


\begin{flushleft}
expansion planning. Optimization with uncertain data
\end{flushleft}


\begin{flushleft}
Introduction to power system optimization problem and their linkages.
\end{flushleft}


\begin{flushleft}
Security states and optimization requirements. Convex and nonconvex
\end{flushleft}


\begin{flushleft}
optimization techniques. Static and dynamic optimization techniques.
\end{flushleft}


\begin{flushleft}
Day ahead and real time market planning. Optimization to handle
\end{flushleft}


\begin{flushleft}
uncertainty in data. Fuzzy and probabilistic techniques. Generation,
\end{flushleft}


\begin{flushleft}
transmission and reactive resources planning. Renewable generation
\end{flushleft}


\begin{flushleft}
integration optimization. Effect of markets and renewable generation
\end{flushleft}


\begin{flushleft}
in resources planning.
\end{flushleft}





\begin{flushleft}
ELL777 Power System operation and control
\end{flushleft}


\begin{flushleft}
3 Credits (3-0-0)
\end{flushleft}


\begin{flushleft}
Control of active power. Turbine, governor and boiler modelling and
\end{flushleft}


\begin{flushleft}
control. Hydro and steam turbines, load frequency control, Automatic
\end{flushleft}


\begin{flushleft}
generation control in single-area and multi-area systems. Underfrequency load shedding, secondary frequency control. Automatic
\end{flushleft}


\begin{flushleft}
voltage regulators, excitation systems -- modelling and control,
\end{flushleft}


\begin{flushleft}
small-signal stability studies, power system stabilizers, on-load tapchanging transformers.
\end{flushleft}





\begin{flushleft}
ELL778 Dynamic Modelling And Control of Sustainable
\end{flushleft}


\begin{flushleft}
Energy Systems
\end{flushleft}


\begin{flushleft}
3 Credits (3-0-0)
\end{flushleft}


\begin{flushleft}
Microgrids and distributed generation; Introduction to renewable
\end{flushleft}


\begin{flushleft}
energy technologies; electrical systems and generators used in wind
\end{flushleft}


\begin{flushleft}
energy conversion systems,diesel generators, combined heat cycle
\end{flushleft}


\begin{flushleft}
plants, inverter based generation, solar PV based systems, fuel cell and
\end{flushleft}


\begin{flushleft}
aqua-electrolyzer, battery and flywheel based storage system; Voltage
\end{flushleft}


\begin{flushleft}
and frequency control in a microgrid; Grid connection interface issues.
\end{flushleft}





\begin{flushleft}
ELL779 Forecasting Techniques for Power System
\end{flushleft}


\begin{flushleft}
3 Credits (3-0-0)
\end{flushleft}


\begin{flushleft}
Principles of forecasting load, wind and price. Statistical and non
\end{flushleft}


\begin{flushleft}
statistical based approaches. AI application for forecasting.
\end{flushleft}





205





\begin{flushleft}
\newpage
Electrical Engineering
\end{flushleft}





\begin{flushleft}
ELD780 Minor Project
\end{flushleft}


\begin{flushleft}
2 Credits (0-0-4)
\end{flushleft}


\begin{flushleft}
ELL780 Mathematical Foundations of Computer
\end{flushleft}


\begin{flushleft}
Technology
\end{flushleft}


\begin{flushleft}
3 Credits (3-0-0)
\end{flushleft}


\begin{flushleft}
Probability theory, stochastic processes, and statistical inference.
\end{flushleft}


\begin{flushleft}
Elements of real and complex analysis, and linear algebra.
\end{flushleft}


\begin{flushleft}
Optimization, with an emphasis on application and implementation.
\end{flushleft}





\begin{flushleft}
ELP780 Software Lab
\end{flushleft}


\begin{flushleft}
3 Credits (0-1-4)
\end{flushleft}


\begin{flushleft}
Experiments related to the following topics: advanced data
\end{flushleft}


\begin{flushleft}
structures and algorithms, compilers, GUI, component-based
\end{flushleft}


\begin{flushleft}
software design, distributed and web based applications, UML,
\end{flushleft}


\begin{flushleft}
firmware, database applications.
\end{flushleft}





\begin{flushleft}
ELV780 Special Module in Computers
\end{flushleft}


\begin{flushleft}
1 Credit (1-0-0)
\end{flushleft}


\begin{flushleft}
ELL781 Software Fundamentals for Computer Technology
\end{flushleft}


\begin{flushleft}
3 Credits (3-0-0)
\end{flushleft}


\begin{flushleft}
Introduction, data structures for combinatorial optimization: heaps,
\end{flushleft}


\begin{flushleft}
union-find, Fibonacci heaps, dynamic trees, dynamic graph structure;
\end{flushleft}


\begin{flushleft}
Asymptotic analysis; Divide \& conquer and graph algorithms: Graph
\end{flushleft}


\begin{flushleft}
search: Breadth first, depth first, topological sorting, Fast Fourier
\end{flushleft}


\begin{flushleft}
Transform, Matrix Multiplication, Shortest path algorithms; Additional
\end{flushleft}


\begin{flushleft}
Data Structures: Suffix trees \& string matching, Splay trees \&
\end{flushleft}


\begin{flushleft}
amortized analysis; Advanced algorithmic design techniques: Dynamic
\end{flushleft}


\begin{flushleft}
programming (edit distance, chains of matrix multiplication, etc.),
\end{flushleft}


\begin{flushleft}
Network flow and its use for solving problems; Linear and integer
\end{flushleft}


\begin{flushleft}
programming, NP-completeness, Randomized algorithms (hashing
\end{flushleft}


\begin{flushleft}
\& global minimum cut), Approximation Algorithms; Object oriented
\end{flushleft}


\begin{flushleft}
Software design, Design of Dependable Software.
\end{flushleft}





\begin{flushleft}
ELP781 Digital Systems Lab
\end{flushleft}


\begin{flushleft}
3 Credits (0-1-4)
\end{flushleft}


\begin{flushleft}
ELV781 Special Modules in Information Processing-I
\end{flushleft}


\begin{flushleft}
1 Credit (1-0-0)
\end{flushleft}


\begin{flushleft}
Pre-requisites: to be decided by the instructor
\end{flushleft}





\begin{flushleft}
Capacity of a Single Layer LTU, Nonlinear Dichotomies, Multilayer
\end{flushleft}


\begin{flushleft}
Networks, Growth networks, Backpropagation and some variants;
\end{flushleft}


\begin{flushleft}
Support Vector Machines: Origin, Formulation of the L1 norm SVM,
\end{flushleft}


\begin{flushleft}
Solution methods (SMO, etc.), L2 norm SVM, Regression, Variants of
\end{flushleft}


\begin{flushleft}
the SVM; Complexity: Origin, Notion of the VC dimension, Derivation
\end{flushleft}


\begin{flushleft}
for an LTU, PAC learning, bounds, VC dimension for SVMS, Learning
\end{flushleft}


\begin{flushleft}
low complexity machines - Structural Risk Minimisation; Unsupervised
\end{flushleft}


\begin{flushleft}
learning: PCA, KPCA; Clustering: Origin, Exposition with some selected
\end{flushleft}


\begin{flushleft}
methods; Feature Selection: Origin, Filter and Wrapper methods, State
\end{flushleft}


\begin{flushleft}
of the art - FCBF, Relief, etc; Semi-supervised learning: introduction;
\end{flushleft}


\begin{flushleft}
Assignments/Short project on these topics.
\end{flushleft}





\begin{flushleft}
ELL785 Computer Communication Networks
\end{flushleft}


\begin{flushleft}
3 Credits (3-0-0)
\end{flushleft}


\begin{flushleft}
Pre-requisites: MTL106/ELL711
\end{flushleft}


\begin{flushleft}
Overlaps with: CSL374, CSL672 (20\%)
\end{flushleft}


\begin{flushleft}
Theory/Lecture: Review of data communication techniques, basic
\end{flushleft}


\begin{flushleft}
networking concepts, layered network and protocol concepts, quality
\end{flushleft}


\begin{flushleft}
of service, motivations for cross-layer protocol design. Motivations for
\end{flushleft}


\begin{flushleft}
performance analysis, forward error correction and re-transmission
\end{flushleft}


\begin{flushleft}
performances, Markov and semi-Markov processes, Little's theorem,
\end{flushleft}


\begin{flushleft}
M/M/m/k, M/G/1 systems, priority queueing, network of queues,
\end{flushleft}


\begin{flushleft}
network traffic behavior. Concepts and analysis of multi-access
\end{flushleft}


\begin{flushleft}
protocols; contention-free and contention multi-access protocols. Basic
\end{flushleft}


\begin{flushleft}
graph theoretic concepts, routing algorithms and analysis.
\end{flushleft}


\begin{flushleft}
Suggested lab Course content:
\end{flushleft}


\begin{flushleft}
Laboratory: Simulation and hardware experiments on different aspects
\end{flushleft}


\begin{flushleft}
of computer communication networks. Network traffic generation
\end{flushleft}


\begin{flushleft}
and analysis, differentiated service queues, network of queues using
\end{flushleft}


\begin{flushleft}
discrete event simulations.
\end{flushleft}





\begin{flushleft}
ELL786 Multimedia Systems
\end{flushleft}


\begin{flushleft}
3 Credits (3-0-0)
\end{flushleft}


\begin{flushleft}
Multimedia signal processing; coding and compression; standards:
\end{flushleft}


\begin{flushleft}
logic, issues, future directions; Multimedia issues governing
\end{flushleft}


\begin{flushleft}
developments in computer architecture and embedded systems,
\end{flushleft}


\begin{flushleft}
computer and communication networks, operating systems;
\end{flushleft}


\begin{flushleft}
Search and retrieval.
\end{flushleft}





\begin{flushleft}
ELL787 Embedded Systems and Applications
\end{flushleft}


\begin{flushleft}
3 Credits (3-0-0)
\end{flushleft}


\begin{flushleft}
Introduction to embedded system. Architectural Issues: CISC, RISC,
\end{flushleft}


\begin{flushleft}
DSP Architectures.
\end{flushleft}





\begin{flushleft}
ELL782 Computer Architecture
\end{flushleft}


\begin{flushleft}
3 Credits (3-0-0)
\end{flushleft}


\begin{flushleft}
Instruction set design, pipelining, memory hierarchy design, parallelism
\end{flushleft}


\begin{flushleft}
in various forms, warehouse scale computers, specific topics such as
\end{flushleft}


\begin{flushleft}
Vector, SIMD, GPU architectures, Embedded Systems, VLIW, EPIC,
\end{flushleft}


\begin{flushleft}
Multi-core architectures.
\end{flushleft}





\begin{flushleft}
ELP782 Computer Networks Lab
\end{flushleft}


\begin{flushleft}
3 Credits (0-1-4)
\end{flushleft}


\begin{flushleft}
Simulation and hardware experiments on different aspects of
\end{flushleft}


\begin{flushleft}
computer communication networks. Network traffic generation and
\end{flushleft}


\begin{flushleft}
analysis, differentiated service queues, network of queues using
\end{flushleft}


\begin{flushleft}
discrete event simulations.
\end{flushleft}





\begin{flushleft}
ELL783 Operating Systems
\end{flushleft}


\begin{flushleft}
4 Credits (3-0-2)
\end{flushleft}


\begin{flushleft}
Processes and threads; CPU scheduling; concurrency, synchronisation;
\end{flushleft}


\begin{flushleft}
deadlocks; Memory management; files and I/O; Real-time operating
\end{flushleft}


\begin{flushleft}
systems; basics of Cloud computing.
\end{flushleft}





\begin{flushleft}
ELL784 Introduction to Machine Learning
\end{flushleft}


\begin{flushleft}
3 Credits (3-0-0)
\end{flushleft}


\begin{flushleft}
Pre-requisites: MTL106
\end{flushleft}


\begin{flushleft}
Overlaps with: ELL409, COL341, COL774, MAL803
\end{flushleft}





\begin{flushleft}
Component Interfacing, Software for Embedded Systems : Program
\end{flushleft}


\begin{flushleft}
Design and Optimisation techniques, O.S for Embedded Systems,
\end{flushleft}


\begin{flushleft}
Real-time Issues. Designing Embedded Systems : Design Issues,
\end{flushleft}


\begin{flushleft}
Hardware- Software Co-design, Use of UML. Embedded Control
\end{flushleft}


\begin{flushleft}
Applications, Networked Embedded Systems : Distributed Embedded
\end{flushleft}


\begin{flushleft}
Architectures, Protocol Design issues, wireless network. Embedded
\end{flushleft}


\begin{flushleft}
Multimedia and Telecommunication Applications: Digital Camera,
\end{flushleft}


\begin{flushleft}
Digital TV, Set-top Box, Voice and Video telephony.
\end{flushleft}





\begin{flushleft}
ELL788 Computational Perception and Cognition
\end{flushleft}


\begin{flushleft}
3 Credits (3-0-0)
\end{flushleft}


\begin{flushleft}
Introduction: Philosophical \& Psychological models, Cognitive models
\end{flushleft}


\begin{flushleft}
\& Bayesian Inferencing framework; Visual Perception of 3D space \&
\end{flushleft}


\begin{flushleft}
scene; Perceptual processes for Object recognition \& memorization;
\end{flushleft}


\begin{flushleft}
Auditory Perception; Haptic Perception; Attentional mechanism
\end{flushleft}


\begin{flushleft}
in multimedia perception; Applications: Image \& video quality
\end{flushleft}


\begin{flushleft}
assessment, compression; Audio quality assessment, compression
\end{flushleft}


\begin{flushleft}
\& indexing; Haptic interfaces; Cognitive Architecture; Computational
\end{flushleft}


\begin{flushleft}
Consciousness, Cognitive Robotics \& Other applications.
\end{flushleft}





\begin{flushleft}
ELL789 Intelligent Systems
\end{flushleft}


\begin{flushleft}
3 Credits (3-0-0)
\end{flushleft}


\begin{flushleft}
Overlaps with: ELL409, COL333, COL770
\end{flushleft}





\begin{flushleft}
Introduction to Machine intelligence and learning; linear learning
\end{flushleft}


\begin{flushleft}
models; Artificial Neural Networks: Single Layer Networks, LTUs,
\end{flushleft}





\begin{flushleft}
Introduction, Search, Markov Decision Process, Game Playing,
\end{flushleft}


\begin{flushleft}
Constraint Satisfaction, Bayesian Network, Logic, Planning, Searching
\end{flushleft}


\begin{flushleft}
with non-deterministic action.
\end{flushleft}





206





\begin{flushleft}
\newpage
Electrical Engineering
\end{flushleft}





\begin{flushleft}
ELL790 Digital Hardware Design
\end{flushleft}


\begin{flushleft}
3 Credits (3-0-0)
\end{flushleft}


\begin{flushleft}
To provide advanced level exposure to digital hardware design and
\end{flushleft}


\begin{flushleft}
interfacing, elements of hardware software co-design, synthesis
\end{flushleft}


\begin{flushleft}
of digital systems at logic/RTL and system levels, simulation
\end{flushleft}


\begin{flushleft}
aspects of synthesis.
\end{flushleft}





\begin{flushleft}
ELL791 Neural Systems and Learning Machines
\end{flushleft}


\begin{flushleft}
3 Credits (3-0-2)
\end{flushleft}


\begin{flushleft}
Introduction to biological neural systems, artificial neural network
\end{flushleft}


\begin{flushleft}
models, feed forward models, recurrent systems, analysis and
\end{flushleft}


\begin{flushleft}
applications.
\end{flushleft}





\begin{flushleft}
ELL792 Computer Graphics
\end{flushleft}


\begin{flushleft}
3 Credits (3-0-0)
\end{flushleft}





\begin{flushleft}
ELL799 Natural Computing
\end{flushleft}


\begin{flushleft}
3 Credits (3-0-0)
\end{flushleft}


\begin{flushleft}
Pre-requisites: COL106, MTL106
\end{flushleft}


\begin{flushleft}
Introduction to natural computing uncertainty handling: probability and
\end{flushleft}


\begin{flushleft}
fuzzy logic; evolutionary computing and problem solving as search;
\end{flushleft}


\begin{flushleft}
swarm intelligence ant colonies, swarm robotics; immunocomputing;
\end{flushleft}


\begin{flushleft}
introduction to DNA computing; basics of quantum computing.
\end{flushleft}





\begin{flushleft}
JVD799 Minor Project
\end{flushleft}


\begin{flushleft}
6 Credits (0-0-12)
\end{flushleft}


\begin{flushleft}
ELD800 Minor Project (EEA)
\end{flushleft}


\begin{flushleft}
3 Credits (0-0-6)
\end{flushleft}


\begin{flushleft}
To be decided by the project supervisor.
\end{flushleft}





\begin{flushleft}
Image formation: the mathematics, as well as photometry and colour;
\end{flushleft}


\begin{flushleft}
transformations; basic graphics primitives; texture mapping; imagebased rendering.
\end{flushleft}





\begin{flushleft}
JTD792 Minor Project
\end{flushleft}


\begin{flushleft}
3 Credits (0-0-6)
\end{flushleft}


\begin{flushleft}
ELL793 Computer Vision
\end{flushleft}


\begin{flushleft}
3 Credits (3-0-0)
\end{flushleft}


\begin{flushleft}
Pre-requisites: ELL715,ELL784
\end{flushleft}


\begin{flushleft}
Overlaps with: COL780
\end{flushleft}


\begin{flushleft}
Link between Computer Vision, Computer Graphics, Image Processing
\end{flushleft}


\begin{flushleft}
and related fields; feature extraction; camera models; multi-view
\end{flushleft}


\begin{flushleft}
geometry; applications of Computer Vision in day-to-day life.
\end{flushleft}





\begin{flushleft}
ELL794 Human-Computer Interface
\end{flushleft}


\begin{flushleft}
3 Credits (3-0-0)
\end{flushleft}


\begin{flushleft}
This course will present some of the necessary background in
\end{flushleft}


\begin{flushleft}
neuroscience and computational methods necessary to begin work
\end{flushleft}


\begin{flushleft}
in this emerging field that is rapidly acquiring growing significance.
\end{flushleft}





\begin{flushleft}
ELL800 Numerical Linear Algebra and Optimization in
\end{flushleft}


\begin{flushleft}
Engineering
\end{flushleft}


\begin{flushleft}
3 Credits (3-0-0)
\end{flushleft}


\begin{flushleft}
Basics of Linear Algebra; Matrix decomposition - LU, LDU, QR and
\end{flushleft}


\begin{flushleft}
Cholesky factorization; Householder reflection, Givens rotation;
\end{flushleft}


\begin{flushleft}
Numerical implications of SVD; Numerical Solution for Linear Systems;
\end{flushleft}


\begin{flushleft}
Algorithm Stability; Problem Conditioning; Pivoting and scaling; Least
\end{flushleft}


\begin{flushleft}
Square Solutions; Numerical Matrix eigenvalue methods; Sparse
\end{flushleft}


\begin{flushleft}
Systems; Iterative methods for large systems; Krylov, Arnoldi, Lanczos
\end{flushleft}


\begin{flushleft}
methods; Numerical Optimization techniques - Conjugate gradient
\end{flushleft}


\begin{flushleft}
method, Linear and quadratic programming, Spectral and Pseudospectral methods.
\end{flushleft}





\begin{flushleft}
ELP800 Control Systems Laboratory
\end{flushleft}


\begin{flushleft}
1 Credit (0-0-2)
\end{flushleft}


\begin{flushleft}
Basics of Sensors and Actuators, Study of AC and DC Motors, Linear
\end{flushleft}


\begin{flushleft}
Systems, Analog and Digital Motors, Synchros, Temperature Control.
\end{flushleft}





\begin{flushleft}
ELD801 Major Project Part-I
\end{flushleft}


\begin{flushleft}
6 Credits (0-0-12)
\end{flushleft}


\begin{flushleft}
To be decided by the project supervisor.
\end{flushleft}





\begin{flushleft}
ELL795 Swarm Intelligence
\end{flushleft}


\begin{flushleft}
3 Credits (3-0-0)
\end{flushleft}


\begin{flushleft}
Swarm intelligence, distributed optimization, ant colony algorithms,
\end{flushleft}


\begin{flushleft}
PSO, firefly, bee, and related methods, applications and
\end{flushleft}


\begin{flushleft}
implementation issues.
\end{flushleft}





\begin{flushleft}
ELL796 Signals and Systems in Biology
\end{flushleft}


\begin{flushleft}
3 Credits (3-0-0)
\end{flushleft}


\begin{flushleft}
Introduction to Cell Biology (DNA and Proteins); Introduction to
\end{flushleft}


\begin{flushleft}
Evolution; Modelling Evolution (Genetic Algorithms, Quasispecies);
\end{flushleft}


\begin{flushleft}
Genomic Signal Processing; Transcriptomic/Proteomic signals;
\end{flushleft}


\begin{flushleft}
Regulatory networks and dynamics; Protein interaction networks;
\end{flushleft}


\begin{flushleft}
Signal transduction and metabolic networks; Evolvability and Learning.
\end{flushleft}


\begin{flushleft}
Project activities on these topics (involving the use of online biological
\end{flushleft}


\begin{flushleft}
databases and bioinformatics software tools); Student presentations
\end{flushleft}


\begin{flushleft}
and Journal Club.
\end{flushleft}





\begin{flushleft}
ELL797 Energy-Efficient Computing
\end{flushleft}


\begin{flushleft}
3 Credits (3-0-0)
\end{flushleft}


\begin{flushleft}
Introduction and Motivation, Energy-Efficient Techniques in Operating
\end{flushleft}


\begin{flushleft}
Systems (Power Aware Scheduling, Adaptation for Multimedia
\end{flushleft}


\begin{flushleft}
Applications, Power aware memory and I/O device management,
\end{flushleft}


\begin{flushleft}
multiprocessor systems.), Storage, Compilers, Networks and Data
\end{flushleft}


\begin{flushleft}
Centers, Power management for Wearable devices and pervasive
\end{flushleft}


\begin{flushleft}
computing.
\end{flushleft}





\begin{flushleft}
ELL801 Nonlinear Control
\end{flushleft}


\begin{flushleft}
3 Credits (3-0-0)
\end{flushleft}


\begin{flushleft}
Overview of nonlinear control, Lyapunov stability for autonomous
\end{flushleft}


\begin{flushleft}
and non-autonomous systems, Input-Output Stability and Input-toState Stability, Passivity analysis and applications, Absolute Stability,
\end{flushleft}


\begin{flushleft}
Incremental stability analysis, Lyapunov-based feedback control
\end{flushleft}


\begin{flushleft}
design, Feedback linearization and backstepping, Sliding mode control,
\end{flushleft}


\begin{flushleft}
Nonlinear observer design.
\end{flushleft}





\begin{flushleft}
ELP801 Advanced Control Laboratory
\end{flushleft}


\begin{flushleft}
2 Credits (0-0-4)
\end{flushleft}


\begin{flushleft}
Magnetic Levitation System, Twin Rotor MIMO System, Gyroscope, Ball
\end{flushleft}


\begin{flushleft}
and Beam System, Embedded Control System, Mobile Robotic System.
\end{flushleft}





\begin{flushleft}
JTD801 Major Project-I
\end{flushleft}


\begin{flushleft}
6 Credits (0-0-12)
\end{flushleft}


\begin{flushleft}
JVD811 Major Project-I
\end{flushleft}


\begin{flushleft}
12 Credits (0-0-24)
\end{flushleft}


\begin{flushleft}
JVS801 Independent Study
\end{flushleft}


\begin{flushleft}
3 Credits (0-3-0)
\end{flushleft}


\begin{flushleft}
ELD802 Major Project Part-II
\end{flushleft}


\begin{flushleft}
12 Credits (0-0-24)
\end{flushleft}





\begin{flushleft}
ELL798 Agent Technologies
\end{flushleft}


\begin{flushleft}
3 Credits (3-0-0)
\end{flushleft}


\begin{flushleft}
The course will comprise lectures on the various topics on agent
\end{flushleft}


\begin{flushleft}
technology and self-study on its applications in various domains.
\end{flushleft}


\begin{flushleft}
The topics are elaborated below. The material of the lectures will be
\end{flushleft}


\begin{flushleft}
gathered from text-books and recent research papers. The self-study
\end{flushleft}


\begin{flushleft}
will comprise study and analysis of typically 5-8 substantial research
\end{flushleft}


\begin{flushleft}
papers and will result in a term paper that will be evaluated.
\end{flushleft}





\begin{flushleft}
To be decided by the project supervisor.
\end{flushleft}





\begin{flushleft}
ELL802 Adaptive and Learning Control
\end{flushleft}


\begin{flushleft}
3 Credits (3-0-0)
\end{flushleft}


\begin{flushleft}
Introduction to adaptive control, Review of Lyapunov stability theory,
\end{flushleft}


\begin{flushleft}
Direct and indirect adaptive control, Model reference adaptive
\end{flushleft}





207





\begin{flushleft}
\newpage
Electrical Engineering
\end{flushleft}





\begin{flushleft}
control, Parameter convergence, persistence of excitation, Adaptive
\end{flushleft}


\begin{flushleft}
backstepping, Adaptive control of nonlinear systems, Composite
\end{flushleft}


\begin{flushleft}
adaptation, Neural Network-based control, Repetitive learning control,
\end{flushleft}


\begin{flushleft}
Reinforcement learning-based control, Predictive control, Robust
\end{flushleft}


\begin{flushleft}
adaptive control.
\end{flushleft}





\begin{flushleft}
ELL808 Advanced Topics in Systems and Control
\end{flushleft}


\begin{flushleft}
3 Credits (3-0-0)
\end{flushleft}





\begin{flushleft}
JVD812 Major Project-II
\end{flushleft}


\begin{flushleft}
12 Credits (0-0-24)
\end{flushleft}





\begin{flushleft}
ELD810 Minor Project (Communication Engineering)
\end{flushleft}


\begin{flushleft}
3 Credits (0-0-6)
\end{flushleft}





\begin{flushleft}
ELL803 Model Reduction in Control
\end{flushleft}


\begin{flushleft}
3 Credits (3-0-0)
\end{flushleft}





\begin{flushleft}
ELL810 Cyber Security and Information Assurance
\end{flushleft}


\begin{flushleft}
3 Credits (3-0-0)
\end{flushleft}





\begin{flushleft}
Introduction to Model Reduction; Sources of Large Models - Circuits,
\end{flushleft}


\begin{flushleft}
Electromagnetic Systems, Mechanical Systems; Discretization Methods
\end{flushleft}


\begin{flushleft}
- Finite Difference Method (FDM), Finite Element Method (FEM);
\end{flushleft}


\begin{flushleft}
Classical Model Reduction Methods - Pade Approximation, Moment
\end{flushleft}


\begin{flushleft}
matching, Routh Approximants; Modern Methods - Modal Model
\end{flushleft}


\begin{flushleft}
Reduction Methods, SVD (Grammian) based methods, Krylov based
\end{flushleft}


\begin{flushleft}
methods, SVD-Krylov based methods; MOR for Nonlinear Systems
\end{flushleft}


\begin{flushleft}
-- SVD \& POD Methods; Model Reduction in Control; Control Design
\end{flushleft}


\begin{flushleft}
on Reduced Models -- Sub- optimal control; Sliding Mode Control as
\end{flushleft}


\begin{flushleft}
model reducing control - First Order SM, Higher Order Sliding Mode.
\end{flushleft}





\begin{flushleft}
ELL804 Robust Control
\end{flushleft}


\begin{flushleft}
3 Credits (3-0-0)
\end{flushleft}


\begin{flushleft}
Modeling of uncertain systems, Signals and Norms, Lyapunov theory
\end{flushleft}


\begin{flushleft}
for LTI systems
\end{flushleft}


\begin{flushleft}
Passive systems -- frequency domain, Passive systems -- time domain,
\end{flushleft}


\begin{flushleft}
Robust Stability and performance, Stabilizing controllers -- Coprime
\end{flushleft}


\begin{flushleft}
factorization, LQR, LQG problems
\end{flushleft}


\begin{flushleft}
Ricatti equations and solutions, H-infinity control and mu-synthesis,
\end{flushleft}


\begin{flushleft}
Linear matrix inequalities for robust control, Ricatti equation solution
\end{flushleft}


\begin{flushleft}
through LMI.
\end{flushleft}





\begin{flushleft}
ELL805 Networked and Multi-Agent Control Systems
\end{flushleft}


\begin{flushleft}
3 Credits (3-0-0)
\end{flushleft}


\begin{flushleft}
Overview of networked systems, Graph Theory Fundamentals, Graphbased Network Models, Network Optimization, Consensus Problem:
\end{flushleft}


\begin{flushleft}
cooperative control, leader-follower architecture.
\end{flushleft}


\begin{flushleft}
Control under Communication Constraints, Formation Control,
\end{flushleft}


\begin{flushleft}
Swarming and Flocking Collision Avoidance, Game Theoretic Control
\end{flushleft}


\begin{flushleft}
of Multi-Agent Systems, Applications: Multi-robot/vehicle coordination,
\end{flushleft}


\begin{flushleft}
Sensor Networks, Social Networks, Smart Grids, Biological Networks.
\end{flushleft}





\begin{flushleft}
EEL806 Scientific Visualization
\end{flushleft}


\begin{flushleft}
3 Credits (3-0-0)
\end{flushleft}


\begin{flushleft}
ELL806 Modeling and Control of Distributed Parameter
\end{flushleft}


\begin{flushleft}
Systems
\end{flushleft}


\begin{flushleft}
3 Credits (3-0-0)
\end{flushleft}


\begin{flushleft}
Overview: Motivation and examples (wave propagation, fluid flow,
\end{flushleft}


\begin{flushleft}
network traffic, electromagnetism), Modeling of Distributed Parameter
\end{flushleft}


\begin{flushleft}
Systems (DPS): Parabolic and Hyperbolic PDEs, Analytic and Numerical
\end{flushleft}


\begin{flushleft}
Solution of PDEs, Lyapunov stability of DPS Boundary control and
\end{flushleft}


\begin{flushleft}
Observer Design of DPS, Discretization of Distributed Parameter
\end{flushleft}


\begin{flushleft}
Models: Finite Difference, Finite Element and Boundary Elements,
\end{flushleft}


\begin{flushleft}
Reduction of FEM models, Applications: Control of systems with time
\end{flushleft}


\begin{flushleft}
delays, control of fluid flow, network control.
\end{flushleft}





\begin{flushleft}
ELL807 Stochastic Control
\end{flushleft}


\begin{flushleft}
3 Credits (3-0-0)
\end{flushleft}


\begin{flushleft}
Overview of stochastic systems with examples, Modeling of Stochastic
\end{flushleft}


\begin{flushleft}
Systems: Continuous and discrete-time models subjected to noise,
\end{flushleft}


\begin{flushleft}
Markov Decision Processes, Introduction to Stochastic Calculus and
\end{flushleft}


\begin{flushleft}
Stochastic Differential Equations, Stochastic Stability, Stochastic
\end{flushleft}


\begin{flushleft}
Optimal Control with complete and partial observations, finite and
\end{flushleft}


\begin{flushleft}
infinite horizon problems, Linear and nonlinear Filtering, Separation
\end{flushleft}


\begin{flushleft}
Principle, Linear quadratic Gaussian Problem, Stochastic Dynamic
\end{flushleft}


\begin{flushleft}
Programming, Stochastic Adaptive Control, Applications: Finance,
\end{flushleft}


\begin{flushleft}
operations research, biology.
\end{flushleft}





\begin{flushleft}
To be decided by the Instructor when floating this course: Can be
\end{flushleft}


\begin{flushleft}
anything that is related to systems and control engineering but is not
\end{flushleft}


\begin{flushleft}
covered in any of the established courses.
\end{flushleft}





\begin{flushleft}
Introduction to cyber security, information assurance, computer
\end{flushleft}


\begin{flushleft}
security and the associated threat, attack, adversary models, identity
\end{flushleft}


\begin{flushleft}
representation, management and access control, intrusion detection,
\end{flushleft}


\begin{flushleft}
security at different levels: network, system, user, program security,
\end{flushleft}


\begin{flushleft}
network security, wireless security, mobile security, hardware security
\end{flushleft}


\begin{flushleft}
and the security of cyber physical systems.
\end{flushleft}





\begin{flushleft}
ELD811 Major Project Part-I (Communication
\end{flushleft}


\begin{flushleft}
Engineering)
\end{flushleft}


\begin{flushleft}
6 Credits (0-0-12)
\end{flushleft}


\begin{flushleft}
ELD812 Major Project Part-II
\end{flushleft}


\begin{flushleft}
12 Credits (0-0-24)
\end{flushleft}


\begin{flushleft}
ELL812 Microwave Propagation and Systems
\end{flushleft}


\begin{flushleft}
3 Credits (3-0-0)
\end{flushleft}


\begin{flushleft}
Frequency bands and allocations. Earth and its effects on propagation.
\end{flushleft}


\begin{flushleft}
Atmosphere and its effects on propagation. Attenuation of millimeter
\end{flushleft}


\begin{flushleft}
waves. Line-of-sight communication links: system configuration,
\end{flushleft}


\begin{flushleft}
multiplexing, link design. Troposcatter propagation and links:
\end{flushleft}


\begin{flushleft}
Fadingand diversity reception, path profile and path loss, link design,
\end{flushleft}


\begin{flushleft}
signal design for fading channels.
\end{flushleft}





\begin{flushleft}
ELL813 Advanced Information Theory
\end{flushleft}


\begin{flushleft}
3 Credits (3-0-0)
\end{flushleft}


\begin{flushleft}
Capacity of single-user Gaussian multi-antenna deterministic channels
\end{flushleft}


\begin{flushleft}
and optimal strategies. Reliable transmission in single user state
\end{flushleft}


\begin{flushleft}
dependent channels. Capacity of Gaussian single-antenna fading
\end{flushleft}


\begin{flushleft}
channels with state (RX CSI, Full CSI). Capacity of single-antenna
\end{flushleft}


\begin{flushleft}
frequency-selective fading channels (OFDM modulation, waterfilling
\end{flushleft}


\begin{flushleft}
across frequency). Capacity of Gaussian multi-antenna single user
\end{flushleft}


\begin{flushleft}
fading channels (RX CSI only, Full CSI). Spatial multiplexing gain, array
\end{flushleft}


\begin{flushleft}
gain. Transmitter and receiver architectures, V-BLAST transmission,
\end{flushleft}


\begin{flushleft}
Zero-Forcing receiver, MMSE receiver, MMSESIC receiver. Optimality of
\end{flushleft}


\begin{flushleft}
MMSE-SIC. Capacity region of the multi-user Gaussian MAC channel.
\end{flushleft}


\begin{flushleft}
Capacity region of the multiuser Gaussian Broadcast channel (BC)
\end{flushleft}


\begin{flushleft}
with single-antenna terminals. Capacity of state dependent channels
\end{flushleft}


\begin{flushleft}
with non-causal side information (Gelfand-Pinsker coding). Dirty paper
\end{flushleft}


\begin{flushleft}
coding to pre-cancel known interference. MAC-BC duality. Capacity
\end{flushleft}


\begin{flushleft}
region of the multi-user Gaussian Broadcast channel with multiantenna terminals (Dirty paper coding achieves the capacity region).
\end{flushleft}


\begin{flushleft}
Capacity region of the Interference channel. There are no laboratory
\end{flushleft}


\begin{flushleft}
or design activities involved in this course.
\end{flushleft}





\begin{flushleft}
ELL814 Wireless Optical Communications
\end{flushleft}


\begin{flushleft}
3 Credits (3-0-0)
\end{flushleft}


\begin{flushleft}
General introduction, optical channel modeling, background noise
\end{flushleft}


\begin{flushleft}
calculations, Modulation techniques: M-PPM, OOK, mxn PAPM,
\end{flushleft}


\begin{flushleft}
subcarrier modulation, DPPM, DHPIM, DAPPM, psd and bandwidth
\end{flushleft}


\begin{flushleft}
requirement evaluation, Detection techniques - Photon counter,
\end{flushleft}


\begin{flushleft}
PMT, coherent techniques, bit error rate evaluation in presence of
\end{flushleft}


\begin{flushleft}
atmospheric turbulence, concept of adaptive threshold, effect of
\end{flushleft}


\begin{flushleft}
turbulence and weather conditions viz., drizzle, haze fog on error
\end{flushleft}


\begin{flushleft}
performance and channel capacity, link availability.
\end{flushleft}





\begin{flushleft}
ELL815 MIMO Wireless Communications
\end{flushleft}


\begin{flushleft}
3 Credits (3-0-0)
\end{flushleft}


\begin{flushleft}
Introduction to space-time diversity, MIMO channel, MIMO information
\end{flushleft}


\begin{flushleft}
theory, error probability analysis, transmit diversity and space-time
\end{flushleft}





208





\begin{flushleft}
\newpage
Electrical Engineering
\end{flushleft}





\begin{flushleft}
coding, linear STBC design, differential coding for MIMO, precoding,
\end{flushleft}


\begin{flushleft}
multiuser MIMO; There are no laboratory or design activities involved
\end{flushleft}


\begin{flushleft}
with this course.
\end{flushleft}





\begin{flushleft}
ELL816 Satellite Communication
\end{flushleft}


\begin{flushleft}
3 Credits (3-0-0)
\end{flushleft}


\begin{flushleft}
Introduction to satellite communication and orbital theory, satellite
\end{flushleft}


\begin{flushleft}
antennas, satellite link design, channel models for satellite links,
\end{flushleft}


\begin{flushleft}
modulation, multiple access techniques for satellite communication,
\end{flushleft}


\begin{flushleft}
VSAT, introduction to MIMO systems and error analysis, multiple
\end{flushleft}


\begin{flushleft}
antenna based satellite communication, hybrid satellite-terrestrial
\end{flushleft}


\begin{flushleft}
communication system.
\end{flushleft}





\begin{flushleft}
ELL823 Selected Topics in Information Processing-I
\end{flushleft}


\begin{flushleft}
3 Credits (3-0-0)
\end{flushleft}


\begin{flushleft}
ELV823 Special Modules in Information Processing-II
\end{flushleft}


\begin{flushleft}
1 Credit (1-0-0)
\end{flushleft}


\begin{flushleft}
ELL824 Selected Topics in Information Processing-II
\end{flushleft}


\begin{flushleft}
3 Credits (3-0-0)
\end{flushleft}


\begin{flushleft}
ELD830 Minor Project
\end{flushleft}


\begin{flushleft}
3 Credits (0-0-6)
\end{flushleft}





\begin{flushleft}
There are no laboratory or design activities involved with this course.
\end{flushleft}





\begin{flushleft}
ELL830 Issues in Deep Submicron VLSI Design
\end{flushleft}


\begin{flushleft}
3 Credits (3-0-0)
\end{flushleft}





\begin{flushleft}
ELL817 Access Networks
\end{flushleft}


\begin{flushleft}
3 Credits (3-0-0)
\end{flushleft}





\begin{flushleft}
VLSI Scaling rules and their impact: Short channel effect, Sub threshold
\end{flushleft}


\begin{flushleft}
leakage current, Gate leakage, VTH and body bias; Low power design:
\end{flushleft}


\begin{flushleft}
Technology level: 3D and 4 terminal MOSFETs, PDSOI, FDSOI, FINFET;
\end{flushleft}


\begin{flushleft}
Sub threshold leakage control: Transistor stacking in digital logic Multiple
\end{flushleft}


\begin{flushleft}
VTH, VDD designs, Dynamically adjustable VTH; Digital Circuit Design:
\end{flushleft}


\begin{flushleft}
Digital Sub-threshold Logic, Noise Immunity, Clock gating, Switching
\end{flushleft}


\begin{flushleft}
activity minimization; Analog Circuit Design: gm/ID Methodology
\end{flushleft}


\begin{flushleft}
for Design, Low power, low voltage opamp design, Subthreshold
\end{flushleft}


\begin{flushleft}
operation of opamps; Architecture level: Array Based Architectures,
\end{flushleft}


\begin{flushleft}
Parallel and Pipelined Architectures; Interconnects \& Noise: Capacitive
\end{flushleft}


\begin{flushleft}
\& Inductive coupling Analysis \& Optimization, Power/Ground Noise,
\end{flushleft}


\begin{flushleft}
L*di/dt noise, Power/Ground Placement Optimization, Decoupling.
\end{flushleft}





\begin{flushleft}
Contents: Types of access networks, wired (copper and optical) and
\end{flushleft}


\begin{flushleft}
wireless access networks, management, dimensioning and scaling of
\end{flushleft}


\begin{flushleft}
access networks, access network design.
\end{flushleft}





\begin{flushleft}
ELL818 Telecommunication Technologies
\end{flushleft}


\begin{flushleft}
3 Credits (3-0-0)
\end{flushleft}


\begin{flushleft}
Types of Data Networks, types of access and edge networks, core
\end{flushleft}


\begin{flushleft}
networks, OSS/NMS and Telecom Management network (TMN),
\end{flushleft}


\begin{flushleft}
Teletraffic Theory and Network analysis.
\end{flushleft}





\begin{flushleft}
ELP830 Semiconductor Processing Laboratory
\end{flushleft}


\begin{flushleft}
3 Credits (0-0-6)
\end{flushleft}





\begin{flushleft}
ELL819 Introduction to Plasmonics
\end{flushleft}


\begin{flushleft}
3 Credits (3-0-0)
\end{flushleft}


\begin{flushleft}
EM Waves, Maxwell's Equations, Origin of Permittivity, Evanescent
\end{flushleft}


\begin{flushleft}
Waves, Surface Plasmons, Scattering and Diffraction, Spoof Surface
\end{flushleft}


\begin{flushleft}
Plasmon, Extraordinary Optical Transmission, Numerical Simulations
\end{flushleft}


\begin{flushleft}
of Surface Plasmons, Negative Index Materials.
\end{flushleft}





\begin{flushleft}
Deposition of Semiconductor Materials and Metals: Sputter Deposition,
\end{flushleft}


\begin{flushleft}
E-Beam Deposition, and Thermal Evaporation; Photolithography;
\end{flushleft}


\begin{flushleft}
Electron-Beam Lithography; Epitaxial Growth of Semiconductors,
\end{flushleft}


\begin{flushleft}
Materials Characterization.
\end{flushleft}





\begin{flushleft}
ELV830 Special Module in Low Power IC Design
\end{flushleft}


\begin{flushleft}
1 Credit (1-0-0)
\end{flushleft}





\begin{flushleft}
ELL820 Photonic Switching and Networking
\end{flushleft}


\begin{flushleft}
3 Credits (3-0-0)
\end{flushleft}


\begin{flushleft}
Study of different types of networks, the enabling technologies and
\end{flushleft}


\begin{flushleft}
devices. Broadcast and Select network. Single and Multi-hop networks
\end{flushleft}


\begin{flushleft}
with example of Access networks, PONS etc., Wavelength Routing
\end{flushleft}


\begin{flushleft}
network, virtual topology, Metro and Wide area networks. Wavelength
\end{flushleft}


\begin{flushleft}
Routing and Assignment, Traffic Grooming and Protection, Network
\end{flushleft}


\begin{flushleft}
Control and Management, Optical packet and burst switching, Network
\end{flushleft}


\begin{flushleft}
Simulation Tools and Design guidelines.
\end{flushleft}





\begin{flushleft}
ELL821 Selected Topics in Communication Systems and
\end{flushleft}


\begin{flushleft}
Networking-I
\end{flushleft}


\begin{flushleft}
3 Credits (3-0-0)
\end{flushleft}





\begin{flushleft}
Special Module that focuses on special topics, development and
\end{flushleft}


\begin{flushleft}
Research problems of importance in the area of Low Power IC Design.
\end{flushleft}





\begin{flushleft}
ELD831 Major Project Part-I (Integrated Electronic
\end{flushleft}


\begin{flushleft}
Circuits)
\end{flushleft}


\begin{flushleft}
6 Credits (0-0-12)
\end{flushleft}


\begin{flushleft}
ELL831 CAD for VLSI, MEMS, and Nanoassembly
\end{flushleft}


\begin{flushleft}
3 Credits (3-0-0)
\end{flushleft}


\begin{flushleft}
Algorithms for design, modelling, and simulation ranging from VLSI,
\end{flushleft}


\begin{flushleft}
MEMS, to nanoassembly; computer aided nano-design for materials.
\end{flushleft}





\begin{flushleft}
ELP831 IEC Laboratory-I
\end{flushleft}


\begin{flushleft}
3 Credits (0-0-6)
\end{flushleft}





\begin{flushleft}
ELP821 Advanced Telecommunication Networks
\end{flushleft}


\begin{flushleft}
Laboratory
\end{flushleft}


\begin{flushleft}
3 Credits (0-1-4)
\end{flushleft}


\begin{flushleft}
To provide advanced level laboratory experiments in telecom signaling
\end{flushleft}


\begin{flushleft}
and transmission.
\end{flushleft}





\begin{flushleft}
Introduction to Cadence, Learning Cadence design framework and
\end{flushleft}


\begin{flushleft}
Virtuoso environment, Design with Virtuoso schematic editor, Layouts,
\end{flushleft}


\begin{flushleft}
Learning and applying Synopsys and Xilinx tools, Circuit simulation
\end{flushleft}


\begin{flushleft}
and SPICE.
\end{flushleft}





\begin{flushleft}
ELV821 Special Module in Communication Systems and
\end{flushleft}


\begin{flushleft}
Networking-II
\end{flushleft}


\begin{flushleft}
1 Credit (1-0-0)
\end{flushleft}





\begin{flushleft}
ELV831 Special Module in VLSI Testing
\end{flushleft}


\begin{flushleft}
1 Credit (1-0-0)
\end{flushleft}





\begin{flushleft}
ELL822 Selected Topics in Communication Systems
\end{flushleft}


\begin{flushleft}
and Networking-II
\end{flushleft}


\begin{flushleft}
3 Credits (3-0-0)
\end{flushleft}





\begin{flushleft}
ELD832 Major Project Part-II
\end{flushleft}


\begin{flushleft}
12 Credits (0-0-24)
\end{flushleft}





\begin{flushleft}
ELP822 Network Software Laboratory
\end{flushleft}


\begin{flushleft}
3 Credits (0-1-4)
\end{flushleft}





\begin{flushleft}
ELL832 Selected Topics in IEC-I
\end{flushleft}


\begin{flushleft}
3 Credits (3-0-0)
\end{flushleft}





\begin{flushleft}
Contents: CASE tools,client-server programming, middleware -- and
\end{flushleft}


\begin{flushleft}
use of Object Request Broker architectures, use of network emulators,
\end{flushleft}


\begin{flushleft}
using networks APIs Parlay/JAIN, service-oriented architectures,
\end{flushleft}


\begin{flushleft}
openflow and SDN, network management software design.
\end{flushleft}





\begin{flushleft}
ELP832 IEC Laboratory-II
\end{flushleft}


\begin{flushleft}
3 Credits (0-0-6)
\end{flushleft}





\begin{flushleft}
Special Module that focuses on special topics, development and
\end{flushleft}


\begin{flushleft}
Research problems of importance in the area of VLSI Testing.
\end{flushleft}





\begin{flushleft}
Introduction to Cadence, Learning Cadence design framework and
\end{flushleft}





209





\begin{flushleft}
\newpage
Electrical Engineering
\end{flushleft}





\begin{flushleft}
Virtuoso environment, Design with Virtuoso schematic editor, Layouts,
\end{flushleft}


\begin{flushleft}
Learning and applying Synopsys and Xilinx tools, Circuit simulation
\end{flushleft}


\begin{flushleft}
and SPICE.
\end{flushleft}





\begin{flushleft}
ELV832 Special Module in Machine Learning
\end{flushleft}


\begin{flushleft}
1 Credit (1-0-0)
\end{flushleft}


\begin{flushleft}
Special Module that focuses on special topics, development and
\end{flushleft}


\begin{flushleft}
Research problems of importance in this area.
\end{flushleft}





\begin{flushleft}
ELL833 CMOS RF IC Design
\end{flushleft}


\begin{flushleft}
3 Credits (3-0-0)
\end{flushleft}


\begin{flushleft}
Historical Aspects -- From Maxwell to Current Wireless standards; The
\end{flushleft}


\begin{flushleft}
bridge between communication system designer and RF IC Designer:
\end{flushleft}


\begin{flushleft}
a) Comm. system characterization, b)RF System Characterization;
\end{flushleft}


\begin{flushleft}
Transceiver Architectures -- Motivation for the individual blocks;
\end{flushleft}


\begin{flushleft}
Lumped, passive RLC, RF properties of MOS, Tuned Amplifiers; LNAs:
\end{flushleft}


\begin{flushleft}
Noise sources, Cascades and LNA Design; Mixers -- passive and active
\end{flushleft}


\begin{flushleft}
mixers ; Oscillators: Analysis Fundamentals, Inductors, LC Oscillators
\end{flushleft}


\begin{flushleft}
and VCOs; Frequency synthesizers: Principles, Integer N vs Fractional
\end{flushleft}


\begin{flushleft}
PLL, Design Concepts.
\end{flushleft}





\begin{flushleft}
ELP833 Device and Materials Characterization
\end{flushleft}


\begin{flushleft}
Laboratory
\end{flushleft}


\begin{flushleft}
3 Credits (0-0-6)
\end{flushleft}


\begin{flushleft}
Skill development in semiconductor modeling and characterization
\end{flushleft}


\begin{flushleft}
through hands on electrical characterization experiments. This
\end{flushleft}


\begin{flushleft}
includes wafer-level DC and RF characterization of p-n junction
\end{flushleft}


\begin{flushleft}
diode, MOS capacitor and transistor, photo-electric characterization
\end{flushleft}


\begin{flushleft}
of solar cells, TCAD and compact modeling of these devices, Materials
\end{flushleft}


\begin{flushleft}
Characterization (SEM, AFM, TEM, etc.).
\end{flushleft}





\begin{flushleft}
ELV833 Special Module in Semiconductor Business
\end{flushleft}


\begin{flushleft}
Management
\end{flushleft}


\begin{flushleft}
1 Credit (1-0-0)
\end{flushleft}


\begin{flushleft}
To educate students about semiconductor business. This includes
\end{flushleft}


\begin{flushleft}
business domains in semiconductors, latest business challenges,
\end{flushleft}


\begin{flushleft}
market trends and forecasts, business planning and incubation,
\end{flushleft}


\begin{flushleft}
execution and delivery, technical and financial analysis of R\&D ,
\end{flushleft}


\begin{flushleft}
business and finance models of chip manufacturing units (or fabs.),
\end{flushleft}


\begin{flushleft}
foundries, and solar power plants.
\end{flushleft}





\begin{flushleft}
ELL834 Selected Topics in IEC-II
\end{flushleft}


\begin{flushleft}
3 Credits (3-0-0)
\end{flushleft}


\begin{flushleft}
ELV834 Special Module in Nanoelectronics
\end{flushleft}


\begin{flushleft}
1 Credit (1-0-0)
\end{flushleft}


\begin{flushleft}
Special Module that focuses on special topics, development and
\end{flushleft}


\begin{flushleft}
Research problems of importance in the area of Nano Electronics.
\end{flushleft}





\begin{flushleft}
ELD850 Minor Project
\end{flushleft}


\begin{flushleft}
3 Credits (0-0-6)
\end{flushleft}


\begin{flushleft}
ELL850 Digital Control of Power Electronics and Drive
\end{flushleft}


\begin{flushleft}
Systems
\end{flushleft}


\begin{flushleft}
3 Credits (3-0-0)
\end{flushleft}


\begin{flushleft}
Review of Digital signal processors, Laplace transforms, Theory
\end{flushleft}


\begin{flushleft}
of sampling, z-transformations, sampling techniques, Digital PWM
\end{flushleft}


\begin{flushleft}
generation schemes, Realization of different PWM's using DSP's,
\end{flushleft}


\begin{flushleft}
Control of DC-DC Converters, Inverters, DC and Ac Machines.
\end{flushleft}





\begin{flushleft}
ELP850 Electrical Machines Laboratory
\end{flushleft}


\begin{flushleft}
1.5 Credits (0-0-3)
\end{flushleft}


\begin{flushleft}
Experiments on Electrical Machines and their control.
\end{flushleft}





\begin{flushleft}
ELT850 Industrial Training and Seminar
\end{flushleft}


\begin{flushleft}
3 Credits (0-0-6)
\end{flushleft}


\begin{flushleft}
ELD851 Major Project Part-I
\end{flushleft}


\begin{flushleft}
6 Credits (0-0-12)
\end{flushleft}





\begin{flushleft}
ELL851 Computer Aided Design of Electrical Machines
\end{flushleft}


\begin{flushleft}
3 Credits (3-0-0)
\end{flushleft}


\begin{flushleft}
Introduction of Standards and standardizations, specifications, frame
\end{flushleft}


\begin{flushleft}
size, basic design methodology and engineering considerations.
\end{flushleft}


\begin{flushleft}
Properties of electric, magnetic and insulating materials. Choice
\end{flushleft}


\begin{flushleft}
of materials, frames etc. Computerization of design procedures.
\end{flushleft}


\begin{flushleft}
Optimization techniques and their application to design problems.
\end{flushleft}


\begin{flushleft}
Design of large and h.p. motors. Database and knowledge based
\end{flushleft}


\begin{flushleft}
expert systems. Development of PC based software.
\end{flushleft}





\begin{flushleft}
ELP851 Power Electronics Laboratory
\end{flushleft}


\begin{flushleft}
1.5 Credits (0-0-3)
\end{flushleft}


\begin{flushleft}
Experiments on Power electronic converters and their control.
\end{flushleft}





\begin{flushleft}
ELD852 Major Project Part-II
\end{flushleft}


\begin{flushleft}
12 Credits (0-0-24)
\end{flushleft}


\begin{flushleft}
ELL852 Condition Monitoring of Electrical Machines
\end{flushleft}


\begin{flushleft}
3 Credits (3-0-0)
\end{flushleft}


\begin{flushleft}
The course includes the need for condition monitoring. Three main
\end{flushleft}


\begin{flushleft}
subdivisions of the course are types of fault and their symptoms,
\end{flushleft}


\begin{flushleft}
diagnostic methods to identify these faults and a deep signal
\end{flushleft}


\begin{flushleft}
processing analysis for fault diagnosis. The various components prone
\end{flushleft}


\begin{flushleft}
to fault are stator, rotor, shaft, gear box, bearing etc. The diagnosis
\end{flushleft}


\begin{flushleft}
methods includes diagnosis based on temperature, infrared signal,
\end{flushleft}


\begin{flushleft}
vibration, noise, motor current signature analysis etc. various signal
\end{flushleft}


\begin{flushleft}
processing techniques such as fuzzy logic, neural network from fault
\end{flushleft}


\begin{flushleft}
diagnosis point of view are also included in this course.
\end{flushleft}





\begin{flushleft}
ELP852 Electrical Drives Laboratory
\end{flushleft}


\begin{flushleft}
1.5 Credits (0-0-3)
\end{flushleft}


\begin{flushleft}
Experiments on drive systems with converter fed dc and ac drives
\end{flushleft}


\begin{flushleft}
and their control.
\end{flushleft}





\begin{flushleft}
ELL853 Advanced Topics in Electrical Machines
\end{flushleft}


\begin{flushleft}
3 Credits (3-0-0)
\end{flushleft}


\begin{flushleft}
Introduction to Advanced Topics in Electrical Machines, Synchronous
\end{flushleft}


\begin{flushleft}
Reluctance Machines, Hybrid Motors, Linear Motors, Super conducting
\end{flushleft}


\begin{flushleft}
Machines, PCB Motors, Micro motors, Written Pole Machines.
\end{flushleft}


\begin{flushleft}
Applications of all these advanced motors in field of Robotics,
\end{flushleft}


\begin{flushleft}
Automation, Electric Vehicles, pumping etc. The rating consideration
\end{flushleft}


\begin{flushleft}
and special advantages with these motors in various practical or
\end{flushleft}


\begin{flushleft}
field conditions is primary objective of this course. Other Advanced
\end{flushleft}


\begin{flushleft}
machines, Case Studies, Computer Aided Simulation of Electrical
\end{flushleft}


\begin{flushleft}
Machines are added for enhanced understanding of the topic.
\end{flushleft}





\begin{flushleft}
ELP853 DSP Based Control of Power Electronics and
\end{flushleft}


\begin{flushleft}
Drives Laboratory
\end{flushleft}


\begin{flushleft}
1.5 Credits (0-0-3)
\end{flushleft}


\begin{flushleft}
Experiments on the DSP/Digital signal controllers, Interfacing peripherals
\end{flushleft}


\begin{flushleft}
to DSP, Assembly language programming, Real-time voltage/ current,
\end{flushleft}


\begin{flushleft}
speed sensing signal and processing, PWM strategies realization
\end{flushleft}


\begin{flushleft}
through DSP and controlling power electronic and Drive Systems.
\end{flushleft}





\begin{flushleft}
ELL854 Selected Topics in Electrical Machines
\end{flushleft}


\begin{flushleft}
3 Credits (3-0-0)
\end{flushleft}


\begin{flushleft}
Recent developments in the area of electrical machines.
\end{flushleft}





\begin{flushleft}
ELP854 Electrical Machines CAD Laboratory
\end{flushleft}


\begin{flushleft}
3 Credits (0-1-4)
\end{flushleft}


\begin{flushleft}
Computer aided design of electrical machines.
\end{flushleft}





\begin{flushleft}
ELL855 High Power Converters
\end{flushleft}


\begin{flushleft}
3 Credits (3-0-0)
\end{flushleft}


\begin{flushleft}
Introduction to High Power devices -- IGBT, Thyristor, IGCT. Different
\end{flushleft}


\begin{flushleft}
topologies of high power converters -- Voltage Source and current
\end{flushleft}


\begin{flushleft}
source converter, 2- level converters, 3 level NPC converter, Cascaded
\end{flushleft}


\begin{flushleft}
H-Bridge Multilevel Converters, Modular multilevel converters. Pulse
\end{flushleft}





210





\begin{flushleft}
\newpage
Electrical Engineering
\end{flushleft}





\begin{flushleft}
width modulation techniques for high power converters -- Level
\end{flushleft}


\begin{flushleft}
shifted PWM, Phase shifted PWM, Space vector PWM for multilevel
\end{flushleft}


\begin{flushleft}
converters. Design of high power converter components, operational
\end{flushleft}


\begin{flushleft}
issues, fault tolerant operat.ion, reliability, mechanical design. Design
\end{flushleft}


\begin{flushleft}
of filters for high power converters. Relevant IEEE and IEC standards
\end{flushleft}


\begin{flushleft}
for high power converters.
\end{flushleft}





\begin{flushleft}
ELL872 Selected Topics in Power System
\end{flushleft}


\begin{flushleft}
3 Credits (3-0-0)
\end{flushleft}


\begin{flushleft}
To be decided by the Instructor when floating this course: It can be
\end{flushleft}


\begin{flushleft}
anything that is related to power system, but is not covered in any
\end{flushleft}


\begin{flushleft}
of the established courses.
\end{flushleft}





\begin{flushleft}
ELL873 Power System Transient
\end{flushleft}


\begin{flushleft}
3 Credits (3-0-0)
\end{flushleft}





\begin{flushleft}
ELP855 Smart-Grids Laboratory
\end{flushleft}


\begin{flushleft}
3 Credits (0-1-4)
\end{flushleft}


\begin{flushleft}
Experiments related to smart-grids measurement and control.
\end{flushleft}





\begin{flushleft}
ELL856 Advanced Topics in Power Electronics
\end{flushleft}


\begin{flushleft}
3 Credits (3-0-0)
\end{flushleft}


\begin{flushleft}
Upcoming power electronic devices- SiC and GaN devices. Design of
\end{flushleft}


\begin{flushleft}
power electronic converters, Introduction to soft-switching in dc-dc
\end{flushleft}


\begin{flushleft}
and dc-ac applications.
\end{flushleft}





\begin{flushleft}
ELL857 Selected Topics in Power Electronics
\end{flushleft}


\begin{flushleft}
3 Credits (3-0-0)
\end{flushleft}





\begin{flushleft}
Origin and nature of transients and surges. Lumped and distributed
\end{flushleft}


\begin{flushleft}
circuit representations. Line energisation and de-energisation
\end{flushleft}


\begin{flushleft}
transients, current chopping, short-line faults, trapped charge effects,
\end{flushleft}


\begin{flushleft}
effect of source, control of transients, Lightening, effect of tower
\end{flushleft}


\begin{flushleft}
footing resistance, travelling waves, insulation coordination, circuit
\end{flushleft}


\begin{flushleft}
breakers duty, surge arresters, overvoltage limiting devices.
\end{flushleft}





\begin{flushleft}
ELL874 Power System Reliability
\end{flushleft}


\begin{flushleft}
3 Credits (3-0-0)
\end{flushleft}


\begin{flushleft}
Review of basic probability theory, reliability theory, network modeling
\end{flushleft}


\begin{flushleft}
and evaluation of simple and complex systems, generation system
\end{flushleft}


\begin{flushleft}
reliability -- concept of loss of load probability, energy not served,
\end{flushleft}


\begin{flushleft}
transmission system reliability, component failure, distribution system
\end{flushleft}


\begin{flushleft}
reliability with perfect and imperfect switching.
\end{flushleft}





\begin{flushleft}
Recent developments in power electronics.
\end{flushleft}





\begin{flushleft}
ELL858 Advanced Topics in Electric Drives
\end{flushleft}


\begin{flushleft}
3 Credits (3-0-0)
\end{flushleft}


\begin{flushleft}
Advanced PWM Techniques. Control of switched reluctance motor drives.
\end{flushleft}


\begin{flushleft}
Control of slip-ring induction motor drives. Self-commissioning and
\end{flushleft}


\begin{flushleft}
self-adaptation techniques in drives. Sensor-less techniques in drives.
\end{flushleft}


\begin{flushleft}
Fault tolerant controllers and converters. Other recent topics on drives.
\end{flushleft}





\begin{flushleft}
ELL859 Selected Topics in Electric Drives
\end{flushleft}


\begin{flushleft}
3 Credits (3-0-0)
\end{flushleft}





\begin{flushleft}
ELD880 Major Project Part-I
\end{flushleft}


\begin{flushleft}
6 Credits (0-0-12)
\end{flushleft}


\begin{flushleft}
ELL880 Special Topics in Computers-I
\end{flushleft}


\begin{flushleft}
3 Credits (3-0-0)
\end{flushleft}





\begin{flushleft}
Recent developments in the area of electric drives.
\end{flushleft}





\begin{flushleft}
ELS880 Independent Study
\end{flushleft}


\begin{flushleft}
3 Credits (3-0-0)
\end{flushleft}





\begin{flushleft}
ELD870 Minor Project-I
\end{flushleft}


\begin{flushleft}
3 Credits (0-0-6)
\end{flushleft}





\begin{flushleft}
ELD881 Major Project Part-II
\end{flushleft}


\begin{flushleft}
12 Credits (0-0-24)
\end{flushleft}





\begin{flushleft}
To be decided by the project supervisor
\end{flushleft}





\begin{flushleft}
ELL881 Special Topics in Computers-II
\end{flushleft}


\begin{flushleft}
3 Credits (3-0-0)
\end{flushleft}





\begin{flushleft}
ELL870 Restructured Power System
\end{flushleft}


\begin{flushleft}
3 Credits (3-0-0)
\end{flushleft}


\begin{flushleft}
Philosophy of market models, Concepts in micro-economics, Centralized
\end{flushleft}


\begin{flushleft}
and de-centralized Dispatch Philosophies, Congestion Management,
\end{flushleft}


\begin{flushleft}
Ancillary Service Management, Transmission Pricing Methods, Loss
\end{flushleft}


\begin{flushleft}
Allocation Algorithms, Locational Marginal Price (LMP) calculation
\end{flushleft}


\begin{flushleft}
and properties, Financial Transmission Rights (FTRs), Transmission
\end{flushleft}


\begin{flushleft}
Expansion Planning, Market Power, Working of International Power
\end{flushleft}


\begin{flushleft}
Markets, Restructuring Issues in Indian Power Sector.
\end{flushleft}





\begin{flushleft}
ELP870 Power System Lab-I
\end{flushleft}


\begin{flushleft}
3 Credits (0-1-4)
\end{flushleft}


\begin{flushleft}
Power flow studies, fault studies, state estimation, security analysis,
\end{flushleft}


\begin{flushleft}
robust power flow methods, power flow with uncertain data.
\end{flushleft}





\begin{flushleft}
ELD871 Major Project Part-I
\end{flushleft}


\begin{flushleft}
6 Credits (0-0-12)
\end{flushleft}





\begin{flushleft}
ELL882 Large-Scale Machine Learning
\end{flushleft}


\begin{flushleft}
3 Credits (3-0-0)
\end{flushleft}


\begin{flushleft}
Introduction, Randomized Algorithms, Matrix Approximations (low-rank
\end{flushleft}


\begin{flushleft}
approximation, decomposition, sparse matrices, matrix completion),
\end{flushleft}


\begin{flushleft}
Large Scale Optimization, Kernel Methods (fast training), Boosted
\end{flushleft}


\begin{flushleft}
Decision trees, Dimensionality Reduction (linear and nonlinear
\end{flushleft}


\begin{flushleft}
methods), Distributed Gibbs Sampling, Sparse Methods/Streaming
\end{flushleft}


\begin{flushleft}
(sparse coding...); Applications.
\end{flushleft}





\begin{flushleft}
ELL883 Embedded Intelligence
\end{flushleft}


\begin{flushleft}
3 Credits (3-0-0)
\end{flushleft}


\begin{flushleft}
Basics of embedded, learning, and adaptive systems; sensors, nature
\end{flushleft}


\begin{flushleft}
of dynamic environments, hardware aspects.
\end{flushleft}





\begin{flushleft}
To be decided by the project supervisor.
\end{flushleft}





\begin{flushleft}
ELL884 Information Retrieval
\end{flushleft}


\begin{flushleft}
3 Credits (3-0-0)
\end{flushleft}





\begin{flushleft}
ELL871 Distribution System Operation and planning
\end{flushleft}


\begin{flushleft}
3 Credits (3-0-0)
\end{flushleft}





\begin{flushleft}
Motivation, evaluation, classical IR models, Indexing, ML techniques,
\end{flushleft}


\begin{flushleft}
Semantic search, MIR, Web-scale information retrieval, Query
\end{flushleft}


\begin{flushleft}
processing, User interfaces.
\end{flushleft}





\begin{flushleft}
Structure of distribution system, modeling of system components,
\end{flushleft}


\begin{flushleft}
power flow, fault studies, state estimation, optimal power flow, optimal
\end{flushleft}


\begin{flushleft}
feeder reconfiguration, optimum resources planning, incorporation of
\end{flushleft}


\begin{flushleft}
DGs in operation and planning.
\end{flushleft}





\begin{flushleft}
ELP871 Power System Lab-II
\end{flushleft}


\begin{flushleft}
3 Credits (0-1-4)
\end{flushleft}


\begin{flushleft}
ELD872 Major Project Part-II
\end{flushleft}


\begin{flushleft}
12 Credits (0-0-24)
\end{flushleft}


\begin{flushleft}
To be decided by the project supervisor.
\end{flushleft}





\begin{flushleft}
ELL885 Machine Learning for Computational Finance
\end{flushleft}


\begin{flushleft}
3 Credits (3-0-0)
\end{flushleft}


\begin{flushleft}
Time series forecasting techniques, Introduction to Portfolio theory,
\end{flushleft}


\begin{flushleft}
Trading Systems, Optimisation methods, Risk Management, Machine
\end{flushleft}


\begin{flushleft}
Learning for Algorithmic Trading.
\end{flushleft}





\begin{flushleft}
ELL886 Big Data Systems
\end{flushleft}


\begin{flushleft}
3 Credits (3-0-0)
\end{flushleft}


\begin{flushleft}
Introduction; Hadoop, Map-Reduce, GFS/HDFS, Bigtable/HBASE;
\end{flushleft}


\begin{flushleft}
Extension of Map-Reduce: iMap-reduce (iterative), incremental mapreduce. SQL and Data-parallel programming, DryadLINQ. Data-flow
\end{flushleft}





211





\begin{flushleft}
\newpage
Electrical Engineering
\end{flushleft}





\begin{flushleft}
parallelism vs. message passing. Data locality. Memory hierarchies.
\end{flushleft}


\begin{flushleft}
Sequential versus random access to secondary storage. NoSQL
\end{flushleft}


\begin{flushleft}
systems. NewSQL systems. Finding similar items and LSH; Search
\end{flushleft}


\begin{flushleft}
Technology: link analysis and Page-rank algorithm; Large Scale Graph
\end{flushleft}


\begin{flushleft}
Processing; Mining Streaming Data and Realtime analytics: Window
\end{flushleft}


\begin{flushleft}
semantics and window joins. Sampling and approximating aggregates
\end{flushleft}


\begin{flushleft}
(no joins). Querying histograms. Maintaining histograms of streams.
\end{flushleft}


\begin{flushleft}
Use of Haar wavelets. Incremental and online query processing:
\end{flushleft}


\begin{flushleft}
online aggregation.
\end{flushleft}





\begin{flushleft}
ELL893 Cyber-Physical Systems
\end{flushleft}


\begin{flushleft}
3 Credits (3-0-0)
\end{flushleft}





\begin{flushleft}
ELL887 Cloud Computing
\end{flushleft}


\begin{flushleft}
3 Credits (3-0-0)
\end{flushleft}





\begin{flushleft}
Network performance models and classifications, Hidden Markov
\end{flushleft}


\begin{flushleft}
Models (HMM), Delay and throughput analysis using Markov models,
\end{flushleft}


\begin{flushleft}
Performance analysis with multi-class traffic, Renewal theory and
\end{flushleft}


\begin{flushleft}
regenerative processes, Performance analysis with semi-Markov traffic
\end{flushleft}


\begin{flushleft}
characteristics, Network performance analysis with interactive servers,
\end{flushleft}


\begin{flushleft}
Practical network traffic characterization, Network performance
\end{flushleft}


\begin{flushleft}
stability, Introduction to dynamic programming, Example network
\end{flushleft}


\begin{flushleft}
modeling scenarios in various engineering applications.
\end{flushleft}





\begin{flushleft}
Introduction; Example System: Apple iCloud, Amazon-AWS;
\end{flushleft}


\begin{flushleft}
Fundamental Concepts: Cloud Characteristics, Cloud delivery models;
\end{flushleft}


\begin{flushleft}
Cloud Enabling Technology: broad-band network,virtualisation
\end{flushleft}


\begin{flushleft}
technology; Cloud Infrastructure Mechanisms: Logical Network
\end{flushleft}


\begin{flushleft}
Perimeter, Virtual Server, Cloud Storage Devices;Cloud Architecture:
\end{flushleft}


\begin{flushleft}
Workload Distribution Architecture, Resource Pooling Architecture,
\end{flushleft}


\begin{flushleft}
Dynamic Scalability Architecture, Hypervisor Clustering Architecture,
\end{flushleft}


\begin{flushleft}
Load Balanced Virtual Server Instances Architecture, Elastic Resource
\end{flushleft}


\begin{flushleft}
Capacity Architecture, Elastic Disk Provisioning Architecture,
\end{flushleft}


\begin{flushleft}
Redundant Storage Architecture; Cloud Security: Encryption, Identity
\end{flushleft}


\begin{flushleft}
and Access management, Cloud-based Security Groups; Working
\end{flushleft}


\begin{flushleft}
with Cloud: Building Service Platforms, Cost Metrics, Pricing Models.
\end{flushleft}





\begin{flushleft}
ELL888 Advanced Machine Learning
\end{flushleft}


\begin{flushleft}
3 Credits (3-0-0)
\end{flushleft}


\begin{flushleft}
Advanced topics in machine learning, including Nonlinear Dimension
\end{flushleft}


\begin{flushleft}
Reduction, Maximum Entropy, Exponential Family Models, Graphical
\end{flushleft}


\begin{flushleft}
Models; Computational Learning Theory, Structured Support Vector
\end{flushleft}


\begin{flushleft}
Machines, Feature Selection, Kernel Selection, Meta-Learning, MultiTask Learning, Semi-Supervised Learning, Reinforcement Learning,
\end{flushleft}


\begin{flushleft}
Approximate Inference, Clustering, and Boosting.
\end{flushleft}





\begin{flushleft}
ELL889 Protocol Engineering
\end{flushleft}


\begin{flushleft}
3 Credits (3-0-0)
\end{flushleft}


\begin{flushleft}
Principles, stages, specification formalisms (UML, SDL, ASN.1) of
\end{flushleft}


\begin{flushleft}
telecom protocol design, protocol software development process,
\end{flushleft}


\begin{flushleft}
computer aided protocol engineering, verification and testing of
\end{flushleft}


\begin{flushleft}
protocols, object oriented techniques in protocol development, kernel
\end{flushleft}


\begin{flushleft}
level development and programming of protocols.
\end{flushleft}





\begin{flushleft}
ELL890 Computational Neuroscience
\end{flushleft}


\begin{flushleft}
3 Credits (3-0-0)
\end{flushleft}


\begin{flushleft}
Fundamentals of brain anatomy and physiology, signals of brain, Brain
\end{flushleft}


\begin{flushleft}
signal recording and imaging techniques, Human experimentation
\end{flushleft}


\begin{flushleft}
study design, Processing the X-D neural data, Machine learning
\end{flushleft}


\begin{flushleft}
approaches, Graph theory and neural networks, Multivariate pattern
\end{flushleft}


\begin{flushleft}
analysis in 4D Imaging data, Statistical inferences, student projects
\end{flushleft}


\begin{flushleft}
and presentations.
\end{flushleft}





\begin{flushleft}
ELL891 Computational Linguistics
\end{flushleft}


\begin{flushleft}
3 Credits (3-0-0)
\end{flushleft}


\begin{flushleft}
Introduction to language and linguistics; Mathematical foundations:
\end{flushleft}


\begin{flushleft}
statistics and machine learning; Introduction to corpus-based
\end{flushleft}


\begin{flushleft}
computational linguistics; Lexical analysis; Syntactic analysis; Semantic
\end{flushleft}


\begin{flushleft}
analysis; Discourse analysis; Psycholinguistics, computational cognitive
\end{flushleft}


\begin{flushleft}
models of language processing and evolution; Assignments and
\end{flushleft}


\begin{flushleft}
practical exercises involving the application of these techniques to
\end{flushleft}


\begin{flushleft}
real-word corpora.
\end{flushleft}





\begin{flushleft}
ELL892 Internet Technologies
\end{flushleft}


\begin{flushleft}
3 Credits (3-0-0)
\end{flushleft}


\begin{flushleft}
Web and service oriented architectures, dynamic web site programming
\end{flushleft}


\begin{flushleft}
(client side and server side), web application development, web
\end{flushleft}


\begin{flushleft}
based repositories, UI design, XML, Web 2.0 and the semantic web,
\end{flushleft}


\begin{flushleft}
applications.
\end{flushleft}





\begin{flushleft}
Introduction: core principles behind CPSs; Specification of CPS, CPS
\end{flushleft}


\begin{flushleft}
models: Continuous, Discrete, Hybrid, Compositional; Abstraction and
\end{flushleft}


\begin{flushleft}
System Architecture, Design by Invariants, Sensing and Fusion, Cloud
\end{flushleft}


\begin{flushleft}
of Robots/CPS; Case Studies: Healthcare, Smart Grid, Transportation.
\end{flushleft}





\begin{flushleft}
ELL894 Network Performance Modeling and Analysis
\end{flushleft}


\begin{flushleft}
3 Credits (3-0-0)
\end{flushleft}





\begin{flushleft}
ELD895 MS Research Project
\end{flushleft}


\begin{flushleft}
36 Credits (0-0-72)
\end{flushleft}


\begin{flushleft}
ELL895 Network Security
\end{flushleft}


\begin{flushleft}
3 Credits (3-0-0)
\end{flushleft}


\begin{flushleft}
Introduction to cryptography, public key distribution and user
\end{flushleft}


\begin{flushleft}
authentication, TLS and wireless network security, secure email
\end{flushleft}


\begin{flushleft}
and PGP, IP security, system security - intrusion, malicious software
\end{flushleft}


\begin{flushleft}
and firewalls.
\end{flushleft}





\begin{flushleft}
ELL896 Mobile Computing
\end{flushleft}


\begin{flushleft}
3 Credits (3-0-0)
\end{flushleft}


\begin{flushleft}
Overview of mobile computing; introduction to GSM, 3GPP, 4G LTE,
\end{flushleft}


\begin{flushleft}
LTE-A standards; wireless networking protocols: mobile IP, ad hoc
\end{flushleft}


\begin{flushleft}
networks, wireless TCP; cognitive radio networks; data broadcasting;
\end{flushleft}


\begin{flushleft}
location and context awareness; QoS, QoE; disconnected or weakly
\end{flushleft}


\begin{flushleft}
connected operations; protocol and resource optimization; wireless
\end{flushleft}


\begin{flushleft}
security issues.
\end{flushleft}





\begin{flushleft}
ELL897 Network Management
\end{flushleft}


\begin{flushleft}
3 Credits (3-0-0)
\end{flushleft}


\begin{flushleft}
Activities, methods, operational procedures, tools, communications
\end{flushleft}


\begin{flushleft}
interfaces, protocols, and human resources that pertain to the
\end{flushleft}


\begin{flushleft}
operation, administration, maintenance, and provisioning of
\end{flushleft}


\begin{flushleft}
communications networks, network management standards,
\end{flushleft}


\begin{flushleft}
technologies; functional areas of fault management, configuration
\end{flushleft}


\begin{flushleft}
management, accounting management, performance management,
\end{flushleft}


\begin{flushleft}
and security management, major Internet and telecommunications
\end{flushleft}


\begin{flushleft}
standards for network management: SNMPv3, RMON, CMIP
\end{flushleft}


\begin{flushleft}
and TMN.
\end{flushleft}





\begin{flushleft}
ELL898 Pervasive Computing
\end{flushleft}


\begin{flushleft}
3 Credits (3-0-0)
\end{flushleft}


\begin{flushleft}
Introduction, computer and network architectures for pervasive
\end{flushleft}


\begin{flushleft}
computing, mobile computing mechanisms, human-computer
\end{flushleft}


\begin{flushleft}
interaction using speech and vision, pervasive software systems,
\end{flushleft}


\begin{flushleft}
location mechanisms, practical techniques for security and userauthentication, and experimental pervasive computing systems.
\end{flushleft}





\begin{flushleft}
ELL899 Testing and Fault Tolerance
\end{flushleft}


\begin{flushleft}
3 Credits (3-0-0)
\end{flushleft}


\begin{flushleft}
Introduction to testing, simulation, fault simulation, automatic
\end{flushleft}


\begin{flushleft}
test pattern generator, sequential logic bests, automatic test
\end{flushleft}


\begin{flushleft}
equipment, design for testability, Built-In-Self-Test (BIST),
\end{flushleft}


\begin{flushleft}
behavioral test and verification.
\end{flushleft}





212





\begin{flushleft}
\newpage
Department of Humanities and Social Sciences
\end{flushleft}


\begin{flushleft}
HUL101 English in Practice
\end{flushleft}


\begin{flushleft}
3 Credits (2-0-2)
\end{flushleft}


\begin{flushleft}
Verb structures and patterns, avoiding common errors, vocabulary
\end{flushleft}


\begin{flushleft}
building, spelling patterns, developing writing skills (composition,
\end{flushleft}


\begin{flushleft}
letter writing) etc. developing listening skills.
\end{flushleft}





\begin{flushleft}
HUP102 Psychology Laboratory
\end{flushleft}


\begin{flushleft}
1 Credit (0-0-2)
\end{flushleft}


\begin{flushleft}
To familiarize students with psychological concepts through practical
\end{flushleft}


\begin{flushleft}
training in a laboratory through experiments pertaining to cognitive
\end{flushleft}


\begin{flushleft}
psychology, environmental psychology and physiological psychology.
\end{flushleft}





\begin{flushleft}
HUL232 Modern Indian Fiction in Translation
\end{flushleft}


\begin{flushleft}
4 Credits (3-1-0)
\end{flushleft}


\begin{flushleft}
Pre--requisites: NLN101
\end{flushleft}


\begin{flushleft}
Students would be introduced to the conditions, beginning in 19th
\end{flushleft}


\begin{flushleft}
century colonial rule in India, which led to the emergent Indian middleclass intelligentsia to experiment with European forms of literature but
\end{flushleft}


\begin{flushleft}
striving for an alternative expression. Indian languages became the
\end{flushleft}


\begin{flushleft}
medium through which writers sought to address issues of identity,
\end{flushleft}


\begin{flushleft}
tradition, modernity, gender, the rural and the urban, the private and
\end{flushleft}


\begin{flushleft}
the public. The course will study the various experiments in narration,
\end{flushleft}


\begin{flushleft}
language, characterization and style undertaken by authors to shape
\end{flushleft}


\begin{flushleft}
these themes.
\end{flushleft}





\begin{flushleft}
HUL235 Rise of the Novel
\end{flushleft}


\begin{flushleft}
4 Credits (3-1-0)
\end{flushleft}


\begin{flushleft}
Pre--requisites: NLN101
\end{flushleft}





\begin{flushleft}
HUL211 Introduction to Economics
\end{flushleft}


\begin{flushleft}
4 Credits (3-1-0)
\end{flushleft}


\begin{flushleft}
Pre-requisites: NLN101
\end{flushleft}


\begin{flushleft}
Current economic problems. Alternative economic systems. An
\end{flushleft}


\begin{flushleft}
overview of the economy. The market mechanism. National product
\end{flushleft}


\begin{flushleft}
and income. Consumption, savings and investment. Determination
\end{flushleft}


\begin{flushleft}
of national income. Aggregate demand and supply. Fiscal policy. The
\end{flushleft}


\begin{flushleft}
nature of money and monetary policy. Inflation and unemployment.
\end{flushleft}


\begin{flushleft}
Basic concepts of price theory. Determination of price by supply and
\end{flushleft}


\begin{flushleft}
demand. Elasticity of demand and supply. Theory of production. Theory
\end{flushleft}


\begin{flushleft}
of costs. Pricing in competitive and monopoly markets. The gains from
\end{flushleft}


\begin{flushleft}
international trade. Theory of exchange rates. Balance of payments.
\end{flushleft}


\begin{flushleft}
Economic growth, and development. Inequality and poverty.	
\end{flushleft}





\begin{flushleft}
HUL212 Microeconomics
\end{flushleft}


\begin{flushleft}
4 Credits (3-1-0)
\end{flushleft}


\begin{flushleft}
Pre-requisites: NLN101
\end{flushleft}


\begin{flushleft}
Micro versus macroeconomics. Theory of consumer behavior and
\end{flushleft}


\begin{flushleft}
demand. Consumer preferences. Indifference curve. Consumer
\end{flushleft}


\begin{flushleft}
equilibrium. Demand function. Income and substitution effects. The
\end{flushleft}


\begin{flushleft}
Slutsky equation. Market demand. Elasticities. Average and marginal
\end{flushleft}


\begin{flushleft}
revenue. Revealed preference theory of firm. Production functions.
\end{flushleft}


\begin{flushleft}
Law of variable proportions. Laws of return to scale. Isoquants. Input
\end{flushleft}


\begin{flushleft}
substitution. Equilibrium of the firm. Expansion path. Cost function.
\end{flushleft}


\begin{flushleft}
Theory of costs. Short Run and Long run costs. Shape of LAC.
\end{flushleft}


\begin{flushleft}
Economies and diseconomies of scale. Market equilibrium under perfect
\end{flushleft}


\begin{flushleft}
competition. Equilibrium under alternative forms of market. Monopoly:
\end{flushleft}


\begin{flushleft}
pure and discriminating. Monopolistic competition. Oligopoly.
\end{flushleft}





\begin{flushleft}
HUL213 Macroeconomics
\end{flushleft}


\begin{flushleft}
4 Credits (3-1-0)
\end{flushleft}


\begin{flushleft}
Pre-requisites: NLN101
\end{flushleft}


\begin{flushleft}
Major economic problems. National income accounting. Expenditure
\end{flushleft}


\begin{flushleft}
and income approaches to GNP. Measuring inflation and unemployment.
\end{flushleft}


\begin{flushleft}
Determination of the equilibrium level of income. Consumption
\end{flushleft}


\begin{flushleft}
function. Investment demand. Aggregate demand and equilibrium
\end{flushleft}


\begin{flushleft}
output. The multiplier process. Government sector. Fiscal policy. Tax
\end{flushleft}


\begin{flushleft}
receipts and Transfer payments. Foreign spending. Money, interest
\end{flushleft}


\begin{flushleft}
and income. Functions of money. Definition of money. Reserve Bank
\end{flushleft}


\begin{flushleft}
of India and Commercial Banks. Creation of money. The instruments
\end{flushleft}


\begin{flushleft}
of monetary control. The demand for money. Investment expenditure
\end{flushleft}


\begin{flushleft}
and rate of interest. The IS curve. Money market and the LM curve.
\end{flushleft}


\begin{flushleft}
Liquidity trap. The IS-LM model. Derivation of the aggregate demand
\end{flushleft}


\begin{flushleft}
curve. Monetary and fiscal polices. Keynesian versus monetarist views.
\end{flushleft}


\begin{flushleft}
The aggregate supply function: Keynesian and classical. Inflation and
\end{flushleft}


\begin{flushleft}
unemployment. Stagflation. The Phillips curve. The long-run Phillips
\end{flushleft}


\begin{flushleft}
curve. Inflation expectations. The rational expectations.
\end{flushleft}





\begin{flushleft}
HUL217 History of Economic Thought
\end{flushleft}


\begin{flushleft}
4 Credits (3-1-0)
\end{flushleft}


\begin{flushleft}
This course will introduce ideas in the history of economic thought,
\end{flushleft}


\begin{flushleft}
from mercantilism, socialism, communism, capitalism to the rise of
\end{flushleft}


\begin{flushleft}
modern economic theory (e.g. utiliarianism), along with questions
\end{flushleft}


\begin{flushleft}
about economic theory (especially from behavioural sciences).
\end{flushleft}





\begin{flushleft}
The socio-politcal contexts which lead to the rise of the novel in
\end{flushleft}


\begin{flushleft}
Europe -- the emergence of print, the expansion of literacy, and
\end{flushleft}


\begin{flushleft}
the establishment of capitalism. Close reading of selected texts
\end{flushleft}


\begin{flushleft}
accompanying concepts like the rise of the modern individual, varied
\end{flushleft}


\begin{flushleft}
narrative techniques and national consciousness. The emerging subgenres of the novel -- the comic, the picaresque, the historical novel
\end{flushleft}


\begin{flushleft}
and the realist novel. The linkage of the novel to the colonial project
\end{flushleft}


\begin{flushleft}
and its influence on world literature.
\end{flushleft}





\begin{flushleft}
HUL236 Introduction to Drama
\end{flushleft}


\begin{flushleft}
4 Credits (3-1-0)
\end{flushleft}


\begin{flushleft}
Pre--requisites: NLN101
\end{flushleft}


\begin{flushleft}
Brief history of the development and importance of drama in Western
\end{flushleft}


\begin{flushleft}
and Indian contexts. Readings from both ancient and contemporary
\end{flushleft}


\begin{flushleft}
drama theorists. Generic differences between different forms of drama
\end{flushleft}


\begin{flushleft}
such as tragedy, comedy, realist, {`}folk', Absurd, etc. Detailed study of
\end{flushleft}


\begin{flushleft}
important examples of different forms of drama.
\end{flushleft}





\begin{flushleft}
HUL237 Contemporary Fiction
\end{flushleft}


\begin{flushleft}
4 Credits (3-1-0)
\end{flushleft}


\begin{flushleft}
Pre--requisites: NLN101
\end{flushleft}


\begin{flushleft}
- Approaches to contemporary fiction - Looking at contemporary styles
\end{flushleft}


\begin{flushleft}
- realism, modernism, postmodernism - Contemporary versions of
\end{flushleft}


\begin{flushleft}
classical genres - the diary, epistolary form, epic, etc. - the relationship
\end{flushleft}


\begin{flushleft}
of society with science and technology through fiction - the relationship
\end{flushleft}


\begin{flushleft}
between self and society through fiction -Race, nationality, culture and
\end{flushleft}


\begin{flushleft}
identity - contemporary forms.
\end{flushleft}





\begin{flushleft}
HUL239 Indian fiction In English
\end{flushleft}


\begin{flushleft}
4 Credits	 (3-1-0)
\end{flushleft}


\begin{flushleft}
Pre--requisites: NLN101
\end{flushleft}


\begin{flushleft}
The course involves a detailed study of 3-4 texts corresponding to the
\end{flushleft}


\begin{flushleft}
distinct phases of literary activity in the genre: the early period of the
\end{flushleft}


\begin{flushleft}
1940s and 50s in which writers like Mulk Raj Anand, Raja Rao and
\end{flushleft}


\begin{flushleft}
R.K. Narayan made their presence felt, before Salman Rushdie, and
\end{flushleft}


\begin{flushleft}
more quietly, Amitav Ghosh and Vikram Seth, erupted into the scene
\end{flushleft}


\begin{flushleft}
in the 1980s, spawning a generation of writers attaining international
\end{flushleft}


\begin{flushleft}
acclaim - Arundhati Roy, Aravind Adiga, Kiran Desai, and many others.
\end{flushleft}


\begin{flushleft}
Some of the questions that will be addressed are: Who constitutes the
\end{flushleft}


\begin{flushleft}
main audience for this writing, and (how) does the writing cater to it?
\end{flushleft}


\begin{flushleft}
How does one position the expatriate Indian writer both residing and
\end{flushleft}


\begin{flushleft}
publishing abroad? How does English become an Indian language? Is
\end{flushleft}


\begin{flushleft}
there a thematic congruence in the novels that fall under this category,
\end{flushleft}


\begin{flushleft}
and does it differ from the thematic concerns of novels written in other
\end{flushleft}


\begin{flushleft}
Indian languages? Students will be encouraged to read a novel in at
\end{flushleft}


\begin{flushleft}
least one other Indian language in order to allow them to pose these
\end{flushleft}


\begin{flushleft}
questions in a more pointed manner.
\end{flushleft}





\begin{flushleft}
HUL240 Indian English Poetry
\end{flushleft}


\begin{flushleft}
4 Credits (3-1-0)
\end{flushleft}


\begin{flushleft}
Pre--requisites: NLN101
\end{flushleft}


\begin{flushleft}
The aim of this course will be to read the poems of Indian English
\end{flushleft}


\begin{flushleft}
Writers (pre and post-Independence), with specific reference to the
\end{flushleft}





213





\begin{flushleft}
\newpage
Humanities and Social Sciences
\end{flushleft}





\begin{flushleft}
articulation of their identity. Some of the perspectives from which
\end{flushleft}


\begin{flushleft}
the poems will be discussed include the notion of home (childhood,
\end{flushleft}


\begin{flushleft}
family and ancestors); land (history, geography, community, caste
\end{flushleft}


\begin{flushleft}
and contemporary politics); language (the dialogue between the
\end{flushleft}


\begin{flushleft}
different languages in the creative repertoire of the poets); and culture
\end{flushleft}


\begin{flushleft}
(ritual, traditions, legends and myths). The course will also look at the
\end{flushleft}


\begin{flushleft}
differences between the resident and expatriate poets vis-a-vis the
\end{flushleft}


\begin{flushleft}
conflicts and resolutions as expressed in their poems.
\end{flushleft}





\begin{flushleft}
HUL242 Fundamentals of language sciences
\end{flushleft}


\begin{flushleft}
4 Credits (3-1-0)
\end{flushleft}


\begin{flushleft}
Pre--requisites: NLN101
\end{flushleft}


\begin{flushleft}
This course provides answers to basic questions about the nature and
\end{flushleft}


\begin{flushleft}
constitution of human language in the mind/brain of native speakers.
\end{flushleft}


\begin{flushleft}
Varied aspects of linguistic organization, including structures of
\end{flushleft}


\begin{flushleft}
sounds, words and sentences are considered to understand the core
\end{flushleft}


\begin{flushleft}
universals of all languages as well as their variations. Cases of feral
\end{flushleft}


\begin{flushleft}
children, language deficiencies and cognition-language interactions
\end{flushleft}


\begin{flushleft}
are also highlighted.
\end{flushleft}





\begin{flushleft}
HUL243 Language and Communication
\end{flushleft}


\begin{flushleft}
4 Credits (3-1-0)
\end{flushleft}


\begin{flushleft}
Pre--requisites: NLN101
\end{flushleft}


\begin{flushleft}
This course offers a wide-ranging introduction to, and analysis of,
\end{flushleft}


\begin{flushleft}
varieties of spoken and written language. From political oratory to
\end{flushleft}


\begin{flushleft}
examination answer scripts to computer codes, not to mention jokes,
\end{flushleft}


\begin{flushleft}
riddles and poetry, human language offers an amazingly rich set of
\end{flushleft}


\begin{flushleft}
structures for expressing and conveying our thoughts, intentions and
\end{flushleft}


\begin{flushleft}
desires. The course will consider some of these linguistic structures
\end{flushleft}


\begin{flushleft}
and communicative strategies in detail, beginning with early childhood
\end{flushleft}


\begin{flushleft}
development. How is it that children in every culture learn language so
\end{flushleft}


\begin{flushleft}
effortlessly despite its great complexity? The course aims to introduce
\end{flushleft}


\begin{flushleft}
students to a set of theories that address this and other puzzles and
\end{flushleft}


\begin{flushleft}
mysteries in the arena of language studies. Finally, since a central
\end{flushleft}


\begin{flushleft}
focus of the course is communication, it will strive to be as interactive
\end{flushleft}


\begin{flushleft}
as possible, with lots of scope for the discussion and working out of
\end{flushleft}


\begin{flushleft}
actual {`}problems' in language use.
\end{flushleft}





\begin{flushleft}
HUL251 Introduction to Logic
\end{flushleft}


\begin{flushleft}
4 Credits (3-1-0)
\end{flushleft}


\begin{flushleft}
In this course, students are introduced to fundamentals of informal
\end{flushleft}


\begin{flushleft}
logic and verbal analysis, material and formal fallacies of reasoning
\end{flushleft}


\begin{flushleft}
often found in ordinary discourse, deductive and Inductive reasoning,
\end{flushleft}


\begin{flushleft}
validity and soundness, formal rules and principles of the deductive
\end{flushleft}


\begin{flushleft}
system of Aristotelian logic, traditional square of opposition;
\end{flushleft}


\begin{flushleft}
propositional calculus; first order predicate calculus; the modern
\end{flushleft}


\begin{flushleft}
square of opposition and the problem of existential import; identity
\end{flushleft}


\begin{flushleft}
and definite descriptions; methods for formulating natural language
\end{flushleft}


\begin{flushleft}
arguments in symbolic forms and techniques for checking their validity;
\end{flushleft}


\begin{flushleft}
various meta-logical theorems and their proofs.
\end{flushleft}





\begin{flushleft}
HUL253 Moral Literacy and Moral choices
\end{flushleft}


\begin{flushleft}
4 Credits (3-1-0)
\end{flushleft}


\begin{flushleft}
Pre--requisites: NLN101
\end{flushleft}





\begin{flushleft}
part of understanding the nature of values. The discussion of the
\end{flushleft}


\begin{flushleft}
above issues will be influenced by three philosophical orientational
\end{flushleft}


\begin{flushleft}
perspectives: Anglo-American Analytic, Continental Phenomenological
\end{flushleft}


\begin{flushleft}
and Classical Indian.
\end{flushleft}





\begin{flushleft}
HUL258 Social and Political Philosophy
\end{flushleft}


\begin{flushleft}
4 Credits (3-1-0)
\end{flushleft}


\begin{flushleft}
Pre--requisites: NLN101
\end{flushleft}


\begin{flushleft}
As closely aligned areas in philosophy-- social philosophy with the
\end{flushleft}


\begin{flushleft}
role of individual in society and political philosophy with the role
\end{flushleft}


\begin{flushleft}
of government- this course bridges divides between social theory,
\end{flushleft}


\begin{flushleft}
political philosophy, and the history of social and political thought as
\end{flushleft}


\begin{flushleft}
also between empirical and normative analysis through perspectives
\end{flushleft}


\begin{flushleft}
from metaphysics, epistemology and axiology. A range of socio-political
\end{flushleft}


\begin{flushleft}
thinkers, theories and concepts will be taught. It will provide a broad
\end{flushleft}


\begin{flushleft}
survey of fundamental social and political questions in current contexts
\end{flushleft}


\begin{flushleft}
discussing philosophical issues central to political thought and radical
\end{flushleft}


\begin{flushleft}
critiques of current political theories.
\end{flushleft}





\begin{flushleft}
HUL261 Introduction to Psychology
\end{flushleft}


\begin{flushleft}
4 Credits (3-1-0)
\end{flushleft}


\begin{flushleft}
Pre--requisites: NLN101
\end{flushleft}


\begin{flushleft}
Psychological Science- Assumptions, schools, methods of doing
\end{flushleft}


\begin{flushleft}
psychology research, The relationship between brain, body and
\end{flushleft}


\begin{flushleft}
mental functioning, Sensation, perception and making sense of
\end{flushleft}


\begin{flushleft}
the world, Consciousness, Life span development and motor and
\end{flushleft}


\begin{flushleft}
language development, Nature and nurture controversy, The
\end{flushleft}


\begin{flushleft}
learning process and some important explanations of how we learn,
\end{flushleft}


\begin{flushleft}
Meaning of motivation and explanations, Theories of emotions and
\end{flushleft}


\begin{flushleft}
expression and regulation of emotions, Basic cognitive processes,
\end{flushleft}


\begin{flushleft}
Language development, why we remember and why we forget- some
\end{flushleft}


\begin{flushleft}
explanations, Different kinds of intelligence, explanations of creativity,
\end{flushleft}


\begin{flushleft}
Differences among individuals and explanations for personality
\end{flushleft}


\begin{flushleft}
differences, Application of psychology to everyday life- enhancing
\end{flushleft}


\begin{flushleft}
health and well-being, performance, social relations, and sensitivity
\end{flushleft}


\begin{flushleft}
to environmental, social and cultural contexts.
\end{flushleft}





\begin{flushleft}
HUL265 Theories of Personality
\end{flushleft}


\begin{flushleft}
4 Credits (3-1-0)
\end{flushleft}


\begin{flushleft}
Pre--requisites: NLN101
\end{flushleft}


\begin{flushleft}
Personality: Meaning \& Assessment. Psychoanalytic \& NeoPsychoanalytic Approach ; Behavioural Approach; Cognitive Approach;
\end{flushleft}


\begin{flushleft}
Social- Cognitive Approach; Humanistic Approach; The Traits
\end{flushleft}


\begin{flushleft}
Approach; Models of healthy personality: the notion of the mature
\end{flushleft}


\begin{flushleft}
person, the self-actualizing personality etc. Personality disorders;
\end{flushleft}


\begin{flushleft}
Psychotherapeutic techniques and Yoga \& Meditation; Indian
\end{flushleft}


\begin{flushleft}
perspective on personality; Personality in Socio- cultural context.
\end{flushleft}





\begin{flushleft}
HUL267 Positive Psychology
\end{flushleft}


\begin{flushleft}
4 Credits (3-1-0)
\end{flushleft}


\begin{flushleft}
Pre--requisites: NLN101
\end{flushleft}





\begin{flushleft}
This is primarily a course in applied ethics. It will focus primarily
\end{flushleft}


\begin{flushleft}
on questions like: What is the meaning of right action? Can ethical
\end{flushleft}


\begin{flushleft}
assertions be true or false? Is morality relative to society? Or can we
\end{flushleft}


\begin{flushleft}
say that acts have universal moral content? The course discussions
\end{flushleft}


\begin{flushleft}
will help to demonstrate that morality is not always self-evident and
\end{flushleft}


\begin{flushleft}
that rational morality must come in place of taboo based moralities.
\end{flushleft}





\begin{flushleft}
Positive Psychology: A historical and contextual overview;
\end{flushleft}


\begin{flushleft}
Relationship between Indian Psychology and Positive Psychology;
\end{flushleft}


\begin{flushleft}
Correlates and predictors of life satisfaction and subjective wellbeing across various cultures; Latest researches on self-esteem,
\end{flushleft}


\begin{flushleft}
optimism, flow, post-traumatic growth, positive ageing, character
\end{flushleft}


\begin{flushleft}
strengths, etc.; Major theories and models within positive psychology
\end{flushleft}


\begin{flushleft}
-- Self-Determination theory, Broaden-and-Build theory, Authentic
\end{flushleft}


\begin{flushleft}
Happiness, Psychological Well-being, etc., Interpersonal character
\end{flushleft}


\begin{flushleft}
strengths \& well- being; Specific Coping Approaches: meditation,
\end{flushleft}


\begin{flushleft}
yoga and spirituality; Future of the Field.
\end{flushleft}





\begin{flushleft}
HUL256 Critical Thinking
\end{flushleft}


\begin{flushleft}
4 Credits (3-1-0)
\end{flushleft}


\begin{flushleft}
Pre--requisites: NLN101
\end{flushleft}





\begin{flushleft}
HUL271 Introduction to Sociology
\end{flushleft}


\begin{flushleft}
4 Credits (3-1-0)
\end{flushleft}


\begin{flushleft}
Pre--requisites: NLN101
\end{flushleft}





\begin{flushleft}
What makes philosophical thinking radically critical? Investigation of
\end{flushleft}


\begin{flushleft}
the nature of knowledge about the world and justification of knowledge
\end{flushleft}


\begin{flushleft}
claims. Metaphysical understanding of the Absolute and Mind-Body
\end{flushleft}


\begin{flushleft}
relation. The nature of ethical and aesthetic beliefs and attitudes as
\end{flushleft}





\begin{flushleft}
Introduction to the discipline of sociology and its emergence as
\end{flushleft}


\begin{flushleft}
a science in the context of the development of modern industrial
\end{flushleft}


\begin{flushleft}
society in Europe. Introduction to key classical and contemporary
\end{flushleft}


\begin{flushleft}
theorists in Sociology.
\end{flushleft}





214





\begin{flushleft}
\newpage
Humanities and Social Sciences
\end{flushleft}





\begin{flushleft}
HUL272 Introduction to Sociology of India
\end{flushleft}


\begin{flushleft}
4 Credits (3-1-0)
\end{flushleft}


\begin{flushleft}
Pre--requisites: NLN101
\end{flushleft}


\begin{flushleft}
This course will begin with a discussion on the various constructions
\end{flushleft}


\begin{flushleft}
of Indian society from colonial to contemporary times. The structural
\end{flushleft}


\begin{flushleft}
and cultural dimensions of Indian society are explored at the level of
\end{flushleft}


\begin{flushleft}
village, city, region, nation and civilization. Sources of differentiation,
\end{flushleft}


\begin{flushleft}
diversity and unity are explored through institutions such as caste,
\end{flushleft}


\begin{flushleft}
class and tribe; kinship, family, marriage and gender systems,
\end{flushleft}


\begin{flushleft}
religious traditions and political organisations. Transformations in
\end{flushleft}


\begin{flushleft}
these institutions are analysed and fault lines explored by studying
\end{flushleft}


\begin{flushleft}
contemporary issues of secularism, communalism, religious
\end{flushleft}


\begin{flushleft}
conversions, caste and identity movements. The sociological
\end{flushleft}


\begin{flushleft}
perspective remains key to interpreting changes in Indian society in
\end{flushleft}


\begin{flushleft}
the era of globalization and rapid economic change.
\end{flushleft}





\begin{flushleft}
HUL282 System and Structure: An Introduction to
\end{flushleft}


\begin{flushleft}
Communication Theory
\end{flushleft}


\begin{flushleft}
4 Credits (3-1-0)
\end{flushleft}


\begin{flushleft}
Pre-requisites: NLN101
\end{flushleft}


\begin{flushleft}
This course is an introduction to theories of communication for which
\end{flushleft}


\begin{flushleft}
there is not sufficient time in the other communication courses, which
\end{flushleft}


\begin{flushleft}
are mainly applied in their orientation. This is an interdisciplinary
\end{flushleft}


\begin{flushleft}
course. It will examine how the notion of communication is used in
\end{flushleft}


\begin{flushleft}
different disciplines in the humanities and the social sciences. It will
\end{flushleft}


\begin{flushleft}
intersect with problems of organizational structure, linguistic structure,
\end{flushleft}


\begin{flushleft}
interpersonal structure and the problem of what is involved in changing
\end{flushleft}


\begin{flushleft}
a structure. This course will include no components of remedial English,
\end{flushleft}


\begin{flushleft}
business correspondence or skill building activities. Only those really
\end{flushleft}


\begin{flushleft}
interested in theoretical questions should enroll.
\end{flushleft}





\begin{flushleft}
HUL286 Social Science Approaches to Development
\end{flushleft}


\begin{flushleft}
4 Credits (3-1-0)
\end{flushleft}


\begin{flushleft}
Pre--requisites: NLN101
\end{flushleft}





\begin{flushleft}
HUL274 Re-thinking the Indian Tradition
\end{flushleft}


\begin{flushleft}
4 Credits (3-1-0)
\end{flushleft}


\begin{flushleft}
Pre-requisites: NLN101
\end{flushleft}


\begin{flushleft}
The examination of sources, the structure, the texts and exemplars
\end{flushleft}


\begin{flushleft}
of the Indian tradition provide the theoretical framework for the
\end{flushleft}


\begin{flushleft}
discussion of contemporary political and social issues. These are
\end{flushleft}


\begin{flushleft}
economic development and social justice religion and the nation,
\end{flushleft}


\begin{flushleft}
communalism and secularism, caste class and gender equity and so
\end{flushleft}


\begin{flushleft}
on. The political misuse of tradition in programs of reform and revival
\end{flushleft}


\begin{flushleft}
both in the past and in modern times will be highlighted to underline
\end{flushleft}


\begin{flushleft}
the need for rethinking tradition in an academically serious manner.
\end{flushleft}





\begin{flushleft}
HUL275 Environment, Development and Society
\end{flushleft}


\begin{flushleft}
4 Credits (3-1-0)
\end{flushleft}


\begin{flushleft}
Pre--requisites: NLN101
\end{flushleft}


\begin{flushleft}
Students will be exposed to contemporary themes and debates
\end{flushleft}


\begin{flushleft}
on connection between environment, development, and society;
\end{flushleft}


\begin{flushleft}
industrialization and risk society; challenge of sustainable development;
\end{flushleft}


\begin{flushleft}
perception of the environment, dependence for livelihood, identity,
\end{flushleft}


\begin{flushleft}
and power on natural resources; social ecology; what is the
\end{flushleft}


\begin{flushleft}
role of religion in determining our world view and relation with
\end{flushleft}


\begin{flushleft}
the environment?; recognition of indigenous knowledge; rise of
\end{flushleft}


\begin{flushleft}
environmental movements, development projects and recent conflict
\end{flushleft}


\begin{flushleft}
over natural resources; understanding major environmental disasters
\end{flushleft}


\begin{flushleft}
and industrial accidents; global climate change negotiations; gender
\end{flushleft}


\begin{flushleft}
and environment.
\end{flushleft}





\begin{flushleft}
HUL276 Sociology of Knowledge
\end{flushleft}


\begin{flushleft}
4 Credits (3-1-0)
\end{flushleft}


\begin{flushleft}
Pre-requisites: NLN101
\end{flushleft}


\begin{flushleft}
The de-mystification of science as a privileged form of knowledge
\end{flushleft}


\begin{flushleft}
since Copernicus. Re-examining the laboratory, the factory and the
\end{flushleft}


\begin{flushleft}
nation-state, structures linked to the West-European model of science.
\end{flushleft}


\begin{flushleft}
Examining systems deemed ethno-science or folk-lore, to set up a
\end{flushleft}


\begin{flushleft}
dialogue with institutionalized science. Comparing science with religion
\end{flushleft}


\begin{flushleft}
as forms of knowledge having competing power over human belief
\end{flushleft}


\begin{flushleft}
and action. Examining Traditional Knowledge (TK) systems and their
\end{flushleft}


\begin{flushleft}
relevance for global economy.
\end{flushleft}





\begin{flushleft}
HUL281 Technology and Governance
\end{flushleft}


\begin{flushleft}
4 Credits (3-1-0)
\end{flushleft}


\begin{flushleft}
Pre--requisites: NLN101
\end{flushleft}


\begin{flushleft}
The course will begin with theories and concepts on the use of
\end{flushleft}


\begin{flushleft}
technologies to improve governance such as efficiency, transparency,
\end{flushleft}


\begin{flushleft}
empowerment, economic gains, decentralization etc. It will
\end{flushleft}


\begin{flushleft}
discuss the concepts of democracy and governance, corruption
\end{flushleft}


\begin{flushleft}
and accountability. Examples and case studies from topics such
\end{flushleft}


\begin{flushleft}
as information and communication technologies for development,
\end{flushleft}


\begin{flushleft}
electronic governance, electronic voting, electronic databases (UID),
\end{flushleft}


\begin{flushleft}
web portals, community radio etc. Public-private partnerships,
\end{flushleft}


\begin{flushleft}
regulation of technology by the state, surveillance, and the role of
\end{flushleft}


\begin{flushleft}
stakeholders in the policy making process.
\end{flushleft}





\begin{flushleft}
Distinction between {`}growth' and {`}development'; historical
\end{flushleft}


\begin{flushleft}
genesis and evolution of the concept of development; theories
\end{flushleft}


\begin{flushleft}
of development and underdevelopment; the political nature of
\end{flushleft}


\begin{flushleft}
the development process. Role of state, market, culture and civil
\end{flushleft}


\begin{flushleft}
society in development. Gendered nature of development. Postindependence Indian experience (centralized planning and socialism)
\end{flushleft}


\begin{flushleft}
of development; selected comparisons with China, East Asia, South
\end{flushleft}


\begin{flushleft}
Asia, Africa, Latin America. Explaining India's slow progress in
\end{flushleft}


\begin{flushleft}
human and social development, poor record in reduction of poverty
\end{flushleft}


\begin{flushleft}
and inequality. Impact of globalization, foreign aid and economic
\end{flushleft}


\begin{flushleft}
reform on India's development. Experiments with decentralization
\end{flushleft}


\begin{flushleft}
and sustainable development.
\end{flushleft}





\begin{flushleft}
HUL289 Science, Technology and Human Development
\end{flushleft}


\begin{flushleft}
4 Credits (3-1-0)
\end{flushleft}


\begin{flushleft}
Pre--requisites: NLN101
\end{flushleft}


\begin{flushleft}
The course will begin by identifying various dimensions of human
\end{flushleft}


\begin{flushleft}
development and mapping the state of India and the world on
\end{flushleft}


\begin{flushleft}
these indicators. It will then discuss theories about how science
\end{flushleft}


\begin{flushleft}
and technology (S\&T) have shaped human development historically
\end{flushleft}


\begin{flushleft}
and the dynamics of technological change. Relationship between
\end{flushleft}


\begin{flushleft}
innovation and human development will be discussed using examples
\end{flushleft}


\begin{flushleft}
from the appropriate technology movement, health, education,
\end{flushleft}


\begin{flushleft}
nutrition, energy, environment, and others. Gender dimensions of S\&T,
\end{flushleft}


\begin{flushleft}
indigenous knowledge, and radical critiques of S\&T will be discussed.
\end{flushleft}





\begin{flushleft}
HUL290 Technology and Culture
\end{flushleft}


\begin{flushleft}
4 Credits (3-1-0)
\end{flushleft}


\begin{flushleft}
Pre-requisites: NLN101
\end{flushleft}


\begin{flushleft}
To examine the relationship between technology and culture through
\end{flushleft}


\begin{flushleft}
a consideration of modern/current developments in various specific
\end{flushleft}


\begin{flushleft}
areas: e.g. Biotechnology and Medicine, IT, AI \& Robotics, Fashion
\end{flushleft}


\begin{flushleft}
Technology, Magic Technology, Communications, Defense and Space
\end{flushleft}


\begin{flushleft}
Research. To focus on the roles played by the IITs themselves in
\end{flushleft}


\begin{flushleft}
creating {`}knowledge societies' - that is, in influencing, formulating
\end{flushleft}


\begin{flushleft}
and envisioning the links between technological {`}solutions' and
\end{flushleft}


\begin{flushleft}
socio-cultural {`}problems' especially in the Indian context. Here we
\end{flushleft}


\begin{flushleft}
will discuss, for example: Patent Laws, Gender Issues, Environmental
\end{flushleft}


\begin{flushleft}
Ethics, Design(er) and Person(al) Technological Aesthetics,
\end{flushleft}


\begin{flushleft}
Technologies for the Disabled, Educational Technologies.
\end{flushleft}





\begin{flushleft}
HUL310 Selected Topics in Policy Studies
\end{flushleft}


\begin{flushleft}
3 Credits (3-0-0)
\end{flushleft}


\begin{flushleft}
Pre--requisites: Any Two courses from HUL2XX category
\end{flushleft}


\begin{flushleft}
Allocation Preferences : HUL281, HUL289, HUL290
\end{flushleft}


\begin{flushleft}
The course will introduce students to selected topics in Policy Studies
\end{flushleft}


\begin{flushleft}
as decided by the instructor.
\end{flushleft}





215





\begin{flushleft}
\newpage
Humanities and Social Sciences
\end{flushleft}





\begin{flushleft}
HUL311 Applied Game Theory
\end{flushleft}


\begin{flushleft}
3 Credits (3-0-0)
\end{flushleft}


\begin{flushleft}
Pre--requisites: Any Two courses from HUL2XX category
\end{flushleft}


\begin{flushleft}
Allocation Preferences : HUL211, HUL212, HUL213
\end{flushleft}





\begin{flushleft}
HUL318 Public Finance and Public Economics
\end{flushleft}


\begin{flushleft}
3 Credits (3-0-0)
\end{flushleft}


\begin{flushleft}
Pre--requisites: Any Two courses from HUL2XX category
\end{flushleft}


\begin{flushleft}
Allocation Preferences : HUL211, HUL212, HUL213
\end{flushleft}





\begin{flushleft}
This module introduces students in economics and other social sciences
\end{flushleft}


\begin{flushleft}
to game theory, a theory of interactive decision making. This module
\end{flushleft}


\begin{flushleft}
provides students with the basic solution concepts for different types
\end{flushleft}


\begin{flushleft}
of non-cooperative games, including static and dynamic games under
\end{flushleft}


\begin{flushleft}
complete and incomplete information. The basic solution concepts that
\end{flushleft}


\begin{flushleft}
this module covers are Nash equilibrium, subgame perfect equilibrium,
\end{flushleft}


\begin{flushleft}
Bayesian equilibrium, and perfect Bayesian equilibrium. This module
\end{flushleft}


\begin{flushleft}
emphasizes the applications of game theory to economics, such as
\end{flushleft}


\begin{flushleft}
duopolies, bargaining, and auctions.
\end{flushleft}





\begin{flushleft}
The course is aimed at developing an understanding of the basics in
\end{flushleft}


\begin{flushleft}
Public Economics and Public Finance. Public economics is the study of
\end{flushleft}


\begin{flushleft}
government policy from the points of view of economic efficiency and
\end{flushleft}


\begin{flushleft}
equity. The course deals with the nature of government intervention
\end{flushleft}


\begin{flushleft}
and its implications for allocation, distribution and stabilization.
\end{flushleft}


\begin{flushleft}
Inherently, this study involves a formal analysis of government taxation
\end{flushleft}


\begin{flushleft}
and expenditures. The subject encompasses a host of topics including
\end{flushleft}


\begin{flushleft}
public goods, market failures and externalities. The course is divided
\end{flushleft}


\begin{flushleft}
into two sections, one dealing with the theory of public economics
\end{flushleft}


\begin{flushleft}
and the other with the Indian public finances. 	
\end{flushleft}





\begin{flushleft}
HUL312 Distribution and Growth
\end{flushleft}


\begin{flushleft}
3 Credits (3-0-0)
\end{flushleft}


\begin{flushleft}
Pre--requisites: Any Two courses from HUL2XX category
\end{flushleft}


\begin{flushleft}
Allocation Preferences : HUL211, HUL212, HUL213
\end{flushleft}





\begin{flushleft}
HUL320 Selected Topics in Economics
\end{flushleft}


\begin{flushleft}
3 Credits (3-0-0)
\end{flushleft}


\begin{flushleft}
Pre--requisites: Any Two courses from HUL2XX category
\end{flushleft}


\begin{flushleft}
Allocation Preferences : HUL211, HUL212, HUL213
\end{flushleft}





\begin{flushleft}
Though empirical questions are central in motivating the issues on
\end{flushleft}


\begin{flushleft}
distribution, this course will mostly draw from theory. Papers published
\end{flushleft}


\begin{flushleft}
in established journals will cover the major references for this course.
\end{flushleft}


\begin{flushleft}
It will start from some empirical pattern of development (Kuznet's
\end{flushleft}


\begin{flushleft}
hypothesis), country experiences, etc. to motivate the subject. Then it
\end{flushleft}


\begin{flushleft}
will try to understand the process of distribution, growth and structural
\end{flushleft}


\begin{flushleft}
change using standard macroeconomic models. This course will be
\end{flushleft}


\begin{flushleft}
heavily dependent on Mathematics - mainly calculus.
\end{flushleft}





\begin{flushleft}
HUL314 International Economics
\end{flushleft}


\begin{flushleft}
3 Credits (3-0-0)
\end{flushleft}


\begin{flushleft}
Pre--requisites: Any Two courses from HUL2XX category
\end{flushleft}


\begin{flushleft}
Allocation Preferences : HUL211, HUL212, HUL213
\end{flushleft}


\begin{flushleft}
Basic concepts of national income accounting, money, and balance of
\end{flushleft}


\begin{flushleft}
payments; output and exchange-rate determination under fixed and
\end{flushleft}


\begin{flushleft}
flexible exchange-rate regimes; fiscal and monetary policies in an
\end{flushleft}


\begin{flushleft}
open economy; international capital movements and their impacts;
\end{flushleft}


\begin{flushleft}
Case Studies: East Asian crisis, global financial crisis; theories of
\end{flushleft}


\begin{flushleft}
international trade including factor-proportions and economies
\end{flushleft}


\begin{flushleft}
of scale; the international trading regime and its implications for
\end{flushleft}


\begin{flushleft}
developing countries.	
\end{flushleft}





\begin{flushleft}
HUL315 Econometric Methods
\end{flushleft}


\begin{flushleft}
3 Credits (3-0-0)
\end{flushleft}


\begin{flushleft}
Pre--requisites: Any Two courses from HUL2XX category
\end{flushleft}


\begin{flushleft}
Allocation Preferences : HUL211, HUL212, HUL213
\end{flushleft}





\begin{flushleft}
The course will introduce students to selected topics in Economics as
\end{flushleft}


\begin{flushleft}
decided by the instructor.
\end{flushleft}





\begin{flushleft}
HUL331 Modernist Fiction
\end{flushleft}


\begin{flushleft}
3 Credits (3-0-0)
\end{flushleft}


\begin{flushleft}
Pre--requisites: Any Two courses from HUL2XX category
\end{flushleft}


\begin{flushleft}
Allocation Preferences : HUL231, HUL232, HUL235, HUL236,
\end{flushleft}


\begin{flushleft}
HUL237, HUL240, HUL239
\end{flushleft}


\begin{flushleft}
The course will undertake a detailed study of some of the most iconic
\end{flushleft}


\begin{flushleft}
Modernist novels by writers such as Virginia Woolf, James Joyce, Franz
\end{flushleft}


\begin{flushleft}
Kafka and Samuel Beckett. It will examine the radical new ways in
\end{flushleft}


\begin{flushleft}
which they grappled with language, turned towards interiority, and
\end{flushleft}


\begin{flushleft}
pushed, in the process, narrative art to its very limits. The discussion
\end{flushleft}


\begin{flushleft}
will highlight the experimental quality of Modernist literature, as well
\end{flushleft}


\begin{flushleft}
as situate it within the context of its emergence - the two world wars,
\end{flushleft}


\begin{flushleft}
the development of psychoanalysis, the growth of metropolitan cities,
\end{flushleft}


\begin{flushleft}
and scientific and technological advancements.
\end{flushleft}





\begin{flushleft}
HUL332 Fantasy Literature
\end{flushleft}


\begin{flushleft}
3 Credits (3-0-0)
\end{flushleft}


\begin{flushleft}
Pre--requisites: Any Two courses from HUL2XX category
\end{flushleft}


\begin{flushleft}
Allocation Preferences : HUL231, HUL232, HUL235, HUL236,
\end{flushleft}


\begin{flushleft}
HUL237, HUL240, HUL239
\end{flushleft}


\begin{flushleft}
Major Themes of Fantasy: Archetypes and Myths; Motifs - journeys,
\end{flushleft}


\begin{flushleft}
theology, devices and aides; creation of alternate worlds; treatment
\end{flushleft}


\begin{flushleft}
of time and space; close readings of individual texts.
\end{flushleft}





\begin{flushleft}
Basics of sample survey; variance and covariance; correlation
\end{flushleft}


\begin{flushleft}
coefficient; simple regression analysis; Gauss-Markov theorem;
\end{flushleft}


\begin{flushleft}
estimation of regression coefficients; confidence intervals and
\end{flushleft}


\begin{flushleft}
hypothesis testing in regression analysis; type-I and type-II
\end{flushleft}


\begin{flushleft}
errors; transformation of variables; multiple regression analysis;
\end{flushleft}


\begin{flushleft}
multicollinearity, heteroscedaticity, dummy variables, basics of timeseries analysis.
\end{flushleft}





\begin{flushleft}
HUL316 Indian Economic Problems and Policies
\end{flushleft}


\begin{flushleft}
3 Credits (3-0-0)
\end{flushleft}


\begin{flushleft}
Pre--requisites: Any Two courses from HUL2XX category
\end{flushleft}


\begin{flushleft}
Allocation Preferences : HUL211, HUL212, HUL213
\end{flushleft}


\begin{flushleft}
The course is aimed at developing an understanding of the economic
\end{flushleft}


\begin{flushleft}
issues in a range of economic activities in the Indian economy. The
\end{flushleft}


\begin{flushleft}
themes that can be covered include performance of the Indian
\end{flushleft}


\begin{flushleft}
Economy since 1951, agricultural growth in India, inter-regional
\end{flushleft}


\begin{flushleft}
variation in growth of output and productivity, farm price policy,
\end{flushleft}


\begin{flushleft}
recent trends in industrial growth, industrial and licensing policy, policy
\end{flushleft}


\begin{flushleft}
changes for industrial growth, economic reforms and liberalization,
\end{flushleft}


\begin{flushleft}
population growth, unemployment, food and nutrition security, and
\end{flushleft}


\begin{flushleft}
education. It will also include some contemporary issues. 	
\end{flushleft}





\begin{flushleft}
HUL333 Theatre of the absurd
\end{flushleft}


\begin{flushleft}
3 Credits (3-0-0)
\end{flushleft}


\begin{flushleft}
Pre--requisites: Any Two courses from HUL2XX category
\end{flushleft}


\begin{flushleft}
Allocation Preferences : HUL231, HUL232, HUL235, HUL236,
\end{flushleft}


\begin{flushleft}
HUL237, HUL240, HUL239
\end{flushleft}


\begin{flushleft}
Socio-political background of the theatre of the Absurd, its basis
\end{flushleft}


\begin{flushleft}
in Existentialist philosophy. The reactions against the conventions
\end{flushleft}


\begin{flushleft}
of realist theater that dominated this theatre. The pre-occupations
\end{flushleft}


\begin{flushleft}
of major playwrights with issues of language and the difficulty of
\end{flushleft}


\begin{flushleft}
communication, the isolation that human beings tend to feel from
\end{flushleft}


\begin{flushleft}
each other and themes of violence.
\end{flushleft}





\begin{flushleft}
HUL334 From Text to Film
\end{flushleft}


\begin{flushleft}
3 Credits (3-0-0)
\end{flushleft}


\begin{flushleft}
Pre--requisites: Any Two courses from HUL2XX category
\end{flushleft}


\begin{flushleft}
Allocation Preferences : HUL231, HUL232, HUL235, HUL236,
\end{flushleft}


\begin{flushleft}
HUL237, HUL240, HUL239
\end{flushleft}


\begin{flushleft}
The course will involve a detailed study of 3-4 texts and their
\end{flushleft}


\begin{flushleft}
corresponding adaptations into film. By means of close reading,
\end{flushleft}


\begin{flushleft}
analysis, and discussion, it will seek to identify the changes that
\end{flushleft}


\begin{flushleft}
take place during the process of adapting one art-form into another
\end{flushleft}





216





\begin{flushleft}
\newpage
Humanities and Social Sciences
\end{flushleft}





\begin{flushleft}
and ask why those modifications occur. An evaluation of what each
\end{flushleft}


\begin{flushleft}
art-form enables and what it restricts or denies will enable a better
\end{flushleft}


\begin{flushleft}
understanding of form per se, and of these two forms in particular.
\end{flushleft}


\begin{flushleft}
Further, the course will address the question of genre and its
\end{flushleft}


\begin{flushleft}
conventions especially with regard to film, and observe the extent to
\end{flushleft}


\begin{flushleft}
which generic expectations shape the process of adaptation of text into
\end{flushleft}


\begin{flushleft}
film. Film screenings will be held outside class hours in the evenings.
\end{flushleft}





\begin{flushleft}
HUL335 Indian Theatre
\end{flushleft}


\begin{flushleft}
3 Credits (3-0-0)
\end{flushleft}


\begin{flushleft}
Pre--requisites: Any Two courses from HUL2XX category
\end{flushleft}


\begin{flushleft}
Allocation Preferences : HUL231, HUL232, HUL235, HUL236,
\end{flushleft}


\begin{flushleft}
HUL237, HUL240, HUL239
\end{flushleft}


\begin{flushleft}
This course will study the various aspects of Indian theatre. The
\end{flushleft}


\begin{flushleft}
linkages between ancient theatre forms and existing forms of
\end{flushleft}


\begin{flushleft}
indigenous performance in various parts of India -- such as the
\end{flushleft}


\begin{flushleft}
nautanki, the tamasha and the jatra. The energies which were
\end{flushleft}


\begin{flushleft}
generated in the urban centres through the encounter with European
\end{flushleft}


\begin{flushleft}
drama -- the Parsi theatre, the nascent Marathi stage, the Hindi theatre
\end{flushleft}


\begin{flushleft}
of Bhartendu Harishchandra and the nationalist theatre of Calcutta --
\end{flushleft}


\begin{flushleft}
will be explored. Special attention would be paid to the transformation
\end{flushleft}


\begin{flushleft}
of theatre values with the intervention of the Indian People's Theatre
\end{flushleft}


\begin{flushleft}
Association (IPTA). The focus for the post-Independence period would
\end{flushleft}


\begin{flushleft}
be on the diverse energies of urban theatre, group theatre and the
\end{flushleft}


\begin{flushleft}
{`}back to the roots' movement. The course would require students to
\end{flushleft}


\begin{flushleft}
study play-scripts as well as look at accompanying literature to form
\end{flushleft}


\begin{flushleft}
a concrete idea of the philosophy behind Indian theatrical practice.
\end{flushleft}





\begin{flushleft}
HUL336 Workshop in Creative Writing
\end{flushleft}


\begin{flushleft}
3 Credits (3-0-0)
\end{flushleft}


\begin{flushleft}
Pre--requisites: Any Two courses from HUL2XX category
\end{flushleft}


\begin{flushleft}
Allocation Preferences : HUL231, HUL232, HUL235, HUL236,
\end{flushleft}


\begin{flushleft}
HUL237, HUL240, HUL239
\end{flushleft}


\begin{flushleft}
The course will begin by seeking to distinguish the notion of {`}creative'
\end{flushleft}


\begin{flushleft}
writing. It will contrast this heterogeneous category with other kinds
\end{flushleft}


\begin{flushleft}
of writing such as the {`}functional' writing found in text-books and
\end{flushleft}


\begin{flushleft}
reportage. Through an analysis of various techniques of writing - in
\end{flushleft}


\begin{flushleft}
master-texts as well as students' own productions - the course will
\end{flushleft}


\begin{flushleft}
explore why and how literary texts continue to be a prime source
\end{flushleft}


\begin{flushleft}
of emotional and intellectual stimulation across cultures. As far as
\end{flushleft}


\begin{flushleft}
possible, the course will focus on contemporary writing, given that
\end{flushleft}


\begin{flushleft}
writers write in the {`}here and now' even as they imagine the future
\end{flushleft}


\begin{flushleft}
or return to past memories. Selected readings will be used to focus
\end{flushleft}


\begin{flushleft}
students' attention on that most difficult of problems: to acquire a
\end{flushleft}


\begin{flushleft}
style of writing that makes a writer's {`}voice' both unique and universal.
\end{flushleft}


\begin{flushleft}
Finally, students will be required to write in some genre(s) of their
\end{flushleft}


\begin{flushleft}
choice. These genres will include the classic areas of poetry, fiction
\end{flushleft}


\begin{flushleft}
and play-writing but will neither exclude non-fiction genres like the
\end{flushleft}


\begin{flushleft}
essay and biography nor forms of writing thrown up by the {`}new
\end{flushleft}


\begin{flushleft}
media' such as blogs, photo-essays and narrative-writing for storyboards and video-games.
\end{flushleft}





\begin{flushleft}
HUL338 Functions of Satire
\end{flushleft}


\begin{flushleft}
3 Credits (3-0-0)
\end{flushleft}


\begin{flushleft}
Pre--requisites: Any Two courses from HUL2XX category
\end{flushleft}


\begin{flushleft}
Allocation Preferences : HUL231, HUL232, HUL235, HUL236,
\end{flushleft}


\begin{flushleft}
HUL237, HUL240, HUL239, HUL256
\end{flushleft}


\begin{flushleft}
Satire is a classical genre that has thrived over the centuries in almost
\end{flushleft}


\begin{flushleft}
all languages and cultures, and is found in a range of media. Life, in
\end{flushleft}


\begin{flushleft}
all aspects, everyday provides grist to the mill of satire, but does satire
\end{flushleft}


\begin{flushleft}
change anything? How do we define satire? Why is it considered the
\end{flushleft}


\begin{flushleft}
social genre? What are the contemporary forms of satire? Who can
\end{flushleft}


\begin{flushleft}
practice satire? It draws upon diverse techniques such as allegory,
\end{flushleft}


\begin{flushleft}
irony, caricature and laughter. Through analyses of examples, this
\end{flushleft}


\begin{flushleft}
course will familiarize students with satirical sub-genres and related
\end{flushleft}


\begin{flushleft}
literary practices, such as parody, burlesque, black humour, the
\end{flushleft}


\begin{flushleft}
grotesque, coarse humour, high and low comedy. It will examine the
\end{flushleft}


\begin{flushleft}
structure of satire, its relation with community, democracy and matters
\end{flushleft}


\begin{flushleft}
of gender, caste race, and religion.
\end{flushleft}





\begin{flushleft}
HUL340 Selected Topics in Literature
\end{flushleft}


\begin{flushleft}
3 Credits (3-0-0)
\end{flushleft}


\begin{flushleft}
Pre--requisites: Any Two courses from HUL2XX category
\end{flushleft}


\begin{flushleft}
Allocation Preferences : HUL231, HUL232, HUL235, HUL236,
\end{flushleft}


\begin{flushleft}
HUL237, HUL240, HUL239,
\end{flushleft}


\begin{flushleft}
The course will introduce students to selected topics in Literature as
\end{flushleft}


\begin{flushleft}
decided by the instructor.
\end{flushleft}





\begin{flushleft}
HUL341: Meaning in Natural Language
\end{flushleft}


\begin{flushleft}
3 Credits (3-0-0)
\end{flushleft}


\begin{flushleft}
Pre--requisites: Any Two courses from HUL2XX category
\end{flushleft}


\begin{flushleft}
Allocation Preferences : HUL242, HUL243, HUL282
\end{flushleft}


\begin{flushleft}
This course examines different aspects of meaning/semantics in
\end{flushleft}


\begin{flushleft}
language. Some specific questions addressed here are: a) what is
\end{flushleft}


\begin{flushleft}
meaning?, b) how do we use words to convey meanings?, and c)
\end{flushleft}


\begin{flushleft}
how does our grammatical knowledge interact with the interpretive
\end{flushleft}


\begin{flushleft}
system? We try to answer these and other questions while introducing
\end{flushleft}


\begin{flushleft}
students to the formal techniques used in research on the semantics
\end{flushleft}


\begin{flushleft}
of natural language.
\end{flushleft}





\begin{flushleft}
HUL350 Selected Topics in Linguistics
\end{flushleft}


\begin{flushleft}
3 Credits (3-0-0)
\end{flushleft}


\begin{flushleft}
Pre--requisites: Any Two courses from HUL2XX category
\end{flushleft}


\begin{flushleft}
Allocation Preferences : HUL242, HUL243, HUL282
\end{flushleft}


\begin{flushleft}
The course will introduce students to selected topics in Linguistics as
\end{flushleft}


\begin{flushleft}
decided by the instructor.
\end{flushleft}





\begin{flushleft}
HUL351 Philosophy of History
\end{flushleft}


\begin{flushleft}
3 Credits (3-0-0)
\end{flushleft}


\begin{flushleft}
Pre--requisites: Any Two courses from HUL2XX category
\end{flushleft}


\begin{flushleft}
Allocation Preferences : HUL251, HUL258, HUL253, HUL256
\end{flushleft}


\begin{flushleft}
What kind of understanding of the past does history provide? Is it
\end{flushleft}


\begin{flushleft}
speculative or analytical? What constitutes historical evidence and
\end{flushleft}


\begin{flushleft}
how does it confine historical understanding?
\end{flushleft}


\begin{flushleft}
Questions of objectivity are the central focus of this course: that
\end{flushleft}


\begin{flushleft}
of historians themselves---constructionist and objectivist--- as they
\end{flushleft}


\begin{flushleft}
debate methodological issues and disagreements about the aim of
\end{flushleft}


\begin{flushleft}
their discipline, and that of philosophers whose interest in history
\end{flushleft}


\begin{flushleft}
springs from their attention on history's objectivist ideals and {``}the
\end{flushleft}


\begin{flushleft}
objectivity crisis'' in history providing a philosophical rationale for
\end{flushleft}


\begin{flushleft}
reframing the two oppositions that dominate debates about the status
\end{flushleft}


\begin{flushleft}
of historical knowledge.
\end{flushleft}





\begin{flushleft}
HUL352 Problems in Classical Indian Philosophy
\end{flushleft}


\begin{flushleft}
(3-0-0) 3 Credits
\end{flushleft}


\begin{flushleft}
Pre--requisites: Any Two courses from HUL2XX category
\end{flushleft}


\begin{flushleft}
Allocation Preferences : HUL251, HUL258, HUL253, HUL256
\end{flushleft}


\begin{flushleft}
The course will begin by exploring the worldview implicit in the Vedas,
\end{flushleft}


\begin{flushleft}
the Upanisads, and the orthodox systems and then move on to the
\end{flushleft}


\begin{flushleft}
rejection of this entire system in Buddhism and Materialism. Emphasis
\end{flushleft}


\begin{flushleft}
will be led on the diversity of systems and healthy dialogue between
\end{flushleft}


\begin{flushleft}
antagonistic schools of thought. Discussions will focus on the nature
\end{flushleft}


\begin{flushleft}
of consciousness in relation to cognition of reality, theories of reality
\end{flushleft}


\begin{flushleft}
in terms of realism and anti-realism; the nature of self and no-self
\end{flushleft}


\begin{flushleft}
theory, theories of perceptual knowledge, theories of error; theories
\end{flushleft}


\begin{flushleft}
of causation and other relations, and key concepts of moral and
\end{flushleft}


\begin{flushleft}
aesthetic thought. Wherever appropriate, problems will be discussed
\end{flushleft}


\begin{flushleft}
in comparison with parallel discussions in western philosophy
\end{flushleft}





\begin{flushleft}
HUL353 Philosophical Themes in Biological Sciences
\end{flushleft}


\begin{flushleft}
3 Credits (3-0-0)
\end{flushleft}


\begin{flushleft}
Pre--requisites: Any Two courses from HUL2XX category
\end{flushleft}


\begin{flushleft}
Allocation Preferences : HUL251, HUL258, HUL253, HUL256
\end{flushleft}


\begin{flushleft}
This course addresses various philosophical questions that arise
\end{flushleft}


\begin{flushleft}
from the recent developments in evolutionary biology, genetics,
\end{flushleft}


\begin{flushleft}
immunology, sociobiology, molecular biology and synthetic biology.
\end{flushleft}





217





\begin{flushleft}
\newpage
Humanities and Social Sciences
\end{flushleft}





\begin{flushleft}
How do these developments affect our ideas about life, evolution and
\end{flushleft}


\begin{flushleft}
the place of man in relation to other living beings. What is the nature of
\end{flushleft}


\begin{flushleft}
explanation in biological sciences? Does the idea of immunity demand
\end{flushleft}


\begin{flushleft}
rethinking on the nature of our embodied self? What can biological
\end{flushleft}


\begin{flushleft}
sciences tell us about healing, pain and death?
\end{flushleft}





\begin{flushleft}
HUL354 Art and Technology
\end{flushleft}


\begin{flushleft}
3 Credits (3-0-0)
\end{flushleft}


\begin{flushleft}
Pre--requisites: Any Two courses from HUL2XX category
\end{flushleft}


\begin{flushleft}
Allocation Preferences: HUL251, HUL258, HUL253, HUL256,
\end{flushleft}


\begin{flushleft}
HUL290
\end{flushleft}


\begin{flushleft}
The course begins by registering the increased presence of technology
\end{flushleft}


\begin{flushleft}
in contemporary art. We shall keep the experiences of both classical
\end{flushleft}


\begin{flushleft}
Greece and Classical India alive where art and technology were not
\end{flushleft}


\begin{flushleft}
clearly separated in the manner familiar to us. By positioning us
\end{flushleft}


\begin{flushleft}
between these two experiences - classical and contemporary we
\end{flushleft}


\begin{flushleft}
shall critically examine the complex relationship between art, science
\end{flushleft}


\begin{flushleft}
and technology which characterizes modernity. The course uses both
\end{flushleft}


\begin{flushleft}
materials from philosophical aesthetics, philosophy of science and
\end{flushleft}


\begin{flushleft}
technology. It also discusses the philosophical writings on specific
\end{flushleft}


\begin{flushleft}
areas like architecture, photography, cinema and digital art.
\end{flushleft}





\begin{flushleft}
HUL355 Philosophy and Intellectual History in India
\end{flushleft}


\begin{flushleft}
3 Credits (3-0-0)
\end{flushleft}


\begin{flushleft}
Pre--requisites: Any Two courses from HUL2XX category
\end{flushleft}


\begin{flushleft}
Allocation Preferences: HUL251, HUL258, HUL253, HUL256,
\end{flushleft}


\begin{flushleft}
What defines the Indian tradition? Is there a singular Indian tradition or
\end{flushleft}


\begin{flushleft}
is there a plurality of Indian traditions in the public sphere today? How
\end{flushleft}


\begin{flushleft}
do these find representation in the modern and textual frameworks?
\end{flushleft}


\begin{flushleft}
Is modernity antithetical to tradition?
\end{flushleft}


\begin{flushleft}
The aim of this course is to take up these varied questions along with their
\end{flushleft}


\begin{flushleft}
nuances to understand and re-negotiate Indian intellectual traditions.
\end{flushleft}


\begin{flushleft}
In this course, the examination of sources, structure, texts and
\end{flushleft}


\begin{flushleft}
exemplars from Indian intellectual tradition provide a theoretical
\end{flushleft}


\begin{flushleft}
framework for the discussion of contemporary political and social
\end{flushleft}


\begin{flushleft}
issues. Economic development, social justice, religion and the nation,
\end{flushleft}


\begin{flushleft}
communalism and secularism, caste, class and gender equality are
\end{flushleft}


\begin{flushleft}
themes to be addressed. The political misuse of tradition in programs
\end{flushleft}


\begin{flushleft}
of reform and revival both in the past and in modern times will be
\end{flushleft}


\begin{flushleft}
highlighted to underline the need for rethinking Indian Philosophy and
\end{flushleft}


\begin{flushleft}
intellectual tradition in an academically rigorous manner.
\end{flushleft}


\begin{flushleft}
This course will also take into cognisance the intellectual history of the
\end{flushleft}


\begin{flushleft}
ancient past as it comes through the Vedic thought and its contestations.
\end{flushleft}





\begin{flushleft}
Allocation Preferences: HUL251, HUL258, HUL253, HUL256,
\end{flushleft}


\begin{flushleft}
Categorial taxonomy of Mental Phenomena: Intentional and
\end{flushleft}


\begin{flushleft}
Phenomenal.
\end{flushleft}


\begin{flushleft}
Theories of the Mind-Body relation: Cartesian Dualism; Behaviourism;
\end{flushleft}


\begin{flushleft}
Identity Theory or Physicalism; Functionalism.
\end{flushleft}


\begin{flushleft}
Personal Identity and the Self: The First-person Point of View.
\end{flushleft}


\begin{flushleft}
Consciousness and Content: Phenomenal Intentionality;
\end{flushleft}


\begin{flushleft}
Representationalism; Internalism and Externalism about Experience;
\end{flushleft}


\begin{flushleft}
Qualia and the Knowledge Argument.
\end{flushleft}


\begin{flushleft}
Consciousness and Self-consciousness: Pre-reflective Selfconsciousness; One-level Accounts of Self-consciousness; Temporality
\end{flushleft}


\begin{flushleft}
and the Limits of Reflective Self-consciousness; Bodily Self-awareness;
\end{flushleft}


\begin{flushleft}
Social Forms of Self-awareness.
\end{flushleft}


\begin{flushleft}
Critique of the Computational Theory of Mind: Searle's Chinese Room
\end{flushleft}


\begin{flushleft}
Argument and the Frame Problem.
\end{flushleft}





\begin{flushleft}
HUL359 Metaphysics of the self
\end{flushleft}


\begin{flushleft}
3 Credits (3-0-0)
\end{flushleft}


\begin{flushleft}
Pre--requisites: Any Two courses from HUL2XX category
\end{flushleft}


\begin{flushleft}
Allocation Preferences: HUL251, HUL258, HUL253, HUL256
\end{flushleft}


\begin{flushleft}
The course is a critical study of the problem of the self taken to be
\end{flushleft}


\begin{flushleft}
a substance by some and denied to have any substantial reality by
\end{flushleft}


\begin{flushleft}
others. Focus will be given on examining the worldview from which
\end{flushleft}


\begin{flushleft}
stems the idea of a continuing self, as a subject of consciousness and
\end{flushleft}


\begin{flushleft}
agent of action. Questions about whether it is material or immaterial,
\end{flushleft}


\begin{flushleft}
real or nominal object will centre the ontological investigation into
\end{flushleft}


\begin{flushleft}
the nature of the self. Special consideration will be given to the issue
\end{flushleft}


\begin{flushleft}
of self-awareness and self-reference, and its relation to the linguistic
\end{flushleft}


\begin{flushleft}
phenomenon of the first-person pronoun {`}I'.
\end{flushleft}





\begin{flushleft}
HUL360 Selected Topics in Philosophy
\end{flushleft}


\begin{flushleft}
3 Credits (3-0-0)
\end{flushleft}


\begin{flushleft}
Pre--requisites: Any Two courses from HUL2XX category
\end{flushleft}


\begin{flushleft}
Allocation Preferences: HUL251, HUL258, HUL253, HUL256
\end{flushleft}


\begin{flushleft}
The course will introduce students to selected topics in Philosophy as
\end{flushleft}


\begin{flushleft}
decided by the instructor.
\end{flushleft}





\begin{flushleft}
HUL361 Applied Positive Psychology
\end{flushleft}


\begin{flushleft}
3 Credits (3-0-0)
\end{flushleft}


\begin{flushleft}
Pre--requisites: Any Two courses from HUL2XX category
\end{flushleft}


\begin{flushleft}
Allocation Preferences: HUL261, HUL265, HUL267
\end{flushleft}





\begin{flushleft}
Literature on Buddhism and Buddhist literature brings out the historical,
\end{flushleft}


\begin{flushleft}
philosophical and political synthesis of Buddhism in ever new cultural
\end{flushleft}


\begin{flushleft}
contexts. Interrogating and contextualizing engagements of Buddhism's
\end{flushleft}


\begin{flushleft}
classical roots in modernity will be a key concerns in this course.
\end{flushleft}





\begin{flushleft}
Meaning and goals of applied positive psychology; Relevant
\end{flushleft}


\begin{flushleft}
research methods of the field; Introduction to intervention
\end{flushleft}


\begin{flushleft}
programmes including internet based intervention; Researches that
\end{flushleft}


\begin{flushleft}
support intervention strategies : Psychological well-being and its
\end{flushleft}


\begin{flushleft}
intervention programmes; emotional intelligence and its intervention
\end{flushleft}


\begin{flushleft}
programmes; Strategies for achieving well-Being; Mindfulness and its
\end{flushleft}


\begin{flushleft}
intervention programmes; Intervention module on stress and time
\end{flushleft}


\begin{flushleft}
management; Character strengths : their role in well being; How
\end{flushleft}


\begin{flushleft}
psychosocial resources enhance health and well being; Intervention
\end{flushleft}


\begin{flushleft}
researches in Indian socio-cultural context; Current issues and future
\end{flushleft}


\begin{flushleft}
directions in this Area.
\end{flushleft}





\begin{flushleft}
HUL357 Philosophy of Science
\end{flushleft}


\begin{flushleft}
3 Credits (3-0-0)
\end{flushleft}


\begin{flushleft}
Pre--requisites: Any Two courses from HUL2XX category
\end{flushleft}


\begin{flushleft}
Allocation Preferences: HUL251, HUL258, HUL253, HUL256,
\end{flushleft}





\begin{flushleft}
HUL362 Organizational Behaviour
\end{flushleft}


\begin{flushleft}
3 Credits (3-0-0)
\end{flushleft}


\begin{flushleft}
Pre--requisites: Any Two courses from HUL2XX category
\end{flushleft}


\begin{flushleft}
Allocation Preferences: HUL261, HUL265, HUL267
\end{flushleft}





\begin{flushleft}
HUL356 Buddhism Across Time and Place
\end{flushleft}


\begin{flushleft}
3 Credits (3-0-0)
\end{flushleft}


\begin{flushleft}
Pre--requisites: Any Two courses from HUL2XX category
\end{flushleft}


\begin{flushleft}
Allocation Preferences: HUL251, HUL258, HUL253, HUL256,
\end{flushleft}





\begin{flushleft}
Science is regarded as the most significant cognitive enterprise of the
\end{flushleft}


\begin{flushleft}
modern society. In view of this, the course addresses the question
\end{flushleft}


\begin{flushleft}
what sets science apart from other epistemic activities. Further It
\end{flushleft}


\begin{flushleft}
concentrates on debates on the nature of scientific methods, logical
\end{flushleft}


\begin{flushleft}
reconstruction of scientific explanation, the relation between theories
\end{flushleft}


\begin{flushleft}
and laws on the one hand, and empirical evidence on the other, the
\end{flushleft}


\begin{flushleft}
nature of the justification and the notion of truth involved in scientific
\end{flushleft}


\begin{flushleft}
knowledge, and the societal influence on scientific practice.
\end{flushleft}





\begin{flushleft}
HUL358 Philosophy of mind
\end{flushleft}


\begin{flushleft}
3 Credits (3-0-0)
\end{flushleft}


\begin{flushleft}
Pre--requisites: Any Two courses from HUL2XX category
\end{flushleft}





\begin{flushleft}
Introduction to organizational behaviour, Historical development
\end{flushleft}


\begin{flushleft}
of the field and some challenges in contemporary times, Learning
\end{flushleft}


\begin{flushleft}
and perceptual processes in organizations and their implications
\end{flushleft}


\begin{flushleft}
for work-life,Work related attitudes- job satisfaction, organizational
\end{flushleft}


\begin{flushleft}
commitment, organizational justice, organizational citizenship
\end{flushleft}


\begin{flushleft}
behaviour, Individual differences related to personality, emotions
\end{flushleft}


\begin{flushleft}
and abilities and functioning in organization, Group processes in
\end{flushleft}


\begin{flushleft}
organizations, Formation of groups and teams, Effective teams,
\end{flushleft}


\begin{flushleft}
Communication in organizations, Social influence processes in
\end{flushleft}


\begin{flushleft}
organizations, influencing people, power dynamics and politics and
\end{flushleft}


\begin{flushleft}
impact on organizational functioning, Theories and styles of leadership
\end{flushleft}


\begin{flushleft}
in organization and their impact on organizational functioning,
\end{flushleft}





218





\begin{flushleft}
\newpage
Humanities and Social Sciences
\end{flushleft}





\begin{flushleft}
Organizational ethos and culture and its impact on productivity and
\end{flushleft}


\begin{flushleft}
well- being, Various kinds of organizational structures and their
\end{flushleft}


\begin{flushleft}
effectiveness, managing organizations in times of change.
\end{flushleft}





\begin{flushleft}
HUL363 Community Psychology
\end{flushleft}


\begin{flushleft}
3 Credits (3-0-0)
\end{flushleft}


\begin{flushleft}
Pre--requisites: Any Two courses from HUL2XX category
\end{flushleft}


\begin{flushleft}
Allocation Preferences: HUL261, HUL265, HUL267
\end{flushleft}


\begin{flushleft}
Introduction to Community Psychology; Understanding Individuals
\end{flushleft}


\begin{flushleft}
within Environments; Individualism collectivism \& community
\end{flushleft}


\begin{flushleft}
Psychology; Understanding Human Diversity; Understanding Coping
\end{flushleft}


\begin{flushleft}
in Context; Community and Social Change; Prevention and Promotion:
\end{flushleft}


\begin{flushleft}
Key Concepts, Current and Future Applications \& implementing
\end{flushleft}


\begin{flushleft}
programs; Overview of Community Interventions; Social support
\end{flushleft}


\begin{flushleft}
research in community psychology; Recent community researches in
\end{flushleft}


\begin{flushleft}
Indian socio- cultural context: Effects of various socio- cultural issues
\end{flushleft}


\begin{flushleft}
on individual and community well-being.
\end{flushleft}





\begin{flushleft}
HUL364 Understanding the Social Being
\end{flushleft}


\begin{flushleft}
3 Credits (3-0-0)
\end{flushleft}


\begin{flushleft}
Pre--requisites: Any Two courses from HUL2XX category
\end{flushleft}


\begin{flushleft}
Allocation Preferences: HUL261, HUL265, HUL267
\end{flushleft}


\begin{flushleft}
The social being- introducing the social psychological approach,
\end{flushleft}


\begin{flushleft}
Historical roots, theories and methods. Person and social perception
\end{flushleft}


\begin{flushleft}
and social judgements. Social cognition and social thinking. The social
\end{flushleft}


\begin{flushleft}
self- cognitive, affective and behavioural aspects of self. Positive social
\end{flushleft}


\begin{flushleft}
behaviours- altruism, cooperation and volunteerism. Individuals in
\end{flushleft}


\begin{flushleft}
groups-social facilitation, loafing, conformity, compliance. Social
\end{flushleft}


\begin{flushleft}
influence, manipulation and power- bases of power. Intergroup
\end{flushleft}


\begin{flushleft}
relations- explanations and managing intergroup relations. Collective
\end{flushleft}


\begin{flushleft}
behaviour- crowds and mobs- negative and positive aspects
\end{flushleft}


\begin{flushleft}
of collective behaviour. Aggression and violence- theories and
\end{flushleft}


\begin{flushleft}
determinants. Reducing aggression and violence. Applications of social
\end{flushleft}


\begin{flushleft}
psychology- health and well-being, law and organizational contexts.
\end{flushleft}





\begin{flushleft}
HUL365 Environmental Issues: Psychological Analysis
\end{flushleft}


\begin{flushleft}
3 Credits (3-0-0)
\end{flushleft}


\begin{flushleft}
Pre--requisites: Any Two courses from HUL2XX category
\end{flushleft}


\begin{flushleft}
Allocation Preferences: HUL261, HUL265, HUL267
\end{flushleft}


\begin{flushleft}
The implications of natural, built and social environment on human
\end{flushleft}


\begin{flushleft}
functioning, Making sense of environment-environmental perception
\end{flushleft}


\begin{flushleft}
and cognition, Nature of environmental attitudes and implications
\end{flushleft}


\begin{flushleft}
for inculcating pro-environmental attitudes, Various kinds of
\end{flushleft}


\begin{flushleft}
environmental stressors and human response to these stressors,
\end{flushleft}


\begin{flushleft}
Psychological analysis of climate change related issues, Psychology
\end{flushleft}


\begin{flushleft}
and energy conservation- social and collective dilemmas and individual
\end{flushleft}


\begin{flushleft}
interests, Environmental disasters and disaster preparedness,
\end{flushleft}


\begin{flushleft}
Assessing environmental risks, Place attachment, territoriality, personal
\end{flushleft}


\begin{flushleft}
space and notion of privacy and identity issues, Designing better
\end{flushleft}


\begin{flushleft}
environments and role of psychological factors in the design process,
\end{flushleft}


\begin{flushleft}
Examining specific built environments.
\end{flushleft}





\begin{flushleft}
HUL370: Selected Topics in Psychology
\end{flushleft}


\begin{flushleft}
3 Credits (3-0-0)
\end{flushleft}


\begin{flushleft}
Pre--requisites: Any Two courses from HUL2XX category
\end{flushleft}


\begin{flushleft}
Allocation Preferences: HUL261, HUL265, HUL267
\end{flushleft}


\begin{flushleft}
The course will introduce students to selected topics in Psychology
\end{flushleft}


\begin{flushleft}
as decided by the instructor.
\end{flushleft}





\begin{flushleft}
HUL371 Science, Technology and Society
\end{flushleft}


\begin{flushleft}
3 Credits (3-0-0)
\end{flushleft}


\begin{flushleft}
Pre--requisites: Any Two courses from HUL2XX category
\end{flushleft}


\begin{flushleft}
Allocation Preferences: HUL271, HUL272, HUL274 HUL275,
\end{flushleft}


\begin{flushleft}
HUL276, HUL281, HUL286, HUL289, HUL290
\end{flushleft}


\begin{flushleft}
The course will begin with social theories on the production of
\end{flushleft}


\begin{flushleft}
technology and scientific knowledge systems, stratification within
\end{flushleft}


\begin{flushleft}
the community of technologists and scientists, discrimination (race,
\end{flushleft}


\begin{flushleft}
class, gender, caste) and the role of power in shaping the production
\end{flushleft}


\begin{flushleft}
of technology and scientific knowledge. Scientific controversies, both
\end{flushleft}





\begin{flushleft}
historical and emerging, and the organization of innovation and its
\end{flushleft}


\begin{flushleft}
geographies will be discussed. Case studies exploring ethical questions
\end{flushleft}


\begin{flushleft}
arising from new technologies such as information technology,
\end{flushleft}


\begin{flushleft}
nanotechnologies, biotechnologies, etc. will be used. Discussions
\end{flushleft}


\begin{flushleft}
on public understanding of science and role of the public and of
\end{flushleft}


\begin{flushleft}
experts in influencing policies related to science and technology will
\end{flushleft}


\begin{flushleft}
conclude the course.
\end{flushleft}





\begin{flushleft}
HUL372 Agrarian India: Past and Present
\end{flushleft}


\begin{flushleft}
3 Credits (3-0-0)
\end{flushleft}


\begin{flushleft}
Pre--requisites: Any Two courses from HUL2XX category
\end{flushleft}


\begin{flushleft}
Allocation Preferences: HUL271, HUL272, HUL274 HUL275,
\end{flushleft}


\begin{flushleft}
HUL276, HUL286
\end{flushleft}


\begin{flushleft}
The course will use interdisciplinary texts to give students a historical
\end{flushleft}


\begin{flushleft}
overview of agrarian India starting from the colonial period,
\end{flushleft}


\begin{flushleft}
plantation and export economies, recurring famines, community
\end{flushleft}


\begin{flushleft}
development programs and land reforms after independence,
\end{flushleft}


\begin{flushleft}
the green revolution, and the neglect of rainfed / dryland regions.
\end{flushleft}


\begin{flushleft}
It will explore various dimensions of development in agriculture
\end{flushleft}


\begin{flushleft}
including the advent of the agricultural sciences and the birth of
\end{flushleft}


\begin{flushleft}
the agricultural extension system. The myth of the ignorant farmer
\end{flushleft}


\begin{flushleft}
and the self-sufficient village will be discussed. Case studies on the
\end{flushleft}


\begin{flushleft}
historical roots of globalization and agricultural commodity chains
\end{flushleft}


\begin{flushleft}
related to new technologies, and the linkages between the market
\end{flushleft}


\begin{flushleft}
and the state in contemporary agriculture will be discussed. The
\end{flushleft}


\begin{flushleft}
growing social and geographical disparity with ecological distress
\end{flushleft}


\begin{flushleft}
and the threat of climate change, farmer suicides, and debt spirals
\end{flushleft}


\begin{flushleft}
on the one hand, and a risky but rewarding cash crop economy
\end{flushleft}


\begin{flushleft}
on the other, will also be explored. Finally the course will discuss
\end{flushleft}


\begin{flushleft}
aspirations of rural youth, opportunities for livelihoods, and gender
\end{flushleft}


\begin{flushleft}
and caste dimensions of the growing urbanization of rural centres.
\end{flushleft}





\begin{flushleft}
HUL375 The Sociology of Religion
\end{flushleft}


\begin{flushleft}
3 Credits (3-0-0)
\end{flushleft}


\begin{flushleft}
Pre--requisites: Any Two courses from HUL2XX category
\end{flushleft}


\begin{flushleft}
Allocation Preferences: HUL271, HUL272, HUL274 HUL275,
\end{flushleft}


\begin{flushleft}
HUL276, HUL286
\end{flushleft}


\begin{flushleft}
This course will introduce students to sociological approaches to the
\end{flushleft}


\begin{flushleft}
study of religion in contemporary society. Religion will be understood
\end{flushleft}


\begin{flushleft}
in terms of its social and cultural structure; in addition the course
\end{flushleft}


\begin{flushleft}
will also encourage a critical perspective on religion and society -- its
\end{flushleft}


\begin{flushleft}
interface with society, polity and the economy. Religious conflict and
\end{flushleft}


\begin{flushleft}
change, syncretism, popular religion, revivalism and fundamentalism
\end{flushleft}


\begin{flushleft}
will also be considered.
\end{flushleft}





\begin{flushleft}
HUL376 Political Ecology of Water
\end{flushleft}


\begin{flushleft}
3 Credits (3-0-0)
\end{flushleft}


\begin{flushleft}
Pre--requisites: Any Two courses from HUL2XX category
\end{flushleft}


\begin{flushleft}
Allocation Preferences: HUL271, HUL272, HUL274 HUL275,
\end{flushleft}


\begin{flushleft}
HUL276, HUL286
\end{flushleft}


\begin{flushleft}
This course is an advanced undergraduate sociology course on the
\end{flushleft}


\begin{flushleft}
political ecology of water. It discusses people's historic and current
\end{flushleft}


\begin{flushleft}
engagement with water, sustainable development and water, the recent
\end{flushleft}


\begin{flushleft}
controversies and emergent resource conflict over water in the context
\end{flushleft}


\begin{flushleft}
of industrial development, design and implementation of hydropower
\end{flushleft}


\begin{flushleft}
projects, water pollution management, and conservation strategies
\end{flushleft}


\begin{flushleft}
(modern and traditional) and relates them to relevant national policies.
\end{flushleft}





\begin{flushleft}
HUL377 Gender, Technology and Society
\end{flushleft}


\begin{flushleft}
3 Credits (3-0-0)
\end{flushleft}


\begin{flushleft}
Pre--requisites: Any Two courses from HUL2XX category
\end{flushleft}


\begin{flushleft}
Allocation Preferences: HUL271, HUL272, HUL274 HUL275,
\end{flushleft}


\begin{flushleft}
HUL276, HUL286, HUL289, HUL290
\end{flushleft}


\begin{flushleft}
The manner in which gender is conceptualized and performed is
\end{flushleft}


\begin{flushleft}
foundational to the understanding of human social relationships.
\end{flushleft}


\begin{flushleft}
Gender identities are not fixed or determined purely by physiology;
\end{flushleft}


\begin{flushleft}
their social construction affects ideas of masculinity and femininity or
\end{flushleft}


\begin{flushleft}
other sexual identities. Besides understanding how sex and gender
\end{flushleft}


\begin{flushleft}
are interrelated, we will look at how gender intertwines with societal
\end{flushleft}





219





\begin{flushleft}
\newpage
Humanities and Social Sciences
\end{flushleft}





\begin{flushleft}
areas of economy, technology, polity, religion and demography.
\end{flushleft}


\begin{flushleft}
The important role played by social structures and institutions such
\end{flushleft}


\begin{flushleft}
as caste, kinship, family, marriage, ethnicity, religion and class in
\end{flushleft}


\begin{flushleft}
structuring gender and vice-versa will be brought out.
\end{flushleft}


\begin{flushleft}
Technologies associated with population and biological sciences have
\end{flushleft}


\begin{flushleft}
transformed and are continuing to transform society and human
\end{flushleft}


\begin{flushleft}
relationships in particular directions. The course will examine these
\end{flushleft}


\begin{flushleft}
transformations at the global and local levels and consider their impact
\end{flushleft}


\begin{flushleft}
on individual lives. Challenges posed to intimate human relationships
\end{flushleft}


\begin{flushleft}
and identities by new reproductive technologies such as invitrofertilization, surrogacy, sex selection will be explored. What does the
\end{flushleft}


\begin{flushleft}
emergence/ institutionalization of new social forms - such as same sex
\end{flushleft}


\begin{flushleft}
marriages and parenthood by surrogacy - tell us about the possibilities
\end{flushleft}


\begin{flushleft}
and limits of human relationships?
\end{flushleft}





\begin{flushleft}
HUL378 Industry and Work Culture under Globalization
\end{flushleft}


\begin{flushleft}
3 Credits (3-0-0)
\end{flushleft}


\begin{flushleft}
Pre--requisites: Any Two courses from HUL2XX category
\end{flushleft}


\begin{flushleft}
Allocation Preferences: HUL271, HUL272, HUL274 HUL275,
\end{flushleft}


\begin{flushleft}
HUL276, HUL286
\end{flushleft}


\begin{flushleft}
Globalization and Globality; Classical theories to understanding
\end{flushleft}


\begin{flushleft}
work and industry; Understanding Work, Work Ethic and Work
\end{flushleft}


\begin{flushleft}
Culture; Post-industrial society and rise of informational economy;
\end{flushleft}


\begin{flushleft}
Job-satisfaction and alienation; Equalization of Opportunities and
\end{flushleft}


\begin{flushleft}
the Flattening of the World; Outsourcing as a Business Strategy;
\end{flushleft}


\begin{flushleft}
Important changes in industry and rise of IT sector and BPO
\end{flushleft}


\begin{flushleft}
industry; Governance and Collective Organization of Workers in
\end{flushleft}


\begin{flushleft}
select sectors; Corporate Social Responsibility.
\end{flushleft}





\begin{flushleft}
HUL380 Selected Topics in Sociology
\end{flushleft}


\begin{flushleft}
3 Credits (3-0-0)
\end{flushleft}


\begin{flushleft}
Pre--requisites: Any Two courses from HUL2XX category
\end{flushleft}


\begin{flushleft}
Allocation Preferences: HUL271, HUL272, HUL274 HUL275,
\end{flushleft}


\begin{flushleft}
HUL276, HUL286
\end{flushleft}


\begin{flushleft}
The course will introduce students to selected topics in Sociology as
\end{flushleft}


\begin{flushleft}
decided by the instructor
\end{flushleft}





\begin{flushleft}
HUL381 Mind, Machines and Language
\end{flushleft}


\begin{flushleft}
3 Credits (3-0-0)
\end{flushleft}


\begin{flushleft}
Pre--requisites: Any Two courses from HUL2XX category
\end{flushleft}


\begin{flushleft}
Allocation Preferences: HUL243
\end{flushleft}





\begin{flushleft}
in theories of language and society, in post-Kantian philosophy, in
\end{flushleft}


\begin{flushleft}
attitudes tor religion. Romantics not only engaged in experimental
\end{flushleft}


\begin{flushleft}
social practices and literary collaborations, but also articulated their
\end{flushleft}


\begin{flushleft}
necessity for the first time. Can we say that romanticism is at an
\end{flushleft}


\begin{flushleft}
end? How does it contribute to both a nationalism rooted in folk
\end{flushleft}


\begin{flushleft}
tradition, and individualism expressed in the cult of the hero, the
\end{flushleft}


\begin{flushleft}
solitary intellectual? How does it both look back to medieval occult
\end{flushleft}


\begin{flushleft}
and forward to novelties of science? Why is romanticism fascinated
\end{flushleft}


\begin{flushleft}
with animals, monsters and machines alike?
\end{flushleft}





\begin{flushleft}
HUD700 Seminar (Case Material-based) Minor Project
\end{flushleft}


\begin{flushleft}
3 Credits (0-0-6)
\end{flushleft}


\begin{flushleft}
Students would under take a supervised research project.
\end{flushleft}





\begin{flushleft}
HUL701 Introduction to Science and Technology Policy
\end{flushleft}


\begin{flushleft}
Studies
\end{flushleft}


\begin{flushleft}
1.5 Credits (1.5-0-0)
\end{flushleft}


\begin{flushleft}
The course will begin with a brief theoretical understanding of
\end{flushleft}


\begin{flushleft}
policy making processes and touch upon the specifics of science and
\end{flushleft}


\begin{flushleft}
technology policy systems in India. It will specifically examine the
\end{flushleft}


\begin{flushleft}
role of stakeholders in the process such as grassroots voices and civil
\end{flushleft}


\begin{flushleft}
society organisations, industry, academia, international actors, and
\end{flushleft}


\begin{flushleft}
policy makers. It will then examine the role of science and technology
\end{flushleft}


\begin{flushleft}
in policies in selected current and emerging key sectors, e.g.
\end{flushleft}


\begin{flushleft}
transport, agriculture, health, energy, environment, or information and
\end{flushleft}


\begin{flushleft}
communication technologies. The course will also explore the inherently
\end{flushleft}


\begin{flushleft}
political and contested nature of decision making in the policy arena.
\end{flushleft}





\begin{flushleft}
HUL702 Approaches to Science and Technology Policy
\end{flushleft}


\begin{flushleft}
Studies
\end{flushleft}


\begin{flushleft}
1.5 Credits (1.5-0-0)
\end{flushleft}


\begin{flushleft}
The course identifies six themes which are key to understanding
\end{flushleft}


\begin{flushleft}
Science and Technology Policy (STP) viz., Safety, Ownership,
\end{flushleft}


\begin{flushleft}
Ethics/morality, Knowledge base, Participation and Choice of policy
\end{flushleft}


\begin{flushleft}
instruments. Through sociological, economic, regulatory and legal
\end{flushleft}


\begin{flushleft}
literature on selected current and emerging key sectors, e.g.
\end{flushleft}


\begin{flushleft}
transport, agriculture, health, energy, environment, or information
\end{flushleft}


\begin{flushleft}
and communication technologies, it invites the students to keenly
\end{flushleft}


\begin{flushleft}
understand the various underlying approaches in STP.
\end{flushleft}





\begin{flushleft}
HUL703 Perspectives on Climate Change: Implications
\end{flushleft}


\begin{flushleft}
for Policy
\end{flushleft}


\begin{flushleft}
3 Credits (3-0-0)
\end{flushleft}





\begin{flushleft}
Exploratory in nature, the course seeks to debate questions such
\end{flushleft}


\begin{flushleft}
as: What are the implications of conceiving the mind as a {`}machine'?
\end{flushleft}


\begin{flushleft}
Can evolutionary theories about language and tool-using help us
\end{flushleft}


\begin{flushleft}
understand how we continually manage today to process the world
\end{flushleft}


\begin{flushleft}
around us {`}online'? On this course, the class will be introduced to
\end{flushleft}


\begin{flushleft}
some state-of-the-art discussions in the interdisciplinary field of
\end{flushleft}


\begin{flushleft}
cognitive studies. These topics will include: (i) the modularity of mind
\end{flushleft}


\begin{flushleft}
(ii) the content of consciousness, (iii) the language bio-programme
\end{flushleft}


\begin{flushleft}
hypothesis, (iv) the relativism versus universals of controversy; (v)
\end{flushleft}


\begin{flushleft}
strong and weak positions on AI, etc. The course will rely on downto-earth examples to demonstrate that such an interconnected area of
\end{flushleft}


\begin{flushleft}
study is not remote or esoteric but part of the intellectual excitement
\end{flushleft}


\begin{flushleft}
of living in the new millennium and attempting to anticipate both how
\end{flushleft}


\begin{flushleft}
it will shape us and how we will shape it.
\end{flushleft}





\begin{flushleft}
HUL382: Romanticism: The Theory of Animals, Monsters
\end{flushleft}


\begin{flushleft}
and Machines
\end{flushleft}


     


\begin{flushleft}
3 Credits (3-0-0)
\end{flushleft}


\begin{flushleft}
Pre--requisites: Any Two courses from HUL2XX category
\end{flushleft}


\begin{flushleft}
Allocation Preferences: HUL231, HUL235
\end{flushleft}


\begin{flushleft}
There is more to romanticism than Wordsworth's poetry, or even
\end{flushleft}


\begin{flushleft}
literature in general. Nor is it confined between 1780s and 1830s. Least
\end{flushleft}


\begin{flushleft}
is it a trend succeeded by Victorianism and realism, and assailed by
\end{flushleft}


\begin{flushleft}
modernism. Romanticism contends with the question of presentation
\end{flushleft}


\begin{flushleft}
-- of representation of and to oneself. It therefore directly participates
\end{flushleft}


\begin{flushleft}
in the philosophical discussions of reason, sensibility, emotion,
\end{flushleft}


\begin{flushleft}
subjectivity, and most importantly the idea of human freedom. This
\end{flushleft}


\begin{flushleft}
course will familiarize students with romantic movements in arts,
\end{flushleft}





\begin{flushleft}
The course will develop a basic understanding of science of climate
\end{flushleft}


\begin{flushleft}
change, the associated uncertainties and the processes that link
\end{flushleft}


\begin{flushleft}
this science with policymaking. The impacts of climate change on
\end{flushleft}


\begin{flushleft}
socio-economic and natural systems and the link between climate
\end{flushleft}


\begin{flushleft}
change, and development policies will be discussed. The global
\end{flushleft}


\begin{flushleft}
distribution of greenhouse gas emissions and possible technological,
\end{flushleft}


\begin{flushleft}
market and regulatory trajectories to mitigate them will be discussed
\end{flushleft}


\begin{flushleft}
with the emphasis on how different trajectories lead to questions on
\end{flushleft}


\begin{flushleft}
geographic, inter-generational and distributional equity. The students
\end{flushleft}


\begin{flushleft}
would examine economic, political and institutional frameworks for
\end{flushleft}


\begin{flushleft}
understanding policies and practices designed to reduce greenhouse
\end{flushleft}


\begin{flushleft}
gas emissions, vulnerability to climate change and facilitate adaptation
\end{flushleft}


\begin{flushleft}
in the face of climate threats and explore how policy can produce or
\end{flushleft}


\begin{flushleft}
reduce vulnerability. The course will draw on theoretical framings and
\end{flushleft}


\begin{flushleft}
methodological tools from multiple disciplines including atmospheric
\end{flushleft}


\begin{flushleft}
sciences, economics, environmental policy, psychology and sociology.
\end{flushleft}





\begin{flushleft}
HUL704 : Inclusive Innovation: Theory and Practice
\end{flushleft}


\begin{flushleft}
4 Credits (2-0-4)
\end{flushleft}


\begin{flushleft}
The course will familiarize students with key concepts in innovation,
\end{flushleft}


\begin{flushleft}
including various elements of the innovation cycle going all the way
\end{flushleft}


\begin{flushleft}
from need identification to deployment. Key aspects relating to
\end{flushleft}


\begin{flushleft}
inclusive innovation - the public goods nature of many basic needs,
\end{flushleft}


\begin{flushleft}
user and market characteristics, delivery and scaling-up considerations,
\end{flushleft}


\begin{flushleft}
and the role of partnerships and policies - will receive particular focus.
\end{flushleft}


\begin{flushleft}
The students will also be introduced to the state-of-the-art thinking in
\end{flushleft}


\begin{flushleft}
organizing for innovation, especially for the bottom of the pyramid. This
\end{flushleft}


\begin{flushleft}
theoretical knowledge will be complemented with hands-on exercises
\end{flushleft}


\begin{flushleft}
aimed to familiarize students with some of the key issues in coming
\end{flushleft}





220





\begin{flushleft}
\newpage
Humanities and Social Sciences
\end{flushleft}





\begin{flushleft}
up with technologies and products for the marginalized, including user
\end{flushleft}


\begin{flushleft}
needs and context analysis, ideation involving user interaction and
\end{flushleft}


\begin{flushleft}
co-creation, and assessment of the potential of technology for success.
\end{flushleft}





\begin{flushleft}
HUL706 Language, Society and Culture
\end{flushleft}


\begin{flushleft}
3 Credits (2-1-0)
\end{flushleft}


\begin{flushleft}
Psycho-linguistics and sociolinguistics; culture and identity studies;
\end{flushleft}


\begin{flushleft}
studies in expressive culture: idea-systems, myths and archtypes.
\end{flushleft}





\begin{flushleft}
HUL707 Social Psychology
\end{flushleft}


\begin{flushleft}
3 Credits (2-1-0)
\end{flushleft}


\begin{flushleft}
Schools of social psychology with special reference to personality and
\end{flushleft}


\begin{flushleft}
social structure. The problems and methods of social psychology.
\end{flushleft}


\begin{flushleft}
The association motive. Interpersonal attraction. Learning in social
\end{flushleft}


\begin{flushleft}
context. Social motives and attitudes. Social influence. Dissonance.
\end{flushleft}


\begin{flushleft}
Consonance and balance. Social status: Its effect on social motives and
\end{flushleft}


\begin{flushleft}
behaviour, social roles. Personality and social phenomenon. Cultural
\end{flushleft}


\begin{flushleft}
influences on personality and social behaviour. Social perception
\end{flushleft}


\begin{flushleft}
communication. Group process. Group task performance : Problem
\end{flushleft}


\begin{flushleft}
solving co-operation and competition. Leaders and leadership. Power
\end{flushleft}


\begin{flushleft}
and politics in organisations. Psychological processes in organizations.
\end{flushleft}


\begin{flushleft}
Aggression and its management.
\end{flushleft}





\begin{flushleft}
HUL709 Social Research Methods
\end{flushleft}


\begin{flushleft}
3 Credits (2-1-0)
\end{flushleft}


\begin{flushleft}
Scientific approach to social research. Concepts and indices. Analytical
\end{flushleft}


\begin{flushleft}
and formal aspects. Hypothesis formulation and testing strategies.
\end{flushleft}


\begin{flushleft}
Design of applied empirical research. Measurement and interpretation
\end{flushleft}


\begin{flushleft}
of social data. Social statistics. Sampling designs, report writing.
\end{flushleft}





\begin{flushleft}
HUL710 Personality Structure and Dynamics
\end{flushleft}


\begin{flushleft}
3 Credits (2-1-0)
\end{flushleft}


\begin{flushleft}
The topics for discussion will be : Coping with stress. Model of success
\end{flushleft}


\begin{flushleft}
and failure in adjustment. Approaches to the study of personality.
\end{flushleft}


\begin{flushleft}
Freud's classical psychoanalytic theory, Jung's analytic theory, Adler's
\end{flushleft}


\begin{flushleft}
individual psychology, Roger's person- centred approach. Lwein's
\end{flushleft}


\begin{flushleft}
field theory, Skinner's operant reinforcement theory. Erikson's theory:
\end{flushleft}


\begin{flushleft}
Psychohistorian perspective of man. Models of healthy personality;
\end{flushleft}


\begin{flushleft}
mature person: Allport's model. Self-actualising person: Maslow's
\end{flushleft}


\begin{flushleft}
model. Here-and-now person : Perl's model. Roger's theory : on
\end{flushleft}


\begin{flushleft}
becoming a person.
\end{flushleft}





\begin{flushleft}
HUL711 Psychological Testing \& Behavioral Assessment
\end{flushleft}


\begin{flushleft}
3 Credits (2-1-0)
\end{flushleft}


\begin{flushleft}
The concept of Behavioral Assessment: Uses and Varieties of
\end{flushleft}


\begin{flushleft}
Psychological Tests, Why Control the use of Psychological Tests? Test
\end{flushleft}


\begin{flushleft}
Administration, Examiner and Situational Variables and Effects of
\end{flushleft}


\begin{flushleft}
Training on Test Performance.
\end{flushleft}


\begin{flushleft}
Technical and Methodological Principles: Test Construction, Norms and
\end{flushleft}


\begin{flushleft}
The Meaning of Test Scores, Reliability and its Types, Validity and its
\end{flushleft}


\begin{flushleft}
Basic Concepts and Item Analysis.
\end{flushleft}


\begin{flushleft}
Other Techniques of Behavioral Assessment: Interview, Questionnaire
\end{flushleft}


\begin{flushleft}
and Schedule, Content Analysis, Observation as a tool of data collection,
\end{flushleft}


\begin{flushleft}
Rating Scales, Survey and Projective Techniques. Brief Review of some
\end{flushleft}


\begin{flushleft}
Selected Psychological Tests and Concluding Comments.
\end{flushleft}


\begin{flushleft}
Ethical and Social Considerations in Testing: Ethical Issues in Behavioral
\end{flushleft}


\begin{flushleft}
Assessment. User Qualifications and Professional Competence,
\end{flushleft}


\begin{flushleft}
Responsibility of Test Publishers, Protection of Privacy, Confidentiality
\end{flushleft}


\begin{flushleft}
and Communicating Test Results.
\end{flushleft}





\begin{flushleft}
HUL712 Microeconomics
\end{flushleft}


\begin{flushleft}
3 Credits (3-0-0)
\end{flushleft}


\begin{flushleft}
Pre-requisites: For UG students: HUL212\&HUL311
\end{flushleft}


\begin{flushleft}
This course provides an introduction to microeconomic theory and
\end{flushleft}


\begin{flushleft}
is the first course in the microeconomic theory series. The course
\end{flushleft}


\begin{flushleft}
will begin with detailed analysis of consumer's choice behavior and
\end{flushleft}


\begin{flushleft}
required mathematical tools from optimization theory and real analysis
\end{flushleft}


\begin{flushleft}
would be reviewed. Producer's behavior is analyzed next where
\end{flushleft}


\begin{flushleft}
emphasis is put on characterization results under different market
\end{flushleft}


\begin{flushleft}
structures, especially strategic aspects in an oligopolistic market.
\end{flushleft}


\begin{flushleft}
The next topic is analysis of decision-making under uncertainty and
\end{flushleft}


\begin{flushleft}
Anscombe-Aumann framework is introduced. Next non-expected utility
\end{flushleft}





\begin{flushleft}
theories are covered, Topics of recent and relevant interest will also
\end{flushleft}


\begin{flushleft}
be covered if time permits.
\end{flushleft}





\begin{flushleft}
HUL713 Macroeconomics
\end{flushleft}


\begin{flushleft}
3 Credits (3-0-0)
\end{flushleft}


\begin{flushleft}
Pre-requisites: For UG students--any ONE of: HUL211, HUL212,
\end{flushleft}


\begin{flushleft}
HUL213, HUL311, HUL312, HUL314, HUL315, HUL318, HUL320
\end{flushleft}


\begin{flushleft}
This course begins with a detailed study of macroeconomic concepts,
\end{flushleft}


\begin{flushleft}
which include an analysis of India's national income and balance
\end{flushleft}


\begin{flushleft}
of payments data. It provides an understanding of the contending
\end{flushleft}


\begin{flushleft}
theories of employment, income distribution, money supply, and
\end{flushleft}


\begin{flushleft}
price-wage relationships. The course also deals with exchange rates
\end{flushleft}


\begin{flushleft}
and other open economy macro issues. It discusses the classical and
\end{flushleft}


\begin{flushleft}
neoclassical theories of the macroeconomy, as also the critiques of
\end{flushleft}


\begin{flushleft}
these theories by Keynes, Kalecki, and their followers. The course
\end{flushleft}


\begin{flushleft}
examines macroeconomic policies, and the challenges faced by
\end{flushleft}


\begin{flushleft}
governments and the Central Banks in implementing them, especially in
\end{flushleft}


\begin{flushleft}
the context of the integrated nature of global finance and production.
\end{flushleft}





\begin{flushleft}
HUL714 International Economics
\end{flushleft}


\begin{flushleft}
3 Credits (3-0-0)
\end{flushleft}


\begin{flushleft}
Pre-requisites: For UG students--any ONE of: HUL211, HUL212,
\end{flushleft}


\begin{flushleft}
HUL213, HUL311, HUL312, HUL314, HUL318, HUL320
\end{flushleft}


\begin{flushleft}
This course discusses the various theories on trade, including the
\end{flushleft}


\begin{flushleft}
Ricardian and Heckscher-Ohlin models. It deals with instruments of
\end{flushleft}


\begin{flushleft}
trade policies, and also the political economy issues such as trade
\end{flushleft}


\begin{flushleft}
agreements under the WTO. It examines how international trade
\end{flushleft}


\begin{flushleft}
affect developing countries, with a particular emphasis on the Indian
\end{flushleft}


\begin{flushleft}
case. Further, the course will trace the emergence of the international
\end{flushleft}


\begin{flushleft}
monetary system, including the international gold standard and the
\end{flushleft}


\begin{flushleft}
Bretton Woods system. The ascent of global finance and its implications
\end{flushleft}


\begin{flushleft}
for macroeconomic policymaking will be covered in this course.
\end{flushleft}


\begin{flushleft}
Theories on finance, financial regulation and financial crises will also
\end{flushleft}


\begin{flushleft}
be discussed in this course.
\end{flushleft}





\begin{flushleft}
HUL715 Time Series Econometrics and Forecasting
\end{flushleft}


\begin{flushleft}
3 Credits (3-0-0)
\end{flushleft}


\begin{flushleft}
Pre-requisites: For UG students: HUL315 / HUL215
\end{flushleft}


\begin{flushleft}
1. Stationary Univariate Models : (a) Difference equation; (b) Wold's
\end{flushleft}


\begin{flushleft}
decomposition; (c) ARMA models; d. Box-Jenkins methodology; (e)
\end{flushleft}


\begin{flushleft}
Model Selection; (f) Forecasting.
\end{flushleft}


\begin{flushleft}
2. Non-stationary univariate models: (a) Trend/cyclical decomposition;
\end{flushleft}


\begin{flushleft}
(b) Deterministic and stochastic trend models; (c) Unit root tests; (d)
\end{flushleft}


\begin{flushleft}
Stationarity tests.
\end{flushleft}


\begin{flushleft}
3. Structural change and non-linear models: (a) Test for structural
\end{flushleft}


\begin{flushleft}
change with unknown change point; (b) Estimation of linear models
\end{flushleft}


\begin{flushleft}
with structural change; (c) Regime switching models.
\end{flushleft}


\begin{flushleft}
4. Stationary multivariate models; (a) Dynamic simulteneous equation
\end{flushleft}


\begin{flushleft}
models; (b) Vector Auto Regression (VAR); (c) Granger causality; (d)
\end{flushleft}


\begin{flushleft}
Impulse response function.
\end{flushleft}


\begin{flushleft}
5. Non-stationary multivariate models: (a) Spurious regression; (b)
\end{flushleft}


\begin{flushleft}
Co-integration; (c) Vector Error Correction (VECM) model.
\end{flushleft}


\begin{flushleft}
6. Time series model of heteroskedasticity: (a) ARCH, GARCH models.
\end{flushleft}





\begin{flushleft}
HUL716 Industrial Economics
\end{flushleft}


\begin{flushleft}
3 Credits (3-0-0)
\end{flushleft}


\begin{flushleft}
Pre-requisites: For UG students: HUL212 \& HUL311
\end{flushleft}


\begin{flushleft}
The course aims to formalize microeconomic treatment of industry
\end{flushleft}


\begin{flushleft}
and firm's behaviour, decision-making in consumer's choice problems,
\end{flushleft}


\begin{flushleft}
rationality theory (as well as its exceptions). Emphasis will be put
\end{flushleft}


\begin{flushleft}
to conceptualize various aspects of firm's and consumer's choice.
\end{flushleft}


\begin{flushleft}
Market structures, pricing under alternative market structures, market
\end{flushleft}


\begin{flushleft}
power and concentration will also form an integral part of the course.
\end{flushleft}


\begin{flushleft}
Behavioural and strategic aspects of the agents would be emphasized
\end{flushleft}


\begin{flushleft}
in various cases e.g. auctions, economic networks etc.
\end{flushleft}





\begin{flushleft}
HUL717 Perspectives on Indian Economy
\end{flushleft}


\begin{flushleft}
3 Credits (3-0-0)
\end{flushleft}


\begin{flushleft}
Pre-requisites: Any ONE of: HUL211, HUL212, HUL213,
\end{flushleft}


\begin{flushleft}
HUL311, HUL312, HUL314, HUL318, HUL320
\end{flushleft}





221





\begin{flushleft}
\newpage
Humanities and Social Sciences
\end{flushleft}





\begin{flushleft}
This course discusses the various phases in India's development
\end{flushleft}


\begin{flushleft}
transition. They include the economic changes during the colonial
\end{flushleft}


\begin{flushleft}
period, development under the planning regime, the transition from
\end{flushleft}


\begin{flushleft}
state to markets in India, and economic growth under liberalization.
\end{flushleft}


\begin{flushleft}
The course will deal with the varied inequalities in the country, along
\end{flushleft}


\begin{flushleft}
the lines of caste, class, and gender, as well as across regions. It will
\end{flushleft}


\begin{flushleft}
feature issues related to Indian agriculture, industry, services, as well
\end{flushleft}


\begin{flushleft}
as trade and investment. The course will aim to provide various points
\end{flushleft}


\begin{flushleft}
of view on each of th ese topics.
\end{flushleft}





\begin{flushleft}
its epistemological status, and the possibility of classifying kinds of
\end{flushleft}


\begin{flushleft}
text. It will also consider the different cultural ways of producing,
\end{flushleft}


\begin{flushleft}
circulating and relating to texts.
\end{flushleft}





\begin{flushleft}
HUL718 Political Economy of Development
\end{flushleft}


\begin{flushleft}
3 Credits (3-0-0)
\end{flushleft}


\begin{flushleft}
Pre-requisites: Any ONE of: HUL211, HUL212, HUL213,
\end{flushleft}


\begin{flushleft}
HUL311, HUL312, HUL314, HUL318, HUL320, HUL271, HUL272,
\end{flushleft}


\begin{flushleft}
HUL275, HUL281, HUL286, HUL289, HUL310
\end{flushleft}





\begin{flushleft}
A particular theoretical position would be explored through the detailed
\end{flushleft}


\begin{flushleft}
study of selected work which trace the history of the development of
\end{flushleft}


\begin{flushleft}
that critical position. The study would also include the analysis of a
\end{flushleft}


\begin{flushleft}
text which would illustrate the critical position being studied.
\end{flushleft}





\begin{flushleft}
The course will be a survey on the theories and issues related to the
\end{flushleft}


\begin{flushleft}
political economy of development. It discusses the emergence of
\end{flushleft}


\begin{flushleft}
industrial capitalism in Europe and North America, as well as its spread
\end{flushleft}


\begin{flushleft}
to third world countries since the mid-twentieth century. The course
\end{flushleft}


\begin{flushleft}
will deal with contemporary issues such as the growth of international
\end{flushleft}


\begin{flushleft}
trade and finance, the emergence of China as a global economic
\end{flushleft}


\begin{flushleft}
power, and the crisis in global capitalism that deepened since 2008.
\end{flushleft}


\begin{flushleft}
Issues related to human development, labour rights, migration and
\end{flushleft}


\begin{flushleft}
environmental sustainability will also be covered.
\end{flushleft}





\begin{flushleft}
HUL719 Advanced Econometrics
\end{flushleft}


\begin{flushleft}
3 Credits (3-0-0)
\end{flushleft}


\begin{flushleft}
Pre-requisites: For UG students: HUL315 / HUL215
\end{flushleft}


\begin{flushleft}
Course contents (about 100 words) (Include laboratory/design
\end{flushleft}


\begin{flushleft}
activities):
\end{flushleft}


\begin{flushleft}
1. Review of Classical Linear Regression Model: Gauss-Markov
\end{flushleft}


\begin{flushleft}
assumptions, finite sample properties, large sample properties.
\end{flushleft}


\begin{flushleft}
2. Instrumental Variable Estimation: Motivation for instrumentation,
\end{flushleft}


\begin{flushleft}
Simultaneity Bias, Endogeneity and Measurement Error; IV
\end{flushleft}


\begin{flushleft}
Estimation; 2SLS Estimation.
\end{flushleft}


\begin{flushleft}
3. Generalized Method of Moments: Single equation linear GMM.
\end{flushleft}


\begin{flushleft}
4. Systems of Equations: Seemingly Unrelated Regressions (SUR)
\end{flushleft}


\begin{flushleft}
model; Simultaneous Equations Models: Identification.
\end{flushleft}


\begin{flushleft}
5. Panel Data models: Pooled Estimation; Unobserved Heterogeneity:
\end{flushleft}


\begin{flushleft}
Fixed vs. Random Effects; ML vs. GMM estimation.
\end{flushleft}


\begin{flushleft}
6. Discrete Choice Models: Binary response models, Multinomial
\end{flushleft}


\begin{flushleft}
Response Models, Ordered Response Models.
\end{flushleft}


\begin{flushleft}
7. Censored Regression Models: Estimation and Inference with
\end{flushleft}


\begin{flushleft}
Censored Tobit.
\end{flushleft}


\begin{flushleft}
8. Estimating Average Treatment Effects: Regression Methods,
\end{flushleft}


\begin{flushleft}
Methods Based on the Propensity Score, Estimating the ATE Using IV.
\end{flushleft}





\begin{flushleft}
HUL720 Development Economics
\end{flushleft}


\begin{flushleft}
3 Credits (3-0-0)
\end{flushleft}


\begin{flushleft}
Pre-requisites: For UG students--any ONE of: HUL211, HUL212,
\end{flushleft}


\begin{flushleft}
HUL213, HUL311, HUL312, HUL314, HUL315, HUL318, HUL320
\end{flushleft}


\begin{flushleft}
This course discusses experiences in economic growth and
\end{flushleft}


\begin{flushleft}
development transitions from around the world. Some of the topics that
\end{flushleft}


\begin{flushleft}
will be covered in this course include poverty, inequality, education,
\end{flushleft}


\begin{flushleft}
health, and gender aspects of development. The course will deal with
\end{flushleft}


\begin{flushleft}
history and persistence in development, as well as with the roles of
\end{flushleft}


\begin{flushleft}
agrarian institutions and credit markets. Other topics covered will
\end{flushleft}


\begin{flushleft}
include culture, social capital, behavior, corruption, violence and
\end{flushleft}


\begin{flushleft}
conflict. The impacts of international trade, foreign aid, and foreign
\end{flushleft}


\begin{flushleft}
investment on development will also feature in this course.
\end{flushleft}





\begin{flushleft}
HUL731 What is a Text
\end{flushleft}


\begin{flushleft}
3 Credits (3-0-0)
\end{flushleft}


\begin{flushleft}
We will study the fundamental assumptions supporting the various
\end{flushleft}


\begin{flushleft}
definitions of text, and their possible mutual incompatibility, for the
\end{flushleft}


\begin{flushleft}
ways in which the question {``}what is a text'' exposes the issues in
\end{flushleft}


\begin{flushleft}
characterizing, interpreting and attributing meaning to text. The
\end{flushleft}


\begin{flushleft}
course will take into account hermeneutic, phenomenological and
\end{flushleft}


\begin{flushleft}
deconstructionist theories of text, the historicality of the idea of text,
\end{flushleft}


\begin{flushleft}
the distinctions between text and work, the metaphysics of text and
\end{flushleft}





\begin{flushleft}
HUL732 Contemporary Critical Theory
\end{flushleft}


\begin{flushleft}
3 Credits (3-0-0)
\end{flushleft}


\begin{flushleft}
Pre-requisites: For Ph.D. students: No prerequisites for all other
\end{flushleft}


\begin{flushleft}
students: Any one of the following: HUL331, HUL332, HUL333,
\end{flushleft}


\begin{flushleft}
HUL334, HUL335, HUL336, HUL337, HUL338, HUL339, HUL340
\end{flushleft}





\begin{flushleft}
Detailed course contents would be announced by the course
\end{flushleft}


\begin{flushleft}
coordinator at the time of offering the course.
\end{flushleft}





\begin{flushleft}
HUL733 Study of an Author/Writer in Focus
\end{flushleft}


\begin{flushleft}
3 Credits (3-0-0)
\end{flushleft}


\begin{flushleft}
$\bullet$ Brief biography and study of the historical/social context of the
\end{flushleft}


\begin{flushleft}
selected writer
\end{flushleft}


\begin{flushleft}
$\bullet$ Intellectual milieu of the writer
\end{flushleft}


\begin{flushleft}
$\bullet$ Overview of the major works and overall trajectory of the
\end{flushleft}


\begin{flushleft}
development of his/her thought
\end{flushleft}


\begin{flushleft}
$\bullet$ Understanding the influences and impact of the work/s
\end{flushleft}


\begin{flushleft}
$\bullet$ Detailed study of the selected text/s
\end{flushleft}





\begin{flushleft}
HUL734 Themes in Modern Indian Thought
\end{flushleft}


\begin{flushleft}
3 Credits (3-0-0)
\end{flushleft}


\begin{flushleft}
This course will focus on significant themes in modern Indian thought
\end{flushleft}


\begin{flushleft}
(Equality, Freedom, Sexuality, Gender, Caste, Religion, Violence,
\end{flushleft}


\begin{flushleft}
Modernity, Education, the Arts etc) and introduce students to major
\end{flushleft}


\begin{flushleft}
works that engage the specific theme(s) that have been chosen. Works
\end{flushleft}


\begin{flushleft}
studied may be cinematic, theatrical, fictional or non-fictional. The
\end{flushleft}


\begin{flushleft}
course will study both the genealogy of significant concepts in modern
\end{flushleft}


\begin{flushleft}
Indian thought (examining English as well as non-English language
\end{flushleft}


\begin{flushleft}
materials) as well as the range of debate about these concepts and
\end{flushleft}


\begin{flushleft}
their deployment. The course is envisaged as an interdisciplinary
\end{flushleft}


\begin{flushleft}
course, though we will pay close attention to questions of reading,
\end{flushleft}


\begin{flushleft}
textuality and interpretation. Lecture outline given for a possible course
\end{flushleft}


\begin{flushleft}
on gender and sexuality., as ONE possible example of a theme that
\end{flushleft}


\begin{flushleft}
could be pursued.
\end{flushleft}





\begin{flushleft}
HUL735 Research Methods in Economics
\end{flushleft}


\begin{flushleft}
2 Credits (1-0-2)
\end{flushleft}


\begin{flushleft}
Pre-requisites: Either HUL701 or HUL 707 or HUL736 OR
\end{flushleft}


\begin{flushleft}
HUL738 or HUL754 or HUL755 or HUL761 or HUL762
\end{flushleft}


\begin{flushleft}
The course will cover theory and practice of doing applied research
\end{flushleft}


\begin{flushleft}
in economics, with special emphasis on primary and secondary data
\end{flushleft}


\begin{flushleft}
uses. The course will familiarize students with sampling techniques,
\end{flushleft}


\begin{flushleft}
questionnaire design, implementation of field-based studies, including
\end{flushleft}


\begin{flushleft}
randomized controlled trials. Students will be provided training in
\end{flushleft}


\begin{flushleft}
STATA for carrying out data analysis, including use of data sets such
\end{flushleft}


\begin{flushleft}
as the National Sample Survey, National Family Health Survey, Indian
\end{flushleft}


\begin{flushleft}
Human Development Survey. Students will be expected to design and
\end{flushleft}


\begin{flushleft}
implement a small study during the course of the semester and will
\end{flushleft}


\begin{flushleft}
be evaluated on this.
\end{flushleft}





\begin{flushleft}
HUL736 Planning and Economic Development
\end{flushleft}


\begin{flushleft}
3 Credits (3-0-0)
\end{flushleft}


\begin{flushleft}
Economic growth. Economic development. Historic growth and
\end{flushleft}


\begin{flushleft}
contemporary development. Lessons and controversies. Characteristics
\end{flushleft}


\begin{flushleft}
of developing countries. Obstacles to development. Structural changes
\end{flushleft}


\begin{flushleft}
in the process of economic development. Relationship between
\end{flushleft}


\begin{flushleft}
agriculture and industry. Strategies of economic development.
\end{flushleft}


\begin{flushleft}
Balanced/ Unbalanced growth. International trade and economic
\end{flushleft}


\begin{flushleft}
development. Population. Planning for economic development. Use of
\end{flushleft}


\begin{flushleft}
input-output model and linear programming techniques in planning.
\end{flushleft}


\begin{flushleft}
Indian plan experience. Strategy of Indian planning. Indian plan models.
\end{flushleft}





222





\begin{flushleft}
\newpage
Humanities and Social Sciences
\end{flushleft}





\begin{flushleft}
HUL737 Advanced Economic Growth Theory
\end{flushleft}


\begin{flushleft}
3 Credits (3-0-0)
\end{flushleft}


\begin{flushleft}
Pre-requisites: either HUL736 or HUL738 or HUL755 or HUL762
\end{flushleft}


\begin{flushleft}
or any other 7XX level economics course.
\end{flushleft}


\begin{flushleft}
Primary objective of this course is to introduce students with the
\end{flushleft}


\begin{flushleft}
process of economic growth and the long run sources of differences
\end{flushleft}


\begin{flushleft}
in economic performances across nations. Emphasis will be placed on
\end{flushleft}


\begin{flushleft}
developing theoretical tool kits in understanding growth mechanics.
\end{flushleft}


\begin{flushleft}
It is intended that this course will make students learn some of the
\end{flushleft}


\begin{flushleft}
workhorse models in modern macroeconomics, namely, Solow model,
\end{flushleft}


\begin{flushleft}
Neo-classical model, overlapping generations' model, models with
\end{flushleft}


\begin{flushleft}
technological change and technology adoption etc.
\end{flushleft}





\begin{flushleft}
HUL738 International Economics
\end{flushleft}


\begin{flushleft}
3 Credits (2-1-0)
\end{flushleft}


\begin{flushleft}
The theory of International Trade. Impact of dynamic factors in
\end{flushleft}


\begin{flushleft}
International Trade. Free Trade, Protection. Economic integration
\end{flushleft}


\begin{flushleft}
and developing countries. The balance of payments. International
\end{flushleft}


\begin{flushleft}
capital movements. Rate of exchange. Relationship between Trade,
\end{flushleft}


\begin{flushleft}
Foreign Aid and Economic Development. Role of multinational
\end{flushleft}


\begin{flushleft}
corporations in developing countries. The IMF and the International
\end{flushleft}


\begin{flushleft}
Monetary System. Trade problems of developing countries. The new
\end{flushleft}


\begin{flushleft}
International Economic order. The structure and trends of India's
\end{flushleft}


\begin{flushleft}
foreign trade. India's balance of payments. India's trade policy. Indian
\end{flushleft}


\begin{flushleft}
and international financial institutions.
\end{flushleft}





\begin{flushleft}
HUL741 Sociolinguistics: Language Variation, Culture
\end{flushleft}


\begin{flushleft}
and Society
\end{flushleft}


\begin{flushleft}
3 Credits (3-0-0)
\end{flushleft}


\begin{flushleft}
Pre-requisites: HUL 234, HUL242 and HUL 350 for UG or Prior
\end{flushleft}


\begin{flushleft}
Permission of Coordinator
\end{flushleft}


\begin{flushleft}
This course aims at understanding variation mainly from a
\end{flushleft}


\begin{flushleft}
sociolinguistics perspective, but while also considering some relevant
\end{flushleft}


\begin{flushleft}
cues from generative views of the phenomenon. It will cover aspects
\end{flushleft}


\begin{flushleft}
of language change (bilingualism, multilingualism, language deaths,
\end{flushleft}


\begin{flushleft}
pidgin and creole formation etc.) as explained by feature-based and
\end{flushleft}


\begin{flushleft}
parameter-based grammars, as well by socio-cultural-political factors.
\end{flushleft}


\begin{flushleft}
The focus will then shift towards how homogenization of language also
\end{flushleft}


\begin{flushleft}
happens - combating the natural tendency towards variation - triggered
\end{flushleft}


\begin{flushleft}
by external factors. Concepts of race, gender, nation and identity will
\end{flushleft}


\begin{flushleft}
also be brought up to show the pervasive role of language in varied
\end{flushleft}


\begin{flushleft}
aspects of our socio-cultural-political lives.
\end{flushleft}





\begin{flushleft}
HUL742 Transformational Theories of Language
\end{flushleft}


\begin{flushleft}
3 Credits (3-0-0)
\end{flushleft}


\begin{flushleft}
Pre-requisites: HUL 234, HUL242 and HUL 350 for UG students
\end{flushleft}


\begin{flushleft}
and/or prior permission of the course coordinator
\end{flushleft}


\begin{flushleft}
This course will cover the fundamental concepts that have defined
\end{flushleft}


\begin{flushleft}
generative/transformational grammars since their inception in the
\end{flushleft}


\begin{flushleft}
1950s. It will introduce students to the main motivations for such
\end{flushleft}


\begin{flushleft}
grammars for natural language, as stated in Chomsky (1957).
\end{flushleft}


\begin{flushleft}
A substantial part of the course will therefore be devoted to
\end{flushleft}


\begin{flushleft}
understanding the inadequacies of immediate constituent analysis
\end{flushleft}


\begin{flushleft}
and the need to include optional and obligatory transformational
\end{flushleft}


\begin{flushleft}
rules in the grammar. This will be followed by a detailed study of later
\end{flushleft}


\begin{flushleft}
theoretical developments, including those found in Standard Theory,
\end{flushleft}


\begin{flushleft}
Extended Standard Theory, Revised Extended Standard Theory and
\end{flushleft}


\begin{flushleft}
Government and Binding Theory.
\end{flushleft}





\begin{flushleft}
HUL743 Language Acquisition, Teaching and Assessment
\end{flushleft}


\begin{flushleft}
3 Credits (3-0-0)
\end{flushleft}


\begin{flushleft}
This is a literature review course that will explore the existing
\end{flushleft}


\begin{flushleft}
literature in the domains of Language Acquisition (both first and
\end{flushleft}


\begin{flushleft}
second), Language teaching (approaches and methods), as well as
\end{flushleft}


\begin{flushleft}
language assessment. In doing so, the course will include aspects of
\end{flushleft}


\begin{flushleft}
the philosophy of language, and the resultant application of these
\end{flushleft}


\begin{flushleft}
philosophical approaches in the form of classroom pedagogy. The
\end{flushleft}


\begin{flushleft}
course will also include substantial literature on {``}action research''
\end{flushleft}


\begin{flushleft}
where language teachers have written about the results of
\end{flushleft}


\begin{flushleft}
implementing various cognitive tasks in their classroom.
\end{flushleft}





\begin{flushleft}
HUL745 Psycholinguistics
\end{flushleft}


\begin{flushleft}
3 Credits (3-0-0)
\end{flushleft}


\begin{flushleft}
We will first introduce the relevant questions, theories, methodologies
\end{flushleft}


\begin{flushleft}
with regard to the historical trajectory of Psycholinguistics. We will
\end{flushleft}


\begin{flushleft}
then look at language processing at different linguistic dimensions.
\end{flushleft}


\begin{flushleft}
We will start with words, their meaning and access. We then look at
\end{flushleft}


\begin{flushleft}
processing sentences. The course will also cover important topics
\end{flushleft}


\begin{flushleft}
such as language and speech production. Reading processes (and its
\end{flushleft}


\begin{flushleft}
relation to processing) will be covered. We course will also cover the
\end{flushleft}


\begin{flushleft}
current theories of Bilingualism and Aphasia.
\end{flushleft}





\begin{flushleft}
HUL746 Phonological Markedness
\end{flushleft}


\begin{flushleft}
3 Credits (2-0-2)
\end{flushleft}


\begin{flushleft}
Pre-requisites: HUL 234, HUL242 and HUL 350 for UG
\end{flushleft}


\begin{flushleft}
This course explores the connection between a unit of acoustic speech
\end{flushleft}


\begin{flushleft}
signal and its environment (sounds preceding or following it). Phonological
\end{flushleft}


\begin{flushleft}
theory is thus composed on context-free and context-sensitive
\end{flushleft}


\begin{flushleft}
notions of markedness. While these are supposed to be universal,
\end{flushleft}


\begin{flushleft}
individual languages might vary significantly in prioritizing between
\end{flushleft}


\begin{flushleft}
these. The course therefore involves a major practical component
\end{flushleft}


\begin{flushleft}
where the speech units of individual languages (vowels, consonants
\end{flushleft}


\begin{flushleft}
and tones) are studied with respect to their phonological contexts.
\end{flushleft}





\begin{flushleft}
HUL748 Community Psychology
\end{flushleft}


\begin{flushleft}
3 Credits (2-1-0)
\end{flushleft}


\begin{flushleft}
Concept of community and their implications for community psychology.
\end{flushleft}


\begin{flushleft}
Community processes and orientations toward change. Examinations
\end{flushleft}


\begin{flushleft}
of the models; the mental health model; the organizational model; the
\end{flushleft}


\begin{flushleft}
social action model; the ecological model. Implications for a psychology
\end{flushleft}


\begin{flushleft}
of the community : the study of community life, interaction strategies;
\end{flushleft}


\begin{flushleft}
implications for manpower and training; family therapy and the
\end{flushleft}


\begin{flushleft}
community; crisis intervention; advocacy and community psychology.
\end{flushleft}





\begin{flushleft}
HUL751 Critical Reading in Philosophical Texts
\end{flushleft}


\begin{flushleft}
3 Credits (3-0-0)
\end{flushleft}


\begin{flushleft}
The instructor will select a seminal text in philosophy and read it along
\end{flushleft}


\begin{flushleft}
with the class. Emphasis will be given to the textual material and issues
\end{flushleft}


\begin{flushleft}
in reading and understanding. An overview of the following will be
\end{flushleft}


\begin{flushleft}
provided: The nature the text, specificity of philosophical texts, text
\end{flushleft}


\begin{flushleft}
and context, issues in translation, interpretation and understanding.
\end{flushleft}





\begin{flushleft}
HUL752 Philosophy of Social Sciences
\end{flushleft}


\begin{flushleft}
3 Credits (3-0-0)
\end{flushleft}


\begin{flushleft}
Some of the key issues which arise in Social Sciences will be discussed
\end{flushleft}


\begin{flushleft}
in this course. These are : (1) What is {`}out there' in the social universe?
\end{flushleft}


\begin{flushleft}
(2) What are the most fundamental properties of the social world?
\end{flushleft}


\begin{flushleft}
(3) What kind(s) of analysis of these properties is (are) possible
\end{flushleft}


\begin{flushleft}
and/ or appropriate ? (4) What are the natures of theory, law, and
\end{flushleft}


\begin{flushleft}
explanation ? (5) Problems of reductionism. (6) Problems of free will
\end{flushleft}


\begin{flushleft}
versus determinism, purposeful behaviour, interpretations of actions.
\end{flushleft}


\begin{flushleft}
(7) Philosophical issues specific to various Social Sciences, e.g.,
\end{flushleft}


\begin{flushleft}
philosophical bases of various economic theories, or of theories of
\end{flushleft}


\begin{flushleft}
psychology, or issues regarding the assumptions concerning human
\end{flushleft}


\begin{flushleft}
nature made by various social science disciplines.
\end{flushleft}





\begin{flushleft}
HUL753 Philosophy of Science
\end{flushleft}


\begin{flushleft}
3 Credits (3-0-0)
\end{flushleft}


\begin{flushleft}
The course will address three sorts of questions. The first set involves the
\end{flushleft}


\begin{flushleft}
status of science as a privileged source of knowledge: what, if anything,
\end{flushleft}


\begin{flushleft}
justifies this status? The second set involves concepts such as 'law' ,
\end{flushleft}


\begin{flushleft}
'cause', and 'explanation' which occur within scientific practice: how
\end{flushleft}


\begin{flushleft}
are these to be understood? The third set involves understanding the
\end{flushleft}


\begin{flushleft}
relationship between different scientific enterprises: is there a hierarchy
\end{flushleft}


\begin{flushleft}
of sciences ranging from physics at one end to the human or social
\end{flushleft}


\begin{flushleft}
sciences at the other? If so, how should this hierarchy be understood?
\end{flushleft}





\begin{flushleft}
HUL754 The Philosophy of Plato
\end{flushleft}


\begin{flushleft}
3.0 Credits (3-0-0)
\end{flushleft}


\begin{flushleft}
This is a survey of Plato's thinking about politics, ethics, epistemology,
\end{flushleft}


\begin{flushleft}
and metaphysics. We will focus on a careful and critical reading of the
\end{flushleft}





223





\begin{flushleft}
\newpage
Humanities and Social Sciences
\end{flushleft}





\begin{flushleft}
primary texts, and attempt to get a sense both of their historical and
\end{flushleft}


\begin{flushleft}
cultural specificity, as well as their interest more generally as sources
\end{flushleft}


\begin{flushleft}
of philosophical insight.
\end{flushleft}





\begin{flushleft}
HUL755 Fascism: Philosophical Perspectives
\end{flushleft}


\begin{flushleft}
3 Credits (3-0-0)
\end{flushleft}


\begin{flushleft}
Fascism is one of the pernicious forms of power that emerged in the
\end{flushleft}


\begin{flushleft}
19th and 20th century and posed a serious to challenge democratic
\end{flushleft}


\begin{flushleft}
forms of life. The course will discuss the philosophical understanding
\end{flushleft}


\begin{flushleft}
of power, state, law, sovereignty and freedom with the special focus
\end{flushleft}


\begin{flushleft}
on gaining a conceptual grasp on fascism. Some of the classical and
\end{flushleft}


\begin{flushleft}
contemporary philosophers who have directly or indirectly contributed
\end{flushleft}


\begin{flushleft}
towards constructing the intellectual edifice of fascism will be studied.
\end{flushleft}


\begin{flushleft}
Philosophical criticisms of fascism and the ethical and political
\end{flushleft}


\begin{flushleft}
explorations of resistance and the possibility of alternative forms of
\end{flushleft}


\begin{flushleft}
power and governance will also be studied. The bio politics of fascism,
\end{flushleft}


\begin{flushleft}
its relationship with Nazism, racism and religious and other forms of
\end{flushleft}


\begin{flushleft}
fanaticism, and the aestheticization of politics will also be covered.
\end{flushleft}





\begin{flushleft}
HUL756 Philosophy and Film
\end{flushleft}


\begin{flushleft}
3 Credits (3-0-0)
\end{flushleft}


\begin{flushleft}
This course develops the conceptual resources of philosophy to
\end{flushleft}


\begin{flushleft}
respond to the cinematic image. The topics include: Ontology of
\end{flushleft}


\begin{flushleft}
the cinematic Image; cognitive and phenomenological approaches
\end{flushleft}


\begin{flushleft}
to perception and imagination; relationship between representation
\end{flushleft}


\begin{flushleft}
and reality and between seeing and saying; space, time and image;
\end{flushleft}


\begin{flushleft}
movement and animation, memory, history, narrative; anthropology
\end{flushleft}


\begin{flushleft}
of images, truth in cinema; Cinema as art; Cinema's relationship to
\end{flushleft}


\begin{flushleft}
painting and literature; cinema and technology, the digital image.
\end{flushleft}





\begin{flushleft}
HUL759 Urban Social Systems
\end{flushleft}


\begin{flushleft}
3 Credits (2-1-0)
\end{flushleft}


\begin{flushleft}
This course intends to impart a comprehensive and systematic
\end{flushleft}


\begin{flushleft}
understanding of urban social systems. Students completing this course
\end{flushleft}


\begin{flushleft}
will have a detailed knowledge of urban-growth and urban behaviour
\end{flushleft}


\begin{flushleft}
analysis, and urban- planning through a feedback analysis approach.
\end{flushleft}


\begin{flushleft}
Following will be the main course contents:
\end{flushleft}


\begin{flushleft}
Nature, types and growth of cities, Some important aspects of urbansystems: migration; neighbourhood; social groups; and voluntary
\end{flushleft}


\begin{flushleft}
associations. Trend of urbanisation. Urban influences on rural areas.
\end{flushleft}


\begin{flushleft}
A profile of urban India and its problems. Solution of the problems
\end{flushleft}


\begin{flushleft}
through various approaches. Urban planning.
\end{flushleft}





\begin{flushleft}
Psychology of legitimacy, Violence, reconciliation and peace will be
\end{flushleft}


\begin{flushleft}
examined.Environment and energy conservation issues and applied
\end{flushleft}


\begin{flushleft}
research and the criminal justice system will be critically examined.
\end{flushleft}


\begin{flushleft}
Social psychology and social policy implications will be discussed.
\end{flushleft}





\begin{flushleft}
HUL763 Cognitive Psychology
\end{flushleft}


\begin{flushleft}
3 Credits (3-0-0)
\end{flushleft}


\begin{flushleft}
The course will cover the following topics: Historical account of
\end{flushleft}


\begin{flushleft}
brain-mind, classification of cognition theories, methods of studying
\end{flushleft}


\begin{flushleft}
cognition, visual perception, top-down-bottom-up processing, visual
\end{flushleft}


\begin{flushleft}
recognition, information processing theories and attention, long-term
\end{flushleft}


\begin{flushleft}
memory, three stage theory of memory, types of memory, working
\end{flushleft}


\begin{flushleft}
memory and executive processing, emotion-cognition, decision making
\end{flushleft}


\begin{flushleft}
and dual process theories.
\end{flushleft}





\begin{flushleft}
HUL764 Psychological Interventions
\end{flushleft}


\begin{flushleft}
3 Credits (3-0-0)
\end{flushleft}


\begin{flushleft}
Introduction: Psychological Intervention modules including Yoga \&
\end{flushleft}


\begin{flushleft}
Meditation; relevant research methods of the field specifically for
\end{flushleft}


\begin{flushleft}
intervention programmes development and evaluation -research
\end{flushleft}


\begin{flushleft}
design, testing, evaluation of results etc; Broad objectives of the field;
\end{flushleft}


\begin{flushleft}
Intervention programmes in the various field of psychology: Applied
\end{flushleft}


\begin{flushleft}
positive psychology, health psychology, applied social psychology,
\end{flushleft}


\begin{flushleft}
community psychology, cognitive psychology etc.; Meta - analysis
\end{flushleft}


\begin{flushleft}
research on intervention programmes; Intervention programmes in
\end{flushleft}


\begin{flushleft}
Indian setting and role of socio --cultural factors; Critical evaluation
\end{flushleft}


\begin{flushleft}
and Future orientation of the field.
\end{flushleft}





\begin{flushleft}
HUL765 Psychological Testing and Behavioral
\end{flushleft}


\begin{flushleft}
Assessment
\end{flushleft}


\begin{flushleft}
3 Credits (3-0-0)
\end{flushleft}


\begin{flushleft}
Psychological testing: Uses and Varieties of Psychological Tests, Item
\end{flushleft}


\begin{flushleft}
Analysis, Norms and the Meaning of Tests Scores; Reliability and its
\end{flushleft}


\begin{flushleft}
Types; Validity and its Basic Concepts; Steps for Test Construction,
\end{flushleft}


\begin{flushleft}
Test adaptation and revalidation; Other Techniques of Behavioral
\end{flushleft}


\begin{flushleft}
Assessment; Ethical and Social Considerations in Testing; Ethical
\end{flushleft}


\begin{flushleft}
Guidelines in Behavioral Assessment.
\end{flushleft}





\begin{flushleft}
HUL771 Sociological Theory
\end{flushleft}


\begin{flushleft}
3 Credits (3-0-0)
\end{flushleft}


\begin{flushleft}
This is an advanced course that introduces students to a range of
\end{flushleft}


\begin{flushleft}
classical and contemporary sociological theory.
\end{flushleft}





\begin{flushleft}
HUL760 Industry and Society
\end{flushleft}


\begin{flushleft}
3 Credits (2-1-0)
\end{flushleft}





\begin{flushleft}
HUL772 Sociology of India
\end{flushleft}


\begin{flushleft}
3 Credits (3-0-0)
\end{flushleft}





\begin{flushleft}
The basic aim of this course is to introduce students from various
\end{flushleft}


\begin{flushleft}
backgrounds scientists, technologists to the study and understanding
\end{flushleft}


\begin{flushleft}
of modern industrial societies. The course material will focus on the
\end{flushleft}


\begin{flushleft}
following topics.
\end{flushleft}





\begin{flushleft}
The major themes covered in this course include the debates on
\end{flushleft}


\begin{flushleft}
continuity and change in relation to colonial rule, ideas of tradition
\end{flushleft}


\begin{flushleft}
and modernity, models of development, agrarian structure and rural
\end{flushleft}


\begin{flushleft}
transformation, marriage and family, caste and kinship, secularism,
\end{flushleft}


\begin{flushleft}
Subaltern religion and religious conflict, class and social mobility. The
\end{flushleft}


\begin{flushleft}
course takes a critical and engaged perspective on concepts such as
\end{flushleft}


\begin{flushleft}
the village, family, caste, region, nation, language, religion, gender,
\end{flushleft}


\begin{flushleft}
class, development, tradition, indigenousness, tribe, modernisation
\end{flushleft}


\begin{flushleft}
and others. Various approaches that have influenced the study of
\end{flushleft}


\begin{flushleft}
Indian society such as Orientalism, Indology, Structuralism, StructuralFunctionSubaltern Studies will also be discussed.
\end{flushleft}





\begin{flushleft}
Nature and type of industrial society. Workers in modern industrial
\end{flushleft}


\begin{flushleft}
societies: the work situation; alienation; and embourgeoisement.
\end{flushleft}


\begin{flushleft}
White collar worker. Trade-unionisation. Industrial democracy. Labourmanagement relations in Indian industries.
\end{flushleft}





\begin{flushleft}
HUL761 Theories of Psychology
\end{flushleft}


\begin{flushleft}
3 Credits (3-0-0)
\end{flushleft}


\begin{flushleft}
The course will provide a history of the discipline of psychology
\end{flushleft}


\begin{flushleft}
and its evolution over the years. Major schools of psychology will
\end{flushleft}


\begin{flushleft}
be discussed. The key theories -psychoanalytic theory, the various
\end{flushleft}


\begin{flushleft}
learning theories, theories of emotion and cognition, humanistic
\end{flushleft}


\begin{flushleft}
approaches and evolutionary perspective will be the focus. Social
\end{flushleft}


\begin{flushleft}
psychology theories, cognitive and neuroscience perspectives and
\end{flushleft}


\begin{flushleft}
positive psychology theories will be discussed in detail.
\end{flushleft}





\begin{flushleft}
HUL762 Social issues:Analysis and Policy
\end{flushleft}


\begin{flushleft}
3 Credits (3-0-0)
\end{flushleft}


\begin{flushleft}
The courses will focus on the following: Social psychological theory and
\end{flushleft}


\begin{flushleft}
application, examine the various methods of research, examine some
\end{flushleft}


\begin{flushleft}
social psychology applications and evaluate them. Issues of Inequality,
\end{flushleft}


\begin{flushleft}
deprivation and justice, in the context of Intergroup relations,
\end{flushleft}





\begin{flushleft}
HUL773 Media, Culture and Society
\end{flushleft}


\begin{flushleft}
3 Credits (3-0-0)
\end{flushleft}


\begin{flushleft}
The course examines contemporary manifestations of the 'mediaevent', the 'spectacle' and the fetishism of the image-object in
\end{flushleft}


\begin{flushleft}
determining the collective consciousness of our times. How are
\end{flushleft}


\begin{flushleft}
'media-events' created? What is the role of the media (this includes
\end{flushleft}


\begin{flushleft}
mass media, advertisements, as well as social and digital media) in
\end{flushleft}


\begin{flushleft}
determining the nature of the 'self' and 'society.'? How do media-trials
\end{flushleft}


\begin{flushleft}
alter the manner in which we relate to issues of justice and fairness?
\end{flushleft}


\begin{flushleft}
What is the manifestation of social movements in a media-saturated
\end{flushleft}


\begin{flushleft}
age? How do we recalibrate our understanding of privacy and what
\end{flushleft}


\begin{flushleft}
does this do to the ways in which we can create ever-changing and
\end{flushleft}


\begin{flushleft}
newer 'selves'? These questions will be examined through case studies
\end{flushleft}


\begin{flushleft}
from South Asia and beyond.
\end{flushleft}





224





\begin{flushleft}
\newpage
Humanities and Social Sciences
\end{flushleft}





\begin{flushleft}
HUL774 Visual Methods in Social Research
\end{flushleft}


\begin{flushleft}
3 Credits (3-0-0)
\end{flushleft}


\begin{flushleft}
The course analyses visual material in cross-cultural contexts and how
\end{flushleft}


\begin{flushleft}
the Internet in particular is being used to disseminate information and
\end{flushleft}


\begin{flushleft}
(re)present content. Importance of visual research methods; visuals as
\end{flushleft}


\begin{flushleft}
texts and framing; representing images and images in social research;
\end{flushleft}


\begin{flushleft}
Internet and online ethnography; analyzing websites - qualitative
\end{flushleft}


\begin{flushleft}
content; politics of digital culture.
\end{flushleft}





\begin{flushleft}
HUL775 Agrarian Societies and Rural Development
\end{flushleft}


\begin{flushleft}
3 Credits (3-0-0)
\end{flushleft}


\begin{flushleft}
Pre-requisites: For UG students: Any ONE course from HUL
\end{flushleft}


\begin{flushleft}
271, HUL 272, HUL 275, HUL 281, HUL 286, HUL 371, HUL
\end{flushleft}


\begin{flushleft}
372, HUL 375, HUL 376, HUL 377, HUL 378 OR HUL 380 OR
\end{flushleft}


\begin{flushleft}
any new 200 or 300 level Sociology course floated in future.
\end{flushleft}


\begin{flushleft}
For PG students: None
\end{flushleft}


\begin{flushleft}
The course will introduce students to theories related to agriculture
\end{flushleft}


\begin{flushleft}
and development including modernization theory, the rational peasant,
\end{flushleft}


\begin{flushleft}
moral economy, the agrarian question, modes of production debates,
\end{flushleft}


\begin{flushleft}
peasantry as a class, etc. Readings from the history of agriculture
\end{flushleft}


\begin{flushleft}
in various countries including the United States, Asia, Africa and
\end{flushleft}


\begin{flushleft}
India will be taught in comparative perspective. The course will
\end{flushleft}


\begin{flushleft}
help students understand the economic, social, cultural, ecological,
\end{flushleft}


\begin{flushleft}
political dimensions to the agrarian question, especially in the light of
\end{flushleft}


\begin{flushleft}
urbanization and globalization over the last 150 years.
\end{flushleft}





\begin{flushleft}
class. Institutionalization of gender via the state, family, marriage,
\end{flushleft}


\begin{flushleft}
religion etc.; the political economy of gender relating to reproduction,
\end{flushleft}


\begin{flushleft}
care, work and property. Issues of gender inequality, patriarchal
\end{flushleft}


\begin{flushleft}
oppression, violence, voice and agency.
\end{flushleft}





\begin{flushleft}
HUL782 Perspectives on Development in India
\end{flushleft}


\begin{flushleft}
3 Credits (3-0-0)
\end{flushleft}


\begin{flushleft}
Pre-requisites: For UG students: ONE course out of (HUL 212,
\end{flushleft}


\begin{flushleft}
HUL 213, HUL 311, HUL 312, HUL 314, HUL 315, HUL 316, HUL
\end{flushleft}


\begin{flushleft}
318, HUL 320 OR any new 200 level or 300 level Economics
\end{flushleft}


\begin{flushleft}
Courses floated in future) AND ONE course out of (HUL 271,
\end{flushleft}


\begin{flushleft}
HUL 272, HUL 275, HUL 281, HUL 286
\end{flushleft}


\begin{flushleft}
This seminar course will undertake a critical examination of the
\end{flushleft}


\begin{flushleft}
development process in India. The course introduces students to a
\end{flushleft}


\begin{flushleft}
historical overview of social, economic and political issues related
\end{flushleft}


\begin{flushleft}
to the ideas of development and growth. Starting from notions of
\end{flushleft}


\begin{flushleft}
improvement mooted under the colonial regime, to the processes
\end{flushleft}


\begin{flushleft}
of planning in independent India, the radical new agrarian policy of
\end{flushleft}


\begin{flushleft}
the 1960s and 70s, down to the era of liberalization in the 1990s and
\end{flushleft}


\begin{flushleft}
beyond, the course familiarises students with the political economy
\end{flushleft}


\begin{flushleft}
of development in India. It uses inter-disciplinary sources and texts
\end{flushleft}


\begin{flushleft}
to expose students to multiple ways of understanding and analyzing
\end{flushleft}


\begin{flushleft}
problems. Other topics covered include poverty and inequality,
\end{flushleft}


\begin{flushleft}
economics of discrimination (gender and caste) and the conflicts over
\end{flushleft}


\begin{flushleft}
land and natural resources in the 21st century.
\end{flushleft}





\begin{flushleft}
HUL783 Science, Technology and Society
\end{flushleft}


\begin{flushleft}
3 Credits (3-0-0)
\end{flushleft}





\begin{flushleft}
HUL776 Capitalism: Theory and Development
\end{flushleft}


\begin{flushleft}
3 Credits (3-0-0)
\end{flushleft}


\begin{flushleft}
What is capitalism and how did it emerge? What are its strengths
\end{flushleft}


\begin{flushleft}
and weaknesses? How have the social scientists analyzed it and
\end{flushleft}


\begin{flushleft}
understood its implications for and relationship with other social
\end{flushleft}


\begin{flushleft}
phenomena? Addressing these questions, this course discusses the
\end{flushleft}


\begin{flushleft}
historical development of capitalist institutions and social relations
\end{flushleft}


\begin{flushleft}
in the context of the advanced industrial and developing societies.
\end{flushleft}


\begin{flushleft}
Particularly, it analyses the various theories and paradigms of
\end{flushleft}


\begin{flushleft}
capitalistic development, such as the Marxist political economy,
\end{flushleft}


\begin{flushleft}
classical liberalism, world systems theory, economic history and neoliberalism.Furthermore, it analyses the relationship between the state
\end{flushleft}


\begin{flushleft}
and market, capitalism and liberal democracy, the religious roots of
\end{flushleft}


\begin{flushleft}
capitalism, social embeddedness of economic activity, and the {`}new
\end{flushleft}


\begin{flushleft}
realities' of capitalism, such as displacement, inequality and rampant
\end{flushleft}


\begin{flushleft}
environmental degradation.
\end{flushleft}





\begin{flushleft}
HUL777 Sociology of Science
\end{flushleft}


\begin{flushleft}
3 Credits (3-0-0)
\end{flushleft}


\begin{flushleft}
Basic theories in the sociology of science such as functionalism,
\end{flushleft}


\begin{flushleft}
the theory of paradigm shift, social construction of scientific facts,
\end{flushleft}


\begin{flushleft}
stratification and discrimination in science, feminist epistemologies
\end{flushleft}


\begin{flushleft}
of science, theories of standardization and objectivity. Historical and
\end{flushleft}


\begin{flushleft}
contemporary debates on scientific and indigenous knowledge from
\end{flushleft}


\begin{flushleft}
India and the world, relationship between science and the state, and
\end{flushleft}


\begin{flushleft}
role of experts and the public in evaluating and regulating science.
\end{flushleft}





\begin{flushleft}
HUL778 Urban Sociology
\end{flushleft}


\begin{flushleft}
3 Credits (3-0-0)
\end{flushleft}


\begin{flushleft}
This seminar course critically examines the production of urban space
\end{flushleft}


\begin{flushleft}
and culture. The {`}urban' denotes an aspect of physical space as much
\end{flushleft}


\begin{flushleft}
as a way of life and a mentality. A critical reading of ethnographic
\end{flushleft}


\begin{flushleft}
studies on the city provides a cross-cultural perspective on how space
\end{flushleft}


\begin{flushleft}
becomes culturally meaningful. The rise of the urban centre and
\end{flushleft}


\begin{flushleft}
metropolis are the product of a certain historical moment, yet they also
\end{flushleft}


\begin{flushleft}
produce distinctive mentalities and cultures that are unique to them.
\end{flushleft}


\begin{flushleft}
The course will explore the structuring and contestation of urban space
\end{flushleft}


\begin{flushleft}
through categories of class, ethnicity, status and gender and study
\end{flushleft}


\begin{flushleft}
the city as the location of discourses of and struggles for citizenship.
\end{flushleft}





\begin{flushleft}
HUL779 Gender and Society
\end{flushleft}


\begin{flushleft}
3 Credits (3-0-0)
\end{flushleft}


\begin{flushleft}
Sex and gender; masculinities, gender as performance and identity;
\end{flushleft}


\begin{flushleft}
sexuality and gender identities, masculinity and femininity. Hegemonic
\end{flushleft}


\begin{flushleft}
masculinity; Inter-sections of gender and race, ethnicity, caste and
\end{flushleft}





\begin{flushleft}
Introduction to the discipline of Science and Technology Studies
\end{flushleft}


\begin{flushleft}
covering topics such as the technological determinism, social
\end{flushleft}


\begin{flushleft}
construction of technology (SCOT), actor-network theory, laboratory
\end{flushleft}


\begin{flushleft}
studies, scientific controversies, the theory of paradigm shift, social
\end{flushleft}


\begin{flushleft}
construction of knowledge, feminist theories of science and technology,
\end{flushleft}


\begin{flushleft}
the idea of technoscience, risk society, ethics in engineering, and
\end{flushleft}


\begin{flushleft}
the role of experts and the public in evaluating and regulating the
\end{flushleft}


\begin{flushleft}
production of science and technology.
\end{flushleft}





\begin{flushleft}
HUL800 Research Writing
\end{flushleft}


\begin{flushleft}
3 Credits (3-0-0)
\end{flushleft}


\begin{flushleft}
The course will include aspects of writing composition and stylistics that
\end{flushleft}


\begin{flushleft}
are essential to write a coherent research paper/abstract. Topics will
\end{flushleft}


\begin{flushleft}
include text structure, common writing mistakes, ethical issues, etc.
\end{flushleft}


\begin{flushleft}
This will be a hands-on course; it will extensively use in-class exercises
\end{flushleft}


\begin{flushleft}
(as well as assignments) to help students learn the necessary skills.
\end{flushleft}





\begin{flushleft}
HUL801 Law, Technology and Citizenship
\end{flushleft}


\begin{flushleft}
3 Credits (3-0-0)
\end{flushleft}


\begin{flushleft}
Although there has been considerable focus in political theory and
\end{flushleft}


\begin{flushleft}
legal studies on the concept of citizenship, and its relationship with
\end{flushleft}


\begin{flushleft}
the law, through the last six decades, the study of the importance
\end{flushleft}


\begin{flushleft}
of technology to this relationship is only an emerging field. Four
\end{flushleft}


\begin{flushleft}
performative sites of citizenship discourse/citizen action viz., (a)
\end{flushleft}


\begin{flushleft}
human rights approaches and the regulation of technology, (b)
\end{flushleft}


\begin{flushleft}
Surveillance state and citizenship, (c) Technological ethics as a site
\end{flushleft}


\begin{flushleft}
for citizenship discourse, and d) Posthuman citizen, are focused
\end{flushleft}


\begin{flushleft}
upon to offer possible (conceptual and practical) implications for the
\end{flushleft}


\begin{flushleft}
ways in which {`}law and technology' impacts existing rights discourse.
\end{flushleft}


\begin{flushleft}
Further, four important sites of contemporary debates on technology
\end{flushleft}


\begin{flushleft}
and citizenship, viz., UID/Aadhaar, Human DNA profiling for crime
\end{flushleft}


\begin{flushleft}
control, nuclear technology and genetically modified technologies
\end{flushleft}


\begin{flushleft}
in agriculture are focussed on to contextualise the key issues that
\end{flushleft}


\begin{flushleft}
are identified in the earlier modules.
\end{flushleft}





\begin{flushleft}
HUL810 Advanced Topics in Policy Studies
\end{flushleft}


\begin{flushleft}
3 Credits (3-0-0)
\end{flushleft}


\begin{flushleft}
This course will introduce students to advanced topics in Policy
\end{flushleft}


\begin{flushleft}
Studies as decided by the instructor.
\end{flushleft}





\begin{flushleft}
HUL811 Advanced Economic Growth Theory
\end{flushleft}


\begin{flushleft}
3 Credits (3-0-0)
\end{flushleft}


\begin{flushleft}
Pre-requisites: For UG students-any ONE of: HUl212, HUL213,
\end{flushleft}





225





\begin{flushleft}
\newpage
Humanities and Social Sciences
\end{flushleft}





\begin{flushleft}
HUL311, HUL312, HUL211, HUL314, HUL318, HUL320
\end{flushleft}


\begin{flushleft}
The course aims to develop an understanding of the process
\end{flushleft}


\begin{flushleft}
of economic growth and income distribution in an economy.
\end{flushleft}


\begin{flushleft}
Historical and contemporary experiences of countries on growth
\end{flushleft}


\begin{flushleft}
and developmental outcomes will be dealt with in this course.
\end{flushleft}


\begin{flushleft}
The impacts of capital accumulation, technological progress, and
\end{flushleft}


\begin{flushleft}
international trade on economic performance will be discussed.
\end{flushleft}


\begin{flushleft}
Further, this course will focus on how global capital movements,
\end{flushleft}


\begin{flushleft}
domestic institutions and political economy can affect economic
\end{flushleft}


\begin{flushleft}
growth and development.
\end{flushleft}





\begin{flushleft}
HUL812 Advanced International Trade
\end{flushleft}


\begin{flushleft}
3 Credits (3-0-0)
\end{flushleft}


\begin{flushleft}
Pre-requisites: For UG students-any ONE of: HUl212, HUL213,
\end{flushleft}


\begin{flushleft}
HUL311, HUL312, HUL211, HUL314, HUL318, HUL320
\end{flushleft}


\begin{flushleft}
The contents of this course will include topics such as - Ricardian
\end{flushleft}


\begin{flushleft}
and Hecksher-Ohlin models, their extension to many goods and
\end{flushleft}


\begin{flushleft}
factors, the role of tariffs, quotas, and other trade policies, trade
\end{flushleft}


\begin{flushleft}
under imperfect competition, outsourcing, political economy,
\end{flushleft}


\begin{flushleft}
multinationals, trade and growth, gravity equation, organization
\end{flushleft}


\begin{flushleft}
of the firms, etc.
\end{flushleft}





\begin{flushleft}
HUL813 Foundations of Decision Theory
\end{flushleft}


\begin{flushleft}
3 Credits (3-0-0)
\end{flushleft}


\begin{flushleft}
Pre-requisites: HUL212
\end{flushleft}


\begin{flushleft}
The course aims to formalize microeconomic treatment of decisionmaking by economic agents. It will encompass consumer's choice
\end{flushleft}


\begin{flushleft}
problems, rationality theory and also bounded rationality theory.
\end{flushleft}


\begin{flushleft}
The course will conceptualize behavioural aspects of firm's decisionmaking- non-cooperation strategies and cartel formation would be
\end{flushleft}


\begin{flushleft}
discussed with respect to various market structures. Basic ideas
\end{flushleft}


\begin{flushleft}
of auction models would also be discussed and reference would
\end{flushleft}


\begin{flushleft}
be made to e-auctions and spectrum (or natural resource) auction
\end{flushleft}


\begin{flushleft}
markets. Latest developments in social and economic networks would
\end{flushleft}


\begin{flushleft}
be introduced and behavioural underpinnings would be discussed.
\end{flushleft}





\begin{flushleft}
HUL814 Research Methods in Economics
\end{flushleft}


\begin{flushleft}
2 Credits (1-0-2)
\end{flushleft}


\begin{flushleft}
Pre-requisites: For M.Tech.: Any two of the following HUL712,
\end{flushleft}


\begin{flushleft}
HUL713, HUL714, HUL715, HUL716, HUL717, HUL718, HUL719,
\end{flushleft}


\begin{flushleft}
HUL720. For UG: Any two of the following: HUL311, HUL312,
\end{flushleft}


\begin{flushleft}
HUL314, HUL316, HUL318, HUL320, HUL712, HUL713, HUL714,
\end{flushleft}


\begin{flushleft}
HUL715, HUL716, HUL717, HUL718, HUL719
\end{flushleft}


\begin{flushleft}
The course will cover theory and practice of doing applied research
\end{flushleft}


\begin{flushleft}
in economics, with special emphasis on primary and secondary
\end{flushleft}


\begin{flushleft}
data uses. The course will familiarize students with sampling
\end{flushleft}


\begin{flushleft}
techniques, questionnaire design, implementation of field-based
\end{flushleft}


\begin{flushleft}
studies, including randomized controlled trials. Students will be
\end{flushleft}


\begin{flushleft}
provided training in STATA for carrying out data analysis, including
\end{flushleft}


\begin{flushleft}
use of data sets such as the National Sample Survey, National Family
\end{flushleft}


\begin{flushleft}
Health Survey, Indian Human Development Survey. Students will be
\end{flushleft}


\begin{flushleft}
expected to design and implement a small study during the course
\end{flushleft}


\begin{flushleft}
of the semester and will be evaluated on this.
\end{flushleft}





\begin{flushleft}
HUL820 Advanced Topics in Economics
\end{flushleft}


\begin{flushleft}
3 Credits (3-0-0)
\end{flushleft}


\begin{flushleft}
Pre-requisites: For UG students-any ONE of: HUL211, HUL212,
\end{flushleft}


\begin{flushleft}
HUL213, HUL215, HUL311, HUL312, HUL314, HUL315, HUL318,
\end{flushleft}


\begin{flushleft}
HUL320
\end{flushleft}


\begin{flushleft}
This course will introduce students to advanced topics in Economics
\end{flushleft}


\begin{flushleft}
as decided by the instructor.
\end{flushleft}





\begin{flushleft}
HUL821 Performance/ Theatre: Theory and Practice
\end{flushleft}


\begin{flushleft}
3 Credits (3-0-0)
\end{flushleft}


\begin{flushleft}
The course would introduce students to the theories of performance
\end{flushleft}


\begin{flushleft}
and a selection of theatrical practices. Reading theatrical perspectives
\end{flushleft}


\begin{flushleft}
on the study of performances, alongside studying the development
\end{flushleft}


\begin{flushleft}
of theatre practices and the insights offered by various theatre
\end{flushleft}


\begin{flushleft}
practitioners would prepare the student for studying the performative.
\end{flushleft}





\begin{flushleft}
HUL823 Contemporary Critical Theory
\end{flushleft}


\begin{flushleft}
3 Credits (2-1-0)
\end{flushleft}


\begin{flushleft}
Recent developments in linguistics, philosophy and the social sciences;
\end{flushleft}


\begin{flushleft}
interdisciplinary cross-talk in these areas, concerning the status of
\end{flushleft}


\begin{flushleft}
canonical literary as well as marginal texts; feminist, post-modernist,
\end{flushleft}


\begin{flushleft}
post-colonial, subaltern, orientalist, new historicist, liberal Marxist
\end{flushleft}


\begin{flushleft}
and critical practice. The aim of the course is to familiarise students
\end{flushleft}


\begin{flushleft}
with some of the vocabulary of theoretical inquiry today, so that they
\end{flushleft}


\begin{flushleft}
are enabled in their own research to question the verities which their
\end{flushleft}


\begin{flushleft}
disciplines seem to offer.
\end{flushleft}





\begin{flushleft}
HUL831 Authorship and Copyright
\end{flushleft}


\begin{flushleft}
3 Credits (3-0-0)
\end{flushleft}


\begin{flushleft}
Pre-requisites: No prerequisite for Ph.D. For UG any ONE of
\end{flushleft}


\begin{flushleft}
the following: HUL331, HUL332, HUL333, HUL334, HUL335,
\end{flushleft}


\begin{flushleft}
HUL336, HUL338, HUL340, HUL351, HUL352, HUL353, HUL354,
\end{flushleft}


\begin{flushleft}
HUL355, HUL356, HUL357, HUL358, HUL359, HUL360, HUL375
\end{flushleft}


\begin{flushleft}
The course would study the history of the print while keeping in
\end{flushleft}


\begin{flushleft}
perspective the changes in transmission of knowledge brought about
\end{flushleft}


\begin{flushleft}
by changes in technologies of representation -- oral, manuscript, print.
\end{flushleft}


\begin{flushleft}
The coming of print is accompanied by the regulation of knowledge
\end{flushleft}


\begin{flushleft}
circulation by systems of profit. This amalgamation leads to the
\end{flushleft}


\begin{flushleft}
emergence of the idea of copyright which is further strengthened
\end{flushleft}


\begin{flushleft}
by the conceptualisation of the author as a genius. The course will
\end{flushleft}


\begin{flushleft}
study the prospects of the concepts of the author and copyright in
\end{flushleft}


\begin{flushleft}
the digital age.
\end{flushleft}





\begin{flushleft}
HUL832 South Asian Writing
\end{flushleft}


\begin{flushleft}
3 Credits (3-0-0)
\end{flushleft}


\begin{flushleft}
The course will include discussions on the place of the English
\end{flushleft}


\begin{flushleft}
language and {``}imported'' literary forms in South Asia, the fragmented
\end{flushleft}


\begin{flushleft}
and divided terrain of the South Asian city/nation, the figure of the
\end{flushleft}


\begin{flushleft}
expatriate writer, and the context within which to understand the
\end{flushleft}


\begin{flushleft}
stylistic and narrative aspects of this writing. It will undertake detailed
\end{flushleft}


\begin{flushleft}
analyses of the works of 3-4 writers, out of a longer list comprising
\end{flushleft}


\begin{flushleft}
Anita Desai, G V Desani, Salman Rushdie, Amitav Ghosh, Vikram
\end{flushleft}


\begin{flushleft}
Seth, Kiran Nagarkar, Aravind Adiga, Jeet Thayyil, Mohammed Hanif,
\end{flushleft}


\begin{flushleft}
Mohsin Hamid, Shyam Sevadurai, Romesh Gunesekhera, and others.
\end{flushleft}





\begin{flushleft}
HUL833 The English Renaissance: Selfhood and Survival
\end{flushleft}


\begin{flushleft}
3 Credits (3-0-0)
\end{flushleft}


\begin{flushleft}
The idea of the Renaissance, the historical, political and social context
\end{flushleft}


\begin{flushleft}
The idea of the self and how it was conceived of during this period
\end{flushleft}


\begin{flushleft}
as different from previous notions The importance of the stage and
\end{flushleft}


\begin{flushleft}
theatre in Elizabethan England Shakespeare, Marlow and other
\end{flushleft}


\begin{flushleft}
dramatists Milton, John Donne and other poets.
\end{flushleft}





\begin{flushleft}
HUL834 Literature and the City
\end{flushleft}


\begin{flushleft}
3 Credits (3-0-0)
\end{flushleft}


\begin{flushleft}
The course examines in some detail the nature of the challenge
\end{flushleft}


\begin{flushleft}
that traditionally preoccupied European writers - how to map the
\end{flushleft}


\begin{flushleft}
experience of the modern city, and what representational strategies
\end{flushleft}


\begin{flushleft}
were adequate for capturing the opacity, the fragmentation, and the
\end{flushleft}


\begin{flushleft}
transitory nature of urban modernity. It goes on to investigate the
\end{flushleft}


\begin{flushleft}
contemporary postcolonial city in order to understand it in relation to
\end{flushleft}


\begin{flushleft}
late capitalism, globalization, migration, and postmodern culture, and
\end{flushleft}


\begin{flushleft}
the challenges these pose to classic modernity. It begins by providing
\end{flushleft}


\begin{flushleft}
an introduction to some of the most important literature on the city
\end{flushleft}


\begin{flushleft}
and the major theoretical debates around it, offering students a set of
\end{flushleft}


\begin{flushleft}
conceptual tools with which to approach the city's incommensurable
\end{flushleft}


\begin{flushleft}
realities, its problems and its potential. It moves on to a detailed
\end{flushleft}


\begin{flushleft}
analysis of a number of literary texts, examining some of the ways
\end{flushleft}


\begin{flushleft}
in which the disjunctive realities of city-life shape new modes of
\end{flushleft}


\begin{flushleft}
experience, creative expression, and solidarity, without losing sight
\end{flushleft}


\begin{flushleft}
of the inequities of gender, culture, class, and race that persist and
\end{flushleft}


\begin{flushleft}
indeed strengthen in the current global economic system.
\end{flushleft}





\begin{flushleft}
HUL835 Modern Indian Theatre
\end{flushleft}


\begin{flushleft}
3 Credits (3-0-0)
\end{flushleft}


\begin{flushleft}
$\bullet$ History of modern Indian theatre through its relationship with colonial
\end{flushleft}





226





\begin{flushleft}
\newpage
Humanities and Social Sciences
\end{flushleft}





\begin{flushleft}
to post-colonial and nationalist concerns. $\bullet$ Understanding the concept
\end{flushleft}


\begin{flushleft}
of modernity and its contested and changing forms in urban Indian
\end{flushleft}


\begin{flushleft}
theatre. $\bullet$ Examining the trajectory of modern Indian theatre from the
\end{flushleft}


\begin{flushleft}
formation of institutions such as the National School of Drama as well
\end{flushleft}


\begin{flushleft}
as movements such as IPTA. $\bullet$ The negotiation of modern theatre with
\end{flushleft}


\begin{flushleft}
its colonial and pre-colonial past -- the Theatre of Roots. $\bullet$ The impact
\end{flushleft}


\begin{flushleft}
and influence of the Parsi Theatre and the Marathi Sangeet Nataks.
\end{flushleft}


\begin{flushleft}
$\bullet$ Development and concerns of original English theatre in India. $\bullet$
\end{flushleft}


\begin{flushleft}
Studies of individual plays and playwrights within the aesthetic and
\end{flushleft}


\begin{flushleft}
political context of their productions.
\end{flushleft}





\begin{flushleft}
HUL836 Performance/Theatre: Theory/Practice
\end{flushleft}


\begin{flushleft}
3 Credits (3-0-0)
\end{flushleft}


\begin{flushleft}
Pre-requisites: No Pre-requisite for Ph.D. for UG any one of the
\end{flushleft}


\begin{flushleft}
following: HUL333, HUL335
\end{flushleft}


\begin{flushleft}
The course will look upon performance practices both within and
\end{flushleft}


\begin{flushleft}
beyond the theatre viz. the spectacular, the digital media, sports etc.
\end{flushleft}


\begin{flushleft}
It will take the students through a variety of performance practices and
\end{flushleft}


\begin{flushleft}
across the world. The history of the development of performance forms
\end{flushleft}


\begin{flushleft}
and conditions of performance would be studied. Special focus would
\end{flushleft}


\begin{flushleft}
be on the theorisation of theatre and performance both by theatre
\end{flushleft}


\begin{flushleft}
practitioners and those emanating from the area of Performance
\end{flushleft}


\begin{flushleft}
Studies. The role of performance in maintaining and countering
\end{flushleft}


\begin{flushleft}
relations of power would be explored. Students would be required to
\end{flushleft}


\begin{flushleft}
observe, study and analyse live performances as part of the course.
\end{flushleft}





\begin{flushleft}
HUL840 Advanced Topics in Literature
\end{flushleft}


\begin{flushleft}
3 Credits (3-0-0)
\end{flushleft}


\begin{flushleft}
This course will introduce students to advanced topics in Literature
\end{flushleft}


\begin{flushleft}
as decided by the instructor.
\end{flushleft}





\begin{flushleft}
HUL841 Minimalist Architecture of Grammar
\end{flushleft}


\begin{flushleft}
3 Credits (3-0-0)
\end{flushleft}


\begin{flushleft}
Pre-requisites: HUL 742: Transformational Theories of
\end{flushleft}


\begin{flushleft}
Language Or prior permission of the course co-ordinator
\end{flushleft}


\begin{flushleft}
his is an advanced course in theoretical syntax and will benefit students
\end{flushleft}


\begin{flushleft}
interested in learning more about recent generative syntactic theories.
\end{flushleft}


\begin{flushleft}
There are two main objectives of the course: a) to provide the rationale
\end{flushleft}


\begin{flushleft}
for the {`}Strong Minimalist Thesis: language as an optimal system'; and,
\end{flushleft}


\begin{flushleft}
b) to learn to generate syntactic structures using fewer transformations/
\end{flushleft}


\begin{flushleft}
operations and features and, with stronger economy considerations.
\end{flushleft}


\begin{flushleft}
On finishing the course, students will be familiar with both theoretical
\end{flushleft}


\begin{flushleft}
(substantive) aspects of the minimalist architecture of grammar,
\end{flushleft}


\begin{flushleft}
as well the technical (methodological) know-how of the system.
\end{flushleft}





\begin{flushleft}
HUL842 Prosodic Morphology
\end{flushleft}


\begin{flushleft}
3 Credits (2-0-2)
\end{flushleft}


\begin{flushleft}
Pre-requisites: HUL 234, HUL242 and HUL 350 for UG
\end{flushleft}


\begin{flushleft}
This course explores the connection between sounds and words.
\end{flushleft}


\begin{flushleft}
While the correlation between the meaning and sounds of a word are
\end{flushleft}


\begin{flushleft}
arbitrary, the string of sounds or phonological shape of words and
\end{flushleft}


\begin{flushleft}
other morphological units follow certain universal well-formedness
\end{flushleft}


\begin{flushleft}
principles. While many of these principles derive from the manner
\end{flushleft}


\begin{flushleft}
in which individual languages form rhythmic/ prosodic domains in
\end{flushleft}


\begin{flushleft}
speech, some make crucial reference to morphological notions such
\end{flushleft}


\begin{flushleft}
as homonymy and synonymy. The aim of the course is to impart
\end{flushleft}


\begin{flushleft}
certain theoretical tools to analyze words from any natural language
\end{flushleft}


\begin{flushleft}
data. Since the objective of the course is to learn to analyze natural
\end{flushleft}


\begin{flushleft}
language data, it involves a substantial practical component.
\end{flushleft}





\begin{flushleft}
HUL843 Reading and Sentence Processing
\end{flushleft}


\begin{flushleft}
3 Credits (2-1-0)
\end{flushleft}


\begin{flushleft}
The course content will cover state-of-the-art models of reading. The
\end{flushleft}


\begin{flushleft}
course will situate itself in the larger domain of sentence processing
\end{flushleft}


\begin{flushleft}
and address the important question of how reading and sentence
\end{flushleft}


\begin{flushleft}
processing are related. Distributed (SWIFT) as well as undistributed
\end{flushleft}


\begin{flushleft}
(EZ Reader) attention models will be discussed. Influence of low-level
\end{flushleft}


\begin{flushleft}
factors (eg. Word length, word frequency, etc) and contribution of highlevel sentential factors on reading patterns will be introduced. Work
\end{flushleft}


\begin{flushleft}
on Indian languages will be discussed. Finally, models that integrate
\end{flushleft}


\begin{flushleft}
sentence processing and reading will be taken up.
\end{flushleft}





\begin{flushleft}
[Note: Once an eyetracker is available, hands on sessions will be
\end{flushleft}


\begin{flushleft}
conducted to run simple reading experiments. See tutorial section
\end{flushleft}


\begin{flushleft}
for more details.
\end{flushleft}





\begin{flushleft}
HUL845 Environmental Ethics
\end{flushleft}


\begin{flushleft}
3 Credits (3-0-0)
\end{flushleft}


\begin{flushleft}
Objectives: To acquaint the student with (a) philosophical concepts
\end{flushleft}


\begin{flushleft}
underlying thinking about the environmental crisis and (b) the
\end{flushleft}


\begin{flushleft}
models of human-nature relationship found in some of the classical
\end{flushleft}


\begin{flushleft}
philosophical systems of India.
\end{flushleft}


\begin{flushleft}
Contents: (a) What is {`}environment'? (b) Conceptual basis for the
\end{flushleft}


\begin{flushleft}
split between {`}nature' and {`}culture' (c) Philosophical theories about
\end{flushleft}


\begin{flushleft}
the environment: Utilitarianism: Deep Ecology: Ecofeminism. (d)
\end{flushleft}


\begin{flushleft}
Non-humans as recipients of moral consideration (e) Environment
\end{flushleft}


\begin{flushleft}
and Gender (f) Environment and Development (g) The Third World
\end{flushleft}


\begin{flushleft}
perspective (h) Revisioning Ethics, Metaphysics and Epistemology in
\end{flushleft}


\begin{flushleft}
the light of the above debates.
\end{flushleft}





\begin{flushleft}
HUL846 Philosophy and Film
\end{flushleft}


\begin{flushleft}
3 Credits (3-0-0)
\end{flushleft}


\begin{flushleft}
Nature of cinematic representation: Illusion, image, reality. Perception
\end{flushleft}


\begin{flushleft}
of image: Analytical, cognitive and phenomenological theories,
\end{flushleft}


\begin{flushleft}
Interpretation of film: meaning, authorship, Intention, Image and
\end{flushleft}


\begin{flushleft}
emotional response.
\end{flushleft}


\begin{flushleft}
Film Theories: Classical theories: Eisenstein, Arnheim, Bazin, Pudovkin,
\end{flushleft}


\begin{flushleft}
Contemporary theories: Semiotics, Psychoanalysis, Marxism, Poststructuralism, Feminism, Auteur theory.
\end{flushleft}


\begin{flushleft}
Aesthetics of Film: Cinema as art, entertainment and technology,
\end{flushleft}


\begin{flushleft}
Cinema's relationship with literature and other arts, Cinema and Digital
\end{flushleft}


\begin{flushleft}
Art, Aesthetics of interactive cinema, Aesthetics of special effects.
\end{flushleft}





\begin{flushleft}
HUL850 Advanced Topics in Linguistics
\end{flushleft}


\begin{flushleft}
3 Credits (3-0-0)
\end{flushleft}


\begin{flushleft}
This course will introduce students to advanced topics in Linguistics
\end{flushleft}


\begin{flushleft}
as decided by the instructor.
\end{flushleft}





\begin{flushleft}
HUL851 Philosophy of Literature
\end{flushleft}


\begin{flushleft}
3 Credits (3-0-0)
\end{flushleft}


\begin{flushleft}
The course examines the philosophical bases and problems that
\end{flushleft}


\begin{flushleft}
define key literary and literary-theoretical concepts, such as text,
\end{flushleft}


\begin{flushleft}
context, paratext, literary history, narration, meaning, interpretation,
\end{flushleft}


\begin{flushleft}
voice, style, literary specificity.Through the study texts of philosophy
\end{flushleft}


\begin{flushleft}
(both Anglo-American and European), literature and literary theory,
\end{flushleft}


\begin{flushleft}
which have influenced or responded to each other, the following
\end{flushleft}


\begin{flushleft}
topics and questions will be addressed: The ontological status of the
\end{flushleft}


\begin{flushleft}
text-context discontinuity; Through what concept of difference do we
\end{flushleft}


\begin{flushleft}
think the specificity of the literary? The epistemology of literature;
\end{flushleft}


\begin{flushleft}
Fictionality, Possible Worlds;Through what concept of existence do we
\end{flushleft}


\begin{flushleft}
distinguish literature from other phenomena, such as, hypotheses, lies,
\end{flushleft}


\begin{flushleft}
counterfactuals, dreams? Literature and/as Moral Philosophy; Is there
\end{flushleft}


\begin{flushleft}
a law of literature or does literature constitute legality itself? How does
\end{flushleft}


\begin{flushleft}
literature relate to non-literary, scientific, and everyday discourses?
\end{flushleft}


\begin{flushleft}
Life as Narrative and theories of narrative self; The relation between
\end{flushleft}


\begin{flushleft}
literature, aesthesis and reason; and emotional response to Fiction.
\end{flushleft}





\begin{flushleft}
HUL852 Political Philosophy
\end{flushleft}


\begin{flushleft}
3 Credits (3-0-0)
\end{flushleft}


\begin{flushleft}
This course will introduce students to key concepts and theories
\end{flushleft}


\begin{flushleft}
in political philosophy, such as justice, democracy, citizenship,
\end{flushleft}


\begin{flushleft}
secularism, sovereignty, equality, rights, and freedom. The approach
\end{flushleft}


\begin{flushleft}
will sometimes be historical, involving an intense engagement with the
\end{flushleft}


\begin{flushleft}
work of a particular thinker or philosophical tradition; at other times is
\end{flushleft}


\begin{flushleft}
will be thematic, taking up a particular notion such as secularism and
\end{flushleft}


\begin{flushleft}
addressing it from many different points of view. The emphasis will be
\end{flushleft}


\begin{flushleft}
on a close and rigorous reading of these texts, while also addressing
\end{flushleft}


\begin{flushleft}
questions about their contemporary relevance. The lecture outline is
\end{flushleft}


\begin{flushleft}
for ONE possible course on key texts in the Western liberal tradition.
\end{flushleft}





\begin{flushleft}
HUL853 Art and Aesthetics
\end{flushleft}


\begin{flushleft}
3 Credits (3-0-0)
\end{flushleft}


\begin{flushleft}
$\bullet$ Aesthetic Attitude and Aesthetic Experience $\bullet$ The Ontology of Art: on
\end{flushleft}


\begin{flushleft}
what kind of a thing is a work of art $\bullet$ Theories of Art: Resemblance,
\end{flushleft}


\begin{flushleft}
Representation, Expression and Form $\bullet$ Aesthetic Judgement:
\end{flushleft}





227





\begin{flushleft}
\newpage
Humanities and Social Sciences
\end{flushleft}





\begin{flushleft}
Perception and Imagination; The Sublime and the Beautiful $\bullet$ Emotional
\end{flushleft}


\begin{flushleft}
Response to Fiction $\bullet$ Criticism and Interpretation: on whethr criticalinterpretative cannons are fixed or open-ended $\bullet$ Art, Tradition and
\end{flushleft}


\begin{flushleft}
Modernity $\bullet$ Art, Morality and Politics: Art as Ideology $\bullet$ Metaphor,
\end{flushleft}


\begin{flushleft}
Narrative and Fictionality $\bullet$ Philosophy and Literature: on Literature
\end{flushleft}


\begin{flushleft}
as Cognitive Thought-Experiment about Human Possibility
\end{flushleft}





\begin{flushleft}
HUL854 Problems in Metaphysics
\end{flushleft}


\begin{flushleft}
3 Credits (3-0-0)
\end{flushleft}


\begin{flushleft}
Ontological issues concerning God or Necessary Being, Mind, Self and
\end{flushleft}


\begin{flushleft}
Personal Identity, Universals and Particulars, Primary and Secondary
\end{flushleft}


\begin{flushleft}
Qualities, and Fictional Objects . Understanding the place of Mind in the
\end{flushleft}


\begin{flushleft}
Natural World, the distinction between Being-in-itself and Being-foritself, Agency and Freedom, Subjective and Objective, Consciousness
\end{flushleft}


\begin{flushleft}
and Self-Consciousness, and the notion of Inter-subjectivity. Special
\end{flushleft}


\begin{flushleft}
study on the conceptualization of reality in terms of Phenomena and
\end{flushleft}


\begin{flushleft}
Noumena will form an important part of the course.
\end{flushleft}





\begin{flushleft}
HUL856 Philosophy of Language
\end{flushleft}


\begin{flushleft}
3 Credits (3-0-0)
\end{flushleft}


\begin{flushleft}
The course is a study of four major topics: Reference and Descriptions;
\end{flushleft}


\begin{flushleft}
Truth and Meaning; Pragmatics and Speech Acts; Language and
\end{flushleft}


\begin{flushleft}
Metaphor. The course covers analyses of the following specific
\end{flushleft}


\begin{flushleft}
concepts: Sense, Reference, Descriptions, Proper Names, Natural
\end{flushleft}


\begin{flushleft}
Kind Terms, Truth, Intentional semantics, Communicative Utterances;
\end{flushleft}


\begin{flushleft}
Figurative Speech.
\end{flushleft}





\begin{flushleft}
HUL857 Epistemology
\end{flushleft}


\begin{flushleft}
3 Credits (3-0-0)
\end{flushleft}


\begin{flushleft}
Necessary and Sufficient conditions of Knowledge and the Gettier
\end{flushleft}


\begin{flushleft}
Problem. Theories of Epistemic Justification: Evidentialism and
\end{flushleft}


\begin{flushleft}
Reliabilism; Internalism and Externalism; Foundationalism and
\end{flushleft}


\begin{flushleft}
Coherentism. Scepticism: Philosophical Scepticism and Ordinary
\end{flushleft}


\begin{flushleft}
Incredulity; Semantic Contextualism and Inferential Contextualism.
\end{flushleft}


\begin{flushleft}
Social Epistemology: Epistemology and Collective Doxastic Agency;
\end{flushleft}


\begin{flushleft}
Epistemology of Testimony. Virtue Epistemology: Knowledge and
\end{flushleft}


\begin{flushleft}
Intellectual Virtues; Epistemic Values. Feminist Epistemology: Critique
\end{flushleft}


\begin{flushleft}
of Rationality and Gender Duality; Feminist Epistemology of Science;
\end{flushleft}


\begin{flushleft}
Feminist Naturalized Epistemology: Standpoint Theory.
\end{flushleft}





\begin{flushleft}
HUL860 Advanced Topics in Philosophy
\end{flushleft}


\begin{flushleft}
3 Credits (3-0-0)
\end{flushleft}


\begin{flushleft}
This course will introduce students to advanced topics in Philosophy
\end{flushleft}


\begin{flushleft}
as decided by the instructor.
\end{flushleft}





\begin{flushleft}
HUL861 Psychology of Decision Making
\end{flushleft}


\begin{flushleft}
3 Credits (3-0-0)
\end{flushleft}


\begin{flushleft}
The course will cover topics on psychological constructs affecting
\end{flushleft}


\begin{flushleft}
decision making (e.g., IQ, memory, motivation, emotion), decisionmaking processes (e.g., information search, risk perception), decisionmaking contexts (e.g., constraints, culture), and applications of
\end{flushleft}


\begin{flushleft}
behavioral decision making related to consumption (e.g., environment,
\end{flushleft}


\begin{flushleft}
technology, consumer decision making).
\end{flushleft}





\begin{flushleft}
HUL862 Special Module in Cognitive Psychology
\end{flushleft}


\begin{flushleft}
3 Credits (2-0-2)
\end{flushleft}


\begin{flushleft}
The course will cover brief history of cognitive psychology, approaches,
\end{flushleft}


\begin{flushleft}
theoretical frameworks, and current issues in cognitive psychology.
\end{flushleft}





\begin{flushleft}
HUL863 Emotion and Cognition
\end{flushleft}


\begin{flushleft}
3 Credits (3-0-0)
\end{flushleft}


\begin{flushleft}
The course will cover the following topics: theoretical approaches to
\end{flushleft}


\begin{flushleft}
emotion (evolutionary, biological, social, cognitive), select emotions
\end{flushleft}


\begin{flushleft}
and emotion expression (e.g., anger, fear, sadness, joy, surprise,
\end{flushleft}


\begin{flushleft}
disgust), and implications of emotion and cognition (e.g., stresshealth, sex-differences)
\end{flushleft}





\begin{flushleft}
HUL870 Advanced Topics in Psychology
\end{flushleft}


\begin{flushleft}
3 Credits (3-0-0)
\end{flushleft}


\begin{flushleft}
This course will introduce students to advanced topics in Policy Studies
\end{flushleft}


\begin{flushleft}
as decided by the instructor.
\end{flushleft}





\begin{flushleft}
HUL871 Ethnographic Perspectives on the State
\end{flushleft}


\begin{flushleft}
3 Credits (3-0-0)
\end{flushleft}


\begin{flushleft}
Traditionally studied by political scientists, the state has increasingly
\end{flushleft}


\begin{flushleft}
come to be regarded as an object of anthropological study.
\end{flushleft}


\begin{flushleft}
Ethnographic perspectives on the state seek to open up the state to
\end{flushleft}


\begin{flushleft}
critical scrutiny, dislodging it as a monolithic conceptual or territorial
\end{flushleft}


\begin{flushleft}
apparatus. These studies allow us to think of the state beyond
\end{flushleft}


\begin{flushleft}
governmentality or bureaucracy, to engaging with the multiple ways
\end{flushleft}


\begin{flushleft}
in which state {`}effects' shape our engagement with it. How does the
\end{flushleft}


\begin{flushleft}
{`}idea' of the state constrain the way in which we {`}think' the state?
\end{flushleft}


\begin{flushleft}
What are the ethnographic sites through which the state emerges
\end{flushleft}


\begin{flushleft}
as an object of study, e.g., bureaucracy, law, sexuality, marriage,
\end{flushleft}


\begin{flushleft}
citizenship, borders etc? The course will consist of seminars designed
\end{flushleft}


\begin{flushleft}
around a set of readings, which will be discussed in detail each week.
\end{flushleft}





\begin{flushleft}
HUL872 Sexuality, Governmentality, and the State
\end{flushleft}


\begin{flushleft}
3 Credits (3-0-0)
\end{flushleft}


\begin{flushleft}
Sexual governance or state surveillance of issues pertaining to
\end{flushleft}


\begin{flushleft}
sexuality, marriage and mobility highlight the problematic division
\end{flushleft}


\begin{flushleft}
between the {`}public' and the {`}private'. The {`}intimate' sphere is no longer
\end{flushleft}


\begin{flushleft}
one that is outside the purview of states; indeed the {`}private' or the
\end{flushleft}


\begin{flushleft}
{`}intimate' is often co-produced as a corollary of the public face of a
\end{flushleft}


\begin{flushleft}
state's legitimacy. Nationalism and patriotism are heavily grounded in
\end{flushleft}


\begin{flushleft}
issues pertaining to culture and sexuality, often thought of as {`}private'.
\end{flushleft}


\begin{flushleft}
This course will provide a historical and sociological perspective to
\end{flushleft}


\begin{flushleft}
how sexual governance -- the control of women's sexuality, conjugality
\end{flushleft}


\begin{flushleft}
and the definition of {`}marriage' and the {`}family' by a patriarchal state
\end{flushleft}


\begin{flushleft}
shows that the state has always concerned itself with the intimate
\end{flushleft}


\begin{flushleft}
lives of its citizens.
\end{flushleft}





\begin{flushleft}
HUL873 Language, Culture and Society
\end{flushleft}


\begin{flushleft}
3 Credits (3-0-0)
\end{flushleft}


\begin{flushleft}
What is language? How does it relate to the {`}collective consciousness'
\end{flushleft}


\begin{flushleft}
of a society? How does language relate to ideology and when does
\end{flushleft}


\begin{flushleft}
language become {`}linguistic capital'? This course introduces students
\end{flushleft}


\begin{flushleft}
to some theoretical approaches to the study of language in social
\end{flushleft}


\begin{flushleft}
anthropology, such as structuralism and Marxism. This will be followed
\end{flushleft}


\begin{flushleft}
by studies of language movements, language policy in colonial and
\end{flushleft}


\begin{flushleft}
postcolonial India. The course concludes with questions of {`}free
\end{flushleft}


\begin{flushleft}
speech', {`}hate speech' and some debates on censorship.
\end{flushleft}





\begin{flushleft}
HUL874 Civil Society and Democracy in India
\end{flushleft}


\begin{flushleft}
3 Credits (3-0-0)
\end{flushleft}


\begin{flushleft}
This is a post-graduate level seminar based course. The objective
\end{flushleft}


\begin{flushleft}
of this course is to discuss the complex and contingent relationships
\end{flushleft}


\begin{flushleft}
between state, market and civil society in India and examine the
\end{flushleft}


\begin{flushleft}
implications of their relationships for the broader processes of
\end{flushleft}


\begin{flushleft}
development, democratization, citizenship rights and governance in
\end{flushleft}


\begin{flushleft}
India. The course begins with an overview of the classical and modern
\end{flushleft}


\begin{flushleft}
theories of the state and civil society. Students read Hobbes, Locke,
\end{flushleft}


\begin{flushleft}
Hegel, Marx, Gramsci, Tocqueville and other political theorists. The
\end{flushleft}


\begin{flushleft}
course will then focus on the role that civil society has played in Indian
\end{flushleft}


\begin{flushleft}
development and democracy. The course will discuss topics such
\end{flushleft}


\begin{flushleft}
as civil society and political society, NGO-ization, non-party political
\end{flushleft}


\begin{flushleft}
processes, social capital and ethnic conflict, economic roots of civil
\end{flushleft}


\begin{flushleft}
society and participatory development and democratization.
\end{flushleft}





\begin{flushleft}
HUL875 Ethnic Identity, Development and
\end{flushleft}


\begin{flushleft}
Democratization in North-east India
\end{flushleft}


\begin{flushleft}
3 Credits (3-0-0)
\end{flushleft}


\begin{flushleft}
Making of Northeast India (NEI) -- past and present; Regional
\end{flushleft}


\begin{flushleft}
identity and Nationalism; Look East policy and Vision 2020; Identity
\end{flushleft}


\begin{flushleft}
politics, ethnic affirmation, territorial sovereignty and ethnic violence;
\end{flushleft}


\begin{flushleft}
underdevelopment and development challenges; social movements,
\end{flushleft}


\begin{flushleft}
ethnic movements for political autonomy and secessionism; responses
\end{flushleft}


\begin{flushleft}
of Indian state and AFSPA; gender, tribalism, race, and religion; civil
\end{flushleft}


\begin{flushleft}
society in NEI; human development report.
\end{flushleft}





\begin{flushleft}
HUL877 Industry and Society
\end{flushleft}


\begin{flushleft}
3 Credits (3-0-0)
\end{flushleft}


\begin{flushleft}
The course material will include the following topics: evolution of
\end{flushleft}





228





\begin{flushleft}
\newpage
Humanities and Social Sciences
\end{flushleft}





\begin{flushleft}
industrial society, industry and industrialization, class and work
\end{flushleft}


\begin{flushleft}
in modern industrial societies, alienation and embourgeoisement,
\end{flushleft}


\begin{flushleft}
labour management relations and labour reforms, family in
\end{flushleft}


\begin{flushleft}
industrial society, formal and informal sector, technology and new
\end{flushleft}


\begin{flushleft}
economy, industry, industrial resources and new social movement,
\end{flushleft}


\begin{flushleft}
post-industrial society.
\end{flushleft}





\begin{flushleft}
HUL878 Globalization
\end{flushleft}


\begin{flushleft}
3 Credits (3-0-0)
\end{flushleft}


\begin{flushleft}
Globalization and Globalism; Economic Globalization and NeoLiberalism; Political Globalization; Social and Cultural Globalization;
\end{flushleft}


\begin{flushleft}
Civil Society and International Politics; Anti-Globalization.
\end{flushleft}





\begin{flushleft}
HUL879 Political Ecology as a Development Critique
\end{flushleft}


\begin{flushleft}
3 Credits (3-0-0)
\end{flushleft}


\begin{flushleft}
Questions of conflict over natural resources, the conservation of
\end{flushleft}


\begin{flushleft}
biodiversity under market environmentalism, the political ecology of
\end{flushleft}


\begin{flushleft}
farming and industry, the emergence of environmental movements,
\end{flushleft}


\begin{flushleft}
the political ecology of indigenous people, feminist political ecology,
\end{flushleft}


\begin{flushleft}
urban ecology, environmental justice, and degrowth comprise core
\end{flushleft}


\begin{flushleft}
concerns of this course. The influence of globalization and neoliberalism provides a rich context to understand these contestations
\end{flushleft}


\begin{flushleft}
and conflict over resource distribution. These propel the debates on
\end{flushleft}


\begin{flushleft}
ecological utopias. Case Studies include (any two per semester):
\end{flushleft}


\begin{flushleft}
forestry; industry and mining; body and health; climate change; water;
\end{flushleft}


\begin{flushleft}
political ecology of tribal areas of India.
\end{flushleft}





\begin{flushleft}
HUL880 Advanced Topics in Sociology
\end{flushleft}


\begin{flushleft}
3 Credits (3-0-0)
\end{flushleft}


\begin{flushleft}
This course will introduce students to advanced topics in Sociology
\end{flushleft}


\begin{flushleft}
as decided by the instructor.
\end{flushleft}





\begin{flushleft}
HUL881 Narratology: Foundations, Domains, Frontiers
\end{flushleft}


\begin{flushleft}
3 Credits (3-0-0)
\end{flushleft}


\begin{flushleft}
The course will familiarise students with the beginnings of this field
\end{flushleft}


\begin{flushleft}
of study in Russian formalism, structural linguistics and anthropology,
\end{flushleft}


\begin{flushleft}
and then its entry into literary studies, discourse and stylistics. The
\end{flushleft}


\begin{flushleft}
course will trace the development of narratological concepts (e.g.
\end{flushleft}


\begin{flushleft}
fabula/sujet, narrative voice, focalisation, paratext metalepsis,
\end{flushleft}


\begin{flushleft}
unreliability, free indirect speech, orientation, evaluation, coda etc.)
\end{flushleft}


\begin{flushleft}
within schools of thought since the 1970s on. It will visit the debates
\end{flushleft}


\begin{flushleft}
on narrativity and gender, race, history, ideology, culture and cognition.
\end{flushleft}


\begin{flushleft}
The spread of narrative theory beyond literary works to other areas
\end{flushleft}


\begin{flushleft}
such as comic books and video games, as well as its relevance to
\end{flushleft}


\begin{flushleft}
other disciplinary inquires in sociology, legal studies, political theory,
\end{flushleft}


\begin{flushleft}
postcolonial theory and psychology will be discussed. The course will
\end{flushleft}


\begin{flushleft}
consider the philosophical questions of narrative and temporality,
\end{flushleft}


\begin{flushleft}
anti-narrative, subjectivity, language, action, personhood, framing,
\end{flushleft}


\begin{flushleft}
closure and evolutionary theory.
\end{flushleft}





\begin{flushleft}
HUL882 CyberPower and Cyber-Protest
\end{flushleft}


\begin{flushleft}
3 Credits (3-0-0)
\end{flushleft}


\begin{flushleft}
Network Society and the Internet, Cyberspace and the Virtual
\end{flushleft}


\begin{flushleft}
Individual; ICT Outreach, Social Inclusion, and Digital Divide in
\end{flushleft}


\begin{flushleft}
Developing countries; Digital Democracy and the Online Public
\end{flushleft}


\begin{flushleft}
Sphere; Cyberpower, and Cyberpolitics; Smart mobs; Cyberprotest;
\end{flushleft}


\begin{flushleft}
Case Studies
\end{flushleft}





\begin{flushleft}
HUL883 History and Revolution
\end{flushleft}


\begin{flushleft}
3 Credits (3-0-0)
\end{flushleft}


\begin{flushleft}
The course will study major concepts of historicality and revolution in
\end{flushleft}


\begin{flushleft}
order to examine the role played by revolution in bringing or blocking
\end{flushleft}


\begin{flushleft}
historical change, in breaking with certain social and intellectual
\end{flushleft}


\begin{flushleft}
patterns. It will respond to questions about the nature of the prerevolutionary moment emergent within existing historical situations
\end{flushleft}


\begin{flushleft}
and yet departing from them; the designation of selective historical
\end{flushleft}


\begin{flushleft}
moments as revolutionary; the variety of domains beyond the narrowly
\end{flushleft}


\begin{flushleft}
defined domain of politics that have seen revolution, for instance, in
\end{flushleft}


\begin{flushleft}
science, technology, social relations, and philosophy. With respect to
\end{flushleft}


\begin{flushleft}
political revolutions, how is a revolution in history analysed, and what
\end{flushleft}


\begin{flushleft}
happens to sovereignty? What is the role of violence in revolution as
\end{flushleft}





\begin{flushleft}
compared to other categories of history (period, epoch), and how does
\end{flushleft}


\begin{flushleft}
it affect language itself, both of literary representations of revolution
\end{flushleft}


\begin{flushleft}
and of historiography?
\end{flushleft}





\begin{flushleft}
HUL884 Environmental Ethics
\end{flushleft}


\begin{flushleft}
3 Credits (3-0-0)
\end{flushleft}


\begin{flushleft}
Pre-requisites: HUL275
\end{flushleft}


\begin{flushleft}
The course introduces different understandings about categories
\end{flushleft}


\begin{flushleft}
of {`}environment' and {`}environmentalism' that have emerged in
\end{flushleft}


\begin{flushleft}
contemporary thought, and its implications to its study in within an
\end{flushleft}


\begin{flushleft}
ethics framework. It seeks to explore three tropes. First, is a foray into
\end{flushleft}


\begin{flushleft}
the nature-culture debate, a debate central to environmental ethics. It
\end{flushleft}


\begin{flushleft}
seeks to lay the basis for the field by tracing key texts in the debate,
\end{flushleft}


\begin{flushleft}
viz., how the category of {`}nature' is understood to be something
\end{flushleft}


\begin{flushleft}
which is external to humans. Second, we seek to understand the
\end{flushleft}


\begin{flushleft}
ways in which the {`}crisis' in environment is constructed, a crisis
\end{flushleft}


\begin{flushleft}
which then would require certain ethical approaches to amelioration
\end{flushleft}


\begin{flushleft}
of our relationship with our surroundings. Third, is an exploration of
\end{flushleft}


\begin{flushleft}
specific themes in the field of contemporary environmental ethics critical environmental aesthetics, applied ethics in agriculture, and
\end{flushleft}


\begin{flushleft}
explore ethical frameworks from non Western realms like in the Indic
\end{flushleft}


\begin{flushleft}
context, and Buddhist environmental ethics. This course looks at the
\end{flushleft}


\begin{flushleft}
imperatives and politics that shaped the literatures and discourses
\end{flushleft}


\begin{flushleft}
that shaped environmental ethics as a distinct discipline.
\end{flushleft}





\begin{flushleft}
HUL885 Criticism, Crisis and Critique
\end{flushleft}


\begin{flushleft}
3 Credits (3-0-0)
\end{flushleft}


\begin{flushleft}
In this course we will inquire into some or all of the following questions
\end{flushleft}


\begin{flushleft}
pertaining to the relation between criticism, crisis, critique and critical
\end{flushleft}


\begin{flushleft}
theory: a) the development of criticism as a part of the literary as
\end{flushleft}


\begin{flushleft}
well as philosophical inquiries into morals and tastes from antiquity
\end{flushleft}


\begin{flushleft}
to the present; b) the manner in which the development of the idea
\end{flushleft}


\begin{flushleft}
of critique in the context of the crisis that accompanied 18th century
\end{flushleft}


\begin{flushleft}
{``}Enlightenment'', the rise of the public sphere, and colonialism; c)
\end{flushleft}


\begin{flushleft}
the relation of criticism and critique to literature and to metaphysical
\end{flushleft}


\begin{flushleft}
inquiry; d) the concepts and concerns that inform theoretical and
\end{flushleft}


\begin{flushleft}
critical activity today, i.e., the critiques of gender, religion, race
\end{flushleft}


\begin{flushleft}
and caste; e) the relation of critical theory to critical practice, e.g.
\end{flushleft}


\begin{flushleft}
application, evaluation, description, self-reflexivity, and resistance, as
\end{flushleft}


\begin{flushleft}
observed in various schools of literary theory and criticism.
\end{flushleft}





\begin{flushleft}
HUL886 American Fiction II
\end{flushleft}


\begin{flushleft}
3 Credits (3-0-0)
\end{flushleft}


\begin{flushleft}
This is a survey course covering American fiction of the post- World
\end{flushleft}


\begin{flushleft}
War--I period. Some of the major novelists of the period will be studied,
\end{flushleft}


\begin{flushleft}
including Hemingway, Scott Fitzgerald, Steinbeck, Richard Wright,
\end{flushleft}


\begin{flushleft}
Ralph Ellison, Saul Bellow, Bernard Malamud, John Barth, John Updike.
\end{flushleft}





\begin{flushleft}
HUL888 Applied Linguistics
\end{flushleft}


\begin{flushleft}
3 Credits (3-0-0)
\end{flushleft}


\begin{flushleft}
Notions of applied linguistics; psycholinguistics; socio-linguistics;
\end{flushleft}


\begin{flushleft}
language learning; language teaching; contrastive analysis; error
\end{flushleft}


\begin{flushleft}
analysis; pedagogic grammars; applied lexicology; communicative
\end{flushleft}


\begin{flushleft}
teaching; discourse analysis; stylistic and literature.
\end{flushleft}





\begin{flushleft}
HUL889 British Fiction -- A Stylistics Approach
\end{flushleft}


\begin{flushleft}
3 Credits (3-0-0)
\end{flushleft}


\begin{flushleft}
Language in prose and poetry; stylistics; deviance; prominence,
\end{flushleft}


\begin{flushleft}
foregrounding; literary relevance; stylistic variants; language and the
\end{flushleft}


\begin{flushleft}
fictional world; the rhetoric of text; discourse situation; conversation,
\end{flushleft}


\begin{flushleft}
speech and thought.
\end{flushleft}





\begin{flushleft}
HUL891 Globalization and Transnationalism
\end{flushleft}


\begin{flushleft}
3 Credits (2-1-0)
\end{flushleft}


\begin{flushleft}
Globalization \& Globalism, Nationalism \& Transnationalism, Dicopora,
\end{flushleft}


\begin{flushleft}
Glocality. Globalisation and Transnational movements of people, ideas
\end{flushleft}


\begin{flushleft}
\& technology, culture, capital and goods. Relationship between locality,
\end{flushleft}


\begin{flushleft}
national boundaries and transnationalism personal and collective
\end{flushleft}


\begin{flushleft}
identity. Transnational migration and global politics of gender and
\end{flushleft}


\begin{flushleft}
work in a global world- Dicopora. Religion and Ethnicity in a global
\end{flushleft}


\begin{flushleft}
world. The State and Democracy in a globalised world.
\end{flushleft}





229





\begin{flushleft}
\newpage
Humanities and Social Sciences
\end{flushleft}





\begin{flushleft}
HUL893 Literature and the City
\end{flushleft}


\begin{flushleft}
The course examines in some detail the nature of the challenge
\end{flushleft}


\begin{flushleft}
that traditionally preoccupied European writers - how to map the
\end{flushleft}


\begin{flushleft}
experience of the modern city, and what representational strategies
\end{flushleft}


\begin{flushleft}
were adequate for capturing the opacity, the fragmentation, and the
\end{flushleft}


\begin{flushleft}
transitory nature of urban modernity. It goes on to investigate the
\end{flushleft}


\begin{flushleft}
contemporary postcolonial city in order to understand it in relation to
\end{flushleft}


\begin{flushleft}
late capitalism, globalization, migration, and postmodern culture, and
\end{flushleft}


\begin{flushleft}
the challenges these pose to classic modernity. It begins by providing
\end{flushleft}


\begin{flushleft}
an introduction to some of the most important literature on the city
\end{flushleft}


\begin{flushleft}
and the major theoretical debates around it, offering students a set of
\end{flushleft}


\begin{flushleft}
conceptual tools with which to approach the city's incommensurable
\end{flushleft}


\begin{flushleft}
realities, its problems and its potential. It moves on to a detailed
\end{flushleft}


\begin{flushleft}
analysis of a number of literary texts, examining some of the ways
\end{flushleft}


\begin{flushleft}
in which the disjunctive realities of city-life shape new modes of
\end{flushleft}


\begin{flushleft}
experience, creative expression, and solidarity, without losing sight
\end{flushleft}


\begin{flushleft}
of the inequities of gender, culture, class, and race that persist and
\end{flushleft}


\begin{flushleft}
indeed strengthen in the current global economic system.
\end{flushleft}





\begin{flushleft}
HUP102 Research Participation
\end{flushleft}


\begin{flushleft}
1 Credits (0-0-1)
\end{flushleft}


\begin{flushleft}
This course will expose students to various experimental methodologies
\end{flushleft}


\begin{flushleft}
in sub-fields of Psychology and Psycho/Cognitive-linguistics. These will
\end{flushleft}


\begin{flushleft}
be behavioral experiments that will investigate theoretical questions
\end{flushleft}


\begin{flushleft}
(e.g., psychological questions related to perception, attention,
\end{flushleft}


\begin{flushleft}
emotion, choice behavior; psycho-linguistic questions related to
\end{flushleft}


\begin{flushleft}
sentence comprehension, sentence production, memory, attention,
\end{flushleft}


\begin{flushleft}
language-perception interaction). The course will illustrate ways
\end{flushleft}


\begin{flushleft}
in which theoretical/practical research query pertaining to human
\end{flushleft}


\begin{flushleft}
cognition is translated into a testable problem with the help of widely
\end{flushleft}


\begin{flushleft}
used behavioral methods.
\end{flushleft}





\begin{flushleft}
HUV731 Critical Reading
\end{flushleft}


\begin{flushleft}
1 Credits (1-0-0)
\end{flushleft}


\begin{flushleft}
The course will introduce students to the tools of critical analysis of a
\end{flushleft}


\begin{flushleft}
variety of verbal texts -- poetry, short stories, essays, non-fiction and
\end{flushleft}


\begin{flushleft}
academic writing. It will require students to study basic semiotics, and
\end{flushleft}


\begin{flushleft}
critical terms and study a variety of texts prescribed for the course.
\end{flushleft}





\begin{flushleft}
HUV734 Dimensions of Language
\end{flushleft}


\begin{flushleft}
1 Credits (1-0-0)
\end{flushleft}


\begin{flushleft}
The course will provide a brief overview of the important contributions
\end{flushleft}


\begin{flushleft}
to the study of language, its origins, diversity, and its metaphysical,
\end{flushleft}


\begin{flushleft}
historical and political dimensions in order to attend to the multiple
\end{flushleft}


\begin{flushleft}
levels at which literature plays with and transforms language on the
\end{flushleft}


\begin{flushleft}
one hand, and is conditioned by on the other. A range of readings
\end{flushleft}


\begin{flushleft}
will be used to focus on: the relation between language use and a
\end{flushleft}


\begin{flushleft}
particular historical and social situation, and the work of literature in
\end{flushleft}


\begin{flushleft}
defining this relation; the politics of language with respect to state,
\end{flushleft}


\begin{flushleft}
religion, nation, gender and caste; subjectivity in language; metaphor
\end{flushleft}


\begin{flushleft}
and metonymy, literary stylistics and rhetoric; agrammaticality.
\end{flushleft}





\begin{flushleft}
HUV735 Narrative Matters
\end{flushleft}


\begin{flushleft}
1 Credits (1-0-0)
\end{flushleft}


\begin{flushleft}
The course will acquaint students with the distinctions of formal and
\end{flushleft}


\begin{flushleft}
conversational, fictional and non-fictional narratives. Students will
\end{flushleft}


\begin{flushleft}
acquire the training to think about and conduct research on discourse
\end{flushleft}


\begin{flushleft}
by going beyond the story and plot and considering discourses
\end{flushleft}


\begin{flushleft}
in terms of salient narrative features such as narration, author,
\end{flushleft}


\begin{flushleft}
reception, motivation, tradition, and framing. The politics of culture,
\end{flushleft}


\begin{flushleft}
representation and the working of power can be better analysed with a
\end{flushleft}


\begin{flushleft}
mind to the role of narrative in action, communication and signification.
\end{flushleft}





\begin{flushleft}
HUV747 Data-driven Analysis and Tools for Linguistic
\end{flushleft}


\begin{flushleft}
research
\end{flushleft}


\begin{flushleft}
2 Credits (2-0-0)
\end{flushleft}


\begin{flushleft}
Pre-requisites: HUL 242 for UG
\end{flushleft}


\begin{flushleft}
The course will cover the following topics: (1) Quick introduction to
\end{flushleft}


\begin{flushleft}
Python (2) It will give a broad overview of how one can use natural
\end{flushleft}





\begin{flushleft}
language text for making linguistic generalization and discovering
\end{flushleft}


\begin{flushleft}
linguistics patterns, (3) We will also look at how the data can be
\end{flushleft}


\begin{flushleft}
used to make automatic tools such as taggers and parsers. In the
\end{flushleft}


\begin{flushleft}
process we will learn to use the following resources/tools: (a) NLTK
\end{flushleft}


\begin{flushleft}
(b) MaltParser (c) WordNet.
\end{flushleft}





\begin{flushleft}
HUV748 Data Analysis for Psycholinguistics using R
\end{flushleft}


\begin{flushleft}
2 Credits (2-0-0)
\end{flushleft}


\begin{flushleft}
Pre-requisites: HUL 242 and HUL 381 for UG
\end{flushleft}


\begin{flushleft}
The course will comprise of 4 broad themes. In the 1st part of the
\end{flushleft}


\begin{flushleft}
course we will introduce the basics of R. R syntax and its libraries will
\end{flushleft}


\begin{flushleft}
be extensively used for other parts of the course. In the 2nd part we
\end{flushleft}


\begin{flushleft}
will introduce basics of statistics that are needed for understanding
\end{flushleft}


\begin{flushleft}
ideas of frequentist-based hypothesis testing methods. We will then
\end{flushleft}


\begin{flushleft}
move on to linear regression which will form the background using
\end{flushleft}


\begin{flushleft}
which we introduce linear-mixed models in the final section of the
\end{flushleft}


\begin{flushleft}
course. The course will also provide assignments and projects where
\end{flushleft}


\begin{flushleft}
the students can practice the course content and apply the learnt
\end{flushleft}


\begin{flushleft}
concepts to real experimental data.
\end{flushleft}





\begin{flushleft}
HUV773 Tools for Sociological Research
\end{flushleft}


\begin{flushleft}
1.5 Credits (1.5-0-0)
\end{flushleft}


\begin{flushleft}
In this course, the students will be introduced to mixed methods
\end{flushleft}


\begin{flushleft}
research (quantitative and qualitative). They will be familiarized with
\end{flushleft}


\begin{flushleft}
specific modes of observation such as surveys, focus group discussions,
\end{flushleft}


\begin{flushleft}
interviews, and participatory rural appraisal, followed by designing
\end{flushleft}


\begin{flushleft}
specific tools of data collection such as questionnaires and interview
\end{flushleft}


\begin{flushleft}
protocols for different modes of observation. Special emphasis will
\end{flushleft}


\begin{flushleft}
be given to household surveys and data from large surveys such as
\end{flushleft}


\begin{flushleft}
census, NFHS, NSS, migration and urban surveys. This will be followed
\end{flushleft}


\begin{flushleft}
by elementary data analysis techniques and inferential statistics,
\end{flushleft}


\begin{flushleft}
converting qualitative data into quantitative data and the use of
\end{flushleft}


\begin{flushleft}
qualitative data analysis software.
\end{flushleft}





\begin{flushleft}
HUV774 Methods in Historical Sociology
\end{flushleft}


\begin{flushleft}
1.5 Credits (1.5-0-0)
\end{flushleft}


\begin{flushleft}
As sociologists increasingly turn to the past for an understanding of
\end{flushleft}


\begin{flushleft}
the present, the discipline has incorporated methods from historical
\end{flushleft}


\begin{flushleft}
research. These include debates on what constitutes an archive,
\end{flushleft}


\begin{flushleft}
the production of the past as an exercise of power, and how to read
\end{flushleft}


\begin{flushleft}
historical sources in an ethnographic vein. The significance of the
\end{flushleft}


\begin{flushleft}
{`}fragment' or documentary evidence for ethnographic research will be
\end{flushleft}


\begin{flushleft}
considered. This module will introduce students to some of these larger
\end{flushleft}


\begin{flushleft}
methodological and theoretical debates from history and sociology.
\end{flushleft}





\begin{flushleft}
HUV781 Introduction to Research Methods
\end{flushleft}


\begin{flushleft}
1.5 Credits (1.5-0-0)
\end{flushleft}


\begin{flushleft}
This course will begin with introducing students to different paradigms
\end{flushleft}


\begin{flushleft}
of inquiry and research with implications for methodology. It will then
\end{flushleft}


\begin{flushleft}
provide an overview of how to formulate research questions and
\end{flushleft}


\begin{flushleft}
hypotheses, identify unit of analysis, conceptualize and operationalize
\end{flushleft}


\begin{flushleft}
variables of interest. The students will be familiarized with random and
\end{flushleft}


\begin{flushleft}
non-random methods of sampling. The discourse on research ethics
\end{flushleft}


\begin{flushleft}
will be an integral part of most discussions in this course.
\end{flushleft}





\begin{flushleft}
HUV886 Special Module in Cognitive Psychology
\end{flushleft}


\begin{flushleft}
2 Credits (1-0-2)
\end{flushleft}


\begin{flushleft}
The course will focus on current relevant and emerging issues, and
\end{flushleft}


\begin{flushleft}
experiments in the field of cognitive psychology.
\end{flushleft}





\begin{flushleft}
HUV887 Special Module on Econometric Tools
\end{flushleft}


\begin{flushleft}
1 Credits (1-0-0)
\end{flushleft}


\begin{flushleft}
Pre-requisites: HUL or SML 700/800 category courses
\end{flushleft}


\begin{flushleft}
Estimation and inference in two variable model; OLS assumption;
\end{flushleft}


\begin{flushleft}
Extension of the two variable model; OLS assumption : autocorrelation,
\end{flushleft}


\begin{flushleft}
multicollinearity, and heteroskedasticity, models with limited dependent
\end{flushleft}


\begin{flushleft}
variables : LPM, logit, and probit; Panel data modelling : fixed effect
\end{flushleft}


\begin{flushleft}
and random effect models; Time series analysis: introduction to nonstationarity, AR and MA modelling.
\end{flushleft}





230





\begin{flushleft}
\newpage
Department of Management Studies
\end{flushleft}


\begin{flushleft}
MSL301 Organization \& People Management
\end{flushleft}


\begin{flushleft}
3 Credits (3-0-0)
\end{flushleft}





\begin{flushleft}
Business Portfolio and Technology Portfolio, Technology- Market matrix.
\end{flushleft}


\begin{flushleft}
Innovation and entry strategy, Flexibility in Technology strategy.
\end{flushleft}





\begin{flushleft}
Lectures on multidisciplinary perspective on organizations,Organizational
\end{flushleft}


\begin{flushleft}
structure \& Design, Organizational stakeholder Ethics, Organizational
\end{flushleft}


\begin{flushleft}
Culture, Organizational Environment, Strategy and Structure,
\end{flushleft}


\begin{flushleft}
Technology and organizational structures, Lifecycle of an organization
\end{flushleft}


\begin{flushleft}
will be supported with case studies \& exercises.
\end{flushleft}





\begin{flushleft}
Module III: Business/technology alliances and networks. Technology
\end{flushleft}


\begin{flushleft}
forecasting and assessment. Technology strategy at business level.
\end{flushleft}


\begin{flushleft}
Strategic Technology Planning, Investment in Technology, Technology
\end{flushleft}


\begin{flushleft}
Strategy and functional strategy. Implementation and Control of
\end{flushleft}


\begin{flushleft}
technology strategy, Managing Corporate culture, structure, and
\end{flushleft}


\begin{flushleft}
interdepartmental linkages.
\end{flushleft}





\begin{flushleft}
MSL302 Managerial Accounting \& Financial Management
\end{flushleft}


\begin{flushleft}
3 Credits (3-0-0)
\end{flushleft}


\begin{flushleft}
On completion of this course the student will be able to: Understand
\end{flushleft}


\begin{flushleft}
accounting for managerial decisions. Assess financial health of a
\end{flushleft}


\begin{flushleft}
corporate firm. Design profit planning. Understand cost concepts and
\end{flushleft}


\begin{flushleft}
financial decision making.
\end{flushleft}





\begin{flushleft}
MSL303 Marketing Management
\end{flushleft}


\begin{flushleft}
3 Credits (3-0-0)
\end{flushleft}


\begin{flushleft}
Marketing concept, Environment of Marketing, Marketing Strategy,
\end{flushleft}


\begin{flushleft}
Marketing Ethics, Marketing Planning, Concept of Product life cycle,
\end{flushleft}


\begin{flushleft}
Pricing, Advertising and Promotion Strategies, Concept of Unique
\end{flushleft}


\begin{flushleft}
selling proposition, Product and Brand Management, Marketing
\end{flushleft}


\begin{flushleft}
Research Methodologies, Case study discussions.
\end{flushleft}





\begin{flushleft}
MSL304 Managing Operations
\end{flushleft}


\begin{flushleft}
3 Credits (3-0-0)
\end{flushleft}


\begin{flushleft}
The objective of the course is to provide the students about the
\end{flushleft}


\begin{flushleft}
application of Industrial management in various functional areas of
\end{flushleft}


\begin{flushleft}
business especially industrial operations such as linear programing,
\end{flushleft}


\begin{flushleft}
assignment and transportation problem, layout/location design,
\end{flushleft}


\begin{flushleft}
quality, materials management, Preventive maintenance, project
\end{flushleft}


\begin{flushleft}
management, supply chain management, scheduling/sequencing,
\end{flushleft}


\begin{flushleft}
ergonomics, operations strategy. The entire course is a case based
\end{flushleft}


\begin{flushleft}
where the participants will be given a case. Participants will be asked to
\end{flushleft}


\begin{flushleft}
tackle the case problem without using linear programming techniques.
\end{flushleft}





\begin{flushleft}
MSL700 Fundamentals of Management of Technology
\end{flushleft}


\begin{flushleft}
3 Credits (3-0-0)
\end{flushleft}


\begin{flushleft}
Module I: Understanding technology: definition, Key concepts, role,
\end{flushleft}


\begin{flushleft}
importance, need. History of technological developments, Today's
\end{flushleft}


\begin{flushleft}
challenges. Issues of concern in Management of New Technology.
\end{flushleft}


\begin{flushleft}
Technology-Management integration, Life cycle approach to
\end{flushleft}


\begin{flushleft}
technology management. Technology innovation process. Managing
\end{flushleft}


\begin{flushleft}
and fostering the Innovation.
\end{flushleft}


\begin{flushleft}
Module II: Technology forecasting and assessment. Technology
\end{flushleft}


\begin{flushleft}
flow and diffusion. Evaluating technology, technology planning and
\end{flushleft}


\begin{flushleft}
strategy, Strategic potential of new technology. Factors promoting
\end{flushleft}


\begin{flushleft}
technology acquisition. Flexibility in Technology Management.
\end{flushleft}


\begin{flushleft}
Technology transfer and absorption, Modes of global technology
\end{flushleft}


\begin{flushleft}
transfer. Technological Entrepreneurship.
\end{flushleft}


\begin{flushleft}
Module III: Technology implementation. Integrating people and
\end{flushleft}


\begin{flushleft}
technology, human factors in technology operations. Organisation
\end{flushleft}


\begin{flushleft}
structure and technology. Investing for technological maintenance
\end{flushleft}


\begin{flushleft}
and growth. Concern of phasing out and upgradation. Market
\end{flushleft}


\begin{flushleft}
factors in technology operations, Science and Technology
\end{flushleft}


\begin{flushleft}
Policy, Technology support systems. Information networking for
\end{flushleft}


\begin{flushleft}
technological updatedness.
\end{flushleft}





\begin{flushleft}
MSL701 Strategic Technology Management
\end{flushleft}


\begin{flushleft}
3 Credits (3-0-0)
\end{flushleft}


\begin{flushleft}
Module I: Emerging technology-strategy relationship in the large
\end{flushleft}


\begin{flushleft}
corporation from the perspective of individual firm, and entire industry.
\end{flushleft}


\begin{flushleft}
Global technology comparison, technological change, sources of
\end{flushleft}


\begin{flushleft}
technology, Technology Information. Criticality of technology for
\end{flushleft}


\begin{flushleft}
growth, core competencies, R\&D productivity, Resource Leverage.
\end{flushleft}


\begin{flushleft}
World Class Organisation.
\end{flushleft}


\begin{flushleft}
Module II: Corporate technology strategy, Generic competitive technology
\end{flushleft}


\begin{flushleft}
strategies. Corporate R\&D, Strategic technology management process,
\end{flushleft}


\begin{flushleft}
relationship between technology strategy and corporate strategy.
\end{flushleft}


\begin{flushleft}
Strategic shifts and resource commitments, technology vision and
\end{flushleft}


\begin{flushleft}
goals, technology leadership. SWOT analysis for technology, Matching
\end{flushleft}





\begin{flushleft}
MSL702 Management of Innovation and R\&D
\end{flushleft}


\begin{flushleft}
3 Credits (3-0-0)
\end{flushleft}


\begin{flushleft}
Module I: Technological innovation systems and processes.
\end{flushleft}


\begin{flushleft}
Understanding the process of technological innovation and the
\end{flushleft}


\begin{flushleft}
factors affecting successful innovation. Management problems from
\end{flushleft}


\begin{flushleft}
the product/service concept-stage to end-product/service marketing.
\end{flushleft}


\begin{flushleft}
Creativity and Innovation- Creativity process, Individual and group
\end{flushleft}


\begin{flushleft}
creativity, Critical functions in the innovation process, Evolving
\end{flushleft}


\begin{flushleft}
innovative culture, teams for innovation.
\end{flushleft}


\begin{flushleft}
Module II: Product and technology life cycle, Management of R\&D
\end{flushleft}


\begin{flushleft}
planning, organising, staffing, scheduling, Controlling, budgeting,
\end{flushleft}


\begin{flushleft}
Selection of R\&D projects. Methodologies for evaluating the
\end{flushleft}


\begin{flushleft}
effectiveness of R\&D, Research Productivity. Protection of Intellectual
\end{flushleft}


\begin{flushleft}
Property Rights. Evolving flexible organisation.
\end{flushleft}


\begin{flushleft}
Module III: Issues relating to managing scientists and technologists
\end{flushleft}


\begin{flushleft}
as individual, in teams, and in large organisations. Human
\end{flushleft}


\begin{flushleft}
Resource Management in R\&D and Innovation, training, motivation,
\end{flushleft}


\begin{flushleft}
communication, group dynamics. Information management for
\end{flushleft}


\begin{flushleft}
innovation and R\&D- strategies, sources, channels, and flows.
\end{flushleft}


\begin{flushleft}
Standardisation and Quality management.
\end{flushleft}





\begin{flushleft}
MSL703 Management of Technology Transfer and
\end{flushleft}


\begin{flushleft}
Absorption
\end{flushleft}


\begin{flushleft}
3 Credits (3-0-0)
\end{flushleft}


\begin{flushleft}
Module I: Transfer of technology from R\&D to field and at international
\end{flushleft}


\begin{flushleft}
level. Commercialization of new technology and new venture
\end{flushleft}


\begin{flushleft}
management, prototyping, test marketing, pilot plant, project viability,
\end{flushleft}


\begin{flushleft}
Technology push and market full. Quality management, customer
\end{flushleft}


\begin{flushleft}
education and awareness. Assessment, justification and financing
\end{flushleft}


\begin{flushleft}
of new technology, source of funds, venture capital financing. New
\end{flushleft}


\begin{flushleft}
venture products and services.
\end{flushleft}


\begin{flushleft}
Module II: Global transfer of technology, Technology transfer
\end{flushleft}


\begin{flushleft}
models: Active, passive. Multi channel approach: from hardware
\end{flushleft}


\begin{flushleft}
technical services acquisitions to strategic partnering and networking
\end{flushleft}


\begin{flushleft}
arrangements. Sourcing technology, technology negotiation, licensing
\end{flushleft}


\begin{flushleft}
agreement. Fee for technology transfer, royalty, equity participation.
\end{flushleft}


\begin{flushleft}
Modes: technological collaboration, joint venture, alliance, acquisition.
\end{flushleft}


\begin{flushleft}
International S\&T cooperation: institutional framework, multilateral/
\end{flushleft}


\begin{flushleft}
bilateral cooperation, pre-emptive R\&D cooperation.
\end{flushleft}


\begin{flushleft}
Module III: Absorbent Strategy: Japanese technology absorption,
\end{flushleft}


\begin{flushleft}
Technology Absorption: product and process technologies, Reverse
\end{flushleft}


\begin{flushleft}
engineering. Appropriate technology. Vendor development. Adaptation
\end{flushleft}


\begin{flushleft}
and assimilation of technology.
\end{flushleft}





\begin{flushleft}
MSL704 Science \& Technology Policy Systems
\end{flushleft}


\begin{flushleft}
3 Credits (3-0-0)
\end{flushleft}


\begin{flushleft}
Pre--requisites: MSL301 \& MSL302
\end{flushleft}


\begin{flushleft}
Module I: Role of S\&T in economic development, Modern analysis
\end{flushleft}


\begin{flushleft}
of growth and structural change, international economic relations,
\end{flushleft}


\begin{flushleft}
liberalisation, globalisation/ regionalisation, industrial/technological
\end{flushleft}


\begin{flushleft}
partnerships, S\&T in Indian Economic Policy. Government policy and its
\end{flushleft}


\begin{flushleft}
impacts on technology development. Living with the new technology,
\end{flushleft}


\begin{flushleft}
social issues. International trends, Technology policy in USA, Japan,
\end{flushleft}


\begin{flushleft}
European Commission, and other select countries.
\end{flushleft}


\begin{flushleft}
Module II: National technology Policies, Regulatory Policies:
\end{flushleft}


\begin{flushleft}
Industries Development and Regulation Act, MRTP, FERA, Intellectual
\end{flushleft}


\begin{flushleft}
Property Rights, Patents act, Environment Protection Act, R\&D Cess
\end{flushleft}


\begin{flushleft}
Rules, Import Export Policy; Development Policies: Industrial Policy
\end{flushleft}





231





\begin{flushleft}
\newpage
Management Studies
\end{flushleft}





\begin{flushleft}
Resolution, Scientific Policy Resolution, Technology Policy Statement,
\end{flushleft}


\begin{flushleft}
New Technology Policy, Policy on Foreign Investments and Technology
\end{flushleft}


\begin{flushleft}
Imports. Role of UN and other International Agencies.
\end{flushleft}


\begin{flushleft}
Module III: Support Systems: Technology infrastructure, technology
\end{flushleft}


\begin{flushleft}
parks, Technology development and utilization schemes by
\end{flushleft}


\begin{flushleft}
government and Financial Institutions, Venture capital financing,
\end{flushleft}


\begin{flushleft}
TIFAC, Technology mission, Standards, Support to Small scale sectors.
\end{flushleft}


\begin{flushleft}
Research laboratories, and institutions. S\&T in five year plans, Fiscal
\end{flushleft}


\begin{flushleft}
incentives. Organization set up for Science and Technology. R\&D in
\end{flushleft}


\begin{flushleft}
corporate sector.
\end{flushleft}





\begin{flushleft}
MSL705 HRM Systems
\end{flushleft}


\begin{flushleft}
1.5 Credits (1.5-0-0)
\end{flushleft}


\begin{flushleft}
This course focuses on various functions of human resource
\end{flushleft}


\begin{flushleft}
management. The course begins with the context and evolution of
\end{flushleft}


\begin{flushleft}
HR, followed by functions of HR via cases and various exercises.
\end{flushleft}


\begin{flushleft}
Manpower planning, job design, recruitment \& selection, training \&
\end{flushleft}


\begin{flushleft}
development, performance appraisal \& management, compensation
\end{flushleft}


\begin{flushleft}
\& reward management and career management, legal issues in HRM
\end{flushleft}


\begin{flushleft}
are the topics covered.
\end{flushleft}





\begin{flushleft}
MSL706 Business Laws
\end{flushleft}


\begin{flushleft}
3 Credits (3-0-0)
\end{flushleft}


\begin{flushleft}
Module I: Nature of Business law, Source of Business law and their
\end{flushleft}


\begin{flushleft}
classification. Mercantile law, statue I Case law, Customs and Usage.
\end{flushleft}


\begin{flushleft}
Agreement and their legal obligations. Essential elements of a valid
\end{flushleft}


\begin{flushleft}
contract, types of contact, Void and voidable contract. Unenforceable
\end{flushleft}


\begin{flushleft}
and illegal agreements. Offer and acceptance over the telephone.
\end{flushleft}


\begin{flushleft}
Law of Arbitration --Definition of Arbitration, Effect of an arbitration.
\end{flushleft}


\begin{flushleft}
Arbitration without Intervention of Court. Powers and duties of
\end{flushleft}


\begin{flushleft}
Arbitrators.
\end{flushleft}


\begin{flushleft}
Module II: Sale of Goods Act. Definition and essentials of a contract
\end{flushleft}


\begin{flushleft}
of sale, Distinction between sale and agreement to sell, sale and
\end{flushleft}


\begin{flushleft}
hire purchase, sale distinguished from contract for work and labour.
\end{flushleft}


\begin{flushleft}
Kinds of goods, perishable goods. Document to the title of goods.
\end{flushleft}


\begin{flushleft}
Rules regarding transfer of property, Transfer of Title on sale. Rules
\end{flushleft}


\begin{flushleft}
regarding delivery of goods. Buyers rights against seller, and unpaid
\end{flushleft}


\begin{flushleft}
seller's right. Consumer protection act. Consumers rights, consumer's
\end{flushleft}


\begin{flushleft}
disputes redressel agencies, consumer protection council.
\end{flushleft}


\begin{flushleft}
Module III: Negotiable Instrumented act. Definition and characteristic
\end{flushleft}


\begin{flushleft}
of Negotiable instrument. Liabilities of Parties to Negotiable
\end{flushleft}


\begin{flushleft}
Instruments. Brief exposure to Company law including incorporation of
\end{flushleft}


\begin{flushleft}
a company - objects, registration, article of association, raising capital
\end{flushleft}


\begin{flushleft}
from public, company management and reconstruction, amalgamation
\end{flushleft}


\begin{flushleft}
and winding up.
\end{flushleft}





\begin{flushleft}
MSL707 Management Accounting
\end{flushleft}


\begin{flushleft}
3 Credits (3-0-0)
\end{flushleft}


\begin{flushleft}
On completion of this course the student will be able to: Understand
\end{flushleft}


\begin{flushleft}
accounting principles governing preparation of financial statements.
\end{flushleft}


\begin{flushleft}
Assess financial health of a corporate firm. Design profit planning.
\end{flushleft}


\begin{flushleft}
Understand cost control systems. Understand techniques of pricing,
\end{flushleft}


\begin{flushleft}
product and capital budgeting decisions.
\end{flushleft}





\begin{flushleft}
MSL708 Financial Management
\end{flushleft}


\begin{flushleft}
3 Credits (3-0-0)
\end{flushleft}


\begin{flushleft}
The course is comprehensive and is designed to equip the students
\end{flushleft}


\begin{flushleft}
with tools and techniques to enable them to make sound financial
\end{flushleft}


\begin{flushleft}
decisions, among others, related to capital budgeting, working capital,
\end{flushleft}


\begin{flushleft}
capital structure and dividend policy.
\end{flushleft}





\begin{flushleft}
MSL709 Business Research Methods
\end{flushleft}


\begin{flushleft}
1.5 Credits (1.5-0-0)
\end{flushleft}


\begin{flushleft}
Pre--requisites: MSL301 \& MSL302
\end{flushleft}


\begin{flushleft}
Introduction to Business Research Methods; Theoretical approaches;
\end{flushleft}


\begin{flushleft}
Problem definition; Research Design; Questionnaires \& Scales;
\end{flushleft}


\begin{flushleft}
Sampling - Probability, size and challenges; Survey \& Observation,
\end{flushleft}


\begin{flushleft}
Experiments; Qualitative Research, Secondary Data; Data Preparation
\end{flushleft}


\begin{flushleft}
\& Analysis, Report Writing.
\end{flushleft}





\begin{flushleft}
MSL710 Creative Problem Solving
\end{flushleft}


\begin{flushleft}
3 Credits (3-0-0)
\end{flushleft}


\begin{flushleft}
Pre--requisites: MSL301 \& MSL302
\end{flushleft}


\begin{flushleft}
Module I: Structure of managerial problems. Open and close
\end{flushleft}


\begin{flushleft}
ended problems, convergent and divergent thinking. The creativity
\end{flushleft}


\begin{flushleft}
process, Individual and group creativity, Idea generation methods:
\end{flushleft}


\begin{flushleft}
Brain storming, Nominal Group Technique, Idea Engineering, Check
\end{flushleft}


\begin{flushleft}
list, Attribute listing, Morphological analysis, Synectics, Mental
\end{flushleft}


\begin{flushleft}
Imaging, Critical Questioning. Total System Intervention, Flexible
\end{flushleft}


\begin{flushleft}
Systems Methodology.
\end{flushleft}


\begin{flushleft}
Module II: Idea Structuring: Graphic tools, Programme Planning
\end{flushleft}


\begin{flushleft}
Linkages, Interpretive Structural Modelling, Relationship Analysis,
\end{flushleft}


\begin{flushleft}
Flexible Systems Management, SAP-LAP Analysis, Flexibility Influence
\end{flushleft}


\begin{flushleft}
Diagrams, Collaboration Diagrams. Scenario Building: Harva method,
\end{flushleft}


\begin{flushleft}
Structural Analysis, Options Field/Profile Methodology.
\end{flushleft}


\begin{flushleft}
Module III: Viable Systems Modelling. Fuzzy sets in multicriteria
\end{flushleft}


\begin{flushleft}
decision making, Analytic Hierarchy Process, Intelligent
\end{flushleft}


\begin{flushleft}
Management Systems, Creativity applications in TQM and Business
\end{flushleft}


\begin{flushleft}
Process Reengineering.
\end{flushleft}





\begin{flushleft}
MSL711 Strategic Management
\end{flushleft}


\begin{flushleft}
3 Credits (3-0-0)
\end{flushleft}


\begin{flushleft}
Pre--requisites: MSL301 \& MSL302
\end{flushleft}


\begin{flushleft}
Understanding new perspectives on strategic management , Content
\end{flushleft}


\begin{flushleft}
and process of strategic management, Formulation and implementation
\end{flushleft}


\begin{flushleft}
of strategies, Developing cross-functional trade-off decision making
\end{flushleft}


\begin{flushleft}
skills, and Help appreciate new themes in strategic management.
\end{flushleft}


\begin{flushleft}
This course will require reading books, articles, case studies and
\end{flushleft}


\begin{flushleft}
literature from the field of Strategic Management. The sessions would
\end{flushleft}


\begin{flushleft}
be interactive where attempt will be made to understand the theories
\end{flushleft}


\begin{flushleft}
and concepts through discussion of the readings and their application
\end{flushleft}


\begin{flushleft}
in cases. Student will be required to prepare and effectively participate
\end{flushleft}


\begin{flushleft}
in class and make impromptu or scheduled presentations of issues
\end{flushleft}


\begin{flushleft}
and learnings. Besides the readings, groups of students will have to
\end{flushleft}


\begin{flushleft}
work on a comprehensive research project to investigate and validate
\end{flushleft}


\begin{flushleft}
some of the key learnings.
\end{flushleft}





\begin{flushleft}
MSL712 Ethics \& Values Based Leadership
\end{flushleft}


\begin{flushleft}
1.5 Credits (1.5-0-0)
\end{flushleft}


\begin{flushleft}
Pre--requisites: MSL301 \& MSL302
\end{flushleft}


\begin{flushleft}
Ethics \& Business, Ethical principles in business, Business and Its
\end{flushleft}


\begin{flushleft}
External Exchanges: Ecology \& Consumers, Business \& Its Internal
\end{flushleft}


\begin{flushleft}
Constituencies.
\end{flushleft}





\begin{flushleft}
MSL713 Information Systems Management
\end{flushleft}


\begin{flushleft}
3 Credits (3-0-0)
\end{flushleft}


\begin{flushleft}
Pre--requisites: MSL301 \& MSL302
\end{flushleft}


\begin{flushleft}
This course may expose the participants to the following topics:
\end{flushleft}


\begin{flushleft}
Information Systems and its impact in Organization and People,
\end{flushleft}


\begin{flushleft}
Information Technologies: concepts, types and usage, Information
\end{flushleft}


\begin{flushleft}
Systems, Organizations and Strategy, Economics of Information
\end{flushleft}


\begin{flushleft}
Systems, Foundations of E-Business, Foundations of Data management,
\end{flushleft}


\begin{flushleft}
Foundations of Business Analytics, Networks and Collaboration as
\end{flushleft}


\begin{flushleft}
Business Solutions, Information Security \& Risk Management, Building
\end{flushleft}


\begin{flushleft}
and Managing Systems, Enterprise Systems, etc. Hands on training
\end{flushleft}


\begin{flushleft}
would also be provided, using specific tools.
\end{flushleft}





\begin{flushleft}
MSL714 Organizational Dynamics and Environment
\end{flushleft}


\begin{flushleft}
3 Credits (3-0-0)
\end{flushleft}


\begin{flushleft}
Pre--requisites: MSL301 \& MSL302
\end{flushleft}


\begin{flushleft}
Module I : Organisational systems vix. a vis., the environment. The
\end{flushleft}


\begin{flushleft}
dialectics of agency and structure- extent of environmental and
\end{flushleft}


\begin{flushleft}
organizational control. External control of organization. Organizations
\end{flushleft}


\begin{flushleft}
and the new institutionalism. Systems for managing chaos and conflict.
\end{flushleft}


\begin{flushleft}
Module II : Constituent systems for organizational functioningplanning, learning, organising, communication and control systems.
\end{flushleft}





232





\begin{flushleft}
\newpage
Management Studies
\end{flushleft}





\begin{flushleft}
Organizational systems and mechanisms related to technology. Systems
\end{flushleft}


\begin{flushleft}
for managing strategy, and structure related to new technology.
\end{flushleft}


\begin{flushleft}
Module III : Systems for managing continuous and radical change
\end{flushleft}


\begin{flushleft}
for organizational renewal and transformation. Adaptiveness and
\end{flushleft}


\begin{flushleft}
flexibility in organisational systems. Systems for managing collective
\end{flushleft}


\begin{flushleft}
action within the organization. Feminism and organizational systems
\end{flushleft}


\begin{flushleft}
for managing gender diversity.
\end{flushleft}





\begin{flushleft}
Tendency, Measures of Dispersion. Introduction to probability theory.
\end{flushleft}


\begin{flushleft}
Probability Theory: Preliminary concepts in Probability, Basic Theorems
\end{flushleft}


\begin{flushleft}
and rules for dependent/independent events, Random Variable,
\end{flushleft}


\begin{flushleft}
Probability distributions. Sampling Techniques, Sampling distributions.
\end{flushleft}


\begin{flushleft}
Hypothesis testing: Z-test, t-test, ANOVA, Chi-square tests, Correlation
\end{flushleft}


\begin{flushleft}
and regression analysis. Business Forecasting. SPSS and its use for
\end{flushleft}


\begin{flushleft}
statistical modeling.
\end{flushleft}





\begin{flushleft}
MSL715 Quality and Environment Management Systems
\end{flushleft}


\begin{flushleft}
3 Credits (3-0-0)
\end{flushleft}


\begin{flushleft}
Pre--requisites: MSL301 \& MSL302
\end{flushleft}





\begin{flushleft}
MSL720 Macroeconomic Environment of Business
\end{flushleft}


\begin{flushleft}
3 Credits (3-0-0)
\end{flushleft}


\begin{flushleft}
Pre--requisites: MSL301 \& MSL302
\end{flushleft}





\begin{flushleft}
Module I : Concept of Total Quality, Quality Management Systems as a
\end{flushleft}


\begin{flushleft}
means of achieving total quality. Linkage of Quality and Environment
\end{flushleft}


\begin{flushleft}
Management System. Strategic concern for Environment. Need and
\end{flushleft}


\begin{flushleft}
relevance of documentation and standardization of Management
\end{flushleft}


\begin{flushleft}
Systems. Various tools of documenting and recording the Management
\end{flushleft}


\begin{flushleft}
Systems, Various standards for Management Systems. Flexibility and
\end{flushleft}


\begin{flushleft}
change in Management Systems and documented procedures.
\end{flushleft}


\begin{flushleft}
Module II : Quality Management Systems, IS0 9000, Quality Policy,
\end{flushleft}


\begin{flushleft}
Data, Records and Traceability. Documenting the Quality System:
\end{flushleft}


\begin{flushleft}
Quality Manual, Quality Audit, Design and Change Control, ISO 9000
\end{flushleft}


\begin{flushleft}
Registration. Six Sigma. Awards and appreciation, DMAIC approach.
\end{flushleft}





\begin{flushleft}
Introduction to macroeconomic environment of business,
\end{flushleft}


\begin{flushleft}
Macroeconomic policies and Business Cycles, Economic Growth
\end{flushleft}


\begin{flushleft}
vs. Economic Development, Measurement of macroeconomic
\end{flushleft}


\begin{flushleft}
performance, Classical Macroeconomic Theory, Keynesian Model
\end{flushleft}


\begin{flushleft}
Income determination, Great Depression of 1930s; South East Asian
\end{flushleft}


\begin{flushleft}
Crisis of 90s, Sub-prime Crisis of 2007 and Euro crisis and Fiscal
\end{flushleft}


\begin{flushleft}
Sustainability; and their impact on the Business. Economic Reforms
\end{flushleft}


\begin{flushleft}
in India, Growth pattern of Indian economy, Fiscal Policy and its
\end{flushleft}


\begin{flushleft}
managerial implications to the industry, Government Budget, Monetary
\end{flushleft}


\begin{flushleft}
policy analysis and its implications to industry. Industrial Policy of India,
\end{flushleft}


\begin{flushleft}
Competition Policy of India, Balance of Payments, WTO and India.
\end{flushleft}





\begin{flushleft}
Module III : Need for proper Environment Management Systems and
\end{flushleft}


\begin{flushleft}
their economic implications. Environment Management Systems,
\end{flushleft}


\begin{flushleft}
Green Products and Strategies, Environment Assessment: Environment
\end{flushleft}


\begin{flushleft}
Protection Act, ISO 14000, Case Studies.
\end{flushleft}





\begin{flushleft}
MSL721 Econometrics
\end{flushleft}


\begin{flushleft}
3 Credits (3-0-0)
\end{flushleft}


\begin{flushleft}
Pre--requisites: MSL301 \& MSL302
\end{flushleft}


\begin{flushleft}
Introduction to Econometrics, Simple linear regression model. Multiple
\end{flushleft}


\begin{flushleft}
linear regression model, Discrete Choice, Panel Data,Time Series,
\end{flushleft}


\begin{flushleft}
Stationarity, VAR, Co-integration and Error correction models.
\end{flushleft}





\begin{flushleft}
MSL716 Fundamentals of Management Systems
\end{flushleft}


\begin{flushleft}
3 Credits (3-0-0)
\end{flushleft}


\begin{flushleft}
Pre--requisites: MSL301 \& MSL302
\end{flushleft}


\begin{flushleft}
Module I : Basics and Variants. The concept of a system, Systems
\end{flushleft}


\begin{flushleft}
Approach to management. Emerging paradigm, customer centred
\end{flushleft}


\begin{flushleft}
management systems, Flexible Management Systems. Management
\end{flushleft}


\begin{flushleft}
of Paradoxes. Management Systems in various countries: Western
\end{flushleft}


\begin{flushleft}
Management Systems, Japanese Management Systems, Chinese
\end{flushleft}


\begin{flushleft}
Management System,Indian Management Systems. Organisational
\end{flushleft}


\begin{flushleft}
Culture and Value System.
\end{flushleft}


\begin{flushleft}
Module II : Management Systems in Operation: Strategic Planning
\end{flushleft}


\begin{flushleft}
Systems, Management Control Systems, Financial Information
\end{flushleft}


\begin{flushleft}
Systems, Marketing Management Systems, Logistics and Distribution
\end{flushleft}


\begin{flushleft}
Systems, Systems for Human Resources Planning and Performance
\end{flushleft}


\begin{flushleft}
Management. System Dynamics Modelling.
\end{flushleft}


\begin{flushleft}
Module III : Methodologies for Development and Improvement.
\end{flushleft}


\begin{flushleft}
Methodology for developing Management System. Optimization
\end{flushleft}


\begin{flushleft}
and Learning Systems methodologies, Microworld, Continuous
\end{flushleft}


\begin{flushleft}
Improvement and Reengineering of Management Systems. Organizing
\end{flushleft}


\begin{flushleft}
to improve systems.
\end{flushleft}





\begin{flushleft}
MSL717 Business Systems Analysis \& Design
\end{flushleft}


\begin{flushleft}
3 Credits (3-0-0)
\end{flushleft}


\begin{flushleft}
Pre--requisites: MSL301 \& MSL302
\end{flushleft}


\begin{flushleft}
This course will have the following topics: System Analysis
\end{flushleft}


\begin{flushleft}
Fundamentals: Introducing SA\&D for Systems Professionals, Analyzing
\end{flushleft}


\begin{flushleft}
the Business Case and Managing Systems Projects, Overview to Data
\end{flushleft}


\begin{flushleft}
Structure in Systems Modeling, Data Flow Diagrams and Modelling
\end{flushleft}


\begin{flushleft}
DFDs, Requirements Modelling and Systems Specification, User Driven
\end{flushleft}


\begin{flushleft}
Business Analysis, Role of the consultant, Object Oriented Modelling:
\end{flushleft}


\begin{flushleft}
Object Relationships, Hierarchies, Use Case Approaches to identify
\end{flushleft}


\begin{flushleft}
and model classes, Process Driven Approaches: Gane, Sarson and
\end{flushleft}


\begin{flushleft}
Yourdon techniques, Data Driven Approaches: Entity Relationship
\end{flushleft}


\begin{flushleft}
Diagrams, Designing the User Interface and Output, Verification \&
\end{flushleft}


\begin{flushleft}
Validation of new systems.
\end{flushleft}





\begin{flushleft}
MSL719 Statistics for Management
\end{flushleft}


\begin{flushleft}
3 Credits (3-0-0)
\end{flushleft}


\begin{flushleft}
Pre--requisites: MSL301 \& MSL302
\end{flushleft}


\begin{flushleft}
Nature and role of statistics for management. Types of data, data
\end{flushleft}


\begin{flushleft}
measurement scales, Descriptive Statistics: Measures of Central
\end{flushleft}





\begin{flushleft}
MSL723 Telecommunications Systems Management
\end{flushleft}


\begin{flushleft}
3 Credits (3-0-0)
\end{flushleft}


\begin{flushleft}
Module I : Telecom Technology Systems Evolution: Recent
\end{flushleft}


\begin{flushleft}
Developments in Telecom Industry, Regulation \& Liberalization
\end{flushleft}


\begin{flushleft}
policy. Techno-managerial aspects of telecommunication, role of
\end{flushleft}


\begin{flushleft}
the telecommunication managers in a dynamic environment. The
\end{flushleft}


\begin{flushleft}
business of telecommunication; telecommunication as a facilitating
\end{flushleft}


\begin{flushleft}
infrastructure for economic development of the country, technical
\end{flushleft}


\begin{flushleft}
survey of the ways and means that voice, data and video traffic are
\end{flushleft}


\begin{flushleft}
moved long distances, data network, the telephone system.
\end{flushleft}


\begin{flushleft}
Module II : Issues of the monopolization and deregulation of telecom,
\end{flushleft}


\begin{flushleft}
national telecom policy, various institutions/organizations like telecom
\end{flushleft}


\begin{flushleft}
regulatory authority etc; conveyance. Telecom service costing,
\end{flushleft}


\begin{flushleft}
economic evaluation of telecom projects, telecom project financing.
\end{flushleft}


\begin{flushleft}
Module III : Telecom marketing, building brand equity for competitive
\end{flushleft}


\begin{flushleft}
advantage, Customer care, total service quality management, preparing
\end{flushleft}


\begin{flushleft}
for the new millennium managing change and people development.
\end{flushleft}





\begin{flushleft}
MSL724 Business Communication
\end{flushleft}


\begin{flushleft}
1.5 Credits (1.5-0-0)
\end{flushleft}


\begin{flushleft}
Pre--requisites: MSL301 \& MSL302
\end{flushleft}


\begin{flushleft}
On completion of this course, students would be able to: Evaluate the
\end{flushleft}


\begin{flushleft}
key purposes of communication in business. Explain the communication
\end{flushleft}


\begin{flushleft}
process model and the barriers to effective communication. Understand
\end{flushleft}


\begin{flushleft}
\& evaluate the changing landscape of business communication. Apply
\end{flushleft}


\begin{flushleft}
techniques for effective communication.
\end{flushleft}





\begin{flushleft}
MSL725 Business Negotiations
\end{flushleft}


\begin{flushleft}
1.5 Credits (1.5-0-0)
\end{flushleft}


\begin{flushleft}
Pre--requisites: MSL301 \& MSL302
\end{flushleft}


\begin{flushleft}
Students who complete this course would be able to: (a) Understand the
\end{flushleft}


\begin{flushleft}
nature, process and structure of negotiations. (b) Understand different
\end{flushleft}


\begin{flushleft}
types of negotiations and the dynamics of cooperative and competitive
\end{flushleft}


\begin{flushleft}
interaction in negotiations. (d) Appreciate and leverage their bargaining
\end{flushleft}


\begin{flushleft}
position in a situation. (e) Learn and apply influence and persuasion
\end{flushleft}


\begin{flushleft}
techniques. (f) Learn ways to build lasting working relationships.
\end{flushleft}


\begin{flushleft}
(g) Understand and appreciate ethical negotiations.
\end{flushleft}





233





\begin{flushleft}
\newpage
Management Studies
\end{flushleft}





\begin{flushleft}
MSL726 Telecom Systems Analysis, Planning and Design
\end{flushleft}


\begin{flushleft}
3 Credits (3-0-0)
\end{flushleft}


\begin{flushleft}
Pre--requisites: MSL301 \& MSL302
\end{flushleft}





\begin{flushleft}
MSL733 Organization Theory
\end{flushleft}


\begin{flushleft}
1.5 Credits (1.5-0-0)
\end{flushleft}


\begin{flushleft}
Pre-requisites: MSL301 \& MSL302
\end{flushleft}





\begin{flushleft}
Module I : An introduction to the basic system analysis tools, the
\end{flushleft}


\begin{flushleft}
procedures for conducting system analysis advanced software
\end{flushleft}


\begin{flushleft}
principles, techniques and processes for designing and implementing
\end{flushleft}


\begin{flushleft}
complex telecommunication systems.
\end{flushleft}





\begin{flushleft}
Different issues related to the organization would be discussed and
\end{flushleft}


\begin{flushleft}
then applied in real life situations, the emphasis will be on application
\end{flushleft}


\begin{flushleft}
of theory to real life situations. The course would be imparted through
\end{flushleft}


\begin{flushleft}
a combination of lectures, cases and simulation exercises.
\end{flushleft}





\begin{flushleft}
Module II : Planning and implementation of telecommunications
\end{flushleft}


\begin{flushleft}
systems from strategic planning through requirements, the initial
\end{flushleft}


\begin{flushleft}
analysis, the general feasibility study, structured analysis, detailed
\end{flushleft}


\begin{flushleft}
analysis, logical design, and implementation.
\end{flushleft}


\begin{flushleft}
Module III : Current system documentation through use of classical
\end{flushleft}


\begin{flushleft}
and structural tools and techniques for describing flows, data flows,
\end{flushleft}


\begin{flushleft}
data structures, file designs, input and output designs, and program
\end{flushleft}


\begin{flushleft}
specifications. The student would gain practical experience through
\end{flushleft}


\begin{flushleft}
a project as part of a term paper.
\end{flushleft}





\begin{flushleft}
MSL734 Management of Small \& Medium Scale
\end{flushleft}


\begin{flushleft}
Industrial Enterprises
\end{flushleft}


\begin{flushleft}
3 Credits (3-0-0)
\end{flushleft}


\begin{flushleft}
Pre--requisites: MSL301 \& MSL302
\end{flushleft}


\begin{flushleft}
Module I : MSME Act 2006; Nature of entrepreneurial management, the
\end{flushleft}


\begin{flushleft}
new entrepreneur, his problems and prospects in the Indian environment.
\end{flushleft}


\begin{flushleft}
Practical aspects of setting up and running of industrial enterprises
\end{flushleft}


\begin{flushleft}
including formulation of projects and feasibility study for new projects.
\end{flushleft}





\begin{flushleft}
MSL727 Interpersonal Behavior \& Team Dynamics
\end{flushleft}


\begin{flushleft}
1.5 Credits (1.5-0-0)
\end{flushleft}


\begin{flushleft}
Pre--requisites: MSL301 \& MSL302
\end{flushleft}





\begin{flushleft}
Module II : Raising resources for new enterprises. Location, design,
\end{flushleft}


\begin{flushleft}
product and process. Choice of technique in small \& medium
\end{flushleft}


\begin{flushleft}
businesses. Survey needs for growth of the enterprise. Monitoring to
\end{flushleft}


\begin{flushleft}
avoid sickness. Development and diversification.
\end{flushleft}





\begin{flushleft}
On completion of this course, students would be able to: Understand
\end{flushleft}


\begin{flushleft}
the nature, structure and formation of teams. Appreciate the
\end{flushleft}


\begin{flushleft}
competitive and collaborative dynamics between teams and subteams. Understand and apply techniques for building and sustaining
\end{flushleft}


\begin{flushleft}
high performing teams. Reflect on their roles within teams and its
\end{flushleft}


\begin{flushleft}
impact on other members.
\end{flushleft}





\begin{flushleft}
Module III : Integration with LSEs and MNCs. Informations network
\end{flushleft}


\begin{flushleft}
for new enterprises. Implication of WTO to SMEs. Globalisation \&
\end{flushleft}


\begin{flushleft}
Competitiveness of SMEs. Entrepreneurship in the globalisation era.
\end{flushleft}





\begin{flushleft}
MSL728 International Telecommunication Management
\end{flushleft}


\begin{flushleft}
3 Credits (3-0-0)
\end{flushleft}


\begin{flushleft}
Module I : Historical development and evolution of telecom, managerial
\end{flushleft}


\begin{flushleft}
issues and structure of industry; evolution and role of international
\end{flushleft}


\begin{flushleft}
institutions; global trends in liberalization and de-regulations, Patterns
\end{flushleft}


\begin{flushleft}
of Transaction in international telecom management; managing the
\end{flushleft}


\begin{flushleft}
market growth; developing, operating and monitoring regulation issues.
\end{flushleft}


\begin{flushleft}
Module II : Role of telecommunications in socio-economic
\end{flushleft}


\begin{flushleft}
development; ICT \& Social change, new technologies and services
\end{flushleft}


\begin{flushleft}
for international telecommunications; data services and business
\end{flushleft}


\begin{flushleft}
applications, Telecom prospectus of WTO \& other international bodies.
\end{flushleft}


\begin{flushleft}
Module III : Current issues and organisational growth; telecom
\end{flushleft}


\begin{flushleft}
implications for the industry, value added services and market drives;
\end{flushleft}


\begin{flushleft}
regional prospectives on development of telecom; Human Resources
\end{flushleft}


\begin{flushleft}
Planning and Industrial relations in ITSM; skill formation for ITSM and
\end{flushleft}


\begin{flushleft}
learning renewal, future directions of growth.
\end{flushleft}





\begin{flushleft}
MSL729 Individual Behavior in Organization
\end{flushleft}


\begin{flushleft}
1.5 Credits (1.5-0-0)
\end{flushleft}


\begin{flushleft}
Pre--requisites: MSL301 \& MSL302
\end{flushleft}





\begin{flushleft}
MSL740 Quantitative Methods in Management
\end{flushleft}


\begin{flushleft}
3 Credits (3-0-0)
\end{flushleft}


\begin{flushleft}
Pre--requisites: MSL301 \& MSL302
\end{flushleft}


\begin{flushleft}
Module I : Role of quantitative methods and operations research
\end{flushleft}


\begin{flushleft}
for managerial decision making and support. Role of mathematical
\end{flushleft}


\begin{flushleft}
models in problem formulation and solving. Structure of decisions,
\end{flushleft}


\begin{flushleft}
statistical decision theory; decision making under uncertainty, risk,
\end{flushleft}


\begin{flushleft}
certainty. Decision Trees; Fuzzy Decision Making. Game theoretic
\end{flushleft}


\begin{flushleft}
applications. Mathematical Programming models- formulation and
\end{flushleft}


\begin{flushleft}
applications. Linear Programming- graphical method, Simplex
\end{flushleft}


\begin{flushleft}
technique; transportation, assignment and transshipment problems.
\end{flushleft}


\begin{flushleft}
Mixed Integer Programming.
\end{flushleft}


\begin{flushleft}
Module II : Non-Linear Programming, introduction to Quadratic
\end{flushleft}


\begin{flushleft}
Programming, Geometric Programming and Direct Search techniques.
\end{flushleft}


\begin{flushleft}
Multiple Criteria Decision making- Goal programming, TOPSIS and AHP.
\end{flushleft}


\begin{flushleft}
Module III : Sequential decisions using Dynamic Programming. PERT
\end{flushleft}


\begin{flushleft}
and CPM. Queuing theory- M/M/1 and M/M/n model. Monte Carlo
\end{flushleft}


\begin{flushleft}
System Simulation concepts and applications. Brief introduction to
\end{flushleft}


\begin{flushleft}
Non-traditional optimization. Case Study applications and use of OR
\end{flushleft}


\begin{flushleft}
software packages.
\end{flushleft}





\begin{flushleft}
MSL745 Operations Management
\end{flushleft}


\begin{flushleft}
3 Credits (3-0-0)
\end{flushleft}





\begin{flushleft}
This course will focus on understanding the nature, composition and
\end{flushleft}


\begin{flushleft}
relevance of organizational behaviour. Students will be introduced to
\end{flushleft}


\begin{flushleft}
the fundamental concepts and theories underpinning organizational
\end{flushleft}


\begin{flushleft}
behaviour. For every concept / theory introduced, its application for
\end{flushleft}


\begin{flushleft}
organizations would be discussed.
\end{flushleft}





\begin{flushleft}
MSL730 Managing With Power
\end{flushleft}


\begin{flushleft}
1.5 Credits (1.5-0-0)
\end{flushleft}


\begin{flushleft}
Pre--requisites: MSL301 \& MSL302
\end{flushleft}


\begin{flushleft}
This course covers power dynamics, the basic art of influencing,
\end{flushleft}


\begin{flushleft}
types of power, display of power at various levels and power vs
\end{flushleft}


\begin{flushleft}
empowerment and ethics. Various cases and readings are included
\end{flushleft}


\begin{flushleft}
for deeper understanding and application of the learnings.
\end{flushleft}





\begin{flushleft}
MSL731 Developing Self Awareness
\end{flushleft}


\begin{flushleft}
1.5 Credits (1.5-0-0)
\end{flushleft}


\begin{flushleft}
Pre--requisites: MSL301 \& MSL302
\end{flushleft}


\begin{flushleft}
Important areas of self awareness: Personal values, moral maturity,
\end{flushleft}


\begin{flushleft}
cognitive styles, attitude towards change, Locus of control, social
\end{flushleft}


\begin{flushleft}
needs of achievement, inclusion, control and affiliation.
\end{flushleft}





\begin{flushleft}
Module I : Managing operations; planning and design of production
\end{flushleft}


\begin{flushleft}
and operations systems. Service characteristics. Facilities planninglocation, layout and movement of materials. Line balancing. Analytical
\end{flushleft}


\begin{flushleft}
tools and techniques for facilities planning and design.
\end{flushleft}


\begin{flushleft}
Module II : Production forecasting. Aggregate planning and operations
\end{flushleft}


\begin{flushleft}
scheduling, Production Planning and Control. Purchasing, Materials
\end{flushleft}


\begin{flushleft}
Management and Inventory control and JIT Material Requirements
\end{flushleft}


\begin{flushleft}
Planning. MRPII, ERP, Optimization techniques applications.
\end{flushleft}


\begin{flushleft}
Module III : Work Study, Value Engineering, Total quality \&
\end{flushleft}


\begin{flushleft}
statistical process control. Maintenance management and equipment
\end{flushleft}


\begin{flushleft}
policies. Network planning and control. Line of Balance, World class
\end{flushleft}


\begin{flushleft}
manufacturing and factories of the future, Case studies.
\end{flushleft}





\begin{flushleft}
MSL760 Marketing Management
\end{flushleft}


\begin{flushleft}
3 Credits (3-0-0)
\end{flushleft}


\begin{flushleft}
Module I : Introduction to Marketing function; genesis, the marketing
\end{flushleft}


\begin{flushleft}
concept. Marketing Management System: objectives, its interfaces with
\end{flushleft}


\begin{flushleft}
other functions in the organisation. Environment of Marketing-Political
\end{flushleft}


\begin{flushleft}
Environment Economic Environment, Market segmentation Consumer
\end{flushleft}


\begin{flushleft}
buying behaviour. Socio- cultural environment. Legal Environment.
\end{flushleft}


\begin{flushleft}
Ethical issues in marketing.
\end{flushleft}





234





\begin{flushleft}
\newpage
Management Studies
\end{flushleft}





\begin{flushleft}
Module II : Marketing Strategy- Marketing planning and Marketing
\end{flushleft}


\begin{flushleft}
programming. The concept of marketing mix, Product policy; the
\end{flushleft}


\begin{flushleft}
concept of product life cycle. New product decisions. Test marketingPricing, Management of distribution: channels of distribution.
\end{flushleft}


\begin{flushleft}
Advertising and promotions. The concept of Unique Selling Proposition.
\end{flushleft}


\begin{flushleft}
Module III : Implementation and Control. The marketing organizationalternative organization structures; the concept of product
\end{flushleft}


\begin{flushleft}
management. Administration of the marketing programme: sales
\end{flushleft}


\begin{flushleft}
forecasting; marketing and sales budgeting; sales management;
\end{flushleft}


\begin{flushleft}
management of sales force. Evaluation of marketing performance;
\end{flushleft}


\begin{flushleft}
sales analysis; control of marketing effort; marketing audit.
\end{flushleft}





\begin{flushleft}
MSL780 Managerial Economics
\end{flushleft}


\begin{flushleft}
1.5 Credits (1.5-0-0)
\end{flushleft}


\begin{flushleft}
Pre--requisites: MSL301 \& MSL302
\end{flushleft}





\begin{flushleft}
Module II : Starting a new technological venture and developing the
\end{flushleft}


\begin{flushleft}
business: Business idea, Business plan, Marketing plan, Financial plan,
\end{flushleft}


\begin{flushleft}
Organisational plan. Financing a new Venture: Sources of Capital,
\end{flushleft}


\begin{flushleft}
Venture Capital, Going public. Enterprenrurship \& liberalization.
\end{flushleft}


\begin{flushleft}
Module III : Managing the new technological venture: Developing
\end{flushleft}


\begin{flushleft}
systems in new venture, Managing doing early operations, Growth
\end{flushleft}


\begin{flushleft}
and expansion, ending the venture. Legal issues, Franchising and
\end{flushleft}


\begin{flushleft}
acquisition. Entrepreneurship, globalisation and Entrepreneurship.
\end{flushleft}





\begin{flushleft}
MSV803 Selected Topics in Information Technology
\end{flushleft}


\begin{flushleft}
Management
\end{flushleft}


\begin{flushleft}
1 Credit (1-0-0)
\end{flushleft}


\begin{flushleft}
Cutting edge will be covered.
\end{flushleft}





\begin{flushleft}
Introduction to managerial economics. Basic concepts, Consumer
\end{flushleft}


\begin{flushleft}
behavior, Demand analysis: Determinants, estimation and managerial
\end{flushleft}


\begin{flushleft}
uses of elasticity of demand. Demand forecasting. Supply function
\end{flushleft}


\begin{flushleft}
and Market equilibrium analysis. Production and Cost analysis and
\end{flushleft}


\begin{flushleft}
Equilibrium of the firm, production Analysis, Productivity Analysis,
\end{flushleft}


\begin{flushleft}
Production efficiency analysis. Pricing and output under different
\end{flushleft}


\begin{flushleft}
market situations: Perfect Competition, Monopolistic Competition,
\end{flushleft}


\begin{flushleft}
Monopoly, Oligopoly and Cartels.
\end{flushleft}





\begin{flushleft}
MSL801 Technology Forecasting \& Assessment
\end{flushleft}


\begin{flushleft}
3 Credits (3-0-0)
\end{flushleft}


\begin{flushleft}
Pre--requisites: MSL301 \& MSL302
\end{flushleft}





\begin{flushleft}
MSL804 Procurement Management
\end{flushleft}


\begin{flushleft}
3 Credits (3-0-0)
\end{flushleft}


\begin{flushleft}
Pre--requisites: MSL301 \& MSL302
\end{flushleft}


\begin{flushleft}
This course will introduce students to purchasing and materials
\end{flushleft}


\begin{flushleft}
management by learning the planning production process, master
\end{flushleft}


\begin{flushleft}
scheduling, material requirements, and forecasting material demands
\end{flushleft}


\begin{flushleft}
and inventory levels. This course is designed to build on the student's
\end{flushleft}


\begin{flushleft}
knowledge of how effective material management improves supply
\end{flushleft}


\begin{flushleft}
chain performance.
\end{flushleft}





\begin{flushleft}
MSV804 Selected Topics in Operations Management
\end{flushleft}


\begin{flushleft}
1 Credit (1-0-0)
\end{flushleft}


\begin{flushleft}
Cutting edge will be covered.
\end{flushleft}





\begin{flushleft}
Module I : Forecasting as an input to technology planning, Futures
\end{flushleft}


\begin{flushleft}
Research, Elements of forecasting process. Types of forecasting
\end{flushleft}


\begin{flushleft}
methods. Quantitative methods of forecasting: time series models,
\end{flushleft}


\begin{flushleft}
growth curves, Precursor, Envelope curves, Experience curves,
\end{flushleft}


\begin{flushleft}
technical assessment.
\end{flushleft}


\begin{flushleft}
Module II : Qualitative methods: Morphological analysis, Relevance
\end{flushleft}


\begin{flushleft}
trees, Delphi, Technological gap analysis, Analogy method, Organising
\end{flushleft}


\begin{flushleft}
for Technology Forecasting.
\end{flushleft}


\begin{flushleft}
Module III : Technology assessment: Components, problem definition,
\end{flushleft}


\begin{flushleft}
Social description, Measure, Impact assessment. Strategies for
\end{flushleft}


\begin{flushleft}
assessment, Economic impact analysis. Assessment of risk and
\end{flushleft}


\begin{flushleft}
uncertainty. Safety and environment considerations.
\end{flushleft}





\begin{flushleft}
MSL805 Services Operations Management
\end{flushleft}


\begin{flushleft}
3 Credits (3-0-0)
\end{flushleft}


\begin{flushleft}
Pre--requisites: MSL301 \& MSL302
\end{flushleft}


\begin{flushleft}
This case course explores the dimensions of successful service firms. It
\end{flushleft}


\begin{flushleft}
prepares students for enlightened management and suggests creative
\end{flushleft}


\begin{flushleft}
entrepreneurial opportunities. The main idea behind the course is:
\end{flushleft}


\begin{flushleft}
To study {``}breakthrough'' services in order to understand the operations
\end{flushleft}


\begin{flushleft}
of successful service firms that can be benchmarks for future
\end{flushleft}


\begin{flushleft}
management practice.
\end{flushleft}


\begin{flushleft}
To develop an understanding of the {``}state of the art'' of service
\end{flushleft}


\begin{flushleft}
management thinking.
\end{flushleft}





\begin{flushleft}
MSV801 Selected Topics in OB \& HR Management
\end{flushleft}


\begin{flushleft}
1 Credit (1-0-0)
\end{flushleft}





\begin{flushleft}
To understand the dimensions of service growth both domestically
\end{flushleft}


\begin{flushleft}
and internationally.
\end{flushleft}





\begin{flushleft}
Cutting edge will be covered.
\end{flushleft}





\begin{flushleft}
MSV805 Selected Topics in Economics
\end{flushleft}


\begin{flushleft}
1 Credit (1-0-0)
\end{flushleft}





\begin{flushleft}
MSL802 Management of Intellectual Property Rights
\end{flushleft}


\begin{flushleft}
3 Credits (3-0-0)
\end{flushleft}


\begin{flushleft}
Pre--requisites: MSL301 \& MSL302
\end{flushleft}





\begin{flushleft}
Cutting edge will be covered.
\end{flushleft}





\begin{flushleft}
Module I : Nature of Intellectual Property; Patents, Industrial Design,
\end{flushleft}


\begin{flushleft}
Trademark and Copyright; Process of patenting and development;
\end{flushleft}


\begin{flushleft}
technological research, innovation, patenting, development;
\end{flushleft}


\begin{flushleft}
International cooperation on Intellectual Property; International
\end{flushleft}


\begin{flushleft}
treaties on IPRs; Patenting under PCT. Procedure for grants of patents.
\end{flushleft}





\begin{flushleft}
MSL806 Mergers \& Acquisitions
\end{flushleft}


\begin{flushleft}
3 Credits (3-0-0)
\end{flushleft}


\begin{flushleft}
Pre--requisites: MSL301 \& MSL302
\end{flushleft}


\begin{flushleft}
(i) Valuation \& Financial framework of M\&A. (ii) The strategic
\end{flushleft}


\begin{flushleft}
perspective of M\&A. (iii) The managerial perspective of M\&A.
\end{flushleft}





\begin{flushleft}
Module II : Scope of Patent Rights;Licensing and transfer of technology;
\end{flushleft}


\begin{flushleft}
Patent information and databases; Geographical Indications.
\end{flushleft}





\begin{flushleft}
MSV806 Selected Topics in Marketing Management
\end{flushleft}


\begin{flushleft}
1 Credit (1-0-0)
\end{flushleft}





\begin{flushleft}
Module III : Administration of Patent System. New developments in
\end{flushleft}


\begin{flushleft}
IPR; IPR of biological systems,plant varieties, computer softwares etc.
\end{flushleft}


\begin{flushleft}
Traditional knowledge; Case Studies; IPR and IITs.
\end{flushleft}





\begin{flushleft}
Cutting edge will be covered.
\end{flushleft}





\begin{flushleft}
MSV802 Selected Topics in Finance
\end{flushleft}


\begin{flushleft}
1 Credit (1-0-0)
\end{flushleft}





\begin{flushleft}
MSL807 Selected Topics in Strategic Management
\end{flushleft}


\begin{flushleft}
1 Credit (1-0-0)
\end{flushleft}


\begin{flushleft}
Pre--requisites: MSL301 \& MSL302
\end{flushleft}


\begin{flushleft}
Open Slot Course (To be decided when the course is floated).
\end{flushleft}





\begin{flushleft}
Cutting edge will be covered.
\end{flushleft}





\begin{flushleft}
MSL803 Technical Entrepreneurship
\end{flushleft}


\begin{flushleft}
3 Credits (3-0-0)
\end{flushleft}


\begin{flushleft}
Module I : Basis and challenges of entrepreneurship Technological
\end{flushleft}


\begin{flushleft}
entrepreneurship, Innovation and entrepreneurship in technology
\end{flushleft}


\begin{flushleft}
based organisations, High tech. entrepreneurship. Entrepreneurial
\end{flushleft}


\begin{flushleft}
characteristics. Concept of new ventures. Technology absorption,
\end{flushleft}


\begin{flushleft}
Appropriate technology. Networking with industries and institutions.
\end{flushleft}





\begin{flushleft}
MSL808 Systems Thinking
\end{flushleft}


\begin{flushleft}
3 Credits (3-0-0)
\end{flushleft}


\begin{flushleft}
Pre--requisites: MSL301 \& MSL302
\end{flushleft}


\begin{flushleft}
Module I: Systems thinking in management; Hard and soft systems
\end{flushleft}


\begin{flushleft}
thinking; open systems thinking; Analytical and systems approaches;
\end{flushleft}


\begin{flushleft}
System concepts, principles and metaphors; General systems theory
\end{flushleft}


\begin{flushleft}
and cybernetics.
\end{flushleft}





235





\begin{flushleft}
\newpage
Management Studies
\end{flushleft}





\begin{flushleft}
Module II: Theory building with causal loop diagrams; Feedback loop
\end{flushleft}


\begin{flushleft}
structures; Linking feedback, stock and flow structures; Tutorial on
\end{flushleft}


\begin{flushleft}
Stella; Case Studies on system dynamics modelling
\end{flushleft}


\begin{flushleft}
Module III: Soft systems methodology; Flexible systems thinking;
\end{flushleft}


\begin{flushleft}
Management of continuity and change; Interpretive systems model.
\end{flushleft}





\begin{flushleft}
MSL809 Cyber Security: Managing Risks
\end{flushleft}


\begin{flushleft}
3 Credits (3-0-0)
\end{flushleft}


\begin{flushleft}
Pre--requisites: MSL301 \& MSL302
\end{flushleft}


\begin{flushleft}
This course introduces students to the interdisciplinary field of
\end{flushleft}


\begin{flushleft}
cybersecurity by discussing the following: cybersecurity theory, and
\end{flushleft}


\begin{flushleft}
the relationship of cybersecurity to nations, businesses, society,
\end{flushleft}


\begin{flushleft}
and people, cybersecurity technologies, processes, and procedures,
\end{flushleft}


\begin{flushleft}
analyzing threats, vulnerabilities and risks present in these
\end{flushleft}


\begin{flushleft}
environments, and develop appropriate strategies to mitigate potential
\end{flushleft}


\begin{flushleft}
cybersecurity problems, advanced policy related topics would also be
\end{flushleft}


\begin{flushleft}
covered through which these risks may be mitigated. Other relevant
\end{flushleft}


\begin{flushleft}
advanced topics may be explored.
\end{flushleft}





\begin{flushleft}
MSL810 Advanced Data Mining for Business Decisions
\end{flushleft}


\begin{flushleft}
1.5 Credits (1.5-0-0)
\end{flushleft}


\begin{flushleft}
Pre--requisites: MSL301 \& MSL302
\end{flushleft}


\begin{flushleft}
This course will expose the participants to the following topics within
\end{flushleft}


\begin{flushleft}
this domain: Understanding advanced models of data mining, advanced
\end{flushleft}


\begin{flushleft}
unsupervised mining methods and approaches, Decision Support
\end{flushleft}


\begin{flushleft}
Systems, Group Decision Support Systems, Consensus based systems,
\end{flushleft}


\begin{flushleft}
Multi-criteria decision systems, Knowledge management systems,
\end{flushleft}


\begin{flushleft}
knowledge management methods, Intelligent systems, Hybrid data
\end{flushleft}


\begin{flushleft}
mining methods, Advanced and emergent topics and applications.
\end{flushleft}





\begin{flushleft}
MSL811 Management Control Systems
\end{flushleft}


\begin{flushleft}
3 Credits (3-0-0)
\end{flushleft}


\begin{flushleft}
Pre--requisites: MSL301 \& MSL302
\end{flushleft}


\begin{flushleft}
Module I : Nature of Management Control Systems: planning and
\end{flushleft}


\begin{flushleft}
control process. Essentials of Management Control System. Behavioural
\end{flushleft}


\begin{flushleft}
aspects of Management Control-motivation and morale, goal
\end{flushleft}


\begin{flushleft}
congruency, and so on. Management Control Process: Programming,
\end{flushleft}


\begin{flushleft}
Budgetary Planning and Procedures, Fixed and Flexible Budgeting,
\end{flushleft}


\begin{flushleft}
Zero Base Budgeting. Internal Audit and Internal Control. Standard
\end{flushleft}


\begin{flushleft}
Cost Accounting Systems as measures of operating performance.
\end{flushleft}


\begin{flushleft}
Module II : Variance Analysis and reporting of financial performance:
\end{flushleft}


\begin{flushleft}
Material, Labour and Overhead Cost Variances, Revenue Variances,
\end{flushleft}


\begin{flushleft}
Profit Variances, Variance Reporting.
\end{flushleft}


\begin{flushleft}
Module III : Management Control Structure: Responsibility Accounting
\end{flushleft}


\begin{flushleft}
System- Concept of Responsibility Centre, Expense Centre, Profit
\end{flushleft}


\begin{flushleft}
Centre, Investment Centre. Inter-Divisional Transfer Pricing System,
\end{flushleft}


\begin{flushleft}
Measurement of Division Performance.
\end{flushleft}





\begin{flushleft}
MSL812 Flexible Systems Management
\end{flushleft}


\begin{flushleft}
3 Credits (3-0-0)
\end{flushleft}


\begin{flushleft}
Pre--requisites: MSL301 \& MSL302
\end{flushleft}


\begin{flushleft}
Module I : Emerging management paradigms: Total Quality
\end{flushleft}


\begin{flushleft}
Management, Business Process Reengineering, Learning Organisation,
\end{flushleft}


\begin{flushleft}
World Class Organisation, Flexibility in Management. Concept of
\end{flushleft}


\begin{flushleft}
systemic flexibility. Liberalisation, Globalisation and change. New
\end{flushleft}


\begin{flushleft}
Organisation forms.
\end{flushleft}


\begin{flushleft}
Module II : Concept and dimensions of Systemic flexibility. Managing
\end{flushleft}


\begin{flushleft}
paradoxes. Methodology and tools of flexible systems management.
\end{flushleft}


\begin{flushleft}
Underlying values, and guiding principles, Case Analysis using SAP-LAP
\end{flushleft}


\begin{flushleft}
framework. SAP-LAP models and linkages.
\end{flushleft}


\begin{flushleft}
Module III : Flexibility in functional systems, Information Systems
\end{flushleft}


\begin{flushleft}
flexibility, manufacturing flexibility, organisational flexibility,
\end{flushleft}


\begin{flushleft}
financial flexibility, and strategic flexibility. Linkage of flexibility with
\end{flushleft}


\begin{flushleft}
organisational performance.
\end{flushleft}





\begin{flushleft}
MSL813 Systems Methodology for Management
\end{flushleft}


\begin{flushleft}
3 Credits (3-0-0)
\end{flushleft}


\begin{flushleft}
Pre--requisites: MSL301 \& MSL302
\end{flushleft}


\begin{flushleft}
Module I : Introduction to systems methodology, Flexible Systems
\end{flushleft}





\begin{flushleft}
Methodology, Need and applicability of Systems methodology for
\end{flushleft}


\begin{flushleft}
management. Nature of managerial problems. System Dynamics
\end{flushleft}


\begin{flushleft}
Methodology- Philosophy, Foundation, Steps, building blocks, feedback
\end{flushleft}


\begin{flushleft}
structures, principles of systems, learning organisation.
\end{flushleft}


\begin{flushleft}
Module II : Validation, Simulation and testing of System Dynamics
\end{flushleft}


\begin{flushleft}
models, Policy analysis, Micro world and Management games,
\end{flushleft}


\begin{flushleft}
Managerial applications of Systems methodology.
\end{flushleft}


\begin{flushleft}
Module III : Management of physical systems. Physical system theory:
\end{flushleft}


\begin{flushleft}
fundamental premises and postulates, modelling of basic processes,
\end{flushleft}


\begin{flushleft}
application to manufacturing, managerial, and socio-economic
\end{flushleft}


\begin{flushleft}
systems. Critical comparison and integration of Physical System Theory
\end{flushleft}


\begin{flushleft}
and System Dynamics. Flexibility in physical system theory.
\end{flushleft}





\begin{flushleft}
MSL814 Data Visualization
\end{flushleft}


\begin{flushleft}
1.5 Credits (1.5-0-0)
\end{flushleft}


\begin{flushleft}
Pre--requisites: MSL301 \& MSL302
\end{flushleft}


\begin{flushleft}
This course would have the following: It would train the participants
\end{flushleft}


\begin{flushleft}
to use visual imagery to present complex information and the trends
\end{flushleft}


\begin{flushleft}
associated with extensive data. Visualization provides a solution to
\end{flushleft}


\begin{flushleft}
address information overload, through a well-designed visual encoding
\end{flushleft}


\begin{flushleft}
to aid comprehension, memory, and decision making. Furthermore,
\end{flushleft}


\begin{flushleft}
visual representations may help engage more diverse audiences in
\end{flushleft}


\begin{flushleft}
the process of analytic thinking. Topics like data and image models,
\end{flushleft}


\begin{flushleft}
heat maps, infographics, multidimensional data visualization and
\end{flushleft}


\begin{flushleft}
representation, graphical perceptions, mapping \& cartography and
\end{flushleft}


\begin{flushleft}
text visualization may be covered. Other relevant topics within the
\end{flushleft}


\begin{flushleft}
subject domain may also be explored.
\end{flushleft}





\begin{flushleft}
MSL815 Decision Support and Expert Systems
\end{flushleft}


\begin{flushleft}
3 Credits (3-0-0)
\end{flushleft}


\begin{flushleft}
Pre--requisites: MSL301 \& MSL302
\end{flushleft}


\begin{flushleft}
Module I : The management support framework for computers.
\end{flushleft}


\begin{flushleft}
Fundamentals of decision theory and decision modelling. Humans and
\end{flushleft}


\begin{flushleft}
information processors and information systems as decision systems.
\end{flushleft}


\begin{flushleft}
Human decision styles.
\end{flushleft}


\begin{flushleft}
Module II : Models, heuristics, and simulation. Overview of DSSdatabase, modelbase, user interface. DSS development methodology
\end{flushleft}


\begin{flushleft}
and tools. Need for expertise in decision models and expert systems.
\end{flushleft}


\begin{flushleft}
Expert systems fundamentals. Knowledge engineering, knowledge
\end{flushleft}


\begin{flushleft}
representation and inferencing. Building expert systems.
\end{flushleft}


\begin{flushleft}
Module III : Integrating expert systems and DSSs. Strategies for
\end{flushleft}


\begin{flushleft}
implementing and maintaining management support systems. Case
\end{flushleft}


\begin{flushleft}
studies, and laboratory and filed projects.
\end{flushleft}





\begin{flushleft}
MSV815 Case Study Writing and Teaching
\end{flushleft}


\begin{flushleft}
1 Credit (1-0-0)
\end{flushleft}


\begin{flushleft}
Various concepts of case study teaching and writing will be covered.
\end{flushleft}





\begin{flushleft}
MSL816 Total Quality Management
\end{flushleft}


\begin{flushleft}
3 Credits (3-0-0)
\end{flushleft}


\begin{flushleft}
Pre--requisites: MSL301 \& MSL302
\end{flushleft}


\begin{flushleft}
Module I : Introduction to TQM; Customer Orientation, Continuous
\end{flushleft}


\begin{flushleft}
Improvement, Quality, Productivity and Flexibility, Approaches and
\end{flushleft}


\begin{flushleft}
philosophies of TQM, Quality Awards, Strategic Quality Management,
\end{flushleft}


\begin{flushleft}
TQM and corporate culture, Total Quality Control; Basic Analytical
\end{flushleft}


\begin{flushleft}
tools-Check Sheets; Histograms; Pareto charts, Cause and Effect
\end{flushleft}


\begin{flushleft}
diagrams; Flow charts.
\end{flushleft}


\begin{flushleft}
Module II : Statistical Process Control; Advanced Analytical toolsStatistical Design of Experiments; Taguchi Approach; Cost of Quality;
\end{flushleft}


\begin{flushleft}
Reliability and failure analysis. FMECA, Quality Function Deployment,
\end{flushleft}


\begin{flushleft}
Benchmarking, Concurrent Engineering.
\end{flushleft}


\begin{flushleft}
Module III : Quality Teams, Employee practices in TQM organisations:
\end{flushleft}


\begin{flushleft}
Leadership, delegation; empowerment and motivation; role of
\end{flushleft}


\begin{flushleft}
communication in Total Quality, Quality Circles; Total Employee
\end{flushleft}


\begin{flushleft}
Involvement; Problem Solving in TQM- Brain storming; Nominal Group
\end{flushleft}


\begin{flushleft}
Technique Team process; Kaizen and Innovation; Measurement and
\end{flushleft}


\begin{flushleft}
audit for TQM; Quality Information Systems, ISO 9000 series of Quality
\end{flushleft}


\begin{flushleft}
Standards; TQM Implementation; Reengineering and TQM.
\end{flushleft}





236





\begin{flushleft}
\newpage
Management Studies
\end{flushleft}





\begin{flushleft}
MSL817 Systems Waste \& Sustainability
\end{flushleft}


\begin{flushleft}
3 Credits (3-0-0)
\end{flushleft}


\begin{flushleft}
Pre--requisites: MSL301 \& MSL302
\end{flushleft}





\begin{flushleft}
developing of newly industrialized countries and Japan. Management
\end{flushleft}


\begin{flushleft}
of Multinational firms.
\end{flushleft}





\begin{flushleft}
Module I : Introduction to waste and waste management. The concept
\end{flushleft}


\begin{flushleft}
of wastivity and its inter-relationship with Productivity Quality and
\end{flushleft}


\begin{flushleft}
Flexibility. Systems concept of waste, complementarily of waste and
\end{flushleft}


\begin{flushleft}
resource management. Functional elements of waste management.
\end{flushleft}


\begin{flushleft}
Waste management and cost reduction. Taxonomy of wastes, JIT,
\end{flushleft}


\begin{flushleft}
TQM and waste.
\end{flushleft}


\begin{flushleft}
Module II : Management of waste in industrial and service sectors.
\end{flushleft}


\begin{flushleft}
Management of manpower waste and unemployment. Management
\end{flushleft}


\begin{flushleft}
of energy waste in the national economy. Energy recycling, Waste
\end{flushleft}


\begin{flushleft}
management and energy conservation. Total energy concept, overall
\end{flushleft}


\begin{flushleft}
energy wastivity.
\end{flushleft}


\begin{flushleft}
Module III : Interfaces of waste management: environment control,
\end{flushleft}


\begin{flushleft}
nature conservation, resource development, Quality and Productivity
\end{flushleft}


\begin{flushleft}
Management, Business Process Reengineering. Role of legislation and
\end{flushleft}


\begin{flushleft}
government. Waste management and national planning.
\end{flushleft}





\begin{flushleft}
MSL818 Industrial Waste Management
\end{flushleft}


\begin{flushleft}
3 Credits (3-0-0)
\end{flushleft}


\begin{flushleft}
Pre--requisites: MSL301 \& MSL302
\end{flushleft}


\begin{flushleft}
Module I : The concept of industrial system. Systems waste and waste
\end{flushleft}


\begin{flushleft}
management. Wastivity and productivity measurement. The categories
\end{flushleft}


\begin{flushleft}
of industrial systems waste. Stages and causes of waste generation in
\end{flushleft}


\begin{flushleft}
industrial systems. Waste reduction measures and systems in industry.
\end{flushleft}


\begin{flushleft}
Collection and disposal system of scrap, surplus and obsolete items.
\end{flushleft}


\begin{flushleft}
Recycling and processing of industrial waste. Industrial pollution and
\end{flushleft}


\begin{flushleft}
environment control.
\end{flushleft}


\begin{flushleft}
Module II : Value engineering, design waste and cost reduction.
\end{flushleft}


\begin{flushleft}
Inspection rejects and quality management. Reliability, maintenance,
\end{flushleft}


\begin{flushleft}
breakdown and management of waste. Space waste and layout
\end{flushleft}


\begin{flushleft}
planning. Time management, manpower waste in industry,
\end{flushleft}


\begin{flushleft}
absenteeism. Capacity utilization. Waste heat recovery and energy
\end{flushleft}


\begin{flushleft}
waste in industry. Resource conversation/loss prevention in process
\end{flushleft}


\begin{flushleft}
industries. Data and information waste, management of hazardous
\end{flushleft}


\begin{flushleft}
waste. Waste treatment. Natural calamities. Accident prevention,
\end{flushleft}


\begin{flushleft}
industrial safety and waste management.
\end{flushleft}


\begin{flushleft}
Module III : Waste management in Indian industries- present practices,
\end{flushleft}


\begin{flushleft}
potentials and perspectives. Management of waste in different
\end{flushleft}


\begin{flushleft}
industrial systems- steel, aluminum, power, automobile, transport
\end{flushleft}


\begin{flushleft}
and other service industries. Economic analysis and system models
\end{flushleft}


\begin{flushleft}
of industrial waste management systems. Analytical and Creative
\end{flushleft}


\begin{flushleft}
techniques to waste control.
\end{flushleft}





\begin{flushleft}
MSL819 Business Process Re-engineering
\end{flushleft}


\begin{flushleft}
3 Credits (3-0-0)
\end{flushleft}


\begin{flushleft}
Pre--requisites: MSL301 \& MSL302
\end{flushleft}


\begin{flushleft}
odule I : Nature, significance and rationale of Business Process
\end{flushleft}


\begin{flushleft}
Reengineering, Reengineering scenarios in major countries, Problems
\end{flushleft}


\begin{flushleft}
issues, scope and trends in BPR, Implementing BPR: Methodology
\end{flushleft}


\begin{flushleft}
and steps, IT enabled reengineering, mediation and collaboration.
\end{flushleft}


\begin{flushleft}
Module II : The paradigm of Mass customization, managing
\end{flushleft}


\begin{flushleft}
organisational change, Transforming/ Reinventing the enterprise, Team
\end{flushleft}


\begin{flushleft}
building. Case studies of success as well as failure.
\end{flushleft}


\begin{flushleft}
Module III : People view, empowering people, reengineering
\end{flushleft}


\begin{flushleft}
management. Issues of purpose, culture, process and performance,
\end{flushleft}


\begin{flushleft}
and people.
\end{flushleft}





\begin{flushleft}
MSL820 Global Business Environment
\end{flushleft}


\begin{flushleft}
3 Credits (3-0-0)
\end{flushleft}


\begin{flushleft}
Pre--requisites: MSL301 \& MSL302
\end{flushleft}


\begin{flushleft}
Module I : Global Scene. Historical and economic background, firms
\end{flushleft}


\begin{flushleft}
and International Business. The global scene and the challenges
\end{flushleft}


\begin{flushleft}
ahead, challenges to free International Trade Political Risk, Protection,
\end{flushleft}


\begin{flushleft}
Accounting, Taxation and Legal practices. The International debt risks.
\end{flushleft}


\begin{flushleft}
Module II : Regional Issues. Global Monetary Institutions and Trade
\end{flushleft}


\begin{flushleft}
Agreements, Regional Trade Agreements and Facts. Socio-cultural
\end{flushleft}


\begin{flushleft}
context of International Business: European countries, U.S.A.
\end{flushleft}





\begin{flushleft}
Module III : Globalization of Indian Economy. Liberalization and
\end{flushleft}


\begin{flushleft}
globalization of Indian business. India's multinationals, Indian laws
\end{flushleft}


\begin{flushleft}
and policies relating to investment in India by international firms and
\end{flushleft}


\begin{flushleft}
outside India by Indian firms.
\end{flushleft}





\begin{flushleft}
MSL821 Strategy Execution Excellence
\end{flushleft}


\begin{flushleft}
3 Credits (3-0-0)
\end{flushleft}


\begin{flushleft}
Pre--requisites: MSL301 \& MSL302
\end{flushleft}


\begin{flushleft}
Maximize your leadership potential by expanding your management
\end{flushleft}


\begin{flushleft}
skills through this one-year graduate certificate management program.
\end{flushleft}


\begin{flushleft}
This comprehensive program offers you the opportunity to broaden
\end{flushleft}


\begin{flushleft}
your perspective on salient management responsibilities and skills in
\end{flushleft}


\begin{flushleft}
key sectors such as health care, not-for-profit, community services, and
\end{flushleft}


\begin{flushleft}
technology and trades. Students must also participate in two weekend
\end{flushleft}


\begin{flushleft}
residency (virtual or on-campus) activities. This program enables you
\end{flushleft}


\begin{flushleft}
to leverage your existing career and educational experiences to move
\end{flushleft}


\begin{flushleft}
into management positions. You will take a series of carefully selected
\end{flushleft}


\begin{flushleft}
business courses that will build and enhance your skills in critical areas
\end{flushleft}


\begin{flushleft}
of management such as finance, marketing, human resources, and
\end{flushleft}


\begin{flushleft}
leadership. In addition, you will have the option to select courses from
\end{flushleft}


\begin{flushleft}
specific industry streams-health care, community services, not-forprofit, and trades/technology. These courses are designed to provide
\end{flushleft}


\begin{flushleft}
industry-specific perspectives that will enhance your employability
\end{flushleft}


\begin{flushleft}
and career advancement.
\end{flushleft}





\begin{flushleft}
MSL822 International Business
\end{flushleft}


\begin{flushleft}
3 Credits (3-0-0)
\end{flushleft}


\begin{flushleft}
Pre--requisites: MSL301 \& MSL302
\end{flushleft}


\begin{flushleft}
Module I : Key Issues in International Business. Socio-cultural,
\end{flushleft}


\begin{flushleft}
economic and political forces facing business. International sourcing.
\end{flushleft}


\begin{flushleft}
Understanding the determinants of competitive advantage in
\end{flushleft}


\begin{flushleft}
international business at the national, industry and firm level. Global
\end{flushleft}


\begin{flushleft}
forces transforming international business. Multinational Corporation.
\end{flushleft}


\begin{flushleft}
Problems and Prospects in an International Environment, competitive
\end{flushleft}


\begin{flushleft}
and cooperative business strategy.
\end{flushleft}


\begin{flushleft}
Module II : International Business Strategy of Indian Industry.
\end{flushleft}


\begin{flushleft}
Competitive position of key Indian Industries. Entry strategies
\end{flushleft}


\begin{flushleft}
for Indian firms: Joint Ventures, strategic/technical alliances/
\end{flushleft}


\begin{flushleft}
collaboration. Strategies employed by Indian firms to develop and
\end{flushleft}


\begin{flushleft}
sustain international business.
\end{flushleft}


\begin{flushleft}
Module III : Globalization Strategy. Globalisation strategy, strategies
\end{flushleft}


\begin{flushleft}
of Multinational Corporation, implications for functional strategies:
\end{flushleft}


\begin{flushleft}
marketing, HR, planning, organisational structure, production, Global
\end{flushleft}


\begin{flushleft}
Information Systems, Strategy Alternatives for Global Market entry
\end{flushleft}


\begin{flushleft}
and expansion, International negotiations.
\end{flushleft}





\begin{flushleft}
MSL823 Strategic Change \& Flexibility
\end{flushleft}


\begin{flushleft}
3 Credits (3-0-0)
\end{flushleft}


\begin{flushleft}
Pre--requisites: MSL301 \& MSL302
\end{flushleft}


\begin{flushleft}
Module I : Patterns of Change and Flexibility. Patterns of change,
\end{flushleft}


\begin{flushleft}
liberalization, globalization and privatization, changes in Social Political
\end{flushleft}


\begin{flushleft}
and Economic environment, Technological and organizational change.
\end{flushleft}


\begin{flushleft}
Changes in customer requirements. Impact of change of business
\end{flushleft}


\begin{flushleft}
and workforce. Need for flexibility, concept of Strategic Flexibility:
\end{flushleft}


\begin{flushleft}
Openness, Adaptiveness, Change, and Resilience. Understanding
\end{flushleft}


\begin{flushleft}
the process of strategic change. Managing chaos strategically.
\end{flushleft}


\begin{flushleft}
Regenerating strategies.
\end{flushleft}


\begin{flushleft}
Module II : Revising Strategies Postures. Corporate restructuring,
\end{flushleft}


\begin{flushleft}
Alliances, joint ventures, acquisitions and merges. Reorganising the firm,
\end{flushleft}


\begin{flushleft}
the impact of mergers and acquisitions on organizational performance.
\end{flushleft}


\begin{flushleft}
Management of continuity and change, Blue Ocean strategy.
\end{flushleft}


\begin{flushleft}
Module III : Energising Strategies Change. Reengineering the
\end{flushleft}


\begin{flushleft}
corporation, identification of key business processes. Organization
\end{flushleft}


\begin{flushleft}
of the future. Implementing Strategic Change. Transforming the
\end{flushleft}


\begin{flushleft}
organization. Sustaining change. Consolidating gains and producing
\end{flushleft}


\begin{flushleft}
more change. Anchoring new approaches in the culture. Leading a highcommitment high-performance organization. Organization Vitalizations
\end{flushleft}





237





\begin{flushleft}
\newpage
Management Studies
\end{flushleft}





\begin{flushleft}
MSL824 Policy Dynamics \& Learning Organization
\end{flushleft}


\begin{flushleft}
3 Credits (3-0-0)
\end{flushleft}


\begin{flushleft}
Pre--requisites: MSL301 \& MSL302
\end{flushleft}


\begin{flushleft}
Module I : Learning Organization. Emergence of learning organization.
\end{flushleft}


\begin{flushleft}
Strategies for organization learning, using Feedback, shared vision,
\end{flushleft}


\begin{flushleft}
team work, personal mastery, mental models, systems thinking, role of
\end{flushleft}


\begin{flushleft}
leader, organizational dynamics. Soft Systems Methodology application
\end{flushleft}


\begin{flushleft}
to policy formulation. Flexibility in policy strategy. Strategy formulation
\end{flushleft}


\begin{flushleft}
in a learning organisation, clarifying vision and opportunities for change
\end{flushleft}


\begin{flushleft}
in a learning organization.
\end{flushleft}


\begin{flushleft}
Module II : Micro World and Policy Dynamics. Systems-linked
\end{flushleft}


\begin{flushleft}
organization model. Micro world for policy learning. System Dynamics
\end{flushleft}


\begin{flushleft}
modeling applied to policy formulations, conceptual model. The
\end{flushleft}


\begin{flushleft}
language of systems thinking links and qualitative system dynamics,
\end{flushleft}


\begin{flushleft}
Flexibility Influence Diagram, Collaboration Diagram, Archetypes,
\end{flushleft}


\begin{flushleft}
leverage points, Integrative simulation models.
\end{flushleft}


\begin{flushleft}
Module III : Frontiers. Role playing games and case studies to
\end{flushleft}


\begin{flushleft}
develop principles for successful management of complex strategies
\end{flushleft}


\begin{flushleft}
in a dynamic world. Strategic Management game for policy planning,
\end{flushleft}


\begin{flushleft}
Interactive Planning. Strategic issues such as business cycles, market
\end{flushleft}


\begin{flushleft}
growth and stagnation. And diffusion of new technologies. Knowledge
\end{flushleft}


\begin{flushleft}
management in learning organizations.
\end{flushleft}





\begin{flushleft}
MSL825 Strategies in Functional Management
\end{flushleft}


\begin{flushleft}
3 Credits (3-0-0)
\end{flushleft}


\begin{flushleft}
Pre--requisites: MSL301 \& MSL302
\end{flushleft}





\begin{flushleft}
Module III : Practitioners Perspectives. Business Models for
\end{flushleft}


\begin{flushleft}
Competitiveness, Functional (e.g. HR, Operational, Financial,
\end{flushleft}


\begin{flushleft}
Technological) Linkages, Partnerships/Cooperation for Competitiveness,
\end{flushleft}


\begin{flushleft}
Emerging Issues/ Practices.
\end{flushleft}





\begin{flushleft}
MSL828 Global Strategic Management
\end{flushleft}


\begin{flushleft}
3 Credits (3-0-0)
\end{flushleft}


\begin{flushleft}
Pre--requisites: MSL301 \& MSL302
\end{flushleft}


\begin{flushleft}
Module I : The Process of Globalization and Global Strategy.
\end{flushleft}


\begin{flushleft}
Globalization of markets and competition, globalization and localization,
\end{flushleft}


\begin{flushleft}
Diagnosing Global Industry Potential, Designing a global strategy,
\end{flushleft}


\begin{flushleft}
Making Global strategies work, Global strategic alliances, M\&A.
\end{flushleft}


\begin{flushleft}
Module II : Regional Strategy and Entry Strategy. Regional Strategy,
\end{flushleft}


\begin{flushleft}
Emerging Markets Assessing Country Attractiveness, Entry Strategies:
\end{flushleft}


\begin{flushleft}
Subsidiaries, acquisitions, joint ventures, Licensing, Franchising,
\end{flushleft}


\begin{flushleft}
Agents and Distributors.
\end{flushleft}


\begin{flushleft}
Module III : Managing Globally and Future Challenges. Designing a
\end{flushleft}


\begin{flushleft}
global organization, Global Marketing and Operations, Cross Cultural
\end{flushleft}


\begin{flushleft}
Management, Leadership and Global manager, Globalization and
\end{flushleft}


\begin{flushleft}
the Internet.
\end{flushleft}





\begin{flushleft}
MSL829 Current and Emerging Issues in Strategic
\end{flushleft}


\begin{flushleft}
Management
\end{flushleft}


\begin{flushleft}
3 Credits (3-0-0)
\end{flushleft}


\begin{flushleft}
Pre--requisites: MSL301 \& MSL302
\end{flushleft}


\begin{flushleft}
(Relevant current and Emerging Issues)
\end{flushleft}





\begin{flushleft}
Module I : Linkage of corporate and Business strategy with various
\end{flushleft}


\begin{flushleft}
Functional strategies, Flexibility in Functional Strategies. Marketing
\end{flushleft}


\begin{flushleft}
Strategy, financial Strategy.
\end{flushleft}


\begin{flushleft}
Module II : Manufacturing Strategy, IT Strategy, Human Resources Strategy.
\end{flushleft}


\begin{flushleft}
Module III : Technology Strategy, Quality and Productivity Strategy,
\end{flushleft}


\begin{flushleft}
Environmental Strategy.
\end{flushleft}





\begin{flushleft}
MSL826 Business Ethics
\end{flushleft}


\begin{flushleft}
3 Credits (3-0-0)
\end{flushleft}


\begin{flushleft}
Pre--requisites: MSL301 \& MSL302
\end{flushleft}


\begin{flushleft}
Module I : Ethics in Business. Historical perspective, culture and
\end{flushleft}


\begin{flushleft}
ethics in India, codes and culture. Economics and the Environment:
\end{flushleft}


\begin{flushleft}
green business, Ethics and Competition. The ethical code, social
\end{flushleft}


\begin{flushleft}
audit. A framework for analysis and action. The sphere of personal
\end{flushleft}


\begin{flushleft}
ethics: consequences, rights and duties, virtue and character. Role of
\end{flushleft}


\begin{flushleft}
objectivity, practicability, judgement and balancing acts. The individual
\end{flushleft}


\begin{flushleft}
and the corporation.
\end{flushleft}


\begin{flushleft}
Module II : Ethical Responsibilities. Ethical responsibilities of economic
\end{flushleft}


\begin{flushleft}
agents: role obligations, obligation to sharesholder, rights and,
\end{flushleft}


\begin{flushleft}
obligations to customers, obligations to pay taxes. Environmental
\end{flushleft}


\begin{flushleft}
protection. Corporate accountability, Ethical conflicts, concern for the
\end{flushleft}


\begin{flushleft}
locality, Attitude to labour. Ethics and Government policies and laws.
\end{flushleft}


\begin{flushleft}
Module III : Ethics in Functions. Ethical responsibilities of organizations
\end{flushleft}


\begin{flushleft}
leader: power, leadership. Obstacles to ethical conduct. Pressures
\end{flushleft}


\begin{flushleft}
for conformity. Evaluation and rewards. Job pressures and issues.
\end{flushleft}


\begin{flushleft}
Organizational change. Ethics in use of Information technology.
\end{flushleft}


\begin{flushleft}
Intellectual Property Rights. Ethics in Marketing. Ethics of advertising
\end{flushleft}


\begin{flushleft}
and sponsorship. Freedom Vs State Control. Acquisitions and Mergers,
\end{flushleft}


\begin{flushleft}
Multinational decision making: Reconciling International norms.
\end{flushleft}





\begin{flushleft}
MSL827 International Competitiveness
\end{flushleft}


\begin{flushleft}
3 Credits (3-0-0)
\end{flushleft}


\begin{flushleft}
Pre--requisites: MSL301 \& MSL302
\end{flushleft}


\begin{flushleft}
Module I : Introduction to Competitiveness. Background, Need, Basics,
\end{flushleft}


\begin{flushleft}
Myths; Global Perspectives, Context, Definitions, Benchmarking \& Key
\end{flushleft}


\begin{flushleft}
Issues; Related concepts: Excellence, Value Creation; Competitiveness
\end{flushleft}


\begin{flushleft}
at Different Levels.
\end{flushleft}


\begin{flushleft}
Module II : Evaluating \& Planning for Competitiveness. Frameworks of
\end{flushleft}


\begin{flushleft}
Competitiveness \& Strategy, Evaluating Competitiveness, Enhancing
\end{flushleft}


\begin{flushleft}
Competitiveness, Competitiveness Processes \& Initiatives, Leadership
\end{flushleft}


\begin{flushleft}
Dimension, Cases.
\end{flushleft}





\begin{flushleft}
MSL830 Organizational Structure and Processes
\end{flushleft}


\begin{flushleft}
3 Credits (3-0-0)
\end{flushleft}


\begin{flushleft}
Pre--requisites: MSL301 \& MSL302
\end{flushleft}


\begin{flushleft}
Module I : Organisational structure- classical and neoclassical theories.
\end{flushleft}


\begin{flushleft}
Strategy and structure. Modern Organizational theory- systems view of
\end{flushleft}


\begin{flushleft}
organisation and integration. Micro, intermediate, macro environment.
\end{flushleft}


\begin{flushleft}
Participative structures.
\end{flushleft}


\begin{flushleft}
Module II : Work culture and organization processes. Decision
\end{flushleft}


\begin{flushleft}
processes, balance and conflict processes. The process of role and
\end{flushleft}


\begin{flushleft}
status development. Influence processes and technological processes.
\end{flushleft}


\begin{flushleft}
Capacity development in organizations.
\end{flushleft}


\begin{flushleft}
Module III : Interface of structure and processes- structural
\end{flushleft}


\begin{flushleft}
functionalism; Allport and Event- Structure theory. Organizational
\end{flushleft}


\begin{flushleft}
Governance- organizations as a subject of political enquiry, Models of
\end{flushleft}


\begin{flushleft}
organizational governance. Making and breaking patterns.
\end{flushleft}





\begin{flushleft}
MSL831 Management of Change
\end{flushleft}


\begin{flushleft}
3 Credits (3-0-0)
\end{flushleft}


\begin{flushleft}
Pre--requisites: MSL301 \& MSL302
\end{flushleft}


\begin{flushleft}
Module I : Process of change and organization theory and practice.
\end{flushleft}


\begin{flushleft}
Elements of change. Achieving Systematic change. Domains of systematic
\end{flushleft}


\begin{flushleft}
change-strategy, technology, structure and people. Planning for change.
\end{flushleft}


\begin{flushleft}
Module II : Change and the use of power. Nature and sources of power.
\end{flushleft}


\begin{flushleft}
Leadership and change- Transactional vs. Transformational change.
\end{flushleft}


\begin{flushleft}
Change cycle including participative and coerced change.
\end{flushleft}


\begin{flushleft}
Module III : Change through behaviour modification. Positive and
\end{flushleft}


\begin{flushleft}
negative reinforcement. Training for change. Managing conflict.
\end{flushleft}


\begin{flushleft}
Implementing change. Adjustment to change and organising
\end{flushleft}


\begin{flushleft}
for growth. Prerequisites and consequence of change. The
\end{flushleft}


\begin{flushleft}
change Dynamics.
\end{flushleft}





\begin{flushleft}
MSL832 Managing Innovation for Organizational
\end{flushleft}


\begin{flushleft}
Effectiveness
\end{flushleft}


\begin{flushleft}
3 Credits (3-0-0)
\end{flushleft}


\begin{flushleft}
Pre--requisites: MSL301 \& MSL302
\end{flushleft}


\begin{flushleft}
Module I : Elements of creativity person, creative organization, nature
\end{flushleft}


\begin{flushleft}
of innovation. Assessing creativity. Tools and techniques for enhancing
\end{flushleft}


\begin{flushleft}
creativity. Innovation and risk.
\end{flushleft}


\begin{flushleft}
Module II : Managing social equity and organisation efficiency paradox,
\end{flushleft}


\begin{flushleft}
blocks to creativity, methods to overcome the blocks. Introducing creativity
\end{flushleft}





238





\begin{flushleft}
\newpage
Management Studies
\end{flushleft}





\begin{flushleft}
in organisation. Structure and creativity. Work culture and innovation.
\end{flushleft}


\begin{flushleft}
Module III : Practices of creativity and intervention strategiesorganization excellence: Criteria and practice-innovation and quality,
\end{flushleft}


\begin{flushleft}
Innovation and BPR/appraisal system- interventions. Innovation
\end{flushleft}


\begin{flushleft}
and competitiveness.
\end{flushleft}





\begin{flushleft}
MSL833 Organizational Development
\end{flushleft}


\begin{flushleft}
3 Credits (3-0-0)
\end{flushleft}


\begin{flushleft}
Pre--requisites: MSL301 \& MSL302
\end{flushleft}


\begin{flushleft}
Module I : Organisation Development- nature and scope. The generic
\end{flushleft}


\begin{flushleft}
and contextual element of developing organisation. Introduction to
\end{flushleft}


\begin{flushleft}
process change. Theories, strategies and techniques of organizational
\end{flushleft}


\begin{flushleft}
diagnosis for improving organisation's problem solving and renewal
\end{flushleft}


\begin{flushleft}
process, legacy factors and organizational growth.
\end{flushleft}


\begin{flushleft}
Module II : Coping with environmental change. Socio-cultural
\end{flushleft}


\begin{flushleft}
dimensions of work and behaviour, Environmental analysis and
\end{flushleft}


\begin{flushleft}
impact. Diagnosis of the ongoing process from symptoms to causes.
\end{flushleft}


\begin{flushleft}
Organisation development and intervention strategies.
\end{flushleft}


\begin{flushleft}
Module III : Personal change. Laboratory learning techniques.
\end{flushleft}


\begin{flushleft}
Managerial Grid. Sensitivity training. Transactional analysis. Intergroup and team building interventions. Management by objectives.
\end{flushleft}


\begin{flushleft}
Total system interventions-stabilising change.
\end{flushleft}





\begin{flushleft}
MSL834 Managing Diversity at Workplace
\end{flushleft}


\begin{flushleft}
1.5 Credits (1.5-0-0)
\end{flushleft}


\begin{flushleft}
Pre-requisites: MSL301 \& MSL302
\end{flushleft}


\begin{flushleft}
The course introduces students to the relational framework towards
\end{flushleft}


\begin{flushleft}
diversity management by discussing the macro, meso and micro
\end{flushleft}


\begin{flushleft}
factors influencing DM. Through analysis of the different organizational
\end{flushleft}


\begin{flushleft}
approaches and initiatives towards diversity management, it highlights
\end{flushleft}


\begin{flushleft}
ways in which inclusive workplaces can be created and diversity
\end{flushleft}


\begin{flushleft}
leveraged for business performance.
\end{flushleft}





\begin{flushleft}
MSL835 Labor Legislation and Industrial Relations
\end{flushleft}


\begin{flushleft}
3 Credits (3-0-0)
\end{flushleft}


\begin{flushleft}
Pre-requisites: MSL301 \& MSL302
\end{flushleft}


\begin{flushleft}
Module I : Introduction of industrial relation and a systematic view of
\end{flushleft}


\begin{flushleft}
personnel. Labour Relations. Introduction to Indian Trade Unionism.
\end{flushleft}


\begin{flushleft}
Industrial relations and conflict in industries. Introduction of Labour
\end{flushleft}


\begin{flushleft}
Regulation Act, Factories Act, Trade Union Act, and Safety Act.
\end{flushleft}


\begin{flushleft}
Module II : Role of Industrial Legislation. Introduction of Industrial
\end{flushleft}


\begin{flushleft}
Dispute Act. Different jurisdiction of Labour Court. Issues in recognition
\end{flushleft}


\begin{flushleft}
of unions. Tribunal and national tribunal. Strategies for resolving
\end{flushleft}


\begin{flushleft}
Industrial Conflict, Collective bargaining. Works committee and joint
\end{flushleft}


\begin{flushleft}
consultative committee, Negotiation process.
\end{flushleft}


\begin{flushleft}
Module III : Influence of Government regulations. Third party
\end{flushleft}


\begin{flushleft}
intervention in industrial disputes. Rules of grievances. Discipline in
\end{flushleft}


\begin{flushleft}
Industry. Contribution of tripartite bodies. Labour Welfare Participative
\end{flushleft}


\begin{flushleft}
Management. Workman's Compensation Act. Productivity in Industry.
\end{flushleft}


\begin{flushleft}
Healthy industrial relations and economic development.
\end{flushleft}





\begin{flushleft}
MSL836 International Human Resources Management
\end{flushleft}


\begin{flushleft}
1.5 Credits (1.5-0-0)
\end{flushleft}


\begin{flushleft}
Pre-requisites: MSL301 \& MSL302
\end{flushleft}


\begin{flushleft}
The course would cover issues pertaining to selecting, managing and
\end{flushleft}


\begin{flushleft}
developing international workforce. It would sensitize students to the
\end{flushleft}


\begin{flushleft}
cross-cultural issues faced by global organizations and emerging issues
\end{flushleft}


\begin{flushleft}
within international HRM.
\end{flushleft}


\begin{flushleft}
Lectures, small group discussions and case study analysis would be
\end{flushleft}


\begin{flushleft}
the primary teaching methods adopted in this course.
\end{flushleft}





\begin{flushleft}
MSL839 Current and Emerging Issues in Organizational
\end{flushleft}


\begin{flushleft}
Management
\end{flushleft}


\begin{flushleft}
3 Credits (3-0-0)
\end{flushleft}


\begin{flushleft}
Pre-requisites: MSL301 \& MSL302
\end{flushleft}


\begin{flushleft}
(Relevant current and Emerging Issues)
\end{flushleft}





\begin{flushleft}
MSL840 Manufacturing Strategy
\end{flushleft}


\begin{flushleft}
3 Credits (3-0-0)
\end{flushleft}


\begin{flushleft}
Pre-requisites: MSL301 \& MSL302
\end{flushleft}


\begin{flushleft}
Module I : Manufacturing and operations strategy-relevance and
\end{flushleft}


\begin{flushleft}
concepts. Strategic issues in manufacturing \& operations, Capacity
\end{flushleft}


\begin{flushleft}
planning, International innovations in manufacturing. Choice of
\end{flushleft}


\begin{flushleft}
technology and manufacturing process in the prevailing environment.
\end{flushleft}


\begin{flushleft}
Module II : Technology-manufacturing process interfaces with
\end{flushleft}


\begin{flushleft}
marketing, engineering, quality, purchasing, finance and accounting.
\end{flushleft}


\begin{flushleft}
Inter-relationship among manufacturing manager and their suppliers,
\end{flushleft}


\begin{flushleft}
customers, competitors, superiors and production workers.
\end{flushleft}


\begin{flushleft}
Module III : Strategic implications of Experience Curve. Focused
\end{flushleft}


\begin{flushleft}
manufacturing-green, lean and mean. Strategic issues in project
\end{flushleft}


\begin{flushleft}
management and implementation of manufacturing policies.
\end{flushleft}


\begin{flushleft}
Perspectives of Manufacturing Strategy. Case Studies.
\end{flushleft}





\begin{flushleft}
MSL841 Supply Chain Analytics
\end{flushleft}


\begin{flushleft}
3 Credits (3-0-0)
\end{flushleft}


\begin{flushleft}
Pre--requisites: MSL301 \& MSL302
\end{flushleft}


\begin{flushleft}
This course will introduce students to supply chain analytics by learning
\end{flushleft}


\begin{flushleft}
the three aspects of supply chain planning and design. The first one is
\end{flushleft}


\begin{flushleft}
Descriptive Analytics of supply chain, which focuses on fundamental
\end{flushleft}


\begin{flushleft}
tools and methods on data analysis and statistics, visual representations
\end{flushleft}


\begin{flushleft}
of data and data modeling. The second major focus is on Predictive
\end{flushleft}


\begin{flushleft}
Analytics of supply chain, which develops approaches for building
\end{flushleft}


\begin{flushleft}
and analyzing predictive models, applying regression, forecasting
\end{flushleft}


\begin{flushleft}
techniques, simulation and risk analysis, etc. The third major focus
\end{flushleft}


\begin{flushleft}
is on Prescriptive Analytics of supply chain, which aims at arriving at
\end{flushleft}


\begin{flushleft}
optimal decisions for the different future scenarios in the supply chain.
\end{flushleft}





\begin{flushleft}
MSL842 Supply Chain Modeling
\end{flushleft}


\begin{flushleft}
3 Credits (3-0-0)
\end{flushleft}


\begin{flushleft}
Pre--requisites: MSL301 \& MSL302
\end{flushleft}


\begin{flushleft}
This course primarily deals with understanding and analyzing problems
\end{flushleft}


\begin{flushleft}
underlying the design, planning and operation of supply chains, with
\end{flushleft}


\begin{flushleft}
a special emphasis on the logistical and other issues related to the
\end{flushleft}


\begin{flushleft}
material and the information flow in these systems. The main objective
\end{flushleft}


\begin{flushleft}
of the course is to introduce methodological description of the various
\end{flushleft}


\begin{flushleft}
issues in supply chain related to design, planning and control problems.
\end{flushleft}


\begin{flushleft}
Mathematical models and techniques are used to support the analysis
\end{flushleft}


\begin{flushleft}
of the identified issues. It also develops understanding of some basic
\end{flushleft}


\begin{flushleft}
tools that can support the functioning of the analytical methodologies.
\end{flushleft}





\begin{flushleft}
MSL843 Supply Chain Logistics Management
\end{flushleft}


\begin{flushleft}
3 Credits (3-0-0)
\end{flushleft}


\begin{flushleft}
Pre-requisites: MSL301 \& MSL302
\end{flushleft}


\begin{flushleft}
Module I : Perspective of Supply Chain Logistics Management. Logistics
\end{flushleft}


\begin{flushleft}
concept, role and scope; Logistics Environment- Integrating Logistics
\end{flushleft}


\begin{flushleft}
of Supply, Logistics of Production and Logistics of Distribution. Internal
\end{flushleft}


\begin{flushleft}
and external factors for logistics strategy, Operational Resources
\end{flushleft}


\begin{flushleft}
of logistics (personnel, warehouse means of transport, warehouse
\end{flushleft}


\begin{flushleft}
transport aids, organizational aids, material stocks, and area/
\end{flushleft}


\begin{flushleft}
spare) Effective supply chain management, customer networking
\end{flushleft}


\begin{flushleft}
and manufacturing, Risk Pooling, Postponement, cross docking in
\end{flushleft}


\begin{flushleft}
supply chain, CPFR, IT-enabled supply chains value of Information,
\end{flushleft}


\begin{flushleft}
Coordination in SCM.
\end{flushleft}


\begin{flushleft}
Module II : Logistics Activity Mix. JIT and Logistics, Synchronised
\end{flushleft}


\begin{flushleft}
manufacturing. Purchasing and Materials Management. Distributional
\end{flushleft}


\begin{flushleft}
logistical systems and facilities-single stage or multistage,
\end{flushleft}


\begin{flushleft}
warehouse(s), their number, location and allocation, Automated
\end{flushleft}


\begin{flushleft}
Warehousing, Materials Handling and Packaging. Simulation aided
\end{flushleft}


\begin{flushleft}
planning of conveyor and warehousing systems.
\end{flushleft}


\begin{flushleft}
Module III : Supply Chain Logistics Mix Management. Logistical
\end{flushleft}


\begin{flushleft}
Connectivity: Transportation modes, rate structure, legal aspects;
\end{flushleft}


\begin{flushleft}
maintenance, spares and repairs; test and support equipment, Routing
\end{flushleft}


\begin{flushleft}
of freight flows. Management and Organization of the Logistics Systems;
\end{flushleft}


\begin{flushleft}
Organization, Information and cost control; Logistical information
\end{flushleft}


\begin{flushleft}
Systems, Computer aided logistics management. Case Studies.
\end{flushleft}





239





\begin{flushleft}
\newpage
Management Studies
\end{flushleft}





\begin{flushleft}
MSL844 Systems Reliability, Safety and Maintenance
\end{flushleft}


\begin{flushleft}
Management
\end{flushleft}


\begin{flushleft}
3 Credits (3-0-0)
\end{flushleft}





\begin{flushleft}
MSL847 Advanced Methods for Management Research
\end{flushleft}


\begin{flushleft}
3 Credits (3-0-0)
\end{flushleft}


\begin{flushleft}
Pre--requisites: MSL301 \& MSL302
\end{flushleft}





\begin{flushleft}
Pre-requisites: MSL301 \& MSL302
\end{flushleft}





\begin{flushleft}
Introduction to management research, types of management research,
\end{flushleft}


\begin{flushleft}
research designs, Portfolio of management research methodologies
\end{flushleft}


\begin{flushleft}
involving qualitative and quantitative tools, optimization approaches,
\end{flushleft}


\begin{flushleft}
Multi-criteria decision making tools, case studies, interpretative models,
\end{flushleft}


\begin{flushleft}
soft system methodology, simulation, etc. Design of a questionnairebased survey instrument, development of data measurement, scale
\end{flushleft}


\begin{flushleft}
development, testing the validity and reliability of data, sampling
\end{flushleft}


\begin{flushleft}
techniques, descriptive statistical analysis, inferential analysis,
\end{flushleft}


\begin{flushleft}
sampling techniques, sampling distribution, hypothesis testing,
\end{flushleft}


\begin{flushleft}
ANOVA, factor analysis, correlation, regression : OLS, Logic, Tobit,
\end{flushleft}


\begin{flushleft}
Probit, Discriminant analysis, Co-integration, unit root testing, Granger,
\end{flushleft}


\begin{flushleft}
causality, VAR, GARCH and its variants. Structural equation modelling
\end{flushleft}


\begin{flushleft}
and other related research tools. Portfolio of optimization tools such
\end{flushleft}


\begin{flushleft}
as linear programming, goal programming, integer programming,
\end{flushleft}


\begin{flushleft}
Data Envelopment Analysis for designing a management research.
\end{flushleft}


\begin{flushleft}
Case study approach with SWOT, SAP-LAP, value chain, PEST, etc.
\end{flushleft}


\begin{flushleft}
AHP, ANP modeling of risk and uncertainty in management, real life
\end{flushleft}


\begin{flushleft}
case development with appropriate research design.
\end{flushleft}





\begin{flushleft}
Module I : Reliability, Safety, Risk Assessment Perspective.
\end{flushleft}


\begin{flushleft}
Introduction to reliability, availability and safety engineering and
\end{flushleft}


\begin{flushleft}
management. Select statistical concepts and probability distributions.
\end{flushleft}


\begin{flushleft}
Optimization techniques for systems reliability, availability and safety.
\end{flushleft}


\begin{flushleft}
Reliability, availability, safety and maintainability. Risk assessment and
\end{flushleft}


\begin{flushleft}
management for reliability and safety.
\end{flushleft}


\begin{flushleft}
Module II : Maintenance Planning and Control.
\end{flushleft}


\begin{flushleft}
Maintenance management objectives and functions. Classification of
\end{flushleft}


\begin{flushleft}
Maintenance system. Maintenance Planning and Scheduling. Issues of
\end{flushleft}


\begin{flushleft}
Replacement versus reconditioning and imperfect repair maintenance
\end{flushleft}


\begin{flushleft}
models. Spare parts Inventory Planning and Control for single and
\end{flushleft}


\begin{flushleft}
multi-echelon systems. Diagnostic tools of failure analysis: Failure
\end{flushleft}


\begin{flushleft}
Mode Effect and Criticality Analysis, Fault Tree Analysis.
\end{flushleft}


\begin{flushleft}
Module III : Information System for Reliability, Safety and Maintenance
\end{flushleft}


\begin{flushleft}
Management.
\end{flushleft}


\begin{flushleft}
Organizational aspects and a computer aided management information
\end{flushleft}


\begin{flushleft}
system for reliability, safety and maintenance. Life cycle costing and
\end{flushleft}


\begin{flushleft}
cost management for maintenance. Human factors in maintenance,
\end{flushleft}


\begin{flushleft}
Maintenance Manpower Planning. Case Studies.
\end{flushleft}





\begin{flushleft}
MSL845 Total Project Systems Management
\end{flushleft}


\begin{flushleft}
3 Credits (3-0-0)
\end{flushleft}


\begin{flushleft}
Pre--requisites: MSL301 \& MSL302
\end{flushleft}


\begin{flushleft}
Module I: Project Systems Management: a life cycle approach, project
\end{flushleft}


\begin{flushleft}
characteristics; project life cycle phases: conception, definition,
\end{flushleft}


\begin{flushleft}
planning and organising, implementation and project clean up. Project
\end{flushleft}


\begin{flushleft}
feasibility analysis. The project manager: role and responsibilities,
\end{flushleft}


\begin{flushleft}
Team Building and Conflict Management. Tools and techniques for
\end{flushleft}


\begin{flushleft}
project management. Environmental impact analysis of a project.
\end{flushleft}


\begin{flushleft}
Module II: Network techniques for project management-PERT, CPM
\end{flushleft}


\begin{flushleft}
and GERT. Accounting for risk, uncertainty and fuzziness. Time
\end{flushleft}


\begin{flushleft}
cost tradeoffs and crashing procedures. Multi project planning and
\end{flushleft}


\begin{flushleft}
scheduling with limited resources. Multi objective, fuzzy and stochastic
\end{flushleft}


\begin{flushleft}
based formulations in a project environment.
\end{flushleft}


\begin{flushleft}
Module III: Funds planning, performance budgeting and control.
\end{flushleft}


\begin{flushleft}
Project materials management. Pricing, estimating, and Contract
\end{flushleft}


\begin{flushleft}
Administration and Management, Building and Bid evaluation and
\end{flushleft}


\begin{flushleft}
analysis. Project implementation and monitoring, Project management
\end{flushleft}


\begin{flushleft}
information and control systems. Project systems management
\end{flushleft}


\begin{flushleft}
performance indices. Software Packages application for Project
\end{flushleft}


\begin{flushleft}
Systems Management. Case studies.
\end{flushleft}





\begin{flushleft}
MSL846 Total Productivity Management
\end{flushleft}


\begin{flushleft}
3 Credits (3-0-0)
\end{flushleft}


\begin{flushleft}
Pre--requisites: MSL301 \& MSL302
\end{flushleft}


\begin{flushleft}
Module I: Total Productivity overview; meaning, relevance and scope
\end{flushleft}


\begin{flushleft}
for productivity and effectiveness. Productivity conceptualisation.
\end{flushleft}


\begin{flushleft}
Productivity mission, objectives, policies and strategies. Productivity
\end{flushleft}


\begin{flushleft}
environment. Corporate culture, management styles, employees
\end{flushleft}


\begin{flushleft}
participation, trade unions and role of governmental agencies.
\end{flushleft}


\begin{flushleft}
Productivity measurement, monitoring and management both at micro
\end{flushleft}


\begin{flushleft}
and macro levels. Corporate and annual productivity plans.
\end{flushleft}


\begin{flushleft}
Module II: Benchmarking: Management issues, modelling, tools and
\end{flushleft}


\begin{flushleft}
techniques; indicators for evaluation of manufacturing, business or
\end{flushleft}


\begin{flushleft}
services organizational performance and its measurement.
\end{flushleft}


\begin{flushleft}
Module III: Productivity Improvement Techniques: modifying
\end{flushleft}


\begin{flushleft}
organizational characteristics and work characteristics. Work study,
\end{flushleft}


\begin{flushleft}
Value Engineering, Waste Management. Human resource development
\end{flushleft}


\begin{flushleft}
strategies to increase productivity. Managing technological change.
\end{flushleft}


\begin{flushleft}
Interfaces of Productivity with Quality, Reliability and Safety. Management
\end{flushleft}


\begin{flushleft}
commitment and involvement for higher productivity. Case Studies.
\end{flushleft}





\begin{flushleft}
MSL848 Applied Operations Research
\end{flushleft}


\begin{flushleft}
3 Credits (3-0-0)
\end{flushleft}


\begin{flushleft}
Pre--requisites: MSL301 \& MSL302
\end{flushleft}


\begin{flushleft}
The objective of the course is to provide the students about the
\end{flushleft}


\begin{flushleft}
application of Operations Research (OR) in various functional areas
\end{flushleft}


\begin{flushleft}
of business such as operation, supply chain management, materials
\end{flushleft}


\begin{flushleft}
management, marketing, finance, and human resource. The entire
\end{flushleft}


\begin{flushleft}
course is a case based where the participants will be given a case.
\end{flushleft}


\begin{flushleft}
Participants will be asked to tackle the case problem without using
\end{flushleft}


\begin{flushleft}
OR using their own experience or any other logical method and then
\end{flushleft}


\begin{flushleft}
they will be asked to tackle the same situation applying OR. At the
\end{flushleft}


\begin{flushleft}
end of the course they will appreciate how OR can help the decision
\end{flushleft}


\begin{flushleft}
makers in an efficient decision making process.
\end{flushleft}





\begin{flushleft}
MSL849 Current and Emerging Issues in Manufacturing
\end{flushleft}


\begin{flushleft}
Management
\end{flushleft}


\begin{flushleft}
3 Credits (3-0-0)
\end{flushleft}


\begin{flushleft}
Pre--requisites: MSL301 \& MSL302
\end{flushleft}


\begin{flushleft}
(Relevant current and Emerging Issues)
\end{flushleft}





\begin{flushleft}
MSL850 Management of Information Technology
\end{flushleft}


\begin{flushleft}
3 Credits (3-0-0)
\end{flushleft}


\begin{flushleft}
Pre--requisites: MSL301 \& MSL302
\end{flushleft}


\begin{flushleft}
Module I : The Strategic Framework for IT Management. Emerging
\end{flushleft}


\begin{flushleft}
information technologies: IT for competitive advantage; IT for internal
\end{flushleft}


\begin{flushleft}
effectiveness; IT for inter- organizational linkage;Module II Strategy
\end{flushleft}


\begin{flushleft}
Development and Planning Techniques.
\end{flushleft}


\begin{flushleft}
Module II : IT Planning (CSFs, Scenario analysis, Linkage analysis, Enterprise
\end{flushleft}


\begin{flushleft}
modeling); Strategy formulation techniques; Nolan's stage model and
\end{flushleft}


\begin{flushleft}
revised models for Nolan's stages; IT investment decisions; methods
\end{flushleft}


\begin{flushleft}
for evaluating IT effectiveness; IT enabled business process redesign.
\end{flushleft}


\begin{flushleft}
Module III : Strategic Issues Related to IT Management. Relating IT
\end{flushleft}


\begin{flushleft}
to organizational leadership, culture, structure, policy and strategy;
\end{flushleft}


\begin{flushleft}
programmer productivity; Managing legacy systems; evaluating
\end{flushleft}


\begin{flushleft}
centralization- issues; IT-forecasting.
\end{flushleft}





\begin{flushleft}
MSL851 Strategic Alliance
\end{flushleft}


\begin{flushleft}
1.5 Credits (1.5-0-0)
\end{flushleft}


\begin{flushleft}
Pre--requisites: MSL301 \& MSL302
\end{flushleft}


\begin{flushleft}
This is an advanced strategy course that focuses on the role of strategic
\end{flushleft}


\begin{flushleft}
alliances and inter-firm networks in the overall strategic adaptation
\end{flushleft}


\begin{flushleft}
process of the firm. Inter-firm networks and strategic alliances have
\end{flushleft}


\begin{flushleft}
emerged as important strategic options for navigating survival and
\end{flushleft}


\begin{flushleft}
creating competitive advantage in times of high velocity turbulent
\end{flushleft}


\begin{flushleft}
environments characterized by pressures to master rapid technological
\end{flushleft}


\begin{flushleft}
developments, counteract new competitors and the never ending need
\end{flushleft}





240





\begin{flushleft}
\newpage
Management Studies
\end{flushleft}





\begin{flushleft}
to acquire and master new capabilities (technical and managerial).
\end{flushleft}


\begin{flushleft}
The course explores crucial success factors that distinguish successful
\end{flushleft}


\begin{flushleft}
from failing strategic alliances. The course utilizes case discussions
\end{flushleft}


\begin{flushleft}
supplemented with readings, lectures, and conceptual discussions.
\end{flushleft}





\begin{flushleft}
MSL852 Network System: Applications and Management
\end{flushleft}


\begin{flushleft}
3 Credits (3-0-0)
\end{flushleft}


\begin{flushleft}
Pre--requisites: MSL301 \& MSL302
\end{flushleft}


\begin{flushleft}
Module I : Networking fundamentals. Communication fundamentals
\end{flushleft}


\begin{flushleft}
(transmission and transmission media; communication techniques;
\end{flushleft}


\begin{flushleft}
transmission efficiency) Wide area networks, local area networks, ISDNs;
\end{flushleft}


\begin{flushleft}
OSI architecture, IBM's SNA, Digitals DNA, Internetworking; network
\end{flushleft}


\begin{flushleft}
applications- EDI, Email, file transfer, conferencing, Enterprise networking.
\end{flushleft}


\begin{flushleft}
Module II : Networking technologies and applications. Design
\end{flushleft}


\begin{flushleft}
and development of enterprise network; Web-based application
\end{flushleft}


\begin{flushleft}
development, Design of large-scale intranets, Network and systems
\end{flushleft}


\begin{flushleft}
management issues, Remote access to computer resources, Network
\end{flushleft}


\begin{flushleft}
and system security.
\end{flushleft}


\begin{flushleft}
Module III : Managing networks. Preparing for doing business on
\end{flushleft}


\begin{flushleft}
the internet; Choosing and costing networks and network services;
\end{flushleft}


\begin{flushleft}
network management requirements; network performance indicators;
\end{flushleft}


\begin{flushleft}
performance monitoring.
\end{flushleft}





\begin{flushleft}
MSL853 Software Project Management
\end{flushleft}


\begin{flushleft}
3 Credits (3-0-0)
\end{flushleft}


\begin{flushleft}
Pre--requisites: MSL301 \& MSL302
\end{flushleft}


\begin{flushleft}
This course may expose the participants to the following topics: IT
\end{flushleft}


\begin{flushleft}
Evolution and its implications for business, IT Productivity Paradox
\end{flushleft}


\begin{flushleft}
- Issues and Implications, Impact of IS in the Networked Economy,
\end{flushleft}


\begin{flushleft}
Reasons for success and failure of IT projects, Disaster planning,
\end{flushleft}


\begin{flushleft}
Approaches to IS Development (e.g. Portfolio approaches), Technology
\end{flushleft}


\begin{flushleft}
Justification and Alignment Models, Strategic impact of IT / IS, Role of
\end{flushleft}


\begin{flushleft}
the CIO and challenges in business continuity.
\end{flushleft}





\begin{flushleft}
MSL854 Big Data Analytics \& Data Science
\end{flushleft}


\begin{flushleft}
1.5 Credits (1.5-0-0)
\end{flushleft}


\begin{flushleft}
Pre--requisites: MSL301 \& MSL302
\end{flushleft}


\begin{flushleft}
This course may expose the student to the following themes within the
\end{flushleft}


\begin{flushleft}
discipline: Introduction to Data Science and Data Scientists, Introduction
\end{flushleft}


\begin{flushleft}
to Big Data, Theories in Data Science, Big data technologies, Large
\end{flushleft}


\begin{flushleft}
query data sets and associated theories, Exploring the Hadoop
\end{flushleft}


\begin{flushleft}
Ecosystem, Information management in Big Data and Emerging Issues.
\end{flushleft}





\begin{flushleft}
MSL855 Electronic Commerce
\end{flushleft}


\begin{flushleft}
3 Credits (3-0-0)
\end{flushleft}


\begin{flushleft}
Pre--requisites: MSL301 \& MSL302
\end{flushleft}


\begin{flushleft}
This course may expose the participants to the following topics:
\end{flushleft}


\begin{flushleft}
Introduction to e-commerce, B2B E-commerce models,B2C
\end{flushleft}


\begin{flushleft}
E-Commerce models, Mercantile processes, E-Commerce Infrastructure
\end{flushleft}


\begin{flushleft}
and Capacity Planning, Web Portals \& Services, Trading, Pricing,
\end{flushleft}


\begin{flushleft}
Auctions, Bartering \& Negotiations, Advanced and emergent topics in
\end{flushleft}


\begin{flushleft}
E-Commerce, Inter-organization information systems, e-procurement
\end{flushleft}


\begin{flushleft}
systems, e-fulfillment systems, e-SCM, Risk management in
\end{flushleft}


\begin{flushleft}
E-commerce. Hands on training may also be provided.
\end{flushleft}





\begin{flushleft}
MSL856 Business Intelligence
\end{flushleft}


\begin{flushleft}
3 Credits (3-0-0)
\end{flushleft}


\begin{flushleft}
Pre--requisites: MSL301 \& MSL302
\end{flushleft}


\begin{flushleft}
The course will consist of the following: Introduction to data
\end{flushleft}


\begin{flushleft}
mining, types of data mining systems, data preprocessing and
\end{flushleft}


\begin{flushleft}
data warehouses (OLAP/OLTP), Data Cube Computation and Data
\end{flushleft}


\begin{flushleft}
Generalization, Mining Frequent Patterns, Associations, Correlations,
\end{flushleft}


\begin{flushleft}
Classification, Prediction, Clustering, time series and sequence data
\end{flushleft}


\begin{flushleft}
analysis, Graph Mining, Social Network Analysis, and Multirelational
\end{flushleft}


\begin{flushleft}
Data Mining, Mining Object, Spatial, Multimedia, Text, and Web Data,
\end{flushleft}


\begin{flushleft}
Applications and trends.
\end{flushleft}





\begin{flushleft}
MSL858 Business Process Management with IT
\end{flushleft}


\begin{flushleft}
1.5 Credits (1.5-0-0)
\end{flushleft}


\begin{flushleft}
Pre--requisites: MSL301 \& MSL302
\end{flushleft}


\begin{flushleft}
This course may expose the participants to the following topics:
\end{flushleft}


\begin{flushleft}
Concepts of process and business process, Processes and workflow
\end{flushleft}


\begin{flushleft}
management systems, Concepts and evolution of BPM technologies,
\end{flushleft}


\begin{flushleft}
Impact of IT in BPM and its road map, BPM Cycle, Process deployment,
\end{flushleft}


\begin{flushleft}
Process monitoring, Process optimization using IT tools, Flowcharting
\end{flushleft}


\begin{flushleft}
and business process mapping and emergent issues in BPM/BPR
\end{flushleft}


\begin{flushleft}
technologies. Other relevant topics within the subject domain may
\end{flushleft}


\begin{flushleft}
also be explored.
\end{flushleft}





\begin{flushleft}
MSL859 Current and Emerging Issues in IT Management
\end{flushleft}


\begin{flushleft}
3 Credits (3-0-0)
\end{flushleft}


\begin{flushleft}
Pre--requisites: MSL301 \& MSL302
\end{flushleft}


\begin{flushleft}
(Relevant current and Emerging Issues)
\end{flushleft}





\begin{flushleft}
MSL861 Market Research
\end{flushleft}


\begin{flushleft}
3 Credits (3-0-0)
\end{flushleft}


\begin{flushleft}
Pre--requisites: MSL301 \& MSL302
\end{flushleft}


\begin{flushleft}
Module I: Research concepts; exploratory, descriptive and conclusive
\end{flushleft}


\begin{flushleft}
research. The market decision-making process and the need of
\end{flushleft}


\begin{flushleft}
different types of research. Types of marketing problems and type
\end{flushleft}


\begin{flushleft}
of marketing research activity. Sources of data; use and appraisal of
\end{flushleft}


\begin{flushleft}
existing information.
\end{flushleft}


\begin{flushleft}
Module II : Information from respondents, sampling design, scaling
\end{flushleft}


\begin{flushleft}
techniques and questionnaire design, interviewing, mail surveys.
\end{flushleft}


\begin{flushleft}
Information from experiment, experimental design for marketing,
\end{flushleft}


\begin{flushleft}
Movtivational research, Advertising research, Analysis and reporting.
\end{flushleft}


\begin{flushleft}
Module III : Marketing information systems, Structure and design, its
\end{flushleft}


\begin{flushleft}
role in planning and control; the place of marketing research.
\end{flushleft}





\begin{flushleft}
MSL862 Product Management
\end{flushleft}


\begin{flushleft}
3 Credits (3-0-0)
\end{flushleft}


\begin{flushleft}
Pre--requisites: MSL301 \& MSL302
\end{flushleft}


\begin{flushleft}
Module I : The product in corporate life, Corporate and product
\end{flushleft}


\begin{flushleft}
objective, product management role, responsibility, scope and functions,
\end{flushleft}


\begin{flushleft}
product strategy and policy, optimum product pattern/line range.
\end{flushleft}


\begin{flushleft}
Module II : New product development and launching. Challenge of
\end{flushleft}


\begin{flushleft}
change-opportunity and risk-product innovation, modification, addition
\end{flushleft}


\begin{flushleft}
and elimination product proposals-sources, generation, processing and
\end{flushleft}


\begin{flushleft}
selection. Establishing techno-economic feasibility product testing and
\end{flushleft}


\begin{flushleft}
test marketing. Developing the strategy and the plan. Implementing
\end{flushleft}


\begin{flushleft}
the plan, coordination and control. Brand identity, Image, Equity,
\end{flushleft}


\begin{flushleft}
Brand Plan and Management, New Product Development Process.
\end{flushleft}


\begin{flushleft}
Brand and Product launch plan.
\end{flushleft}


\begin{flushleft}
Module III : Organization for Product Management, Marketing
\end{flushleft}


\begin{flushleft}
manager-product manager-brand manager concept, approaches and
\end{flushleft}


\begin{flushleft}
organizational role, product manager-functions and tasks-tools and
\end{flushleft}


\begin{flushleft}
techniques. Brand extensions, acquisitions, Brand value, Consumer
\end{flushleft}


\begin{flushleft}
insight. Strategies brand management.
\end{flushleft}





\begin{flushleft}
MSL863 Advertising and Sales Promotion Management
\end{flushleft}


\begin{flushleft}
3 Credits (3-0-0)
\end{flushleft}


\begin{flushleft}
Pre--requisites: MSL301 \& MSL302
\end{flushleft}


\begin{flushleft}
Module I : Mass communication theory and practices, marketing and
\end{flushleft}


\begin{flushleft}
promotion mix- interrelationship and interdependence advertising.
\end{flushleft}


\begin{flushleft}
Sales Promotion, Publicity and Public Relations- Scope, Objectives,
\end{flushleft}


\begin{flushleft}
activities and creative role. Advertising, objectives tasks and
\end{flushleft}


\begin{flushleft}
process, market segmentation and target audience- Message and
\end{flushleft}


\begin{flushleft}
copy development. Mass media, selection, planning, budgeting
\end{flushleft}


\begin{flushleft}
and scheduling. Integrated programme and budget planning.
\end{flushleft}


\begin{flushleft}
Implementing the programme, coordination and control. Advertising
\end{flushleft}


\begin{flushleft}
Agencies in India, their services and terms, advertisement campaign
\end{flushleft}


\begin{flushleft}
development, Agency selection and appointment; Agency Organization
\end{flushleft}


\begin{flushleft}
and operation, Getting the best of the agency services. Analysis of
\end{flushleft}


\begin{flushleft}
effectiveness of advertisement and promotional campaign.
\end{flushleft}





241





\begin{flushleft}
\newpage
Management Studies
\end{flushleft}





\begin{flushleft}
Module II : Why and when sales promotion support, Sales promotion
\end{flushleft}


\begin{flushleft}
activities; Consumer Oriented-Sales channel Oriented-Sales staff
\end{flushleft}


\begin{flushleft}
oriented, Planning, budgeting, implementing and controlling
\end{flushleft}


\begin{flushleft}
campaigns. Advertisement development brief.
\end{flushleft}


\begin{flushleft}
Module III : Valuation and measurement of advertising and sales
\end{flushleft}


\begin{flushleft}
promotion effectiveness, Company organization for advertising: sales
\end{flushleft}


\begin{flushleft}
manager, Sales Promotion Manager, Market Development ManagerRole of Tasks, advertising ethics, economics and social relevance. The
\end{flushleft}


\begin{flushleft}
Public Relations Activities, Public relations and mass media. Media
\end{flushleft}


\begin{flushleft}
planning and budgeting control.
\end{flushleft}





\begin{flushleft}
trade, Identifying geographical territories for expansion. Cultural
\end{flushleft}


\begin{flushleft}
factors affecting business in global market.
\end{flushleft}


\begin{flushleft}
Module III : Export credit system preshipment and post-shipment,
\end{flushleft}


\begin{flushleft}
finance, medium and long term credit financing; ECGC; Transportation
\end{flushleft}


\begin{flushleft}
and shipment of cargo; Marine insurance of cargo; procedure for
\end{flushleft}


\begin{flushleft}
claiming rebate of excise duty. Import replenishment licensing
\end{flushleft}


\begin{flushleft}
procedures. Generalized scheme of preferences. Sourcing and Transfer
\end{flushleft}


\begin{flushleft}
pricing mechanism. WTO related issues and IPR related issues
\end{flushleft}


\begin{flushleft}
impacting global trade.
\end{flushleft}





\begin{flushleft}
MSL867 Industrial Marketing Management
\end{flushleft}


\begin{flushleft}
3 Credits (3-0-0)
\end{flushleft}


\begin{flushleft}
Pre--requisites: MSL301 \& MSL302
\end{flushleft}





\begin{flushleft}
MSL864 Corporate Communication
\end{flushleft}


\begin{flushleft}
3 Credits (3-0-0)
\end{flushleft}


\begin{flushleft}
Pre--requisites: MSL301 \& MSL302
\end{flushleft}


\begin{flushleft}
Corporate communications is a strategic tool that is leveraged to
\end{flushleft}


\begin{flushleft}
gain strategic advantage. Organizations use it to lead, motivate,
\end{flushleft}


\begin{flushleft}
persuade and inform both employees and outside stakeholders.
\end{flushleft}


\begin{flushleft}
How organizations set objectives, define messages and reach their
\end{flushleft}


\begin{flushleft}
employees, extended audiences, the media and customers, and how
\end{flushleft}


\begin{flushleft}
the company or group articulates its vision and brings its values to
\end{flushleft}


\begin{flushleft}
life, will all be discussed.
\end{flushleft}


\begin{flushleft}
The course will familiarize students with some of the issues that
\end{flushleft}


\begin{flushleft}
specifically affect organisations and challenge the corporate
\end{flushleft}


\begin{flushleft}
communications function. Some of these issues include a change in
\end{flushleft}


\begin{flushleft}
CEO, mergers and acquisitions, imposition of government regulation
\end{flushleft}


\begin{flushleft}
and public pressure groups. Focus will be placed on crafting corporate
\end{flushleft}


\begin{flushleft}
messages for internal and external stakeholders. Specific subject.
\end{flushleft}





\begin{flushleft}
MSL865 Sales Management
\end{flushleft}


\begin{flushleft}
3 Credits (3-0-0)
\end{flushleft}


\begin{flushleft}
Pre--requisites: MSL301 \& MSL302
\end{flushleft}


\begin{flushleft}
Module I : Organisational framework of the field sales force. Types and
\end{flushleft}


\begin{flushleft}
methods of field sales organisations-Career in Field Sales Management.
\end{flushleft}


\begin{flushleft}
Field Sales Manager- coordinating and controlling the Marketing
\end{flushleft}


\begin{flushleft}
mix, Tasks and responsibilities, team relations with Salesman and
\end{flushleft}


\begin{flushleft}
interaction and reporting relationship with Top Management. Operating
\end{flushleft}


\begin{flushleft}
environment for Field Sales Managers. Sales forecasting.
\end{flushleft}


\begin{flushleft}
Module II : Sales Information and Planning, The qualities and role of
\end{flushleft}


\begin{flushleft}
a Field Sales Manager- Hierarchy of objectives and goals, concept of
\end{flushleft}


\begin{flushleft}
sales strategies and tactics; types of Planning. Marketing Intelligence
\end{flushleft}


\begin{flushleft}
and Sales Management. Relationship and contribution of Marketing
\end{flushleft}


\begin{flushleft}
Research to the sales development as decision making process.
\end{flushleft}


\begin{flushleft}
Designing and planning of sales territories, procedure for designing
\end{flushleft}


\begin{flushleft}
sales territories. Determining sales manpower requirements to
\end{flushleft}


\begin{flushleft}
establish sales territories- Recruiting salesman- selection process and
\end{flushleft}


\begin{flushleft}
system. Distribution and chamel selection \& Management.
\end{flushleft}


\begin{flushleft}
Module III : Operational Management, Staffing: Its advantages,
\end{flushleft}


\begin{flushleft}
responsibility for staffing, tools and methods of selection. Sales
\end{flushleft}


\begin{flushleft}
training: Its objectives, programme content, Methods of training,
\end{flushleft}


\begin{flushleft}
concepts of territorial management for field sales force. Measurement
\end{flushleft}


\begin{flushleft}
and control: General considerations governing evaluation and sales
\end{flushleft}


\begin{flushleft}
performance and control. Sales audit, Sales budgeting, Key account
\end{flushleft}


\begin{flushleft}
management, Route Planning and control. Sales Promotion Customer
\end{flushleft}


\begin{flushleft}
relationship management.
\end{flushleft}





\begin{flushleft}
MSL866 International Marketing
\end{flushleft}


\begin{flushleft}
3 Credits (3-0-0)
\end{flushleft}


\begin{flushleft}
Pre--requisites: MSL301 \& MSL302
\end{flushleft}


\begin{flushleft}
Module I : International marketing-its scope and tasks- world
\end{flushleft}


\begin{flushleft}
economy prospects and Challenges; India's external trade. Analysis
\end{flushleft}


\begin{flushleft}
of export performance. Why all organisations cannot go global
\end{flushleft}


\begin{flushleft}
Shipping terms and international trade terms. Information needs
\end{flushleft}


\begin{flushleft}
of exports. Costing and pricing in international trade. Advantages
\end{flushleft}


\begin{flushleft}
and disadvantages of globalisation.
\end{flushleft}


\begin{flushleft}
Module II : Strategic export planning. Handling an export transaction.
\end{flushleft}


\begin{flushleft}
Export marketing Checklist; Selection of Markets: Choosing Markets;
\end{flushleft}


\begin{flushleft}
Export pricing; Management of export logistics. Documentation for
\end{flushleft}


\begin{flushleft}
export; processing of an export trade. Sales forcasting in international
\end{flushleft}





\begin{flushleft}
Module I : Industrial marketing and Environment. Application of
\end{flushleft}


\begin{flushleft}
industrial buyer behaviour theories. Marketing plan to implement the
\end{flushleft}


\begin{flushleft}
marketing concept.
\end{flushleft}


\begin{flushleft}
Module II : The new product development process. Personal selling
\end{flushleft}


\begin{flushleft}
(negotiations, systems selling, targets setting, fact finding, training);
\end{flushleft}


\begin{flushleft}
sales communications.
\end{flushleft}


\begin{flushleft}
Module III : Marketing Research for industrial product Marketing
\end{flushleft}


\begin{flushleft}
control (variance analysis audit). Industrial purchase behaviour and
\end{flushleft}


\begin{flushleft}
processes, new product launch. Forecasting methods.
\end{flushleft}





\begin{flushleft}
MSL868 Digital Research Methods
\end{flushleft}


\begin{flushleft}
1.5 Credits (1.5-0-0)
\end{flushleft}


\begin{flushleft}
Pre--requisites: MSL301 \& MSL302
\end{flushleft}


\begin{flushleft}
The course will have the following coverage: Internet as a research
\end{flushleft}


\begin{flushleft}
medium; Research design; Sampling methods; Online surveys;
\end{flushleft}


\begin{flushleft}
Nonreactive data collection; virtual ethnography; Online focus groups;
\end{flushleft}


\begin{flushleft}
secondary qualitative data analysis; blogs \& videos as source of data;
\end{flushleft}


\begin{flushleft}
data analysis approaches; tools.
\end{flushleft}





\begin{flushleft}
MSL869 Current and Emerging Issues in Marketing
\end{flushleft}


\begin{flushleft}
3 Credits (3-0-0)
\end{flushleft}


\begin{flushleft}
Pre--requisites: MSL301 \& MSL302
\end{flushleft}


\begin{flushleft}
(Relevant current and Emerging Issues)
\end{flushleft}





\begin{flushleft}
MSL870 Corporate Governance
\end{flushleft}


\begin{flushleft}
1.5 Credits (1.5-0-0)
\end{flushleft}


\begin{flushleft}
Pre--requisites: MSL301 \& MSL302
\end{flushleft}


\begin{flushleft}
The course would broadly be divided into three modules. Module
\end{flushleft}


\begin{flushleft}
1 would provide a global perspective to the students on the
\end{flushleft}


\begin{flushleft}
concept of corporate governance. Module 2 would focus on
\end{flushleft}


\begin{flushleft}
India and present the framework of corporate governance for
\end{flushleft}


\begin{flushleft}
Indian organizations. Module 3 would focus on corporate social
\end{flushleft}


\begin{flushleft}
responsibility (CSR) and its manifestations. Apart from the regular
\end{flushleft}


\begin{flushleft}
lectures and assignments, there would be a course pack provided to
\end{flushleft}


\begin{flushleft}
the students containing international and national reports, articles,
\end{flushleft}


\begin{flushleft}
studies and cases to help them build an international perspective
\end{flushleft}


\begin{flushleft}
through the self-study component.
\end{flushleft}





\begin{flushleft}
MSL871 Banking and Financial Services
\end{flushleft}


\begin{flushleft}
1.5 Credits (1.5-0-0)
\end{flushleft}


\begin{flushleft}
Pre--requisites: MSL301 \& MSL302
\end{flushleft}


\begin{flushleft}
The course will comprise of two broad sections; banking and
\end{flushleft}


\begin{flushleft}
financial services. Banking portion will cover banking sector reforms,
\end{flushleft}


\begin{flushleft}
bank management, financial statements of banks, sources and
\end{flushleft}


\begin{flushleft}
uses of bank funds, credit monitoring and management by banks,
\end{flushleft}


\begin{flushleft}
bank capital and Basel norms. Financial services will encompass
\end{flushleft}


\begin{flushleft}
both fund based and fee based services that are an integral part
\end{flushleft}


\begin{flushleft}
of modern financial systems; it will include lease financing, hire
\end{flushleft}


\begin{flushleft}
purchase financing, consumer credit, factoring, housing finance,
\end{flushleft}


\begin{flushleft}
investment banking, credit rating, stock broking, depository and
\end{flushleft}


\begin{flushleft}
custodial services. The course work will encompass problem solving
\end{flushleft}


\begin{flushleft}
on relevant topics and inputs from real life cases to give a practical
\end{flushleft}


\begin{flushleft}
insight to the theoretical concepts.
\end{flushleft}





242





\begin{flushleft}
\newpage
Management Studies
\end{flushleft}





\begin{flushleft}
MSL872 Working Capital Management
\end{flushleft}


\begin{flushleft}
3 Credits (3-0-0)
\end{flushleft}


\begin{flushleft}
Pre--requisites: MSL301 \& MSL302
\end{flushleft}





\begin{flushleft}
MSL876 Economics of Digital Business
\end{flushleft}


\begin{flushleft}
1.5 Credits (1.5-0-0)
\end{flushleft}


\begin{flushleft}
Pre--requisites: MSL301 \& MSL302
\end{flushleft}





\begin{flushleft}
Module I : Nature and Financial of Working Capital. Nature of Working
\end{flushleft}


\begin{flushleft}
Capita, Trade-off between Profitability and Risk, Determinants of
\end{flushleft}


\begin{flushleft}
Working Capital. Factoring as a Sources Finance. Forecasting Working
\end{flushleft}


\begin{flushleft}
Capital requirements. Sources of financing Working Capital. Factoring
\end{flushleft}


\begin{flushleft}
as a source of finance. Bank credit and working capital Finance.
\end{flushleft}


\begin{flushleft}
Approaches to determine Financing Mix. Working Capital Leverage.
\end{flushleft}


\begin{flushleft}
Cases and Practical Problems.
\end{flushleft}





\begin{flushleft}
This course may expose the participants to the following topics:
\end{flushleft}


\begin{flushleft}
Impact of diffusion of ICTs in Business and People,Trade-offs \&
\end{flushleft}


\begin{flushleft}
Network effects, Economics of Data communication including pricing,
\end{flushleft}


\begin{flushleft}
Firms, Networks, Centralization, Decentralization in 2 sided markets,
\end{flushleft}


\begin{flushleft}
Factors affecting organizational structure and size, Dynamics of Open
\end{flushleft}


\begin{flushleft}
Source and Open Innovation, Information, Search, Switching and Price
\end{flushleft}


\begin{flushleft}
dispersion, Information goods pricing and bundling. Other similar
\end{flushleft}


\begin{flushleft}
themes may also be explored.
\end{flushleft}





\begin{flushleft}
Module II : Current Assets Management. Cash Management, Inventory
\end{flushleft}


\begin{flushleft}
Management, Receivables Management. Cases and Practical Problems.
\end{flushleft}


\begin{flushleft}
Module III : Analysis aTools and New Development. Operating Cycle,
\end{flushleft}


\begin{flushleft}
Ratio Analysis, Funds-flow Analysis and Cash-Flow Statement as
\end{flushleft}


\begin{flushleft}
tools of Working Capital Management. Recent changes and new
\end{flushleft}


\begin{flushleft}
developments. Practical Problems.
\end{flushleft}





\begin{flushleft}
MSL873 Security Analysis \& Portfolio Management
\end{flushleft}


\begin{flushleft}
3 Credits (3-0-0)
\end{flushleft}


\begin{flushleft}
Pre--requisites: MSL301 \& MSL302
\end{flushleft}


\begin{flushleft}
Module I : Investment Environment. Saving and Financial flows,
\end{flushleft}


\begin{flushleft}
Financial Intermediation, Investment in Corporate Securities and other
\end{flushleft}


\begin{flushleft}
Investment Outlets, New Issue market and Secondary Markets. Sources
\end{flushleft}


\begin{flushleft}
of investment information. Theoretical framework for investment
\end{flushleft}


\begin{flushleft}
Decision. Regulatory Framework of Securities Markets in India.
\end{flushleft}


\begin{flushleft}
Module II : Valuation of Securities. Valuation of Variable Income
\end{flushleft}


\begin{flushleft}
Securities (Equity Shares): Theory of Valuation-Earnings and Dividend
\end{flushleft}


\begin{flushleft}
Model. Fundamental Analysis, Aggregate Economic Analysis, Industry
\end{flushleft}


\begin{flushleft}
Analysis, Company Analysis, Technical Analysis, Growth Shares, Under
\end{flushleft}


\begin{flushleft}
and Overvalued Shares. Analysis of Fixed Income Securities like
\end{flushleft}


\begin{flushleft}
Preference Shares, Debentures/Bonds and other Financial Instruments.
\end{flushleft}


\begin{flushleft}
Interest Rate structure and yield to Maturity Curve. Convertible Bonds:
\end{flushleft}


\begin{flushleft}
Warrants and Options.
\end{flushleft}


\begin{flushleft}
Module III : Portfolio Management. General principles. Measures of
\end{flushleft}


\begin{flushleft}
Risk and Return, Required Rate of Return and CAPM, Markkowitz
\end{flushleft}


\begin{flushleft}
Portfolio Theory. Efficient Capital Market Theory. Alternative Efficient
\end{flushleft}


\begin{flushleft}
Market Hypotheses. Constructing the Optimum Portfolio.
\end{flushleft}





\begin{flushleft}
MSL874 Indian Financial System
\end{flushleft}


\begin{flushleft}
1.5 Credits (1.5-0-0)
\end{flushleft}


\begin{flushleft}
Pre--requisites: MSL301 \& MSL302
\end{flushleft}





\begin{flushleft}
MSL877 Electronic Government
\end{flushleft}


\begin{flushleft}
1.5 Credits (1.5-0-0)
\end{flushleft}


\begin{flushleft}
Pre--requisites: MSL301 \& MSL302
\end{flushleft}


\begin{flushleft}
This course may expose the participants to the following topics:
\end{flushleft}


\begin{flushleft}
Introduction to E-Governance, E-Governance models and frameworks,
\end{flushleft}


\begin{flushleft}
E-Governance infrastructure and stages in evolution, Information
\end{flushleft}


\begin{flushleft}
Management in Electronic Governance. Issues in Emerging and
\end{flushleft}


\begin{flushleft}
Developing Economies, Selective Case Studies in E-Governance,
\end{flushleft}


\begin{flushleft}
Emerging initiatives in electronic governance, Role of policy. Other
\end{flushleft}


\begin{flushleft}
relevant topics within the subject domain may also be explored.
\end{flushleft}





\begin{flushleft}
MSL878 Electronic Payments
\end{flushleft}


\begin{flushleft}
1.5 Credits (1.5-0-0)
\end{flushleft}


\begin{flushleft}
Pre--requisites: MSL301 \& MSL302
\end{flushleft}


\begin{flushleft}
This course may expose the participants to the following topics:
\end{flushleft}


\begin{flushleft}
Different business models in electronic payments, Digital certificates
\end{flushleft}


\begin{flushleft}
and certificate chains, Automated clearing and settlement systems,
\end{flushleft}


\begin{flushleft}
Banking systems and foreign exchanges, Other players in the
\end{flushleft}


\begin{flushleft}
ecosystem, E-Payment and Card security, Micro-payments, P2P
\end{flushleft}


\begin{flushleft}
Payments, Electronic Cash, Challenges and role of policy. Other
\end{flushleft}


\begin{flushleft}
relevant topics within the subject domain may also be explored.
\end{flushleft}





\begin{flushleft}
MSL879 Current and Emerging Issues in Finance
\end{flushleft}


\begin{flushleft}
3 Credits (3-0-0)
\end{flushleft}


\begin{flushleft}
Pre--requisites: MSL301 \& MSL302
\end{flushleft}


\begin{flushleft}
(Relevant current and Emerging Issues)
\end{flushleft}





\begin{flushleft}
This course is an introduction to the Indian financial system and tends
\end{flushleft}


\begin{flushleft}
to appraise students with its components, functions and integration
\end{flushleft}


\begin{flushleft}
of its sub components with each other. It covers different types of
\end{flushleft}


\begin{flushleft}
financial institutions, financial markets and financial instruments
\end{flushleft}


\begin{flushleft}
and services through which the financial system operates. Also, the
\end{flushleft}


\begin{flushleft}
students would develop an understanding of the role played by the
\end{flushleft}


\begin{flushleft}
different financial intermediaries in developing a robust financial
\end{flushleft}


\begin{flushleft}
environment for any country. The course will also give insight into the
\end{flushleft}


\begin{flushleft}
role played by financial market regulators and the challenges being
\end{flushleft}


\begin{flushleft}
faced by them in the modern internationally integrated economies.
\end{flushleft}





\begin{flushleft}
MSL875 International Financial Management
\end{flushleft}


\begin{flushleft}
3 Credits (3-0-0)
\end{flushleft}


\begin{flushleft}
Pre--requisites: MSL301 \& MSL302
\end{flushleft}


\begin{flushleft}
Module I : Foreign Exchange Market and Risk Management :
\end{flushleft}


\begin{flushleft}
Environment of International Financial Management: Balance of
\end{flushleft}


\begin{flushleft}
Payments. Means of International Payments, Foreign Exchange
\end{flushleft}


\begin{flushleft}
Market, Currency Futures and Options Markets, Foreign Exchange
\end{flushleft}


\begin{flushleft}
Risk Management, Political Risk, Interest Rate Risk.
\end{flushleft}


\begin{flushleft}
Module II : Financing of International Operations : Determination
\end{flushleft}


\begin{flushleft}
of Exchange Rate, Exchange Market and Arbitrage, Exchange
\end{flushleft}


\begin{flushleft}
Rate Control, Financing of Exports and International Investments,
\end{flushleft}


\begin{flushleft}
International Monetary Systems, European Monetary System,
\end{flushleft}


\begin{flushleft}
International monetary and Financial Institutions.
\end{flushleft}


\begin{flushleft}
Module III : Financial Management of MNCs : Capital Budgeting
\end{flushleft}


\begin{flushleft}
Decisions for Multinational Corporation, Financing Decisions- Cost
\end{flushleft}


\begin{flushleft}
of Capital and Financial Structure, Working Capital Management and
\end{flushleft}


\begin{flushleft}
Control, International Banking, International Transfer Pricing.
\end{flushleft}





\begin{flushleft}
MSL880 Selected Topics in Management Methodology
\end{flushleft}


\begin{flushleft}
3 Credits (3-0-0)
\end{flushleft}


\begin{flushleft}
Pre--requisites: MSL301 \& MSL302
\end{flushleft}


\begin{flushleft}
MSL881 Management of Public Sector Enterprises in India
\end{flushleft}


\begin{flushleft}
3 Credits (3-0-0)
\end{flushleft}


\begin{flushleft}
Pre--requisites: MSL301 \& MSL302
\end{flushleft}


\begin{flushleft}
This course will expose the participants to the following topics:
\end{flushleft}


\begin{flushleft}
Concepts of cloud computing and its impact, Technology Road Map
\end{flushleft}


\begin{flushleft}
to Cloud Computing, Virtualization, Practical usage of virtualization,
\end{flushleft}


\begin{flushleft}
Cloud Computing Frameworks and Deployment models. Cloud
\end{flushleft}


\begin{flushleft}
resource utilization and optimization, Cloud and Web Services, Service
\end{flushleft}


\begin{flushleft}
Model Architectures, SLA and QoS, Service Oriented Architecture and
\end{flushleft}


\begin{flushleft}
Cloud Computing.
\end{flushleft}





\begin{flushleft}
MSL882 Enterprise Cloud Computing
\end{flushleft}


\begin{flushleft}
1.5 Credits (1.5-0-0)
\end{flushleft}


\begin{flushleft}
Pre--requisites: MSL301 \& MSL302
\end{flushleft}


\begin{flushleft}
This course will expose the participants to the following topics:
\end{flushleft}


\begin{flushleft}
Concepts of cloud computing and its impact, Technology Road Map
\end{flushleft}


\begin{flushleft}
to Cloud Computing, Virtualization, Practical usage of virtualization,
\end{flushleft}


\begin{flushleft}
Cloud Computing Frameworks and Deployment models. Cloud
\end{flushleft}


\begin{flushleft}
resource utilization and optimization, Cloud and Web Services, Service
\end{flushleft}


\begin{flushleft}
Model Architectures, SLA and QoS, Service Oriented Architecture and
\end{flushleft}


\begin{flushleft}
Cloud Computing.
\end{flushleft}





243





\begin{flushleft}
\newpage
Management Studies
\end{flushleft}





\begin{flushleft}
MSL883 ICTs, Development and Business
\end{flushleft}


\begin{flushleft}
3 Credits (3-0-0)
\end{flushleft}


\begin{flushleft}
Pre--requisites: MSL301 \& MSL302
\end{flushleft}


\begin{flushleft}
The course will cover the following topics: Introduction, Development
\end{flushleft}


\begin{flushleft}
agendas and place of ICTs, ICTs as appropriate technologies, ICTs in
\end{flushleft}


\begin{flushleft}
education, health, industry \& enterprises; ICT policy \& regulations.
\end{flushleft}


\begin{flushleft}
Politics of open technology standards; ICT consulting for government;
\end{flushleft}


\begin{flushleft}
ICTs, Bottom of Pyramid \& Business.
\end{flushleft}





\begin{flushleft}
techniques and ETL, SQL. Data warehousing requirements for ETL; Data
\end{flushleft}


\begin{flushleft}
Warehousing Risks, OLAP and OLTP Management Issues, designing and
\end{flushleft}


\begin{flushleft}
supporting applications, Expanding a data warehouse. Other relevant
\end{flushleft}


\begin{flushleft}
topics within the subject domain may also be explored.
\end{flushleft}





\begin{flushleft}
MSL889 Current and Emerging Issues in Public Sector
\end{flushleft}


\begin{flushleft}
management
\end{flushleft}


\begin{flushleft}
3 Credits (3-0-0)
\end{flushleft}


\begin{flushleft}
Pre--requisites: MSL301 \& MSL302
\end{flushleft}


\begin{flushleft}
(Relevant current and Emerging Issues)
\end{flushleft}





\begin{flushleft}
MSL884 Information System Strategy
\end{flushleft}


\begin{flushleft}
3 Credits (3-0-0)
\end{flushleft}


\begin{flushleft}
Pre--requisites: MSL301 \& MSL302
\end{flushleft}


\begin{flushleft}
This course may expose the participants to the following topics: IT
\end{flushleft}


\begin{flushleft}
Evolution and its implications for business, IT Productivity Paradox
\end{flushleft}


\begin{flushleft}
- Issues and Implications, Impact of IS in the Networked Economy,
\end{flushleft}


\begin{flushleft}
Reasons for success and failure of IT projects, Disaster planning,
\end{flushleft}


\begin{flushleft}
Approaches to IS Development (e.g. Portfolio approaches), Technology
\end{flushleft}


\begin{flushleft}
Justification and Alignment Models, Strategic impact of IT / IS, Role of
\end{flushleft}


\begin{flushleft}
the CIO and challenges in business continuity.
\end{flushleft}





\begin{flushleft}
MSL885 Digital Marketing-Analytics \& Optimization
\end{flushleft}


\begin{flushleft}
3 Credits (3-0-0)
\end{flushleft}


\begin{flushleft}
Pre--requisites: MSL301 \& MSL302
\end{flushleft}





\begin{flushleft}
MSD890 Major Project (Unique Core)
\end{flushleft}


\begin{flushleft}
6 Credits (0-0-12)
\end{flushleft}


\begin{flushleft}
MSD891 Major Project (Unique Core)
\end{flushleft}


\begin{flushleft}
6 Credits (0-0-12)
\end{flushleft}


\begin{flushleft}
MSL891 Data Analytics using SP S S
\end{flushleft}


\begin{flushleft}
1.5 Credits (1.5-0-0)
\end{flushleft}


\begin{flushleft}
Pre--requisites: MSL301 \& MSL302
\end{flushleft}


\begin{flushleft}
MSD892 Major Project (Unique Core)
\end{flushleft}


\begin{flushleft}
6 Credits (0-0-12)
\end{flushleft}





\begin{flushleft}
The course may cover the following topics: Introduction and
\end{flushleft}


\begin{flushleft}
Perspectives in internet marketing, Online consumer behaviour and
\end{flushleft}


\begin{flushleft}
technology adoption theories, Managing the Word of Web, Mapping
\end{flushleft}


\begin{flushleft}
online communities \& networks, Online pricing mechanisms, Social
\end{flushleft}


\begin{flushleft}
Network Analytics \& Optimization, Web Analytics and Optimization,
\end{flushleft}


\begin{flushleft}
Traffic analytics, Online campaign and channel management, Managing
\end{flushleft}


\begin{flushleft}
the Web 2.0, Search Engine \& Social Media Optimization, SMAC, Social
\end{flushleft}


\begin{flushleft}
CRMs, Metrics for E-Commerce Analytics, KPIs, Revenue Analytics.
\end{flushleft}





\begin{flushleft}
MSL886 IT Consulting \& Practice
\end{flushleft}


\begin{flushleft}
3 Credits (3-0-0)
\end{flushleft}


\begin{flushleft}
Pre--requisites: MSL301 \& MSL302
\end{flushleft}


\begin{flushleft}
This course may expose the participants to the following topics: Trends
\end{flushleft}


\begin{flushleft}
in the IT consulting industry, IT consulting issues and pain points,
\end{flushleft}


\begin{flushleft}
Critical IT issues and their organizational contexts, Marketing and
\end{flushleft}


\begin{flushleft}
selling IT consulting projects, Project Entry Strategies, Contracting,
\end{flushleft}


\begin{flushleft}
Proposal Writing and making the sales pitch, Frameworks for
\end{flushleft}


\begin{flushleft}
technology evaluation. Frameworks for consulting intervention, change
\end{flushleft}


\begin{flushleft}
management and project closure,Implementation Planning for IT
\end{flushleft}


\begin{flushleft}
Projects, Managing Consulting Firms and Knowledge Management.
\end{flushleft}


\begin{flushleft}
Other relevant topics may also be explored.
\end{flushleft}





\begin{flushleft}
MSL887 Mobile Commerce
\end{flushleft}


\begin{flushleft}
3 Credits (3-0-0)
\end{flushleft}


\begin{flushleft}
Pre--requisites: MSL301 \& MSL302
\end{flushleft}


\begin{flushleft}
This course may expose the participants to the following topics:
\end{flushleft}


\begin{flushleft}
Introduction to Ubiquitous computing, Mobile communication and
\end{flushleft}


\begin{flushleft}
emerging technologies, Ubiquitous business models and challenges,
\end{flushleft}


\begin{flushleft}
Security issues and information risk management in mobile commerce.
\end{flushleft}


\begin{flushleft}
Mobile services and location based services, Interface with Social
\end{flushleft}


\begin{flushleft}
Media and Cloud, Mobile banking and payment systems, Socioeconomic development with m-Commerce, Mobile based services
\end{flushleft}


\begin{flushleft}
for e-governance. Introduction to mobile apps in the context of ICT
\end{flushleft}


\begin{flushleft}
ecosystem; explaining success of apps; app entrepreneurship; app
\end{flushleft}


\begin{flushleft}
economy, challenges of entrepreneurship and economy. Business
\end{flushleft}


\begin{flushleft}
models of app stores; mobile gaming; app customer segmentation;
\end{flushleft}


\begin{flushleft}
case studies.
\end{flushleft}





\begin{flushleft}
MSL892 Predictive Analytics
\end{flushleft}


\begin{flushleft}
1.5 Credits (1.5-0-0)
\end{flushleft}


\begin{flushleft}
Pre--requisites: MSL301 \& MSL302
\end{flushleft}


\begin{flushleft}
This course may expose the participants to the following topics:
\end{flushleft}


\begin{flushleft}
Introduction to the different predictive analytics models, using
\end{flushleft}


\begin{flushleft}
predictive analytics in decision making, types of predictive modeling,
\end{flushleft}


\begin{flushleft}
agent modeling, Case Based Reasoning and Predictive Expert Systems.
\end{flushleft}


\begin{flushleft}
Text mining, Social Network Analytics, Heuristics, Swarm algorithms,
\end{flushleft}


\begin{flushleft}
Hybrid Methods and algorithms. Other relevant topics within the
\end{flushleft}


\begin{flushleft}
subject domain may also be explored.
\end{flushleft}





\begin{flushleft}
MSL893 Public Policy Issues in the Information Age
\end{flushleft}


\begin{flushleft}
1.5 Credits (1.5-0-0)
\end{flushleft}


\begin{flushleft}
Pre--requisites: MSL301 \& MSL302
\end{flushleft}


\begin{flushleft}
This course may expose the participants to the following topics within
\end{flushleft}


\begin{flushleft}
this domain: Cyber Security Policies - National Cyber Security Policy,
\end{flushleft}


\begin{flushleft}
US, UK, EU; Global cyber security norms; Cloud computing policies;
\end{flushleft}


\begin{flushleft}
ICT Supply Chain trustworthiness; Social Media, Internet freedom of
\end{flushleft}


\begin{flushleft}
expression; Security v/s Privacy - surveillance; Internet Governance;
\end{flushleft}


\begin{flushleft}
Encryption - national security v/s economic growth; International
\end{flushleft}


\begin{flushleft}
Cooperation - treaties, norms, conventions. Other relevant topics
\end{flushleft}


\begin{flushleft}
within the subject domain may also be explored.
\end{flushleft}





\begin{flushleft}
MSL894 Social Media \& Business Practices
\end{flushleft}


\begin{flushleft}
3 Credits (3-0-0)
\end{flushleft}


\begin{flushleft}
Pre--requisites: MSL301 \& MSL302
\end{flushleft}


\begin{flushleft}
Introduction, Definition, Types, and Dimensions; Status in India \& the
\end{flushleft}


\begin{flushleft}
World; Different Revenue \& Business Models; Situating Social Media
\end{flushleft}


\begin{flushleft}
in Business; Adoption in Organizations; Social Media \& Applications:
\end{flushleft}


\begin{flushleft}
Viral marketing; Tool for SMEs, Customer Relationship Management,
\end{flushleft}


\begin{flushleft}
Researching Competitors; Digital Brand Management; Social Media
\end{flushleft}


\begin{flushleft}
Program Management; ROI; Influencers Index; and Social Media Audit
\end{flushleft}


\begin{flushleft}
\& Policy in Organizations.
\end{flushleft}





\begin{flushleft}
MSL895 Advance Data Analysis for Management
\end{flushleft}


\begin{flushleft}
3 Credits (3-0-0)
\end{flushleft}


\begin{flushleft}
Pre--requisites: MSL301 \& MSL302
\end{flushleft}


\begin{flushleft}
Module I : Descriptive vs. Inferential Analysis, Parametric vs.
\end{flushleft}


\begin{flushleft}
Nanoparametric Analysis, Univariate, Bivariate and multivariate
\end{flushleft}


\begin{flushleft}
analysis, Hypothesis Testing and Estimation
\end{flushleft}





\begin{flushleft}
MSL888 Data Warehousing for Business Decisions
\end{flushleft}


\begin{flushleft}
1.5 Credits (1.5-0-0)
\end{flushleft}


\begin{flushleft}
Pre--requisites: MSL301 \& MSL302
\end{flushleft}


\begin{flushleft}
This course may expose the participants to the following topics
\end{flushleft}


\begin{flushleft}
within this domain: Introduction to Database Management Systems,
\end{flushleft}


\begin{flushleft}
Hierarchical modelling, Multi-dimensional modeling of data, Design
\end{flushleft}





\begin{flushleft}
Module II : ANCOVA, MANOVA, Logit Regression, Tobbit Regression,
\end{flushleft}


\begin{flushleft}
Panel Regression.
\end{flushleft}


\begin{flushleft}
Module III : Factor Analysis, Cluster Analysis, Discriminant Analysis,
\end{flushleft}


\begin{flushleft}
Data Envelopment Analysis, Structural Equation Modelling.
\end{flushleft}





244





\begin{flushleft}
\newpage
Management Studies
\end{flushleft}


\begin{flushleft}
MSL896 International Economic Policy
\end{flushleft}


\begin{flushleft}
3 Credits (3-0-0)
\end{flushleft}


\begin{flushleft}
Pre--requisites: MSL301 \& MSL302
\end{flushleft}


\begin{flushleft}
World trade;The standard trade model; economies of scale and
\end{flushleft}


\begin{flushleft}
international trade; international factor movements; instruments of
\end{flushleft}


\begin{flushleft}
trade policy; exchange rates and foreign exchange markets; money,
\end{flushleft}


\begin{flushleft}
interest rates and exchange rates; price, output and exchange rates,
\end{flushleft}


\begin{flushleft}
different exchange rate regimes and policy, optimum currency area;
\end{flushleft}


\begin{flushleft}
Global capital markets; Financial crisis and contagion, Transition
\end{flushleft}


\begin{flushleft}
economies: crisis and reform.
\end{flushleft}





\begin{flushleft}
MSL897 Consultancy Process and Skills
\end{flushleft}


\begin{flushleft}
3 Credits (3-0-0)
\end{flushleft}


\begin{flushleft}
Pre--requisites: MSL301 \& MSL302
\end{flushleft}


\begin{flushleft}
Module I : Introduction to Consultancy-its evolution, growth \& status,
\end{flushleft}


\begin{flushleft}
Types of Consulting Services, firms and role of consultants, clientconsultant relationship. Marketing of Consultancy Services.
\end{flushleft}


\begin{flushleft}
Module II : The Consulting Process-Entry, Diagnosis, Action Planning,
\end{flushleft}


\begin{flushleft}
Implementation and Termination/Closing;
\end{flushleft}


\begin{flushleft}
Module III : Methods of selection of consultants, Costs and fee
\end{flushleft}


\begin{flushleft}
calculation, Preparation of Consultancy proposals and Agreements,
\end{flushleft}


\begin{flushleft}
Technical Report Writing and Presentation.
\end{flushleft}





\begin{flushleft}
MSL898 Consultancy Professional Practice
\end{flushleft}


\begin{flushleft}
3 Credits (3-0-0)
\end{flushleft}


\begin{flushleft}
Pre--requisites: MSL301 \& MSL302
\end{flushleft}





\begin{flushleft}
Module I : Negotiation Skills, Professional Ethics and Code of Conduct.
\end{flushleft}


\begin{flushleft}
Managing a Consultancy firm-fundamentals of consulting firm
\end{flushleft}


\begin{flushleft}
management, consulting firms and IT in consulting firms, management
\end{flushleft}


\begin{flushleft}
of consulting assignments.
\end{flushleft}


\begin{flushleft}
Module II : Consulting in various areas of Management-Consulting
\end{flushleft}


\begin{flushleft}
in general and strategic management, consulting in financial
\end{flushleft}


\begin{flushleft}
management, consulting in marketing and distribution management,
\end{flushleft}


\begin{flushleft}
consulting in production and operation management, consulting in
\end{flushleft}


\begin{flushleft}
HRM, consulting in IT.
\end{flushleft}


\begin{flushleft}
Module III : R\&D-Consultancy relation-ship, Careers and Compensation
\end{flushleft}


\begin{flushleft}
in Consulting, Training and development of Consultants, Future
\end{flushleft}


\begin{flushleft}
Challenges and Opportunities in Consultancy.
\end{flushleft}





\begin{flushleft}
MSL899 Current and Emerging Issues in Consultancy
\end{flushleft}


\begin{flushleft}
Management
\end{flushleft}


\begin{flushleft}
3 Credits (3-0-0)
\end{flushleft}


\begin{flushleft}
Pre--requisites: MSL301 \& MSL302
\end{flushleft}


\begin{flushleft}
MST893 Corporate Sector Attachment
\end{flushleft}


\begin{flushleft}
2 Credits (0-0-4)
\end{flushleft}


\begin{flushleft}
MSC894 Seminar
\end{flushleft}


\begin{flushleft}
3 Credits (0-0-6)
\end{flushleft}


\begin{flushleft}
MST894 Social Sector Attachment
\end{flushleft}


\begin{flushleft}
1 Credit (3-0-0)
\end{flushleft}





245





\begin{flushleft}
\newpage
Department of Mathematics
\end{flushleft}


\begin{flushleft}
MTL100 Calculus
\end{flushleft}


\begin{flushleft}
4 Credits (3-1-0)
\end{flushleft}


\begin{flushleft}
Review of limit, continuity and differentiability, uniform continuity.
\end{flushleft}


\begin{flushleft}
Mean value theorems and applications, Taylor's theorem, maxima and
\end{flushleft}


\begin{flushleft}
minima. Sequences and series, limsup, liminf, convergence of sequences
\end{flushleft}


\begin{flushleft}
and series of real numbers, absolute and conditional convergence.
\end{flushleft}


\begin{flushleft}
Riemann integral, fundamental theorem of integral calculus, applications
\end{flushleft}


\begin{flushleft}
of definite integrals, improper integrals, beta and gamma functions.
\end{flushleft}


\begin{flushleft}
Functions of several variables, limit and continuity, partial derivatives
\end{flushleft}


\begin{flushleft}
and differentiability, gradient, directional derivatives, chain rule, Taylor's
\end{flushleft}


\begin{flushleft}
theorem, maxima and minima and the method of Lagrange multipliers.
\end{flushleft}


\begin{flushleft}
Double and triple integration, Jacobian and change of variables
\end{flushleft}


\begin{flushleft}
formula. Parameterization of curves and surfaces, vector fields,
\end{flushleft}


\begin{flushleft}
divergence and curl. Line integrals, Green's theorem, surface integral,
\end{flushleft}


\begin{flushleft}
Gauss and Stokes' theorems with applications.
\end{flushleft}





\begin{flushleft}
MTL101 Linear Algebra and Differential Equations
\end{flushleft}


\begin{flushleft}
4 Credits (3-1-0)
\end{flushleft}


\begin{flushleft}
Vector spaces over Q, R and C, subspaces, linear independence, linear span
\end{flushleft}


\begin{flushleft}
of a set of vectors, basis and dimension of a vector space, sum and direct sum.
\end{flushleft}


\begin{flushleft}
Systems of linear (homogeneous and non-homogeneous) equations,
\end{flushleft}


\begin{flushleft}
matrices and Gauss elimination, elementary row operations, row space,
\end{flushleft}


\begin{flushleft}
column space, null space and rank of a matrix.
\end{flushleft}


\begin{flushleft}
Linear transformation, rank-nullity theorem and its applications, matrix
\end{flushleft}


\begin{flushleft}
representation of a linear transformation, change of basis and similarity.
\end{flushleft}


\begin{flushleft}
Eigenvalues and eigenvectors, characteristic and minimal polynomials,
\end{flushleft}


\begin{flushleft}
Cayley-Hamilton theorem (without proof) and applications.
\end{flushleft}


\begin{flushleft}
Review of first order differential equations, Picard's theorem, linear
\end{flushleft}


\begin{flushleft}
dependence and Wronskian. Dimensionality of space of solutions,
\end{flushleft}


\begin{flushleft}
linear ODE with constant coefficients of second and higher order,
\end{flushleft}


\begin{flushleft}
Cauchy-Euler equations, Method of undetermined coefficients and
\end{flushleft}


\begin{flushleft}
method of variation of parameters. Boundary Value Problems: SturmLiouville eigenvalue problems. System of linear differential equations
\end{flushleft}


\begin{flushleft}
with constant coefficients, fundamental matrix, matrix methods. Power
\end{flushleft}


\begin{flushleft}
Series and its convergence, power series method, Fourier series,
\end{flushleft}


\begin{flushleft}
Laplace Transform Method.
\end{flushleft}





\begin{flushleft}
MTL102 Differential Equations
\end{flushleft}


\begin{flushleft}
3 Credits (3-0-0)
\end{flushleft}


\begin{flushleft}
Overlaps with: MTL260
\end{flushleft}


\begin{flushleft}
Systems of differential equations, Existence and uniqueness theorems
\end{flushleft}


\begin{flushleft}
for initial value problems of semilinear and nonlinear ODEs, continuous
\end{flushleft}


\begin{flushleft}
dependence and well-posed ness; Comparison theorems of Sturms,
\end{flushleft}


\begin{flushleft}
Sturm-Liouville eigenvalue problems; Phase-plane analysis, Linear and
\end{flushleft}


\begin{flushleft}
Non-linear stability, Liapunov functions and applications;First order
\end{flushleft}


\begin{flushleft}
Partial differential equations, Method of characteristics, local and global
\end{flushleft}


\begin{flushleft}
solutions, envelop of solutions, complete and general solutions; Second
\end{flushleft}


\begin{flushleft}
order equations: Heat and Wave equation, fundamental solutions,
\end{flushleft}


\begin{flushleft}
method of eigenfunctions, Duhamel's principle. Maximum priciples
\end{flushleft}


\begin{flushleft}
for Heat and Laplace equation,Greens functions.
\end{flushleft}





\begin{flushleft}
MTL104 Linear Algebra and Applications
\end{flushleft}


\begin{flushleft}
3 Credits (3-0-0)
\end{flushleft}


\begin{flushleft}
Pre-requisites: MTL101
\end{flushleft}


\begin{flushleft}
Overlaps with: MTL502
\end{flushleft}


\begin{flushleft}
Introduce Fields: fields of numbers, finite fields. Review basis and
\end{flushleft}


\begin{flushleft}
dimension of a vector space, linear transformations, eigenvalue and
\end{flushleft}


\begin{flushleft}
eigenvector of an operator. LU Factorization. Some applications giving
\end{flushleft}


\begin{flushleft}
rise to Linear Systems Problems
\end{flushleft}


\begin{flushleft}
Dual and double dual of a vector space and transpose of a linear
\end{flushleft}


\begin{flushleft}
transformation. Diagonalizability of linear operators of finite
\end{flushleft}


\begin{flushleft}
dimensional vector spaces, simultaneous triangulization and
\end{flushleft}


\begin{flushleft}
simultaneous diagonalization. The primary decomposition theorem diagonal and nilpotent parts.
\end{flushleft}


\begin{flushleft}
Inner product spaces, Gram-Schmidt orthogonalization, best
\end{flushleft}


\begin{flushleft}
approximation of a vector by a vector belonging a given subspace and
\end{flushleft}


\begin{flushleft}
application to least square problems. Adjoint of an operator, hermitian,
\end{flushleft}


\begin{flushleft}
unitary and normal operators. Singular Value Decomposition and its
\end{flushleft}


\begin{flushleft}
applications. Spectral decomposition. Introduction of bilinear and
\end{flushleft}


\begin{flushleft}
quadratic forms.
\end{flushleft}





\begin{flushleft}
MTL105 Algebra
\end{flushleft}


\begin{flushleft}
3 Credits (3-0-0)
\end{flushleft}


\begin{flushleft}
Overlaps with: MTL501
\end{flushleft}


\begin{flushleft}
Preliminaries: Equivalence relations and partitions.
\end{flushleft}


\begin{flushleft}
Groups: Subgroups, Cyclic groups, Abelian groups, permutation
\end{flushleft}


\begin{flushleft}
groups; Langrange's theorem, normal subgroups, quotient groups,
\end{flushleft}


\begin{flushleft}
isomorphism theorems. Direct product of groups, structure theorem of
\end{flushleft}


\begin{flushleft}
finitely generated abelian groups, Sylow's theorems and applications.
\end{flushleft}


\begin{flushleft}
Rings: Definition and examples, units and zero divisors. Ideals and
\end{flushleft}


\begin{flushleft}
quotients, principal ideals, prime ideals, maximal ideals, integral
\end{flushleft}


\begin{flushleft}
domain, PID, Euclidean domain, UFD. Modules over a commutative
\end{flushleft}


\begin{flushleft}
ring with unity: Free module, quotient module, exact sequences.
\end{flushleft}


\begin{flushleft}
Fields: Finite fields, field extensions, splitting fields.
\end{flushleft}





\begin{flushleft}
MTL106 Probability and Stochastic Processes
\end{flushleft}


\begin{flushleft}
4 Credits (3-1-0)
\end{flushleft}


\begin{flushleft}
Overlaps with: MTL108
\end{flushleft}


\begin{flushleft}
Axioms of probability, Probability space, Conditional probability,
\end{flushleft}


\begin{flushleft}
Independence, Bayes' rule, Random variable, Some common
\end{flushleft}


\begin{flushleft}
discrete and continuous distributions, Distribution of Functions of
\end{flushleft}


\begin{flushleft}
Random Variable, Moments, Generating functions, Two and higher
\end{flushleft}


\begin{flushleft}
dimensional distributions, Functions of random variables, Order
\end{flushleft}


\begin{flushleft}
statistics, Conditional distributions, Covariance, Correlation coefficient,
\end{flushleft}


\begin{flushleft}
conditional expectation, Modes of convergences, Laws of large
\end{flushleft}


\begin{flushleft}
numbers, Central limit theorem, Definition of Stochastic process,
\end{flushleft}


\begin{flushleft}
Classification and properties of stochastic processes, Simple Markovian
\end{flushleft}


\begin{flushleft}
stochastic processes, Gaussian processes, Stationary processes,
\end{flushleft}


\begin{flushleft}
Discrete and continuous time Markov chains, Classification of states,
\end{flushleft}


\begin{flushleft}
Limiting distribution, Birth and death process, Poisson process, Steady
\end{flushleft}


\begin{flushleft}
state and transient distributions, Simple Markovian queuing models
\end{flushleft}


\begin{flushleft}
(M/M/1, M/M/1/N, M/M/c/N, M/M/N/N, M/M/$\infty$).
\end{flushleft}





\begin{flushleft}
MTL107 Numerical Methods and Computations
\end{flushleft}


\begin{flushleft}
3 Credits (3-0-0)
\end{flushleft}


\begin{flushleft}
Overlaps with: MTL509, CLL113, CVL763
\end{flushleft}





\begin{flushleft}
MTL103 Optimization Methods and Applications
\end{flushleft}


\begin{flushleft}
3 Credits (3-0-0)
\end{flushleft}


\begin{flushleft}
Overlaps with: MTL508, CLL782, MCL261
\end{flushleft}


\begin{flushleft}
Linear programming - formulation through examples from
\end{flushleft}


\begin{flushleft}
engineering / business decision making problems, preliminary
\end{flushleft}


\begin{flushleft}
theory and geometry of linear programs, basic feasible solution,
\end{flushleft}


\begin{flushleft}
simplex method, variants of simplex method. Duality and its
\end{flushleft}


\begin{flushleft}
principles, interpretation of dual variables, dual simplex method.
\end{flushleft}


\begin{flushleft}
Linear integer programming, applications in real decision making
\end{flushleft}


\begin{flushleft}
problems, methods to solve linear integer programs, transportation
\end{flushleft}


\begin{flushleft}
problems: theory and methodology, assignment problems. Zerosum matrix games, saddle point, linear programming formulation
\end{flushleft}


\begin{flushleft}
of matrix games, network optimization problems LPP formulation.
\end{flushleft}


\begin{flushleft}
Nonlinear programming, Lagrange function, KKT optimality
\end{flushleft}


\begin{flushleft}
conditions, sufficiency of KKT under convexity of quadratic
\end{flushleft}


\begin{flushleft}
programming, Wolfe's method, applications of quadratic programs.
\end{flushleft}





\begin{flushleft}
Errors in computation: source and types of errors, error propagation.
\end{flushleft}


\begin{flushleft}
Computer representation of numbers: floating point representation,
\end{flushleft}


\begin{flushleft}
rounding error and floating point arithmetic. Roots of nonlinear
\end{flushleft}


\begin{flushleft}
equation in one variable: Direct and iterative methods, order of
\end{flushleft}


\begin{flushleft}
convergence. Iterative methods for roots of nonlinear system
\end{flushleft}


\begin{flushleft}
of equations. Linear systems of equations: Direct and iterative
\end{flushleft}


\begin{flushleft}
methods, rate of convergence of iterative methods, Condition
\end{flushleft}


\begin{flushleft}
number and ill-conditioned systems. Interpolation: Lagrange,
\end{flushleft}


\begin{flushleft}
Newton divided difference formula, Newton's interpolations,
\end{flushleft}


\begin{flushleft}
errors in interpolation. Approximation: least square and uniform
\end{flushleft}


\begin{flushleft}
approximations. Differentiation: differentiation using interpolation
\end{flushleft}


\begin{flushleft}
formulas. Integration using interpolation: Newton-Cotes formulas,
\end{flushleft}


\begin{flushleft}
Gauss quadrature rules. Ordinary differential equations: Taylor, Euler
\end{flushleft}


\begin{flushleft}
and Runge-Kutta methods. Implementation of these methods.
\end{flushleft}





246





\begin{flushleft}
\newpage
Mathematics
\end{flushleft}





\begin{flushleft}
MTL108 Introduction to Statistics
\end{flushleft}


\begin{flushleft}
3 Credits (3-0-0)
\end{flushleft}


\begin{flushleft}
Overlaps with: MTL106, MTL390
\end{flushleft}





\begin{flushleft}
MTL260 Boundary Value Problems
\end{flushleft}


\begin{flushleft}
3 Credits (3-0-0)
\end{flushleft}


\begin{flushleft}
Pre-requisites: MTL100, MTL101
\end{flushleft}





\begin{flushleft}
Measures of central tendency, Dispersion, Skewness, Kurtosis,
\end{flushleft}


\begin{flushleft}
Data Representation using Histogram, Pie Chart, Boxplot, Biplot,
\end{flushleft}


\begin{flushleft}
Multi Dimensional Scaling etc. Concept of Random Variable, Some
\end{flushleft}


\begin{flushleft}
common discrete and continuous distributions, Distribution of
\end{flushleft}


\begin{flushleft}
Functions of Random Variables, Bivariate and Multivariate random
\end{flushleft}


\begin{flushleft}
variables. Sampling Distribution, Theory of Estimation, Properties of
\end{flushleft}


\begin{flushleft}
an estimator, Cramer Rao Theorem, Rao Blackwellization, One-sample
\end{flushleft}


\begin{flushleft}
and Two sample tests of Proportion, mean, variance, Critical regions,
\end{flushleft}


\begin{flushleft}
Neyman Pearson Lemma. Tests for Goodness of fit, Chi-square Test,
\end{flushleft}


\begin{flushleft}
Kolmogorov Smirnov Test, One sample and paired sample tests: Sign
\end{flushleft}


\begin{flushleft}
Test, Signed-rank Test, Run tests etc. Linear regression, Non-linear
\end{flushleft}


\begin{flushleft}
regression, Logit and Probit Methods.
\end{flushleft}





\begin{flushleft}
Sturm Liouville problem, Boundary Value Problems for nonhomogeneous
\end{flushleft}


\begin{flushleft}
ODEs, Green's Functions. Fourier Series and Integrals: Periodic
\end{flushleft}


\begin{flushleft}
Functions and Fourier Series, Arbitrary Period and Half-Range
\end{flushleft}


\begin{flushleft}
Expansions, Fourier Integral theorem and convergence of series
\end{flushleft}


\begin{flushleft}
Parabolic equations: Heat equation, Fourier series solution, Different
\end{flushleft}


\begin{flushleft}
Boundary Conditions, Generalities on the Heat Conduction Problems on
\end{flushleft}


\begin{flushleft}
bounded and unbounded domains and applications in Option pricing.
\end{flushleft}





\begin{flushleft}
MTL122 Real and Complex Analysis
\end{flushleft}


\begin{flushleft}
4 Credits (3-1-0)
\end{flushleft}


\begin{flushleft}
Pre-requisites: MTL100
\end{flushleft}


\begin{flushleft}
Overlaps with: MTL503, MTL506
\end{flushleft}





\begin{flushleft}
Higher Dimensions and Other Coordinates: Two-Dimensional Wave
\end{flushleft}


\begin{flushleft}
Equation: Derivation, Parabolic equation, Solution by Fourier series,
\end{flushleft}


\begin{flushleft}
Problems in Polar Coordinates, Temperature in a Cylinder, Vibrations
\end{flushleft}


\begin{flushleft}
of a Circular Membrane
\end{flushleft}





\begin{flushleft}
Metric spaces: Definition and examples. Open, closed and bounded
\end{flushleft}


\begin{flushleft}
sets. Interior, closure and boundary. Convergence and completeness.
\end{flushleft}


\begin{flushleft}
Continuity and uniform continuity. Connectedness, compactness and
\end{flushleft}


\begin{flushleft}
separability. Heine-Borel theorem. Pointwise and uniform convergence
\end{flushleft}


\begin{flushleft}
of real-valued functions. Equicontinuity. Ascoli-Arzela theorem. Limits,
\end{flushleft}


\begin{flushleft}
continuity and differentiability of functions of a complex variable.
\end{flushleft}


\begin{flushleft}
Analytic functions, the Cauchy-Riemann equations. Definition of
\end{flushleft}


\begin{flushleft}
contour integrals, Cauchy's integral formula and derivatives of analytic
\end{flushleft}


\begin{flushleft}
functions. Morera's and Liouville's theorems. Maximum modulus
\end{flushleft}


\begin{flushleft}
principle. Taylor and Laurent series. Isolated singular points and
\end{flushleft}


\begin{flushleft}
residues. Cauchy's residue theorem and applications.
\end{flushleft}





\begin{flushleft}
MTL145 Number Theory
\end{flushleft}


\begin{flushleft}
3 Credits (3-0-0)
\end{flushleft}


\begin{flushleft}
Divisibility: basic definition, properties, prime numbers, some
\end{flushleft}


\begin{flushleft}
results on distribution of primes; Congruences: basic definitions
\end{flushleft}


\begin{flushleft}
and properties, complete and reduced residue systems, theorems
\end{flushleft}


\begin{flushleft}
of Fermat, Euler \& Wilson, application to RSA cryptosystem,
\end{flushleft}


\begin{flushleft}
linear congruences and Chinese Remainder theorem, quadratic
\end{flushleft}


\begin{flushleft}
congruences, and Quadratic Reciprocity law; Arithmetical functions:
\end{flushleft}


\begin{flushleft}
examples, with some properties and their rate of growth; Continued
\end{flushleft}


\begin{flushleft}
fractions, and their connections with Diophantine approximatins,
\end{flushleft}


\begin{flushleft}
applications tolinear and Pell's equations; Binary quadratic forms;
\end{flushleft}


\begin{flushleft}
Partition: basic properties and results; Diophatine equations: linear
\end{flushleft}


\begin{flushleft}
and quadratic, some general equations.
\end{flushleft}





\begin{flushleft}
MTL146 Combinatorics
\end{flushleft}


\begin{flushleft}
3 Credits (3-0-0)
\end{flushleft}


\begin{flushleft}
Pre-requisites: MTL180
\end{flushleft}


\begin{flushleft}
Basic counting techniques; principle of inclusion and exclusion;
\end{flushleft}


\begin{flushleft}
recurrences and generating functions; Systems of Distinct
\end{flushleft}


\begin{flushleft}
Representatives \& Hall's theorem; Extremal Set theory; Projective
\end{flushleft}


\begin{flushleft}
and combinatorial geometries; Latin squares; Designs \& Steiner Triple
\end{flushleft}


\begin{flushleft}
Systems; Ramsey theory.
\end{flushleft}





\begin{flushleft}
MTL180 Discrete Mathematical Structures
\end{flushleft}


\begin{flushleft}
4 Credits (3-1-0)
\end{flushleft}


\begin{flushleft}
Overlaps with: COL202
\end{flushleft}


\begin{flushleft}
Logic : Propositional Logic: language of propositional logic, truth
\end{flushleft}


\begin{flushleft}
table, natural deduction, predicate logic: language of predicate logic,
\end{flushleft}


\begin{flushleft}
Logical inference with Quantifiers. Proof techniques: Introduction
\end{flushleft}


\begin{flushleft}
to different standard proof techniques. Set Theory: Review of Basic
\end{flushleft}


\begin{flushleft}
Set Operations, cardinality of a set. Relations : Types of relations,
\end{flushleft}


\begin{flushleft}
operations of relations and applications, Poset, topological ordering;
\end{flushleft}


\begin{flushleft}
Congruence arithmetic; Combinatorics: Counting techniques: Pigeon
\end{flushleft}


\begin{flushleft}
Hole principle, inclusion exclusion principle, recurrence relation and
\end{flushleft}


\begin{flushleft}
generating function; Graph Theory : Graph as a discrete structure,
\end{flushleft}


\begin{flushleft}
Modeling applications using graphs, Hamiltonian graphs, Planar
\end{flushleft}


\begin{flushleft}
graphs, Graph coloring, Matching.
\end{flushleft}





\begin{flushleft}
The Wave Equation: The Vibrating String, Solution of the Vibrating
\end{flushleft}


\begin{flushleft}
String Problem, d'Alembert's Solution, One-Dimensional Wave Equation
\end{flushleft}


\begin{flushleft}
The Potential Equation: Potential Equation in a Rectangle, Fourier
\end{flushleft}


\begin{flushleft}
series method, Potential equation in Unbounded Regions, Fourier
\end{flushleft}


\begin{flushleft}
integral representations, Potential in a Disk and Limitations.
\end{flushleft}





\begin{flushleft}
Finite dimensional approximations of solutions, piecewise linear
\end{flushleft}


\begin{flushleft}
polynomials and introduction to different methods like Galerkin and
\end{flushleft}


\begin{flushleft}
Petrov-Galerkin method.
\end{flushleft}





\begin{flushleft}
MTL265 Mathematical Programming Techniques
\end{flushleft}


\begin{flushleft}
3 Credits (3-0-0)
\end{flushleft}


\begin{flushleft}
Pre-requisites: MTL103
\end{flushleft}


\begin{flushleft}
Overlaps with: COL756
\end{flushleft}


\begin{flushleft}
Recall of linear programming simplex algorithm and dual problem;
\end{flushleft}


\begin{flushleft}
primal-dual simplex method, linear programs with upper bounds,
\end{flushleft}


\begin{flushleft}
network optimization, network simplex method for non-capacitated and
\end{flushleft}


\begin{flushleft}
capacitated networks; dynamic programming, principle of optimality,
\end{flushleft}


\begin{flushleft}
general insight followed by in-depth examples; complexity of simplex
\end{flushleft}


\begin{flushleft}
algorithm, Karmarkar's interior point method; nonlinear programming,
\end{flushleft}


\begin{flushleft}
KKT conditions, convex programs, linear fractional programming
\end{flushleft}


\begin{flushleft}
problems, Charnes and Cooper technique, convex simplex method,
\end{flushleft}


\begin{flushleft}
Rosen projection method; multiobjective programming problems,
\end{flushleft}


\begin{flushleft}
applications to engineering and sciences, Pareto efficient solution,
\end{flushleft}


\begin{flushleft}
linear multiobjective programs, weighted sum approach, scalarization
\end{flushleft}


\begin{flushleft}
schemes, goal programming.
\end{flushleft}





\begin{flushleft}
MTL270 Measure Integral and Probability
\end{flushleft}


\begin{flushleft}
3 Credits (3-0-0)
\end{flushleft}


\begin{flushleft}
Overlaps with: MTL510
\end{flushleft}


\begin{flushleft}
Measurable spaces, measurable sets, measurable functions, measure,
\end{flushleft}


\begin{flushleft}
outer measures and generation of measure, Lebesgue integration,
\end{flushleft}


\begin{flushleft}
basic integration theorem, comparison of Lebesgue and Riemann
\end{flushleft}


\begin{flushleft}
integrals, various modes of convergence of measurable functions,
\end{flushleft}


\begin{flushleft}
signed measure, Hahn and Jordan decomposition theorems, the
\end{flushleft}


\begin{flushleft}
Radon-Nikodym theorem, product measures and Fubini's theorem,
\end{flushleft}


\begin{flushleft}
probability measures and spaces, independent events, conditional
\end{flushleft}


\begin{flushleft}
probability, theorem of total probability, random variables, distribution
\end{flushleft}


\begin{flushleft}
and distribution function of a random variable, independent random
\end{flushleft}


\begin{flushleft}
variable, expectation, convergence in distribution of a sequence
\end{flushleft}


\begin{flushleft}
of random variables, weak and strong laws of large numbers,
\end{flushleft}


\begin{flushleft}
Kolmogorov's zero-one law, the central limit theorem, identically
\end{flushleft}


\begin{flushleft}
distributed summands, the Linderberg and Lyapounov theorems.
\end{flushleft}





\begin{flushleft}
MTP290 Computing Laboratory
\end{flushleft}


\begin{flushleft}
2 Credits (0-0-4)
\end{flushleft}


\begin{flushleft}
Pre-requisites: MTL101
\end{flushleft}


\begin{flushleft}
Programming concepts. Implementation of matrix operations,
\end{flushleft}


\begin{flushleft}
Complexity in Matrix Operations, Implementation of linear algebraic
\end{flushleft}


\begin{flushleft}
solvers; solution of systems of linear equations. Gauss elimination,
\end{flushleft}


\begin{flushleft}
LU decomposition and Iterative methods. Implementation of several
\end{flushleft}


\begin{flushleft}
Numerical Integration algorithms, Initial value problems and Boundary
\end{flushleft}


\begin{flushleft}
Value Problems for ODEs.
\end{flushleft}





247





\begin{flushleft}
\newpage
Mathematics
\end{flushleft}





\begin{flushleft}
MTL342 Analysis and Design of Algorithms
\end{flushleft}


\begin{flushleft}
4 Credits (3-1-0)
\end{flushleft}


\begin{flushleft}
Pre-requisites: MTL180
\end{flushleft}


\begin{flushleft}
Overlaps with: COL351
\end{flushleft}





\begin{flushleft}
MTD421 B.Tech. Project
\end{flushleft}


\begin{flushleft}
4 Credits (0-0-8)
\end{flushleft}


\begin{flushleft}
Pre-requisites: EC 100
\end{flushleft}





\begin{flushleft}
Models of computation: RAM and Turing Machines; Algorithm Analysis
\end{flushleft}


\begin{flushleft}
techniques; Basic techniques for designing algorithms: dynamic
\end{flushleft}


\begin{flushleft}
programming, divide-and-conquer and Greedy; DFS , BFS and their
\end{flushleft}


\begin{flushleft}
applications; Some Basic Graph Algorithms; linear time sorting
\end{flushleft}


\begin{flushleft}
algorithms; NP-Completeness and Approximation Algorithms.
\end{flushleft}





\begin{flushleft}
MTD350 Mini Project
\end{flushleft}


\begin{flushleft}
3 Credits (0-0-6)
\end{flushleft}


\begin{flushleft}
Pre-requisites: EC 80
\end{flushleft}


\begin{flushleft}
Depends on the project topic.
\end{flushleft}





\begin{flushleft}
MTL390 Statistical Methods
\end{flushleft}


\begin{flushleft}
4 Credits (3-1-0)
\end{flushleft}


\begin{flushleft}
Pre-requisites: MTL106
\end{flushleft}


\begin{flushleft}
Overlaps with: MTL108
\end{flushleft}


\begin{flushleft}
Basic concepts and Data Visualization: Measures of central tendency,
\end{flushleft}


\begin{flushleft}
Dispersion, Skewness, Kurtosis, Data Representation using Histogram,
\end{flushleft}


\begin{flushleft}
Pie Chart, Boxplot, Biplot, Multi Dimensional Scaling etc. Revision of
\end{flushleft}


\begin{flushleft}
Probability Distribution: Emphasis on Normal, Chi-Square, Student's
\end{flushleft}


\begin{flushleft}
T, F distributions; Order Statistics: Different Order Statistics and their
\end{flushleft}


\begin{flushleft}
single and joint Distribution; Sampling Distribution of Mean, Variance;
\end{flushleft}


\begin{flushleft}
Generation of Random Numbers following certain distributions;
\end{flushleft}


\begin{flushleft}
Theory of Estimation (Point and Interval)Properties of an estimator,
\end{flushleft}


\begin{flushleft}
MVUE, BLUE, Cramer-Rao Inequality, Rao-Blackwell Theorem; Testing
\end{flushleft}


\begin{flushleft}
of Hypothesis: Mean and Variance, Confidence Interval, NeymanPearson Lemma; Non-Parametric Methods Run Tests, Rank Tests,
\end{flushleft}


\begin{flushleft}
Signed Rank Tests, Kruskal Wallis Test, Kolmogorov-Smirnov Test
\end{flushleft}


\begin{flushleft}
etc.; Regression Analysis Linear Regression, Multiple Regression,
\end{flushleft}


\begin{flushleft}
Logit, Probit, Regression.
\end{flushleft}





\begin{flushleft}
MTL411 Functional Analysis
\end{flushleft}


\begin{flushleft}
3 Credits (3-0-0)
\end{flushleft}


\begin{flushleft}
Pre-requisites: MTL104 and MTL122
\end{flushleft}


\begin{flushleft}
Overlaps with: MTL602
\end{flushleft}


\begin{flushleft}
Review of some basic concepts in metric spaces and topological spaces;
\end{flushleft}


\begin{flushleft}
Normed linear spaces and Banach spaces, Examples of Banach spaces,
\end{flushleft}


\begin{flushleft}
Bounded linear operators and examples, Finite dimensional Banach
\end{flushleft}


\begin{flushleft}
spaces; Introduction of Lebesgue integration on real line, Fatou's
\end{flushleft}


\begin{flushleft}
lemma, monotone convergence theorem, dominated convergence
\end{flushleft}


\begin{flushleft}
theorem, Lp spaces; Hahn Banach extension theorem, Hahn Banach
\end{flushleft}


\begin{flushleft}
separation theorem, Uniform boundedness principle, Open mapping
\end{flushleft}


\begin{flushleft}
theorem, Closed graph theorem; Characterization of dual of certain
\end{flushleft}


\begin{flushleft}
concrete Banach spaces; Schauder basis and separability, Reflexive
\end{flushleft}


\begin{flushleft}
Banach spaces, Best approximation in Banach spaces; Hilbert spaces
\end{flushleft}


\begin{flushleft}
and their geometry; Basic operator theory.
\end{flushleft}





\begin{flushleft}
MTL415 Parallel Algorithms
\end{flushleft}


\begin{flushleft}
3 Credits (3-0-0)
\end{flushleft}


\begin{flushleft}
Pre-requisites: MTL342
\end{flushleft}


\begin{flushleft}
Overlaps with: MTL765
\end{flushleft}





\begin{flushleft}
Contents will be related to topic from the courses undertaken by the
\end{flushleft}


\begin{flushleft}
student in the programme.
\end{flushleft}





\begin{flushleft}
MTL445 Computational Methods for Differential
\end{flushleft}


\begin{flushleft}
Equations
\end{flushleft}


\begin{flushleft}
4 Credits (3-0-2)
\end{flushleft}


\begin{flushleft}
Pre-requisites: MTL107
\end{flushleft}


\begin{flushleft}
Overlaps with: MTL712, CLL113
\end{flushleft}


\begin{flushleft}
Numerical methods for solving IVPs for ODEs: Difference equations,
\end{flushleft}


\begin{flushleft}
Routh-Hurwitz criterion, Test Equation. Single step methods: Taylor
\end{flushleft}


\begin{flushleft}
series method, explicit Runge-Kutta methods, convergence, order,
\end{flushleft}


\begin{flushleft}
relative and absolute stability. Multistep methods: Development
\end{flushleft}


\begin{flushleft}
of linear multistep method using interpolation and undetermined
\end{flushleft}


\begin{flushleft}
parameter approach, convergence, order, relative and absolute
\end{flushleft}


\begin{flushleft}
stability, Predictor Corrector methods. Solution of initial value problems
\end{flushleft}


\begin{flushleft}
of systems of ODES. BVP: Finite difference methods for second order
\end{flushleft}


\begin{flushleft}
ODEs, Eigenvalue problems.
\end{flushleft}


\begin{flushleft}
PDEs: Finite difference methods for Elliptic PDEs, Consistency,
\end{flushleft}


\begin{flushleft}
stability and convergence. Boundary Conditions. FD methods for
\end{flushleft}


\begin{flushleft}
Parabolic equations in 1D and 2D. Operator splitting methods,
\end{flushleft}


\begin{flushleft}
Convergence, stability and consistency of difference methods.
\end{flushleft}


\begin{flushleft}
Higher order methods. Introduction to Hyperbolic PDEs, FD methods.
\end{flushleft}


\begin{flushleft}
Upwind schemes, Consistency, stability and convergence of schemes.
\end{flushleft}


\begin{flushleft}
Second order schemes.
\end{flushleft}





\begin{flushleft}
MTL458 Operating Systems
\end{flushleft}


\begin{flushleft}
4 Credits (3-0-2)
\end{flushleft}


\begin{flushleft}
Pre-requisites: MTL342
\end{flushleft}


\begin{flushleft}
Overlaps with: COL331, ELL405
\end{flushleft}


\begin{flushleft}
Operating Systems functions, Basic Concepts, Notion of a process,
\end{flushleft}


\begin{flushleft}
concurrent processes, problem of mutual exclusion, Deadlock, process
\end{flushleft}


\begin{flushleft}
Scheduling, memory management, multiprogramming, File systems;
\end{flushleft}


\begin{flushleft}
time sharing systems and their design consideration.
\end{flushleft}





\begin{flushleft}
MTL501 Algebra
\end{flushleft}


\begin{flushleft}
4 Credits (3-1-0)
\end{flushleft}


\begin{flushleft}
Groups, subgroups, Lagrange theorem, quotient groups, isomorphism
\end{flushleft}


\begin{flushleft}
theorems; cyclic groups, dihedral groups, symmetric groups,
\end{flushleft}


\begin{flushleft}
alternating groups; simple groups, simplicity of alternating groups;
\end{flushleft}


\begin{flushleft}
Group action, Sylow theorems and applications; free abelian groups,
\end{flushleft}


\begin{flushleft}
structure of finitely generated abelian groups; Solvable and nilpotent
\end{flushleft}


\begin{flushleft}
groups, composition series, Jordan-Holder theorem.
\end{flushleft}


\begin{flushleft}
Rings, examples: polynomial rings, formal power series, matrix rings,
\end{flushleft}


\begin{flushleft}
group rings; prime ideals, maximal ideals, quotient rings, isomorphism
\end{flushleft}


\begin{flushleft}
theorems; Integral domains, PID, UFD, Euclidean domains, division
\end{flushleft}


\begin{flushleft}
rings, field of fractions; primes and irreducibles, irreducibility criteria;
\end{flushleft}


\begin{flushleft}
product of rings, Chinese remainder theorem.
\end{flushleft}


\begin{flushleft}
Field extension, algebraic extension, algebraic closure, straight
\end{flushleft}


\begin{flushleft}
edge and compass constructions, splitting fields, separable and
\end{flushleft}


\begin{flushleft}
inseparable extensions, fundamental theorem of Galois theory;
\end{flushleft}


\begin{flushleft}
solvability by radicals.
\end{flushleft}





\begin{flushleft}
Parallel architecture: Shared/local memory systems, pipelining,
\end{flushleft}


\begin{flushleft}
hypercubes, mesh, linear array etc. Degree of parallelism, Speed-up
\end{flushleft}


\begin{flushleft}
and efficiency of a parallel algorithm. Principles of parallel algorithm
\end{flushleft}


\begin{flushleft}
design. Basic communication operations. Parallel algorithms: searching
\end{flushleft}


\begin{flushleft}
and sorting, matrix-vector and matrix-matrix multiplication for dense,
\end{flushleft}


\begin{flushleft}
band and triangular matrices. Parallel algorithms for direct methods
\end{flushleft}


\begin{flushleft}
for dense, band and triangular matrices. Cholesky method. Solving
\end{flushleft}


\begin{flushleft}
recurrence relations. Parallel iterative methods for finite difference
\end{flushleft}


\begin{flushleft}
equations of elliptic boundary value problems: point-Jacobi, line
\end{flushleft}


\begin{flushleft}
Jacobi, block Jacobi methods, 2 colour and multicolour Gauss-Seidel,
\end{flushleft}


\begin{flushleft}
SOR, SSOR methods. Domain decomposition method in one and two
\end{flushleft}


\begin{flushleft}
dimensions. Parallel preconditioned conjugate gradient methods.
\end{flushleft}


\begin{flushleft}
Quadrant interlocking factorization.
\end{flushleft}





\begin{flushleft}
MTL502 Linear Algebra
\end{flushleft}


\begin{flushleft}
4 Credits (3-1-0)
\end{flushleft}


\begin{flushleft}
Revision of existence-uniqueness of solutions of a system of linear
\end{flushleft}


\begin{flushleft}
equations, elementary row operations, row-reduced echelon matrices.
\end{flushleft}


\begin{flushleft}
Vector spaces, span of a subset, bases and dimension, quotient spaces,
\end{flushleft}


\begin{flushleft}
direct sums. Linear transformations, rank-nullity, matrix representation
\end{flushleft}


\begin{flushleft}
of a linear transformation, algebra of linear transformations,
\end{flushleft}


\begin{flushleft}
dual space, transpose of a linear transformation. Eigenvalues,
\end{flushleft}


\begin{flushleft}
eigenvectors, annihilating polynomials, Cayley-Hamilton theorem,
\end{flushleft}


\begin{flushleft}
invariant subspaces, triangulable and diagonalizable linear operators.
\end{flushleft}


\begin{flushleft}
Simultaneous triangulation and diagonalization, Primary decomposition
\end{flushleft}


\begin{flushleft}
theorem, Jordan decomposition.
\end{flushleft}





248





\begin{flushleft}
\newpage
Mathematics
\end{flushleft}





\begin{flushleft}
Inner product spaces over R (real numbers) and C (complex numbers),
\end{flushleft}


\begin{flushleft}
Gram-Schmidt orthogonalization process, orthogonal projection,
\end{flushleft}


\begin{flushleft}
best approximation. Adjoint of a linear operator, unitary and normal
\end{flushleft}


\begin{flushleft}
operators, spectral theory of normal operators. Bilinear forms,
\end{flushleft}


\begin{flushleft}
symmetric and skew-symmetric bilinear forms.
\end{flushleft}





\begin{flushleft}
MTL503 Real Analysis
\end{flushleft}


\begin{flushleft}
4 Credits (3-1-0)
\end{flushleft}


\begin{flushleft}
Elementary set theory, Countable and Uncountable sets, Real number
\end{flushleft}


\begin{flushleft}
system and its order completeness.
\end{flushleft}


\begin{flushleft}
Metric spaces, Continuous and uniformly continuous functions,
\end{flushleft}


\begin{flushleft}
Homeomorphism and isometry, Completeness, Fixed Points,
\end{flushleft}


\begin{flushleft}
Baire's Category Theorem, Totally bounded metrics, Compactness,
\end{flushleft}


\begin{flushleft}
Connectedness.
\end{flushleft}


\begin{flushleft}
Sequences and series of functions, Pointwise and uniform convergence
\end{flushleft}


\begin{flushleft}
of sequences of functions, Equicontinuity, Arzel\`{a}-Ascoli Theorem, Dini's
\end{flushleft}


\begin{flushleft}
Theorem, Stone-Weierstrass Theorems (Lattice and algebra versions).
\end{flushleft}


\begin{flushleft}
Functions of several variables, Linear transformations, Differentiation,
\end{flushleft}


\begin{flushleft}
Inverse function theorem, Implicit Function theorem, Derivatives of
\end{flushleft}


\begin{flushleft}
higher order.
\end{flushleft}





\begin{flushleft}
MTL504 Ordinary Differential Equations
\end{flushleft}


\begin{flushleft}
4 Credits (3-1-0)
\end{flushleft}


\begin{flushleft}
Initial value problems, Cauchy-Picard Theorem. General theory of linear
\end{flushleft}


\begin{flushleft}
differential systems. Sturms theory on separation and comparison
\end{flushleft}


\begin{flushleft}
properties of solutions, Boundary value problems, Green functions,
\end{flushleft}


\begin{flushleft}
Sturm-Liouville problems, Weyl-Titschmarsh theorem for unbounded
\end{flushleft}


\begin{flushleft}
interval- limit cycle, limit point cases. Power series method, regular
\end{flushleft}


\begin{flushleft}
singular points, Legendre ansBassel equations, Linear system with
\end{flushleft}


\begin{flushleft}
constant coefficients, fundamental matrix, linear systems with periodic
\end{flushleft}


\begin{flushleft}
coefficients. Critical points, phase plane analysis and concepts of
\end{flushleft}


\begin{flushleft}
linear and nonlinear stability. Autonomous systems and applications.
\end{flushleft}





\begin{flushleft}
MTL505 Computer Programming
\end{flushleft}


\begin{flushleft}
4 Credits (3-1-0)
\end{flushleft}


\begin{flushleft}
Introduction to Computers - CPU, ALU, I/O devices,
\end{flushleft}


\begin{flushleft}
Introduction to C Programming - Data types, Looping Statements,
\end{flushleft}


\begin{flushleft}
Arrays, Structure, Functions (Both simple and Recursive function), Call
\end{flushleft}


\begin{flushleft}
by Value and Call by reference, Pointers, File Handling in C
\end{flushleft}


\begin{flushleft}
Introduction to C++ Programming , Looping Statements ,arrays and
\end{flushleft}


\begin{flushleft}
Structures in C++, Functions in C++,Basic OOPS concepts.
\end{flushleft}





\begin{flushleft}
MTL506 Complex Analysis
\end{flushleft}


\begin{flushleft}
4 Credits (3-1-0)
\end{flushleft}


\begin{flushleft}
Field of complex numbers, complex plane, polar representation,
\end{flushleft}


\begin{flushleft}
stereographic projection.
\end{flushleft}


\begin{flushleft}
Analytic functions, Cauchy-Riemann equation, harmonic conjugates,
\end{flushleft}


\begin{flushleft}
power series, M\"{O}bius transforms.
\end{flushleft}


\begin{flushleft}
Contour integrals, power series representation of an analytic function,
\end{flushleft}


\begin{flushleft}
zeros of an analytic function, Liouville's theorem and applications.
\end{flushleft}


\begin{flushleft}
Index of a closed curve, Cauchy's theorem, Cauchy integral formula,
\end{flushleft}


\begin{flushleft}
Open mapping theorem, Goursat's theorem.
\end{flushleft}


\begin{flushleft}
Isolated singularities, Laurent Series, Residue theorem and application
\end{flushleft}


\begin{flushleft}
to real integrals. Meromorphic functions, Argument principle and
\end{flushleft}


\begin{flushleft}
Rouche's theorem.
\end{flushleft}


\begin{flushleft}
Maximum modulus principle and Schwarz's Lemma.
\end{flushleft}





\begin{flushleft}
MTL507 Topology
\end{flushleft}


\begin{flushleft}
4 Credits (3-1-0)
\end{flushleft}


\begin{flushleft}
Topological spaces: Definitions and Examples, Basis and Subbasis
\end{flushleft}


\begin{flushleft}
for a Topology, limit points, closure, interior; Continuous functions,
\end{flushleft}


\begin{flushleft}
Homeomorphisms; Subspace Topology, Metric Topology, Product \&
\end{flushleft}


\begin{flushleft}
Box Topology, Order Topology; Quotient spaces.
\end{flushleft}


\begin{flushleft}
Connectedness and Compactness: Connectedness, Path connectedness;
\end{flushleft}


\begin{flushleft}
Connected subspaces of the real line; Components and local
\end{flushleft}


\begin{flushleft}
connectedness; Compact spaces, Limit point compactness, Sequential
\end{flushleft}


\begin{flushleft}
compactness; Local compactness, One point compactification;
\end{flushleft}


\begin{flushleft}
Tychonoff theorem, characterizations of compact metric spaces.
\end{flushleft}


\begin{flushleft}
Countability Axioms: First countable spaces, Second countable spaces,
\end{flushleft}





\begin{flushleft}
Separable spaces, Lindeloff spaces.
\end{flushleft}


\begin{flushleft}
Separation Axioms: Hausdorff, Regular and Normal spaces; Urysohn's
\end{flushleft}


\begin{flushleft}
lemma; Uryohn's Metrization theorem; Tietze extension theorem.
\end{flushleft}


\begin{flushleft}
Completely metrizable spaces, Baire's category theorem and
\end{flushleft}


\begin{flushleft}
Function Spaces.
\end{flushleft}





\begin{flushleft}
MTL508 Mathematical Programming
\end{flushleft}


\begin{flushleft}
4 Credits (3-1-0)
\end{flushleft}


\begin{flushleft}
Linear programs formulation through examples from
\end{flushleft}


\begin{flushleft}
engineering / business decision making problems, preliminary theory
\end{flushleft}


\begin{flushleft}
and geometry of linear programs, basic feasible solution, simplex
\end{flushleft}


\begin{flushleft}
method, variants of simplex method, like two phase method and
\end{flushleft}


\begin{flushleft}
revised simplex method; duality and its principles, interpretation
\end{flushleft}


\begin{flushleft}
of dual variables, dual simplex method, primal-dual method; linear
\end{flushleft}


\begin{flushleft}
integer programs, their applications in real decision making problems,
\end{flushleft}


\begin{flushleft}
cutting plane and branch and bound methods, transportation problems,
\end{flushleft}


\begin{flushleft}
assignment problems, network maximum flow problems; complexity
\end{flushleft}


\begin{flushleft}
of simplex method, ellipsoid method, Karmarkar's interior point
\end{flushleft}


\begin{flushleft}
method; nonlinear programming, Lagrange multipliers, Farkas lemma,
\end{flushleft}


\begin{flushleft}
constraint qualification, KKT optimality conditions, sufficiency of KKT
\end{flushleft}


\begin{flushleft}
under convexity; quadratic programs, Wolfe method, applications of
\end{flushleft}


\begin{flushleft}
quadratic programs in some domains like portfolio optimization and
\end{flushleft}


\begin{flushleft}
support vector machines, etc.
\end{flushleft}





\begin{flushleft}
MTL509 Numerical Analysis
\end{flushleft}


\begin{flushleft}
4 Credits (3-1-0)
\end{flushleft}


\begin{flushleft}
Numerical Algorithms and errors, Floating point systems, Roundoff
\end{flushleft}


\begin{flushleft}
error accumulations. Interpolation: Lagrange Interpolation Newton's
\end{flushleft}


\begin{flushleft}
divided difference interpolation. Finite differences. Hermite
\end{flushleft}


\begin{flushleft}
Interpolation. Cubic splines. Numerical differentiation. Numerical
\end{flushleft}


\begin{flushleft}
Integration: Newton cotes formulas, Gaussian Quadrature composite
\end{flushleft}


\begin{flushleft}
quadrature formulas
\end{flushleft}


\begin{flushleft}
Approximation: Least squares approximation, minimum maximum error
\end{flushleft}


\begin{flushleft}
techniques. Legendre and Chebyshev polynomials. Solution of Nonlinear
\end{flushleft}


\begin{flushleft}
equations: Fixed point iteration,bisection, Secant,Regula-Falsi,
\end{flushleft}


\begin{flushleft}
Newton-Raphson methods. Solution of linear systems: Direct methods,
\end{flushleft}


\begin{flushleft}
Gauss elimination, LU and Cholesky factorizations. Iterative methods --
\end{flushleft}


\begin{flushleft}
Jacobi, Gauss- Seidel and SOR methods. System of nonlinear equation,
\end{flushleft}


\begin{flushleft}
Eigen-Value problems: Power and Inverse power method. Numerical
\end{flushleft}


\begin{flushleft}
Solution of ODE. Taylor series, Euler and Runge-Kutta methods.
\end{flushleft}





\begin{flushleft}
MTL510 Measure and Integration
\end{flushleft}


\begin{flushleft}
4 Credits (3-1-0)
\end{flushleft}


\begin{flushleft}
Outer measures, measures and measurable sets, Lebesgue measure
\end{flushleft}


\begin{flushleft}
on R, Borel measure
\end{flushleft}


\begin{flushleft}
Measurable functions, simple functions, Egoroff's theorem, Lebesgue
\end{flushleft}


\begin{flushleft}
integral and its properties, monotone convergence theorem,
\end{flushleft}


\begin{flushleft}
Fatou's Lemma, Dominated convergence theorem various modes of
\end{flushleft}


\begin{flushleft}
convergence and their relations
\end{flushleft}


\begin{flushleft}
Signed measures, Hahn and Jordan decomposition theorems,
\end{flushleft}


\begin{flushleft}
Lebesgue-Radon-Nikodym theorem, Lebesgue decomposition
\end{flushleft}


\begin{flushleft}
theorem, the representation of positive linear functionals on Cc(X)
\end{flushleft}


\begin{flushleft}
Product measures, iterated integrals, Fubini's and Tonelli's theorems
\end{flushleft}


\begin{flushleft}
Lp spaces and their completeness, conjugate space of Lp for 1 $<$ p$<$
\end{flushleft}


\begin{flushleft}
infinity, conjugate space of L1 for sigma-finite measure space
\end{flushleft}


\begin{flushleft}
Differentiation of monotone functions, functions of bounded variation,
\end{flushleft}


\begin{flushleft}
differentiation of an integral, absolute continuity.
\end{flushleft}





\begin{flushleft}
MTL601 Probability and Statistics
\end{flushleft}


\begin{flushleft}
4 Credits (3-1-0)
\end{flushleft}


\begin{flushleft}
Probability definition, conditional probability, Bayes theorem,
\end{flushleft}


\begin{flushleft}
random variables, expectation and variance, specific discrete
\end{flushleft}


\begin{flushleft}
and continuous distributions, e.g. uniform, Binomial, Poisson,
\end{flushleft}


\begin{flushleft}
geometric, Pascal, hypergeometric, exponential, normal, gamma,
\end{flushleft}


\begin{flushleft}
beta, moment generating function, Poisson process, Chebyshev's
\end{flushleft}


\begin{flushleft}
inequality, bivariate and multivariate distributions, joint, marginal
\end{flushleft}


\begin{flushleft}
and conditional distributions, order statistics, law of large numbers,
\end{flushleft}


\begin{flushleft}
central limit theorem, sampling distributions - Chi-sq, Student's t, F,
\end{flushleft}


\begin{flushleft}
theory of estimation, maximum likelihood test, testing of hypothesis,
\end{flushleft}


\begin{flushleft}
nonparametric analysis, test of goodness of fit.
\end{flushleft}





249





\begin{flushleft}
\newpage
Mathematics
\end{flushleft}





\begin{flushleft}
MTL602 Functional Analysis
\end{flushleft}


\begin{flushleft}
4 Credits (3-1-0)
\end{flushleft}


\begin{flushleft}
Normed linear spaces, Banach spaces and their examples, quotient
\end{flushleft}


\begin{flushleft}
spaces, bounded linear operators, finite dimensional Banach spaces,
\end{flushleft}


\begin{flushleft}
Lp Spaces, Lp spaces as examples for Banach spaces
\end{flushleft}


\begin{flushleft}
Hahn Banach theorems, Uniform boundedness principle, open mapping
\end{flushleft}


\begin{flushleft}
theorem, closed graph theorem, transpose of an operator
\end{flushleft}


\begin{flushleft}
Characterization of the dual of certain Banach spaces
\end{flushleft}


\begin{flushleft}
Geometry of Banach spaces - Weak and weak* convergence,
\end{flushleft}


\begin{flushleft}
Geometry of Hilbert spaces - Inner product spaces and its properties,
\end{flushleft}


\begin{flushleft}
Hilbert spaces and examples, best approximation in Hilbert spaces,
\end{flushleft}


\begin{flushleft}
orthogonal complements, orthonormal basis, dual of a Hilbert space
\end{flushleft}


\begin{flushleft}
Basic operator theory - Adjoint of an operator, self-adjoint operators,
\end{flushleft}


\begin{flushleft}
normal and unitary operators, projections
\end{flushleft}


\begin{flushleft}
Compact operators, examples and properties, spectral theorem for
\end{flushleft}


\begin{flushleft}
the compact self-adjoint operator.
\end{flushleft}





\begin{flushleft}
MTL603 Partial Differential Equations
\end{flushleft}


\begin{flushleft}
4 Credits (3-1-0)
\end{flushleft}


\begin{flushleft}
Linear and semi-linear equations, Cauchy problem, Method of
\end{flushleft}


\begin{flushleft}
characteristics. Cauchy-Kowalewsky theorem, Holmgren's Uniqueness
\end{flushleft}


\begin{flushleft}
Theorem. Classification of second order equations, wave equation in
\end{flushleft}


\begin{flushleft}
one space dimension, classical and weak solutions, Duhamel's principle.
\end{flushleft}


\begin{flushleft}
Laplace equation, fundamental solutions, maximum principles and
\end{flushleft}


\begin{flushleft}
mean value formulas, Properties of harmonic functions, Green's
\end{flushleft}


\begin{flushleft}
function, Energy methods, Perron's method, Parabolic equations in one
\end{flushleft}


\begin{flushleft}
space dimension, fundamental solution, maximum principle, existence
\end{flushleft}


\begin{flushleft}
and uniqueness theorems. Wave equation, Solutions by spherical
\end{flushleft}


\begin{flushleft}
means, Non-Homogeneous Problems, Duhamel's principle, Energy
\end{flushleft}


\begin{flushleft}
Methods. Nonlinear first order PDE's: Complete integrals, Envelopes
\end{flushleft}


\begin{flushleft}
and singular solutions. Some special methods for finding solutions:
\end{flushleft}


\begin{flushleft}
Similarity solutions, Hopf-Cole transformation.
\end{flushleft}





\begin{flushleft}
MTL625 Principles of Optimization Theory
\end{flushleft}


\begin{flushleft}
3 Credits (3-0-0)
\end{flushleft}





\begin{flushleft}
MTL712 Computational Methods for Differential
\end{flushleft}


\begin{flushleft}
Equations
\end{flushleft}


\begin{flushleft}
4 Credits (3-0-2)
\end{flushleft}


\begin{flushleft}
Pre-requisites: MTL107/MTL509
\end{flushleft}


\begin{flushleft}
Numerical methods for solving IVPs for ODEs: Difference equations,
\end{flushleft}


\begin{flushleft}
Routh-Hurwitz criterion, Test Equation. Single step methods: Taylor
\end{flushleft}


\begin{flushleft}
series method, explicit Runge-Kutta methods, convergence, order,
\end{flushleft}


\begin{flushleft}
relative and absolute stability. Multistep methods: Development
\end{flushleft}


\begin{flushleft}
of linear multistep method using interpolation and undetermined
\end{flushleft}


\begin{flushleft}
parameter approach, convergence, order, relative and absolute
\end{flushleft}


\begin{flushleft}
stability, Predictor Corrector methods. Solution of initial value problems
\end{flushleft}


\begin{flushleft}
of systems of ODES. BVP: Finite difference methods for second order
\end{flushleft}


\begin{flushleft}
ODEs, Eigenvalue problems.
\end{flushleft}


\begin{flushleft}
PDEs: Finite difference methods for Elliptic PDEs, Consistency, stability and
\end{flushleft}


\begin{flushleft}
convergnce. Boundary Conditions. FD methods for Parabolic equations
\end{flushleft}


\begin{flushleft}
in 1D and 2D. Operator splitting methods, Convergence, stability and
\end{flushleft}


\begin{flushleft}
consistency of difference methods. Higher order methods. Introduction
\end{flushleft}


\begin{flushleft}
to Hyperbolic PDEs, FD methods. Upwind schemes, Consistency,
\end{flushleft}


\begin{flushleft}
stability and convergence of schemes. Second order schemes.
\end{flushleft}





\begin{flushleft}
MTL717 Fuzzy Sets and Applications
\end{flushleft}


\begin{flushleft}
3 Credits (3-0-0)
\end{flushleft}


\begin{flushleft}
Fuzzy sets, fuzzy relations, matrix representation of fuzzy relations,
\end{flushleft}


\begin{flushleft}
fuzzy numbers, fuzzy arithmetic, Zadeh's extension principle,
\end{flushleft}


\begin{flushleft}
ordering fuzzy numbers, ranking functions; Fuzzy aggregation,
\end{flushleft}


\begin{flushleft}
t-norm, t-conorm, fuzzy negation, other aggregation operators, OWA
\end{flushleft}


\begin{flushleft}
operators; Fuzzy relational equations (FRE), algorithms to solve system
\end{flushleft}


\begin{flushleft}
of FRE; Fuzzy optimization, fuzzy linear program; Fuzzy measures,
\end{flushleft}


\begin{flushleft}
belief and plausibility, necessity and possibility , Sugeno and Choquet
\end{flushleft}


\begin{flushleft}
integrals on finite sets; Fuzzy logic and approximate reasoning, Ifthen-else rules, Mamdani model, TSK model, SAM model; Applications
\end{flushleft}


\begin{flushleft}
of fuzzy sets and logics in areas of image processing, control, AI,
\end{flushleft}


\begin{flushleft}
computing with words, etc.; Generalized fuzzy sets - like type2 fuzzy
\end{flushleft}


\begin{flushleft}
sets, rough sets, and Intuitionistic fuzzy sets.
\end{flushleft}





\begin{flushleft}
Convex set, hyperplane, relative interior and closure, separation
\end{flushleft}


\begin{flushleft}
theorems, theorems of alternatives for linear systems, convex
\end{flushleft}


\begin{flushleft}
functions and properties of continuity, differentiability etc.,
\end{flushleft}


\begin{flushleft}
quasiconvex and pseudoconvex functions and their properties and
\end{flushleft}


\begin{flushleft}
interrelationships, minimax theorems for convex and quasiconvex
\end{flushleft}


\begin{flushleft}
functions, nonlinear programming, Lagrange function, saddle point,
\end{flushleft}


\begin{flushleft}
Fritz John optimality conditions, constraint qualifications, KarushKuhn-Tucker (KKT) necessary and sufficient optimality conditions,
\end{flushleft}


\begin{flushleft}
Wolfe and Mond-Weir duals, Wolfe method for quadratic programs,
\end{flushleft}


\begin{flushleft}
Projection gradient method, steepest descent method, conjugate
\end{flushleft}


\begin{flushleft}
gradient method, rank-1 methods, convergence, conjugate function,
\end{flushleft}


\begin{flushleft}
Fenchel duality, subgradient and subdifferential, nonsmooth
\end{flushleft}


\begin{flushleft}
optimization, tangent cone, normal cone, nonsmooth KKT conditions,
\end{flushleft}


\begin{flushleft}
nonsmooth optimality conditions, subgradient method, proximal
\end{flushleft}


\begin{flushleft}
method, convergence of these methods, applications to support
\end{flushleft}


\begin{flushleft}
vector machines optimization problems.
\end{flushleft}





\begin{flushleft}
MTL720 Neurocomputing and Applications
\end{flushleft}


\begin{flushleft}
3 Credits (3-0-0)
\end{flushleft}





\begin{flushleft}
MTD701 Project-I
\end{flushleft}


\begin{flushleft}
5 Credits (0-0-10)
\end{flushleft}





\begin{flushleft}
Stochastic processes, specification of stochastic processes, stationary
\end{flushleft}


\begin{flushleft}
processes, discrete time and continuous time Markov chains, birth and
\end{flushleft}


\begin{flushleft}
death processes, applications in queueing theory. Markov processes
\end{flushleft}


\begin{flushleft}
with continuous state space, martingales, applications in financial
\end{flushleft}


\begin{flushleft}
mathematics. Renewal processes and theory, Markov renewal and
\end{flushleft}


\begin{flushleft}
semi-Markov processes, branching processes.
\end{flushleft}





\begin{flushleft}
MTD702 Project-II
\end{flushleft}


\begin{flushleft}
6 Credits (0-0-12)
\end{flushleft}


\begin{flushleft}
MTL704 Numerical Optimization
\end{flushleft}


\begin{flushleft}
3 Credits (3-0-0)
\end{flushleft}


\begin{flushleft}
Pre-requisites: MTL103/MTL508
\end{flushleft}





\begin{flushleft}
Biological and Artificial Neuron, Perceptron model, Adaline model,
\end{flushleft}


\begin{flushleft}
Multilayered feedforward networks, Activation functions, Backpropagation algorithm and its improvements, Conjugate Gradient
\end{flushleft}


\begin{flushleft}
Neural Network, Applications of Back-propagation algorithm to
\end{flushleft}


\begin{flushleft}
Statistical Pattern Classification, Feature selection, Classification and
\end{flushleft}


\begin{flushleft}
regression problems, General Regression Neural Networks, Hopfield
\end{flushleft}


\begin{flushleft}
Network, Recurrent networks, Probabilistic Neural Networks, Kohonen's
\end{flushleft}


\begin{flushleft}
self-organizing maps with quadratic junctions and its applications to
\end{flushleft}


\begin{flushleft}
character recognition, Adaptive Resonance Theory model, Applications
\end{flushleft}


\begin{flushleft}
of ART model for knowledge acquisition.
\end{flushleft}





\begin{flushleft}
MTL725 Stochastic Processes and its Applications
\end{flushleft}


\begin{flushleft}
3 Credits (3-0-0)
\end{flushleft}





\begin{flushleft}
MTL728 Category Theory
\end{flushleft}


\begin{flushleft}
3 Credits (3-0-0)
\end{flushleft}





\begin{flushleft}
Unconstrained optimization techniques - one dimensional methods
\end{flushleft}


\begin{flushleft}
like Fibonacci method, Golden section method; higher dimension
\end{flushleft}


\begin{flushleft}
methods: pattern search method, Nelder and Meed method; gradient
\end{flushleft}


\begin{flushleft}
based methods: Steepest descent method, Newton method, Conjugate
\end{flushleft}


\begin{flushleft}
direction and gradient method, Quasi-Newton methods. Constrained
\end{flushleft}


\begin{flushleft}
optimization techniques - penalty method, barrier method, cutting
\end{flushleft}


\begin{flushleft}
plane method, projection gradient method. Heuristic technique: like
\end{flushleft}


\begin{flushleft}
Genetic programming method to solve non-convex programs.
\end{flushleft}





\begin{flushleft}
Categories, functors and natural transformations, adjoints (of
\end{flushleft}


\begin{flushleft}
functors), representable functors, Yoneda Lemma and applications.
\end{flushleft}


\begin{flushleft}
Limits and colimits, interaction between functors and limits. Limits in
\end{flushleft}


\begin{flushleft}
terms of representables and adjoints, limits and colimits of presheaves,
\end{flushleft}


\begin{flushleft}
interaction between adjoint functors and limits. Application to abelian
\end{flushleft}


\begin{flushleft}
category: complexes of R-modules, long exact sequence, mapping
\end{flushleft}


\begin{flushleft}
cone and cylinder, projective and injective resolution, derived functors,
\end{flushleft}


\begin{flushleft}
right and left exactness, Ext and Tor. Concept of presheaf and sheaf,
\end{flushleft}


\begin{flushleft}
group scheme and Hopf algebra.
\end{flushleft}





250





\begin{flushleft}
\newpage
Mathematics
\end{flushleft}





\begin{flushleft}
MTL729 Computational Algebra and its Applications
\end{flushleft}


\begin{flushleft}
3 Credits (3-0-0)
\end{flushleft}


\begin{flushleft}
Finite fields: Construction and examples. Polynomials over finite
\end{flushleft}


\begin{flushleft}
fields, their factorization/irreducibility and their applications to
\end{flushleft}


\begin{flushleft}
coding theory. Combinatorial applications. Symmetric and Public key
\end{flushleft}


\begin{flushleft}
cryptosystems particularly on Elliptic curves. Combinatorial group
\end{flushleft}


\begin{flushleft}
theory: investigation of groups on computers, finitely presented
\end{flushleft}


\begin{flushleft}
groups, coset enumeration. Fundamental problem of combinatorial
\end{flushleft}


\begin{flushleft}
group theory. Coset enumeration, Nielsen transformations. Braid
\end{flushleft}


\begin{flushleft}
Group cryptography. Automorphism groups. Computational methods
\end{flushleft}


\begin{flushleft}
for determining automorphism groups of certain finite groups.
\end{flushleft}


\begin{flushleft}
Computations of characters and representations of finite groups.
\end{flushleft}


\begin{flushleft}
Computer algebra programs. Computations of units in rings and group
\end{flushleft}


\begin{flushleft}
rings. Calculations in Lie algebras.
\end{flushleft}





\begin{flushleft}
MTL730 Cryptography
\end{flushleft}


\begin{flushleft}
3 Credits (3-0-0)
\end{flushleft}


\begin{flushleft}
Overlaps with: COL759
\end{flushleft}


\begin{flushleft}
Applying the corresponding algorithms programmes. (laboratory/
\end{flushleft}


\begin{flushleft}
design activities could also be included) Classical cryptosystems,
\end{flushleft}


\begin{flushleft}
Preview from number theory, Congruences and residue class rings,
\end{flushleft}


\begin{flushleft}
DES- security and generalizations, Prime number generation. Public
\end{flushleft}


\begin{flushleft}
Key Cryptosystems of RSA, Rabin, etc. their security and cryptanalysis.
\end{flushleft}


\begin{flushleft}
Primality, factorization and quadratic sieve, efficiency of other factoring
\end{flushleft}


\begin{flushleft}
algorithms.
\end{flushleft}


\begin{flushleft}
Finite fields: Construction and examples. Diffie-Hellman key exchange.
\end{flushleft}


\begin{flushleft}
Discrete logarithm problem in general and on finite fields.
\end{flushleft}


\begin{flushleft}
Cryptosystems based on Discrete logarithm problem such as MasseyOmura cryptosystems. Algorithms For finding discrete logarithms,
\end{flushleft}


\begin{flushleft}
their analysis. Polynomials on finite fields and Their factorization/
\end{flushleft}


\begin{flushleft}
irreducibility and their application to coding theory.
\end{flushleft}


\begin{flushleft}
Elliptic curves, Public key cryptosystems particularly on Elliptic curves.
\end{flushleft}


\begin{flushleft}
Problems of key exchange, discrete logarithms and the elliptic curve
\end{flushleft}


\begin{flushleft}
logarithm problem.
\end{flushleft}


\begin{flushleft}
Implementation of elliptic curve cryptosystems. Counting of points on
\end{flushleft}


\begin{flushleft}
Elliptic Curves over Galois Fields of order 2m. Other systems such as
\end{flushleft}


\begin{flushleft}
Hyper Elliptic Curves And cryptosystems based on them. Combinatorial
\end{flushleft}


\begin{flushleft}
group theory: investigation of groups on computers, finitely presented
\end{flushleft}


\begin{flushleft}
groups, coset enumeration. Fundamental problems of combinatorial
\end{flushleft}


\begin{flushleft}
group theory. Coset enumeration, Nielsen and Tietze transformations.
\end{flushleft}


\begin{flushleft}
Braid Group cryptography.
\end{flushleft}


\begin{flushleft}
Cryptographic hash functions. Authentication, Digital Signatures,
\end{flushleft}


\begin{flushleft}
Identification, certification infrastructure and other applied aspects.
\end{flushleft}





\begin{flushleft}
MTL731 Introduction to Chaotic Dynamical Systems
\end{flushleft}


\begin{flushleft}
3 Credits (3-0-0)
\end{flushleft}


\begin{flushleft}
Topics to be covered include chaos, elementary bifurcations.
\end{flushleft}


\begin{flushleft}
Sarkovski's theorem, recurrence and equidistribution, codes, symbolic
\end{flushleft}


\begin{flushleft}
dynamics and chaotic behaviour. Higher dimensional dynamics,
\end{flushleft}


\begin{flushleft}
including horseshoes, Henon map. Stability of systems.
\end{flushleft}





\begin{flushleft}
MTL732 Financial Mathematics
\end{flushleft}


\begin{flushleft}
3 Credits (3-0-0)
\end{flushleft}


\begin{flushleft}
Pre-requisites: MTL103/MTL508
\end{flushleft}


\begin{flushleft}
Overlaps with: MCL363/MSL873
\end{flushleft}


\begin{flushleft}
Financial markets, Interest computation, value, growth and discount
\end{flushleft}


\begin{flushleft}
factors, derivative products, basic option theory: single and multiperiod binomial pricing models, Cox-Ross-Rubinstein (CRR) model,
\end{flushleft}


\begin{flushleft}
volatility, Black-Scholes formula for option pricing as a limit of CRR
\end{flushleft}


\begin{flushleft}
model, Greeks and hedging, Mean-Variance portfolio theory: Markowitz
\end{flushleft}


\begin{flushleft}
model, Capital Asset Pricing Model (CAPM), factor models, interest
\end{flushleft}


\begin{flushleft}
rates and interest rate derivatives, Binomial tree models.
\end{flushleft}





\begin{flushleft}
MTL733 Stochastic of Finance
\end{flushleft}


\begin{flushleft}
3 Credits (3-0-0)
\end{flushleft}


\begin{flushleft}
Pre-requisites: MTL106/MTL601
\end{flushleft}


\begin{flushleft}
Stochastic Processes; Brownian and Geometric Brownian Motion;
\end{flushleft}


\begin{flushleft}
Levy Processes, Jump-Diffusion Processes; Conditional Expectations
\end{flushleft}





\begin{flushleft}
and Martingales; Ito Integrals, Ito's Formula; Stochastic Differential
\end{flushleft}


\begin{flushleft}
Equations; Change of Measure, Girsanov Theorem, Martingale
\end{flushleft}


\begin{flushleft}
Representation Theorem and Feymann-Kac Theorem; Applications
\end{flushleft}


\begin{flushleft}
of Stochastic Calculus in Finance, Option Pricing, Interest Rate
\end{flushleft}


\begin{flushleft}
Derivatives, Levy Processes in Credit Risk.
\end{flushleft}





\begin{flushleft}
MTL735 Advanced Number Theory
\end{flushleft}


\begin{flushleft}
3 Credits (3-0-0)
\end{flushleft}


\begin{flushleft}
Overlaps with: MTL145
\end{flushleft}


\begin{flushleft}
Divisibility, prime numbers, Bertrand's theorem, Congruences,
\end{flushleft}


\begin{flushleft}
complete \& reduced residue systems, theorems of Fermat, Euler,
\end{flushleft}


\begin{flushleft}
Wilson \& Wolstenholme, solutions of general congruences, study
\end{flushleft}


\begin{flushleft}
of linear and system of linear congruences, Chinese Remainder
\end{flushleft}


\begin{flushleft}
theorem, study of quadratic congruences, Quadratic, Cubic \&
\end{flushleft}


\begin{flushleft}
Biquadratic Reciprocity laws, binary and ternary quadratic forms,
\end{flushleft}


\begin{flushleft}
Continued fractions, Diophantine approximations and applications
\end{flushleft}


\begin{flushleft}
to linear and Pell's equations, Arithmetical functions, properties,
\end{flushleft}


\begin{flushleft}
rate of growth, Distribution of primes, Dirichlet's theorem on primes
\end{flushleft}


\begin{flushleft}
in arithmetic progression, Prime Number theorem, Diophantine
\end{flushleft}


\begin{flushleft}
equations, special cases of the Fermat equation, introduction to
\end{flushleft}


\begin{flushleft}
classic and modern techniques.
\end{flushleft}





\begin{flushleft}
MTL737 Differential Geometry
\end{flushleft}


\begin{flushleft}
3 Credits (3-0-0)
\end{flushleft}


\begin{flushleft}
Curves in plane and space, arc-length, reparametrization, curvature
\end{flushleft}


\begin{flushleft}
of a plane cure, curvature and torsion of a space curve
\end{flushleft}


\begin{flushleft}
Simple closed curves, isoperimetric inequality, Four-vertex theorem
\end{flushleft}


\begin{flushleft}
Surfaces, smooth surfaces and examples, level surfaces, quadric
\end{flushleft}


\begin{flushleft}
surfaces, surfaces of revolution, ruled surfaces smooth maps, tangent
\end{flushleft}


\begin{flushleft}
space, derivatives, orientability of surfaces
\end{flushleft}


\begin{flushleft}
The first fundamental form, lengths of curves on surfaces, isometries,
\end{flushleft}


\begin{flushleft}
conformal mappings, equiareal maps
\end{flushleft}


\begin{flushleft}
The second fundamental form, Gauss and Weingarten maps, normal
\end{flushleft}


\begin{flushleft}
and geodesic curvatures, Gaussian and mean curvatures, principal
\end{flushleft}


\begin{flushleft}
curvatures
\end{flushleft}


\begin{flushleft}
Surfaces of constant Gaussian curvature, surfaces of constant mean
\end{flushleft}


\begin{flushleft}
curvature, flat surfaces
\end{flushleft}


\begin{flushleft}
Parallel transport, geodesics and their examples, properties, geodesic
\end{flushleft}


\begin{flushleft}
equations, geodesics as shortest paths, Gauss and Codazzi-Minardi
\end{flushleft}


\begin{flushleft}
equations, Theorema Egregium
\end{flushleft}


\begin{flushleft}
Gauss-Bonnet Theorem. Introduction to hyperbolic and spherical
\end{flushleft}


\begin{flushleft}
geometry.
\end{flushleft}





\begin{flushleft}
MTL738 Commutative Algebra
\end{flushleft}


\begin{flushleft}
3 Credits (3-0-0)
\end{flushleft}


\begin{flushleft}
Pre-requisites: MTL105/MTL501
\end{flushleft}


\begin{flushleft}
Revision of Rings and Ideals: Prime and maximal ideals. Chinese
\end{flushleft}


\begin{flushleft}
remainder theorem, Nilradical, Jacobson radical, operations on ideals,
\end{flushleft}


\begin{flushleft}
extension and contraction; Module, submodule, quotient module, sums
\end{flushleft}


\begin{flushleft}
and products, Nakayama's lemma; Homomorphism, kernel, cokernel,
\end{flushleft}


\begin{flushleft}
direct sum, direct product, universal properties, free module, exact
\end{flushleft}


\begin{flushleft}
sequences, tensor product of modules and its exactness property;
\end{flushleft}


\begin{flushleft}
Rings and modules of fractions and functorial properties of fractions;
\end{flushleft}


\begin{flushleft}
Primary decomposition; Integral dependence, going-up and going
\end{flushleft}


\begin{flushleft}
down theorems, valuation rings; Chain conditions, Noetherian rings,
\end{flushleft}


\begin{flushleft}
Artinian rings, discrete valuation ring and Dedikind domains, fractional
\end{flushleft}


\begin{flushleft}
ideals; Completion: filtration, graded rings and modules.
\end{flushleft}





\begin{flushleft}
MTL739 Representation of Finite Groups
\end{flushleft}


\begin{flushleft}
3 Credits (3-0-0)
\end{flushleft}


\begin{flushleft}
Pre-requisites: MTL105/MTL501
\end{flushleft}


\begin{flushleft}
Revision of basic group theory. Definition and examples of representation.
\end{flushleft}


\begin{flushleft}
Subrepresentation, sum and tensor product of represenations,
\end{flushleft}


\begin{flushleft}
irreducible representations; Character Theory: Character of a
\end{flushleft}


\begin{flushleft}
representation, Schur's Lemma, Maschke's theorem, Orthogonality
\end{flushleft}


\begin{flushleft}
relations for characters, decomposition of regular representation,
\end{flushleft}


\begin{flushleft}
number of irreducible representations of a group; Representation of
\end{flushleft}


\begin{flushleft}
subgroups and product of groups, induced representations; Group
\end{flushleft}


\begin{flushleft}
Algebra: Representations and modules; Decomposition of complex
\end{flushleft}





251





\begin{flushleft}
\newpage
Mathematics
\end{flushleft}





\begin{flushleft}
algebra C[G] and Integrability properties of characters. Induced
\end{flushleft}


\begin{flushleft}
representations, restriction to subgroups, Reciprocity formula, Mackey's
\end{flushleft}


\begin{flushleft}
irreducibility criterion; Irreducible representations of symmertric groups
\end{flushleft}


\begin{flushleft}
(S\_n) and alternating groups (A\_n).
\end{flushleft}





\begin{flushleft}
MTL741 Fractal Geometry
\end{flushleft}


\begin{flushleft}
3 Credits (3-0-0)
\end{flushleft}


\begin{flushleft}
Code spaces, Hausdorff metric, Hausdorff measures, fractal
\end{flushleft}


\begin{flushleft}
dimensions, Hausdorff dimension, Box-counting dimensions, groups
\end{flushleft}


\begin{flushleft}
and rings of fractal dimension, semigroups of iterated function
\end{flushleft}


\begin{flushleft}
schemes(IFS) and self-similarity, Cantor sets, Cantor dusts, Koch
\end{flushleft}


\begin{flushleft}
Snowflake, Sierpinski's triangle, Diophantine approximation, chaos
\end{flushleft}


\begin{flushleft}
games, attractors, fractals, superfractals and multi fractal measures,
\end{flushleft}


\begin{flushleft}
Mandelbrot and Julia sets, random fractals, fractals in Brownian motion.
\end{flushleft}





\begin{flushleft}
MTL742 Operator Theory
\end{flushleft}


\begin{flushleft}
3 Credits (3-0-0)
\end{flushleft}


\begin{flushleft}
Pre-requisites: MTL411/MTL602
\end{flushleft}





\begin{flushleft}
The structure of cyclic codes, encoding and decoding with a
\end{flushleft}


\begin{flushleft}
cyclic code, minimal codes, Some special cyclic codes including
\end{flushleft}


\begin{flushleft}
BCH codes and their decoding algorithm, Reed-Solomon codes,
\end{flushleft}


\begin{flushleft}
quadratic residue codes. Burst errors, Burst error-correcting codes,
\end{flushleft}


\begin{flushleft}
decoding of cyclic burst-error-correcting codes. Generalized ReedSolomon codes, Alternant codes, Goppa codes, Sudan decoding for
\end{flushleft}


\begin{flushleft}
generalized RS codes.
\end{flushleft}





\begin{flushleft}
MTL745 Advanced Matrix Theory
\end{flushleft}


\begin{flushleft}
3 Credits (3-0-0)
\end{flushleft}


\begin{flushleft}
Review of Linear Algebra; Matrix calculus, Diagonalization, Canonical
\end{flushleft}


\begin{flushleft}
forms and invariant Factors. Quadratic forms, Courant-Fischer
\end{flushleft}


\begin{flushleft}
minimax and related Theorems. Perron-Frobenius theory, Matrix
\end{flushleft}


\begin{flushleft}
stability, Inequalities, g-inverses. Direct, iterative, projection
\end{flushleft}


\begin{flushleft}
and rotation methods for solving linear systems and eigenvalue
\end{flushleft}


\begin{flushleft}
problems. Applications.
\end{flushleft}





\begin{flushleft}
MTL746 Methods of Applied Mathematics
\end{flushleft}


\begin{flushleft}
3 Credits (3-0-0)
\end{flushleft}





\begin{flushleft}
Weak and weak*-topologies, closed convex sets, weak compactness,
\end{flushleft}


\begin{flushleft}
Alaoglu's theorem, locally convex topologies, separation of points by
\end{flushleft}


\begin{flushleft}
linear functionals, Krein-Milman theorem, Stone-Weierstrass theorem.
\end{flushleft}





\begin{flushleft}
Expansion in Eigen functions, Fourier series and Fourier Integral,
\end{flushleft}


\begin{flushleft}
orthogonal expansion, mean square approximation, completeness,
\end{flushleft}


\begin{flushleft}
orthogonal polynomials and their properties.
\end{flushleft}





\begin{flushleft}
Normed algebras, resolvent, spectrum, spectral radius, functional
\end{flushleft}


\begin{flushleft}
calculus, spectral mapping theorem, Gelfand's theory of commutative
\end{flushleft}


\begin{flushleft}
Banach algebras.
\end{flushleft}





\begin{flushleft}
Integral transform and their applications.
\end{flushleft}





\begin{flushleft}
Basic properties of compact operators, spectral theory of compact
\end{flushleft}


\begin{flushleft}
operators, Fredholm alternative, General theory of Schatten-von
\end{flushleft}


\begin{flushleft}
Neumann classes, Hilbert-Schmidt operators, trace and trace duality
\end{flushleft}


\begin{flushleft}
in finite dimensions, duality for Schatten-von Neumann classes.
\end{flushleft}





\begin{flushleft}
Integral equations of voltera and Fredhlom type, seperable and
\end{flushleft}


\begin{flushleft}
symmetric kernels, Hilbert-Schmidth theory, Singular integral
\end{flushleft}


\begin{flushleft}
equations, approximation methods of solving integral equations.
\end{flushleft}





\begin{flushleft}
Functional calculus for self-adjoint operators, square root of positive
\end{flushleft}


\begin{flushleft}
operators, polar decomposition, some topologies on B(H), spectral
\end{flushleft}


\begin{flushleft}
measures, the spectral theorem for normal operators.
\end{flushleft}





\begin{flushleft}
MTL743 Fourier Analysis
\end{flushleft}


\begin{flushleft}
3 Credits (3-0-0)
\end{flushleft}


\begin{flushleft}
Pre-requisites: MTL122/MTL503
\end{flushleft}


\begin{flushleft}
Fourier Series - Definition, uniqueness, convolution, summability,
\end{flushleft}


\begin{flushleft}
convergence of Fourier series, Fourier series for square integrable
\end{flushleft}


\begin{flushleft}
functions, Plancheral theorem, Riesz-Fischer theorem, Gibb's
\end{flushleft}


\begin{flushleft}
phenomenon, divergence of Fourier series
\end{flushleft}


\begin{flushleft}
Applications of Fourier series -- Isoperimetric inequality, Weierstrass
\end{flushleft}


\begin{flushleft}
approximation theorem, Weyl's equidistribution theorem, heat equation
\end{flushleft}


\begin{flushleft}
on the circle.
\end{flushleft}


\begin{flushleft}
Fourier transform -- Schwartz space on R, Fourier transform on the
\end{flushleft}


\begin{flushleft}
Schwartz space, Fourier transform of integrable and square-integrable
\end{flushleft}


\begin{flushleft}
functions, Poisson summation formula.
\end{flushleft}


\begin{flushleft}
Tempered distributions -- Topology on the Schwartz space, tempered
\end{flushleft}


\begin{flushleft}
distributions and its properties, Fourier transform of tempered
\end{flushleft}


\begin{flushleft}
distributions.
\end{flushleft}


\begin{flushleft}
Applications -- Uncertainty principle, Paley-Wiener theorem, Wiener's
\end{flushleft}


\begin{flushleft}
theorem, Shannon sampling theorem, multiplier theorem for
\end{flushleft}


\begin{flushleft}
integrable functions.
\end{flushleft}





\begin{flushleft}
MTL744 Mathematical Theory of Coding
\end{flushleft}


\begin{flushleft}
3 Credits (3-0-0)
\end{flushleft}


\begin{flushleft}
Pre-requisites: MTL105/MTL501
\end{flushleft}


\begin{flushleft}
Overlaps with: ELL710
\end{flushleft}


\begin{flushleft}
Review of communication channels, maximum likelihood and nearest
\end{flushleft}


\begin{flushleft}
neighbour decoding schemes, Hamming distance, Distance of a code.
\end{flushleft}


\begin{flushleft}
Structure of finite fields,Linear codes and their duals, Equivalence of
\end{flushleft}


\begin{flushleft}
linear codes, encoding with a linear code, decoding of a linear code,
\end{flushleft}


\begin{flushleft}
ISBN Code, Hamming codes. Hadamard matrix codes, Golay codes,
\end{flushleft}


\begin{flushleft}
Codes and Latin squares. Non-linear codes, Nordstrom-Robinson code,
\end{flushleft}


\begin{flushleft}
Kerdock codes, Preparata codes. Bounds in coding theory: Spherecovering bound, Hamming bound and perfect codes, Singleton bound
\end{flushleft}


\begin{flushleft}
and MDS codes, Gilbert-Varshamov bound, Plotkin bound, Griesmer
\end{flushleft}


\begin{flushleft}
bound. Weight enumerators, MacWilliams Identity. Construction of
\end{flushleft}


\begin{flushleft}
new codes: Propagation rules, Reed-Muller codes, subfield codes.
\end{flushleft}





\begin{flushleft}
Linear function, general variation of a functional, direct variation
\end{flushleft}


\begin{flushleft}
methods for solution of boundary value problems.
\end{flushleft}





\begin{flushleft}
MTL747 Mathematical Logic
\end{flushleft}


\begin{flushleft}
3 Credits (3-0-0)
\end{flushleft}


\begin{flushleft}
Propositional Logic - Syntax, Semantics and Normal Forms, First Order
\end{flushleft}


\begin{flushleft}
Logic Syntax, Semantics and Normal Forms, Herbrand interpretation,
\end{flushleft}


\begin{flushleft}
Resolution of PL and FL, Proofs in PL and FL, Axiomatic Systems,
\end{flushleft}


\begin{flushleft}
Adequacy and Compactness, Program Verification, Hoare Proof, Godels
\end{flushleft}


\begin{flushleft}
completeness and incompleteness Theorem, Turing Machines and
\end{flushleft}


\begin{flushleft}
undecidability of Predicate calculus, Gentzen systems, Introduction to
\end{flushleft}


\begin{flushleft}
other logics - Description Logic, Default \& Defeasible Logic, Courteous
\end{flushleft}


\begin{flushleft}
Logic, Modal Logic, Fuzzy logic.
\end{flushleft}





\begin{flushleft}
MTL751 Symbolic Dynamics
\end{flushleft}


\begin{flushleft}
3 Credits (3-0-0)
\end{flushleft}


\begin{flushleft}
Shift Spaces, languages, subshifts of finite type, their graph
\end{flushleft}


\begin{flushleft}
representation, sofic shifts, their presentation and characterization,
\end{flushleft}


\begin{flushleft}
entropy, its properties, conjugacy, shift equivalence and dimension
\end{flushleft}


\begin{flushleft}
groups, zeta functions.
\end{flushleft}





\begin{flushleft}
MTL754 Principles of Computer Graphics
\end{flushleft}


\begin{flushleft}
3 Credits (3-0-0)
\end{flushleft}


\begin{flushleft}
Overlaps with: COL781, ELL792
\end{flushleft}


\begin{flushleft}
Overview of Graphics Systems; Raster Graphics: line and circle
\end{flushleft}


\begin{flushleft}
drawing algorithms, Windowing and clipping: Cohen - Sutherland line
\end{flushleft}


\begin{flushleft}
clipping, Cyrus beck clipping method, Polygon Clipping; 2D and 3D
\end{flushleft}


\begin{flushleft}
Geometrical Transformations: scaling, translation, rotation, reflection;
\end{flushleft}


\begin{flushleft}
3D Object representation: Curves and Surfaces: cubic splines, Bezier
\end{flushleft}


\begin{flushleft}
curves B-splines, surface of revolution, sweep surfaces;, viewing
\end{flushleft}


\begin{flushleft}
Transformations: parallel and perspective projection; Hidden line/
\end{flushleft}


\begin{flushleft}
surface removal methods; illuminations model; shading: Gouraud,
\end{flushleft}


\begin{flushleft}
Phong; Introduction to Ray-tracing; Programming practices with
\end{flushleft}


\begin{flushleft}
standard graphics libraries like open GL.
\end{flushleft}





\begin{flushleft}
MTL755 Algebraic Geometry
\end{flushleft}


\begin{flushleft}
3 Credits (3-0-0)
\end{flushleft}


\begin{flushleft}
Pre-requisites: MTL105/MTL501
\end{flushleft}


\begin{flushleft}
Rings of polynomials and their quotients, local rings, DVR, modules,
\end{flushleft}


\begin{flushleft}
free modules, exact sequences. Affine algebraic sets, The Hilbert
\end{flushleft}


\begin{flushleft}
basis theorem. Hilbert's Nullstellansatz. Affine varieties: Coordinate
\end{flushleft}


\begin{flushleft}
rings, polynomial maps, coordinate changes, rational functions.
\end{flushleft}


\begin{flushleft}
Local Properties of plane curves: Multiple points, tangent lines,
\end{flushleft}





252





\begin{flushleft}
\newpage
Mathematics
\end{flushleft}





\begin{flushleft}
multiplicities and local rings, intersection number. Projective varieties:
\end{flushleft}


\begin{flushleft}
projective algebraic sets, projective plane curves, linear systems of
\end{flushleft}


\begin{flushleft}
curves, Bezout's theorem, Max Noether's fundamental theorem and
\end{flushleft}


\begin{flushleft}
its applications. Variety, Morphisms and Rational maps: The Zariski
\end{flushleft}


\begin{flushleft}
topology, varieties and their morphism, dimension of varieties,
\end{flushleft}


\begin{flushleft}
rational maps. Resolution of sigularies: Blowing up a point in affine
\end{flushleft}


\begin{flushleft}
and projective planes, quadratic transformations and nonsingular
\end{flushleft}


\begin{flushleft}
models of curves.
\end{flushleft}





\begin{flushleft}
MTL756 Lie Algebras and Lie Groups
\end{flushleft}


\begin{flushleft}
3 Credits (3-0-0)
\end{flushleft}


\begin{flushleft}
Pre-requisites: MTL105/MTL501
\end{flushleft}


\begin{flushleft}
Overlaps with: MTL856
\end{flushleft}


\begin{flushleft}
Definition and examples, solvable and nilpotent Lie algebras, the Engel's
\end{flushleft}


\begin{flushleft}
theorem, Lie's theorem, Cartan's theorem, killing form. Representation
\end{flushleft}


\begin{flushleft}
theory of finite dimensional semisimple Lie algebras. The Weyl's
\end{flushleft}


\begin{flushleft}
theorem, representations of sl(2,C), root space decomposition.
\end{flushleft}


\begin{flushleft}
Weyl group, Cartan subalgebras and classification of root systems;
\end{flushleft}


\begin{flushleft}
Definition and examples of matrix Lie groups. Exponential mapping,
\end{flushleft}


\begin{flushleft}
Baker-Campbell-Hausdorff formula. Representation theory of matrix
\end{flushleft}


\begin{flushleft}
Lie groups. Representation theory of SU(2) and SU(3).
\end{flushleft}





\begin{flushleft}
MTL757 Introduction to Algebraic Topology
\end{flushleft}


\begin{flushleft}
3 Credits (3-0-0)
\end{flushleft}


\begin{flushleft}
Pre-requisites: MTL122/MTL507
\end{flushleft}


\begin{flushleft}
Homotopy of paths, fundamental group, covering spaces, fundamental
\end{flushleft}


\begin{flushleft}
group of the circle, Retraction and application, van Kampen theorem
\end{flushleft}


\begin{flushleft}
and application. Universal cover and classification of covering spaces.
\end{flushleft}


\begin{flushleft}
Deck transformation and group actions. Simplicial and Singular
\end{flushleft}


\begin{flushleft}
homology, homotopy invariance, exact sequences- Mayer-Vietories
\end{flushleft}


\begin{flushleft}
Sequences, the equivalence of simplicial and singular homology.
\end{flushleft}





\begin{flushleft}
MTL760 Advanced Algorithms
\end{flushleft}


\begin{flushleft}
3 Credits (3-0-0)
\end{flushleft}


\begin{flushleft}
Pre-requisites: MTL342
\end{flushleft}


\begin{flushleft}
Overlaps with: COL758
\end{flushleft}





\begin{flushleft}
Two dimension random variables, joint distributions, marginal
\end{flushleft}


\begin{flushleft}
distributions, operations on random variables and their corresponding
\end{flushleft}


\begin{flushleft}
distributions, multidimensional random variables and their distributions.
\end{flushleft}


\begin{flushleft}
Expectation of a random variable, expectation of a discrete and
\end{flushleft}


\begin{flushleft}
a continuous random variable, moments and moment generating
\end{flushleft}


\begin{flushleft}
function, correlation, covariance and regression.
\end{flushleft}


\begin{flushleft}
Various modes of convergence, Weak law of large numbers, strong
\end{flushleft}


\begin{flushleft}
law of large numbers.
\end{flushleft}


\begin{flushleft}
Convergence in distribution, weak convergence of generalized
\end{flushleft}


\begin{flushleft}
distributions, Helly-Bray theorems, Scheffe's theorem.
\end{flushleft}


\begin{flushleft}
Characteristic function -- definition and examples, properties,
\end{flushleft}


\begin{flushleft}
uniqueness and inversion theorems, moments using characteristic
\end{flushleft}


\begin{flushleft}
function, Paul Levy's continuity property of characteristic functions,
\end{flushleft}


\begin{flushleft}
characterization of independent random variables.
\end{flushleft}


\begin{flushleft}
Central limit theorem -- Liapunov's and Lindberg's condition, LindebergLevy form.
\end{flushleft}


\begin{flushleft}
Infinite divisibility, Levy-Khintchine theorem.
\end{flushleft}





\begin{flushleft}
MTL763 Introduction to Game Theory
\end{flushleft}


\begin{flushleft}
3 Credits (3-0-0)
\end{flushleft}


\begin{flushleft}
Game Trees, Choice Functions and Strategies, Choice Subtrees,
\end{flushleft}


\begin{flushleft}
Equilibrium N-tuples Strategies, Normal Forms, Non-cooperative
\end{flushleft}


\begin{flushleft}
games, Nash Equilibrium and its computation, The von Neumann
\end{flushleft}


\begin{flushleft}
Minimax Theorem, Mixed strategies, Best Response Strategies, Matrix
\end{flushleft}


\begin{flushleft}
Games and Linear Programming, Simplex Algorithm, Avoiding cycles
\end{flushleft}


\begin{flushleft}
and Achieving Feasibility, Dual-Simplex Algorithm, Duality Theorem,
\end{flushleft}


\begin{flushleft}
2x2 Bimatrix Games, Nonlinear Programming Methods for Non-zero
\end{flushleft}


\begin{flushleft}
Sum Two-Person Games, Coalitions and Characteristic Functions,
\end{flushleft}


\begin{flushleft}
Imputations and their Dominance, The Core of a game, Strategic
\end{flushleft}


\begin{flushleft}
Equivalence, Stable Sets of Imputations, Shapley Values, N-Person
\end{flushleft}


\begin{flushleft}
Non-Zero Sum Games with continuum of strategies -- Duels, Auctions,
\end{flushleft}


\begin{flushleft}
Nash Model with Security Point, Threats, Evolution, Stable Strategies,
\end{flushleft}


\begin{flushleft}
Population Games, Bayesian Games.
\end{flushleft}





\begin{flushleft}
MTL766 Multivariate Statistical Methods
\end{flushleft}


\begin{flushleft}
3 Credits (3-0-0)
\end{flushleft}


\begin{flushleft}
Pre-requisites: MTL390/MTL601
\end{flushleft}





\begin{flushleft}
MST: Fibonacci Heaps and O(m log log n) time implementation of
\end{flushleft}


\begin{flushleft}
MST, Linear time MST verification Algorithm, A linear time randomized
\end{flushleft}


\begin{flushleft}
algorithm for MST,Finding min-cost arborescences; Dynamic Graph
\end{flushleft}


\begin{flushleft}
Algorithms; Review of NP-completeness; Introduction to NPhard optimization problems; A brief introduction to LPP; Integer
\end{flushleft}


\begin{flushleft}
Programming Problem; Primal-Dual Algorithm; Approximation
\end{flushleft}


\begin{flushleft}
Algorithms: Primal-Dual Approximation Scheme; vertex cover, set
\end{flushleft}


\begin{flushleft}
cover, TSP; Hardness of Approximation; Introduction to Randomized
\end{flushleft}


\begin{flushleft}
Algorithms; Some basic Randomized algorithms; Probabilistic Method:
\end{flushleft}


\begin{flushleft}
Lovasz Local Lemma.
\end{flushleft}





\begin{flushleft}
MTL761 Basic Ergodic Theory
\end{flushleft}


\begin{flushleft}
3 Credits (3-0-0)
\end{flushleft}


\begin{flushleft}
Measure spaces, Haar Measure, Poincare Recurrence Theorem,
\end{flushleft}


\begin{flushleft}
Hopf's Maximal Ergodic Theorem, Birkhoff's Ergodic Theorem, von
\end{flushleft}


\begin{flushleft}
Neumann's Ergodic Theorem, Isomorphism, Conjugacy and Spectral
\end{flushleft}


\begin{flushleft}
Isomorphism, Entropy, Topological Pressure and its relationship with
\end{flushleft}


\begin{flushleft}
Invariant Measures.
\end{flushleft}





\begin{flushleft}
MTL762 Probability Theory
\end{flushleft}


\begin{flushleft}
3 Credits (3-0-0)
\end{flushleft}


\begin{flushleft}
Axiomatic definition of a probability measure, examples, properties
\end{flushleft}


\begin{flushleft}
of the probability measure, finite probability space, conditional
\end{flushleft}


\begin{flushleft}
probability and Bayes formula, countable probability space, general
\end{flushleft}


\begin{flushleft}
probability space.
\end{flushleft}


\begin{flushleft}
Random variables, examples, sigma-field generated by a random variable,
\end{flushleft}


\begin{flushleft}
tail sigma-field, probability space on R induced by a random variable.
\end{flushleft}


\begin{flushleft}
Independent events, sigma-fields and random variables, Borel 0-1
\end{flushleft}


\begin{flushleft}
criteria, Kolmogorov 0-1 criteria.
\end{flushleft}


\begin{flushleft}
Distribution - definition and examples, properties, characterization,
\end{flushleft}


\begin{flushleft}
Jordan decomposition theorem, discrete, continuous and mixed
\end{flushleft}


\begin{flushleft}
random variables, standard discrete and continuous distributions,
\end{flushleft}


\begin{flushleft}
convolution of distributions.
\end{flushleft}





\begin{flushleft}
Introduction to Multivariate data, Geometry of a sample, Mean and
\end{flushleft}


\begin{flushleft}
Covariance, Generalized Variance; Sample value of Linear combination
\end{flushleft}


\begin{flushleft}
of variables; Multivariate Normal Distribution, and its properties,
\end{flushleft}


\begin{flushleft}
Sampling from a Multivariate Normal population, Sampling distribution
\end{flushleft}


\begin{flushleft}
and Large sample Behaviour of Mean and Covariance, Inference
\end{flushleft}


\begin{flushleft}
about Mean Vector, Hotelling's T-square and Likelihood Ratio test,
\end{flushleft}


\begin{flushleft}
Confidence Region , Comparison of several Multivariate Populations,
\end{flushleft}


\begin{flushleft}
Multivariate Linear Regression Models, Inferences about regression
\end{flushleft}


\begin{flushleft}
models and parameters, Model checking, Principal Component
\end{flushleft}


\begin{flushleft}
Analysis, Introduction to Factor Analysis, Orthogonal Factor Models,
\end{flushleft}


\begin{flushleft}
Factor Rotation, Strategy for Factor Analysis; Canonical Correlation
\end{flushleft}


\begin{flushleft}
Analysis, Interpreting population by Canonical variables , Large
\end{flushleft}


\begin{flushleft}
Sample Inferences.
\end{flushleft}





\begin{flushleft}
MTL768 Graph Theory
\end{flushleft}


\begin{flushleft}
3 Credits (3-0-0)
\end{flushleft}


\begin{flushleft}
Overlaps with: MTL776
\end{flushleft}


\begin{flushleft}
Introduction to Graphs: Definition and basic concepts; Trees:
\end{flushleft}


\begin{flushleft}
characterizations, counting of minimum spanning tree; Paths and
\end{flushleft}


\begin{flushleft}
Distance in Graphs: Basic Definitions, center and median of a graph,
\end{flushleft}


\begin{flushleft}
activity digraph and critical path; Eulerian Graphs: Definition and
\end{flushleft}


\begin{flushleft}
Characterization; Hamiltonian Graphs: Necessary and sufficient
\end{flushleft}


\begin{flushleft}
conditions, Planar Graphs: properties, dual, genus of a graph; Graph
\end{flushleft}


\begin{flushleft}
Coloring: vertex coloring, chromatic polynomials, edge coloring, planar
\end{flushleft}


\begin{flushleft}
graph coloring; Matching and Factorizations: maximum matching in
\end{flushleft}


\begin{flushleft}
bipartite graphs, maximum matching in general graphs, Hall's marriage
\end{flushleft}


\begin{flushleft}
theorem, factorization; Networks: The Max-flow min-cut theorem,
\end{flushleft}


\begin{flushleft}
connectivity and edge connectivity, Menger's theorem; Graph and Matrices.
\end{flushleft}





\begin{flushleft}
MTL773 Wavelets and Applications
\end{flushleft}


\begin{flushleft}
3 Credits (3-0-0)
\end{flushleft}


\begin{flushleft}
Pre-requisites: MTL411/MTL602
\end{flushleft}


\begin{flushleft}
Basic Fourier Analysis: Fourier Series, convergence of Fourier
\end{flushleft}





253





\begin{flushleft}
\newpage
Mathematics
\end{flushleft}





\begin{flushleft}
series, Riesz Fischer theorem, Fourier transform of square
\end{flushleft}


\begin{flushleft}
integrable functions, Plancheral formula, Poisson Summation
\end{flushleft}


\begin{flushleft}
formula,	 Shannon sampling theorem, Heisenberg Uncertainty
\end{flushleft}


\begin{flushleft}
principle. Continuous Wavelet transform, Plancherel formula,
\end{flushleft}


\begin{flushleft}
Inversion formulas. Frames, Riesz Systems, discrete wavelet
\end{flushleft}


\begin{flushleft}
transform, Numerical algorithms. Orthogonal bases of wavelets,
\end{flushleft}


\begin{flushleft}
multi resolution analysis, smoothness of wavelets, compactly
\end{flushleft}


\begin{flushleft}
supported wavelets, cardinal spline wavelets. Tensor products
\end{flushleft}


\begin{flushleft}
of wavelets, Decomposition and reconstruction algorithms for
\end{flushleft}


\begin{flushleft}
wavelets, wavelet packets, recent development and applications.
\end{flushleft}





\begin{flushleft}
MTL781 Finite Element Theory and Applications
\end{flushleft}


\begin{flushleft}
3 Credits (3-0-0)
\end{flushleft}


\begin{flushleft}
Pre-requisites: MTL107/MTL509 and MTL411/MTL602
\end{flushleft}


\begin{flushleft}
Variational formulation of elliptic boundary value problems; Lax
\end{flushleft}


\begin{flushleft}
Milgram Lemma; Existence and uniqueness of solutions; equivalence
\end{flushleft}


\begin{flushleft}
of Galerkin and Ritz variational formulations; Triangulation of
\end{flushleft}


\begin{flushleft}
ordinary domains-rectangles, polygons, circles, ellipses, etc. Finite
\end{flushleft}


\begin{flushleft}
element problems; conforming and non-conforming methods, Ce'a's
\end{flushleft}


\begin{flushleft}
Lemma, Interpolation on simplexes in Rn, different Lagrange and
\end{flushleft}


\begin{flushleft}
Hermite finite elements, Affine, isoparametric, sub-parametric,
\end{flushleft}


\begin{flushleft}
super parametric finite elements; Triangulation using isoparametric
\end{flushleft}


\begin{flushleft}
mapping; approximation of boundary; Numerical Integration,
\end{flushleft}


\begin{flushleft}
construction of element stiffness matrices and assembly into
\end{flushleft}


\begin{flushleft}
global stiffness matrix, Skyline method of solution of finite element
\end{flushleft}


\begin{flushleft}
equations; Solution of model problems and computer implementation
\end{flushleft}


\begin{flushleft}
procedures; Asymptotic error estimate results; Eigenvalue problems
\end{flushleft}


\begin{flushleft}
of Laplace operator.
\end{flushleft}





\begin{flushleft}
MTL785 Natural Language Processing
\end{flushleft}


\begin{flushleft}
3 Credits (3-0-0)
\end{flushleft}


\begin{flushleft}
Overlaps with: COL772
\end{flushleft}


\begin{flushleft}
Linguistic Essentials: Parts of speech and morphology, inflectional
\end{flushleft}


\begin{flushleft}
versus derivational morphology, Phrase structure and link grammar,
\end{flushleft}


\begin{flushleft}
Syntax and syntactic theory, Semantics - semantic annotations,
\end{flushleft}


\begin{flushleft}
semantic similarity, Syntactic and semantic Ambiguity, Anaphora and
\end{flushleft}


\begin{flushleft}
cataphora - resolution. Study of Words: Frequency N-grams, Word
\end{flushleft}


\begin{flushleft}
alignment in parallel corpora - length based, word based, cognate
\end{flushleft}


\begin{flushleft}
based, Word Sense Disambiguation - Supervised, unsupervised
\end{flushleft}


\begin{flushleft}
Techniques, Grammar: Markov Models, POS tagging, Context Free
\end{flushleft}


\begin{flushleft}
Grammar, Parsing - Example-based parsing, Study of Divergence,
\end{flushleft}


\begin{flushleft}
Applications: Machine Translation - Example Based, Rule Based,
\end{flushleft}


\begin{flushleft}
Statistical, Summarization - Word Space Model, Random Indexing,
\end{flushleft}


\begin{flushleft}
Multi Document summarization, Information Retireval - vector based,
\end{flushleft}


\begin{flushleft}
term distribution based, Sentiment Analysis.
\end{flushleft}





\begin{flushleft}
MTV791 Special Module in Dynamical System
\end{flushleft}


\begin{flushleft}
1 Credit (1-0-0)
\end{flushleft}


\begin{flushleft}
Basics - minimality, equicontinuity, recurrence, distality. Interplay of
\end{flushleft}


\begin{flushleft}
dynamical properties. Ergodicity. Symbolic dynamics. Relations arising
\end{flushleft}


\begin{flushleft}
from dynamical transformations and their Ellis semigroups. Entropy.
\end{flushleft}


\begin{flushleft}
Structure theorems. Decomposition theorems.
\end{flushleft}





\begin{flushleft}
MTL792 Modern Methods in Partial Differential equations
\end{flushleft}


\begin{flushleft}
3 Credits (3-0-0)
\end{flushleft}


\begin{flushleft}
Pre-requisites: MTL411/MTL602
\end{flushleft}


\begin{flushleft}
Review of Lebesgue integration and Classical function spaces, Spaces
\end{flushleft}


\begin{flushleft}
of infinitely differentiable functions and Holder spaces. Sobolev spaces:
\end{flushleft}


\begin{flushleft}
L\^{}p spaces, Weak derivatives, Sobolev spaces, approximation to
\end{flushleft}


\begin{flushleft}
identity, approximation with smooth functions, trace spaces, Sobolev
\end{flushleft}


\begin{flushleft}
and Poincare inequalities, compact embeddings and negative order
\end{flushleft}


\begin{flushleft}
Sobolev spaces. Second order elliptic equations: Weak solutions, LaxMilgram Theorem, Energy estimates, Fredlhom-alternative, Regularity
\end{flushleft}


\begin{flushleft}
of weak solutions, Maximum principles and eigenvalue problems.
\end{flushleft}


\begin{flushleft}
Mountain Pass lemma and applications. Hardy's inequalities and their
\end{flushleft}


\begin{flushleft}
relation with Elliptic equations. Linear Evolution equations: Second
\end{flushleft}


\begin{flushleft}
order parabolic equations, existence and regularity of weak solutions,
\end{flushleft}


\begin{flushleft}
Maximum principles. Semi-group Theory: Generating, contraction
\end{flushleft}


\begin{flushleft}
semi-groups and applications.
\end{flushleft}





\begin{flushleft}
MTL793 Numerical Methods for Hyperbolic PDEs
\end{flushleft}


\begin{flushleft}
3 Credits (3-0-0)
\end{flushleft}


\begin{flushleft}
Scalar conservation laws: Method of characteristics, Shocks,
\end{flushleft}


\begin{flushleft}
Rarefactions, weak and entropy solutions, existence and uniqueness
\end{flushleft}


\begin{flushleft}
results, Finite volume schemes, Riemann solvers, Convergence of first
\end{flushleft}


\begin{flushleft}
order schemes. Higher-order schemes: Lax-Wendroff, TVD schemes,
\end{flushleft}


\begin{flushleft}
Limiters, ENO schemes, Higher order Runge-Kutta methods. Linear
\end{flushleft}


\begin{flushleft}
systems: Exact solutions, First- and higher-order finite volume
\end{flushleft}


\begin{flushleft}
schemes. Non-linear Systems: Solutions of Riemann problems, FirstOrder finite volume schemes for systems. Higher-order schemes
\end{flushleft}


\begin{flushleft}
for systems: TVD Limiters. Finite-volume schemes on unstructured
\end{flushleft}


\begin{flushleft}
meshes. Hyperbolic systems with source term.
\end{flushleft}





\begin{flushleft}
MTL794 Advanced Probability Theory
\end{flushleft}


\begin{flushleft}
3 Credits (3-0-0)
\end{flushleft}


\begin{flushleft}
Notions of Stochastic Convergence and Related Convergence Theorems,
\end{flushleft}


\begin{flushleft}
Uniform Integrability, Weak and Strong Laws of Large Numbers, Speed
\end{flushleft}


\begin{flushleft}
of Convergence in the Strong Laws of Large Numbers, Martingales,
\end{flushleft}


\begin{flushleft}
Processes, Filtrations, Stopping Times, Discrete Stochastic Integral,
\end{flushleft}


\begin{flushleft}
Martingale Convergence Theorems and Their Applications, Levy's
\end{flushleft}


\begin{flushleft}
Continuity Theorem and Various Versions of Central Limit Theorem,
\end{flushleft}


\begin{flushleft}
Markov Chains, Discrete Markov Chains, Convergence of Markov
\end{flushleft}


\begin{flushleft}
Chains, Applications of Probability Theory to Fourier Series-Examples.
\end{flushleft}





\begin{flushleft}
MTL795 Numerical Method for Partial Differential
\end{flushleft}


\begin{flushleft}
Equations
\end{flushleft}


\begin{flushleft}
4 Credits (3-1-0)
\end{flushleft}


\begin{flushleft}
Two point boundary value problem: Variational approach, Discretization
\end{flushleft}


\begin{flushleft}
and convergence of numerical schemes. Second order Elliptic boundary
\end{flushleft}


\begin{flushleft}
value problem, Variational formulation and Boundary conditions, Finite
\end{flushleft}


\begin{flushleft}
element Methods, Galerkin Discretization, Implementation, Finite
\end{flushleft}


\begin{flushleft}
difference and Finite volume methods, Convergence and Accuracy.
\end{flushleft}


\begin{flushleft}
Parabolic initial value problems, Heat equations, variational
\end{flushleft}


\begin{flushleft}
formulation, Method of lines, Convergence.
\end{flushleft}


\begin{flushleft}
Wave Equations, Method of lines, Timestepping.
\end{flushleft}





\begin{flushleft}
MTL843 Mathematical Modeling of Credit Risk
\end{flushleft}


\begin{flushleft}
3 Credits (3-0-0)
\end{flushleft}


\begin{flushleft}
Pre-requisites: MTL106/MTL601
\end{flushleft}


\begin{flushleft}
Review of elementary stochastic calculus and Black - Scholes - Merton
\end{flushleft}


\begin{flushleft}
theory of option pricing. Corporate liabilities and contingent claims.
\end{flushleft}


\begin{flushleft}
Risk structure of interest rates. Statistical techniques for analyzing
\end{flushleft}


\begin{flushleft}
defaults. Credit scoring modeling using logistic regression, Discriminant
\end{flushleft}


\begin{flushleft}
Analysis and support vector machines. Rating based term structure
\end{flushleft}


\begin{flushleft}
models. Credit risk and interest rate swaps. Credit default swaps
\end{flushleft}


\begin{flushleft}
(CDS), collateralized debt obligations (CDO's) and other related
\end{flushleft}


\begin{flushleft}
products. The copula approach. Portfolio Credit risk analysis using
\end{flushleft}


\begin{flushleft}
coherant risk measures.
\end{flushleft}





\begin{flushleft}
MTL851 Applied Numerical Analysis
\end{flushleft}


\begin{flushleft}
3 Credits (3-0-0)
\end{flushleft}


\begin{flushleft}
Error analysis and stability of algorithms. Nonlinear equations: Newton
\end{flushleft}


\begin{flushleft}
Raphson method, Muller's method, criterion for acceptance of a
\end{flushleft}


\begin{flushleft}
root, system of non-linear equations. Roots of polynomial equations.
\end{flushleft}


\begin{flushleft}
Linear system of algebraic equations : Gauss elimination method,
\end{flushleft}


\begin{flushleft}
LU-decomposition method; matrix inversion, iterative methods,
\end{flushleft}


\begin{flushleft}
ill- conditioned systems. Eigenvalue problems : Jacobi, Given's and
\end{flushleft}


\begin{flushleft}
Householder's methods for symmetric matrices, Rutishauser method
\end{flushleft}


\begin{flushleft}
for general matrices, Power and inverse power methods. Interpolation
\end{flushleft}


\begin{flushleft}
and approximation : Newton's, Lagrange and Hermite interpolating
\end{flushleft}


\begin{flushleft}
polynomials, cubic splines; least square and minimax approximations.
\end{flushleft}


\begin{flushleft}
Numerical differentiation and integration: Newton-Cotes and Gaussian
\end{flushleft}


\begin{flushleft}
type quadrature methods.
\end{flushleft}


\begin{flushleft}
Ordinary differential equations : Initial value problems: single step and
\end{flushleft}


\begin{flushleft}
multistep methods, stability and their convergence. Boundary value
\end{flushleft}


\begin{flushleft}
problems: Shooting and difference methods.
\end{flushleft}


\begin{flushleft}
Partial Differential Equations : Difference methods for solution of
\end{flushleft}


\begin{flushleft}
parabolic and hyperbolic equations in one and two-space dimensions,
\end{flushleft}


\begin{flushleft}
stability and their convergence, difference methods for elliptic equations.
\end{flushleft}





254





\begin{flushleft}
\newpage
Mathematics
\end{flushleft}





\begin{flushleft}
MTL854 Interpolation and Approximation
\end{flushleft}


\begin{flushleft}
3 Credits (3-0-0)
\end{flushleft}


\begin{flushleft}
Interpolation : general problem, representation theorems, remainder
\end{flushleft}


\begin{flushleft}
theory, convergence of interpolatory processes. Approximation : best,
\end{flushleft}


\begin{flushleft}
uniform and least-squares, degree of approximation. Approximation
\end{flushleft}


\begin{flushleft}
of linear functionals : Optimal approximations in Hilbert spaces, roots
\end{flushleft}


\begin{flushleft}
and extremals : Convergence of Newton's method in Banach spaces,
\end{flushleft}


\begin{flushleft}
minimizing functionals on normed linear spaces, applications to integral
\end{flushleft}


\begin{flushleft}
equations and control theory.
\end{flushleft}


\begin{flushleft}
Splines : applications to computer-aided design.
\end{flushleft}


\begin{flushleft}
Filters : linear, least-squares and Chebyshev.
\end{flushleft}


\begin{flushleft}
Applications to signal processing.
\end{flushleft}





\begin{flushleft}
MTL855 Multiple Decision Procedures in Ranking and
\end{flushleft}


\begin{flushleft}
Selection
\end{flushleft}


\begin{flushleft}
3 Credits (3-0-0)
\end{flushleft}


\begin{flushleft}
The problem of ranking and selection, different approaches to the
\end{flushleft}


\begin{flushleft}
solution of problem. Indifference zone formulation : Ranking normal
\end{flushleft}


\begin{flushleft}
population in terms of means single and two stage procedures.
\end{flushleft}


\begin{flushleft}
Ranking normal population in terms of variances. Ranking binomial
\end{flushleft}


\begin{flushleft}
population-fixed sample size and multistage procedures, play the
\end{flushleft}


\begin{flushleft}
winner rules and vector at a time sampling. Ranking Gamma population
\end{flushleft}


\begin{flushleft}
with largest (smallest) scale parameter. Optimal properties of fixed
\end{flushleft}


\begin{flushleft}
subset size proceduresBayes, minimax and admissibilities properties,
\end{flushleft}


\begin{flushleft}
subset selection formulation : Decision theoretical formulation,
\end{flushleft}


\begin{flushleft}
best invariant rules. Restricted subset selection. Subset selection of
\end{flushleft}


\begin{flushleft}
normal population w.r.t. means and variances, selection of t-best.
\end{flushleft}


\begin{flushleft}
Subset selection in binomial and gamma populations. Comparison
\end{flushleft}


\begin{flushleft}
of population with a control. Normal and exponential populations.
\end{flushleft}





\begin{flushleft}
fields, integral extensions, conjugate elements and conjugate fields,
\end{flushleft}


\begin{flushleft}
norms and traces. The discriminant. Noetherian rings and Dedekind
\end{flushleft}


\begin{flushleft}
domains. Finiteness of the class group. Dirichlet's unit theorem and
\end{flushleft}


\begin{flushleft}
its applications.
\end{flushleft}





\begin{flushleft}
MTL874 Analysis
\end{flushleft}


\begin{flushleft}
3 Credits (3-0-0)
\end{flushleft}


\begin{flushleft}
Review of Banach and Hilbert spaces. The Hahn-Banach, Open
\end{flushleft}


\begin{flushleft}
mapping and Banach-Steinhaus theorems. The Riesz representation
\end{flushleft}


\begin{flushleft}
theorem, the spaces Lp(0,1) and L2(0,1) Spectral theory and SturmLiouville systems, fixed point theory. The theorems by Banach, Browder
\end{flushleft}


\begin{flushleft}
and Schauder and applications. Picard's theorem. Integral equation
\end{flushleft}


\begin{flushleft}
of Fredholm, Volterra and Hammerstein. Nonlinear operators : The
\end{flushleft}


\begin{flushleft}
complementarity problem and its uses. Banach algebras and C*
\end{flushleft}


\begin{flushleft}
algebras. Best approximation in normed linear spaces.
\end{flushleft}





\begin{flushleft}
MTL882 Applied Analysis
\end{flushleft}


\begin{flushleft}
3 Credits (3-0-0)
\end{flushleft}


\begin{flushleft}
Review of Normed Linear spaces, Banach spaces and Hilbert spaces.
\end{flushleft}


\begin{flushleft}
Weak and weak* convergence, Spectrum of Bounded Linear operators.
\end{flushleft}


\begin{flushleft}
Browder and Schauder fixed point theorems and applications to
\end{flushleft}


\begin{flushleft}
Differential and integral equations, LP spaces.
\end{flushleft}


\begin{flushleft}
Distributions and Fourier transforms: Schwartz space, tempered
\end{flushleft}


\begin{flushleft}
distributions, Fourier transform of tempered distributions, Fourier
\end{flushleft}


\begin{flushleft}
transform of LP functions and applications.
\end{flushleft}


\begin{flushleft}
Sobolev spaces: Density, embedding and extension theorems.
\end{flushleft}


\begin{flushleft}
Differential Calculus: Derivatives of maps on Banach spaces, inverse
\end{flushleft}


\begin{flushleft}
and implicit function theorems, Direct methods of Calculus of variations
\end{flushleft}


\begin{flushleft}
and applications.
\end{flushleft}





\begin{flushleft}
MTL883 Physical Fluid Mechanics
\end{flushleft}


\begin{flushleft}
3 Credits (3-0-0)
\end{flushleft}





\begin{flushleft}
MTL856 Lie Algebras
\end{flushleft}


\begin{flushleft}
3 Credits (3-0-0)
\end{flushleft}


\begin{flushleft}
Overlaps with: MTL756
\end{flushleft}


\begin{flushleft}
Definitions and examples. Basic concepts. Solvable and Nilpotent
\end{flushleft}


\begin{flushleft}
Lie algebras, The Engel's theorem, Lie's theorem, Cartan's criterion,
\end{flushleft}


\begin{flushleft}
Killing form, Finite dimensional semi-simple Lie algebras and their
\end{flushleft}


\begin{flushleft}
representation theory. The Weyl's theorem. Representations of sl
\end{flushleft}


\begin{flushleft}
(2,C). Root space decomposition. Rationality properties. Root systems,
\end{flushleft}


\begin{flushleft}
The Weyl group. Isomorphism and conjugacy theorems (Cartan
\end{flushleft}


\begin{flushleft}
subalgebras, Borel subalgebras). Universal enveloping algebras, PBW
\end{flushleft}


\begin{flushleft}
theorem, Serre's theorem. Representation theory and characters.
\end{flushleft}


\begin{flushleft}
Formulas of Weyl, Kostant and Steinberg. Introduction to infinite
\end{flushleft}


\begin{flushleft}
dimensional Lie algebras.
\end{flushleft}





\begin{flushleft}
MTL860 Linear Algebra
\end{flushleft}


\begin{flushleft}
3 Credits (3-0-0)
\end{flushleft}


\begin{flushleft}
Vector spaces, linear transformations, Eigenvalues and eigenvectors,
\end{flushleft}


\begin{flushleft}
Diagonalization, Simultaneous triangulation and diagonalization.
\end{flushleft}


\begin{flushleft}
The primary decomposition theorem. Cyclic decomposition and
\end{flushleft}


\begin{flushleft}
the rational and Jordan canonical forms. Computation of invariant
\end{flushleft}


\begin{flushleft}
factors. Inner product spaces, unitary operators, spectral theorem
\end{flushleft}


\begin{flushleft}
for normal operators, polar decomposition. Bilinear and quadratic
\end{flushleft}


\begin{flushleft}
forms, Symmetric and Skew-symmetric bilinear forms. Non-negative
\end{flushleft}


\begin{flushleft}
matrices, Perron-Frobenius theory, generalized inverse of a matrix.
\end{flushleft}





\begin{flushleft}
MTL863 Algebraic Number Theory
\end{flushleft}


\begin{flushleft}
3 Credits (3-0-0)
\end{flushleft}


\begin{flushleft}
Algebraic number fields, cyclotomic fields, quadratic and cubic
\end{flushleft}





\begin{flushleft}
Description of principles of flow phenomena : pipe and channel flow
\end{flushleft}


\begin{flushleft}
laminar flow, transition, turbulance; flow past an object; boundary
\end{flushleft}


\begin{flushleft}
layer, wake, separation, vortices, drag, convection in horizontal layers,
\end{flushleft}


\begin{flushleft}
transition from periodic to chaotic behaviour; equations of motion;
\end{flushleft}


\begin{flushleft}
dynamical scaling, sample viscous flows; inviscid flows. Flow in rotating
\end{flushleft}


\begin{flushleft}
fluids; hydrodynamic stability.
\end{flushleft}





\begin{flushleft}
MTL888 Boundary Elements Methods with Computer
\end{flushleft}


\begin{flushleft}
Implementation
\end{flushleft}


\begin{flushleft}
3 Credits (3-0-0)
\end{flushleft}


\begin{flushleft}
Distributions and Sobolev spaces of fractional order. Elliptic boundary
\end{flushleft}


\begin{flushleft}
value problems on unbounded domains in IRn (n=2,3).
\end{flushleft}


\begin{flushleft}
Fundamental solution of elliptic equations.
\end{flushleft}


\begin{flushleft}
Simple layer and double layer potentials Fredholm integral equations
\end{flushleft}


\begin{flushleft}
of first and second kinds. Singular and hypersingular kernels.
\end{flushleft}


\begin{flushleft}
Interior and exterior Dirichlet problems and integral representations
\end{flushleft}


\begin{flushleft}
of their solutions.
\end{flushleft}


\begin{flushleft}
Variational formulation of problems defined on boundary. Solution of
\end{flushleft}


\begin{flushleft}
some model problems by boundary element methods, approximate
\end{flushleft}


\begin{flushleft}
integrations over boundary, solution methods of algebraic
\end{flushleft}


\begin{flushleft}
equations; computer implementation of boundary element methods
\end{flushleft}


\begin{flushleft}
for a model problem. Coupling of boundary element and finite
\end{flushleft}


\begin{flushleft}
element methods.
\end{flushleft}


\begin{flushleft}
Some advanced topics of boundary integral methods integrals with
\end{flushleft}


\begin{flushleft}
hypersingular kernel, a method of elimination of singularity, Lagrange
\end{flushleft}


\begin{flushleft}
multiplier method.
\end{flushleft}





255





\begin{flushleft}
\newpage
Department of Mechanical Engineering
\end{flushleft}


\begin{flushleft}
MCP100 Introduction to Engineering Visualization
\end{flushleft}


\begin{flushleft}
2 Credits (0-0-4)
\end{flushleft}





\begin{flushleft}
automotive components, casting of light alloys -- Aluminum,
\end{flushleft}


\begin{flushleft}
Magnesium and Titanium alloys.
\end{flushleft}





\begin{flushleft}
Sketching of engineering objects and interpretation of drawings as
\end{flushleft}


\begin{flushleft}
a visualisation and communication tool. Creating 3D components
\end{flushleft}


\begin{flushleft}
through the use of a CAD package. Simple assemblies, generation of
\end{flushleft}


\begin{flushleft}
assembly views from part drawings, animation of simple assemblies.
\end{flushleft}





\begin{flushleft}
MCP101 Product Realization through Manufacturing
\end{flushleft}


\begin{flushleft}
2 Credits (0-0-4)
\end{flushleft}


\begin{flushleft}
Exposing role of manufacturing processes in product realization;
\end{flushleft}


\begin{flushleft}
Understanding product realization by endeavouring hands on activities;
\end{flushleft}


\begin{flushleft}
Experience of product realization by undertaking manufacturing
\end{flushleft}


\begin{flushleft}
exercises and assembly activity in teams.
\end{flushleft}





\begin{flushleft}
MCL111 Kinematics and Dynamics of Machines
\end{flushleft}


\begin{flushleft}
4 Credits (3-0-2)
\end{flushleft}


\begin{flushleft}
Pre-requisites: APL100
\end{flushleft}


\begin{flushleft}
Kinematic pairs, Kinematic diagram and inversions. Mobility and
\end{flushleft}


\begin{flushleft}
range of movements. Displacement, velocity and acceleration analysis
\end{flushleft}


\begin{flushleft}
of planar linkages, graphical and analytical methods. Dimensional
\end{flushleft}


\begin{flushleft}
synthesis for motion, function and path generation. Force analysis of
\end{flushleft}


\begin{flushleft}
planar mechanisms. Cam profile synthesis, graphical and analytical
\end{flushleft}


\begin{flushleft}
method. Gear tooth profile, interference in gears. Gear types,
\end{flushleft}


\begin{flushleft}
gear trains including compound epicyclic gears. Design of flywheel
\end{flushleft}


\begin{flushleft}
and governors. Inertia forces and their balancing for rotating and
\end{flushleft}


\begin{flushleft}
reciprocating machines.
\end{flushleft}


\begin{flushleft}
Free and forced vibration of SDOF system. Introduction to 2 DOF
\end{flushleft}


\begin{flushleft}
systems, vibration absorbers.
\end{flushleft}





\begin{flushleft}
MCL131 Manufacturing Processes-I
\end{flushleft}


\begin{flushleft}
3 Credits (3-0-0)
\end{flushleft}


\begin{flushleft}
Pre-requisites: MCP101
\end{flushleft}


\begin{flushleft}
Overlaps with: With three core courses of ME2 (30\% each)
\end{flushleft}


\begin{flushleft}
CASTING: Sand casting, Gating system and its design, Riser design
\end{flushleft}


\begin{flushleft}
and its placement, Melting, Pouring and Fluidity, Solidification of pure
\end{flushleft}


\begin{flushleft}
metals and alloys, Casting defects, Inspection and testing. Other
\end{flushleft}


\begin{flushleft}
casting processes, advantages and applications.
\end{flushleft}


\begin{flushleft}
WELDING: Shielded metal arc welding, other arc welding processes like
\end{flushleft}


\begin{flushleft}
TIG, MIG and SAW processes, Types of metal transfer in arc welding,
\end{flushleft}


\begin{flushleft}
Gas welding and Gas cutting, Resistance welding, Solid state welding
\end{flushleft}


\begin{flushleft}
processes, Brazing, Soldering and their applications, Surfacing and
\end{flushleft}


\begin{flushleft}
its applications.
\end{flushleft}


\begin{flushleft}
FORMING: Plastic deformation of metals, stress-strain relationships,
\end{flushleft}


\begin{flushleft}
Yield criteria, Hot working and Cold working, Friction and lubrication
\end{flushleft}


\begin{flushleft}
in metal working, Analysis of bulk forming and sheet metal forming
\end{flushleft}


\begin{flushleft}
processes. Unconventional forming processes.
\end{flushleft}


\begin{flushleft}
Powder Metallurgy: Powder production methods, compaction and
\end{flushleft}


\begin{flushleft}
sintering. Applications of powder metallurgy.
\end{flushleft}





\begin{flushleft}
MCL132 Metal Forming and Press Tools
\end{flushleft}


\begin{flushleft}
3 Credits (3-0-0)
\end{flushleft}


\begin{flushleft}
Pre-requisites: MCP101
\end{flushleft}


\begin{flushleft}
Overlaps with: 30\% with MCL131
\end{flushleft}


\begin{flushleft}
Mechanical behaviour of metals and alloys in plastic deformation,
\end{flushleft}


\begin{flushleft}
Stress-strain relationships, Yield criteria, Fundamentals of plasticity,
\end{flushleft}


\begin{flushleft}
Tensile properties, Flow stress and flow curves, Fundamentals of metal
\end{flushleft}


\begin{flushleft}
forming processes, Strain rate and temperature in metal working, Hot
\end{flushleft}


\begin{flushleft}
working, Cold working and annealing, Analysis of forming processes
\end{flushleft}


\begin{flushleft}
like forging, rolling, extrusion, wire drawing and sheet metal forming
\end{flushleft}


\begin{flushleft}
by slab method, Equipment and tools used in metal forming operations,
\end{flushleft}


\begin{flushleft}
Types of presses, different types of dies and their design aspects,
\end{flushleft}


\begin{flushleft}
Unconventional forming processes.
\end{flushleft}





\begin{flushleft}
MCL133 Near Net Shape Manufacturing
\end{flushleft}


\begin{flushleft}
3 Credits (3-0-0)
\end{flushleft}


\begin{flushleft}
Pre-requisites: MCP101
\end{flushleft}


\begin{flushleft}
Introduction and fundamentals of Casting of complicated shapes:
\end{flushleft}





\begin{flushleft}
Injection moulding: Thermoplastics, thermoset plastics and composites
\end{flushleft}


\begin{flushleft}
-- processing methodologies.
\end{flushleft}


\begin{flushleft}
Powder Metallurgy: fabrication routes, powder size determination
\end{flushleft}


\begin{flushleft}
-- micro and nano level, powder consolidation routes, compacting,
\end{flushleft}


\begin{flushleft}
sintering, hot pressing, sintering, hot iso static pressing, field assisted
\end{flushleft}


\begin{flushleft}
sintering technologies.
\end{flushleft}


\begin{flushleft}
Advances in near net shape manufacturing: Metal Injection moulding,
\end{flushleft}


\begin{flushleft}
Laser engineered net shaping.
\end{flushleft}





\begin{flushleft}
MCL134 Metrology and Quality Assurance
\end{flushleft}


\begin{flushleft}
3.5 Credits (3-0-1)
\end{flushleft}


\begin{flushleft}
Pre-requisites: MCP101
\end{flushleft}


\begin{flushleft}
Overlaps with: MCL231
\end{flushleft}


\begin{flushleft}
Introduction to Metrology and its relevance, standardization,
\end{flushleft}


\begin{flushleft}
dimensional measurement, limits, fits and tolerances, limit gauging,
\end{flushleft}


\begin{flushleft}
linear and angular measurements and their applications, surface
\end{flushleft}


\begin{flushleft}
roughness-quantification \& measurement, Feature Inspection, Online
\end{flushleft}


\begin{flushleft}
inspection, Calibration.
\end{flushleft}


\begin{flushleft}
Introduction to Quality Assurance and Quality Control, Various elements
\end{flushleft}


\begin{flushleft}
in Quality Assurance, On-line and Off-line quality control, Statistical
\end{flushleft}


\begin{flushleft}
concepts in quality, Central limit theorem, Quality Characteristics, QC
\end{flushleft}


\begin{flushleft}
Tools. Process capability studies, Remedial / Corrective actions.
\end{flushleft}


\begin{flushleft}
Design of sampling plans, Economics of product inspection, Quality
\end{flushleft}


\begin{flushleft}
costs, Problems and illustrations in Quality Assurance.
\end{flushleft}





\begin{flushleft}
MCL135 Welding and Allied Processes
\end{flushleft}


\begin{flushleft}
3 Credits (3-0-0)
\end{flushleft}


\begin{flushleft}
Pre-requisites: MCP101
\end{flushleft}


\begin{flushleft}
Principles of arc welding, basic physics of arc and flame, Gas welding
\end{flushleft}


\begin{flushleft}
and Gas cutting, manual metal arc welding, GTAW, GMAW. Metal
\end{flushleft}


\begin{flushleft}
transfer mechanisms in arc welding, Weld bead characterization,
\end{flushleft}


\begin{flushleft}
Electrogas and electroslag welding, Resistance welding, Heat flow
\end{flushleft}


\begin{flushleft}
characteristics and metallurgical changes in fusion welding, Solid
\end{flushleft}


\begin{flushleft}
state welding processes, Radiant energy welding processes, Brazing,
\end{flushleft}


\begin{flushleft}
Soldering and their applications, Joint design, welding symbols and
\end{flushleft}


\begin{flushleft}
Joint evaluation through destructive and non destructive testing
\end{flushleft}


\begin{flushleft}
methods, welding defects, causes and remedies, residual stress
\end{flushleft}


\begin{flushleft}
and distortion. Plasma cutting, surfacing and plasma spray forming,
\end{flushleft}


\begin{flushleft}
surfacing applications. Advances in welding.
\end{flushleft}





\begin{flushleft}
MCL136 Material Removal Processes
\end{flushleft}


\begin{flushleft}
3 Credits (3-0-0)
\end{flushleft}


\begin{flushleft}
Pre-requisites: MCP101
\end{flushleft}


\begin{flushleft}
Introduction to various material removal processes, Nomenclature
\end{flushleft}


\begin{flushleft}
and geometry of cutting tools, Mechanics of Conventional and Non
\end{flushleft}


\begin{flushleft}
Conventional Machining including force, temperature, surface integrity.
\end{flushleft}


\begin{flushleft}
Methods of measurement of forces, temperature and surface finish
\end{flushleft}


\begin{flushleft}
(experimentally and analytically), Tool wear mechanisms and tool life
\end{flushleft}


\begin{flushleft}
criteria, Basic concepts of cost and economics of machining.
\end{flushleft}


\begin{flushleft}
Various types of machine tools and their structures, Workholding and
\end{flushleft}


\begin{flushleft}
tool holding devices for machine tools.
\end{flushleft}


\begin{flushleft}
Ultraprecision machining and grinding methods and the machine tools
\end{flushleft}


\begin{flushleft}
used for such processes. Manufacturing of micro tools, Nano-finishing
\end{flushleft}


\begin{flushleft}
of materials using advanced machining methods.
\end{flushleft}





\begin{flushleft}
MCL136 Material Removal Processes
\end{flushleft}


\begin{flushleft}
3 Credits (3-0-0)
\end{flushleft}


\begin{flushleft}
Pre-requisites: MCP101
\end{flushleft}


\begin{flushleft}
Introduction to various material removal processes, Nomenclature
\end{flushleft}


\begin{flushleft}
and geometry of cutting tools, Mechanics of Conventional and Non
\end{flushleft}


\begin{flushleft}
Conventional Machining including force, temperature, surface integrity.
\end{flushleft}


\begin{flushleft}
Methods of measurement of forces, temperature and surface finish
\end{flushleft}


\begin{flushleft}
(experimentally and analytically), Tool wear mechanisms and tool life
\end{flushleft}


\begin{flushleft}
criteria, Basic concepts of cost and economics of machining.
\end{flushleft}


\begin{flushleft}
Various types of machine tools and their structures, Workholding and
\end{flushleft}


\begin{flushleft}
tool holding devices for machine tools.
\end{flushleft}





256





\begin{flushleft}
\newpage
Mechanical Engineering
\end{flushleft}





\begin{flushleft}
Ultraprecision machining and grinding methods and the machine tools
\end{flushleft}


\begin{flushleft}
used for such processes. Manufacturing of micro tools, Nano-finishing
\end{flushleft}


\begin{flushleft}
of materials using advanced machining methods.
\end{flushleft}





\begin{flushleft}
MCL140 Engineering Thermodynamics
\end{flushleft}


\begin{flushleft}
4 Credits (3-1-0)
\end{flushleft}


\begin{flushleft}
Introduction: microscopic and macroscopic points of view. Basic
\end{flushleft}


\begin{flushleft}
concepts and definitions -- system, boundary, equilibrium, steady
\end{flushleft}


\begin{flushleft}
state, zeroth law, temperature scale. Work and heat -- definition and
\end{flushleft}


\begin{flushleft}
applications; various forms of work. Thermodynamic properties of a
\end{flushleft}


\begin{flushleft}
pure substance -- saturated and other states, real gases, compressibility
\end{flushleft}


\begin{flushleft}
chart. The First Law of Thermodynamics for control mass/ volume,
\end{flushleft}


\begin{flushleft}
Internal Energy, Enthalpy, The SSSF and USUF Processes. Second Law
\end{flushleft}


\begin{flushleft}
-- corollaries, Carnot cycle. Clausius inequality, entropy. Irreversibility
\end{flushleft}


\begin{flushleft}
and exergy analysis. Thermodynamic Relations. Vapor power cycles --
\end{flushleft}


\begin{flushleft}
Rankine cycle and its modifications. Brayton/ Otto/ Dual cycles. Vapor
\end{flushleft}


\begin{flushleft}
compression refrigeration cycle. Thermodynamics of non-reacting
\end{flushleft}


\begin{flushleft}
mixtures, psychrometry.
\end{flushleft}





\begin{flushleft}
MCL141 Thermal Science for Manufacturing
\end{flushleft}


\begin{flushleft}
4 Credits (3-1-0)
\end{flushleft}


\begin{flushleft}
Overlaps with: MCL140, MCL242 (50\%), CLL110 (50\%)
\end{flushleft}


\begin{flushleft}
Overview and the importance of the knowledge of thermal science in
\end{flushleft}


\begin{flushleft}
manufacturing processes. Basics of thermodynamics: closed and open
\end{flushleft}


\begin{flushleft}
systems, work and heat. First law of thermodynamics for control mass
\end{flushleft}


\begin{flushleft}
and control volume. Second law of thermodynamics. Irreversibilities
\end{flushleft}


\begin{flushleft}
and examples of irreversibilities in manufacturing.
\end{flushleft}


\begin{flushleft}
Introduction to transport phenomena : various modes of transport
\end{flushleft}


\begin{flushleft}
of momentum, energy and mass- diffusion and advective transport.
\end{flushleft}


\begin{flushleft}
Convective heat and mass transfer - Concept of momentum, thermal
\end{flushleft}


\begin{flushleft}
and concentration boundary layers; relevant correlations. Radiation
\end{flushleft}


\begin{flushleft}
heat transfer. Blackbody radiation. Gray and diffuse surfaces.
\end{flushleft}


\begin{flushleft}
Surface radiation. Case studies of manufacturing processes involving
\end{flushleft}


\begin{flushleft}
application of the above concepts.
\end{flushleft}





\begin{flushleft}
MCL142 Thermal Science for Electrical Engineers
\end{flushleft}


\begin{flushleft}
3 Credits (3-0-0)
\end{flushleft}


\begin{flushleft}
Overlaps with: MCL140, MCL141, CLL121
\end{flushleft}


\begin{flushleft}
Introduction to applications. Basic concepts and definitions -- system,
\end{flushleft}


\begin{flushleft}
boundary, equilibrium, steady state and others. Thermodynamic
\end{flushleft}


\begin{flushleft}
properties of a pure substance -- saturated and other states. Work
\end{flushleft}


\begin{flushleft}
and heat -- definition and applications. 1st Law -- internal energy and
\end{flushleft}


\begin{flushleft}
enthalpy, applications to non-flow/closed and flow/open systems (SSSF
\end{flushleft}


\begin{flushleft}
and USUF). 2nd Law -- corollaries, Clausius inequality, entropy. Carnot
\end{flushleft}


\begin{flushleft}
cycle. Basics of gas-vapor mixtures. Vapor power cycles -- Rankine cycle
\end{flushleft}


\begin{flushleft}
and its modifications. Steam generation and its use -- power plants,
\end{flushleft}


\begin{flushleft}
co-generation, combined cycles. Introduction to various equipment
\end{flushleft}


\begin{flushleft}
in thermal power plant.
\end{flushleft}


\begin{flushleft}
Introduction to transport phenomena: various modes of transport of
\end{flushleft}


\begin{flushleft}
momentum and energy - diffusion and advective transport. Modes
\end{flushleft}


\begin{flushleft}
of heat transfer in various applications. Conduction: Heat diffusion
\end{flushleft}


\begin{flushleft}
equation, 1-D steady state conduction in extended surfaces, infinite
\end{flushleft}


\begin{flushleft}
and semi-infinite walls, heat generation, lumped capacitance.
\end{flushleft}


\begin{flushleft}
Convection: Forced and free convection - mass, momentum and energy
\end{flushleft}


\begin{flushleft}
conservation equations, non-dimensional numbers, hydrodynamic
\end{flushleft}


\begin{flushleft}
and thermal boundary layers, basics of heat transfer in external
\end{flushleft}


\begin{flushleft}
and internal laminar and turbulent flows, and use of co-relations.
\end{flushleft}


\begin{flushleft}
Radiation: properties, Laws, 3-surface network for diffuse-gray
\end{flushleft}


\begin{flushleft}
surfaces. Familiarization with heat exchangers. Application area
\end{flushleft}


\begin{flushleft}
example: cooling of electronics.
\end{flushleft}





\begin{flushleft}
MCL201 Mechanical Engineering Drawing
\end{flushleft}


\begin{flushleft}
3.5 Credits (2-0-3)
\end{flushleft}


\begin{flushleft}
Pre-requisites: MCP100
\end{flushleft}





\begin{flushleft}
Introduction to limits, fits and tolerances, dimensional and geometric
\end{flushleft}


\begin{flushleft}
tolerances, surface finish symbols.
\end{flushleft}


\begin{flushleft}
Generation of assembly drawings including sectioning and bill of
\end{flushleft}


\begin{flushleft}
materials.
\end{flushleft}


\begin{flushleft}
Evolving details of components from assembly considerations. Detailing
\end{flushleft}


\begin{flushleft}
of components involving shafts, bearing, pulleys, gears, belts, brackets
\end{flushleft}


\begin{flushleft}
for assembly.
\end{flushleft}


\begin{flushleft}
Solid modeling of above assembly and incorporating assembly
\end{flushleft}


\begin{flushleft}
constraints for animation of motion of machine assemblies.
\end{flushleft}





\begin{flushleft}
MCL211 Design of Machines
\end{flushleft}


\begin{flushleft}
4 Credits (3-0-2)
\end{flushleft}


\begin{flushleft}
Pre-requisites: APL104, MCL100, MCL201
\end{flushleft}


\begin{flushleft}
Conceptualization of a machine in terms of geometrical requirements
\end{flushleft}


\begin{flushleft}
specified in terms of functional degrees of freedom, degrees of
\end{flushleft}


\begin{flushleft}
constraints and stiffness. Synthesis of an assembly from machine
\end{flushleft}


\begin{flushleft}
components to meet the functional requirements. Sizing machine
\end{flushleft}


\begin{flushleft}
components and selecting material through use of free body
\end{flushleft}


\begin{flushleft}
diagrams, failure theories in static and repeated loading. Design
\end{flushleft}


\begin{flushleft}
and selection of certain machine elements (i.e. cams, gears, beltpulleys, bearings, springs, shaft/axle, plates, nuts and bolts, brake/
\end{flushleft}


\begin{flushleft}
clutch) as exemplars. Case studies (like Gearbox driven by motor
\end{flushleft}


\begin{flushleft}
using belt drive) through use of parametric software to carry out
\end{flushleft}


\begin{flushleft}
iteration in the design space.
\end{flushleft}





\begin{flushleft}
MCL212 Control theory and applications
\end{flushleft}


\begin{flushleft}
4 Credits (3-0-2)
\end{flushleft}


\begin{flushleft}
Pre-requisites: MTL100, MTL101
\end{flushleft}


\begin{flushleft}
Overlaps with: 50-60\% with ELL301 and CLL261
\end{flushleft}


\begin{flushleft}
Introduction; Fourier and Laplace transforms; Mathematical Modeling
\end{flushleft}


\begin{flushleft}
of simple physical systems; Transfer function; Block diagrams; Signal
\end{flushleft}


\begin{flushleft}
flow graph; Transient response analysis using Laplace transform;
\end{flushleft}


\begin{flushleft}
Frequency response; Design/performance specifications in time and
\end{flushleft}


\begin{flushleft}
frequency domain; Steady state error and error constants;
\end{flushleft}


\begin{flushleft}
Proportional, integral, derivative, PD and PID control; Sensors and
\end{flushleft}


\begin{flushleft}
actuators for temperature, pressure, flow and motion control systems;
\end{flushleft}


\begin{flushleft}
Realization of standard controllers using hydraulic, pneumatic,
\end{flushleft}


\begin{flushleft}
electronic, electro-hydraulic and electro-pneumatic systems;
\end{flushleft}


\begin{flushleft}
Stability; Routh's criterion; Nyquist stability criterion, Bode plots;
\end{flushleft}


\begin{flushleft}
Control system design using Root Locus and Frequency response;
\end{flushleft}


\begin{flushleft}
Lead and lag compensation; Gain margin, Phase margin; Introduction
\end{flushleft}


\begin{flushleft}
to Modern control: State space representation; Control with state
\end{flushleft}


\begin{flushleft}
feedback; Review of applications of control in: Machine tools,
\end{flushleft}


\begin{flushleft}
Aerospace, Boiler, Engine Governing, Active vibration control.
\end{flushleft}





\begin{flushleft}
MCL231 Manufacturing Processes-II
\end{flushleft}


\begin{flushleft}
3 Credits (3-0-0)
\end{flushleft}


\begin{flushleft}
Pre-requisites: MCL131
\end{flushleft}


\begin{flushleft}
Overlaps with: MCL134, MCL136
\end{flushleft}


\begin{flushleft}
Introduction to Metal Machining and Machine Tools, Geometry of
\end{flushleft}


\begin{flushleft}
cutting tools, Mechanics of Machining including force and temperature
\end{flushleft}


\begin{flushleft}
generation, Methods of measurement of forces and temperature
\end{flushleft}


\begin{flushleft}
(experimentally and analytically), Tool wear mechanisms and tool life
\end{flushleft}


\begin{flushleft}
criteria, Basic concepts of cost and economics of machining.
\end{flushleft}


\begin{flushleft}
Various types of machine tools and their development with regard to
\end{flushleft}


\begin{flushleft}
productivity \& accuracy requirements, Workholding and tool holding
\end{flushleft}


\begin{flushleft}
devices for machine tools.
\end{flushleft}


\begin{flushleft}
Introduction to non conventional machining processes and
\end{flushleft}


\begin{flushleft}
understanding basic mechanisms of material removal in such processes
\end{flushleft}


\begin{flushleft}
Introduction to metrology, Dimensional Inspection, Inspection by
\end{flushleft}


\begin{flushleft}
measurement, Limit gauging, Design of Limit gauges, Surface quality
\end{flushleft}


\begin{flushleft}
inspection, Feature inspection.
\end{flushleft}





\begin{flushleft}
Sectioning, dimensioning, notes and version control in drawings.
\end{flushleft}





\begin{flushleft}
MCP231 Manufacturing Laboratory-I
\end{flushleft}


\begin{flushleft}
1 Credit (0-0-2)
\end{flushleft}


\begin{flushleft}
Pre-requisites: MCL131
\end{flushleft}


\begin{flushleft}
Overlaps with: MCP232 (60\%)
\end{flushleft}





\begin{flushleft}
Standardized representation of threads, fasteners, welds, bearings,
\end{flushleft}


\begin{flushleft}
springs and related components.
\end{flushleft}





\begin{flushleft}
Experiments on casting, joining, forming, injection molding and powder
\end{flushleft}


\begin{flushleft}
metallurgical processes.
\end{flushleft}





\begin{flushleft}
Introduction to generation of drawings as a design process for machine
\end{flushleft}


\begin{flushleft}
assembly. Use of datum planes to locate features and machine
\end{flushleft}


\begin{flushleft}
elements uniquely in assemblies.
\end{flushleft}





257





\begin{flushleft}
\newpage
Mechanical Engineering
\end{flushleft}





\begin{flushleft}
MCP232 Production Engineering Laboratory-I
\end{flushleft}


\begin{flushleft}
1 Credit (0-0-2)
\end{flushleft}


\begin{flushleft}
Pre-requisites: MCL132, MCL133, MCL134
\end{flushleft}


\begin{flushleft}
Overlaps with: MCP231 (60\%)
\end{flushleft}





\begin{flushleft}
Introduction to Continuous Time Markov Chains (CTMC), Transient
\end{flushleft}


\begin{flushleft}
and Limiting analysis of DTMC, Applications, Discrete Event Simulation
\end{flushleft}


\begin{flushleft}
- Introduction, Generation of Random Variables, Simulation modeling
\end{flushleft}


\begin{flushleft}
through case studies.
\end{flushleft}





\begin{flushleft}
Experiments on casting, forming, injection molding and powder
\end{flushleft}


\begin{flushleft}
metallurgical processes.
\end{flushleft}





\begin{flushleft}
MCL241 Energy systems and Technologies
\end{flushleft}


\begin{flushleft}
4 Credits (3-0.5-1)
\end{flushleft}


\begin{flushleft}
Pre-requisites: MCL140
\end{flushleft}


\begin{flushleft}
Overlaps with: ESL714 ($>$50\%)
\end{flushleft}





\begin{flushleft}
MCP301 Mechanical Engineering Laboratory-I
\end{flushleft}


\begin{flushleft}
1.5 Credits (0-0-3)
\end{flushleft}


\begin{flushleft}
Pre-requisites: APL104, APL106, MCL111, MCL140, MCL241
\end{flushleft}


\begin{flushleft}
Experiments pertaining to applications of the concepts learnt in the
\end{flushleft}


\begin{flushleft}
theory courses of Fluid Mech, Solid Mech, Thermodynamics, Kinematics
\end{flushleft}


\begin{flushleft}
and dynamics and Energy Systems.
\end{flushleft}





\begin{flushleft}
Energy sources :
\end{flushleft}


\begin{flushleft}
Fuels : Fossil fuels, Nuclear fuels, Direct Solar, Indirect solar - Biomass,
\end{flushleft}


\begin{flushleft}
Ocean, Tidal, Hydro, Wind etc. Energy demand/ Growth/ economics;
\end{flushleft}


\begin{flushleft}
Fuel upgradation: gasification of coal and biomass; biogas
\end{flushleft}


\begin{flushleft}
Energy conversion: Direct Conversion: Solar PV, Fuel Cells,
\end{flushleft}


\begin{flushleft}
Thermoelectric Conversion. Thermal to electric: IC Engines, Gas and
\end{flushleft}


\begin{flushleft}
Steam Turbines; Electromechanical conversion; Hydraulic turbines.
\end{flushleft}


\begin{flushleft}
Chemical to Thermal: Combustion and stoichiometry.
\end{flushleft}


\begin{flushleft}
Energy utilization : Refrigeration, HVAC, Desalination, Polygeneration;
\end{flushleft}


\begin{flushleft}
pumps and compressors
\end{flushleft}


\begin{flushleft}
Energy storage : Thermal / Mechanical / Electric / Chemical
\end{flushleft}


\begin{flushleft}
Environmental Impact : Air/ water / soil / nuclear waste.
\end{flushleft}





\begin{flushleft}
MCL242 Heat and Mass Transfer
\end{flushleft}


\begin{flushleft}
4 Credits (3-1-0)
\end{flushleft}


\begin{flushleft}
Pre-requisites: MCL140 \& APL106
\end{flushleft}


\begin{flushleft}
Overlaps with: CLL251
\end{flushleft}





\begin{flushleft}
MCL311 CAD and Finite Element Analysis
\end{flushleft}


\begin{flushleft}
4 Credits (3-0-2)
\end{flushleft}


\begin{flushleft}
Pre-requisites: APL104, MCL211
\end{flushleft}


\begin{flushleft}
Overlaps with: AML705, 706, 710 (course should be mutually
\end{flushleft}


\begin{flushleft}
exclusive w.r.t these courses)
\end{flushleft}


\begin{flushleft}
Introduction and overview. Need and Scope of Computer Aided
\end{flushleft}


\begin{flushleft}
Machine Design. Role of Geometric Modelling, FE and Optimization;
\end{flushleft}


\begin{flushleft}
2D and 3D Geometric transformations and projections. The Viewing
\end{flushleft}


\begin{flushleft}
pipeline; Geometric modeling; Modelling of curves, cubics, splines,
\end{flushleft}


\begin{flushleft}
beziers and b-splines, NURBS; Modeling of surfaces; Modeling of
\end{flushleft}


\begin{flushleft}
solids--b-rep, CSG, octree, feature based modeling; Introduction to
\end{flushleft}


\begin{flushleft}
the Finite Element Method, principle of potential energy; 1D elements,
\end{flushleft}


\begin{flushleft}
Derivation of Stiffness and Mass matrices for a bar, a beam and a shaft,
\end{flushleft}


\begin{flushleft}
FEA using 2D and 3D elements; Plain strain and plain stress problems,
\end{flushleft}


\begin{flushleft}
plates / shell elements; Importance of Finite element mesh, Automatic
\end{flushleft}


\begin{flushleft}
meshing techniques; Interfacing with CAD software.
\end{flushleft}


\begin{flushleft}
Introduction to Thermal analysis, Dynamic analysis using eigen values,
\end{flushleft}


\begin{flushleft}
and Non linear analysis; Limitations of FEM.
\end{flushleft}





\begin{flushleft}
Modes of heat transfer, energy carriers and continuum approximation.
\end{flushleft}


\begin{flushleft}
Mechanisms of mass transfer. Unified view of momentum, heat and
\end{flushleft}


\begin{flushleft}
mass transfer.
\end{flushleft}


\begin{flushleft}
Conduction: Fourier's law, heat diffusion equation, 1-D steady state
\end{flushleft}


\begin{flushleft}
conduction in extended surfaces, heat generation, lumped capacitance
\end{flushleft}


\begin{flushleft}
and 1D transient models, semi-infinite wall. Diffusion mass transfer
\end{flushleft}


\begin{flushleft}
in 1D: steady state and transient.
\end{flushleft}


\begin{flushleft}
Convection: Forced and free convection - mass, momentum and
\end{flushleft}


\begin{flushleft}
energy conservation equations, scaling analysis and significance of
\end{flushleft}


\begin{flushleft}
non-dimensional numbers, thermal boundary layers, heat transfer
\end{flushleft}


\begin{flushleft}
in external and internal laminar and turbulent flows, and use of
\end{flushleft}


\begin{flushleft}
correlations. Convective mass transfer. Boiling and condensation:
\end{flushleft}


\begin{flushleft}
physical phenomena and correlations.
\end{flushleft}


\begin{flushleft}
Heat exchanger types and analysis: LMTD and effectiveness-NTU method.
\end{flushleft}


\begin{flushleft}
Radiation: properties, Laws, view factor, 3-surface network for diffusegray surfaces. Gas radiation.
\end{flushleft}





\begin{flushleft}
MCL261 Introduction to Operations Research
\end{flushleft}


\begin{flushleft}
3 Credits (3-0-0)
\end{flushleft}


\begin{flushleft}
Pre-requisites: MTL108
\end{flushleft}


\begin{flushleft}
Introduction to Modeling, Linear Programming - Formulation, Solution
\end{flushleft}


\begin{flushleft}
methods including Simplex, Primal-Dual, Integer ProgrammingFormulation, Solution methods, Introduction to Dynamic Programming,
\end{flushleft}


\begin{flushleft}
Software Tools and Case Studies.
\end{flushleft}





\begin{flushleft}
MCP261 Industrial Engineering Laboratory-I
\end{flushleft}


\begin{flushleft}
1 Credit (0-0-2)
\end{flushleft}


\begin{flushleft}
Pre-requisites: MCL261, MCL262
\end{flushleft}


\begin{flushleft}
Deterministic optimization problem formulation, solution using CPLEX,
\end{flushleft}


\begin{flushleft}
sensitivity analysis; Conceptualization/Visualization of problem situation,
\end{flushleft}


\begin{flushleft}
formulation of simulation model, simulation runs and output analysis.
\end{flushleft}





\begin{flushleft}
MCL262 Stochastic Modelling and Simulation
\end{flushleft}


\begin{flushleft}
3 Credits (3-0-0)
\end{flushleft}


\begin{flushleft}
Pre-requisites: MTL108
\end{flushleft}


\begin{flushleft}
Overview of Probability Basics, Introduction to Discrete Time
\end{flushleft}


\begin{flushleft}
Markov Chains (DTMC), Transient and Limiting analysis of DTMC,
\end{flushleft}





\begin{flushleft}
MCL314 Acoustics and Noise Control
\end{flushleft}


\begin{flushleft}
4 Credits (3-0-2)
\end{flushleft}


\begin{flushleft}
Pre-requisites: APL100
\end{flushleft}


\begin{flushleft}
Overlaps with: MEL733 (20\%), MEL746 (50\%), ITL 760 (60\%)
\end{flushleft}


\begin{flushleft}
Fundamentals of acoustics, Reflection and transmission of waves,
\end{flushleft}


\begin{flushleft}
Sound sources and generation mechanisms, Human physiological
\end{flushleft}


\begin{flushleft}
response to noise, Sound measurement, Sound in enclosed
\end{flushleft}


\begin{flushleft}
spaces, Sound absorption, Acoustic enclosures and barriers, Sound
\end{flushleft}


\begin{flushleft}
propagation in ducts, Vibration control, Active noise control, Overview
\end{flushleft}


\begin{flushleft}
of Numerical acoustics.
\end{flushleft}





\begin{flushleft}
MCL321 Automotive Systems
\end{flushleft}


\begin{flushleft}
4 Credits (3-0-2)
\end{flushleft}


\begin{flushleft}
Overlaps with: MEL311 (10\%),
\end{flushleft}


\begin{flushleft}
Review of basic engine management systems, alternative fuel systems,
\end{flushleft}


\begin{flushleft}
fuel ignition systems, hybrid electric vehicles, exhaust emission
\end{flushleft}


\begin{flushleft}
systems, drivetrain systems, chassis, environmental management
\end{flushleft}


\begin{flushleft}
and service information systems. Introduction of torque converters,
\end{flushleft}


\begin{flushleft}
planetary gears, clutches, differentials, all-wheel drive, heating and
\end{flushleft}


\begin{flushleft}
air conditioning systems, and interaction of tyre and road interface.
\end{flushleft}


\begin{flushleft}
History of engine technology. Detail of starting and charging systems.
\end{flushleft}


\begin{flushleft}
Details of steering and suspension systems. Details of bearing and
\end{flushleft}


\begin{flushleft}
lubrication systems.
\end{flushleft}





\begin{flushleft}
MCL322 Power Train Design
\end{flushleft}


\begin{flushleft}
3 Credits (3-0-0)
\end{flushleft}


\begin{flushleft}
Pre-requisites: MCL211
\end{flushleft}


\begin{flushleft}
Overlaps with: MEL311 (10\%)
\end{flushleft}


\begin{flushleft}
Introduction of components of automotive powertrain system, viz.,
\end{flushleft}


\begin{flushleft}
engines, transmission, clutches and brakes. Engine characteristics.
\end{flushleft}


\begin{flushleft}
Throttle system, Turbochargers, History and design of valve train.
\end{flushleft}


\begin{flushleft}
Design of variable valve timing system, Exhaust gas recirculation.
\end{flushleft}


\begin{flushleft}
Materials in powertrain components. Lubrication systems to minimize
\end{flushleft}


\begin{flushleft}
life cycle costs. Modelling and design of gearbox. Role of control
\end{flushleft}


\begin{flushleft}
system in advanced (i.e. direct injection, active boosting, camless)
\end{flushleft}


\begin{flushleft}
powertrain system.
\end{flushleft}





258





\begin{flushleft}
\newpage
Mechanical Engineering
\end{flushleft}





\begin{flushleft}
MCL330 Special Topics Production Engineering
\end{flushleft}


\begin{flushleft}
3 Credits (3-0-0)
\end{flushleft}


\begin{flushleft}
Pre-requisites: To be defined by the course coordinator at the
\end{flushleft}


\begin{flushleft}
time of offering the course if required
\end{flushleft}


\begin{flushleft}
Specialized topics in Production Engineering. The detailed contents
\end{flushleft}


\begin{flushleft}
will be decided by the faculty who will reach the course.
\end{flushleft}





\begin{flushleft}
MCL331 Micro and Nano Manufacturing
\end{flushleft}


\begin{flushleft}
3 Credits (3-0-0)
\end{flushleft}


\begin{flushleft}
Pre-requisites: for ME1: MCL131, MCL231
\end{flushleft}


\begin{flushleft}
for ME2: MCL136
\end{flushleft}


\begin{flushleft}
An overview of micro and nano mechanical systems and their
\end{flushleft}


\begin{flushleft}
applications in Mechanical Engineering, MEMS Microfabrication
\end{flushleft}


\begin{flushleft}
methods, Silicon Micromachining methods, Laser,Electron and
\end{flushleft}


\begin{flushleft}
Ion beam micromachining methods, Mechanical Micromachining
\end{flushleft}


\begin{flushleft}
techniques, Nanomanufacturing methods, nanomaterials and
\end{flushleft}


\begin{flushleft}
nano metrology.
\end{flushleft}





\begin{flushleft}
MCP331 Manufacturing Laboratory-II
\end{flushleft}


\begin{flushleft}
1 Credit (0-0-2)
\end{flushleft}


\begin{flushleft}
Pre-requisites: for ME1: MCL131, MCL231
\end{flushleft}


\begin{flushleft}
for ME2: MCL136
\end{flushleft}


\begin{flushleft}
Experiments on machining and metrology.
\end{flushleft}





\begin{flushleft}
MCP332 Production Engineering Laboratory-II
\end{flushleft}


\begin{flushleft}
1 Credit (0-0-2)
\end{flushleft}


\begin{flushleft}
Pre-requisites: MCL135, MCL136, MCP232
\end{flushleft}


\begin{flushleft}
Experiments on machining and welding processes.
\end{flushleft}





\begin{flushleft}
MCL334 Industrial Automation
\end{flushleft}


\begin{flushleft}
4 Credits (3-0-2)
\end{flushleft}


\begin{flushleft}
Pre-requisites: ELL100 and APL106 or MCL141
\end{flushleft}


\begin{flushleft}
Overlaps with: 5\% with MEL312
\end{flushleft}


\begin{flushleft}
Introduction to Automation technologies, applications around us and in
\end{flushleft}


\begin{flushleft}
manufacturing. Types of systems - mechanical, electrical, electronics;
\end{flushleft}


\begin{flushleft}
Sensors, Factory Automation Sensors, Electrical sensors, Process
\end{flushleft}


\begin{flushleft}
Automation Sensors and their interfaces; Hydraulics \& Pneumatic
\end{flushleft}


\begin{flushleft}
Systems and components; Circuit design approach and examples;
\end{flushleft}


\begin{flushleft}
Sequence operation of more than two cylinders and motors; Electro
\end{flushleft}


\begin{flushleft}
Pneumatic \& Electro Hydraulic Systems, Relay Logic circuits, Feedback
\end{flushleft}


\begin{flushleft}
control systems; Programmable Logic Controllers, programming
\end{flushleft}


\begin{flushleft}
languages \& instruction set, ladder logic, functional blocks, structured
\end{flushleft}


\begin{flushleft}
text, and applications. Human Machine Interface \& SCADA; Motion
\end{flushleft}


\begin{flushleft}
controller, stepper \& servo motors, multi axes coordinated motion,
\end{flushleft}


\begin{flushleft}
CNC control; RFID technology and its application; Machine vision and
\end{flushleft}


\begin{flushleft}
control applications.
\end{flushleft}


\begin{flushleft}
Laboratory work will be hands-on design and operation of automatic
\end{flushleft}


\begin{flushleft}
systems.
\end{flushleft}





\begin{flushleft}
MCL336 Advances in Welding
\end{flushleft}


\begin{flushleft}
4 Credits (3-0-2)
\end{flushleft}


\begin{flushleft}
Pre-requisites: MCL131 or MCL135
\end{flushleft}


\begin{flushleft}
Introduction to joining technology, General survey and classification
\end{flushleft}


\begin{flushleft}
of Welding processes, importance of advanced materials and joining
\end{flushleft}


\begin{flushleft}
technologies, welding technologies related to industries: automotive,
\end{flushleft}


\begin{flushleft}
aerospace, nuclear, oil and gas industries.
\end{flushleft}





\begin{flushleft}
MCL337 Advanced Machining Processes
\end{flushleft}


\begin{flushleft}
3 Credits (3-0-0)
\end{flushleft}


\begin{flushleft}
Prerequisites: MCL231 or MCL136
\end{flushleft}


\begin{flushleft}
Overlaps with: \~{}20\% overlap with MCL231 and MCL136
\end{flushleft}


\begin{flushleft}
Introduction to advanced machining processes -- need for such
\end{flushleft}


\begin{flushleft}
processes and application areas.
\end{flushleft}


\begin{flushleft}
Mechanical Energy utilized advanced machining processes like
\end{flushleft}


\begin{flushleft}
ultrasonic machining, abrasive flow machining, magnetic abrasive
\end{flushleft}


\begin{flushleft}
finishing, magneto-rheological finishing, abrasive water jet machining --
\end{flushleft}





\begin{flushleft}
mechanics of cutting, process parametric analysis, process capabilities,
\end{flushleft}


\begin{flushleft}
applications.
\end{flushleft}


\begin{flushleft}
Thermoelectric based advanced machining processes like electro
\end{flushleft}


\begin{flushleft}
discharge machining, wire EDM, Plasma Arc Machining, Laser Beam
\end{flushleft}


\begin{flushleft}
Machining, Focussed Ion Beam Machining -- working principles,
\end{flushleft}


\begin{flushleft}
material removal mechanisms, process capabilities and applications.
\end{flushleft}


\begin{flushleft}
Electrochemical and Chemical Advanced Machining -- ECG,
\end{flushleft}


\begin{flushleft}
Electrostream Drilling, Chemical Machining -- process characteristics,
\end{flushleft}


\begin{flushleft}
numerical modelling of the processes, applications and limitations.
\end{flushleft}





\begin{flushleft}
MCL338 Mechatronic Applications in Manufacturing
\end{flushleft}


\begin{flushleft}
4 Credits (3-0-2)
\end{flushleft}


\begin{flushleft}
Pre-requisites: ELL100
\end{flushleft}


\begin{flushleft}
Overlaps with: MEL749, EEL482
\end{flushleft}


\begin{flushleft}
Introduction to mechatronic systems and components, Review of
\end{flushleft}


\begin{flushleft}
manufacturing and need and integration of mechatronics at different
\end{flushleft}


\begin{flushleft}
levels, Principles of basic electronics, Digital electronics review:
\end{flushleft}


\begin{flushleft}
number system, gates, flip-flops, counters, registers, tri-state concept,
\end{flushleft}


\begin{flushleft}
TTL and CMOS circuits, memories. Embedded electronics, Basics of
\end{flushleft}


\begin{flushleft}
Microcontroller \& Microprocessors architecture and instruction set,
\end{flushleft}


\begin{flushleft}
machine cycles, interrupts, instruction set, memory and I/O interfacing,
\end{flushleft}


\begin{flushleft}
programming techniques, Timer/Counters, Serial Interfacing and
\end{flushleft}


\begin{flushleft}
communications, Interfacing to keyboards and displays, Standard
\end{flushleft}


\begin{flushleft}
busses. Microcontrollers and their applications, integrated circuits,
\end{flushleft}


\begin{flushleft}
sensors, actuators, and other electrical/electronic hardware in
\end{flushleft}


\begin{flushleft}
mechatronic systems. Microprocessor based measurement and
\end{flushleft}


\begin{flushleft}
control: D/A and A/D conversion, data acquisition systems, encoders,
\end{flushleft}


\begin{flushleft}
interfacing of motors and transducers. Selection of mechatronic
\end{flushleft}


\begin{flushleft}
components, namely sensors like encoders and resolvers. Stepper
\end{flushleft}


\begin{flushleft}
and servomotors; Solenoid like actuators; Transmission elements
\end{flushleft}


\begin{flushleft}
like Ball screw and Controllers. Analysis of mechatronic systems with
\end{flushleft}


\begin{flushleft}
applications to motion control, robotics, CNC systems, and others.
\end{flushleft}


\begin{flushleft}
Case studies of applications in process and discrete manufacturing.
\end{flushleft}


\begin{flushleft}
Laboratory work will be hands-on Microcontroller \& Microprocessor
\end{flushleft}


\begin{flushleft}
interfacing and programming, Motion controller, motors, sensors,
\end{flushleft}


\begin{flushleft}
and actuators.
\end{flushleft}





\begin{flushleft}
MCL341 Gas Dynamics and Propulsion
\end{flushleft}


\begin{flushleft}
4 Credits (3-0-2)
\end{flushleft}


\begin{flushleft}
Pre-requisites: MCL140 \& MCL241
\end{flushleft}


\begin{flushleft}
Revision of fundamentals. Thermodynamics of compressible flow
\end{flushleft}


\begin{flushleft}
-- wave motion in compressible medium, Mach number and cone,
\end{flushleft}


\begin{flushleft}
properties. Steady one-dimensional compressible flow through
\end{flushleft}


\begin{flushleft}
variable area ducts. Converging and converging-diverging nozzles
\end{flushleft}


\begin{flushleft}
and diffusers. Effects of heating and friction in duct flow, Rayleigh
\end{flushleft}


\begin{flushleft}
and Fanno lines. Flows with normal shocks. Oblique shocks and
\end{flushleft}


\begin{flushleft}
reflection. Expansion waves. Prandtl-Meyer flow. Flow over bodies.
\end{flushleft}


\begin{flushleft}
Measurements and applications. Jet propulsion -- types of engines,
\end{flushleft}


\begin{flushleft}
propulsion fundamentals. Compressor, combustor and turbines
\end{flushleft}


\begin{flushleft}
construction and performance. Rocket propulsion -- basics, solid and
\end{flushleft}


\begin{flushleft}
liquid propelled engines, parametric studies, construction features,
\end{flushleft}


\begin{flushleft}
single and multi-stage rockets. Thrust chamber and nozzle models.
\end{flushleft}


\begin{flushleft}
Studies of in-use engines. Environmental aspects.
\end{flushleft}





\begin{flushleft}
MCL343 Introduction to Combustion
\end{flushleft}


\begin{flushleft}
3 Credits (3-0-0)
\end{flushleft}


\begin{flushleft}
Pre-requisites: (MCL140 and MCL242) or MCL141
\end{flushleft}


\begin{flushleft}
Introduction. Fuels: gaseous, liquid and solid. Physical and chemical
\end{flushleft}


\begin{flushleft}
characterizations. Chemical thermodynamics and kinetics. Conservation
\end{flushleft}


\begin{flushleft}
equations for multi-component systems. Premixed systems: laminar
\end{flushleft}


\begin{flushleft}
flame problems and effects of different variables. Measurement of
\end{flushleft}


\begin{flushleft}
flame speed. Flammability limits. Ignition and quenching. Turbulent
\end{flushleft}


\begin{flushleft}
premixed flames. Non-premixed systems: laminar diffusion flamejet,
\end{flushleft}


\begin{flushleft}
droplet burning. Combustion of solids: drying, devolatilization and
\end{flushleft}


\begin{flushleft}
char combustion. Biomass combustion devices. Coal combustion.
\end{flushleft}


\begin{flushleft}
Pollution: Main pollutants and their environmental impact. NOx, CO
\end{flushleft}


\begin{flushleft}
and SOx formation chemistry. Particulate pollutants. Emissions from
\end{flushleft}


\begin{flushleft}
engines, power plants and industrial applications. Low NOx burners
\end{flushleft}


\begin{flushleft}
and furnace design.
\end{flushleft}





259





\begin{flushleft}
\newpage
Mechanical Engineering
\end{flushleft}





\begin{flushleft}
MCL344 Refrigeration and Air-conditioning
\end{flushleft}


\begin{flushleft}
4 Credits (3-0-2)
\end{flushleft}


\begin{flushleft}
Pre-requisites: (MCL140 and MCL242) or MCL141
\end{flushleft}


\begin{flushleft}
Overlaps with: ESL850
\end{flushleft}


\begin{flushleft}
Introduction and applications, recapitulation of fundamentals. Vapor
\end{flushleft}


\begin{flushleft}
compression systems: Ideal and real cycle analyses, Refrigerants:
\end{flushleft}


\begin{flushleft}
designation, desirable properties, environmental considerations.
\end{flushleft}


\begin{flushleft}
Advanced vapor compression cycles. Components: condensers,
\end{flushleft}


\begin{flushleft}
evaporators, compressors and expansion devices -- construction,
\end{flushleft}


\begin{flushleft}
operation and performance. Vapor absorption and gas cycle
\end{flushleft}


\begin{flushleft}
refrigeration. Psychrometry. Processes - heating, humidification,
\end{flushleft}


\begin{flushleft}
cooling and dehumidification etc. Air-conditioning calculations,
\end{flushleft}


\begin{flushleft}
Cooling load estimation.
\end{flushleft}





\begin{flushleft}
MCL345 Reciprocating Internal Combustion Engines
\end{flushleft}


\begin{flushleft}
4 Credits (3-0-2)
\end{flushleft}


\begin{flushleft}
Pre-requisites: MCL140 or MCL141
\end{flushleft}


\begin{flushleft}
Overlaps with: $<$10\% with PG I.C. Engine course
\end{flushleft}


\begin{flushleft}
Introduction, Engine design and operating parameters, Ideal
\end{flushleft}


\begin{flushleft}
properties, Models of engine processes and cycles, combustion
\end{flushleft}


\begin{flushleft}
thermodynamics, fuel/air cycle analysis, Spark-Ignition engine
\end{flushleft}


\begin{flushleft}
combustion, SI and Diesel engine emissions, IC Engines: the future.
\end{flushleft}





\begin{flushleft}
MCL347 Intermediate Heat Transfer
\end{flushleft}


\begin{flushleft}
3 Credits (3-0-0)
\end{flushleft}


\begin{flushleft}
Pre-requisites: MTL100, MTL101, MCL242
\end{flushleft}


\begin{flushleft}
Overlaps with: MCL441 (\~{}20\%)
\end{flushleft}


\begin{flushleft}
Heat conduction: Governing equation, Analytical solution of steady and
\end{flushleft}


\begin{flushleft}
unsteady 2D heat conduction, Heat transfer from convective radiative
\end{flushleft}


\begin{flushleft}
fins; Convection: Governing equations, Forced convection heat transfer,
\end{flushleft}


\begin{flushleft}
scale analysis, similarity solutions and momentum integral method for
\end{flushleft}


\begin{flushleft}
laminar flows, energy integral method for turbulent flows; Natural
\end{flushleft}


\begin{flushleft}
convection: similarity solutions and energy integral methods, mixed
\end{flushleft}


\begin{flushleft}
convection; Heat transfer from pipes: analytical solutions; Heat transfer
\end{flushleft}


\begin{flushleft}
at high speed; Radiative heat transfer: blackbody radiation, radiative
\end{flushleft}


\begin{flushleft}
heat transfer between gray, diffusive surfaces, radiative heat exchange
\end{flushleft}


\begin{flushleft}
between non-gray surfaces,gas radiation, enclosure theory, governing
\end{flushleft}


\begin{flushleft}
equation for radiatively participating medium. Boiling: pool boiling,
\end{flushleft}


\begin{flushleft}
development of a correlation for nucleate boiling, critical heat flux,
\end{flushleft}


\begin{flushleft}
flow boiling. Condensation: film condensation over flat and circular
\end{flushleft}


\begin{flushleft}
geometries, Nusselt theory.
\end{flushleft}





\begin{flushleft}
MCL348 Thermal Management of Electronics
\end{flushleft}


\begin{flushleft}
3 Credits (3-0-0)
\end{flushleft}


\begin{flushleft}
Pre-requisites: APL106 and (MCL242 or MCL141)
\end{flushleft}


\begin{flushleft}
Electronics packaging and cooling technologies; Heat sinks: principle,
\end{flushleft}


\begin{flushleft}
types, modelling, and design; Contact resistance; Heat pipes and two
\end{flushleft}


\begin{flushleft}
phase systems: principle, types, modelling and design; Microchannel
\end{flushleft}


\begin{flushleft}
heat exchangers: single phase and two phase; Radiative heat transfer
\end{flushleft}


\begin{flushleft}
and importance in space applications; Thermoelectric devices;
\end{flushleft}


\begin{flushleft}
Measurement and characterisation techniques; Case studies of thermal
\end{flushleft}


\begin{flushleft}
management of electronics.
\end{flushleft}





\begin{flushleft}
MCL350 Mechanical Engineering Product Synthesis
\end{flushleft}


\begin{flushleft}
2 Credits (1-0-2)
\end{flushleft}


\begin{flushleft}
Pre-requisites: Product-related core courses, as specified by
\end{flushleft}


\begin{flushleft}
the instructor
\end{flushleft}


\begin{flushleft}
Study of product specifications, GA drawings, sub-systems and
\end{flushleft}


\begin{flushleft}
component functionalities. Component-wise study of engineering
\end{flushleft}


\begin{flushleft}
design, including, material selection, stress analysis, fluid flow analysis,
\end{flushleft}


\begin{flushleft}
heat transfer analysis, etc. Implications of geometrical dimensioning
\end{flushleft}


\begin{flushleft}
and tolerancing. Materials and manufacturing processes. Assembly.
\end{flushleft}


\begin{flushleft}
Wear, performance deterioration and failure. Testing and certification.
\end{flushleft}


\begin{flushleft}
Failure modes and effects analysis. Modifications and their implications.
\end{flushleft}


\begin{flushleft}
Regulatory requirements. Standards.
\end{flushleft}





\begin{flushleft}
MCL361 Manufacturing System Design
\end{flushleft}


\begin{flushleft}
3 Credits (3-0-0)
\end{flushleft}


\begin{flushleft}
Pre-requisites: MTL108, MCL261
\end{flushleft}





\begin{flushleft}
Manufacturing strategy, Manufacturing flexibility, Manufacturing
\end{flushleft}


\begin{flushleft}
complexity, Investment decisions using life cycle costing, System
\end{flushleft}


\begin{flushleft}
reliability and maintenance models, Economic design of quality
\end{flushleft}


\begin{flushleft}
control plans, Single and mixed model assembly line balancing, Shop
\end{flushleft}


\begin{flushleft}
floor scheduling algorithms, Lot sizing and inventory control models,
\end{flushleft}


\begin{flushleft}
Performance modeling of manufacturing systems, Production control
\end{flushleft}


\begin{flushleft}
mechanisms like Kanban, CONWIP and POL2.
\end{flushleft}





\begin{flushleft}
MCP361 Industrial Engineering Laboratory-II
\end{flushleft}


\begin{flushleft}
1 Credit (0-0-2)
\end{flushleft}


\begin{flushleft}
Pre-requisites: MCP261, MCL361
\end{flushleft}


\begin{flushleft}
Design of optimal acceptance sampling plans, Design of optimal
\end{flushleft}


\begin{flushleft}
control charts, Simulation of process failures, Simulation of machine
\end{flushleft}


\begin{flushleft}
failures and Simulation of job shops and production lines with various
\end{flushleft}


\begin{flushleft}
production control mechanisms.
\end{flushleft}





\begin{flushleft}
MCL363 Investment Planning
\end{flushleft}


\begin{flushleft}
3 Credits (3-0-0)
\end{flushleft}


\begin{flushleft}
Pre-requisites: MCL261 \& MCL262
\end{flushleft}


\begin{flushleft}
Introduction to investment and rate of return, Markowitz theory
\end{flushleft}


\begin{flushleft}
and its applications to optimal portfolio management, Introduction
\end{flushleft}


\begin{flushleft}
to Bonds, Introduction to Derivatives and Options, Concept of Risk
\end{flushleft}


\begin{flushleft}
Neutral Pricing, Single period and multiple period binomial models for
\end{flushleft}


\begin{flushleft}
option pricing, Introduction to Black Scholes model and the formula.
\end{flushleft}





\begin{flushleft}
MCL364 Value Engineering
\end{flushleft}


\begin{flushleft}
4 Credits (3-0-2)
\end{flushleft}


\begin{flushleft}
Overlaps with: MEL671
\end{flushleft}


\begin{flushleft}
Introduction to Value Engineering and Value Analysis, Methodology
\end{flushleft}


\begin{flushleft}
of V.E., Quantitative definition of value, Use value and prestige value,
\end{flushleft}


\begin{flushleft}
Estimation of product quality/performance, Classification of functions,
\end{flushleft}


\begin{flushleft}
functional cost and functional worth, Effect of value improvement
\end{flushleft}


\begin{flushleft}
on profitability. Introduction to V.E. job plan / Functional approach to
\end{flushleft}


\begin{flushleft}
value improvement, Various phases and techniques of the job plan,
\end{flushleft}


\begin{flushleft}
Life Cycle Costing for managing the total value of a product, Cash
\end{flushleft}


\begin{flushleft}
flow diagrams, Concepts in LCC, Present Value concept, Annuity cost
\end{flushleft}


\begin{flushleft}
concept, Net Present Value, Pay Back period, Internal rate of return on
\end{flushleft}


\begin{flushleft}
investment (IRR), Continuous discounting, Examples and illustrations.
\end{flushleft}


\begin{flushleft}
Creative thinking and creative judgment, False material, labor and
\end{flushleft}


\begin{flushleft}
overhead saving, System reliability, Reliability elements in series and
\end{flushleft}


\begin{flushleft}
in parallel, Decision Matrix, Evaluation of value alternatives, Estimation
\end{flushleft}


\begin{flushleft}
of weights and efficiencies, Sensitivity analysis, Utility transformation
\end{flushleft}


\begin{flushleft}
functions, Fast diagramming, Critical path of functions, DARSIRI
\end{flushleft}


\begin{flushleft}
method of value analysis.
\end{flushleft}





\begin{flushleft}
MCL366 OR Methods in Policy Governance
\end{flushleft}


\begin{flushleft}
3 Credits (3-0-0)
\end{flushleft}


\begin{flushleft}
Pre-requisites: MCL261
\end{flushleft}


\begin{flushleft}
Mixed Integer Linear Programming, Markov Decision Processes,
\end{flushleft}


\begin{flushleft}
Applications of OR techniques to aviation security, resource allocation,
\end{flushleft}


\begin{flushleft}
energy policy, railways systems, management of natural resources,
\end{flushleft}


\begin{flushleft}
Public Service Delivery.
\end{flushleft}





\begin{flushleft}
MCL368 Quality and Reliability Engineering
\end{flushleft}


\begin{flushleft}
3 Credits (3-0-0)
\end{flushleft}


\begin{flushleft}
Pre-requisites: MTL108
\end{flushleft}


\begin{flushleft}
Process capability analysis, Process quality improvement approaches,
\end{flushleft}


\begin{flushleft}
Economics of quality control, Reliability data analysis, Component and
\end{flushleft}


\begin{flushleft}
system reliability models, Reliability test plans, Warranty analysis,
\end{flushleft}


\begin{flushleft}
Maintenance models.
\end{flushleft}





\begin{flushleft}
MCL370 Special Topics in Industrial Engineering
\end{flushleft}


\begin{flushleft}
3 Credits (3-0-0)
\end{flushleft}


\begin{flushleft}
Pre-requisites: To be defined by the course coordinator at the
\end{flushleft}


\begin{flushleft}
time of offering the course if required
\end{flushleft}


\begin{flushleft}
Specialized topics in Industrial Engineering. The detailed contents will
\end{flushleft}


\begin{flushleft}
be decided by the faculty who will teach the course.
\end{flushleft}





260





\begin{flushleft}
\newpage
Mechanical Engineering
\end{flushleft}





\begin{flushleft}
MCL380 Special Topics in Mechanical Engineering
\end{flushleft}


\begin{flushleft}
3 Credits (3-0-0)
\end{flushleft}


\begin{flushleft}
Pre-requisites: courses as specified by the instructor and EC 50
\end{flushleft}


\begin{flushleft}
Course details shall be announced at the time of offering of the course.
\end{flushleft}


\begin{flushleft}
The assessment will be based on a combination of assignments,
\end{flushleft}


\begin{flushleft}
quizzes, and term paper and tests.
\end{flushleft}





\begin{flushleft}
MCV390 Special module in Mechanical Engineering
\end{flushleft}


\begin{flushleft}
1 Credit (1-0-0)
\end{flushleft}


\begin{flushleft}
Pre-requisites: courses as specified by the instructor and EC 50
\end{flushleft}


\begin{flushleft}
Course details shall be announced at the time of offering of the
\end{flushleft}


\begin{flushleft}
course. The lectures will be supplemented by reading materials. The
\end{flushleft}


\begin{flushleft}
assessment will be based on a combination of assignments, quizzes,
\end{flushleft}


\begin{flushleft}
and term papers (to be announced by the instructor) and tests.
\end{flushleft}





\begin{flushleft}
MCP401 Mechanical Engineering Laboratory-II
\end{flushleft}


\begin{flushleft}
2 Credits (0-0-4)
\end{flushleft}


\begin{flushleft}
Pre-requisites: MCL211, MCL212, MCL242, MCP301
\end{flushleft}


\begin{flushleft}
The experiments would involve full or partial fabrication of setups and
\end{flushleft}


\begin{flushleft}
then taking readings and analysis of its behavior, instead of using ready
\end{flushleft}


\begin{flushleft}
made setups. The knowledge gained in control engineering course
\end{flushleft}


\begin{flushleft}
would also be used for setting up computerised measurements using
\end{flushleft}


\begin{flushleft}
Data acquisition cards.
\end{flushleft}





\begin{flushleft}
MCD411 B.Tech. Project-I
\end{flushleft}


\begin{flushleft}
4 Credits (0-0-8)
\end{flushleft}


\begin{flushleft}
Pre-requisites: EC 100
\end{flushleft}


\begin{flushleft}
A broad outline of the contents is as follows and a project may include
\end{flushleft}


\begin{flushleft}
some or all of these activities:
\end{flushleft}


\begin{flushleft}
Team formation for designing, manufacturing and operating a
\end{flushleft}


\begin{flushleft}
selected product, formulating project management procedures. Need
\end{flushleft}


\begin{flushleft}
identification, assessment of alternative designs, selection of design
\end{flushleft}


\begin{flushleft}
for development, defining design and performance specifications, and
\end{flushleft}


\begin{flushleft}
testing procedure. Detailed mechanical, thermal and manufacturingrelated design of systems, assemblies, sub-assemblies and components
\end{flushleft}


\begin{flushleft}
culminating in engineering drawings and material specifications;
\end{flushleft}


\begin{flushleft}
preparing bill of materials and identification of standard components
\end{flushleft}


\begin{flushleft}
and bought-out parts.
\end{flushleft}


\begin{flushleft}
Using engineering drawings, the process sheets are developed based
\end{flushleft}


\begin{flushleft}
on available materials, machine tools and other fabrication facilities.
\end{flushleft}


\begin{flushleft}
Materials and standard components are procured and manufacturing is
\end{flushleft}


\begin{flushleft}
carried out. After inspection, parts are accepted. Assembly procedure
\end{flushleft}


\begin{flushleft}
is finalized and the machine is assembled. Acceptance tests are
\end{flushleft}


\begin{flushleft}
carried out vis-\`{a}-vis specifications. Professional quality documentation
\end{flushleft}


\begin{flushleft}
of all designs, data, drawings, and results, change history, overall
\end{flushleft}


\begin{flushleft}
assessment, etc. is mandatory, along with a final presentation.
\end{flushleft}





\begin{flushleft}
MCD412 B.Tech. Project-II
\end{flushleft}


\begin{flushleft}
7 Credits (0-0-14)
\end{flushleft}


\begin{flushleft}
Pre-requisites: EC 100
\end{flushleft}


\begin{flushleft}
MCL421 Automotive Structural Design
\end{flushleft}


\begin{flushleft}
3 Credits (2-0-2)
\end{flushleft}


\begin{flushleft}
Pre-requisites: MCL211, MCL321
\end{flushleft}


\begin{flushleft}
Overlaps with: MEL736 (40\% - Students should be allowed to
\end{flushleft}


\begin{flushleft}
do only one of the two courses)
\end{flushleft}


\begin{flushleft}
History of automotive design, Design cycle for an Automobile,
\end{flushleft}


\begin{flushleft}
Styling, Loads on the chassis, Chassis and structural Design for static
\end{flushleft}


\begin{flushleft}
loads, Dynamic and impact loads, Energy absorption in the vehicle,
\end{flushleft}


\begin{flushleft}
Designing for NVH, Designing the suspension system, Designing the
\end{flushleft}


\begin{flushleft}
brake system.
\end{flushleft}





\begin{flushleft}
MCL422 Design of Brake Systems
\end{flushleft}


\begin{flushleft}
3 Credits (2-0-2)
\end{flushleft}


\begin{flushleft}
Pre-requisites: APL104, MCL111, MCL321
\end{flushleft}


\begin{flushleft}
Types of brakes, Friction materials in brakes and their characteristics,
\end{flushleft}


\begin{flushleft}
Design of brakes in passenger cars / vans: weight transfer, effect
\end{flushleft}





\begin{flushleft}
of tyre / road adhesion, wheel lock, brake efficiency / adhesion
\end{flushleft}


\begin{flushleft}
utilization; Design of brakes in vehicle -- trailer combinations: in light
\end{flushleft}


\begin{flushleft}
trailers, overrun brakes, center axle trailer, chassis trailer; Brakedesign analysis: Brake and shoe factors in different types of brakes,
\end{flushleft}


\begin{flushleft}
Comparison of estimation by analytical and FE methods; Thermal
\end{flushleft}


\begin{flushleft}
effects in friction brakes (thermal analysis and heat dissipation); Issues
\end{flushleft}


\begin{flushleft}
in electronic control of brakes: features of anti-lock brake system,
\end{flushleft}


\begin{flushleft}
Traction Control System, Electronic Stability Control, Adaptive Cruise
\end{flushleft}


\begin{flushleft}
Control, trailer Sway Control; Brake Noise: Sources, its analysis (using
\end{flushleft}


\begin{flushleft}
analytical and FE based approaches) and control.
\end{flushleft}





\begin{flushleft}
MCL431 CAM and Automation
\end{flushleft}


\begin{flushleft}
3 Credits (2-0-2)
\end{flushleft}


\begin{flushleft}
Pre-requisites: for ME1: MCL131, MCL231
\end{flushleft}


\begin{flushleft}
for ME2: MCL136
\end{flushleft}


\begin{flushleft}
Automation need and types of automation, economics of automation,
\end{flushleft}


\begin{flushleft}
FMS, CIM. Basics of electro-mechanical automation technologies,
\end{flushleft}


\begin{flushleft}
Circuit design and applications of hydraulic, pneumatic, electropneumatic, electro-hydraulic and programmable logic control (PLC)
\end{flushleft}


\begin{flushleft}
systems. Numerical control, NC and CNC hardware and programming,
\end{flushleft}


\begin{flushleft}
Machine controls, HMI design and implementation, DNC system,
\end{flushleft}


\begin{flushleft}
Control engineering in production systems: open loop and closed loop
\end{flushleft}


\begin{flushleft}
control systems, Automated material handling technologies, Group
\end{flushleft}


\begin{flushleft}
technology, Computer aided process planning, Inspection automation
\end{flushleft}


\begin{flushleft}
and reverse engineering, Rapid prototyping and tooling concepts and
\end{flushleft}


\begin{flushleft}
applications, virtual manufacturing.
\end{flushleft}





\begin{flushleft}
MCL441 Modelling and Experiments in Heat Transfer
\end{flushleft}


\begin{flushleft}
4 Credits (2-0-4)
\end{flushleft}


\begin{flushleft}
Pre-requisites: (MCL242 or MCL141) and MCP301
\end{flushleft}


\begin{flushleft}
Modelling heat transfer phenomena,comparison with experimental
\end{flushleft}


\begin{flushleft}
data, assumptions and their implications. Mathematical modeling:
\end{flushleft}


\begin{flushleft}
dimensional analysis, scaling, physical similarity, self-similarity,
\end{flushleft}


\begin{flushleft}
physical laws and constitutive relations. Solution methodologies:
\end{flushleft}


\begin{flushleft}
separation of variables, self-similar solutions, boundary layer
\end{flushleft}


\begin{flushleft}
analysis. Results: representation and interpretation, uncertainty
\end{flushleft}


\begin{flushleft}
and error bands. Heat transfer experiments: design, uncertainty
\end{flushleft}


\begin{flushleft}
analysis, selection of geometrical and physical parameters,
\end{flushleft}


\begin{flushleft}
instrumentation, and rig calibration. Temperature, pressure and flow
\end{flushleft}


\begin{flushleft}
rate measurements. Systemic errors in temperature measurement:
\end{flushleft}


\begin{flushleft}
thermocouples and thermowell. Data acquisition systems: basics and
\end{flushleft}


\begin{flushleft}
applications. Data analysis and error estimation. Project-type work
\end{flushleft}


\begin{flushleft}
involving modeling, designing and performing experiments related
\end{flushleft}


\begin{flushleft}
to heat transfer applications.
\end{flushleft}





\begin{flushleft}
MCL442 Thermofluid Analysis of Biosystems
\end{flushleft}


\begin{flushleft}
3 Credits (3-0-0)
\end{flushleft}


\begin{flushleft}
Pre-requisites: APL106 and [(MCL140 \& MCL242) or MCL141]
\end{flushleft}


\begin{flushleft}
and EC 80
\end{flushleft}


\begin{flushleft}
Applications of fluid mechanics, heat transfer, and thermodynamics
\end{flushleft}


\begin{flushleft}
to biological processes,including blood flow in the circulatory
\end{flushleft}


\begin{flushleft}
system, heart function, effects of heating and cooling on cells,
\end{flushleft}


\begin{flushleft}
tissues, and proteins.
\end{flushleft}





\begin{flushleft}
MCL443 Electrochemical Energy Systems
\end{flushleft}


\begin{flushleft}
3 Credits (3-0-0)
\end{flushleft}


\begin{flushleft}
Pre-requisites: MCL140 \& MCL242 or MCL141 and EC 80
\end{flushleft}


\begin{flushleft}
Overlaps with: CLL722 (30\%), CLL720 (10\%), CLL721 (15\%)
\end{flushleft}


\begin{flushleft}
Introduction to electrochemical systems -- electrochemical power
\end{flushleft}


\begin{flushleft}
sources, nomenclature, survey of common types. Thermodynamics
\end{flushleft}


\begin{flushleft}
-- thermodynamic functions, chemical and electrochemical potentials,
\end{flushleft}


\begin{flushleft}
temperature dependence, activity dependence. Reaction kinetics --
\end{flushleft}


\begin{flushleft}
electrical double layer, kinetics, activation energy of reactions, currentvoltage relationship, polarization and losses, charge transfer kinetics,
\end{flushleft}


\begin{flushleft}
performance criteria. Transport processes -- infinitely dilute solutions,
\end{flushleft}


\begin{flushleft}
concentrated solutions, thermal effects, fluid mechanics. Modeling
\end{flushleft}


\begin{flushleft}
of electrochemical systems -- governing equations, assumptions,
\end{flushleft}


\begin{flushleft}
boundary conditions of species and charge. Thermal management.
\end{flushleft}


\begin{flushleft}
Environmental impact.
\end{flushleft}





261





\begin{flushleft}
\newpage
Mechanical Engineering
\end{flushleft}





\begin{flushleft}
MCL701 Advanced Thermodynamics
\end{flushleft}


\begin{flushleft}
3 Credits (3-0-0)
\end{flushleft}





\begin{flushleft}
MCL722 Mechanical Design of Prime Mover Elements
\end{flushleft}


\begin{flushleft}
3 Credits (3-0-0)
\end{flushleft}





\begin{flushleft}
Review of basic fundamentals, closed system and open system
\end{flushleft}


\begin{flushleft}
formulations, laws of thermodynamics, the maximum entropy principle,
\end{flushleft}


\begin{flushleft}
concept of equations of state, ideal gas, van der Waals equations
\end{flushleft}


\begin{flushleft}
and other variants, compressibility, maximum work theorem, exergy,
\end{flushleft}


\begin{flushleft}
energy minimum principle, thermodynamic potentials and relationships
\end{flushleft}


\begin{flushleft}
for compressible, elastic, electric and magnetic systems, stability
\end{flushleft}


\begin{flushleft}
conditions of potentials, multicomponent systems, entropy of mixing,
\end{flushleft}


\begin{flushleft}
chemical potential, mixtures, conditions of equilibrium and stability
\end{flushleft}


\begin{flushleft}
of multicomponent systems, thermodynamics of reactive mixtures.
\end{flushleft}





\begin{flushleft}
Introduction of prime-movers. Introduction of transmission-systems,
\end{flushleft}


\begin{flushleft}
clutches and brakes. Engine Characteristics. Throttle system,
\end{flushleft}


\begin{flushleft}
Turbochargers, History of valve train. Design of valve train. Design of
\end{flushleft}


\begin{flushleft}
variable valve timing system, flywheel. Material selection. Lubrication
\end{flushleft}


\begin{flushleft}
systems to minimize the maintenance requirement. Modelling and
\end{flushleft}


\begin{flushleft}
design of gearbox. Role of control system in advanced (i.e. direct
\end{flushleft}


\begin{flushleft}
injection, active boosting, camless) powertrain system
\end{flushleft}





\begin{flushleft}
MCL702 Advanced Fluid Mechanics
\end{flushleft}


\begin{flushleft}
3 Credits (3-0-0)
\end{flushleft}


\begin{flushleft}
Formulation of Navier-Stokes equations. Exact solutions of the NavierStokes equations for select unsteady/steady flows, potential flows,
\end{flushleft}


\begin{flushleft}
boundary layer theory and its applications, turbulent flows; special
\end{flushleft}


\begin{flushleft}
topics in fluid mechanics such as capillary and electrokinetic flows.
\end{flushleft}





\begin{flushleft}
MCL703 Advanced Heat and Mass Transfer
\end{flushleft}


\begin{flushleft}
3 Credits (3-0-0)
\end{flushleft}


\begin{flushleft}
Derivation of governing equation for three dimensional transient heat
\end{flushleft}


\begin{flushleft}
conduction problems. Two-dimensional steady state heat conduction.
\end{flushleft}


\begin{flushleft}
Transient one-dimensional heat conduction in finite length bodies.
\end{flushleft}


\begin{flushleft}
Diffusive Mass Transfer -- Fick's law and governing equation. Melting
\end{flushleft}


\begin{flushleft}
and solidification.
\end{flushleft}


\begin{flushleft}
Newton's law of cooling-Derivation of energy equation- Self-similar
\end{flushleft}


\begin{flushleft}
solution for laminar boundary flow over a flat plate -- energy integral
\end{flushleft}


\begin{flushleft}
method for laminar boundary layer flow over a flat surface-Laminar
\end{flushleft}


\begin{flushleft}
internal flows-thermally fully developed flows-Graetz problem - Natural
\end{flushleft}


\begin{flushleft}
convection over a vertical flat plate: similarity solutions and energy
\end{flushleft}


\begin{flushleft}
integral method- natural convection in enclosures-mixed convectionTurbulent flow and heat transfer: Reynolds averaged equationsTurbulent boundary layer flows -- The law of wall-integral solutions.
\end{flushleft}


\begin{flushleft}
Convective mass transfer.
\end{flushleft}


\begin{flushleft}
Convection with phase change: Pool boiling regimes- Condensation: dropwise condensation-Laminar film condensation over a vertical surface.
\end{flushleft}


\begin{flushleft}
Radiative heat transfer: Black body radiation-radiative properties of
\end{flushleft}


\begin{flushleft}
non-black bodies-surface radiation heat transfer in enclosures with
\end{flushleft}


\begin{flushleft}
gray diffused walls and non-gray surfaces.
\end{flushleft}





\begin{flushleft}
MCL704 Applied Mathematics for Thermofluids
\end{flushleft}


\begin{flushleft}
3 Credits (3-0-0)
\end{flushleft}


\begin{flushleft}
Initial-boundary value problems, Linear and Non-linear systems;
\end{flushleft}


\begin{flushleft}
Theory of linear homogeneous and nonhomogeneous equations;
\end{flushleft}


\begin{flushleft}
Non-linear systems; Series solutions of linear ordinary differential
\end{flushleft}


\begin{flushleft}
equations; special functions; 1st order PDEs, classification of PDEs: 2nd
\end{flushleft}


\begin{flushleft}
order PDE - Planar, cylindrical and spherical geometries, Homogeneous
\end{flushleft}


\begin{flushleft}
and non homogeneous PDEs, Strum-Liouville theory; Stability and
\end{flushleft}


\begin{flushleft}
instability of regular system.
\end{flushleft}





\begin{flushleft}
MCL705 Experimental Methods
\end{flushleft}


\begin{flushleft}
4 Credits (3-0-2)
\end{flushleft}


\begin{flushleft}
Methodology and planning of experimental work and reporting results.
\end{flushleft}


\begin{flushleft}
Types of errors, uncertainty propagation and statistical basis of
\end{flushleft}


\begin{flushleft}
uncertainty. Statics and data interpretation: population and sample,
\end{flushleft}


\begin{flushleft}
mean and standard deviation, standard deviation of mean, probability
\end{flushleft}


\begin{flushleft}
distributions and sample size selection. Design of experiments.
\end{flushleft}


\begin{flushleft}
Instruments: specifications, characteristics, and sources of error. Data
\end{flushleft}


\begin{flushleft}
acquisition and signal processing: analog to digital conversion, Fourier
\end{flushleft}


\begin{flushleft}
series and transform, sampling, aliasing, and filtering. Cross-correlation
\end{flushleft}


\begin{flushleft}
and autocorrelation. Digital image analysis.
\end{flushleft}





\begin{flushleft}
MCL721 Automotive Prime Movers
\end{flushleft}


\begin{flushleft}
3 Credits (3-0-0)
\end{flushleft}


\begin{flushleft}
Introduction to current technologies, Design and Performance analysis
\end{flushleft}


\begin{flushleft}
of MPFI-SI and CRDI-CI engines. Advanced thermodynamic modeling
\end{flushleft}


\begin{flushleft}
of near DI-SI and HCCI-DI engines. Analysis of five stroke and six
\end{flushleft}


\begin{flushleft}
stroke cycles for SI and CI engines. Design of Electric and Hydraulic
\end{flushleft}


\begin{flushleft}
Prime Movers. Concept, classification and analysis of Hybrid Prime
\end{flushleft}


\begin{flushleft}
mover systems.
\end{flushleft}





\begin{flushleft}
MCL723 Vehicle Dynamics
\end{flushleft}


\begin{flushleft}
3 Credits (2-0-2)
\end{flushleft}


\begin{flushleft}
Basics of Modeling: Euler Angles, Vehicle fixed and Earth fixed coordinate systems. Application of Newton's second law. Role of inertia.
\end{flushleft}


\begin{flushleft}
Vehicle Traction: Engine Performance Curves, Traction curves, Loads on
\end{flushleft}


\begin{flushleft}
grades, Aerodynamic resistance, ideal gear shifting positions, Rolling
\end{flushleft}


\begin{flushleft}
Resistance, Transverse weight shift due to accelerations, maximum
\end{flushleft}


\begin{flushleft}
acceleration, Traction limits.
\end{flushleft}


\begin{flushleft}
Braking Dynamics: Tire road friction, braking efficiency, wheel lock up,
\end{flushleft}


\begin{flushleft}
wheel skidding, performance with anti-lock brake systems.
\end{flushleft}


\begin{flushleft}
Vehicle Vibrations: Measures of Ride quality, predictions of vibrations,
\end{flushleft}


\begin{flushleft}
suspension stiffness and damping, road roughness models, response to
\end{flushleft}


\begin{flushleft}
speed breakers, Heave, pitch and roll phenomenon of dynamic Motion,
\end{flushleft}


\begin{flushleft}
quarter car, half car and full car model, Seat suspension, relation to
\end{flushleft}


\begin{flushleft}
human body vibrations.
\end{flushleft}


\begin{flushleft}
Steering Dynamics: Steering system moments and forces, FWD,RWD
\end{flushleft}


\begin{flushleft}
and 4WD steering systems, understeer and over steer, roll steer, Tire
\end{flushleft}


\begin{flushleft}
cornering forces, slip angle, critical speed, roll stability, steering of
\end{flushleft}


\begin{flushleft}
heavy vehicles, steering dynamics of tractor trailer systems.
\end{flushleft}


\begin{flushleft}
Role of Suspensions in Vehicle Dynamics: Independent Suspensions,
\end{flushleft}


\begin{flushleft}
and Solid axles, roll center analysis important class of suspensions,
\end{flushleft}


\begin{flushleft}
roll over stability of vehicles on suspensions.
\end{flushleft}





\begin{flushleft}
MCL724 Biomechanics in Trauma and Automotive Design
\end{flushleft}


\begin{flushleft}
3 Credits (3-0-0)
\end{flushleft}


\begin{flushleft}
Introduction to Biomechanics of human body esp. its Musculo -skeletal
\end{flushleft}


\begin{flushleft}
systems. Biomechanics (including injury indices) of the Head, Neck,
\end{flushleft}


\begin{flushleft}
Thorax and Lower Extremity, Crashworthiness standards-crash tests
\end{flushleft}


\begin{flushleft}
and crash simulations, Human Body Models for analysis-Lumped
\end{flushleft}


\begin{flushleft}
mass, multi body, FE and integrated models, modeling of contact,
\end{flushleft}


\begin{flushleft}
modeling of muscles, belts, modeling of airbags and other safety
\end{flushleft}


\begin{flushleft}
devices. Human Anthropometry, Crash Dummies and Dummy
\end{flushleft}


\begin{flushleft}
models, Goals of crashworthiness in automotive design. Design of
\end{flushleft}


\begin{flushleft}
the automotive chassis for crashworthiness, Different aspects of
\end{flushleft}


\begin{flushleft}
crashworthiness-designing for frontal, side impact, rear impact,
\end{flushleft}


\begin{flushleft}
rollover and pedestrian/2 wheeler impacts, The future-current trends
\end{flushleft}


\begin{flushleft}
in automotive crashworthiness.
\end{flushleft}





\begin{flushleft}
MCL725 Design Electronic Assist Systems in Automobiles
\end{flushleft}


\begin{flushleft}
3 Credits (3-0-0)
\end{flushleft}


\begin{flushleft}
Introduction to Automotive Electrical and Electronic Systems,
\end{flushleft}


\begin{flushleft}
Electric and Electronic components, Instrumentation: Gauges and
\end{flushleft}


\begin{flushleft}
sensors, Introduction to body computer: Microprocessor, Antilock
\end{flushleft}


\begin{flushleft}
braking system: the mechanism and control implementation, Power
\end{flushleft}


\begin{flushleft}
steering, Suspension, Air conditioning: controls and implementation,
\end{flushleft}


\begin{flushleft}
Accessories: Intelligent windshield wipers, Power door locks, Power
\end{flushleft}


\begin{flushleft}
windows, Power seats, Vehicle audio entertainment system, Navigation
\end{flushleft}


\begin{flushleft}
system, airbags and belt tensioners.
\end{flushleft}





\begin{flushleft}
MCL726 Design of Steering Systems
\end{flushleft}


\begin{flushleft}
3 Credits (3-0-0)
\end{flushleft}


\begin{flushleft}
Introduction of steering requirements and system; steady-state
\end{flushleft}


\begin{flushleft}
cornering -- slip angle effects, steady-state turns, calculating steadystate steering characteristics, lateral weight transfer effect, traction
\end{flushleft}


\begin{flushleft}
effect, neutral steer point and static margin, swing axle; steady-state
\end{flushleft}


\begin{flushleft}
cornering -- steer effects, roll effects, wheel control, understeer and
\end{flushleft}


\begin{flushleft}
oversteer effects, torque steer, lateral deflection steer, straight running,
\end{flushleft}


\begin{flushleft}
suspension geometry effects, effect of road surface, wind handling;
\end{flushleft}


\begin{flushleft}
transient cornering, steering when moving forward, steering when
\end{flushleft}


\begin{flushleft}
moving reverse, boat steering and truck in reverse; examples of
\end{flushleft}


\begin{flushleft}
steering system; conclusion.
\end{flushleft}





262





\begin{flushleft}
\newpage
Mechanical Engineering
\end{flushleft}





\begin{flushleft}
MCL728 Nanotribology
\end{flushleft}


\begin{flushleft}
3 Credits (3-0-0)
\end{flushleft}


\begin{flushleft}
Topics will include surface force and adhesion models for soft and hard
\end{flushleft}


\begin{flushleft}
solids; friction laws for nano, micro and macro contacts; atomic-scale
\end{flushleft}


\begin{flushleft}
stick-slip phenomenon; the roles of surface energy and surface forces
\end{flushleft}


\begin{flushleft}
on friction and wear; molecular structure effects on friction; nanolubrication and design of nano-lubricants (self-assembled monolayers,
\end{flushleft}


\begin{flushleft}
ultra-thin films of functionalized polymers); nano-texturing and surface
\end{flushleft}


\begin{flushleft}
roughness effects; surface chemisorptions and physisorption effects;
\end{flushleft}


\begin{flushleft}
friction-induced effects such as wear, molecular alignments, tribocharging, surface oxidation, third-body generation etc. The above
\end{flushleft}


\begin{flushleft}
principles would be applied to modern technologies such as data
\end{flushleft}


\begin{flushleft}
storage (head-disk interface) tribology, various micromachines such
\end{flushleft}


\begin{flushleft}
as micro-electromechanical systems (MEMS) tribology and nature's
\end{flushleft}


\begin{flushleft}
solutions to tribological problems through a few case studies.
\end{flushleft}





\begin{flushleft}
MCL729 Nanomechanics
\end{flushleft}


\begin{flushleft}
3 Credits (2-0-2)
\end{flushleft}


\begin{flushleft}
Introduction to nanomechanics, need for studying nanomechanics,
\end{flushleft}


\begin{flushleft}
its scope and limitations; Dynamics of 2-atom, 3-atom molecules,
\end{flushleft}


\begin{flushleft}
and an N-atom chain; Crystal Lattice and Reciprocal Lattice; Dynamic
\end{flushleft}


\begin{flushleft}
Interaction Potentials and Periodic Boundary Conditions in molecular
\end{flushleft}


\begin{flushleft}
dynamics simulations; Role of different ensembles; Evaluation of
\end{flushleft}


\begin{flushleft}
atomic stresses and strains, Evaluation of Specific Heat, Dissipation
\end{flushleft}


\begin{flushleft}
of Energy in nano-mechanical Systems; Solutions for classical Nano
\end{flushleft}


\begin{flushleft}
scale structural components such as Carbon Nanotubes, Nano rods,
\end{flushleft}


\begin{flushleft}
Nanowires and Polymers; Correlations between Nano-mechanics
\end{flushleft}


\begin{flushleft}
and classical continuum theory of solids; Size effect; Introduction to
\end{flushleft}


\begin{flushleft}
multiscale modeling.
\end{flushleft}





\begin{flushleft}
MCL730 Designing with advanced materials
\end{flushleft}


\begin{flushleft}
4 Credits (3-0-2)
\end{flushleft}


\begin{flushleft}
Introduction to polymers, composites and smart materials. Polymer
\end{flushleft}


\begin{flushleft}
microstructure and mechanical properties. Thermosets and
\end{flushleft}


\begin{flushleft}
thermoplastics. Viscoelastic creep and relaxation behavior, mechanical
\end{flushleft}


\begin{flushleft}
models, and polymer failure. Design considerations and practices
\end{flushleft}


\begin{flushleft}
for polymeric components with case studies. Composite materials
\end{flushleft}


\begin{flushleft}
and their applications. Micro and macro mechanics of lamina, failure
\end{flushleft}


\begin{flushleft}
criteria of lamina, classical laminate theory, strength of laminates.
\end{flushleft}


\begin{flushleft}
Design considerations and practices for composite structures with case
\end{flushleft}


\begin{flushleft}
studies. Structure, applications and design considerations of smart
\end{flushleft}


\begin{flushleft}
materials such as shape memory alloys and piezoelectric materials.
\end{flushleft}





\begin{flushleft}
power and intensity and its measurement, Sound Intensity, Various
\end{flushleft}


\begin{flushleft}
Sound Fields, Concept of Monopoles, Dipoles and Quadrupoles, Sound
\end{flushleft}


\begin{flushleft}
Power measurement, Transmission loss, Design of partitions, barriers,
\end{flushleft}


\begin{flushleft}
acoustic enclosures, Design of Mufflers; Sound Absorbing Materials,
\end{flushleft}


\begin{flushleft}
Noise Control approaches, case studies.
\end{flushleft}





\begin{flushleft}
MCL735 CAD and Finite Element Analysis
\end{flushleft}


\begin{flushleft}
4 Credits (3-0-2)
\end{flushleft}


\begin{flushleft}
Introduction and overview. Need and Scope of Computer Aided
\end{flushleft}


\begin{flushleft}
Machine Design. Role of Geometric Modelling, FE and Optimization;
\end{flushleft}


\begin{flushleft}
Introduction to the Finite Element Method, principles of minimization
\end{flushleft}


\begin{flushleft}
of potential energy: Rayleig Ritz and Galerkin Methods, 1D elements
\end{flushleft}


\begin{flushleft}
and their analysis, analysis of bars, beams and trusses, axissymmetric solids, 2D/3D solids, solution methods in FE, error analysis,
\end{flushleft}


\begin{flushleft}
Introduction to Dynamic and Non linear analysis; Limitations of FEM;
\end{flushleft}


\begin{flushleft}
2D and 3D Geometric transformations, Orthographic and Perspective
\end{flushleft}


\begin{flushleft}
Projections. Euler angles, Windowing, view-porting and viewing
\end{flushleft}


\begin{flushleft}
transformations, Modeling of cubics, beziers, b-splines, NURBS and
\end{flushleft}


\begin{flushleft}
advanced curves; Modeling of surfaces: ruled surfaces, surfaces of
\end{flushleft}


\begin{flushleft}
revolution, Bicubic, Bezier, B-splines; Feature Based Modeling and
\end{flushleft}


\begin{flushleft}
Feature Recognition.
\end{flushleft}





\begin{flushleft}
MCL736 Automotive Design
\end{flushleft}


\begin{flushleft}
4 Credits (3-0-2)
\end{flushleft}


\begin{flushleft}
History of automotive design, Design cycle for an Automobile,
\end{flushleft}


\begin{flushleft}
Styling, Loads on the chassis, Chassis and structural Design for static
\end{flushleft}


\begin{flushleft}
loads, Dynamic and impact loads, Energy absorption in the vehicle,
\end{flushleft}


\begin{flushleft}
Computational tools for structural design, vehicle occupant system
\end{flushleft}


\begin{flushleft}
analysis, biomechanics of the human body and its implications for
\end{flushleft}


\begin{flushleft}
structural design, Designing for NVH, Designing the suspension
\end{flushleft}


\begin{flushleft}
system, Designing the brake system, Design of engine characteristics,
\end{flushleft}


\begin{flushleft}
Design requirements of the transmission and the driveline.
\end{flushleft}





\begin{flushleft}
MCL738 Dynamics of Multibody Systems
\end{flushleft}


\begin{flushleft}
3 Credits (2-0-2)
\end{flushleft}


\begin{flushleft}
Overview of kinematic descriptions of serial, tree, and closed-loop
\end{flushleft}


\begin{flushleft}
chains, Degrees of freedom, and Kinematic constraints of rigid and
\end{flushleft}


\begin{flushleft}
flexible systems; Basics of Euler-Lagrange and other classical dynamic
\end{flushleft}


\begin{flushleft}
formulations, and those with orthogonal complements; Dynamic
\end{flushleft}


\begin{flushleft}
algorithms (inverse and forward dynamics); Efficiency and numerical
\end{flushleft}


\begin{flushleft}
stability aspects of the algorithms; Introduction to commercial software
\end{flushleft}


\begin{flushleft}
like RecurDyn.
\end{flushleft}





\begin{flushleft}
MCL740 Advanced Lubrication
\end{flushleft}


\begin{flushleft}
4 Credits (3-0-2)
\end{flushleft}





\begin{flushleft}
MCL731 Analytical Dynamics
\end{flushleft}


\begin{flushleft}
3 Credits (3-0-0)
\end{flushleft}


\begin{flushleft}
Review of Newtonian dynamics; Degrees of freedom; Generalized
\end{flushleft}


\begin{flushleft}
coordinates and constraints; Holonomic and nonholonomic systems;
\end{flushleft}


\begin{flushleft}
Principle of Virtual work; D'Alembert's principle; Euler-Lagrange
\end{flushleft}


\begin{flushleft}
equations of motion; Hamilton's principle; Rotating coordinate
\end{flushleft}


\begin{flushleft}
systems; Euler angles; Coordinate transformation; Kinematics of a
\end{flushleft}


\begin{flushleft}
rigid body; Euler's equations of rotation; Computer-oriented dynamic
\end{flushleft}


\begin{flushleft}
modeling; Orthogonal-complement based formulation of dynamic
\end{flushleft}


\begin{flushleft}
equations; Geometric theory; Stability; Lyapunov's direct method;
\end{flushleft}


\begin{flushleft}
Introduction to flexible-body dynamics.
\end{flushleft}





\begin{flushleft}
MCL733 Vibration and Noise Engineering
\end{flushleft}


\begin{flushleft}
3 Credits (3-0-0)
\end{flushleft}


\begin{flushleft}
Elements of vibration analysis: modeling practical systems through
\end{flushleft}


\begin{flushleft}
discrete system/lumped parameters approach and its subsequent
\end{flushleft}


\begin{flushleft}
analysis for different types of excitations encountered in practice.
\end{flushleft}


\begin{flushleft}
Alternate mathematical models of damping, two and multi-DOF
\end{flushleft}


\begin{flushleft}
systems, tuned mass dampers. Introduction to vibration isolation:
\end{flushleft}


\begin{flushleft}
displacement/force isolation, approaches for MDOF system analysis
\end{flushleft}


\begin{flushleft}
with emphasis on modal approach. Numerical and Continuum Analysis:
\end{flushleft}


\begin{flushleft}
Finite Element Method for dynamic analysis. Distributed parameter
\end{flushleft}


\begin{flushleft}
models of rods, bars, beams, membranes and plates. Introduction
\end{flushleft}


\begin{flushleft}
to Modal testing, Vibration Testing. Spatial, Modal and Response
\end{flushleft}


\begin{flushleft}
models of vibrating systems. Non-linear and Random vibrations:
\end{flushleft}


\begin{flushleft}
Introduction to non-linear vibrations, response to random excitation.
\end{flushleft}


\begin{flushleft}
Engineering acoustics: Wave approach to sound, wave equation,
\end{flushleft}


\begin{flushleft}
Noise measurement and instrumentation standards. Sound pressure,
\end{flushleft}





\begin{flushleft}
Introduction: surface topography and its 2-D and 3-D characterizations,
\end{flushleft}


\begin{flushleft}
interactions of surfaces, friction, wear, lubrication; Regimes of
\end{flushleft}


\begin{flushleft}
lubrication: hydrodynamic, elastohydrodynamic, mixed, boundary,
\end{flushleft}


\begin{flushleft}
Stribeck curve; Lubricant: mineral oil, synthetic oil, grease, emulsions,
\end{flushleft}


\begin{flushleft}
gases, properties of lubricants, various rheology models; Derivation
\end{flushleft}


\begin{flushleft}
of governing equations: conservations of mass, momentum, energy,
\end{flushleft}


\begin{flushleft}
establishing 3-D Reynolds equation and energy equation for lubrication
\end{flushleft}


\begin{flushleft}
simulations; Cavitation and turbulence models; Contact mechanics:
\end{flushleft}


\begin{flushleft}
2-D and 3-D contacts, surface and subsurface stresses, asperity
\end{flushleft}


\begin{flushleft}
contact models, elastic deformation at contacts; Applications of
\end{flushleft}


\begin{flushleft}
governing equations in design and performance analysis of journal
\end{flushleft}


\begin{flushleft}
bearings, thrust bearings, squeeze film bearings, hydrostatic bearings,
\end{flushleft}


\begin{flushleft}
rolling bearings, gear sets, seals, and piston rings; Lubrication in metal
\end{flushleft}


\begin{flushleft}
forming; Dynamic coefficients: stiffness and damping calculations,
\end{flushleft}


\begin{flushleft}
rotor vibrations, oil-whirl instability, and friction instabilities; Failure
\end{flushleft}


\begin{flushleft}
analysis of lubricated contacts/interfaces, Immerging technology:
\end{flushleft}


\begin{flushleft}
surface textures and bionic surfaces.
\end{flushleft}





\begin{flushleft}
MCL741 Control
\end{flushleft}


\begin{flushleft}
4 Credits (3-0-2)
\end{flushleft}


\begin{flushleft}
An introduction to control systems; transfer function representation of
\end{flushleft}


\begin{flushleft}
mechanical and mechatronic systems; stability analysis, gain setting for
\end{flushleft}


\begin{flushleft}
stability; transient and steady-state response analyses; control system
\end{flushleft}


\begin{flushleft}
analysis and design by the Root-Locus method and the FrequencyResponse method; PID controllers design and realization; State-Space
\end{flushleft}


\begin{flushleft}
representation, controllability and observability; control system design
\end{flushleft}


\begin{flushleft}
in State Space; digital implementation of classical controllers.
\end{flushleft}





263





\begin{flushleft}
\newpage
Mechanical Engineering
\end{flushleft}





\begin{flushleft}
MCL742 Design \& Optimization
\end{flushleft}


\begin{flushleft}
4 Credits (3-0-2)
\end{flushleft}


\begin{flushleft}
Review of machine element design based on strength and distortion
\end{flushleft}


\begin{flushleft}
criterion; review of choice of materials and their treatment: Designing
\end{flushleft}


\begin{flushleft}
for fatigue, creep; Design criterion for fracture; Application of advanced
\end{flushleft}


\begin{flushleft}
design criterion to machine elements (like shafts, spur / bevel / worm
\end{flushleft}


\begin{flushleft}
gears); Design of structures, machines and equipment; Classical
\end{flushleft}


\begin{flushleft}
methods of unconstrained optimization (single variable and multi
\end{flushleft}


\begin{flushleft}
variable), classical methods of constrained optimization, Numerical
\end{flushleft}


\begin{flushleft}
optimisation techniques including i. genetic algorithms, (binary and
\end{flushleft}


\begin{flushleft}
real coded) ii. Simulated annealing. Case studies of Optimum Design
\end{flushleft}


\begin{flushleft}
(Gear Box, Power Transmission, shape and topology using FE).
\end{flushleft}





\begin{flushleft}
MCL743 Plant Equipment Design
\end{flushleft}


\begin{flushleft}
3 Credits (3-0-0)
\end{flushleft}


\begin{flushleft}
Introduction to various kinds of plant equipment, and technological
\end{flushleft}


\begin{flushleft}
considerations in their design. Special considerations for typical
\end{flushleft}


\begin{flushleft}
industries such as petrochemicals, food-processing, power plants, and
\end{flushleft}


\begin{flushleft}
for mass production. Pressure vessel types and shapes. Design analysis
\end{flushleft}


\begin{flushleft}
of thin walled vessel for low pressure applications. Design analysis
\end{flushleft}


\begin{flushleft}
of thick walled vessels for high pressures and special applications.
\end{flushleft}


\begin{flushleft}
Vessel opening, closures and seals. Manufacturing considerations
\end{flushleft}


\begin{flushleft}
for pressure vessels. Configuration of various kinds of pumps used
\end{flushleft}


\begin{flushleft}
in process plants. Pump design considerations. Centrifugal pump
\end{flushleft}


\begin{flushleft}
selection. Design of pipes and piping joints, Layout of piping systems.
\end{flushleft}


\begin{flushleft}
Material Handling Equipment, Types and use. Design considerations
\end{flushleft}


\begin{flushleft}
for hoisting equipment, Surface and Overhead equipment Stackers
\end{flushleft}


\begin{flushleft}
and elevators, and conveyors. Design consideration in rotating
\end{flushleft}


\begin{flushleft}
machinery, bearing characteristics and selection, placement of
\end{flushleft}


\begin{flushleft}
critical speeds, effect of seals and foundation effects. Materials and
\end{flushleft}


\begin{flushleft}
manufacturing considerations in various plant equipment systems,
\end{flushleft}


\begin{flushleft}
and use of applicable standards, and available software packages.
\end{flushleft}





\begin{flushleft}
MCL744 Design for Manufacture and Assembly
\end{flushleft}


\begin{flushleft}
3 Credits (2-0-2)
\end{flushleft}


\begin{flushleft}
Product design for life-cycle, concurrent engineering, dfx, design
\end{flushleft}


\begin{flushleft}
for manufacture, rule-based and plan based DFM, automated
\end{flushleft}


\begin{flushleft}
manufacturability assessment, Automated manufacturability
\end{flushleft}


\begin{flushleft}
assessment, Commonly used dfx tools including, QFD, POKA YOKE,
\end{flushleft}


\begin{flushleft}
FMEA, Design for manual assembly and automated assembly, design
\end{flushleft}


\begin{flushleft}
for environment, Industrial and real life case studies of dfx.
\end{flushleft}





\begin{flushleft}
MCL745 Robotics
\end{flushleft}


\begin{flushleft}
4 Credits (3-0-2)
\end{flushleft}


\begin{flushleft}
Type and components of robots; Classification of closed- and openloop kinematic systems; Definition of mechanisms and manipulators;
\end{flushleft}


\begin{flushleft}
Kinematic constraints; Degrees of freedom and mobility; Rotation
\end{flushleft}


\begin{flushleft}
representation; Coordinate transformation; DH parameters; Matrix
\end{flushleft}


\begin{flushleft}
methods for forward and inverse kinematics analyses; Jacobian and
\end{flushleft}


\begin{flushleft}
singularity; Dynamic modeling; Euler-Lagrange and Newton-Euler
\end{flushleft}


\begin{flushleft}
equations of motion for serial type manipulators; DeNOC-based
\end{flushleft}


\begin{flushleft}
dynamic formulation; Inverse and forward dynamics algorithms;
\end{flushleft}


\begin{flushleft}
Parallel robots; Inverse and forward kinematics of parallel robots; Gain
\end{flushleft}


\begin{flushleft}
singularity of parallel robots; Introduction to control of robotic systems.
\end{flushleft}





\begin{flushleft}
MCL746 Design for Noise Vibration and Harshness
\end{flushleft}


\begin{flushleft}
4 Credits (3-0-2)
\end{flushleft}


\begin{flushleft}
Fundamentals of Vibrations and their manifestations in real life
\end{flushleft}


\begin{flushleft}
systems. Review of Design of a Vibration Absorber. Vibration
\end{flushleft}


\begin{flushleft}
Reduction Measures, Unconstrained and constrained layer damping
\end{flushleft}


\begin{flushleft}
treatment, add on dampers, and stiffeners. Changing the dynamic
\end{flushleft}


\begin{flushleft}
characteristics of a structure, Structural dynamics modification.
\end{flushleft}


\begin{flushleft}
Predicting the modification (dynamic design) Design of Isolators
\end{flushleft}


\begin{flushleft}
in machine foundations. Role of materials damping. Balancing
\end{flushleft}


\begin{flushleft}
of rotating machinery. Rigid and flexible rotor balancing. Active
\end{flushleft}


\begin{flushleft}
Vibrations control. Introduction of wave analysis of structures and
\end{flushleft}


\begin{flushleft}
spaces. , Characteristics of Duct and Cabin Noise. Stationary modes.
\end{flushleft}


\begin{flushleft}
Random noise. Measures of a sound acoustic design, importance of
\end{flushleft}


\begin{flushleft}
reverberations time. Various types of acoustic testing chambers. Noise
\end{flushleft}


\begin{flushleft}
measurement and control instruments. Sound Intensity Mapping Noise
\end{flushleft}


\begin{flushleft}
isolation design. Noise absorber design. Design of silencers, mufflers.
\end{flushleft}


\begin{flushleft}
Acoustic Design of Buildings.
\end{flushleft}





\begin{flushleft}
MCL747 Design of Precision Machines
\end{flushleft}


\begin{flushleft}
3 Credits (2-0-2)
\end{flushleft}


\begin{flushleft}
Pre-requisites: For UG:AML140, MCL111, MCL211
\end{flushleft}


\begin{flushleft}
Fundamental concepts in precision design; design for stiffness;
\end{flushleft}


\begin{flushleft}
controlling Degrees-of-Freedom, exact-constrained design; design
\end{flushleft}


\begin{flushleft}
of elastic mechanisms/flexures/compliant mechanisms; friction,
\end{flushleft}


\begin{flushleft}
hysteresis and micro-slip; actuators and sensors for precision motion;
\end{flushleft}


\begin{flushleft}
materials selection in precision machine design; slideways for long
\end{flushleft}


\begin{flushleft}
range precision motion; and dynamics of precision mechanisms.
\end{flushleft}





\begin{flushleft}
MCL748 Tribological Systems Design
\end{flushleft}


\begin{flushleft}
4 Credits (3-0-2)
\end{flushleft}


\begin{flushleft}
Lubrication, Friction and Wear aspects in Design; Tribological Surfaces
\end{flushleft}


\begin{flushleft}
-- Measures of Roughness and associated mechanisms of Lubrication,
\end{flushleft}


\begin{flushleft}
Regimes of Lubrication; Boundary lubrication and lubricants. Friction
\end{flushleft}


\begin{flushleft}
and wear at different length scales. Viscosity - its representation and
\end{flushleft}


\begin{flushleft}
measurement, apparent viscosity. Selection of Bearings - Rubbing,
\end{flushleft}


\begin{flushleft}
Fluid Film, Rolling Element. Lubricants - Types and Selection, Bearing
\end{flushleft}


\begin{flushleft}
Design - Rubbing, Fluid Film Journal and Thrust, Dynamically Loaded,
\end{flushleft}


\begin{flushleft}
Rolling Element, Design of lubrication Systems. Introduction to
\end{flushleft}


\begin{flushleft}
maintenance of Bearings, Seals, Linear Bearing Design, Slideways.
\end{flushleft}


\begin{flushleft}
Material considerations for selected tribological applications.
\end{flushleft}





\begin{flushleft}
MCL749 Mechatronic Product Design
\end{flushleft}


\begin{flushleft}
4 Credits (3-0-2)
\end{flushleft}


\begin{flushleft}
Pre-requisites: For UG : ELL100, MCL338
\end{flushleft}


\begin{flushleft}
Overlaps with: EEL482
\end{flushleft}


\begin{flushleft}
Introduction tokey elements of Mechatronic products - Physical Systems
\end{flushleft}


\begin{flushleft}
Modeling, Sensors and Actuators, Signals and Systems, Computers
\end{flushleft}


\begin{flushleft}
and Logic Systems, Software and Data Acquisition; Mechatronic
\end{flushleft}


\begin{flushleft}
Design Approach, System Interfacing, Instrumentation and Control
\end{flushleft}


\begin{flushleft}
Systems; Microprocessor-Based Controllers and Microelectronics;
\end{flushleft}


\begin{flushleft}
Product functional block diagram, schematic and PCB Design, Product
\end{flushleft}


\begin{flushleft}
enclosure design, Microcontroller interfacing and programming,
\end{flushleft}


\begin{flushleft}
Interfacing with sensors and actuators, driver circuits, motion control,
\end{flushleft}


\begin{flushleft}
Stepper and servo motion control. Software and hardware tools to build
\end{flushleft}


\begin{flushleft}
mechatronic systems. Design and selection of mechatronic elements
\end{flushleft}


\begin{flushleft}
namely sensors like encoders and resolvers; stepper and servomotors,
\end{flushleft}


\begin{flushleft}
ballscrews, solenoid like actuators, and controllers with applications
\end{flushleft}


\begin{flushleft}
to CNC systems, robotics, consumer electronic products etc. Design
\end{flushleft}


\begin{flushleft}
of a mechatronic product using available software CAD packages.
\end{flushleft}


\begin{flushleft}
Laboratory work will be hands-on Microcontroller \& Microprocessor
\end{flushleft}


\begin{flushleft}
interfacing and programming, Motion controller, motors, sensors,
\end{flushleft}


\begin{flushleft}
and actuators.
\end{flushleft}





\begin{flushleft}
MCL750 Product design and Manufacturing
\end{flushleft}


\begin{flushleft}
3 Credits (1-0-4)
\end{flushleft}


\begin{flushleft}
Product design for a given need or identified need, Development
\end{flushleft}


\begin{flushleft}
and evaluation of multiple solutions and concepts, Manufacturability
\end{flushleft}


\begin{flushleft}
assessments of given design, Product Costing and Bill of
\end{flushleft}


\begin{flushleft}
Materials, Process planning for components and assembly, Product
\end{flushleft}


\begin{flushleft}
manufacturing and Testing.
\end{flushleft}





\begin{flushleft}
MCL751 Industrial Engineering Systems
\end{flushleft}


\begin{flushleft}
3 Credits (1-0-4)
\end{flushleft}


\begin{flushleft}
Overview of IE methods and tools such as decision making under
\end{flushleft}


\begin{flushleft}
uncertainty (Pay-off tables, decision trees, utility theory etc.),
\end{flushleft}


\begin{flushleft}
Probability based methods for outcome prediction (Logistic regression,
\end{flushleft}


\begin{flushleft}
Bayesian belief networks, Monte Carlo simulation etc.), Multicriteria
\end{flushleft}


\begin{flushleft}
decision making (AHP, ANP, Graph theory etc.), System Simulation
\end{flushleft}


\begin{flushleft}
(through games like the Beer game for supply chain), Queuing
\end{flushleft}


\begin{flushleft}
theory games, Economic analysis (NPV, IRR etc. for deterministic and
\end{flushleft}


\begin{flushleft}
stochastic scenarios), Algorithms (branch and bound, Metaheuristics
\end{flushleft}


\begin{flushleft}
etc.), Formulation of bigger optimization problems and solving using
\end{flushleft}


\begin{flushleft}
available solvers (eg. CPLEX). Shop-floor scheduling.
\end{flushleft}





\begin{flushleft}
MCL753 Manufacturing Informatics
\end{flushleft}


\begin{flushleft}
4 Credits (3-0-2)
\end{flushleft}


\begin{flushleft}
Pre-requisites: MCL361
\end{flushleft}


\begin{flushleft}
Introduction to manufacturing analytics (manufacturing analytics
\end{flushleft}


\begin{flushleft}
concepts, contemporary issues in high-value manufacturing, and
\end{flushleft}





264





\begin{flushleft}
\newpage
Mechanical Engineering
\end{flushleft}





\begin{flushleft}
opportunities provided by analytics and big data technologies), data
\end{flushleft}


\begin{flushleft}
types and applications (point of sale data, service touch point data,
\end{flushleft}


\begin{flushleft}
service centre data, warranty data, machine condition data, machine
\end{flushleft}


\begin{flushleft}
failure history, machine utilisation data, work in process data and
\end{flushleft}


\begin{flushleft}
online quality control data), optimisation of manufacturing processes
\end{flushleft}


\begin{flushleft}
(optimisation concepts, evolutionary computing, multi-objective
\end{flushleft}


\begin{flushleft}
optimisation, and applications of optimisation for sequential and
\end{flushleft}


\begin{flushleft}
assembly processes), and latest advancements in manufacturing
\end{flushleft}


\begin{flushleft}
analytics (virtual reality, augmented reality, and motion capture gaming
\end{flushleft}


\begin{flushleft}
technologies for manufacturing).
\end{flushleft}





\begin{flushleft}
MCL754 Operations Planning and Control
\end{flushleft}


\begin{flushleft}
3 Credits (3-0-0)
\end{flushleft}


\begin{flushleft}
Evolution of Scientific Management and Buzzwords, Inventory
\end{flushleft}


\begin{flushleft}
Management and Control, MRP and ERP, JIT, Modeling of Processes
\end{flushleft}


\begin{flushleft}
and Systems, Measuring and Improving Performance, Scheduling,
\end{flushleft}


\begin{flushleft}
Aggregate Production Planning, Facility Location.
\end{flushleft}





\begin{flushleft}
MCL755 Service System Design
\end{flushleft}


\begin{flushleft}
3 Credits (2-0-2)
\end{flushleft}


\begin{flushleft}
Pre-requisites: MTL108
\end{flushleft}


\begin{flushleft}
Need for servitization, Service system types, Key dimensions of service
\end{flushleft}


\begin{flushleft}
systems, Frameworks for service system design, tools for service
\end{flushleft}


\begin{flushleft}
system design, Value co-creation, Service quality models, Economics
\end{flushleft}


\begin{flushleft}
of service systems, Service contract design, CMMI-SVC model for
\end{flushleft}


\begin{flushleft}
service systems, Case studies on service system design.
\end{flushleft}





\begin{flushleft}
MCL756 Supply Chain Management
\end{flushleft}


\begin{flushleft}
3 Credits (3-0-0)
\end{flushleft}


\begin{flushleft}
Pre-requisites: MCL361
\end{flushleft}





\begin{flushleft}
system, the project as an agent of change. Project Identification
\end{flushleft}


\begin{flushleft}
considering objectives and SWOT analysis, Screening of project ideas,
\end{flushleft}


\begin{flushleft}
Technical, Market, Financial, Socio-economic and Ecological Appraisal
\end{flushleft}


\begin{flushleft}
of a project. Work break down structure and network development.
\end{flushleft}


\begin{flushleft}
Basic Scheduling, Critical Path and four kinds of floats, Scheduling
\end{flushleft}


\begin{flushleft}
under probabilistic durations, Time Cost tradeoffs, Project Monitoring
\end{flushleft}


\begin{flushleft}
with PERT/Cost, Organizational aspects, Computer packages and
\end{flushleft}


\begin{flushleft}
project completion.
\end{flushleft}





\begin{flushleft}
MCL761 Probability and Statistics
\end{flushleft}


\begin{flushleft}
3 Credits (3-0-0)
\end{flushleft}


\begin{flushleft}
Probability Laws, Random Variables, Conditional Probability and
\end{flushleft}


\begin{flushleft}
Bayes Theorem, Important Random Variables and their properties,
\end{flushleft}


\begin{flushleft}
Joint Probability Distributions, Law of Total Probability, Law of Large
\end{flushleft}


\begin{flushleft}
Numbers, Central Limit Theorem, Estimation Theory, Parameter
\end{flushleft}


\begin{flushleft}
Estimation, Hypothesis Testing using Parametric and Non-Parametric
\end{flushleft}


\begin{flushleft}
Methods, Goodness of fit tests, ANOVA, Linear Regression (Simple,
\end{flushleft}


\begin{flushleft}
Generalized) and Logistics Regression.
\end{flushleft}





\begin{flushleft}
MCL765 Operations Research
\end{flushleft}


\begin{flushleft}
3 Credits (3-0-0)
\end{flushleft}


\begin{flushleft}
The art and science of modeling, Linear Programming, Solution
\end{flushleft}


\begin{flushleft}
methods including Simplex, Sensitivity Analysis, Shadow Pricing and
\end{flushleft}


\begin{flushleft}
Duality Theory, Integer Programming and Solution methods, Dynamic
\end{flushleft}


\begin{flushleft}
Programming with applications, Large Canonical problems such as
\end{flushleft}


\begin{flushleft}
Transportation Problem, Traveling Salesman Problem, Network Flow
\end{flushleft}


\begin{flushleft}
Problem, Case Studies.
\end{flushleft}





\begin{flushleft}
MCL769 Metal Forming Analysis
\end{flushleft}


\begin{flushleft}
4 Credits (3-0-2)
\end{flushleft}





\begin{flushleft}
Supply Chain Orientation and Management, Various flows in a typical
\end{flushleft}


\begin{flushleft}
supply chain, Supply chain strategy -- its context, components and
\end{flushleft}


\begin{flushleft}
structure, Location Decisions, Inventory Decisions, Information
\end{flushleft}


\begin{flushleft}
Decisions -- Bull whip effect and its ramifications, remedies,
\end{flushleft}


\begin{flushleft}
Transportation Decisions - including planning techniques, Supply
\end{flushleft}


\begin{flushleft}
chain modeling and analysis, Performance measurement; Various
\end{flushleft}


\begin{flushleft}
frameworks including Balanced Score Card, SCOR etc., Customer
\end{flushleft}


\begin{flushleft}
Service level selection and supply chain vulnerabilities, Reverse
\end{flushleft}


\begin{flushleft}
Logistics and decision making involved, Supply chain integration and
\end{flushleft}


\begin{flushleft}
web enabled supply management.
\end{flushleft}





\begin{flushleft}
MCL757 Logistics
\end{flushleft}


\begin{flushleft}
3 Credits (3-0-0)
\end{flushleft}


\begin{flushleft}
Logistics Management is the part of supply chain management that
\end{flushleft}


\begin{flushleft}
plans, implements, and controls the efficient, effective forward and
\end{flushleft}


\begin{flushleft}
reverse flow and storage of goods, services, and related information
\end{flushleft}


\begin{flushleft}
between the point of origin and the point of consumption in order to
\end{flushleft}


\begin{flushleft}
meet customers' requirements.
\end{flushleft}


\begin{flushleft}
This course provides a practical, management perspective of the
\end{flushleft}


\begin{flushleft}
following areas of logistics: distribution, transportation, international
\end{flushleft}


\begin{flushleft}
logistics, inventory control, sustainable logistics practices, key
\end{flushleft}


\begin{flushleft}
performance indicators, supply chain finance, leadership in a supply
\end{flushleft}


\begin{flushleft}
chain role, and an introduction to logistics technology including RFID
\end{flushleft}


\begin{flushleft}
and ERP systems.
\end{flushleft}





\begin{flushleft}
MCL758 OPTIMIZATION
\end{flushleft}


\begin{flushleft}
3 Credits (3-0-0)
\end{flushleft}


\begin{flushleft}
Optimization Theory in single and multiple dimensions, Karush-KuhnTucker Conditions, Non Linear Programming, Solution Methods,
\end{flushleft}


\begin{flushleft}
Stochastic Programming, Applications and case studies.
\end{flushleft}





\begin{flushleft}
MCL759 Entrepreneurship
\end{flushleft}


\begin{flushleft}
3 Credits (3-0-0)
\end{flushleft}


\begin{flushleft}
Ideation, Team Building, Making of a Business Plan, Securing Funding,
\end{flushleft}


\begin{flushleft}
Legal Procedures, Case studies of successful and failed attempts.
\end{flushleft}





\begin{flushleft}
Revision of fundamentals of plastic deformation and metal forming,
\end{flushleft}


\begin{flushleft}
Constitutive equations for plastic deformation, effect of strain, strain
\end{flushleft}


\begin{flushleft}
rate and temperature, Theory of plasticity, Analysis of important bulk
\end{flushleft}


\begin{flushleft}
forming processes and sheet metal forming processes, Workability,
\end{flushleft}


\begin{flushleft}
Upper and lower bound methods, Slipline field theory, Defects in
\end{flushleft}


\begin{flushleft}
sheet metal forming, Introduction to FE analysis of forming processes.
\end{flushleft}





\begin{flushleft}
MCL770 Stochastic Modeling and Simulation
\end{flushleft}


\begin{flushleft}
3 Credits (3-0-0)
\end{flushleft}


\begin{flushleft}
Overview of Probability Basics, Introduction to Discrete Time
\end{flushleft}


\begin{flushleft}
Markov Chains (DTMC), Transient and Limiting analysis of DTMC,
\end{flushleft}


\begin{flushleft}
Introduction to Continuous Time Markov Chains (CTMC), Transient
\end{flushleft}


\begin{flushleft}
and Limiting analysis of DTMC, Applications, Discrete Event Simulation
\end{flushleft}


\begin{flushleft}
- Introduction, Generation of Random Variables, Simulation modeling
\end{flushleft}


\begin{flushleft}
through case studies.
\end{flushleft}





\begin{flushleft}
MCL771 Value Engineering and Life Cycle Costing
\end{flushleft}


\begin{flushleft}
3 Credits (3-0-0)
\end{flushleft}


\begin{flushleft}
Introduction to Value Engineering and Value Analysis, Methodology
\end{flushleft}


\begin{flushleft}
of V.E., Quantitative definition of use value and prestige value,
\end{flushleft}


\begin{flushleft}
Estimation of product quality/performance, Classification of Functions,
\end{flushleft}


\begin{flushleft}
Functional Cost and Functional Worth, Effect of value improvement
\end{flushleft}


\begin{flushleft}
on profitability. Introduction to V.E. Job plan / Functional Approach to
\end{flushleft}


\begin{flushleft}
Value Improvement, Various phases and techniques of the job plan.
\end{flushleft}


\begin{flushleft}
Life Cycle Costing for managing the Total Value of a Product, Cash
\end{flushleft}


\begin{flushleft}
flow diagrams, Concepts in LCC, Present Value concept, Annuity cost
\end{flushleft}


\begin{flushleft}
concept, Net Present Value, Pay Back period, Internal rate of return on
\end{flushleft}


\begin{flushleft}
investment (IRR), Continuous discounting, Examples and illustrations.
\end{flushleft}


\begin{flushleft}
Creative thinking and creative judgment, False savings, System
\end{flushleft}


\begin{flushleft}
Reliability, Evaluation Matrix, Assessment of value alternatives,
\end{flushleft}


\begin{flushleft}
Estimation of weights and efficiencies, Sensitivity analysis, Utility
\end{flushleft}


\begin{flushleft}
transformation functions, Fast diagramming, Critical path of functions,
\end{flushleft}


\begin{flushleft}
DARSIRI method of value analysis. Critical review of some industry
\end{flushleft}


\begin{flushleft}
oriented projects and case studies.
\end{flushleft}





\begin{flushleft}
MCL772 Reliability Engineering
\end{flushleft}


\begin{flushleft}
3 Credits (3-0-0)
\end{flushleft}





\begin{flushleft}
MCL760 Project Management
\end{flushleft}


\begin{flushleft}
3 Credits (3-0-0)
\end{flushleft}


\begin{flushleft}
Pre-requisites: MTL108
\end{flushleft}


\begin{flushleft}
The nature of projects, the project as a non-repetitive unit production
\end{flushleft}





\begin{flushleft}
Time to failure distributions; Parameter estimation for nonrepairable systems; Reliability models for series, parallel and mixed
\end{flushleft}


\begin{flushleft}
configurations; Reliability models for active/passive redundancy,
\end{flushleft}





265





\begin{flushleft}
\newpage
Mechanical Engineering
\end{flushleft}





\begin{flushleft}
load sharing systems, mixed population, competing failure modes;
\end{flushleft}


\begin{flushleft}
Stress-Strength models; Conditional reliability models and residual
\end{flushleft}


\begin{flushleft}
life calculation; Reliability models for multiple operational phases;
\end{flushleft}


\begin{flushleft}
Shock based reliability models; Reliability models for non-repairable
\end{flushleft}


\begin{flushleft}
systems; Parameter estimation for repairable systems, Failure Mode
\end{flushleft}


\begin{flushleft}
and Effects Analysis, Fault Tree Analysis; Failure simulation; Warranty
\end{flushleft}


\begin{flushleft}
cost analysis; Reliability allocation; Reliability of production systems;
\end{flushleft}


\begin{flushleft}
Test plan design for non-accelerated life tests; Accelerated life testing
\end{flushleft}


\begin{flushleft}
models; Burn-in test plans.
\end{flushleft}





\begin{flushleft}
MCL773 Quality Systems
\end{flushleft}


\begin{flushleft}
3 Credits (3-0-0)
\end{flushleft}


\begin{flushleft}
Introduction to quality systems through approaches proposed by Juran,
\end{flushleft}


\begin{flushleft}
Deming, Baldridge, Taguchi, Crossby etc., Quality costs, Requirements
\end{flushleft}


\begin{flushleft}
analysis using methods like Kano's analysis, Requirement mapping
\end{flushleft}


\begin{flushleft}
using QFD, Product and process analysis using Design and Process
\end{flushleft}


\begin{flushleft}
FMEA, Robust design and process improvement using online and
\end{flushleft}


\begin{flushleft}
offline methods for design and analysis of experiments, Shainin's
\end{flushleft}


\begin{flushleft}
tools for variability reduction, Process capability analysis and loss
\end{flushleft}


\begin{flushleft}
functions, Statistical tolerancing, Design of control charts and
\end{flushleft}


\begin{flushleft}
acceptance sampling plans, Quality standards like IS0 9000, ISO
\end{flushleft}


\begin{flushleft}
14000, CMMI etc.; Service quality models and Service blueprints and
\end{flushleft}


\begin{flushleft}
Service FMEA, Case studies.
\end{flushleft}





\begin{flushleft}
MCL775 Special Topics in Industrial Engineering
\end{flushleft}


\begin{flushleft}
3 Credits (3-0-0)
\end{flushleft}


\begin{flushleft}
To be decided by the instructor at the time of offering.
\end{flushleft}





\begin{flushleft}
MCL776 Advances in Metal Forming
\end{flushleft}


\begin{flushleft}
3 Credits (3-0-0)
\end{flushleft}


\begin{flushleft}
Pre-requisites: For UG: MCL131 or MCL132; for PG MCL769
\end{flushleft}


\begin{flushleft}
Advanced metal forming processes such as tube and sheet hydroforming,
\end{flushleft}


\begin{flushleft}
High energy rate forming processes such as EMF, EHF and explosive
\end{flushleft}


\begin{flushleft}
forming. Design of dies for forging, extrusion and wire drawing, Die
\end{flushleft}


\begin{flushleft}
design for sheet metal forming processes such as single and multi-stage
\end{flushleft}


\begin{flushleft}
deep drawing, bending and stretch forming. Materials used for making
\end{flushleft}


\begin{flushleft}
forming tools, Lubrication mechanisms, Metal forming equipment,
\end{flushleft}


\begin{flushleft}
Formability testing of sheet metals, Determination of Forming Limit
\end{flushleft}


\begin{flushleft}
Diagrams and their applications, Warm forming, Micro forming.
\end{flushleft}





\begin{flushleft}
MCL777 Machine Tool Design
\end{flushleft}


\begin{flushleft}
4 Credits (3-0-2)
\end{flushleft}


\begin{flushleft}
Pre-requisites: MCL231 or MCL136
\end{flushleft}


\begin{flushleft}
Course will cover machine tool design process which will include
\end{flushleft}


\begin{flushleft}
machine tool specifications, conceptual design, configuration design,
\end{flushleft}


\begin{flushleft}
mechanical structure design, design of drives \& controls. Methods
\end{flushleft}


\begin{flushleft}
of achieving required mechanical accuracies considering static,
\end{flushleft}


\begin{flushleft}
dynamic and thermal loads, geometric, kinematic and thermal error
\end{flushleft}


\begin{flushleft}
compensation. Machine tool acceptance tests and characterization of
\end{flushleft}


\begin{flushleft}
machine tools for no-load and load conditions.
\end{flushleft}





\begin{flushleft}
MCL778 Design and Metallurgy of Welded Joints
\end{flushleft}


\begin{flushleft}
4 Credits (3-0-2)
\end{flushleft}


\begin{flushleft}
Importance of welding in fabrication, Problems and difficulties in
\end{flushleft}


\begin{flushleft}
welded structures, service and fabrication tests and their importance,
\end{flushleft}


\begin{flushleft}
weld testing and qualification, causes and remedies for weld defects,
\end{flushleft}


\begin{flushleft}
weld symbols, weld joint design for strength and quality and
\end{flushleft}


\begin{flushleft}
automation in welding.
\end{flushleft}





\begin{flushleft}
MCL780 Casting Technology
\end{flushleft}


\begin{flushleft}
4 Credits (3-0-2)
\end{flushleft}


\begin{flushleft}
Sand casting: sand molding techniques, Core sand and core fabrication.
\end{flushleft}


\begin{flushleft}
Other casting: Permanent mold, pressure die casting, squeeze
\end{flushleft}


\begin{flushleft}
casting, centrifugal casting, continuous casting, stir casting, defects
\end{flushleft}


\begin{flushleft}
and inspection.
\end{flushleft}


\begin{flushleft}
Gating system, risering system, casting design: Metallurgical
\end{flushleft}


\begin{flushleft}
consideration, design consideration, economical consideration. Fluidity
\end{flushleft}


\begin{flushleft}
testing, Application of CAD\ensuremath{\backslash}CAM in foundry.
\end{flushleft}





\begin{flushleft}
Casting of complicated shapes: automotive components, casting of
\end{flushleft}


\begin{flushleft}
light alloys -- Aluminum, magnesium and Titanium alloys.
\end{flushleft}


\begin{flushleft}
Advances in near net shape manufacturing: Metal Injection moulding,
\end{flushleft}


\begin{flushleft}
Laser engineered net shaping.
\end{flushleft}





\begin{flushleft}
MCL781 Machining Processes and Analysis
\end{flushleft}


\begin{flushleft}
4 Credits (3-0-2)
\end{flushleft}


\begin{flushleft}
Introduction to basic traditional machining processes -- the need and
\end{flushleft}


\begin{flushleft}
requirements for such processes and their brief application areas.
\end{flushleft}


\begin{flushleft}
Specifications and geometry of various cutting tools such as turning
\end{flushleft}


\begin{flushleft}
tools, drills, milling cutters in different referencing systems such as
\end{flushleft}


\begin{flushleft}
work reference, tool reference and machine reference systems.
\end{flushleft}


\begin{flushleft}
Methods and techniques used for sharpening / resharpening of these
\end{flushleft}


\begin{flushleft}
cutting tools.
\end{flushleft}


\begin{flushleft}
Mechanisms of chip formation by single point, drilling and milling
\end{flushleft}


\begin{flushleft}
tools. Different types of chips obtained during machining. Concept of
\end{flushleft}


\begin{flushleft}
effective rake angle during machining.
\end{flushleft}


\begin{flushleft}
Mechanics of machining of single point cutting tool, drill and milling
\end{flushleft}


\begin{flushleft}
cutter -- estimation of cutting forces using analytical models,
\end{flushleft}


\begin{flushleft}
Experimental methods and instruments used for cutting force
\end{flushleft}


\begin{flushleft}
determination during machining processes, essential design features
\end{flushleft}


\begin{flushleft}
of the dynamometers used for such measurement.
\end{flushleft}


\begin{flushleft}
Heat transfer during machining processes, identification of the different
\end{flushleft}


\begin{flushleft}
sources of heat generation and development of suitable models for
\end{flushleft}


\begin{flushleft}
analytical estimation of the cutting temperature, Experimental methods
\end{flushleft}


\begin{flushleft}
used for estimating cutting temperature.
\end{flushleft}


\begin{flushleft}
Use of cutting fluids in machining -- purposes, proper selection and
\end{flushleft}


\begin{flushleft}
methods of application of such cutting fluids.
\end{flushleft}


\begin{flushleft}
Advanced cutting tool materials and processes used for development
\end{flushleft}


\begin{flushleft}
of such tools.
\end{flushleft}


\begin{flushleft}
Mechanics of cutting tool wear and development of models for
\end{flushleft}


\begin{flushleft}
assessing the tool life.
\end{flushleft}


\begin{flushleft}
Economics of machining -- identifying the major parameters in
\end{flushleft}


\begin{flushleft}
machining and their roles on cutting force, surface finish and cutting
\end{flushleft}


\begin{flushleft}
temperature, selection of optimal conditions of process parameters
\end{flushleft}


\begin{flushleft}
to reduce machining costs through suitable models.
\end{flushleft}


\begin{flushleft}
Introduction to Grinding processes and understanding of the
\end{flushleft}


\begin{flushleft}
differences between machining and grinding.
\end{flushleft}


\begin{flushleft}
Grinding for bulk material removal -- creep feed grinding -- fast feed
\end{flushleft}


\begin{flushleft}
grinding.
\end{flushleft}


\begin{flushleft}
Superabrasive grinding wheels -- both monolayer and multilayer,
\end{flushleft}


\begin{flushleft}
advantages of the monolayer wheel and its applications.
\end{flushleft}


\begin{flushleft}
Grinding Geometry and Kinematics -- contact length -- measurement
\end{flushleft}


\begin{flushleft}
of contact length definition of active grits and methods used for
\end{flushleft}


\begin{flushleft}
estimating active grits, use of single grit experiments to develop models
\end{flushleft}


\begin{flushleft}
for estimating forces and specific energy requirement in grinding.
\end{flushleft}


\begin{flushleft}
Wheel Conditioning -- truing and dressing techniques and parameters
\end{flushleft}


\begin{flushleft}
-- effect of dressing on grinding wheel parameters -- dressing of superabrasive wheels.
\end{flushleft}


\begin{flushleft}
Temperature generation during grinding process and thermal modelling
\end{flushleft}


\begin{flushleft}
of the process.
\end{flushleft}


\begin{flushleft}
Thermal damages in grinding -- burning -- oxidation -- tempering
\end{flushleft}


\begin{flushleft}
-- residual stresses -- effect of residual stresses on job quality --
\end{flushleft}


\begin{flushleft}
measurement of residual stresses -- introduction to X-ray diffractometry
\end{flushleft}


\begin{flushleft}
-- XRD measurement of residual stress -- application of Barkhausen
\end{flushleft}


\begin{flushleft}
Noise Technique in grinding.
\end{flushleft}


\begin{flushleft}
Special Machining and grinding processes such as $\bullet$ Ductile Regime
\end{flushleft}


\begin{flushleft}
Grinding $\bullet$ Diamond Turning.
\end{flushleft}





\begin{flushleft}
MCL782 Computational Methods
\end{flushleft}


\begin{flushleft}
2 Credits (2-0-0)
\end{flushleft}


\begin{flushleft}
Errors in numerical calculations and series approximations, Solution of
\end{flushleft}


\begin{flushleft}
algebraic and transcendental equations, Interpolation of data, finite
\end{flushleft}


\begin{flushleft}
differences, Curve fitting, Numerical differentiation and integration,
\end{flushleft}


\begin{flushleft}
Matrices and linear system of equations, Numerical solution of
\end{flushleft}


\begin{flushleft}
ordinary differential and partial differential equations, Solution
\end{flushleft}


\begin{flushleft}
of integral equations, Numerical solution of important production
\end{flushleft}


\begin{flushleft}
engineering problems.
\end{flushleft}





266





\begin{flushleft}
\newpage
Mechanical Engineering
\end{flushleft}





\begin{flushleft}
MCL783 Automation in Manufacturing
\end{flushleft}


\begin{flushleft}
4 Credits (3-0-2)
\end{flushleft}





\begin{flushleft}
MCL788 Surface Engineering
\end{flushleft}


\begin{flushleft}
4 Credits (3-0-2)
\end{flushleft}





\begin{flushleft}
Introduction to Automation of different manufacturing processes.
\end{flushleft}


\begin{flushleft}
Types of systems - mechanical, electrical, electronics; Data conversion
\end{flushleft}


\begin{flushleft}
devices, transducers, signal processing devices, relays, contactors and
\end{flushleft}


\begin{flushleft}
timers. Sensors and their interfaces; Hydraulics \& Pneumatic Systems
\end{flushleft}


\begin{flushleft}
design and their application to manufacturing equipment; Sequence
\end{flushleft}


\begin{flushleft}
operation of hydraulic and pneumatic cylinders and motors; Electro
\end{flushleft}


\begin{flushleft}
Pneumatic \& Electro Hydraulic Systems design, Relay Logic circuits,
\end{flushleft}


\begin{flushleft}
Feedback control systems, PID Controller; Drives and mechanisms of
\end{flushleft}


\begin{flushleft}
an automated system: stepper motors, servo drives. Ball screws, linear
\end{flushleft}


\begin{flushleft}
motion bearings, electronic caming and gearing, indexing mechanisms,
\end{flushleft}


\begin{flushleft}
tool magazines, and transfer systems. Programmable Logic Controllers,
\end{flushleft}


\begin{flushleft}
I/Os, system interfacing, ladder logic, functional blocks, structured
\end{flushleft}


\begin{flushleft}
text, and applications. Human Machine Interface \& SCADA; Motion
\end{flushleft}


\begin{flushleft}
controller and their programming, PLCOpen Motion Control blocks,
\end{flushleft}


\begin{flushleft}
multi axes coordinated motion, CNC control; RFID technology and
\end{flushleft}


\begin{flushleft}
its application; Machine vision and control applications. Modular
\end{flushleft}


\begin{flushleft}
Production Systems -- Distribution, Conveying, Pick \& Place etc.
\end{flushleft}





\begin{flushleft}
Introduction to surface engineering -- importance and scope of surface
\end{flushleft}


\begin{flushleft}
engineering, conventional surface engineering practices like pickling,
\end{flushleft}


\begin{flushleft}
grinding, buffing etc., surface engineering by material addition
\end{flushleft}


\begin{flushleft}
like electroplating, surface modification of ferrous and non-ferrous
\end{flushleft}


\begin{flushleft}
materials like nitriding, cyaniding, aluminizing etc.
\end{flushleft}





\begin{flushleft}
Laboratory work will be hands-on design and operation of automatic
\end{flushleft}


\begin{flushleft}
systems.
\end{flushleft}





\begin{flushleft}
MCL784 Computer Aided Manufacturing
\end{flushleft}


\begin{flushleft}
4 Credits (3-0-2)
\end{flushleft}


\begin{flushleft}
An overview of Computer Aided Manufacturing. Use and Programming
\end{flushleft}


\begin{flushleft}
of Computer Controlled Machines such as CNC, 3D Printing, CMM,
\end{flushleft}


\begin{flushleft}
Robots etc. Constructional aspects of computer controlled machines.
\end{flushleft}


\begin{flushleft}
Geometric modeling and computational geometry for manufacturing.
\end{flushleft}


\begin{flushleft}
Product Life-cycle modeling. Virtual and Distributed Manufacturing.
\end{flushleft}





\begin{flushleft}
MCL785 Advanced Machining Processes
\end{flushleft}


\begin{flushleft}
3 Credits (3-0-0)
\end{flushleft}


\begin{flushleft}
Introduction to advanced machining processes -- need for such
\end{flushleft}


\begin{flushleft}
processes and application areas
\end{flushleft}


\begin{flushleft}
Mechanical Energy utilized advanced machining processes like
\end{flushleft}


\begin{flushleft}
ultrasonic machining, abrasive flow machining, magnetic abrasive
\end{flushleft}


\begin{flushleft}
finishing,magneto-rheological finishing, abrasive water jet machining
\end{flushleft}


\begin{flushleft}
-- mechanics of cutting, process parametric analysis, process
\end{flushleft}


\begin{flushleft}
capabilities, applications.
\end{flushleft}


\begin{flushleft}
Thermoelectric based advanced machining processes like electro
\end{flushleft}


\begin{flushleft}
discharge machining, wire EDM, Plasma Arc Machining, Laser Beam
\end{flushleft}


\begin{flushleft}
Machining, Focussed Ion Beam Machining -- working principles,
\end{flushleft}


\begin{flushleft}
material removal mechanisms, process capabilities and applications.
\end{flushleft}


\begin{flushleft}
Electrochemical and Chemical Advanced Machining -- ECG,
\end{flushleft}


\begin{flushleft}
Electrostream Drilling, Chemical Machining -- process characteristics,
\end{flushleft}


\begin{flushleft}
numerical modelling of the processes, applications and limitations.
\end{flushleft}





\begin{flushleft}
MCL786 Metrology
\end{flushleft}


\begin{flushleft}
3 Credits (2-0-2)
\end{flushleft}


\begin{flushleft}
Introduction to Dimensional Metrology, standardization,
\end{flushleft}


\begin{flushleft}
interchangeability, selective assembly, Indian standard specifications,
\end{flushleft}


\begin{flushleft}
application of tolerances, Limit gauging- Taylor's principles of
\end{flushleft}


\begin{flushleft}
limit gauging, Design of Gauges, Inspection by measurement;
\end{flushleft}


\begin{flushleft}
interferometers. GD\&T, Applications of Dimensional Inspection,
\end{flushleft}


\begin{flushleft}
Inspection of Surface Quality, Feature inspection- straightness,
\end{flushleft}


\begin{flushleft}
flatness, parallelism, squareness, circularity and roundness.
\end{flushleft}


\begin{flushleft}
Automated Dimensional Measurements: Introduction, Automatic
\end{flushleft}


\begin{flushleft}
Gauging, Automatic Measuring Machines for inspecting multiple
\end{flushleft}


\begin{flushleft}
work piece dimensions, Automatic Gauging Machine Part-Matching
\end{flushleft}


\begin{flushleft}
Functions, Coordinate Measuring Machines-Types, Probes, Accessories,
\end{flushleft}


\begin{flushleft}
Measurement, Computer supported Coordinate Measurements.
\end{flushleft}





\begin{flushleft}
MCL787 Welding Science and Technology
\end{flushleft}


\begin{flushleft}
4 Credits (3-0-2)
\end{flushleft}


\begin{flushleft}
General survey and classification of welding processes, importance
\end{flushleft}


\begin{flushleft}
of advanced materials and joining technologies, weld arc physics,
\end{flushleft}


\begin{flushleft}
power sources and their characteristics, welding technologies related
\end{flushleft}


\begin{flushleft}
to industries: automotive, aerospace, nuclear, oil and gas industries.
\end{flushleft}





\begin{flushleft}
Advanced surface engineering practices like laser assisted surface
\end{flushleft}


\begin{flushleft}
modification, electron beam assisted modification, spraying techniques
\end{flushleft}


\begin{flushleft}
like flame and plasma spraying, high velocity oxyfuel, cold spray techniques.
\end{flushleft}


\begin{flushleft}
Sputter deposition processes, PVD and CVD methods of surface
\end{flushleft}


\begin{flushleft}
coatings, surface modification by ion implantation and ion beam mixing
\end{flushleft}


\begin{flushleft}
Characterisation of the engineered surface and coatings like
\end{flushleft}


\begin{flushleft}
thickness, porosity and adhesion of coatings, surface microscopy and
\end{flushleft}


\begin{flushleft}
spectroscopic analysis of the modified surfaces.
\end{flushleft}


\begin{flushleft}
Functional coatings and their applications.
\end{flushleft}





\begin{flushleft}
MCP790 Process Engineering
\end{flushleft}


\begin{flushleft}
4 Credits (2-0-4)
\end{flushleft}


\begin{flushleft}
Understanding relation between geometry, materials and manufacturing
\end{flushleft}


\begin{flushleft}
in relation to process planning. Design for Manufacture and Assembly,
\end{flushleft}


\begin{flushleft}
Product life-cycle considerations, Selection of raw material geometries,
\end{flushleft}


\begin{flushleft}
process selection, selection of manufacturing equipment, process
\end{flushleft}


\begin{flushleft}
sequencing, tooling, work holding and In-process inspection. Process
\end{flushleft}


\begin{flushleft}
Planning for Assembly \& Inspection, Computer Aided Process Planning,
\end{flushleft}


\begin{flushleft}
Lean concepts in manufacturing.
\end{flushleft}





\begin{flushleft}
MCL791 Processing and Mechanics of Composite
\end{flushleft}


\begin{flushleft}
Materials
\end{flushleft}


\begin{flushleft}
4 Credits (3-0-2)
\end{flushleft}


\begin{flushleft}
Introduction to matrix materials - polymers, metals and ceramics.
\end{flushleft}


\begin{flushleft}
Introduction to reinforcements -- fibers, flakes, particulates: macro,
\end{flushleft}


\begin{flushleft}
micro and nano. Hand layup, tape layup, autoclave moulding, vacuum
\end{flushleft}


\begin{flushleft}
bag moulding, compression moulding, resin transfer moulding, reaction
\end{flushleft}


\begin{flushleft}
injection moulding, filament winding, pultrusion, braiding, and other
\end{flushleft}


\begin{flushleft}
manufacturing variants. Macromechanics of a lamina as a building
\end{flushleft}


\begin{flushleft}
block of a composite structure, stress-strain relation, strain --stress
\end{flushleft}


\begin{flushleft}
relation for anisotropic material, orthotropic materials, material
\end{flushleft}


\begin{flushleft}
property matrix estimation, micromechanics of a lamina, effect of
\end{flushleft}


\begin{flushleft}
fiber volume fraction on properties, failure theories of a lamina.
\end{flushleft}


\begin{flushleft}
Laminate analysis, failure of a laminate, design principles of tailor
\end{flushleft}


\begin{flushleft}
made material systems.
\end{flushleft}





\begin{flushleft}
MCL792 Injection Molding and Mold Design
\end{flushleft}


\begin{flushleft}
3 Credits (2-0-2)
\end{flushleft}


\begin{flushleft}
Introduction to Injection molding fundamentals, flow of non-Newtonian
\end{flushleft}


\begin{flushleft}
fluids, flow of various polymer melts in a cavity, molding cycle, injection
\end{flushleft}


\begin{flushleft}
molding machine characteristics- injection unit design, clamping unit
\end{flushleft}


\begin{flushleft}
design, shrinkage, warpage, defect free product, Moldflow analysis --
\end{flushleft}


\begin{flushleft}
fundamentals of FE analysis for fill, cool, warp, stress, DOE, results
\end{flushleft}


\begin{flushleft}
interpretation. Mold design fundamentals, type of molds - two plate,
\end{flushleft}


\begin{flushleft}
three plate, feeding system -- sprue, runner, gate design, ejection
\end{flushleft}


\begin{flushleft}
system - pin, sleeve, stripper plate, air ejection design. Moldings with
\end{flushleft}


\begin{flushleft}
undercuts -- internal, external, threads, split cavity, split core designs.
\end{flushleft}


\begin{flushleft}
Advances in injection molding process- microcellular, gas assisted,
\end{flushleft}


\begin{flushleft}
insert, outsert, push-pull, multilive, vibration assisted, micro lamellar,
\end{flushleft}


\begin{flushleft}
lost core. Designing with plastics, applications and future research.
\end{flushleft}





\begin{flushleft}
MCL796 Additive Manufacturing
\end{flushleft}


\begin{flushleft}
4 Credits (3-0-2)
\end{flushleft}


\begin{flushleft}
Re v i e w o f s o l i d m o d e l i n g t e c h n i q u e s w i t h c o m p a r i s o n .
\end{flushleft}


\begin{flushleft}
Product development. Simultaneous Engineering and Additive
\end{flushleft}


\begin{flushleft}
Manufacturing(AM). Basic Principle of AM processes. Support structure
\end{flushleft}


\begin{flushleft}
in Additive Manufacturing. Containment and critical applications.
\end{flushleft}


\begin{flushleft}
Generation of the physical layer model. Classification of AM Processes.
\end{flushleft}


\begin{flushleft}
Virtual Prototyping. Tessellation (STL format) and tessellation
\end{flushleft}


\begin{flushleft}
algorithms. Defects in STL files and repairing algorithms. Slicing and
\end{flushleft}


\begin{flushleft}
various slicing procedures. Accuracy and Surface quality in Additive
\end{flushleft}


\begin{flushleft}
Manufacturing. Effect of part deposition orientation on accuracy,
\end{flushleft}





267





\begin{flushleft}
\newpage
Mechanical Engineering
\end{flushleft}





\begin{flushleft}
surface finish, build time, support structure, cost etc. Various Rapid
\end{flushleft}


\begin{flushleft}
tooling techniques. Introduction to Reverse Engineering. Reverse
\end{flushleft}


\begin{flushleft}
engineering and Additive Manufacturing.
\end{flushleft}





\begin{flushleft}
MCD800 Professional Project Activity
\end{flushleft}


\begin{flushleft}
3 Credits (0-0-6)
\end{flushleft}


\begin{flushleft}
Project Work.
\end{flushleft}





\begin{flushleft}
mixed convection - Pool boiling : nucleate boiling-film boiling, flow
\end{flushleft}


\begin{flushleft}
boiling-condensation : dropwise condensation-film condensation
\end{flushleft}


\begin{flushleft}
Nusselt theory-Special topics-Convective heat transfer in rotating
\end{flushleft}


\begin{flushleft}
systems, Microscale convective heat transfer, Convective heat
\end{flushleft}


\begin{flushleft}
transfer with nano-fluids, Combined convection and radiation,
\end{flushleft}


\begin{flushleft}
Double diffusive convection.
\end{flushleft}





\begin{flushleft}
MCL815 Fire Dynamics and Engineering
\end{flushleft}


\begin{flushleft}
4 Credits (2-0-4)
\end{flushleft}





\begin{flushleft}
MCD810 Major Project Part-I (Thermal Engineering)
\end{flushleft}


\begin{flushleft}
12 Credits (0-0-24)
\end{flushleft}


\begin{flushleft}
MCD811 Major Project Part-I (Thermal Engineering)
\end{flushleft}


\begin{flushleft}
6 Credits (0-0-12)
\end{flushleft}


\begin{flushleft}
MCL811 Advanced Power Generation Systems
\end{flushleft}


\begin{flushleft}
3 Credits (3-0-0)
\end{flushleft}


\begin{flushleft}
General Introduction to current power generation technology and
\end{flushleft}


\begin{flushleft}
need for advances systems. Analysis of Advanced Ultra super-ciritical
\end{flushleft}


\begin{flushleft}
power plants, Organic Rankine Cycle based systems, Power systems
\end{flushleft}


\begin{flushleft}
using mixtures as working fluids. Sizing of components for the selected
\end{flushleft}


\begin{flushleft}
systems. Design of power systems for solar, biomass and geothermal
\end{flushleft}


\begin{flushleft}
sources. Thermo-fluid analysis of solar PV systems. Hybrid solar
\end{flushleft}


\begin{flushleft}
PV-thermal systems. Recent developments in hydro power systems.
\end{flushleft}





\begin{flushleft}
MCD812 Major Project Part 2 (Thermal Engineering)
\end{flushleft}


\begin{flushleft}
12 Credits (0-0-24)
\end{flushleft}


\begin{flushleft}
MCL812 Combustion
\end{flushleft}


\begin{flushleft}
3 Credits (3-0-0)
\end{flushleft}


\begin{flushleft}
Introduction - importance of combustion. Chemical thermodynamics
\end{flushleft}


\begin{flushleft}
and chemical kinetics. Important chemical mechanisms. Coupling
\end{flushleft}


\begin{flushleft}
chemical and thermal analysis of reacting systems. Premixed systems:
\end{flushleft}


\begin{flushleft}
detonation and deflagration, laminar flames, burning velocity,
\end{flushleft}


\begin{flushleft}
flammability limits, quenching and ignition. Turbulent premixed flames.
\end{flushleft}


\begin{flushleft}
Non-premixed systems: laminar diffusion flame jet, droplet burning.
\end{flushleft}


\begin{flushleft}
Combustion of solids: drying, devolatilization and char combustion.
\end{flushleft}


\begin{flushleft}
Practical aspects of coal combustion, woodstove combustion.
\end{flushleft}





\begin{flushleft}
MCL813 Computational Heat Transfer
\end{flushleft}


\begin{flushleft}
4 Credits (3-0-2)
\end{flushleft}


\begin{flushleft}
Mathematical Description of the Physical Phenomena- Governing
\end{flushleft}


\begin{flushleft}
equations---mass, momentum, energy, species, General form of the
\end{flushleft}


\begin{flushleft}
scalar transport equation, Elliptic, parabolic and hyperbolic equations.
\end{flushleft}


\begin{flushleft}
Discretization Methods- Introduction to finite difference and finite
\end{flushleft}


\begin{flushleft}
volume method, Consistency, stability and convergence.
\end{flushleft}


\begin{flushleft}
Diffusion Equation- 1D-2D steady diffusion, Source terms, non-linearity,
\end{flushleft}


\begin{flushleft}
Boundary conditions, interface diffusion coefficient, Under-relaxation,
\end{flushleft}


\begin{flushleft}
Solution of linear equations (preliminary), Unsteady diffusion, Explicit,
\end{flushleft}


\begin{flushleft}
Implicit and Crank-Nicolson scheme.
\end{flushleft}


\begin{flushleft}
Convection and Diffusion- Steady one-dimensional convection and
\end{flushleft}


\begin{flushleft}
diffusion, Upwind, exponential, hybrid, power, QUICK scheme, Twodimensional convection-diffusion.
\end{flushleft}


\begin{flushleft}
Flow Field Calculation- Incompressibility issues and pressure-velocity
\end{flushleft}


\begin{flushleft}
coupling, Primitive variable versus other methods, Vorticity-stream
\end{flushleft}


\begin{flushleft}
function formulation, Staggered grid, SIMPLE family of algorithms.
\end{flushleft}


\begin{flushleft}
Radiative heat transfer - Computation of surface radiation using zone
\end{flushleft}


\begin{flushleft}
method, Solution of radiative transfer equation using discrete transfer,
\end{flushleft}


\begin{flushleft}
discrete ordinates and finite volume methods.
\end{flushleft}





\begin{flushleft}
MCL814 Convective Heat Transfer
\end{flushleft}


\begin{flushleft}
3 Credits (3-0-0)
\end{flushleft}


\begin{flushleft}
Derivation of energy equation-Similarity solutions for laminar
\end{flushleft}


\begin{flushleft}
external flows-Laminar internal flows-Transition flow-Heat transfer
\end{flushleft}


\begin{flushleft}
in transition flow-Reynolds averaged equations of motion, Averaged
\end{flushleft}


\begin{flushleft}
energy equations-Turbulent flow and heat transfer over a flat
\end{flushleft}


\begin{flushleft}
plate-Turbulent flow and heat transfer in pipes and channelsLaminar and turbulent natural convection-laminar and turbulent
\end{flushleft}





\begin{flushleft}
Basics of Conservation equations, Turbulence, radiation and
\end{flushleft}


\begin{flushleft}
thermochemistry. Ignition of solids- Burning and heat release rates.
\end{flushleft}


\begin{flushleft}
Properties of fire plumes- buoyant plumes and interactions with
\end{flushleft}


\begin{flushleft}
surfaces. Turbulent diffusion flames- structure, modeling, soot
\end{flushleft}


\begin{flushleft}
formation and radiation effects. Toxic products. Fire chemistry, thermal
\end{flushleft}


\begin{flushleft}
decomposition of bulk fuel, pyrolysis, nitrogen and halogen chemistry.
\end{flushleft}


\begin{flushleft}
Fire growth- ignition, initial conditions, flame and fire spread theory,
\end{flushleft}


\begin{flushleft}
feedback to fuel. Compartment zone models. Flashover, post-flashover
\end{flushleft}


\begin{flushleft}
and control. Fire detection, suppression methods, codes, standards
\end{flushleft}


\begin{flushleft}
and laws. Case studies of real fires- buildings, transport, industries,
\end{flushleft}


\begin{flushleft}
shamiana and jhuggi-jhonpdi etc.
\end{flushleft}





\begin{flushleft}
MCL816 Gas Dynamics
\end{flushleft}


\begin{flushleft}
4 Credits (3-0-2)
\end{flushleft}


\begin{flushleft}
Revision of fundamentals. Thermodynamics of compressible flow
\end{flushleft}


\begin{flushleft}
-- wave motion in compressible medium, Mach number and cone,
\end{flushleft}


\begin{flushleft}
properties. Steady one-dimensional compressible flow through
\end{flushleft}


\begin{flushleft}
variable area ducts. Converging and converging-diverging nozzles
\end{flushleft}


\begin{flushleft}
and diffusers. Effects of heating and friction in duct flow, Rayleigh
\end{flushleft}


\begin{flushleft}
and Fanno lines. Flows with normal shocks. Oblique shocks and
\end{flushleft}


\begin{flushleft}
reflection. Expansion waves. Prandtl-Meyer flow. Flow over bodies.
\end{flushleft}


\begin{flushleft}
Measurements and applications. Jet propulsion -- types of engines,
\end{flushleft}


\begin{flushleft}
propulsion fundamentals. Compressor, combustor and turbines
\end{flushleft}


\begin{flushleft}
construction and performance. Rocket propulsion -- basics, solid and
\end{flushleft}


\begin{flushleft}
liquid propelled engines, parametric studies, construction features,
\end{flushleft}


\begin{flushleft}
single and multi-stage rockets. Thrust chamber and nozzle models.
\end{flushleft}


\begin{flushleft}
Studies of in-use engines. Environmental aspects.
\end{flushleft}





\begin{flushleft}
MCL817 Heat Exchangers
\end{flushleft}


\begin{flushleft}
3 Credits (3-0-0)
\end{flushleft}


\begin{flushleft}
Applications. Basic design methodologies -- LMTD and effectivenessNTU methods. Overall heat transfer coefficient, fouling. Correlations for
\end{flushleft}


\begin{flushleft}
heat transfer coefficient and friction factor. Classification and types of
\end{flushleft}


\begin{flushleft}
heat exchangers and construction details. Design and rating of double
\end{flushleft}


\begin{flushleft}
pipe heat exchangers, shell and tube heat exchangers, compact heat
\end{flushleft}


\begin{flushleft}
exchangers, plate and heat pipe type, condensers, cooling towers.
\end{flushleft}


\begin{flushleft}
Heat exchanger standards and testing,Heat transfer enhancement
\end{flushleft}


\begin{flushleft}
and efficient surfaces.
\end{flushleft}





\begin{flushleft}
MCL818 Heating, Ventilating and Air-conditioning
\end{flushleft}


\begin{flushleft}
3 Credits (3-0-0)
\end{flushleft}


\begin{flushleft}
Introduction, psychrometry of airconditioning processes. HVAC
\end{flushleft}


\begin{flushleft}
technologies. Thermal comfort - factors influencing thermal comfort.
\end{flushleft}


\begin{flushleft}
Cooling and Heating load calculations. Room air distribution principles.
\end{flushleft}


\begin{flushleft}
Design of air duct systems.
\end{flushleft}


\begin{flushleft}
Indoor air quality. Ventilation - need, principles. Various types of air
\end{flushleft}


\begin{flushleft}
conditioning systems. Cooling, dehumidification and humidification
\end{flushleft}


\begin{flushleft}
equipment. Temperature, pressure and humidity controllers. Various
\end{flushleft}


\begin{flushleft}
types of controls and control strategies.
\end{flushleft}





\begin{flushleft}
MCL819 Lattice Boltzmann method
\end{flushleft}


\begin{flushleft}
3 Credits (3-0-0)
\end{flushleft}


\begin{flushleft}
Introduction, Kinetic theory and statistical mechanics, Lattice gas
\end{flushleft}


\begin{flushleft}
cellular automata, LBM, Thermal LBM, Boundary conditions, Body forces,
\end{flushleft}


\begin{flushleft}
Multiple relaxation time model, Single component multiphase models,
\end{flushleft}


\begin{flushleft}
Multicomponent models single phase models, Applications of LBM.
\end{flushleft}





\begin{flushleft}
MCL820 Micro/nano scale heat transfer
\end{flushleft}


\begin{flushleft}
4 Credits (3-0-2)
\end{flushleft}


\begin{flushleft}
Introduction to micro/ nano scale transport phenomena, size effect
\end{flushleft}


\begin{flushleft}
behaviour, overview of engg. applications, fundamentals of micro/
\end{flushleft}





268





\begin{flushleft}
\newpage
Mechanical Engineering
\end{flushleft}





\begin{flushleft}
nano scale fluid mechanics and heat transfer -- kinetic theory,
\end{flushleft}


\begin{flushleft}
quantum mechanics considerations, Boltzmann transport equation,
\end{flushleft}


\begin{flushleft}
molecular dynamics modelling, microfluidics, Knudsen number,
\end{flushleft}


\begin{flushleft}
slip theory, micro/nano scale heat conduction - classical/ quantum
\end{flushleft}


\begin{flushleft}
size effects, thermal conductivity models, specific heat, thin films,
\end{flushleft}


\begin{flushleft}
convection in microtubes and channels, nanoparticles and nanofluids
\end{flushleft}


\begin{flushleft}
-- preparation \& transport properties, microfluidics - electrokinetic
\end{flushleft}


\begin{flushleft}
flows, microscale radiative heat transfer -- modelling, properties,
\end{flushleft}


\begin{flushleft}
measurements at microscale.
\end{flushleft}





\begin{flushleft}
MCL821 Radiative Heat Transfer
\end{flushleft}


\begin{flushleft}
3 Credits (3-0-0)
\end{flushleft}


\begin{flushleft}
Introduction to Radiation- Recapitulation: Radiative properties of opaque
\end{flushleft}


\begin{flushleft}
surfaces, Intensity, emissive power, radiosity, Planck's law, Wien's
\end{flushleft}


\begin{flushleft}
displacement law, Black and Gray surfaces, View factors.	
\end{flushleft}


\begin{flushleft}
Enclosure with Transparent Medium- Enclosure analysis for diffuse-gray
\end{flushleft}


\begin{flushleft}
surfaces and non-diffuse, non-gray surfaces, net radiation method.
\end{flushleft}


\begin{flushleft}
Radiative heat transfer in Participating Medium- Radiation in absorbing,
\end{flushleft}


\begin{flushleft}
emitting and scattering media. Absorption, scattering and extinction
\end{flushleft}


\begin{flushleft}
coefficients, Radiative transfer equation. Analytical solution of radiative
\end{flushleft}


\begin{flushleft}
transfer equation.
\end{flushleft}


\begin{flushleft}
Introduction to different radiation model- Discrete transfer method,
\end{flushleft}


\begin{flushleft}
discrete ordinates method Radiation from particulate media,
\end{flushleft}


\begin{flushleft}
Dependent versus independent scattering Non-gray radiation,
\end{flushleft}


\begin{flushleft}
Modelling of non-gray radiation. Transient radiation and its solution.
\end{flushleft}


\begin{flushleft}
Radiative transfer in porous media. Combined Heat Transfer ModesRadiation with conduction, combined boundary layer.
\end{flushleft}





\begin{flushleft}
MCL822 Steam and Gas Turbines
\end{flushleft}


\begin{flushleft}
4 Credits (3-0-2)
\end{flushleft}


\begin{flushleft}
Introduction, Recapitulation of heat cycles of steam power plants
\end{flushleft}


\begin{flushleft}
and gas turbine engines,Thermodynamics and fluid dynamics of
\end{flushleft}


\begin{flushleft}
compressible flow through turbines, meanline analysis and design
\end{flushleft}


\begin{flushleft}
of axial flow turbines,Three dimensional flows in axial flow turbines,
\end{flushleft}


\begin{flushleft}
Partial admission turbines, Turbines for nuclear power plants, Steam
\end{flushleft}


\begin{flushleft}
turbines for co-generation, turbine for super critical thermal power
\end{flushleft}


\begin{flushleft}
plant, operation of turbine plants- start up and shut-down of a turbine,
\end{flushleft}


\begin{flushleft}
steady state operation.
\end{flushleft}





\begin{flushleft}
MCL823 Thermal Design
\end{flushleft}


\begin{flushleft}
4 Credits (3-0-2)
\end{flushleft}


\begin{flushleft}
Introduction to design, modelling and simulation, components
\end{flushleft}


\begin{flushleft}
and systems. Component design. Design of heat sink - single fin
\end{flushleft}


\begin{flushleft}
optimization, multiple fin array. Design of compact heat exchangers Fundamentals, shell and tube heat exchangers, plate heat exchangers,
\end{flushleft}


\begin{flushleft}
finned tube heat exchangers. Design of Heat pipe - Fundamentals,
\end{flushleft}


\begin{flushleft}
design procedure. Design of thermoelectric devices - Fundamentals,
\end{flushleft}


\begin{flushleft}
thermoelectric generator, thermoelectric cooler, module design.
\end{flushleft}


\begin{flushleft}
System design:
\end{flushleft}


\begin{flushleft}
Design of thermal systems: System identification and description with
\end{flushleft}


\begin{flushleft}
mathematical modelling: Examples with Power plant, refrigeration plant,
\end{flushleft}


\begin{flushleft}
HVAC systems, pump pipe network, electric space heaters, wind tunnel.
\end{flushleft}


\begin{flushleft}
Development of a numerical model, mathematical techniques, solution
\end{flushleft}


\begin{flushleft}
of non-linear equations, numerical model for a system, system
\end{flushleft}


\begin{flushleft}
simulation, methods of numerical simulation.
\end{flushleft}


\begin{flushleft}
Optimization - basic concepts, optimization of thermal systems,
\end{flushleft}


\begin{flushleft}
Lagrange multiplier, optimization of unconstrained problems, search
\end{flushleft}


\begin{flushleft}
based methods, Genetic algorithm, Differential Evolution method.
\end{flushleft}


\begin{flushleft}
Thermal design based on inverse methods - Definition, estimation of
\end{flushleft}


\begin{flushleft}
boundary condition, conjugate gradient method.
\end{flushleft}





\begin{flushleft}
MCL824 Turbocompressors
\end{flushleft}


\begin{flushleft}
3 Credits (3-0-0)
\end{flushleft}


\begin{flushleft}
Introduction, Fluid mechanics and thermodynamics of axial and
\end{flushleft}


\begin{flushleft}
radial flow compressors, operation and performance of compressors,
\end{flushleft}


\begin{flushleft}
compressor cascades, blade to blade flow for axial compressors with
\end{flushleft}


\begin{flushleft}
subsonic inlet flow, blade-to-blade flow for axial flow compressors
\end{flushleft}


\begin{flushleft}
with supersonic inlet flow, loss correlations, performance analysis
\end{flushleft}





\begin{flushleft}
of axial flow compressors, Centrifugal compressor - the centrifugal
\end{flushleft}


\begin{flushleft}
impeller, diffuser of centrifugal compressor, stall and surge, supersonic
\end{flushleft}


\begin{flushleft}
compressors, compressor instrumentation and testing.
\end{flushleft}





\begin{flushleft}
MCL825 Design of Wind Power Farms
\end{flushleft}


\begin{flushleft}
4 Credits (3-0-2)
\end{flushleft}


\begin{flushleft}
General Introduction to Wind Turbines, Analysis of wind source, 2-D
\end{flushleft}


\begin{flushleft}
Aerodynamics, 3-D Aerodynamics, Momentum Theory for an Ideal
\end{flushleft}


\begin{flushleft}
Wind Turbine, wind turbine performance, Design of HAWT, Design of
\end{flushleft}


\begin{flushleft}
VAWT, Component sizing, Analysis and design of wind farms, Optimal
\end{flushleft}


\begin{flushleft}
selection of layouts.
\end{flushleft}





\begin{flushleft}
MCS830 Independent Study
\end{flushleft}


\begin{flushleft}
3 Credits (0-3-0)
\end{flushleft}


\begin{flushleft}
MCD831 Major Project Part-I
\end{flushleft}


\begin{flushleft}
6 Credits (0-0-12)
\end{flushleft}


\begin{flushleft}
MCD832 Major Project Part-II
\end{flushleft}


\begin{flushleft}
12 Credits (0-0-24)
\end{flushleft}


\begin{flushleft}
MCL834 Vibroacoustics
\end{flushleft}


\begin{flushleft}
3 Credits (2-0-2)
\end{flushleft}


\begin{flushleft}
Excitation of vibrations, Wave types in fluids and solids. Modes of
\end{flushleft}


\begin{flushleft}
vibrations in solids. The mobility and impedance concepts, for beams
\end{flushleft}


\begin{flushleft}
and plates. Wave/boundary matching. Radiation and transmission:
\end{flushleft}


\begin{flushleft}
Acoustical radiation from structures. Transmission between structures.
\end{flushleft}


\begin{flushleft}
Fluid structure interaction: Fundamentals of fluid structure interaction.
\end{flushleft}


\begin{flushleft}
Vibroacoustic Coupling: Effects of fluid-loading on vibrating infinite
\end{flushleft}


\begin{flushleft}
and finite plates and shells. Acoustic reflection from elastic plates and
\end{flushleft}


\begin{flushleft}
shells, acoustic excitation of elastic plates and coupling between panels
\end{flushleft}


\begin{flushleft}
and acoustic spaces. Prediction models, Sound transmission loss of
\end{flushleft}


\begin{flushleft}
structures. Enclosures: Acoustic fields in enclosures, low- and highmodal density fields, Sound-isolation techniques. Numerical models
\end{flushleft}


\begin{flushleft}
and analysis use of FEM, BEM. Vibroacoustic condition monitoring,
\end{flushleft}


\begin{flushleft}
Source identification and fault detection from noise and vibration
\end{flushleft}


\begin{flushleft}
signals in Mechanical systems such as bearings, gears, fans, blower
\end{flushleft}


\begin{flushleft}
and pumps, electrical equipment etc.
\end{flushleft}





\begin{flushleft}
MCL837 Advanced Mechanisms
\end{flushleft}


\begin{flushleft}
3 Credits (2-0-2)
\end{flushleft}


\begin{flushleft}
MCL838 Rotor Dynamics
\end{flushleft}


\begin{flushleft}
4 Credits (3-0-2)
\end{flushleft}


\begin{flushleft}
Importance of dynamics of rotors, issues involved in rotor vibration
\end{flushleft}


\begin{flushleft}
analysis, Rigid rotor and flexible rotor analysis, Lateral and Torsional
\end{flushleft}


\begin{flushleft}
vibration analysis, Response to steady state and transient excitations,
\end{flushleft}


\begin{flushleft}
bending critical speeds and response to unbalance for simple and
\end{flushleft}


\begin{flushleft}
complex rotor bearing system, orbital analysis and cascade plots,
\end{flushleft}


\begin{flushleft}
critical speed map, Campbell diagram. Disc gyroscopics, synchronous
\end{flushleft}


\begin{flushleft}
and nonsynchronous whirl, forward and backward whirl, Role of
\end{flushleft}


\begin{flushleft}
fluid film bearings and seals, analysis of rotors mounted on rolling
\end{flushleft}


\begin{flushleft}
element bearings, hydrodynamic bearings, two spool and multi-spool
\end{flushleft}


\begin{flushleft}
rotors, Dynamics of rotors with stiffness asymmetry, bend, crack and
\end{flushleft}


\begin{flushleft}
misalignment, etc.
\end{flushleft}


\begin{flushleft}
Parametric excitations, instabilities due to fluid film forces and
\end{flushleft}


\begin{flushleft}
hysteresis, influence of nonlinear supports. Balancing techniques,
\end{flushleft}


\begin{flushleft}
such as rigid rotor balancing, modal balancing, etc. Introduction to
\end{flushleft}


\begin{flushleft}
smart rotor systems. Use of finite element based approach for solving
\end{flushleft}


\begin{flushleft}
rotor dynamic problems.
\end{flushleft}


\begin{flushleft}
Application of vibration based condition monitoring, signal processing
\end{flushleft}


\begin{flushleft}
for rotor fault identification, application of expert systems for
\end{flushleft}


\begin{flushleft}
automated condition monitoring and rotor fault diagnosis, remote
\end{flushleft}


\begin{flushleft}
monitoring and other commercial systems.
\end{flushleft}


\begin{flushleft}
The course involves extensive coding in Matlab for dynamic response
\end{flushleft}


\begin{flushleft}
analysis of a general rotor bearing system. It involves exercises
\end{flushleft}


\begin{flushleft}
on vibration signal processing and analysis. The course involves
\end{flushleft}


\begin{flushleft}
experimentation on Machinery Fault Simulator.
\end{flushleft}





269





\begin{flushleft}
\newpage
Mechanical Engineering
\end{flushleft}





\begin{flushleft}
MCL840 Experimental Modal Analysis and Dynamic
\end{flushleft}


\begin{flushleft}
Design
\end{flushleft}


\begin{flushleft}
3 Credits (2-0-2)
\end{flushleft}





\begin{flushleft}
work on all aspects of system design for that theme topic. Theme topic
\end{flushleft}


\begin{flushleft}
can be parts of aircraft design, automotive design, robotic design,
\end{flushleft}


\begin{flushleft}
energy system, biomedical equipment etc.
\end{flushleft}





\begin{flushleft}
Introduction to modal testing. Dynamic test data measurement and
\end{flushleft}


\begin{flushleft}
processing methods including Laser vibrometery. Frequency response
\end{flushleft}


\begin{flushleft}
functions for multi-degree of- freedoms systems, forced response.
\end{flushleft}


\begin{flushleft}
Experimental and theoretical modal analysis - algorithms and codes.
\end{flushleft}


\begin{flushleft}
Applications of modal testing in system and force identification,
\end{flushleft}


\begin{flushleft}
structural dynamic modification, sensitivity analysis and frequency
\end{flushleft}


\begin{flushleft}
response coupling of substructure etc. Introduction to non-linear
\end{flushleft}


\begin{flushleft}
vibration analysis. Introduction to discrete systems and finite element
\end{flushleft}


\begin{flushleft}
modeling. Approaches to Joint modeling. Numerical model correlation;
\end{flushleft}


\begin{flushleft}
Introduction to FE model updating; Direct and iterative methods of FE
\end{flushleft}


\begin{flushleft}
model updating including optimization based updating; Brief idea of
\end{flushleft}


\begin{flushleft}
operational Modal Analysis, frequency and time domain methods of
\end{flushleft}


\begin{flushleft}
Operational Modal Analysis; Dynamic design of structures of products,
\end{flushleft}


\begin{flushleft}
machines and equipment via model testing, structural dynamic
\end{flushleft}


\begin{flushleft}
modification and FE model updating.
\end{flushleft}





\begin{flushleft}
MCV849 Special Module in Systems Design
\end{flushleft}


\begin{flushleft}
1 Credit (1-0-0)
\end{flushleft}





\begin{flushleft}
MCL845 Advanced Robotics
\end{flushleft}


\begin{flushleft}
3 Credits (2-0-2)
\end{flushleft}





\begin{flushleft}
Advanced Linear Programming and Simplex Method, Advanced
\end{flushleft}


\begin{flushleft}
Dynamic Programming, Integer Programming - Branch and Bound,
\end{flushleft}


\begin{flushleft}
Branch and Cut, Interior Point Methods, Selected Topics in Applied
\end{flushleft}


\begin{flushleft}
Probability, Applications.
\end{flushleft}





\begin{flushleft}
Review of different robotic systems; Types of wheeled mobile robots
\end{flushleft}


\begin{flushleft}
and walking machines; Jacobian; Forward and inverse kinematic
\end{flushleft}


\begin{flushleft}
algorithms; Non-recursive and recursive dynamic algorithms; Dynamics
\end{flushleft}


\begin{flushleft}
of mobile robots and walking machines; Kinematic design of robotic
\end{flushleft}


\begin{flushleft}
systems based on singularity, manipulability, etc.; Control of robots.
\end{flushleft}


\begin{flushleft}
Mechanical design of links.
\end{flushleft}





\begin{flushleft}
MCL848 Special topics in Systems Design-I
\end{flushleft}


\begin{flushleft}
2 Credits (2-0-0)
\end{flushleft}


\begin{flushleft}
Some sample topics are given below and depending upon the
\end{flushleft}


\begin{flushleft}
availability of experts, the department will decide the topic for a
\end{flushleft}


\begin{flushleft}
given semester:
\end{flushleft}


\begin{flushleft}
1.	Design of aircraft fuselage and wing
\end{flushleft}


\begin{flushleft}
2.	Engineering materials selection in mechanical design: Stages of
\end{flushleft}


\begin{flushleft}
design, stiffness/strength based design, damage tolerant design,
\end{flushleft}


\begin{flushleft}
life cycle design, systems approach to materials selection etc.
\end{flushleft}


\begin{flushleft}
3.	Tribo-systems design
\end{flushleft}


\begin{flushleft}
4.	Elecro-mechanical machines etc.
\end{flushleft}





\begin{flushleft}
Content will be decided by department based on the availability of
\end{flushleft}


\begin{flushleft}
experts in a particular semester. Topics will be related to systems
\end{flushleft}


\begin{flushleft}
design on the theme selected by the department.
\end{flushleft}





\begin{flushleft}
MCD861 M.Tech. Project Part-I
\end{flushleft}


\begin{flushleft}
12 Credits (0-0-24)
\end{flushleft}


\begin{flushleft}
MCD862 M.Tech. Project Part-II
\end{flushleft}


\begin{flushleft}
12 Credits (0-0-24)
\end{flushleft}


\begin{flushleft}
MCL865 Advanced Operations Research
\end{flushleft}


\begin{flushleft}
3 Credits (3-0-0)
\end{flushleft}





\begin{flushleft}
MCL866 Maintenance management
\end{flushleft}


\begin{flushleft}
3 Credits (3-0-0)
\end{flushleft}


\begin{flushleft}
Introduction to maintenance management, Reliability basics,
\end{flushleft}


\begin{flushleft}
Asset criticality Analysis, Reliability centered maintenance, Basic
\end{flushleft}


\begin{flushleft}
maintenance models for age and time based replacement, block
\end{flushleft}


\begin{flushleft}
and group replacement, inspection and shock based replacement,
\end{flushleft}


\begin{flushleft}
imperfect maintenance models, Maintainability models, Availability
\end{flushleft}


\begin{flushleft}
models, Life cycle cost models, Simulation based approach for
\end{flushleft}


\begin{flushleft}
maintenance planning, Queuing models for maintenance planning,
\end{flushleft}


\begin{flushleft}
Models for condition monitoring, Models for Maintenance scheduling,
\end{flushleft}


\begin{flushleft}
Maintenance performance measurement, Asset management
\end{flushleft}


\begin{flushleft}
practices, Case studies.
\end{flushleft}





\begin{flushleft}
MCD881 Major Project Part-I
\end{flushleft}


\begin{flushleft}
6 Credits (0-0-12)
\end{flushleft}


\begin{flushleft}
MCD882 Major Project Part-II
\end{flushleft}


\begin{flushleft}
12 Credits (0-0-24)
\end{flushleft}





\begin{flushleft}
MCL849 Special topics in Systems Design-II
\end{flushleft}


\begin{flushleft}
3 Credits (3-0-0)
\end{flushleft}


\begin{flushleft}
The course topic(s) will be decided by the department for each semester
\end{flushleft}


\begin{flushleft}
this course will be offered depending on the expertise available.
\end{flushleft}


\begin{flushleft}
Generally, each time a theme topic will be selected and students will
\end{flushleft}





\begin{flushleft}
MCD895 MS Research Project
\end{flushleft}


\begin{flushleft}
36 Credits (0-0-72)
\end{flushleft}





270





\begin{flushleft}
\newpage
Department of Physics
\end{flushleft}


\begin{flushleft}
PYL100 Electromagnetic Waves and Quantum Mechanics
\end{flushleft}


\begin{flushleft}
3 Credits (3-0-0)
\end{flushleft}


\begin{flushleft}
Electric and magnetic fields in a medium, Susceptibility and
\end{flushleft}


\begin{flushleft}
Conductivity, Maxwell's equations, Boundary conditions; EM wave
\end{flushleft}


\begin{flushleft}
equation, Plane wave solutions, Polarization of the EM waves, Poynting
\end{flushleft}


\begin{flushleft}
vector and intensity of the EM wave; Wave packet, Phase and Group
\end{flushleft}


\begin{flushleft}
velocities; Reflection and refraction of EM waves at a dielectric
\end{flushleft}


\begin{flushleft}
interface; Brewster angle; Total internal reflection at a dielectric
\end{flushleft}


\begin{flushleft}
interface; EM waves in a conducting medium and plasma.
\end{flushleft}


\begin{flushleft}
Wave-particle duality, de-Broglie waves; Quantum mechanical
\end{flushleft}


\begin{flushleft}
operators; Schroedinger equation, Wave function, Statistical
\end{flushleft}


\begin{flushleft}
interpretation, Superposition Principle, Continuity equation for
\end{flushleft}


\begin{flushleft}
probability density; Stationary states, Bound states, Free-particle
\end{flushleft}


\begin{flushleft}
solution, 1-D infinite potential well, Expectation values and uncertainty
\end{flushleft}


\begin{flushleft}
relations; 1-D finite potential well, Quantum mechanical tunneling and
\end{flushleft}


\begin{flushleft}
alpha-decay, Kronig-Penny model and emergence of bands.
\end{flushleft}





\begin{flushleft}
PYP100 Physics Laboratory
\end{flushleft}


\begin{flushleft}
2 Credits (0-0-4)
\end{flushleft}


\begin{flushleft}
Experiments based on Design and Study of Power sources, Charging
\end{flushleft}


\begin{flushleft}
and discharging of a capacitor, Electromagnetic Induction, Phase
\end{flushleft}


\begin{flushleft}
Measurement. Experiments on geometrical and wave optics including
\end{flushleft}


\begin{flushleft}
interference, diffraction, dispersion and polarization. Experiments
\end{flushleft}


\begin{flushleft}
based on mechanics, heat, sound, fluids, resonance, like linear
\end{flushleft}


\begin{flushleft}
air track, coupled pendulum and oscillators, thermal conductivity,
\end{flushleft}


\begin{flushleft}
elasticity. Experiments in the area of modern physics, like Planck's
\end{flushleft}


\begin{flushleft}
constant, lasers, semiconductor band gap, wave motion, mechanical
\end{flushleft}


\begin{flushleft}
transmission lines.
\end{flushleft}





\begin{flushleft}
PYL102 Principles of Electronic Materials
\end{flushleft}


\begin{flushleft}
3 Credits (3-0-0)
\end{flushleft}


\begin{flushleft}
Pre-requisites: PYL100
\end{flushleft}


\begin{flushleft}
(Program Linked Course: Not available to B.Tech. (Engineering
\end{flushleft}


\begin{flushleft}
Physics) students)
\end{flushleft}


\begin{flushleft}
Energy bands in solids(KP model), Classification of electronic
\end{flushleft}


\begin{flushleft}
materials: metals, semiconductors and insulators. Free electron
\end{flushleft}


\begin{flushleft}
model, Conductivity in metals and Concepts of Fermi level, effective
\end{flushleft}


\begin{flushleft}
mass and holes, Concept of phonons, Thermoelectricity, Intrinsic,
\end{flushleft}


\begin{flushleft}
extrinsic and degenerate semiconductors, Fermi level variation
\end{flushleft}


\begin{flushleft}
by carrier concentration and temperature, Metal-semiconductor
\end{flushleft}


\begin{flushleft}
junction, p-n junction, Diffusion and drift transport, carrier life
\end{flushleft}


\begin{flushleft}
time and diffusion length; Direct and indirect band gaps, Optical
\end{flushleft}


\begin{flushleft}
transitions, photon absorption, Exciton, photovoltaic effect,
\end{flushleft}


\begin{flushleft}
Dielectrics and electrical polarization, Depolarization field, ClausiusMossotti relation; Drude model, Electronic polarization and its
\end{flushleft}


\begin{flushleft}
mechanisms, Dielectric breakdown; Piezoelectricity, Pyroelectricity
\end{flushleft}


\begin{flushleft}
and Ferroelectricity; Magnetism in materials -- types of interactions,
\end{flushleft}


\begin{flushleft}
Magnetic susceptibility, Curie and Neel temperatures; Domains,
\end{flushleft}


\begin{flushleft}
Magnetic anisotropies, Spin-orbit interaction.
\end{flushleft}





\begin{flushleft}
PYL103 Physics of Nanomaterials
\end{flushleft}


\begin{flushleft}
3 Credits (3-0-0)
\end{flushleft}


\begin{flushleft}
Pre-requisites: PYL100
\end{flushleft}


\begin{flushleft}
(Program linked course: Not available to B.Tech. (Engineering
\end{flushleft}


\begin{flushleft}
Physics) students)
\end{flushleft}


\begin{flushleft}
Basics semiconductor concepts; Quantum dot, nanoparticle
\end{flushleft}


\begin{flushleft}
and clusters; critical size for low dimensional effects and magic
\end{flushleft}


\begin{flushleft}
numbers; Size induced modifications in band gap; Tight binding
\end{flushleft}


\begin{flushleft}
and effective mass approximations; Density of states of 0-D, 1-D,
\end{flushleft}


\begin{flushleft}
2-D, superlattice and monolayer structures; Quantum Hall effect;
\end{flushleft}


\begin{flushleft}
Thermoelectrical properties of nanostructured materials. Optical
\end{flushleft}


\begin{flushleft}
properties of bulk, metal nanoparticles, Core-shell nanoparticles;
\end{flushleft}


\begin{flushleft}
Size, shape and matrix effects; surface plasmon resonance; intrinsic
\end{flushleft}


\begin{flushleft}
and extrinsic effects; Applications of surface plasmon resonance in
\end{flushleft}


\begin{flushleft}
sensor devices; Magnetic properties of bulk and nanostructured
\end{flushleft}


\begin{flushleft}
materials; Single domain and multiple domain super paramagnetic
\end{flushleft}


\begin{flushleft}
phases; ZFC and FC measurements; Giant magnetoresistance effect.
\end{flushleft}


\begin{flushleft}
Chemical and Physical methods of synthesis of nano-particles and
\end{flushleft}





\begin{flushleft}
nano-structures, size selection methods; Measurements of size and
\end{flushleft}


\begin{flushleft}
its distribution; Characterization by AFM, STM and STS; Applications
\end{flushleft}


\begin{flushleft}
- Single electron effect and resonant tunneling devices. QW lasers,
\end{flushleft}


\begin{flushleft}
CNT and Graphene, GMR magnetic sensor, Nanostructured solar
\end{flushleft}


\begin{flushleft}
cell, Thermoelectric devices.
\end{flushleft}





\begin{flushleft}
PYL104 Advanced Mechanics and Thermodynamics
\end{flushleft}


\begin{flushleft}
3 Credits (3-0-0)
\end{flushleft}


\begin{flushleft}
Pre-requisites: PYL100
\end{flushleft}


\begin{flushleft}
(Program linked course: Not available to B.Tech. (Engineering
\end{flushleft}


\begin{flushleft}
Physics) students)
\end{flushleft}


\begin{flushleft}
Dynamics of a single and system of particles through energy
\end{flushleft}


\begin{flushleft}
approach, Hamilton's principle, the principle of least action,
\end{flushleft}


\begin{flushleft}
Canonical transformation, implication to mechanical systems,
\end{flushleft}


\begin{flushleft}
Poission, bracket, Concepts of phase space, Lioville's theorem,
\end{flushleft}


\begin{flushleft}
principle of stochastic cooling. Concepts of non linear dynamics,
\end{flushleft}


\begin{flushleft}
contraction of phase space volume, attractors, classical chaos,
\end{flushleft}


\begin{flushleft}
periodic motion, chaotic trajectories, bifurcations, driven damped
\end{flushleft}


\begin{flushleft}
harmonic oscillator, fractals and dimensionalities, various examples
\end{flushleft}


\begin{flushleft}
of nature, transition from discrete to continuous systems and fields.
\end{flushleft}


\begin{flushleft}
Laws of thermodynamics, Carnot's cycle, adiabatic and isothermal
\end{flushleft}


\begin{flushleft}
processes, principle of ideal engine and refrigeration, Definition
\end{flushleft}


\begin{flushleft}
of entropy, enthalpy, free energy, Maxwell's relations, Concepts
\end{flushleft}


\begin{flushleft}
of transport of heat and mass, Heat diffusion equations with and
\end{flushleft}


\begin{flushleft}
without heat source in one and three dimension based on thermal
\end{flushleft}


\begin{flushleft}
circuit concepts, Applications in nuclear reactors and Fourier
\end{flushleft}


\begin{flushleft}
transform based analysis of heat exchange process.
\end{flushleft}





\begin{flushleft}
PYL105 Optics and Lasers
\end{flushleft}


\begin{flushleft}
3 Credits (3-0-0)
\end{flushleft}


\begin{flushleft}
Pre-requisites: PYL100
\end{flushleft}


\begin{flushleft}
(Program linked course: Not available to B.Tech. (Engineering
\end{flushleft}


\begin{flushleft}
Physics) students)
\end{flushleft}


\begin{flushleft}
Basic optics: Interference and interferometers, phase change on
\end{flushleft}


\begin{flushleft}
reflection, anti-reflection film; Fresnel and Fraunhofer diffraction and
\end{flushleft}


\begin{flushleft}
examples, limit of resolution, diffraction grating, resolving power.
\end{flushleft}


\begin{flushleft}
Polarization optics, examples and applications. Lasers: Laser principles,
\end{flushleft}


\begin{flushleft}
interaction of radiation and matter, amplification and resonator
\end{flushleft}


\begin{flushleft}
conditions for laser oscillation, modes of laser, some laser systems
\end{flushleft}


\begin{flushleft}
and applications. Fiber optics: Light propagation in optical fibers, fiber
\end{flushleft}


\begin{flushleft}
communication, attenuation and dispersion, single and multi-mode
\end{flushleft}


\begin{flushleft}
fibers, fiber amplifiers and lasers, fiber optic sensors. Fourier Optics
\end{flushleft}


\begin{flushleft}
and Holography: Basics of Fourier transformation, spatial frequency,
\end{flushleft}


\begin{flushleft}
spatial filtering and some applications; Holographic principles, on-axis
\end{flushleft}


\begin{flushleft}
and off-axis holograms, types of holograms and some applications.
\end{flushleft}





\begin{flushleft}
PYL111 Electrodynamics
\end{flushleft}


\begin{flushleft}
4 Credits (3-1-0)
\end{flushleft}


\begin{flushleft}
Electrostatics and magnetostatics. Laplace and Poisson equations
\end{flushleft}


\begin{flushleft}
(solution), method of images. Multipole expansion. Maxwell's
\end{flushleft}


\begin{flushleft}
equations. Wave equation. Frequency dependence of permittivity.
\end{flushleft}


\begin{flushleft}
Absorption and dispersion. Kramers-Kronig relations. Conservation
\end{flushleft}


\begin{flushleft}
laws: Continuity equation, Poynting theorem, stress-energy tensor
\end{flushleft}


\begin{flushleft}
and Conservation of momentum. Solutions of Maxwell's equations in
\end{flushleft}


\begin{flushleft}
terms of potentials. Gauge transformations. Continuous distribution
\end{flushleft}


\begin{flushleft}
and retarded potentials. Lienard-Wiechert potentials. Field of moving
\end{flushleft}


\begin{flushleft}
point charge. Radiation, Electric dipole radiation, magnetic dipole
\end{flushleft}


\begin{flushleft}
radiation, Radiation from an arbitrary source. Power radiated by a
\end{flushleft}


\begin{flushleft}
point charge. Radiation reaction. Four vectors, Transformations of
\end{flushleft}


\begin{flushleft}
four vectors and tensors under Lorentz transformations. Formulation
\end{flushleft}


\begin{flushleft}
of Maxwell's equations in relativistic notations. Transformations of
\end{flushleft}


\begin{flushleft}
electric and the magnetic field vectors. Magnetism as a relativistic
\end{flushleft}


\begin{flushleft}
phenomenon. Lagrangian formulation of the electromagnetic field
\end{flushleft}


\begin{flushleft}
equations. Euler-Lagrange equations.
\end{flushleft}





\begin{flushleft}
PYP111 Engineering Physics Laboratory-I
\end{flushleft}


\begin{flushleft}
3 Credits (0-0-6)
\end{flushleft}


\begin{flushleft}
Experiments with various Lasers, Optical spectrometer, Microwaves,
\end{flushleft}


\begin{flushleft}
Fundamentals of Quantum Mechanics, Atomic spectroscopy and Tunneling.
\end{flushleft}





271





\begin{flushleft}
\newpage
Physics
\end{flushleft}





\begin{flushleft}
PYL112 Quantum Mechanics
\end{flushleft}


\begin{flushleft}
4 Credits (3-1-0)
\end{flushleft}


\begin{flushleft}
Dirac's bra-ket algebra, projection operator. Matrix representation of
\end{flushleft}


\begin{flushleft}
vectors and operators. Reformulating postulates in bra-ket language,
\end{flushleft}


\begin{flushleft}
Examples. 1D harmonic oscillator, ladder operators and construction
\end{flushleft}


\begin{flushleft}
of the stationary state wave functions, number operator and its
\end{flushleft}


\begin{flushleft}
eigenstates. Quantum mechanics in 2 and 3 dimensions in Cartesian
\end{flushleft}


\begin{flushleft}
coordinates. Quantum theory of angular momentum, eigenvalues and
\end{flushleft}


\begin{flushleft}
eigenfunctions. Quantum theory of spin angular momentum, addition
\end{flushleft}


\begin{flushleft}
of angular momenta and Clebsch-Gordan coefficients. Schroedinger
\end{flushleft}


\begin{flushleft}
equation in spherical coordinates, Free particle solution and solutions
\end{flushleft}


\begin{flushleft}
for spherically symmetric potentials, Hydrogen atom. Many particle
\end{flushleft}


\begin{flushleft}
Schroedinger equation, independent particles and reduction to the
\end{flushleft}


\begin{flushleft}
system of single-particle equations. Identical particles, exchange
\end{flushleft}


\begin{flushleft}
symmetry and degeneracy, Pauli principle and its applications. EPR
\end{flushleft}


\begin{flushleft}
paradox, Entangled states, hidden variables, Bell's inequality.
\end{flushleft}





\begin{flushleft}
PYL113 Mathematical Physics
\end{flushleft}


\begin{flushleft}
4 Credits (3-1-0)
\end{flushleft}


\begin{flushleft}
Linear algebra, complex variables, partial differential equations, special
\end{flushleft}


\begin{flushleft}
functions, Fourier and Laplace transforms, integral equations, vector
\end{flushleft}


\begin{flushleft}
and tensor analysis, brief introduction to group theory.
\end{flushleft}





\begin{flushleft}
PYL114 Solid State Physics
\end{flushleft}


\begin{flushleft}
4 Credits (3-1-0)
\end{flushleft}


\begin{flushleft}
Crystal Structure, concepts of reciprocal lattice and Brillouin zones,
\end{flushleft}


\begin{flushleft}
Defects in Crystals, Phonons, Crystal Vibrations with monoatomic
\end{flushleft}


\begin{flushleft}
and diatomic basis, Phonon Heat Capacity: Density of states in
\end{flushleft}


\begin{flushleft}
one dimension, Debye and Einstein models, thermal expansion,
\end{flushleft}


\begin{flushleft}
Free Electron Fermi Gas, Effect of temperature on the Fermi-Dirac
\end{flushleft}


\begin{flushleft}
Distribution, E-k diagrams, Effective Mass, Nearly free electron model,
\end{flushleft}


\begin{flushleft}
Bloch function, Kronig Penny Model, Atomic origin of magnetism:
\end{flushleft}


\begin{flushleft}
Diamagnetism, Langevin theory of paramagnetism, Curie-Weiss
\end{flushleft}


\begin{flushleft}
Law, Pauli paramagnetism, Ferromagnetism, Weiss molecular theory,
\end{flushleft}


\begin{flushleft}
Ferromagnetic domains, magnetic anisotropy , Superconductivity,
\end{flushleft}


\begin{flushleft}
types of superconductors, Heat capacity, energy gap, Thermodynamics
\end{flushleft}


\begin{flushleft}
of the superconducting transition, London equation, coherence length,
\end{flushleft}


\begin{flushleft}
BCS theory of superconductivity (qualitative), Brief introduction to
\end{flushleft}


\begin{flushleft}
high temperature superconductors.
\end{flushleft}





\begin{flushleft}
PYL115 Applied Optics
\end{flushleft}


\begin{flushleft}
4 Credits (3-1-0)
\end{flushleft}


\begin{flushleft}
Geometrical and Wave Optics: Fermat's Principle, Solution of ray
\end{flushleft}


\begin{flushleft}
equation, and applications. Review of Maxwell's equations and
\end{flushleft}


\begin{flushleft}
propagation of e. m. waves, reflection and refraction, total internal
\end{flushleft}


\begin{flushleft}
reflection and evanescent waves. Surface plasmons, Meta-materials.
\end{flushleft}


\begin{flushleft}
Plane waves in anisotropic media, Wave refractive index, Uniaxial
\end{flushleft}


\begin{flushleft}
crystals, some polarization devices. Interference and Diffraction:
\end{flushleft}


\begin{flushleft}
Concept of Coherence, Interference by division of wavefront and
\end{flushleft}


\begin{flushleft}
division of amplitude; Stoke's relations; Non-reflecting films; Michelson
\end{flushleft}


\begin{flushleft}
interferometer; Fabry-Perot interferometer and etalon. Fraunhofer
\end{flushleft}


\begin{flushleft}
diffraction: Single slit, circular aperture; limit of resolution. Diffraction
\end{flushleft}


\begin{flushleft}
grating, Resolving power. Fresnel diffraction: Half-period zones and
\end{flushleft}


\begin{flushleft}
the zone plate. Diffraction of a Gaussian beam. Lasers and Fiber
\end{flushleft}


\begin{flushleft}
Optics: Interaction of radiation and matter, Einstein coefficients,
\end{flushleft}


\begin{flushleft}
condition for amplification. Optical resonators, Condition for laser
\end{flushleft}


\begin{flushleft}
oscillation. Some Laser Systems. Light propagation in optical fibers,
\end{flushleft}


\begin{flushleft}
Attenuation and dispersion; Single-mode fibers, material dispersion,
\end{flushleft}


\begin{flushleft}
Fiber amplifiers and lasers. Fiber optic sensors. Introduction to Fourier
\end{flushleft}


\begin{flushleft}
Optics and Holography
\end{flushleft}





\begin{flushleft}
PYL116 Elements of Materials Processing
\end{flushleft}


\begin{flushleft}
4 Credits (3-1-0)
\end{flushleft}


\begin{flushleft}
Fundamentals of thermodynamic and kinetic aspects during nucleation
\end{flushleft}


\begin{flushleft}
and growth processes, Film growth modes, 2-D growth, Epitaxy and
\end{flushleft}


\begin{flushleft}
lattice misfits, Molecular beam epitaxy, Basics of vacuum, plasma
\end{flushleft}


\begin{flushleft}
discharge and sputtering, importance for material growth, Energy
\end{flushleft}


\begin{flushleft}
enhanced processes for low temperature processing, Reactive
\end{flushleft}


\begin{flushleft}
sputtering, Ion-beam deposition, Pulsed Laser Deposition, Plasma
\end{flushleft}


\begin{flushleft}
etching, E-beam and Ion-beam patterning, Chemical Vapor Deposition,
\end{flushleft}


\begin{flushleft}
Chemical Bath Deposition and Electrodeposition, Chemical epitaxy,
\end{flushleft}





\begin{flushleft}
Need for Epitaxy and its role in semiconductor devices, quantum wells,
\end{flushleft}


\begin{flushleft}
superlattices and hybrid structures. Mechanisms for confined materials
\end{flushleft}


\begin{flushleft}
growth for 0-D, 1-D and 2-D architecture and other complex forms,
\end{flushleft}


\begin{flushleft}
Case studies of material design by taking examples from current and
\end{flushleft}


\begin{flushleft}
emerging aspects of technologies and applications.
\end{flushleft}





\begin{flushleft}
PYL202 Statistical Physics
\end{flushleft}


\begin{flushleft}
4 Credits (3-1-0)
\end{flushleft}


\begin{flushleft}
Pre-requisites: PYL112
\end{flushleft}


\begin{flushleft}
Elementary Probability Theory: Binomial, Poisson and Gaussian
\end{flushleft}


\begin{flushleft}
Distribution, random walk problem, central limit theorem and its
\end{flushleft}


\begin{flushleft}
significance, average and distributions; diffusion and Brownian motion
\end{flushleft}


\begin{flushleft}
and their relation to random walk problem; Macrostate and microstate,
\end{flushleft}


\begin{flushleft}
Postulates of Statistical Mechanics, rules of calculations through
\end{flushleft}


\begin{flushleft}
microcanonical, canonical and grand canonical ensembles; Derivation
\end{flushleft}


\begin{flushleft}
of the thermodynamic relations from the statistical mechanics ;
\end{flushleft}


\begin{flushleft}
Application to classical systems: Systems of ideal gas molecules,
\end{flushleft}


\begin{flushleft}
Maxwell Boltzmann velocity distribution, paramagnetism of non
\end{flushleft}


\begin{flushleft}
interacting spins; specific heat of solids; Concept of Thermodynamic
\end{flushleft}


\begin{flushleft}
stability and Phase Transition: Vander Waal equation of state,
\end{flushleft}


\begin{flushleft}
Ising model, critical exponents; Indistinguishability of particles
\end{flushleft}


\begin{flushleft}
and Quantum Statistical Mechanics; Bose Einstein and Fermi-Dirac
\end{flushleft}


\begin{flushleft}
distribution: Black Body radiation, Bose Einstein Condensation, Fermi
\end{flushleft}


\begin{flushleft}
level and electronic contribution to specific heat, White Dwarf stars
\end{flushleft}


\begin{flushleft}
and Chandrasekhar Limit.
\end{flushleft}





\begin{flushleft}
PYL203 Classical Mechanics \& Relativity
\end{flushleft}


\begin{flushleft}
4 Credits (3-1-0)
\end{flushleft}


\begin{flushleft}
Dynamics of a particle moving under Central Force, Canonical
\end{flushleft}


\begin{flushleft}
transformation and Poission bracket formulation, HamiltonJacobi's theory, Non inertial (rotating) frames of references,
\end{flushleft}


\begin{flushleft}
Relativistic Mechanics.
\end{flushleft}





\begin{flushleft}
PYL204 Computational Physics
\end{flushleft}


\begin{flushleft}
4 Credits (3-1-0)
\end{flushleft}


\begin{flushleft}
Pre-requisites: PYL113
\end{flushleft}


\begin{flushleft}
Introduction to the basic numerical tools, such as locating roots of
\end{flushleft}


\begin{flushleft}
equations, interpolation, numerical differentiation and integration,
\end{flushleft}


\begin{flushleft}
solutions of algebraic and differential equations, discrete Fourier
\end{flushleft}


\begin{flushleft}
transform, etc. Applications of Monte-Carlo simulations, optimization
\end{flushleft}


\begin{flushleft}
and variational methods etc. to problems of interest in multiple
\end{flushleft}


\begin{flushleft}
areas of Physics.
\end{flushleft}





\begin{flushleft}
PYP212 Engineering Physics Laboratory-II
\end{flushleft}


\begin{flushleft}
3 Credits (0-0-6)
\end{flushleft}


\begin{flushleft}
Pre-requisites: PYL115
\end{flushleft}


\begin{flushleft}
Characterisation of optoelectronic/semiconductor devices, Holography,
\end{flushleft}


\begin{flushleft}
Determination of characteristic parameters of different types of
\end{flushleft}


\begin{flushleft}
optical fibers, Applications of Fiber Optics: experiments related to
\end{flushleft}


\begin{flushleft}
communication and sensors.
\end{flushleft}





\begin{flushleft}
PYP221 Engineering Physics Laboratory-III
\end{flushleft}


\begin{flushleft}
4 Credits (0-0-8)
\end{flushleft}


\begin{flushleft}
Pre-requisites: PYL114
\end{flushleft}


\begin{flushleft}
Synthesis of thin films, multilayers, nanoparticles by physical and
\end{flushleft}


\begin{flushleft}
chemical vapor deposition techniques, phase diagrams, study of
\end{flushleft}


\begin{flushleft}
surface, design of thin film resistor and magnetic field sensor.
\end{flushleft}





\begin{flushleft}
PYP222 Engineering Physics Laboratory-IV
\end{flushleft}


\begin{flushleft}
4 Credits (0-0-8)
\end{flushleft}


\begin{flushleft}
Pre-requisites: PYL114
\end{flushleft}


\begin{flushleft}
Resistivity of metals and semiconductors, band gap, charge carrier
\end{flushleft}


\begin{flushleft}
density and mobilities of semiconductor, basics of junction diode and
\end{flushleft}


\begin{flushleft}
its characteristics in solar cell configuration, study of crystal structure,
\end{flushleft}


\begin{flushleft}
dielectric constant, specific heat and superconductivity.
\end{flushleft}





\begin{flushleft}
PYS 300 Independent Study
\end{flushleft}


\begin{flushleft}
3 Credits (0-3-0)
\end{flushleft}


\begin{flushleft}
The course details to be worked out by the faculty giving the course
\end{flushleft}


\begin{flushleft}
keeping in view the learning needs of the students.
\end{flushleft}





272





\begin{flushleft}
\newpage
Physics
\end{flushleft}





\begin{flushleft}
PYL 301 Vacuum Technology and Surface Science
\end{flushleft}


\begin{flushleft}
3 Credits (3-0-0)
\end{flushleft}


\begin{flushleft}
Need of Vacuum and basic concepts: Mean free path, Particle flux;
\end{flushleft}


\begin{flushleft}
Monolayer formation, Gas Flow regimes ; Gas release from Solids:
\end{flushleft}


\begin{flushleft}
Vaporization, Thermal Desorption, Permeation, Surface diffusion,
\end{flushleft}


\begin{flushleft}
Physisorption and Chemisorption; Measurement of Pressure: Gauges,
\end{flushleft}


\begin{flushleft}
Residual Gas Analyses; Production of Vacuum: Roughing - Rotary
\end{flushleft}


\begin{flushleft}
pumps, Oil free pumps; HV \& UHV - Turbomolecular pumps, Cryopumps,
\end{flushleft}


\begin{flushleft}
Getter and Sputter Ion pumps; Materials and components in vacuum;
\end{flushleft}


\begin{flushleft}
Bulk versus surface; Electronic properties of surfaces: Contact potential
\end{flushleft}


\begin{flushleft}
and work function, Surface Plasmons; Atomic motion: Surface lattice
\end{flushleft}


\begin{flushleft}
dynamics, Surface diffusion, Surface melting and chemisorption;
\end{flushleft}


\begin{flushleft}
Adsorption of atoms and molecules; Experimental techniques for
\end{flushleft}


\begin{flushleft}
surface analysis: XPS, AES, SEXAFS, TEM, SEM, STM, AFM and RHEED.
\end{flushleft}





\begin{flushleft}
PYL302 Nuclear Science and Engineering
\end{flushleft}


\begin{flushleft}
3 Credits (3-0-0)
\end{flushleft}


\begin{flushleft}
Pre-requisites: PYL112
\end{flushleft}


\begin{flushleft}
Introduction to nuclear structure, Radioactivity and applications,
\end{flushleft}


\begin{flushleft}
Nuclear detection and acceleration technology, Nuclear reactors
\end{flushleft}


\begin{flushleft}
engineering, Nuclear techniques for composition analysis, Nuclear
\end{flushleft}


\begin{flushleft}
radiation in biology.
\end{flushleft}





\begin{flushleft}
PYL303 Materials Science and Engineering
\end{flushleft}


\begin{flushleft}
3 Credits (3-0-0)
\end{flushleft}


\begin{flushleft}
Pre-requisites: PYL114
\end{flushleft}


\begin{flushleft}
Elementary materials science concepts, thermally activated
\end{flushleft}


\begin{flushleft}
processes, diffusion in solids, phase diagram of pure substances,
\end{flushleft}


\begin{flushleft}
Gibbs phase rule, binary isomorphous systems, the Lever rule, zone
\end{flushleft}


\begin{flushleft}
refining, homogeneous and heterogeneous nucleation, martensitic
\end{flushleft}


\begin{flushleft}
transformation \& spinodal decomposition, Temperature dependence of
\end{flushleft}


\begin{flushleft}
resistivity,Matthiessen's rule, TCR, Nordheim's rule, mixture rules and
\end{flushleft}


\begin{flushleft}
electrical switches, high frequency resistance of a conductor, thin metal
\end{flushleft}


\begin{flushleft}
films and integrated circuit inter-connections, thermoelectricity, seebeck,
\end{flushleft}


\begin{flushleft}
Thomson and Peltier effects, thermoelectric heating and refrigeration,
\end{flushleft}


\begin{flushleft}
thermoelectric generators, the figure of merit, Bonding characteristics
\end{flushleft}


\begin{flushleft}
and elastic modulii, anelasticity, thermoelasticity, anelasticity energy
\end{flushleft}


\begin{flushleft}
losses, viscoelastic deformation, displacement models, Corrosion
\end{flushleft}


\begin{flushleft}
and Degradation of Materials: Electrochemical considerations,
\end{flushleft}


\begin{flushleft}
corrosion rates and their prediction, passivity environmental effects,
\end{flushleft}


\begin{flushleft}
forms of corrosiion, corrosion environments, corrosion prevention,
\end{flushleft}


\begin{flushleft}
oxidation, protective and non-protective oxides, PB ratio, mechanisms
\end{flushleft}


\begin{flushleft}
of oxide growth, Materials Selection and Design Considerations.
\end{flushleft}





\begin{flushleft}
PYL304 Superconductivity and Applications
\end{flushleft}


\begin{flushleft}
3 Credits (3-0-0)
\end{flushleft}


\begin{flushleft}
Pre-requisites: PYL114
\end{flushleft}


\begin{flushleft}
Basic properties: zero resistance, perfect diamagnetism, difference
\end{flushleft}


\begin{flushleft}
from perfect conductors; Critical temperature, Basic Introduction to
\end{flushleft}


\begin{flushleft}
High Temperature superconductors, Meissner effect, London equations,
\end{flushleft}


\begin{flushleft}
penetration depth, flux quantization, critical current and critical magnetic
\end{flushleft}


\begin{flushleft}
field, Thermodynamics of superconducting state, Type I and Type II
\end{flushleft}


\begin{flushleft}
superconductors, BCS theory, electron pairs; coherence length; energy
\end{flushleft}


\begin{flushleft}
gap; Isotope effect, Ginzburg-Landau Theory, tunneling of electron
\end{flushleft}


\begin{flushleft}
in M/I/S, tunneling of electron pairs in S/I/S: DC and AC Josephson
\end{flushleft}


\begin{flushleft}
effect, Some applications: Electromagnet, SQUID, Oscillators, basics of
\end{flushleft}


\begin{flushleft}
superconducting electronics and superconducting quantum computing.
\end{flushleft}





\begin{flushleft}
PYL305 Engineering Applications of Plasmas
\end{flushleft}


\begin{flushleft}
3 Credits (3-0-0)
\end{flushleft}


\begin{flushleft}
Pre-requisites: PYL111
\end{flushleft}


\begin{flushleft}
Plasma processing of materials, surface cleaning, etching, power/
\end{flushleft}


\begin{flushleft}
fusion energy, coherent radiation generation, plasma processing of
\end{flushleft}


\begin{flushleft}
textiles, nitriding, surface modification, plasma based charged particle
\end{flushleft}


\begin{flushleft}
accelerators, Hall thrusters.
\end{flushleft}





\begin{flushleft}
PYL306 Microelectronic Devices
\end{flushleft}


\begin{flushleft}
3 Credits (3-0-0)
\end{flushleft}


\begin{flushleft}
Pre-requisites: PYL201
\end{flushleft}


\begin{flushleft}
Brief overview of semiconductor fundamentals; pn junction diode -
\end{flushleft}





\begin{flushleft}
energy-band diagrams, electrostatics, current - voltage relationship,
\end{flushleft}


\begin{flushleft}
junction-breakdown mechanisms. Metal-semiconductor contacts:
\end{flushleft}


\begin{flushleft}
Schottky barrier diode, C-V and I-V characteristics of Schottky diode;
\end{flushleft}


\begin{flushleft}
ohmic contacts in semiconductors. MOS structure: Accumulation,
\end{flushleft}


\begin{flushleft}
depletion and inversion modes of operation, charge - voltage and
\end{flushleft}


\begin{flushleft}
capacitance - voltage behaviour, threshold and flatband voltages,
\end{flushleft}


\begin{flushleft}
fixed oxide and interface charge effects. MOSFET: Output and transfer
\end{flushleft}


\begin{flushleft}
characteristics, I-V relations, nonideal effects, MOSFET scaling. BJT:
\end{flushleft}


\begin{flushleft}
BJT action, current gain factors, modes of operation, I-V characteristics
\end{flushleft}


\begin{flushleft}
of a BJT, non-ideal effects, cutoff frequency of a BJT.
\end{flushleft}





\begin{flushleft}
PYL311 Lasers
\end{flushleft}


\begin{flushleft}
3 Credits (3-0-0)
\end{flushleft}


\begin{flushleft}
Pre-requisites: PYL115
\end{flushleft}


\begin{flushleft}
Interaction of Radiation with Matter: Einstein coefficients; Line shape
\end{flushleft}


\begin{flushleft}
function, Line-broadening mechanisms, Condition for amplification by
\end{flushleft}


\begin{flushleft}
stimulated emission, the meta-stable state and laser action. 3-level and
\end{flushleft}


\begin{flushleft}
4-level pumping schemes. Laser Rate Equations: Two-, three- and fourlevel laser systems, condition for population inversion, gain saturation;
\end{flushleft}


\begin{flushleft}
Laser amplifiers; Rare earth doped fiber amplifiers. Optical Resonators:
\end{flushleft}


\begin{flushleft}
Modes of a rectangular cavity, Plane mirror resonators, spherical
\end{flushleft}


\begin{flushleft}
mirror resonators, ray paths in the resonator, stable and unstable
\end{flushleft}


\begin{flushleft}
resonators, resonator stability condition; ring resonators; Transverse
\end{flushleft}


\begin{flushleft}
modes of laser resonators. Gaussian beams in laser resonators. Laser
\end{flushleft}


\begin{flushleft}
Oscillation: Optical feedback, threshold condition, variation of laser
\end{flushleft}


\begin{flushleft}
power near threshold, optimum output coupling, Characteristics of the
\end{flushleft}


\begin{flushleft}
laser output, oscillation frequency, frequency pulling, hole burning and
\end{flushleft}


\begin{flushleft}
the Lamb dip; Mode selection, single-frequency lasers; Methods of
\end{flushleft}


\begin{flushleft}
pulsing lasers, Q-switching, mode-locking. Some Laser Systems: Ruby,
\end{flushleft}


\begin{flushleft}
Nd: YAG, He-Ne, CO2 and excimer lasers, Tunable lasers: Ti Sapphire
\end{flushleft}


\begin{flushleft}
and dye lasers, Fiber lasers, Semiconductor lasers;Laser safety.
\end{flushleft}





\begin{flushleft}
PYL312 Semiconductor Optoelectronics
\end{flushleft}


\begin{flushleft}
3 Credits (3-0-0)
\end{flushleft}


\begin{flushleft}
Pre-requisites: PYL201
\end{flushleft}


\begin{flushleft}
Energy bands in solids, density of states, occupation probability,
\end{flushleft}


\begin{flushleft}
Fermi level and quasi Fermi levels, p-n junctions, Semiconductor
\end{flushleft}


\begin{flushleft}
optoelectronic materials, bandgap modification, Heterostructures
\end{flushleft}


\begin{flushleft}
and Quantum Wells. Rates of emission and absorption, condition for
\end{flushleft}


\begin{flushleft}
amplification by stimulated emission, the laser amplifier. Semiconductor
\end{flushleft}


\begin{flushleft}
Photon Sources: Electroluminescence. The LED, Semiconductor Laser,
\end{flushleft}


\begin{flushleft}
Single-frequency lasers; DFB and DBR lasers, VCSEL; Quantum-well
\end{flushleft}


\begin{flushleft}
lasers and quantum cascade lasers. Laser diode arrays. Semiconductor
\end{flushleft}


\begin{flushleft}
optical amplifiers (SOA), Electro-absorption modulators based on FKE
\end{flushleft}


\begin{flushleft}
and QCSE. Semiconductor Photodetectors: Types of photodetectors,
\end{flushleft}


\begin{flushleft}
Photoconductors, Photodiodes, PIN diodes and APDs. Quantum well
\end{flushleft}


\begin{flushleft}
infrared photodetectors (QWIP); Noise in photodetection; Photonic
\end{flushleft}


\begin{flushleft}
integrated circuits (PICs).
\end{flushleft}





\begin{flushleft}
PYL313 Fourier Optics and Holography
\end{flushleft}


\begin{flushleft}
3 Credits (3-0-0)
\end{flushleft}


\begin{flushleft}
Pre-requisites: PYL115
\end{flushleft}


\begin{flushleft}
Signals and systems, Fourier transform (FT), FT theorems, sampling
\end{flushleft}


\begin{flushleft}
theorem, Space-bandwidth product; Review of diffraction theory:
\end{flushleft}


\begin{flushleft}
Fresnel-Kirchhoff formulation, Fresnel and Fraunhofer Diffraction and
\end{flushleft}


\begin{flushleft}
angular spectrum method, FT properties of lenses and image formation
\end{flushleft}


\begin{flushleft}
by a lens; Frequency response of a diffraction-limited system under
\end{flushleft}


\begin{flushleft}
coherent and incoherent illumination. Basics of holography, in-line and
\end{flushleft}


\begin{flushleft}
off-axis holography, plane and volume holograms, diffraction efficiency;
\end{flushleft}


\begin{flushleft}
Recording medium for holograms; Applications of holography: display,
\end{flushleft}


\begin{flushleft}
microscopy; memories, interferometry, NDT of engineering objects,
\end{flushleft}


\begin{flushleft}
Digital Holography etc.; Holographic optical elements. Analog optical
\end{flushleft}


\begin{flushleft}
information processing: Abbe-Porter experiment, phase contrast
\end{flushleft}


\begin{flushleft}
microscopy and other simple applications; Coherent image processing:
\end{flushleft}


\begin{flushleft}
vander Lugt filter; joint-transform correlator; pattern recognition,
\end{flushleft}


\begin{flushleft}
image restoration.
\end{flushleft}





\begin{flushleft}
PYL321 Low Dimensional Physics
\end{flushleft}


\begin{flushleft}
3 Credits (3-0-0)
\end{flushleft}


\begin{flushleft}
Pre-requisites: PYL201 (Only for students opting for Minor Area)
\end{flushleft}





273





\begin{flushleft}
\newpage
Physics
\end{flushleft}





\begin{flushleft}
Brief overview of band structure and density of states function for
\end{flushleft}


\begin{flushleft}
0D, 1D and 2D systems, band gap engineetring and semiconductor
\end{flushleft}


\begin{flushleft}
heterostructures. Quantum wells and their optical properties, multiple
\end{flushleft}


\begin{flushleft}
quantum wells and superlattices, Bloch oscillations. Two dimensional
\end{flushleft}


\begin{flushleft}
electron gas, modulation doped heterostructures, Quantum Hall effect.
\end{flushleft}


\begin{flushleft}
Quantum wires and nanowires, electronic transport, properties and
\end{flushleft}


\begin{flushleft}
applications. Quantum dots and their optical properties, Coulomb
\end{flushleft}


\begin{flushleft}
blockade. Device applications of low dimensional systems: Double
\end{flushleft}


\begin{flushleft}
heterostructure laser, quantum cascade laser, high electron mobility
\end{flushleft}


\begin{flushleft}
transistors. 2D materials: Graphene, topological insulators, WS2 /
\end{flushleft}


\begin{flushleft}
MoS2and their properties.
\end{flushleft}





\begin{flushleft}
PYL322 Nanoscale Fabrication
\end{flushleft}


\begin{flushleft}
3 Credits (3-0-0)
\end{flushleft}


\begin{flushleft}
Pre-requisites: PYL201
\end{flushleft}





\begin{flushleft}
PYL332 General Theory of Relativity \& Cosmology
\end{flushleft}


\begin{flushleft}
3 Credits (3-0-0)
\end{flushleft}


\begin{flushleft}
Pre-requisites: PYL203
\end{flushleft}


\begin{flushleft}
Revision of special relativity, Notations, Equivalence principle,
\end{flushleft}


\begin{flushleft}
Introduction to tensor calculus, Metric, Parallel transport, covariant
\end{flushleft}


\begin{flushleft}
derivative and Christoffel symbols, Geodesic, Riemann curvature
\end{flushleft}


\begin{flushleft}
tensor, Ricci tensor, Geodesic deviation equation, Stress-Energy
\end{flushleft}


\begin{flushleft}
tensor, Einstein equation, Meaning of Einstein equation, Schwarzschild
\end{flushleft}


\begin{flushleft}
solution, Trajectories in Schwarzschild space-time, Perihelion shift,
\end{flushleft}


\begin{flushleft}
Binary pulsars, Gravitational deflection of light, Gravitational lensing,
\end{flushleft}


\begin{flushleft}
Gravitational collapse, Black holes, Hawking Radiation, Gravitational
\end{flushleft}


\begin{flushleft}
waves, Cosmology: Models of the universe and the cosmological
\end{flushleft}


\begin{flushleft}
principle, Cosmological metrics, Types of universe, Robertson-Walker
\end{flushleft}


\begin{flushleft}
universes, Big Bang, Dark energy.
\end{flushleft}





\begin{flushleft}
Nucleation and growth, Basic principles involved in growth with
\end{flushleft}


\begin{flushleft}
controllable dimensions, Chemical and physical techniques for
\end{flushleft}


\begin{flushleft}
growth of nanoparticle, nanorod, ultrathin films, monolayer materials,
\end{flushleft}


\begin{flushleft}
multilayer structures, nanocomposite materials. Self organized
\end{flushleft}


\begin{flushleft}
growth on substrates and templates. Micro and nanoscale pattering
\end{flushleft}


\begin{flushleft}
techniques.
\end{flushleft}





\begin{flushleft}
PYD411 Project-I
\end{flushleft}


\begin{flushleft}
4 Credits (0-0-8)
\end{flushleft}





\begin{flushleft}
PYL323 Nanoscale Microscopy
\end{flushleft}


\begin{flushleft}
2 Credits (2-0-0)
\end{flushleft}


\begin{flushleft}
Pre-requisites: PYL201
\end{flushleft}





\begin{flushleft}
PYL411 Quantum Electronics
\end{flushleft}


\begin{flushleft}
3 Credits (3-0-0)
\end{flushleft}


\begin{flushleft}
Pre-requisites: PYL112
\end{flushleft}





\begin{flushleft}
Scanning probe microscopy such as scanning electron microscope,
\end{flushleft}


\begin{flushleft}
atomic force microscope, scanning electron micoscope. Transmission
\end{flushleft}


\begin{flushleft}
electron microscope with high resolution and near field optical
\end{flushleft}


\begin{flushleft}
microscopy.
\end{flushleft}





\begin{flushleft}
PYL324 Spectroscopy of Nanomaterials
\end{flushleft}


\begin{flushleft}
2 Credits (2-0-0)
\end{flushleft}


\begin{flushleft}
Pre-requisites: PYL201
\end{flushleft}


\begin{flushleft}
Absorption and Reflection spectroscopy, molecular spectroscopy
\end{flushleft}


\begin{flushleft}
fundamentals, band-gaps and quantum confinement effects,
\end{flushleft}


\begin{flushleft}
Photoluminescence and Electroluminescence spectroscopy: Origin of
\end{flushleft}


\begin{flushleft}
emissions, Infrared and Raman Spectroscopy: Vibration spectroscopy
\end{flushleft}


\begin{flushleft}
principles , Time-domain spectroscopy, Nonlinear optical spectroscopy,
\end{flushleft}


\begin{flushleft}
Single molecule single nanoparticle detection. X-Ray Diffraction:
\end{flushleft}


\begin{flushleft}
Overview of basics, Intensities of diffracted beams, structure of
\end{flushleft}


\begin{flushleft}
polycrystalline aggregates, determination of crystallite size. X-Ray
\end{flushleft}


\begin{flushleft}
Absorption Spectroscopy: Fundamentals, Qualitative analysis of
\end{flushleft}


\begin{flushleft}
XANES and EXAFS data. X-Ray Photoelectron Spectroscopy and Auger
\end{flushleft}


\begin{flushleft}
Electron Spectroscopy: Principles of the method, initial- and final-state
\end{flushleft}


\begin{flushleft}
effects, Applications and case studies using all techniques specific to
\end{flushleft}


\begin{flushleft}
nanomaterials, Introduction to synchrotron radiation and its application
\end{flushleft}


\begin{flushleft}
to study nanomaterials.
\end{flushleft}





\begin{flushleft}
To set the objectives, deliverables, work plan, logistics planning and
\end{flushleft}


\begin{flushleft}
milestones with discernible outputs, and to demonstrate the feasibility
\end{flushleft}


\begin{flushleft}
through some specific aspects of a project.
\end{flushleft}





\begin{flushleft}
Light propagation though anisotropic media, nonlinear effects,
\end{flushleft}


\begin{flushleft}
nonlinear polarization, Second harmonic generation, sum and
\end{flushleft}


\begin{flushleft}
difference frequency generation, parametric amplification, parametric
\end{flushleft}


\begin{flushleft}
fluorescence and oscillation, concept of quasi-phase matching;
\end{flushleft}


\begin{flushleft}
periodically poled materials and their applications. Third-order effects:
\end{flushleft}


\begin{flushleft}
self-phase modulations, temporal and spatial solitons, cross-phase
\end{flushleft}


\begin{flushleft}
modulation, stimulated Raman and Brilloun scattering, four-wave
\end{flushleft}


\begin{flushleft}
mixing, phase conjugation. Quantization of the electromagnetic field;
\end{flushleft}


\begin{flushleft}
number states, coherent states and their properties: squeezed states
\end{flushleft}


\begin{flushleft}
of light and their properties, application of optical parametric processes
\end{flushleft}


\begin{flushleft}
to generate squeezed states of light, entangled states and their
\end{flushleft}


\begin{flushleft}
properties; Generation of entangled states; Quantum eraser, Ghost
\end{flushleft}


\begin{flushleft}
interference effects; Applications in quantum information science.
\end{flushleft}


\begin{flushleft}
Ultra-intense laser-matter interactions.
\end{flushleft}





\begin{flushleft}
PYD412 Project-II
\end{flushleft}


\begin{flushleft}
8 Credits (0-0-16)
\end{flushleft}


\begin{flushleft}
Pre-requisites: PYD 411
\end{flushleft}


\begin{flushleft}
Open to only those students opting for Departmental Specialization.
\end{flushleft}


\begin{flushleft}
The Project can be a continuation of the project undertaken for PYD
\end{flushleft}


\begin{flushleft}
411. The students will be eligible to do this project, if he/she secures
\end{flushleft}


\begin{flushleft}
a grade not below B in PYD411
\end{flushleft}





\begin{flushleft}
PYL412 Ultrafast Laser Systems and Applications
\end{flushleft}


\begin{flushleft}
3 Credits (3-0-0)
\end{flushleft}


\begin{flushleft}
Pre-requisites: PYL311
\end{flushleft}





\begin{flushleft}
PYL331 Applied Quantum Mechanics
\end{flushleft}


\begin{flushleft}
3 Credits (3-0-0)
\end{flushleft}


\begin{flushleft}
Pre-requisites: PYL112
\end{flushleft}


\begin{flushleft}
Electron in a magnetic field, Landau levels, Quantum Hall effect,
\end{flushleft}


\begin{flushleft}
Aharonov-Bohm effect. Non-degenerate and Degenerate Timeindependent perturbation theory, Examples: Stark effect, Atomic
\end{flushleft}


\begin{flushleft}
fine-structure, Atomic Hyperfine-structure, Zeeman Effect. Variational
\end{flushleft}


\begin{flushleft}
method, Examples, WKB Approximation, Examples and comparison.
\end{flushleft}


\begin{flushleft}
Time-dependent Perturbation theory, Examples, Fermi Golden Rule.
\end{flushleft}


\begin{flushleft}
Interaction of radiation with matter: Absorption and emission of
\end{flushleft}


\begin{flushleft}
radiation, Selection rules. Scattering theory: Scattering amplitude,
\end{flushleft}


\begin{flushleft}
Differential and total cross-sections, Born's Approximation, Scattering
\end{flushleft}


\begin{flushleft}
by spherically symmetric potentials, Examples, Rutherford's formula
\end{flushleft}


\begin{flushleft}
for Coulomb scattering, Partial wave analysis and Optical theorem,
\end{flushleft}


\begin{flushleft}
Examples. Relativistic Quantum Mechanics: Klein-Gordon equation,
\end{flushleft}


\begin{flushleft}
Properties of the free-particle KG equation including negative energy
\end{flushleft}


\begin{flushleft}
solutions. Dirac equation: The Dirac matrices and Dirac algebra. Spin
\end{flushleft}


\begin{flushleft}
of the Dirac particle. Dirac particle in an electromagnetic field, including
\end{flushleft}


\begin{flushleft}
the Pauli equation, magnetic moment and the g-factor, Free particle
\end{flushleft}


\begin{flushleft}
plane wave solutions, including negative and positive energy solutions.
\end{flushleft}





\begin{flushleft}
Review of Laser Physics: Gain media, laser oscillation, spectral
\end{flushleft}


\begin{flushleft}
line broadening, mode selection, Q-switching and mode-locking.
\end{flushleft}


\begin{flushleft}
Generation of Ultrashort Pulses: Temporal, spectral and spatial
\end{flushleft}


\begin{flushleft}
properties of pulses, Group velocity dispersion, Self-phase
\end{flushleft}


\begin{flushleft}
modulation; Pulse chirping, broadening and compression; Optical
\end{flushleft}


\begin{flushleft}
solitons, Chirp filters; High repetition-rate, high-energy few-cycle
\end{flushleft}


\begin{flushleft}
pulses. Measurement of Ultrashort Pulses: Optical and electronic
\end{flushleft}


\begin{flushleft}
pulse profiling; Intensity autocorrelation; Spectral measurement and
\end{flushleft}


\begin{flushleft}
frequency gating, FROG; Spectral interferometry, SPIDER. Ultrafast
\end{flushleft}


\begin{flushleft}
Optical Processes: Higher harmonic generation, Supercontinuum
\end{flushleft}


\begin{flushleft}
generation, Attosecond generation, Ultra-wideband optical parametric
\end{flushleft}


\begin{flushleft}
amplification. Femtosecond Laser Systems: Solid-state laser and fiber
\end{flushleft}


\begin{flushleft}
laser based systems, next-generation mid-IR lasers. Ultrafast Laser
\end{flushleft}


\begin{flushleft}
Processing: Laser ablation and surface micro/nano-structuring, Laser
\end{flushleft}


\begin{flushleft}
inscription of photonic devices, fabrication of optical waveguides and
\end{flushleft}


\begin{flushleft}
micro-fluidic chips. Ultrafast Spectroscopy: Transient absorption and
\end{flushleft}


\begin{flushleft}
emission spectroscopy, Terahertz spectroscopy; Femtosecond optical
\end{flushleft}


\begin{flushleft}
frequency combs and their applications.
\end{flushleft}





274





\begin{flushleft}
\newpage
Physics
\end{flushleft}





\begin{flushleft}
PYL413 Fiber and Integrated Optics
\end{flushleft}


\begin{flushleft}
3 Credits (3-0-0)
\end{flushleft}


\begin{flushleft}
Pre-requisites: PYL115
\end{flushleft}


\begin{flushleft}
Modes in planar optical waveguides: TE and TM modes, Modes
\end{flushleft}


\begin{flushleft}
in channel waveguides: Effective index and Perturbation method.
\end{flushleft}


\begin{flushleft}
Directional coupler: coupled mode theory, Integrated Optical
\end{flushleft}


\begin{flushleft}
devices: Prism Coupling, optical switching and wavelength filtering
\end{flushleft}


\begin{flushleft}
etc. Step Index and graded index fibers, Attenuation in optical
\end{flushleft}


\begin{flushleft}
fibers, LP Guided Modes of a step-index fiber, Single-mode fibers,
\end{flushleft}


\begin{flushleft}
Gaussian approximation and splice loss. Pulse dispersion, Dispersion
\end{flushleft}


\begin{flushleft}
compensation, Basics of Optical Communication Systems, and recent
\end{flushleft}


\begin{flushleft}
trends. Fiber fabrication technology and fiber characterization. Periodic
\end{flushleft}


\begin{flushleft}
interaction in waveguides: Coupled Mode Theory, Fiber Bragg Gratings,
\end{flushleft}


\begin{flushleft}
Long period Gratings and applications, Optical fiber sensors: basic
\end{flushleft}


\begin{flushleft}
principles and applications.
\end{flushleft}





\begin{flushleft}
PYD414 Project III
\end{flushleft}


\begin{flushleft}
4 Credits (0-0-8)
\end{flushleft}


\begin{flushleft}
Pre-requisites: PYD411
\end{flushleft}


\begin{flushleft}
Working out the detailed work plan and implementation of the project.
\end{flushleft}


\begin{flushleft}
The Project can be a continunation of the project undertaken for PYD 411.
\end{flushleft}





\begin{flushleft}
PYL414 Engineering Optics
\end{flushleft}


\begin{flushleft}
3 Credits (3-0-0)
\end{flushleft}


\begin{flushleft}
Pre-requisites: PYL115
\end{flushleft}





\begin{flushleft}
anisotropies and exchange bias, Spin valves with AF and SAF layers,
\end{flushleft}


\begin{flushleft}
Magnetization switching in AF and SAF layers, Magnetic domains and
\end{flushleft}


\begin{flushleft}
domain walls, single domain nano-particles; Pure spin and chage
\end{flushleft}


\begin{flushleft}
currents, spin-Hall effect and inverse spin-Hall effect, spin Seebeck
\end{flushleft}


\begin{flushleft}
effect, magneto-caloric effect, generation of spin current by charge
\end{flushleft}


\begin{flushleft}
and thermal current; Current induced magnetization switching, Spin
\end{flushleft}


\begin{flushleft}
torque effect and spin torque oscillators of tunable GHz frequency;
\end{flushleft}


\begin{flushleft}
High density data storage: MRAM, two stable states, half-select
\end{flushleft}


\begin{flushleft}
problem, Savtchenko switching and Toggle MRAM; Ultra high density
\end{flushleft}


\begin{flushleft}
devices: Current \& STT driven DW motion, Race track memory, Shift
\end{flushleft}


\begin{flushleft}
resistor; Q-bits and spin logic.
\end{flushleft}





\begin{flushleft}
PYL423 Nanoscale Energy Materials and Devices
\end{flushleft}


\begin{flushleft}
3 Credits(3-0-0)
\end{flushleft}


\begin{flushleft}
Pre-requisites: PYL201
\end{flushleft}


\begin{flushleft}
Basics of photovoltaics, Quantum confinement and plasmonics in
\end{flushleft}


\begin{flushleft}
photovoltaic devices, Nanorod solar cells, Principle of operation of
\end{flushleft}


\begin{flushleft}
hybrid and dye-sensitized solar cells, Nanoscale materials for improving
\end{flushleft}


\begin{flushleft}
thermoelectric figure of merit, Photoelectrochemical cells.
\end{flushleft}





\begin{flushleft}
PYV428 Selected Topics in Nanotechnology
\end{flushleft}


\begin{flushleft}
Pre-requisites: PYL201
\end{flushleft}


\begin{flushleft}
2 Credits (2-0-0)
\end{flushleft}


\begin{flushleft}
Topics from the emerging areas of Nanotechnology will form the
\end{flushleft}


\begin{flushleft}
basics and the faculty offering the course will provide the detailed
\end{flushleft}


\begin{flushleft}
course contents.
\end{flushleft}





\begin{flushleft}
Lens systems and basic concepts in their design; Optical components:
\end{flushleft}


\begin{flushleft}
Mirrors, prisms, gratings and filters; Sources, detectors and their
\end{flushleft}


\begin{flushleft}
characteristics; Optical systems: Telescopes, microscopes, projection
\end{flushleft}


\begin{flushleft}
systems, photographic systems, interferometers and spectrometers;
\end{flushleft}


\begin{flushleft}
Concepts in design of optical systems; Applications in industry,
\end{flushleft}


\begin{flushleft}
defense, space and medicine; CCD, compact disc, scanner, laser
\end{flushleft}


\begin{flushleft}
printer, photocopy, laser shows, satellite cameras, IR imagers, LCD,
\end{flushleft}


\begin{flushleft}
Spatial Light modulators.
\end{flushleft}





\begin{flushleft}
PYV418 Selected Topics in Photonics
\end{flushleft}


\begin{flushleft}
2 Credits (2-0-0)
\end{flushleft}


\begin{flushleft}
Pre-requisites: PYL115
\end{flushleft}


\begin{flushleft}
Topics from the emerging areas of Photonics will form the basics,
\end{flushleft}


\begin{flushleft}
and the faculty offering the course will provide the detailed
\end{flushleft}


\begin{flushleft}
course contents.
\end{flushleft}





\begin{flushleft}
PYV419 Special Topics in Photonics
\end{flushleft}


\begin{flushleft}
1 Credit (1-0-0)
\end{flushleft}


\begin{flushleft}
Pre-requisites: PYL115
\end{flushleft}


\begin{flushleft}
Topics from the emerging areas of Photonics will form the basics,
\end{flushleft}


\begin{flushleft}
and the faculty offering the course will provide the detailed
\end{flushleft}


\begin{flushleft}
course contents.
\end{flushleft}





\begin{flushleft}
PYL421 Functional Nanostructures
\end{flushleft}


\begin{flushleft}
3 Credits (3-0-0)
\end{flushleft}


\begin{flushleft}
Pre-requisites: PYL201
\end{flushleft}


\begin{flushleft}
Basics of low dimensional structures, QD, QW, nanostructures for
\end{flushleft}


\begin{flushleft}
optical and electronic applications, QD lasers, detectors, SET, Carbon
\end{flushleft}


\begin{flushleft}
based nanostructures, CNT, CNT optical, electrical, mechanical,
\end{flushleft}


\begin{flushleft}
chemical properties, sensors, drug delivery, photonic crystals, GMR,
\end{flushleft}


\begin{flushleft}
nanostructured magnetism, hydrogen storage, nanoclays, colloids,
\end{flushleft}


\begin{flushleft}
nanomachines, organic and biological nanostructures.
\end{flushleft}





\begin{flushleft}
PYL422 Spintronics
\end{flushleft}


\begin{flushleft}
3 Credits (3-0-0)
\end{flushleft}


\begin{flushleft}
Pre-requisites: PYL112
\end{flushleft}


\begin{flushleft}
Spintronics, its need and future vision; Basics of magnetic materials,
\end{flushleft}


\begin{flushleft}
spin orbit interaction, spin polarized current and their injection,
\end{flushleft}


\begin{flushleft}
accumulation and detection, Magnetoresistance and concepts of
\end{flushleft}


\begin{flushleft}
spin detection and magnetic memory; Spin valves \& GMR, CIP and
\end{flushleft}


\begin{flushleft}
CPP transport, Semiclassical transport models; Basics of spin valve
\end{flushleft}


\begin{flushleft}
and magnetic tunnel junctions, Tunnel magneto resistance, Quantum
\end{flushleft}


\begin{flushleft}
mechanical model of coherent tunneling and Giant TMR; Magnetic
\end{flushleft}





\begin{flushleft}
PYV429 Special Topics in Nanotechnology
\end{flushleft}


\begin{flushleft}
1 Credit (1-0-0)
\end{flushleft}


\begin{flushleft}
Topics from the emerging areas of nanotechnology will form the
\end{flushleft}


\begin{flushleft}
basics and the faculty offering the course will provide the detailed
\end{flushleft}


\begin{flushleft}
course contents.
\end{flushleft}





\begin{flushleft}
PYL431 Relativistic Quantum Mechanics
\end{flushleft}


\begin{flushleft}
2 Credits (2-0-0)
\end{flushleft}


\begin{flushleft}
Pre-requisites: PYL331
\end{flushleft}


\begin{flushleft}
Revision of Lorentz transformations, relativistic notations, Lorentz
\end{flushleft}


\begin{flushleft}
group. The Klein-Gordon equation, negative and positive energy
\end{flushleft}


\begin{flushleft}
solutions. Charged spin-zero particle, Difficulties with K-G theory. The
\end{flushleft}


\begin{flushleft}
Dirac equation, Relativistic invariance, Relativistic invariance, spin
\end{flushleft}


\begin{flushleft}
and energy projection operators. Nonrelativistic limit, Pauli equation,
\end{flushleft}


\begin{flushleft}
Solutions and their properties. Dirac sea, Anti-particle, Klein paradox,
\end{flushleft}


\begin{flushleft}
Fodly-Wouthuysen representation. Hydrogen atom, Dirac electron in
\end{flushleft}


\begin{flushleft}
an electromagnetic field, Charge conjugation.
\end{flushleft}





\begin{flushleft}
PYL432 Quantum Electrodynamics
\end{flushleft}


\begin{flushleft}
3 Credits (3-0-0)
\end{flushleft}


\begin{flushleft}
Pre-requisites: PYL331
\end{flushleft}


\begin{flushleft}
Lagrangian formulation of classical field theory, Field equations,
\end{flushleft}


\begin{flushleft}
symmetries, Noether's theorem and conservation laws. Energymomentum tensor. Classical field equations: Neutral and charged
\end{flushleft}


\begin{flushleft}
scalar fields, Electromagnetic field, Dirac field, Momentum
\end{flushleft}


\begin{flushleft}
representation, Second quantization of the free fields, Interacting
\end{flushleft}


\begin{flushleft}
fields, interaction picture, Dyson-series,Feynman diagrams and
\end{flushleft}


\begin{flushleft}
Feynman rules for quantum electrodynamics. Wick's theorem.
\end{flushleft}


\begin{flushleft}
Cross-section and S-matrix, Moeller and Bhabha scattering, Compton
\end{flushleft}


\begin{flushleft}
scattering, photoelectric effect etc. Divergence, Renormalization
\end{flushleft}


\begin{flushleft}
technique, Mass and charge renormalization.
\end{flushleft}





\begin{flushleft}
PYL433 Introduction to Gauge Field Theories
\end{flushleft}


\begin{flushleft}
2 Credits (2-0-0)
\end{flushleft}


\begin{flushleft}
Pre-requisites: PYL111 \& PYL112
\end{flushleft}


\begin{flushleft}
Maxwell's equations and Gauge invariance,Quantum mechanics
\end{flushleft}


\begin{flushleft}
of a charged particle as a gauge theory,Vector potential as
\end{flushleft}


\begin{flushleft}
phase, Aharonov-Bohm Effect,Superconductivity and Magnetic
\end{flushleft}


\begin{flushleft}
flux quantization in superconductors, Introduction to continuous
\end{flushleft}


\begin{flushleft}
symmetry groups, U(1) and SU(2) symmetry groups,Classical field
\end{flushleft}


\begin{flushleft}
theories, Local gauge invariance and the gauge fields,Yang-Mills
\end{flushleft}


\begin{flushleft}
gauge theories,Spontaneous symmetry breaking, Goldstone bosons,
\end{flushleft}


\begin{flushleft}
Higgs machanism,Weinberg-Salam Model.
\end{flushleft}





275





\begin{flushleft}
\newpage
Physics
\end{flushleft}





\begin{flushleft}
PYL434 Particle Accelerators
\end{flushleft}


\begin{flushleft}
2 Credits (2-0-0)
\end{flushleft}


\begin{flushleft}
Pre-requisites: PYL111 \& PYL112
\end{flushleft}


\begin{flushleft}
Electrostatic and electromagnetic accelerators: Van de Graff, Tandem
\end{flushleft}


\begin{flushleft}
acceleration, Linear accelerators, Synchrocyclotron, Storage ring, Free
\end{flushleft}


\begin{flushleft}
electron laser, High energy colliders.
\end{flushleft}





\begin{flushleft}
PYV438 Selected Topics in Theoretical Physics
\end{flushleft}


\begin{flushleft}
2 Credits(2-0-0)
\end{flushleft}


\begin{flushleft}
Pre-requisites: PYL112
\end{flushleft}


\begin{flushleft}
Topics from the emerging areas of Theoretical Physics will form the
\end{flushleft}


\begin{flushleft}
basics, and the faculty offering the course will provide the detailed
\end{flushleft}


\begin{flushleft}
course contents.
\end{flushleft}





\begin{flushleft}
thermodynamics: macrostates, microstates, Gibb's paradox. Gibb's
\end{flushleft}


\begin{flushleft}
ensemble theory: phase space perspective, Liouville's theorem,
\end{flushleft}


\begin{flushleft}
microcanonical, canonical and grand canonical ensembles, partition
\end{flushleft}


\begin{flushleft}
function, calculations of physical properties of classical systems
\end{flushleft}


\begin{flushleft}
using ensemble approach, thermodynamic relations. Applications
\end{flushleft}


\begin{flushleft}
of ensemble theory, quantum statistical mechanics: density matrix
\end{flushleft}


\begin{flushleft}
approach, statistical mechanics of Bosons and Fermions, Bose-Einstein
\end{flushleft}


\begin{flushleft}
condensation, Pauli paramagnetism, Landau diamagnetism, quantum
\end{flushleft}


\begin{flushleft}
statistics of harmonic oscillators, non-ideal gases, virial expansion,
\end{flushleft}


\begin{flushleft}
brief introduction to phase transitions, critical phenomena, transfer
\end{flushleft}


\begin{flushleft}
matrix approach, application to 1-D Ising model.
\end{flushleft}





\begin{flushleft}
PYL560 Applied Optics
\end{flushleft}


\begin{flushleft}
4 Credits (3-1-0)
\end{flushleft}





\begin{flushleft}
Constraints, generalized coordinates, action principle, symmetries and
\end{flushleft}


\begin{flushleft}
conservation laws, Hamilton's equations, Poisson brackets, canonical
\end{flushleft}


\begin{flushleft}
transformations, central potentials, small oscillations, normal modes,
\end{flushleft}


\begin{flushleft}
rigid body dynamics.
\end{flushleft}





\begin{flushleft}
Electromagnetic waves in a medium: review of Maxwell's equations and
\end{flushleft}


\begin{flushleft}
propagation of electromagnetic waves, various states of polarization
\end{flushleft}


\begin{flushleft}
and their analysis. Anisotropic media, plane waves in anisotropic
\end{flushleft}


\begin{flushleft}
media, uniaxial crystals, some polarization devices. Diffraction: scalar
\end{flushleft}


\begin{flushleft}
waves, the diffraction integral, Fresnel and Fraunhofer diffraction,
\end{flushleft}


\begin{flushleft}
diffraction of a Gaussian beam, diffraction grating. Fourier optics
\end{flushleft}


\begin{flushleft}
and holography: spatial frequency and transmittance function,
\end{flushleft}


\begin{flushleft}
Fourier transform by diffraction and by lens, spatial-frequency
\end{flushleft}


\begin{flushleft}
filtering, phase-contrast microscope. Holography: on-axis and off-axis
\end{flushleft}


\begin{flushleft}
hologram recording and reconstruction, types of hologram and some
\end{flushleft}


\begin{flushleft}
applications. Coherence and Interferometry: Spatial and temporal
\end{flushleft}


\begin{flushleft}
coherence, fringe visibility, Michelson stellar interferometer, optical
\end{flushleft}


\begin{flushleft}
beats, multiple beam interference, Fourier transform spectroscopy.
\end{flushleft}


\begin{flushleft}
Guided wave optics: Modes of a planar waveguide, optical fibers:
\end{flushleft}


\begin{flushleft}
step-index and graded index fibers, waveguide theory and quantum
\end{flushleft}


\begin{flushleft}
mechanics, applications of optical fibers in communication and sensing.
\end{flushleft}





\begin{flushleft}
PYL552 Electrodynamics
\end{flushleft}


\begin{flushleft}
4 Credits (3-1-0)
\end{flushleft}





\begin{flushleft}
PYD561 Project-I
\end{flushleft}


\begin{flushleft}
3 Credits (0-0-6)
\end{flushleft}





\begin{flushleft}
PYV439 Special Topics in Theoretical Physics
\end{flushleft}


\begin{flushleft}
1 Credit (1-0-0)
\end{flushleft}


\begin{flushleft}
Pre-requisites: PYL112
\end{flushleft}


\begin{flushleft}
Topics from the emerging areas of Theoretical Physics will form the
\end{flushleft}


\begin{flushleft}
basics, and the faculty offering the course will provide the detailed
\end{flushleft}


\begin{flushleft}
course contents.
\end{flushleft}





\begin{flushleft}
PYL551 Classical Mechanics
\end{flushleft}


\begin{flushleft}
4 Credits (3-1-0)
\end{flushleft}





\begin{flushleft}
Electrostatics, conductors, dielectrics, magnetostatics, boundary value
\end{flushleft}


\begin{flushleft}
problems, time dependent fields, waves in a medium, relativistic
\end{flushleft}


\begin{flushleft}
formulation of Maxwell's equations, radiation from accelerating
\end{flushleft}


\begin{flushleft}
charges, scattering of electromagnetic waves.
\end{flushleft}





\begin{flushleft}
PYL553 Mathematical Physics
\end{flushleft}


\begin{flushleft}
4 Credits (3-1-0)
\end{flushleft}





\begin{flushleft}
PYP561 Laboratory-I
\end{flushleft}


\begin{flushleft}
4 Credits (0-0-8)
\end{flushleft}


\begin{flushleft}
PYD562 Project-II
\end{flushleft}


\begin{flushleft}
6 Credits (0-0-12)
\end{flushleft}





\begin{flushleft}
Linear Algebra, complex analysis, Fourier transform, Sturm-Liouville's
\end{flushleft}


\begin{flushleft}
theorem and orthogonal functions, Ordinary differential equations,
\end{flushleft}


\begin{flushleft}
Green Functions/old methods.
\end{flushleft}





\begin{flushleft}
PYL555 Quantum Mechanics-I
\end{flushleft}


\begin{flushleft}
4 Credits (3-1-0)
\end{flushleft}


\begin{flushleft}
Introduction, quantum mechanical wave function, Born interpretation,
\end{flushleft}


\begin{flushleft}
basic formalism (Dirac bra-ket formalism), state vectors, operators
\end{flushleft}


\begin{flushleft}
and their representation, review of one dimensional examples, one
\end{flushleft}


\begin{flushleft}
dimensional harmonic oscillator, creation and annihilation operators,
\end{flushleft}


\begin{flushleft}
Landau problem, symmetries in quantum mechanics, hydrogen atom,
\end{flushleft}


\begin{flushleft}
entanglement.
\end{flushleft}





\begin{flushleft}
PYL556 Quantum Mechanics-II
\end{flushleft}


\begin{flushleft}
3 Credits (3-0-0)
\end{flushleft}


\begin{flushleft}
Time independent perturbation theory, time dependent perturbation
\end{flushleft}


\begin{flushleft}
theory, cross-section, scattering theory, approximation techniques,
\end{flushleft}


\begin{flushleft}
identical particles, interaction of atoms with radiation, relativistic
\end{flushleft}


\begin{flushleft}
equations.
\end{flushleft}





\begin{flushleft}
PYL557 Electronics
\end{flushleft}


\begin{flushleft}
4 Credits (3-1-0)
\end{flushleft}


\begin{flushleft}
Basics of semiconductor devices such as diode, transistor, FET and
\end{flushleft}


\begin{flushleft}
MOSFET; BJT and FET based amplifiers, oscillators, switches, circuit
\end{flushleft}


\begin{flushleft}
analysis by hybrid and r-parameters, operational amplifiers and their
\end{flushleft}


\begin{flushleft}
applications, timer circuit, dc power supplies, filters and digital circuits,
\end{flushleft}


\begin{flushleft}
counters, registers, ADC, DAC and microprocessor.
\end{flushleft}





\begin{flushleft}
PYL558 Statistical Mechanics
\end{flushleft}


\begin{flushleft}
4 Credits (3-1-0)
\end{flushleft}


\begin{flushleft}
Introduction to statistical methods. Some basic notions of random
\end{flushleft}


\begin{flushleft}
walks, Poisson distribution, Gaussian distribution. Statistical basis for
\end{flushleft}





\begin{flushleft}
PYP562 Laboratory-II
\end{flushleft}


\begin{flushleft}
4 Credits (0-0-8)
\end{flushleft}


\begin{flushleft}
PYL563 Solid State Physics
\end{flushleft}


\begin{flushleft}
4 Credits (3-1-0)
\end{flushleft}


\begin{flushleft}
Crystal lattices, Reciprocal lattice, equivalence of Bragg and Laue
\end{flushleft}


\begin{flushleft}
formulations, Ewald Construction, bonding \& packing in crystals.
\end{flushleft}


\begin{flushleft}
Free electron theory: Drude and Sommerfield's model of conductivity.
\end{flushleft}


\begin{flushleft}
Electrons in a Periodic Potential, Bloch theorem in lattice and reciprocal
\end{flushleft}


\begin{flushleft}
space, origin of band gap in a weak periodic potential, Kronig-Penney
\end{flushleft}


\begin{flushleft}
model, band structures, metal, insulator, semiconductor, concepts
\end{flushleft}


\begin{flushleft}
of effective mass, light and heavy holes in a semiconductor, optical
\end{flushleft}


\begin{flushleft}
properties of semiconductors. Wannier functions, Tight binding model
\end{flushleft}


\begin{flushleft}
and calculation of band structure, Fermi Surfaces. Thermal Properties:
\end{flushleft}


\begin{flushleft}
classical \& quantum theory of harmonic crystal in one, two, \& three
\end{flushleft}


\begin{flushleft}
dimensions, specific heat at high and low temperatures, normal modes
\end{flushleft}


\begin{flushleft}
\& phonons, Einstein \& Debye models of specific heat. Ferroelectric,
\end{flushleft}


\begin{flushleft}
Piezoelectric Magnetism: Diamagnetism, Paramagnetism, Hunds Rule,
\end{flushleft}


\begin{flushleft}
Curie's Law, Cooling by Diamagnetism, Pauli Paramagnetism, CurieWeiss Law Ferromagnetism and Antiferromagnetic ordering, Domains.
\end{flushleft}


\begin{flushleft}
Superconductivity: Basic Phenomenology, Meissner effect, London
\end{flushleft}


\begin{flushleft}
penetration depth, coherence length, Flux quantization.
\end{flushleft}





\begin{flushleft}
PYP563 Advanced Laboratory
\end{flushleft}


\begin{flushleft}
4 Credits (0-0-8)
\end{flushleft}


\begin{flushleft}
PYL567 Atomic and Molecular Physics
\end{flushleft}


\begin{flushleft}
3 Credits (3-0-0)
\end{flushleft}


\begin{flushleft}
Hydrogen and alkali metals, double fine structure of atoms, two
\end{flushleft}





276





\begin{flushleft}
\newpage
Physics
\end{flushleft}





\begin{flushleft}
electron atom, Zeeman and Paschen-Back effect, X-ray spectra,
\end{flushleft}


\begin{flushleft}
general factors influencing spectral line width (Collision, Doppler effect,
\end{flushleft}


\begin{flushleft}
Heisenberg) and line intensities (transition probability, population
\end{flushleft}


\begin{flushleft}
of states, Beer- Lambert law), Molecular symmetry, irreducible
\end{flushleft}


\begin{flushleft}
representations, Rotational and vibrational spectra of diatomic
\end{flushleft}


\begin{flushleft}
molecules, FTIR and Laser Raman spectroscopy, electronic spectra,
\end{flushleft}


\begin{flushleft}
Franck-Condon principle, bond dissociation energies, Molecular orbital
\end{flushleft}


\begin{flushleft}
and models, laser cooling of atom.
\end{flushleft}





\begin{flushleft}
PYL569 Nuclear and Particle Physics
\end{flushleft}


\begin{flushleft}
3 Credits (3-0-0)
\end{flushleft}


\begin{flushleft}
N-N interaction, iso-spin symmetry, nuclear models, beta decay,
\end{flushleft}


\begin{flushleft}
detectors and particle accelerators, Quark model, Deep inelastic
\end{flushleft}


\begin{flushleft}
scattering, Basics of nuclear astrophysics, Fundamental particles
\end{flushleft}


\begin{flushleft}
and their properties.
\end{flushleft}





\begin{flushleft}
PYL650 Fiber and Integrated Optics
\end{flushleft}


\begin{flushleft}
3 Credits (3-0-0)
\end{flushleft}


\begin{flushleft}
Modes in planar optical waveguides: TE and TM modes. Modal
\end{flushleft}


\begin{flushleft}
analysis of a parabolic index medium. Modes in channel waveguides:
\end{flushleft}


\begin{flushleft}
Effective index method, Perturbation method and Variational method.
\end{flushleft}


\begin{flushleft}
Modes in multilayered waveguides: Matrix method. Directional
\end{flushleft}


\begin{flushleft}
coupler: coupled mode theory, Integrated Optical devices: Prism
\end{flushleft}


\begin{flushleft}
Coupling, optical switching, modulators and wavelength filters, etc.
\end{flushleft}


\begin{flushleft}
Step Index and graded index fibers, Attenuation in optical fibers, LP
\end{flushleft}


\begin{flushleft}
Guided Modes of a step-index fiber, Single-mode fibers, Gaussian
\end{flushleft}


\begin{flushleft}
approximation and splice losses. Dispersion in optical fibers, Pulse
\end{flushleft}


\begin{flushleft}
dispersion, Dispersion management. Fabrication and characterization
\end{flushleft}


\begin{flushleft}
of optical waveguides. Fiber optic components and devices. Optical
\end{flushleft}


\begin{flushleft}
fiber sensors; Basic principles and applications.
\end{flushleft}





\begin{flushleft}
PYL651 Advanced Solid State Physics
\end{flushleft}


\begin{flushleft}
3 Credits (3-0-0)
\end{flushleft}


\begin{flushleft}
Pre-requisites: PYL563
\end{flushleft}


\begin{flushleft}
Semiclassical model of electron dynamics, Electrons in static electric
\end{flushleft}


\begin{flushleft}
and magnetic fields, DC and AC electrical conductivity in metals,
\end{flushleft}


\begin{flushleft}
Sources of electron scattering, Boltzmann equation, Temperature
\end{flushleft}


\begin{flushleft}
dependence of electronic conductivity, Dielectric properties of
\end{flushleft}


\begin{flushleft}
insulators, Pizoelectric, Ferroelectric, Pyroelectric, Optical properties
\end{flushleft}


\begin{flushleft}
of solids, Electrons in magnetic fields, Landau Levels, Cylotron
\end{flushleft}


\begin{flushleft}
resonance, Density of states in magnetic field, De-Haas Van Alfen
\end{flushleft}


\begin{flushleft}
effect, Quantum Hall effect, Models for ferromagnetism, Magnetic
\end{flushleft}


\begin{flushleft}
phase transition, Properties of Superconductors, GinzburgLandau theory, Josephson effect, Squids Mircoscopic Theory of
\end{flushleft}


\begin{flushleft}
superconductivity: Cooper pairs, BCS theory.
\end{flushleft}





\begin{flushleft}
PYL652 Magnetism and Spintronics
\end{flushleft}


\begin{flushleft}
3 Credits (3-0-0)
\end{flushleft}


\begin{flushleft}
Pre-requisites: PYL563
\end{flushleft}


\begin{flushleft}
Magnetism of metals, Spontaneous spin split bands, Magnetic
\end{flushleft}


\begin{flushleft}
anisotropy, Competing interactions, One and two-dimensional
\end{flushleft}


\begin{flushleft}
magnets, Spin dependent transport in magnetic metals - Anisotropic
\end{flushleft}


\begin{flushleft}
magnetoresistance, Giant magnetoresistance, Spin dependent
\end{flushleft}


\begin{flushleft}
tunneling, Tunneling magnetoresistance, Spin-Orbit interaction and
\end{flushleft}


\begin{flushleft}
Hall effects --Spin Hall Effect and Inverse Spin Hall Effect; Spin injection
\end{flushleft}


\begin{flushleft}
phenomena - Spin Transfer Torque, Spin injection magnetization
\end{flushleft}


\begin{flushleft}
reversal; High frequency phenomena.
\end{flushleft}





\begin{flushleft}
PYL653 Semiconductor Electronics
\end{flushleft}


\begin{flushleft}
3 Credits (3-0-0)
\end{flushleft}


\begin{flushleft}
Pre-requisites: PYL563 or equivalent
\end{flushleft}


\begin{flushleft}
Semiconductors junction review; charge storage and transient
\end{flushleft}


\begin{flushleft}
behavior, equivalent circuit of diode, p-n hetero-structure: band
\end{flushleft}


\begin{flushleft}
discontinuity and its effect on junction properties; Junction
\end{flushleft}


\begin{flushleft}
breakdown mechanisms; Static characteristics of Bipolar transistor;
\end{flushleft}


\begin{flushleft}
Frequency response and switching behavior, Non-ideal effects:
\end{flushleft}


\begin{flushleft}
base width modulation, early effect, current crowding and high
\end{flushleft}


\begin{flushleft}
injection effect; Hetero-junction transistor; SCR, M-S junctions:
\end{flushleft}


\begin{flushleft}
Basic structure, Energy band relation, I-V characteristics; Ohmic
\end{flushleft}


\begin{flushleft}
contacts; MOS capacitors, JFET and MESFET basic principles,
\end{flushleft}





\begin{flushleft}
MOSFET: structure and operation, basic characteristics and
\end{flushleft}


\begin{flushleft}
analysis; linear quadratic model; equivalent circuit; Threshold
\end{flushleft}


\begin{flushleft}
voltage calculation; Substrate biasing effect; LED, Laser,
\end{flushleft}


\begin{flushleft}
Photodiode and solar cells, Tunnel, IMPATT \& Gunn diodes and
\end{flushleft}


\begin{flushleft}
comparison of microwave devices.
\end{flushleft}





\begin{flushleft}
PYL655 Laser Physics
\end{flushleft}


\begin{flushleft}
3 Credits (3-0-0)
\end{flushleft}


\begin{flushleft}
Pre-requisites: PYL560
\end{flushleft}


\begin{flushleft}
Introduction. Physics of interaction between Radiation and Atomic
\end{flushleft}


\begin{flushleft}
systems including: Stimulated emission, emission line shapes and
\end{flushleft}


\begin{flushleft}
dispersion effects. Gain saturation in laser media and theory of FabryPerot laser. Techniques for the control of laser output employing Qswitching, mode-locking and mode-dumping. Optical cavity design and
\end{flushleft}


\begin{flushleft}
laser stability criteria. Description of common types of conventional
\end{flushleft}


\begin{flushleft}
lasers. Physics of semiconducting optical materials, degenerate
\end{flushleft}


\begin{flushleft}
semiconductors and their Homo-junctions and Hetero-junctions. Light
\end{flushleft}


\begin{flushleft}
emitting diodes (LED's) junction lasers. Characteristics of diode laser
\end{flushleft}


\begin{flushleft}
arrays and applications.
\end{flushleft}





\begin{flushleft}
PYL656 Microwaves
\end{flushleft}


\begin{flushleft}
3 Credits (3-0-0)
\end{flushleft}


\begin{flushleft}
Pre-requisites: PYL552
\end{flushleft}


\begin{flushleft}
Maswell's equations, Wave equation, Boundary conditions, Ideal
\end{flushleft}


\begin{flushleft}
transmission line, Terminated line, Wave solutions, TEM, TE, and TM
\end{flushleft}


\begin{flushleft}
waves, Rectangular and circular wave guides, power and attenuation,
\end{flushleft}


\begin{flushleft}
Smith chart, Impedance matching, Double and triple stub tuners,
\end{flushleft}


\begin{flushleft}
Quarter wave and half wave transforms, Equivalent voltage and currents,
\end{flushleft}


\begin{flushleft}
Impedance description, Impedance, admittance and scattering matrix
\end{flushleft}


\begin{flushleft}
formulation, Signal flow graph, Attenuators, Phase shifters, Directional
\end{flushleft}


\begin{flushleft}
couplers, Junctions, Power dividers, Isolators and circulators,
\end{flushleft}


\begin{flushleft}
Resonant circuits, Transmission line resonators, Rectangular and
\end{flushleft}


\begin{flushleft}
circular wave guide resonators, Electron beams, Velocity modulation,
\end{flushleft}


\begin{flushleft}
Klystron, Magnetron, Traveling wave tubes, Gunn oscillator, Transistor
\end{flushleft}


\begin{flushleft}
and FET amplifiers, biasing, stability, power gain, noise, Mixers.
\end{flushleft}





\begin{flushleft}
PYL657 Plasma Physics
\end{flushleft}


\begin{flushleft}
3 Credits (3-0-0)
\end{flushleft}


\begin{flushleft}
Pre-requisites: PYL552
\end{flushleft}


\begin{flushleft}
Introduction to plasma, Debye shielding, Single particle motion in
\end{flushleft}


\begin{flushleft}
E and B fields, Mirror confinement, Plasma oscillations, Waves in
\end{flushleft}


\begin{flushleft}
unmagnetized plasmas, Solitons, Two stream instability, Rayleigh Taylor
\end{flushleft}


\begin{flushleft}
instability, Vlasov equation and Landau damping, Waves in magnetized
\end{flushleft}


\begin{flushleft}
plasmas (fluid theory), Plasma production \& characterization, Plasma
\end{flushleft}


\begin{flushleft}
processing of materials, Laser driven fusion, Cerenkov free electron
\end{flushleft}


\begin{flushleft}
laser, Applications to astrophysics and astronomy.
\end{flushleft}





\begin{flushleft}
PYD658 Mini Project
\end{flushleft}


\begin{flushleft}
3 Credits (0-0-6)
\end{flushleft}


\begin{flushleft}
PYL658 Advanced Plasma Physics
\end{flushleft}


\begin{flushleft}
3 Credits (3-0-0)
\end{flushleft}


\begin{flushleft}
Pre-requisites: PYL657
\end{flushleft}


\begin{flushleft}
Nonlinearity and dispersion, solitary waves and solitons, KortewegdeVries (kdv) equation, Electromagnetic (EM) radiation from free
\end{flushleft}


\begin{flushleft}
charges, Absorption of em waves in plasmas, Radiation by coulomb
\end{flushleft}


\begin{flushleft}
collisions, Plasma based Terahertz radiation generation, Hall thrusters,
\end{flushleft}


\begin{flushleft}
Rayleigh-Taylor instability, Resistive instability, Electron transport,
\end{flushleft}


\begin{flushleft}
Waveguide modes in the presence of plasma, Ponderomotive force,
\end{flushleft}


\begin{flushleft}
wakefield, Particle acceleration, Dusty plasma, Current flow in dust
\end{flushleft}


\begin{flushleft}
grains, Waves in dusty plasma.
\end{flushleft}





\begin{flushleft}
PYL659 Laser Spectroscopy
\end{flushleft}


\begin{flushleft}
3 Credits (3-0-0)
\end{flushleft}


\begin{flushleft}
Review of lasers as spectroscopic source, Absorption spectroscopy,
\end{flushleft}


\begin{flushleft}
High sensitive methods, Cavity ring down spectroscopy, Doppler limited
\end{flushleft}


\begin{flushleft}
spectroscopy: Photo-ionization and Photo-acoustic spectroscopy,
\end{flushleft}


\begin{flushleft}
Laser-induced breakdown spectroscopy (LIBS), Laser induced
\end{flushleft}





277





\begin{flushleft}
\newpage
Physics
\end{flushleft}





\begin{flushleft}
fluorescence spectroscopy, Nonlinear spectroscopy: Linear and
\end{flushleft}


\begin{flushleft}
nonlinear absorption, saturation spectroscopy, Two-photon and multiphoton spectroscopy, Laser Raman spectroscopy: Stimulated Raman
\end{flushleft}


\begin{flushleft}
spectroscopy, Coherent anti-Stokes Raman spectroscopy (CARS),
\end{flushleft}


\begin{flushleft}
Time-resolved spectroscopy: Short pulse generation and detection,
\end{flushleft}


\begin{flushleft}
Life time measurements, Pump-and-probe techniques, Time-resolved
\end{flushleft}


\begin{flushleft}
absorption, Fluorescence and Raman spectroscopy, Applications of
\end{flushleft}


\begin{flushleft}
laser spectroscopy: Single molecule detection, Trace level detection
\end{flushleft}


\begin{flushleft}
of explosives and hazardous gases, LIDAR.
\end{flushleft}





\begin{flushleft}
PYL701 Physical Foundations of Materials Science
\end{flushleft}


\begin{flushleft}
3 Credits (3-0-0)
\end{flushleft}


\begin{flushleft}
Overlap with : PYL303
\end{flushleft}


\begin{flushleft}
Imperfections in solids: Points defects, thermodynamics of point
\end{flushleft}


\begin{flushleft}
defects. Dislocations: Grain Boundaries: Low and high angle grain
\end{flushleft}


\begin{flushleft}
boundaries. Phase Transformations: Kinetics of phase transformations,
\end{flushleft}


\begin{flushleft}
homogeneous and heterogeneous nucleation, kinetic considerations
\end{flushleft}


\begin{flushleft}
of solid-state transformations. Diffusion: Diffusion Mechanisms,
\end{flushleft}


\begin{flushleft}
Steady and non-steady state diffusion, factors influencing diffusion.
\end{flushleft}


\begin{flushleft}
Phase Diagrams: Unary phase diagram, Gibbs Phase Rule, Binary
\end{flushleft}


\begin{flushleft}
Isomorphous Systems, Lever Rule, interpretation of phase diagrams,
\end{flushleft}


\begin{flushleft}
determination of phase amounts, Equilibrium and non-equilibrium
\end{flushleft}


\begin{flushleft}
solidification, Binary Eutectic Systems, Equilibrium Diagrams
\end{flushleft}


\begin{flushleft}
having intermediate phases or compounds, Eutectoid and Peritectic
\end{flushleft}


\begin{flushleft}
Reactions, Congruent Phase transformations, Ternary phase diagrams.
\end{flushleft}


\begin{flushleft}
Microstructural and Property changes in Iron-Carbon Alloys. Corrosion
\end{flushleft}


\begin{flushleft}
and degradation of materials: Electrochemical Corrosion of Metals,
\end{flushleft}


\begin{flushleft}
Galvanic cells, Corrosion rates, Corrosion reactions, passivation, types
\end{flushleft}


\begin{flushleft}
of corrosion, Mechanisms of oxidation, oxidation rates, corrosion
\end{flushleft}


\begin{flushleft}
control. Materials Selection and Design Considerations.
\end{flushleft}





\begin{flushleft}
PYP701 Solid State Materials Laboratory-I
\end{flushleft}


\begin{flushleft}
3 Credits (0-0-6)
\end{flushleft}


\begin{flushleft}
This laboratory course is designed to make the students familiar with
\end{flushleft}


\begin{flushleft}
fundamental experiments related with materials synthesis and their
\end{flushleft}


\begin{flushleft}
primary characterization. Experiments are based on materials synthesis
\end{flushleft}


\begin{flushleft}
by solid-state reaction route, Spray-pyrolysis, spin and dip coating,
\end{flushleft}


\begin{flushleft}
thermal evaporation and sputtering, Dry and Wet Oxidation of Silicon,
\end{flushleft}


\begin{flushleft}
Understanding of binary eutectic phase diagrams, phase transitions,
\end{flushleft}


\begin{flushleft}
etc. and study of the optical, electrical, semiconducting and dielectric
\end{flushleft}


\begin{flushleft}
properties of the synthesized materials. Simulation experiments to
\end{flushleft}


\begin{flushleft}
understand the properties of solid state materials (e.g. ion-matter
\end{flushleft}


\begin{flushleft}
interaction, properties of low dimensional materials and band structure
\end{flushleft}


\begin{flushleft}
estimation) are also included.
\end{flushleft}





\begin{flushleft}
PYL702 Physics of Semiconductor Devices
\end{flushleft}


\begin{flushleft}
3 Credits (3-0-0)
\end{flushleft}


\begin{flushleft}
Overlap with : PYL201
\end{flushleft}


\begin{flushleft}
Charge carriers in semiconductors: Intrinsic and extrinsic semiconductors,
\end{flushleft}


\begin{flushleft}
position of Fermi energy level. Carrier transport phenomenon: Carrier
\end{flushleft}


\begin{flushleft}
drift and diffusion, Hall effect. Carrier generation and recombination.
\end{flushleft}


\begin{flushleft}
PN junction: Energy band diagram, electrostatics of pn junction, PN
\end{flushleft}


\begin{flushleft}
junction current, ideal current-voltage relationship, junction breakdown
\end{flushleft}


\begin{flushleft}
mechanisms, heterojunctions. Metal-semiconductor contacts: Schottky
\end{flushleft}


\begin{flushleft}
barrier diodes, current transport in Schotty diodes, I-V characteristics,
\end{flushleft}


\begin{flushleft}
Ohmic contacts. MOS structure: Ideal MOS structure, energy band
\end{flushleft}


\begin{flushleft}
diagrams under accumulation, depletion and inversion conditions, C-V
\end{flushleft}


\begin{flushleft}
characteristics, various oxide charges in Si/SiO2 MOS and their effect
\end{flushleft}


\begin{flushleft}
on C-V graph, MOSFET, basics about the operation of a MOSFET, I-V
\end{flushleft}


\begin{flushleft}
relationships of a MOSFET, non ideal effects. Optical devices: Basics
\end{flushleft}


\begin{flushleft}
of Solar cells and photodetectors.
\end{flushleft}





\begin{flushleft}
PYP702 Solid State Materials Laboratory-II
\end{flushleft}


\begin{flushleft}
3 Credits (0-0-6)
\end{flushleft}


\begin{flushleft}
In this course, the emphasis is given on some advanced experiments
\end{flushleft}


\begin{flushleft}
related with materials characterization, such as X-ray diffraction,
\end{flushleft}


\begin{flushleft}
X-ray fluorescence, determination of transition temperature in
\end{flushleft}


\begin{flushleft}
a high temperature superconductor frequency dependence of
\end{flushleft}


\begin{flushleft}
dielectric constants Lock-in detection technique, Solar cells, Minority
\end{flushleft}


\begin{flushleft}
carrier life-time measurements, capacitance-voltage measurements
\end{flushleft}





\begin{flushleft}
on semiconductor devices, current-voltage characteristics of
\end{flushleft}


\begin{flushleft}
varistors, synthesis and electrical characterization of thermistors,
\end{flushleft}


\begin{flushleft}
disaccommodation factor in ferrites, etc.
\end{flushleft}





\begin{flushleft}
PYL703 Electronic Properties of Materials
\end{flushleft}


\begin{flushleft}
3 Credits (3-0-0)
\end{flushleft}


\begin{flushleft}
Overlap with : PYL102
\end{flushleft}


\begin{flushleft}
Drude and Sommerfeld theory of metals, Periodic Potential, Bloch's
\end{flushleft}


\begin{flushleft}
theorem, Kronig-Penney Model and Origin of bands, example of real
\end{flushleft}


\begin{flushleft}
band structure, Ferromagnetism, Molecular field theory, Exchange
\end{flushleft}


\begin{flushleft}
interactions, Band theory of ferromagnetism, Ferrimagnetism, Ferrites,
\end{flushleft}


\begin{flushleft}
Molecular field theory for ferrimagnets, Magnetic Domains, Type of
\end{flushleft}


\begin{flushleft}
Domain walls, Structure of domain walls, Soft and Hard magnetic
\end{flushleft}


\begin{flushleft}
materials, Spin waves, Magnon dispersion relation, Introduction to
\end{flushleft}


\begin{flushleft}
nanomagnetism, Dielectric constants of solids and liquids, ClaussiusMossoti relation, dielectric dispersion and losses, piezo, ferro- and
\end{flushleft}


\begin{flushleft}
pyroelectricity, Optical constants, atomistic theory of optical properties,
\end{flushleft}


\begin{flushleft}
quantum mechanical treatment, band transitions, dispersion, plasma
\end{flushleft}


\begin{flushleft}
oscillations and excitons.
\end{flushleft}





\begin{flushleft}
PYL704 Science and Technology of Thin Films
\end{flushleft}


\begin{flushleft}
3 Credits (3-0-0)
\end{flushleft}


\begin{flushleft}
Overlap with : PYL116
\end{flushleft}


\begin{flushleft}
Kinetic Theory of Gases and basics of vacuum science and
\end{flushleft}


\begin{flushleft}
technology, Physical Vapor Deposition - Hertz Knudsen equation;
\end{flushleft}


\begin{flushleft}
mass evaporation rate; Knudsen cell, Directional distribution of
\end{flushleft}


\begin{flushleft}
evaporating species, Evaporation of elements, compounds, alloys,
\end{flushleft}


\begin{flushleft}
Raoult's law, Homogenous and Heterogenous Nucleation, capillarity
\end{flushleft}


\begin{flushleft}
theory, atomistic and kinetic models of nucleation, basic modes
\end{flushleft}


\begin{flushleft}
of thin film growth, stages of film growth \& cluster coalescence.
\end{flushleft}


\begin{flushleft}
E-beam beam evaporation, Molecular beam epitaxy and Pulsed Laser
\end{flushleft}


\begin{flushleft}
Deposition, Epitaxy--homo, hetero and coherent epilayers, lattice misfit
\end{flushleft}


\begin{flushleft}
and imperfections, epitaxy of compound semiconductors, scope of
\end{flushleft}


\begin{flushleft}
devices and applications, Glow Discharge and Plasma, Sputtering--
\end{flushleft}


\begin{flushleft}
mechanisms and yield, dc and rf sputtering, Bias sputtering,
\end{flushleft}


\begin{flushleft}
magnetically enhanced sputtering systems, reactive sputtering,
\end{flushleft}


\begin{flushleft}
Hybrid and Modified PVD- Ion plating, reactive evaporation, ion beam
\end{flushleft}


\begin{flushleft}
assisted deposition, Chemical Vapor Deposition - reaction chemistry
\end{flushleft}


\begin{flushleft}
and thermodynamics of CVD; Thermal CVD, Laser \& plasma enhanced
\end{flushleft}


\begin{flushleft}
CVD, Atomic layer deposition, Electrodeposition, Spray pyrolysis.
\end{flushleft}





\begin{flushleft}
PYL705 Nanostructured Materials
\end{flushleft}


\begin{flushleft}
3 Credits (3-0-0)
\end{flushleft}


\begin{flushleft}
Introduction and importance of nanostructured materials. Differences
\end{flushleft}


\begin{flushleft}
in the properties of bulk, nanoparticles, quantum dots, clusters,
\end{flushleft}


\begin{flushleft}
superlattices and nanostructured layers. Quantum confinement, surface
\end{flushleft}


\begin{flushleft}
enhanced properties, effective mass and tight binding approximations.
\end{flushleft}


\begin{flushleft}
Properties of 0D, 1D, 2D and superlattice structures. Plasmonic and
\end{flushleft}


\begin{flushleft}
optical properties of metal nanoparticles, Properties of magnetic
\end{flushleft}


\begin{flushleft}
nanoparticles, Structure and physical properties of nanomaterials.
\end{flushleft}


\begin{flushleft}
Chemical and physical methods for low dimensional growth with size
\end{flushleft}


\begin{flushleft}
control and size selection. Synthesis methods and growth mechanism
\end{flushleft}


\begin{flushleft}
for nanorods and nanowires. Growth and properties of graphene and
\end{flushleft}


\begin{flushleft}
other monolayer materials. Application of semiconductor, metal and
\end{flushleft}


\begin{flushleft}
magnetic nanoparticles. Application of nanostructures in catalytics,
\end{flushleft}


\begin{flushleft}
solar cell, resistive memory, thermoelectric, photoelectrochemical
\end{flushleft}


\begin{flushleft}
and sensor devices.
\end{flushleft}





\begin{flushleft}
PYL707 Characterization Techniques for Materials
\end{flushleft}


\begin{flushleft}
3 Credits (3-0-0)
\end{flushleft}


\begin{flushleft}
Pre-requisites: PYL563 (for MSc), PYL114 (for UG)
\end{flushleft}


\begin{flushleft}
Introduction to structure property correlation in materials, basic
\end{flushleft}


\begin{flushleft}
crystallography basic revision in 2-3 classes, k-space, X-ray diffraction,
\end{flushleft}


\begin{flushleft}
Reitveld refinement method and its fundamentals, Ewald sphere,
\end{flushleft}


\begin{flushleft}
Transmission electron microscopy in patterns, Microstructural
\end{flushleft}


\begin{flushleft}
investigations using Scanning electron microscope and Transmission
\end{flushleft}


\begin{flushleft}
electron microscopes, Kinetics of phase transformations in solids Thermal analysis using differential thermal analysis and Differential
\end{flushleft}


\begin{flushleft}
scanning calorimetry, other techniques like Thermogravimetric analysis,
\end{flushleft}


\begin{flushleft}
Dynamic mechanical thermal analysis, Thin film DSC, Modulate
\end{flushleft}


\begin{flushleft}
DSC, Raman and Micro Raman spectroscopy, Photoluminescence
\end{flushleft}





278





\begin{flushleft}
\newpage
Physics
\end{flushleft}





\begin{flushleft}
spectroscopy, Material compositional analysis like Energy dispersive
\end{flushleft}


\begin{flushleft}
x-ray(EDX) and Electron probe micro analysis (EPMA).
\end{flushleft}





\begin{flushleft}
PYL728 Quantum Heterostructures
\end{flushleft}


\begin{flushleft}
2 Credits (2-0-0)
\end{flushleft}





\begin{flushleft}
PYL723 Vacuum Science and Cryogenics
\end{flushleft}


\begin{flushleft}
3 Credits (3-0-0)
\end{flushleft}


\begin{flushleft}
Overlap with : PYL301
\end{flushleft}





\begin{flushleft}
Semiconductor heterostructures, Quantum confined systems,
\end{flushleft}


\begin{flushleft}
Electron transport in quantum structures, 2DEG, Excitons in quantum
\end{flushleft}


\begin{flushleft}
structures, Quantum confined Stark effect, Integer Quantum Hall
\end{flushleft}


\begin{flushleft}
effect, quantum well and quantum cascade lasers, quantum well
\end{flushleft}


\begin{flushleft}
infrared photodetectors (QWIPD), resonant tunneling devices (RTD),
\end{flushleft}


\begin{flushleft}
high electron mobility transistors (HEMT), quantum interference
\end{flushleft}


\begin{flushleft}
transistors (QIT) and hot electron transistors (HET).
\end{flushleft}





\begin{flushleft}
Behavior of Gases; Gas Transport Phenomenon, Viscous, molecular
\end{flushleft}


\begin{flushleft}
and transition flow regimes, Measurement of Pressure, Residual
\end{flushleft}


\begin{flushleft}
Gas Analyses; Production of Vacuum - Mechanical pumps, Diffusion
\end{flushleft}


\begin{flushleft}
pump, Getter and Ion pumps, Cryopumps, Materials in Vacuum; High
\end{flushleft}


\begin{flushleft}
Vacuum and Ultra High Vacuum Systems; Leak Detection. Cryogenic
\end{flushleft}


\begin{flushleft}
Fluids - Helium 3, Helium 4, Superfluidity, Liquefaction of Helium,
\end{flushleft}


\begin{flushleft}
Experimental Methods at Low Temperature: Closed Cycle Refrigerators,
\end{flushleft}


\begin{flushleft}
Cryostat systems based on He4 and He3, He3-He4 dilution refrigerator,
\end{flushleft}


\begin{flushleft}
Pomeranchuk Cooling, Magnetic Refrigerators, Thermoelectric coolers;
\end{flushleft}


\begin{flushleft}
Cryostat Design: Cryogenic level sensors, Handling of cryogenic liquids,
\end{flushleft}


\begin{flushleft}
Cryogenic thermometry.
\end{flushleft}





\begin{flushleft}
PYL724 Advances in Spintronics
\end{flushleft}


\begin{flushleft}
3 Credits (3-0-0)
\end{flushleft}


\begin{flushleft}
Overlap with : PYL422 and PYL652
\end{flushleft}


\begin{flushleft}
Spin dependent transport in magnetic metals - Anisotropic
\end{flushleft}


\begin{flushleft}
Magnetoresistance, Giant Magnetoresistance, Spin dependent
\end{flushleft}


\begin{flushleft}
tunneling, Tunneling magnetoresistance, Spin-Orbit interaction and
\end{flushleft}


\begin{flushleft}
Hall effects --Spin Hall Effect and Inverse Spin Hall Effect; Spin injection
\end{flushleft}


\begin{flushleft}
phenomena and applications - Spin Transfer Torque, Spin injection
\end{flushleft}


\begin{flushleft}
magnetization reversal; High frequency phenomena; Spin Transfer
\end{flushleft}


\begin{flushleft}
Torque applications, Dilute magnetic semiconductors, Spintronic
\end{flushleft}


\begin{flushleft}
properties of ferromagnetic semiconductors, Materials for Spin
\end{flushleft}


\begin{flushleft}
Electronics, Spintronic devices and their applications.
\end{flushleft}





\begin{flushleft}
PYL725 Surface Physics and Analysis
\end{flushleft}


\begin{flushleft}
3 Credits (3-0-0)
\end{flushleft}


\begin{flushleft}
Surface structure, stability and reactivity, surface crystallography,
\end{flushleft}


\begin{flushleft}
surface stress, reconstructions and relaxation, surface sensitivity,
\end{flushleft}


\begin{flushleft}
clean surface preparation, physisorption, chemisorption, Langmuir,
\end{flushleft}


\begin{flushleft}
kinematics and dynamics of surface processes, properties of interfaces,
\end{flushleft}


\begin{flushleft}
adhesion and segregation, surface diffusion, chemical shift in
\end{flushleft}


\begin{flushleft}
electronic structure, surface states, plasmons, chemical potential/work
\end{flushleft}


\begin{flushleft}
function, experimental methods for surface structure: photoemission
\end{flushleft}


\begin{flushleft}
spectroscopy (PES), inverse photoemission spectroscopy (IPES), low
\end{flushleft}


\begin{flushleft}
energy electron diffraction (LEED), Reflection high energy electron
\end{flushleft}


\begin{flushleft}
diffraction (RHEED), Auger electron spectroscopy (AES), Secondary
\end{flushleft}


\begin{flushleft}
ion mass spectroscopy (SIMS), scanning tunneling microscopy
\end{flushleft}


\begin{flushleft}
(STM), Grazing incidence XRD, x-reflectivity (XRR), scanning electron
\end{flushleft}


\begin{flushleft}
microscope (SEM), electron energy loss spectroscopy (EELS), etc.
\end{flushleft}





\begin{flushleft}
PYL726 Semiconductor Device Technology
\end{flushleft}


\begin{flushleft}
3 Credits (3-0-0)
\end{flushleft}


\begin{flushleft}
Silicon wafer fabrication and oxidation techniques, Growth kinetics and
\end{flushleft}


\begin{flushleft}
oxide measurement techniques, defects in silicon and silicon dioxide,
\end{flushleft}


\begin{flushleft}
interface defects, polysilicon, silicon nitride and silicide formation,
\end{flushleft}


\begin{flushleft}
Lithography and etching techniques, diffusion and ion implantation,
\end{flushleft}


\begin{flushleft}
modeling and measurement of dopant profile, Thick and thin film
\end{flushleft}


\begin{flushleft}
device technology, Processes involved in ink preparation, screen
\end{flushleft}


\begin{flushleft}
printing, laser trimming, mounting, mask making and packaging, Thin
\end{flushleft}


\begin{flushleft}
film deposition, metallization etc.
\end{flushleft}





\begin{flushleft}
PYL727 Energy Materials and Devices
\end{flushleft}


\begin{flushleft}
3 Credits (3-0-0)
\end{flushleft}


\begin{flushleft}
Importance of energy materials and devices in present technology.
\end{flushleft}


\begin{flushleft}
PV materials and devices: Definition and basic physical quantities;
\end{flushleft}


\begin{flushleft}
Energy band diagram and operation of Schottky, homojunction and
\end{flushleft}


\begin{flushleft}
heterojunction solar cells. Amorphous silicon and thin film based solar
\end{flushleft}


\begin{flushleft}
cell devices. Physics of tandem solar cell devices. New generation up
\end{flushleft}


\begin{flushleft}
conversion and down conversion devices. Materials for Si based solar
\end{flushleft}


\begin{flushleft}
cell, thin film solar cells, role of nanomaterials, dye sensitized solar
\end{flushleft}


\begin{flushleft}
cells. Introduction to PV panels, domestic and industrial applications.
\end{flushleft}


\begin{flushleft}
Materials and device concept for thermoelectric devices, Methods
\end{flushleft}


\begin{flushleft}
for improving the thermoelectrical properties, application for heating
\end{flushleft}


\begin{flushleft}
and cooling applications. Operation of photoelectrochemical cell for
\end{flushleft}


\begin{flushleft}
hydrogen production, Energy band and materials requirements.
\end{flushleft}





\begin{flushleft}
PYL729 Nanoprobe Techniques
\end{flushleft}


\begin{flushleft}
1 Credit (1-0-0)
\end{flushleft}


\begin{flushleft}
Historical perspectives for invention of STM, Optical \& Electron microscopy,
\end{flushleft}


\begin{flushleft}
Atom-scale tunnelling, Imaging atomic states, STM Instrumentation,
\end{flushleft}


\begin{flushleft}
Imaging modes, Constant current, Constant height, Feedback circuitry,
\end{flushleft}


\begin{flushleft}
surface topography, local density of the states, Single molecule
\end{flushleft}


\begin{flushleft}
vibrational spectroscopy, Image processing and analysis, Atomic
\end{flushleft}


\begin{flushleft}
Force Microscopy, Capacitance detection system, Optical detection
\end{flushleft}


\begin{flushleft}
systems, Imaging modes, Representative applications in biological
\end{flushleft}


\begin{flushleft}
sciences, Force Spectroscopy, Interpreting force curve, Adhesion,
\end{flushleft}


\begin{flushleft}
Nanoindentation, Magnetic Force Microscopy, Scanning Capacitance
\end{flushleft}


\begin{flushleft}
Microscopy, Thermal Methods at the Nanoscale, Dip-pen lithography,
\end{flushleft}


\begin{flushleft}
Near field Scanning Optical Microscopy, Hard X-ray nanoprobe.
\end{flushleft}





\begin{flushleft}
PYL739 Computational Techniques for Solid State
\end{flushleft}


\begin{flushleft}
Materials
\end{flushleft}


\begin{flushleft}
3 Credits (3-0-0)
\end{flushleft}


\begin{flushleft}
Pre-requisites: PYL563/PYL114 or equivalent
\end{flushleft}


\begin{flushleft}
Numerical solution of equations of motion, NVT molecular dynamics
\end{flushleft}


\begin{flushleft}
(MD), Application of MD for continuous and discontinuous potentials,
\end{flushleft}


\begin{flushleft}
Probability, Markov chains and master equations, Simple sampling by
\end{flushleft}


\begin{flushleft}
Monte carlo (MC) methods, MC simulations for non-equilibrium and
\end{flushleft}


\begin{flushleft}
irreversible systems, Schrodinger Equation, The Born-Oppenheimer
\end{flushleft}


\begin{flushleft}
approximation, Wave-funtion based methods, Hartree theory, HartreeFock Theory, Density Functional Theory (DFT), Exchange Correlation
\end{flushleft}


\begin{flushleft}
Functional.
\end{flushleft}





\begin{flushleft}
PYL740 Advanced Condensed Matter Theory
\end{flushleft}


\begin{flushleft}
3 Credits (3-0-0)
\end{flushleft}


\begin{flushleft}
Pre-requisites: PYL563/PYL114 or equivalent
\end{flushleft}


\begin{flushleft}
Quantum Fields and their roles in describing collective modes. Particle
\end{flushleft}


\begin{flushleft}
creation and annihilation operators: Commutation relations for Bosons
\end{flushleft}


\begin{flushleft}
and Fermions. Second quantization. Equivalence with the many body
\end{flushleft}


\begin{flushleft}
Schroedinger Equation. Identical Conserved particles in equilibrium and
\end{flushleft}


\begin{flushleft}
thermodynamic properties, Simple Examples of Second Quantization,
\end{flushleft}


\begin{flushleft}
Bosonic and Fermionic systems. Cooper instability and BCS
\end{flushleft}


\begin{flushleft}
Hamiltonian, Mean field description of BCS condensate, Quasiparticle
\end{flushleft}


\begin{flushleft}
excitation and Bogoliubov de-Gennes theory. Phase transition and
\end{flushleft}


\begin{flushleft}
broken symmetry, Order parameter concept, Landau theory and
\end{flushleft}


\begin{flushleft}
Landau Ginzburg theory and some examples from condensed matter
\end{flushleft}


\begin{flushleft}
Spin systems and magnetism, Heitler London theory and Heisenberg
\end{flushleft}


\begin{flushleft}
model, Ferromagnets, Spin waves, Antiferromagnets, Spin-chains.
\end{flushleft}





\begin{flushleft}
PYL741 Field Theory and Quantum Electrodynamics
\end{flushleft}


\begin{flushleft}
3 Credits (3-0-0)
\end{flushleft}


\begin{flushleft}
Quantization of free fields; Discrete symmetries; Gauge symmetries;
\end{flushleft}


\begin{flushleft}
QED; Elementary processes; Higher order effects; Renormalization;
\end{flushleft}


\begin{flushleft}
Novel effects of QED.
\end{flushleft}





\begin{flushleft}
PYL742 General Relativity
\end{flushleft}





\begin{flushleft}
3 Credits (3-0-0)
\end{flushleft}


\begin{flushleft}
Overlaps with : PYL332
\end{flushleft}


\begin{flushleft}
Brief review of special Relativity, Principle of equivalence: weak,
\end{flushleft}


\begin{flushleft}
strong and Einstein, Experimental evidence for equivalence principle
\end{flushleft}


\begin{flushleft}
of covariance, Curvilinear coordinates, Tensor algebra and tensor
\end{flushleft}


\begin{flushleft}
analysis, Parallel transport, Curvature tensor, Ricci and Einstein
\end{flushleft}


\begin{flushleft}
tensors, Einstein equation, The Schwarschild metric, Shift in
\end{flushleft}


\begin{flushleft}
perihelion of planets, Bending of light ray, Modern tests with light
\end{flushleft}


\begin{flushleft}
delay, Gravitational lensing, Gravitational waves, and their detection,
\end{flushleft}


\begin{flushleft}
Freidman-Robertston-Walker metric, the Hubble expansion.
\end{flushleft}





279





\begin{flushleft}
\newpage
Physics
\end{flushleft}





\begin{flushleft}
PYL743 Group Theory and its Applications
\end{flushleft}


\begin{flushleft}
3 Credits (3-0-0)
\end{flushleft}


\begin{flushleft}
Concept of a group, multiplication tables, cyclic and permutation
\end{flushleft}


\begin{flushleft}
groups, subgroups, cosets, Isomorphism and Homomorphism,
\end{flushleft}


\begin{flushleft}
conjugate elements and classes, normal sub-groups and factor
\end{flushleft}


\begin{flushleft}
group, direct product of groups, Group representations, Unitary
\end{flushleft}


\begin{flushleft}
and Irreducible, representations, Schur's Lemmas, orthonormality
\end{flushleft}


\begin{flushleft}
theorems, Character tables, Basis functions for irreducible
\end{flushleft}


\begin{flushleft}
representations. Continuous groups, Lie groups, The rotation group,
\end{flushleft}


\begin{flushleft}
Special orthogonal and unitary groups, crystallographic point groups
\end{flushleft}


\begin{flushleft}
and their representations. Applications in quantum mechanics and
\end{flushleft}


\begin{flushleft}
solid state physics.
\end{flushleft}





\begin{flushleft}
PYL744 High Energy Physics
\end{flushleft}


\begin{flushleft}
3 Credits (3-0-0)
\end{flushleft}


\begin{flushleft}
Overlaps with : PYL433
\end{flushleft}


\begin{flushleft}
Fundamental interactions; QED; QCD; Marshak-Sudarshan theory
\end{flushleft}


\begin{flushleft}
of weak interactions; Parity violation; Higgs mechanism; GlashowSalam-Weinberg model; The standard model of particle physics;
\end{flushleft}


\begin{flushleft}
Open problems.
\end{flushleft}





\begin{flushleft}
PYL745 Advanced Statistical Mechanics
\end{flushleft}


\begin{flushleft}
3 Credits (3-0-0)
\end{flushleft}


\begin{flushleft}
Pre-requisites: PYL558/PYL202/equivalent
\end{flushleft}





\begin{flushleft}
scattering; Applications of stimulated processes. Electro optic,
\end{flushleft}


\begin{flushleft}
photorefractive and acousto optic effects and their applications,
\end{flushleft}


\begin{flushleft}
Ultrafast and intense field nonlinear optics. Special topics.
\end{flushleft}





\begin{flushleft}
PYL748 Quantum Optics
\end{flushleft}


\begin{flushleft}
3 Credits (3-0-0)
\end{flushleft}


\begin{flushleft}
Pre-requisites: PYL556/PYL112
\end{flushleft}


\begin{flushleft}
HBT effect, Quantization of the EM field, Quantum states of light,
\end{flushleft}


\begin{flushleft}
correlation functions, Detection of quantum light and techniques,
\end{flushleft}


\begin{flushleft}
coincidence-counting, Phase-sensitive detection, Quantum
\end{flushleft}


\begin{flushleft}
treatment of linear optics, Quantum light by non-linear optical
\end{flushleft}


\begin{flushleft}
processes, SPDC, Signatures of quantum behaviour, Landmark
\end{flushleft}


\begin{flushleft}
experiments in quantum optics, Applications: Laser cooling and
\end{flushleft}


\begin{flushleft}
BEC, Ion trapping, CPT, EIT, Slow light, Introduction to quantum
\end{flushleft}


\begin{flushleft}
communication: Quantum teleportation, Entanglement swapping,
\end{flushleft}


\begin{flushleft}
Quantum repeaters, Quantum cryptography.
\end{flushleft}





\begin{flushleft}
PYL749 Quantum Information and Computation
\end{flushleft}


\begin{flushleft}
3 Credits (3-0-0)
\end{flushleft}


\begin{flushleft}
Basic classical and Quantum mechanics; Basic information theory;
\end{flushleft}


\begin{flushleft}
Bits, Qubits and ebits; Non-locality and entanglement; Quantum gates
\end{flushleft}


\begin{flushleft}
and circuits; Teleportation, Superdense coding, Quantum oracles;
\end{flushleft}


\begin{flushleft}
Quantum algorothms; Quantum encryption; Quantum error correction;
\end{flushleft}


\begin{flushleft}
Quantum computers.
\end{flushleft}





\begin{flushleft}
Review of basic thermodynamics, Thermodynamic potentials, Equation
\end{flushleft}


\begin{flushleft}
of state. Theory of ensembles, Density matrix. Thermodynamics of
\end{flushleft}


\begin{flushleft}
phase transitions, Concept of thermodynamic stability, Metastability
\end{flushleft}


\begin{flushleft}
and instability, Vander Waal equation of state, Phase coexistence:
\end{flushleft}


\begin{flushleft}
and Gibbs phase rule. Lattice models to describe phase transition
\end{flushleft}


\begin{flushleft}
e.g. Ising model, Heisenberg model Landau theory of second
\end{flushleft}


\begin{flushleft}
order phase transitions, Scaling hypothesis, Critical exponents
\end{flushleft}


\begin{flushleft}
and universality classes, Spatial correlation, Correlation length,
\end{flushleft}


\begin{flushleft}
Importance of fluctuations near critical point. Mean Field theory,
\end{flushleft}


\begin{flushleft}
Transfer matrix method. Concept of renormalization group. Ising
\end{flushleft}


\begin{flushleft}
model, Renormalization in one dimension. Related numerical methods,
\end{flushleft}


\begin{flushleft}
Monte-Carlo simulations of spin systems.
\end{flushleft}





\begin{flushleft}
PYL751 Optical sources, photometry and metrology
\end{flushleft}


\begin{flushleft}
3 Credits (3-0-0)
\end{flushleft}





\begin{flushleft}
PYL746 Non-equilibrium Statistical Mechanics with
\end{flushleft}


\begin{flushleft}
Interdisciplinary Applications
\end{flushleft}


\begin{flushleft}
3 Credits (3-0-0)
\end{flushleft}


\begin{flushleft}
Pre-requisites: PYL558/PYL202/equivalent
\end{flushleft}





\begin{flushleft}
Optical metrology: Surface inspection, Optical gauging and profiling,
\end{flushleft}


\begin{flushleft}
Techniques for non-destructive testing, Moire self imaging and Speckle
\end{flushleft}


\begin{flushleft}
metrology, Sensing elements.
\end{flushleft}





\begin{flushleft}
Review of equilibrium systems. Systems out of equilibrium, Kinetic
\end{flushleft}


\begin{flushleft}
theory of gases, Boltzman equation and its application to transport
\end{flushleft}


\begin{flushleft}
problems, Master equation and irreversibility. Time correlation
\end{flushleft}


\begin{flushleft}
functions, linear response theory, Kubo formula, Onsager relations.
\end{flushleft}


\begin{flushleft}
Random walks, Brownian motion and diffusion, Langevin equation,
\end{flushleft}


\begin{flushleft}
Fluctuation dissipation theorem, Einstein relation, Fokker-Planck
\end{flushleft}


\begin{flushleft}
equation. Some selected topics in rachets, Driven diffusive systems.
\end{flushleft}


\begin{flushleft}
Fluctuation theorems, Jarzynski Equality. Percolation, Polymers,
\end{flushleft}


\begin{flushleft}
Soft condensed matter systems. Biological systems, Applications
\end{flushleft}


\begin{flushleft}
to Molecular motors, Stochasticity in gene expression. Stochastic
\end{flushleft}


\begin{flushleft}
growth models. Monte-Carlo simulations of Random walks and their
\end{flushleft}


\begin{flushleft}
applications to polymers, Percolation, Diffusion limited aggregation
\end{flushleft}


\begin{flushleft}
and other growth models.
\end{flushleft}





\begin{flushleft}
PYL747 Non-linear Optics
\end{flushleft}


\begin{flushleft}
3 Credits (3-0-0)
\end{flushleft}


\begin{flushleft}
Pre-requisites: PYL560
\end{flushleft}





\begin{flushleft}
Eye and vision: Visual system, Sensitivity, Acuity; Radiometry
\end{flushleft}


\begin{flushleft}
and Photmetry: Radiometric quantities and their measurements,
\end{flushleft}


\begin{flushleft}
Photometric quantities, Radiation from a surface; Brightness and
\end{flushleft}


\begin{flushleft}
luminous intensity distribution; Integrating sphere; Illumination from
\end{flushleft}


\begin{flushleft}
a line, Surface and volume sources; Colorimetry: Fundamentals,
\end{flushleft}


\begin{flushleft}
Trichromatic specifications, Colorimeters, CIE system; Conventional
\end{flushleft}


\begin{flushleft}
light sources: Point and extended sources; Incandescent, fluorescent,
\end{flushleft}


\begin{flushleft}
discharge lamps; LEDs; Lighting fundamentals, Optical detectors;
\end{flushleft}


\begin{flushleft}
Detector characteristics, Noise considerations, Single \& multi-element
\end{flushleft}


\begin{flushleft}
detectors, CCDs.
\end{flushleft}





\begin{flushleft}
PYL752 Laser systems and applications
\end{flushleft}


\begin{flushleft}
3 Credits (3-0-0)
\end{flushleft}


\begin{flushleft}
Review of Laser theory, properties of laser radiation, and laser safety;
\end{flushleft}


\begin{flushleft}
CW lasers systems: Ruby-, Nd: YAG- and Nd: Glass lasers, DPSS
\end{flushleft}


\begin{flushleft}
lasers, Fiber lasers, Gas lasers, Pulsed lasers: ns, ps, and fs lasers,
\end{flushleft}


\begin{flushleft}
Excimer-, dye-, X-ray- and Free-electron lasers; Semiconductor lasers:
\end{flushleft}


\begin{flushleft}
DH, QW, QCL, VCSEL, DFB- and DBR lasers; Application of lasers
\end{flushleft}


\begin{flushleft}
in data storage,communication and information technology; Laser
\end{flushleft}


\begin{flushleft}
applications in optical metrology; Surface profile and dimensional
\end{flushleft}


\begin{flushleft}
measurements; Laser Applications in material processing and
\end{flushleft}


\begin{flushleft}
manufacturing; 3D-printing, Marking, Drilling, Cutting, Welding,
\end{flushleft}


\begin{flushleft}
Hardening and Manufacturing; Laser Doppler velocimetry, LIDAR,
\end{flushleft}


\begin{flushleft}
laser spectroscopy, LIF, LIBS, Bio-medical applications of lasers, Laser
\end{flushleft}


\begin{flushleft}
tweezers and applications, Laser applications in defence.
\end{flushleft}





\begin{flushleft}
PYL753 Optical systems design
\end{flushleft}


\begin{flushleft}
3 Credits (3-0-0)
\end{flushleft}





\begin{flushleft}
Wave propagation in anisotropic media. Origin of optical nonlinearity,
\end{flushleft}


\begin{flushleft}
Nonlinear optical polarization; Second order and third order processes;
\end{flushleft}


\begin{flushleft}
Nonlinear optical wave equation; Second order nonlinear processes;
\end{flushleft}


\begin{flushleft}
Second harmonic generation, difference and sum frequency generation,
\end{flushleft}


\begin{flushleft}
Phase insensitive and phase sensitive optical parametric amplifiers,
\end{flushleft}


\begin{flushleft}
Spontaneous parametric down conversion; Birefringence and quasi
\end{flushleft}


\begin{flushleft}
phase matching; Optical parametric oscillators. Third order nonlinear
\end{flushleft}


\begin{flushleft}
processes; Third harmonic generation, Self phase modulation, Cross
\end{flushleft}


\begin{flushleft}
phase modulation and four wave mixing; Impact of nonlinear effects
\end{flushleft}


\begin{flushleft}
in lightwave communication systems; supercontinuum generation;
\end{flushleft}


\begin{flushleft}
Phase conjugation and applications, Stimulated Raman and Brillouin
\end{flushleft}





\begin{flushleft}
Gaussian theory of optical system; Aberrations: Transverse ray and
\end{flushleft}


\begin{flushleft}
wave aberrations; Chromatic aberration; Third order aberrations;
\end{flushleft}


\begin{flushleft}
Position and shape factors; Meridional ray tracing; Paraxial rays and
\end{flushleft}


\begin{flushleft}
first order optics; Primary chromatic aberration: Achromat doublet,
\end{flushleft}


\begin{flushleft}
Triplet and dialyte, tolerances, Chromatic aberration at finite aperture;
\end{flushleft}


\begin{flushleft}
Spherical aberration: Surface contribution formulas; Spherically
\end{flushleft}


\begin{flushleft}
corrected achromat; Oblique pencils : Tracings of oblique meridional
\end{flushleft}


\begin{flushleft}
and skew rays; Coma and sine condition; Image evaluation: Geometric
\end{flushleft}


\begin{flushleft}
OTF, Strehl ratio, Spot diagram; definition of Merit function; Cooks
\end{flushleft}


\begin{flushleft}
Triplet and its derivatives; Double Gauss lens, Introduction to Zoom
\end{flushleft}


\begin{flushleft}
lenses and Aspherics, Examples of modern optical, GRIN optics.
\end{flushleft}





280





\begin{flushleft}
\newpage
Physics
\end{flushleft}





\begin{flushleft}
PYL755 Basic optics and optical instrumentation
\end{flushleft}


\begin{flushleft}
3 Credits (3-0-0)
\end{flushleft}





\begin{flushleft}
PYL760 Biomedical optics and Bio-photonics
\end{flushleft}


\begin{flushleft}
3 Credits (3-0-0)
\end{flushleft}





\begin{flushleft}
Reflection and refraction of plane waves and by spherical surfaces;
\end{flushleft}


\begin{flushleft}
Lens aberrations; Polarization and Polarizing components; Diffraction:
\end{flushleft}


\begin{flushleft}
Diffraction by single and multiple slits and circular aperture, Gaussian
\end{flushleft}


\begin{flushleft}
beams, Interference: Two beam and multiple beam interference.
\end{flushleft}





\begin{flushleft}
Introduction to Biophotonics: Photobiology: Light-tissue interactions
\end{flushleft}


\begin{flushleft}
and light induced effects in Biological systems. Optical properties of
\end{flushleft}


\begin{flushleft}
tissue -- absorption, scattering, diffraction, and emission. Spectroscopy:
\end{flushleft}


\begin{flushleft}
Fluorescence, Raman and diffuse reflectance spectroscopy: Physics
\end{flushleft}


\begin{flushleft}
and their applications. Basic principles of optical imaging and
\end{flushleft}


\begin{flushleft}
spectroscopy systems. Principles of standard optical microscopy/
\end{flushleft}


\begin{flushleft}
fluorescence microscopy/ endoscopy and instrumentation. Confocal
\end{flushleft}


\begin{flushleft}
microscopy: Principles, instrumentation and applications. Two-photon
\end{flushleft}


\begin{flushleft}
and multi-photon microscopy. Physics of optical tweezers and it's
\end{flushleft}


\begin{flushleft}
applications in biology. Bio-medical applications of lasers: Laser
\end{flushleft}


\begin{flushleft}
scissors, Photo-dynamic therapy. Optical coherence tomography
\end{flushleft}


\begin{flushleft}
(OCT): Physics, Imaging concepts and applications. Photo-acoustic
\end{flushleft}


\begin{flushleft}
tomography (PAT): Physics, Imaging concepts and applications.
\end{flushleft}


\begin{flushleft}
Quantitative phase microscopy; Principles and imaging concepts,
\end{flushleft}


\begin{flushleft}
Imaging beyond diffraction limit; SIM, STED, NSOM, Image processing
\end{flushleft}


\begin{flushleft}
and image recovery methods.
\end{flushleft}





\begin{flushleft}
Inteferometers: Shearing and Scanning interferometers, Interferometric
\end{flushleft}


\begin{flushleft}
instrumentation for testing, Polarization interferometers; Spectroscopic
\end{flushleft}


\begin{flushleft}
instrumentation, Fourier transform spectroscopy; Imaging and super
\end{flushleft}


\begin{flushleft}
resolution imaging, near-field imaging techniques; Adaptive optics;
\end{flushleft}


\begin{flushleft}
Wavefront sensing and correction,reconstruction, Opto-medical
\end{flushleft}


\begin{flushleft}
instruments; Optical coherence tomography, Infrared instrumentation;
\end{flushleft}


\begin{flushleft}
I.R. telescopes, Focal plane arrays; Light field camera, Space optics;
\end{flushleft}


\begin{flushleft}
Satellite cameras, High-resolution radiometers, Space telescopes,
\end{flushleft}


\begin{flushleft}
Space based sensors.
\end{flushleft}





\begin{flushleft}
PYL756 Fourier optics and holography
\end{flushleft}


\begin{flushleft}
3 Credits (3-0-0)
\end{flushleft}


\begin{flushleft}
Signals and systems, Fourier Transform(FT), Sampling theorem;
\end{flushleft}


\begin{flushleft}
Diffraction theory; Fresnel-Kirchhoff formulation and angular spectrum
\end{flushleft}


\begin{flushleft}
method, brief discussion of Fresnel and Fraunhofer diffraction, FT
\end{flushleft}


\begin{flushleft}
Properties of lenses and Image formation by a lens; Frequency
\end{flushleft}


\begin{flushleft}
response of a diffraction-limited system under coherent and incoherent
\end{flushleft}


\begin{flushleft}
illumination, OTF-effects of aberration and apodization, Comparison
\end{flushleft}


\begin{flushleft}
of coherent and incoherent imaging, Super-resolution; Techniques
\end{flushleft}


\begin{flushleft}
for measurement of OTF; Analog optical information processing:
\end{flushleft}


\begin{flushleft}
Abbe-Porter experiment, phase contrast microscopy and other simple
\end{flushleft}


\begin{flushleft}
applications; Coherent image processing: Vander Lugt filter; Jointtransform correlator; Pattern recognition, Synthetic Aperture Radar.
\end{flushleft}


\begin{flushleft}
Basics of holography, in-line and off-axis holography; transmission
\end{flushleft}


\begin{flushleft}
and reflection holograms, Amplitude and phase holograms, Recording
\end{flushleft}


\begin{flushleft}
materials. Thick and thin holograms.
\end{flushleft}





\begin{flushleft}
PYL757 Statistical and Quantum optics
\end{flushleft}


\begin{flushleft}
3 Credits (3-0-0)
\end{flushleft}


\begin{flushleft}
Overlaps with : PYL748
\end{flushleft}


\begin{flushleft}
Probability theory, Generating function, Characteristic function;
\end{flushleft}


\begin{flushleft}
Analytic signal representation, Correlation and spectral properties,
\end{flushleft}


\begin{flushleft}
Temporal, spatial and partial coherence, Law of interference,
\end{flushleft}


\begin{flushleft}
spectral interference, Coherent mode representation, Propagation
\end{flushleft}


\begin{flushleft}
of coherence; Higher order correlations; Photodetection probability,
\end{flushleft}


\begin{flushleft}
Mandel's photon counting formula; Intensity interferometry, Speckle
\end{flushleft}


\begin{flushleft}
statistics and applications, Field quantization, Number states,
\end{flushleft}


\begin{flushleft}
Coherent states, Glauber-Sudarshan representation, Tests for nonclassicality, Quantum correlations, Two photon coherence function
\end{flushleft}


\begin{flushleft}
and coincidence count rate, Quantum treatment of beamsplitter and
\end{flushleft}


\begin{flushleft}
simple interferometers.
\end{flushleft}





\begin{flushleft}
PYL758 Advanced Quantum optics and applications
\end{flushleft}


\begin{flushleft}
3 Credits (3-0-0)
\end{flushleft}


\begin{flushleft}
Pre-requisites: PYL757
\end{flushleft}


\begin{flushleft}
Quantization of the EM field, Quantum states of light, Correlation
\end{flushleft}


\begin{flushleft}
functions, Photodetection techniques, Generation of quantum light,
\end{flushleft}


\begin{flushleft}
Detection of quantum light, coincidence-counting, Phase-sensitive
\end{flushleft}


\begin{flushleft}
detection, Quantum treatment of linear optics, Quantum light by nonlinear optical processes, Signatures of quantum behaviour, Squeezed
\end{flushleft}


\begin{flushleft}
states and applications, Landmark experiments in quantum optics,
\end{flushleft}


\begin{flushleft}
Light-matter interaction, Quantum memories, Experimental quantum
\end{flushleft}


\begin{flushleft}
communications : Quantum teleportation, Entanglement swapping,
\end{flushleft}


\begin{flushleft}
Quantum repeaters.
\end{flushleft}





\begin{flushleft}
PYL759 Computational optical imaging
\end{flushleft}


\begin{flushleft}
3 Credits (3-0-0)
\end{flushleft}


\begin{flushleft}
Revision of Fourier optics and basic concepts in optical imaging,
\end{flushleft}


\begin{flushleft}
Mathematical preliminaries on inverse problems in imaging,
\end{flushleft}


\begin{flushleft}
Compressive imaging, Multi-view imaging systems, Point-spread
\end{flushleft}


\begin{flushleft}
function engineering, Phase retrieval, Interferometric imaging
\end{flushleft}


\begin{flushleft}
methods such as digital holography and optical coherence tomography,
\end{flushleft}


\begin{flushleft}
Imaging through turbulent media, Super-resolution through structured
\end{flushleft}


\begin{flushleft}
illumination, Correlation/Ghost imaging.
\end{flushleft}





\begin{flushleft}
PYL761 Liquid Crystals
\end{flushleft}


\begin{flushleft}
3 Credits (3-0-0)
\end{flushleft}


\begin{flushleft}
Nematic, Cholesteric, Smectic and Ferro-electric liquid crystals,
\end{flushleft}


\begin{flushleft}
Landau-de Gennes and Frank-Oseen free energy, Nematic-isotropic
\end{flushleft}


\begin{flushleft}
phase transition, Landau theory and Maier-Saupe theory, Kerr effect,
\end{flushleft}


\begin{flushleft}
Pockel effect, Polarizing Microscopy, Differential Scanning calorimetery,
\end{flushleft}


\begin{flushleft}
Dielectric Spectroscopy, Bent core liquid crystals, Twist bent liquid
\end{flushleft}


\begin{flushleft}
crystals, Display applications.
\end{flushleft}





\begin{flushleft}
PYP761 Optical fabrication and metrology laboratory
\end{flushleft}


\begin{flushleft}
3 Credits (0-0-6)
\end{flushleft}


\begin{flushleft}
Trepanning, Grinding, Curve generation, Smoothing and polishing,
\end{flushleft}


\begin{flushleft}
Centering and Edging, Optical coating, Autocollimator, Newton
\end{flushleft}


\begin{flushleft}
interferometer, Twyman-Green interferometer, Shack Hartmann
\end{flushleft}


\begin{flushleft}
Sensor and Moire, Talbot interferometry for measurement of optical
\end{flushleft}


\begin{flushleft}
performance parameters of the optical elements, Spherometers, Abbe
\end{flushleft}


\begin{flushleft}
refractometer, White light, Fabry-Perot interferometers.
\end{flushleft}





\begin{flushleft}
PYL762 Statistical Optics and Optical coherence theory
\end{flushleft}


\begin{flushleft}
3 Credits (3-0-0)
\end{flushleft}


\begin{flushleft}
Review of probability and random variables. Probability and Statistics
\end{flushleft}


\begin{flushleft}
in Optics. Stochastic processes to represent optical fields. Ergodicity
\end{flushleft}


\begin{flushleft}
and stationarity, Auto-correlation, cross-correlation, and WienerKhinchin theorem, Gaussian and Poisson random processes. First-order
\end{flushleft}


\begin{flushleft}
properties of optical fields: Radiation from sources of any state of
\end{flushleft}


\begin{flushleft}
coherence. Monochromatic, polychromatic and broad light sources.
\end{flushleft}


\begin{flushleft}
Polarized, partially polarized and unpolarized thermal light and pseudothermal light. Second-order coherence theory in space-time domain:
\end{flushleft}


\begin{flushleft}
Temporal coherence and complex degree of self coherence. Spatial
\end{flushleft}


\begin{flushleft}
coherence and complex degree of mutual coherence, Cross-spectral
\end{flushleft}


\begin{flushleft}
density, propagation of mutual coherence, The Van Cittert-Zernike
\end{flushleft}


\begin{flushleft}
theorem and it's application to stellar interferometry. Higher-order
\end{flushleft}


\begin{flushleft}
coherence theory: Hanbury-Brown and Twiss experiment, Intensityintensity correlation and Ghost imaging. Second order coherence theory
\end{flushleft}


\begin{flushleft}
in space-frequency domain: Concept of cross-spectral density, spectral
\end{flushleft}


\begin{flushleft}
degree of coherence, Wiener-Khintchin theorem, Electromagnetic
\end{flushleft}


\begin{flushleft}
coherence, Degree of polarization and applications. Applications of
\end{flushleft}


\begin{flushleft}
second-order coherence theory: Optical coherence tomography, stellar
\end{flushleft}


\begin{flushleft}
interferometry, Laser speckle and speckle metrology, Fourier transform
\end{flushleft}


\begin{flushleft}
spectroscopy, Partial coherence in imaging systems, Propagation
\end{flushleft}


\begin{flushleft}
through random inhomogeneous media.
\end{flushleft}





\begin{flushleft}
PYP762 Advanced optics laboratory
\end{flushleft}


\begin{flushleft}
3 Credits (0-0-6)
\end{flushleft}


\begin{flushleft}
Experiments related to recording and development of holograms,
\end{flushleft}


\begin{flushleft}
Laser Speckles, Fresnel hologram, Reflection and Rainbow hologram,
\end{flushleft}


\begin{flushleft}
Polarization, Spatial filtering, Digital holography, Optical security
\end{flushleft}


\begin{flushleft}
systems, Optical singularity, Nonlinear optical processes, Tomography,
\end{flushleft}


\begin{flushleft}
Profilometry, Polarizing microscope, Strain viewer.
\end{flushleft}





281





\begin{flushleft}
\newpage
Physics
\end{flushleft}





\begin{flushleft}
PYP763 Computational Optics laboratory
\end{flushleft}


\begin{flushleft}
3 Credits (0-0-6)
\end{flushleft}


\begin{flushleft}
Pre-requisites: PYL756
\end{flushleft}


\begin{flushleft}
Ray tracing in optical systems with commercial software, Image
\end{flushleft}


\begin{flushleft}
handling in MatLab or similar environment for optics experiments,
\end{flushleft}


\begin{flushleft}
Simulation of Fresnal and Fraunhofer diffraction, Fourier transforms
\end{flushleft}


\begin{flushleft}
and applications in optics, Simulation of spatial filtering, Matched
\end{flushleft}


\begin{flushleft}
filtering and pattern recognition, Simulation of Joint Transform and
\end{flushleft}


\begin{flushleft}
Vander Lugt correlators, Synthesis of computer generated hologram
\end{flushleft}


\begin{flushleft}
and optical reconstruction, Simulation of recording and reconstruction
\end{flushleft}


\begin{flushleft}
of digital holograms, Interferogram analysis using Fourier and Phase
\end{flushleft}


\begin{flushleft}
shifting methods, Stoke's parameters of optical beams and plotting of
\end{flushleft}


\begin{flushleft}
polarization ellipse, Simulation of multi-beam interference for photonic
\end{flushleft}


\begin{flushleft}
crystal designs, Simulation of multi-beam interference for photonic
\end{flushleft}


\begin{flushleft}
crystal designs, Design Project.
\end{flushleft}





\begin{flushleft}
PYP764 Advanced Optical Workshop
\end{flushleft}


\begin{flushleft}
3 Credits (0-0-6)
\end{flushleft}


\begin{flushleft}
Development of metal optics, Infrared imaging, Fabrication of Total
\end{flushleft}


\begin{flushleft}
Internal Reflection Prisms, Measurement of thin coating, Fabrication
\end{flushleft}


\begin{flushleft}
of Shearing plate, Shearing interferometry, Talbot interferometry,
\end{flushleft}


\begin{flushleft}
Moire interferometry.
\end{flushleft}





\begin{flushleft}
PYL770 Ultra-fast optics and applications
\end{flushleft}


\begin{flushleft}
3 Credits (3-0-0)
\end{flushleft}


\begin{flushleft}
Overlaps with : PYL412
\end{flushleft}


\begin{flushleft}
Generating and measuring Ultrashort Optical Pulses.- Ultra-Broadband
\end{flushleft}


\begin{flushleft}
Optical Parametric Amplifiers.- Advances in Solid-State Ultrafast Laser
\end{flushleft}


\begin{flushleft}
Oscillators.- Ultrafast Quantum Control in Atoms and Molecules.Femtosecond Optical Frequency Combs.- Ultrafast Material Science
\end{flushleft}


\begin{flushleft}
Probed using Coherent X-Ray Pulses from High-Harmonic Generation.Ultrafast Nonlinear Fibre Optics and Supercontinuum Generation.Nonlinear Wavelength Conversion and Pulse Propagation in Optical
\end{flushleft}


\begin{flushleft}
Fibres.- Applications of Ultra-Intense, Short Laser Pulses.- Utilising
\end{flushleft}


\begin{flushleft}
Ultrafast Lasers for Multiphoton Biomedical Imaging.- Femtosecond
\end{flushleft}


\begin{flushleft}
Laser Micromachining.- Technology and Applications of THz waves,
\end{flushleft}


\begin{flushleft}
Ultrafast Nonlinear Microscopy,- Attosecond Generation.
\end{flushleft}





\begin{flushleft}
PYL771 Green Photonics
\end{flushleft}


\begin{flushleft}
3 Credits (3-0-0)
\end{flushleft}


\begin{flushleft}
Need for green photonics, Overview of solid-state lighting technologies
\end{flushleft}


\begin{flushleft}
and their advantages. Inorganic and Organic LEDs: Fundamentals,
\end{flushleft}


\begin{flushleft}
Device Physics, Diode structures and operating principles. Materials
\end{flushleft}


\begin{flushleft}
for LEDs, OLEDs and PLEDs: Phosphor materials and their
\end{flushleft}


\begin{flushleft}
characterisation. LEDs and OLED fabrication, encapsulation and
\end{flushleft}


\begin{flushleft}
Packaging techniques. Electro-optical properties of LEDs and OLEDs;
\end{flushleft}


\begin{flushleft}
Electric drive circuits, internal, external and power efficiency, Spectral
\end{flushleft}


\begin{flushleft}
distribution, and encasulants. Design and development of light outcoupling techniques. Photometry and colorimetry of LEDs and OLEDs.
\end{flushleft}


\begin{flushleft}
Free-form optics and design of LEDs and OLEDs based illumination
\end{flushleft}


\begin{flushleft}
systems: General lighting, Traffic lights, Automotive, Street \& flood
\end{flushleft}


\begin{flushleft}
lighting, and Backlights for displays.
\end{flushleft}


\begin{flushleft}
Sunlight Harvesting Technologies, Non-imaging Solar Concentrators
\end{flushleft}


\begin{flushleft}
and illuminators: Parabolic and Fresnel lens, Diffractive, Microoptics and
\end{flushleft}


\begin{flushleft}
Free-form optics for lighting and illumination engineering of day light
\end{flushleft}


\begin{flushleft}
saving, light guiding devices and diffuse lighting materials and devices.
\end{flushleft}





\begin{flushleft}
refractive index and other analytes sensing, multichannel sensing,
\end{flushleft}


\begin{flushleft}
multianalyte sensing; Factors affecting performance of the sensor:
\end{flushleft}


\begin{flushleft}
fiber parameters, change of metal, high index dielectric material,
\end{flushleft}


\begin{flushleft}
probe design, Temperature and ionic fluid.
\end{flushleft}





\begin{flushleft}
PYL780 Diffractive and micro optics
\end{flushleft}


\begin{flushleft}
3 Credits (3-0-0)
\end{flushleft}


\begin{flushleft}
Diffractive optics, Micro optics, Design of diffractive optics, Amplitude
\end{flushleft}


\begin{flushleft}
and Phase Diffractive Optics, Application of Diffractive optics,
\end{flushleft}


\begin{flushleft}
Fabrication of Diffractive and micro optics, Photo-Lithography,
\end{flushleft}


\begin{flushleft}
Interferometric, profilometric and other testing techniques for
\end{flushleft}


\begin{flushleft}
Diffractive optics, Plastic optics, Injection Moulding of plastic optics,
\end{flushleft}


\begin{flushleft}
Applications of Micro optics in Beam shaping, MOEMS, Optical
\end{flushleft}


\begin{flushleft}
information technology and Aspheric optics, Freeform optics.
\end{flushleft}





\begin{flushleft}
PYL790 Integrated Optics
\end{flushleft}


\begin{flushleft}
3 Credits (3-0-0)
\end{flushleft}


\begin{flushleft}
Guided TE and TM Modes of Symmetric and Asymmetric Planar
\end{flushleft}


\begin{flushleft}
waveguides: Step-index and graded-index waveguides. Strip and
\end{flushleft}


\begin{flushleft}
channel waveguides, anisotropic waveguides, Marcatili's Method,
\end{flushleft}


\begin{flushleft}
Effective-Index method and Perturbation method of analysis.
\end{flushleft}


\begin{flushleft}
Directional couplers, Coupled mode analysis of uniform and reverse
\end{flushleft}


\begin{flushleft}
delta-beta couplers. Applications as power splitters, Y-junction, optical
\end{flushleft}


\begin{flushleft}
switch; phase and amplitude modulators, filters, A/D converters,
\end{flushleft}


\begin{flushleft}
Y-splitters, Mode splitters, polarization splitters; Mach-Zehnder
\end{flushleft}


\begin{flushleft}
interferometer based devices, Acoustooptic waveguide devices.
\end{flushleft}


\begin{flushleft}
Arrayed waveguide devices, Nano-photonic-devices: Metal/dielectric
\end{flushleft}


\begin{flushleft}
plasmonic waveguides, Long and short range surface Plasmon modes
\end{flushleft}


\begin{flushleft}
supported by thin metal films, applications in waveguide polarizers
\end{flushleft}


\begin{flushleft}
and bio-sensing. Fabrication of integrated optical waveguides and
\end{flushleft}


\begin{flushleft}
devices. Waveguide characterisation, end-fire and prism coupling;
\end{flushleft}


\begin{flushleft}
grating and tapered couplers, Fiber pigtailing, Nonlinear effects in
\end{flushleft}


\begin{flushleft}
integrated optical waveguides.
\end{flushleft}





\begin{flushleft}
JOP791 Laboratory-I (Fiber Optics and Opt. Comm. Lab)
\end{flushleft}


\begin{flushleft}
3 Credits (0-0-6)
\end{flushleft}


\begin{flushleft}
Experiments on characterisation of optical fibers, sources,
\end{flushleft}


\begin{flushleft}
detectors and modulators, in the Physics Department and
\end{flushleft}


\begin{flushleft}
experiments on electronics and communication in the Electrical
\end{flushleft}


\begin{flushleft}
Engineering Department.
\end{flushleft}





\begin{flushleft}
PYL791 Fiber Optics
\end{flushleft}


\begin{flushleft}
3 Credits (3-0-0)
\end{flushleft}


\begin{flushleft}
Overlaps with : PYL413 and PYL650
\end{flushleft}


\begin{flushleft}
Rays and ray paths in optical fibers; Numerical aperture; Step index
\end{flushleft}


\begin{flushleft}
and graded index fibers; Attenuation in optical fibers; Modal analysis of
\end{flushleft}


\begin{flushleft}
symmetric planar waveguides; TE and TM modes, mode cutoff, power
\end{flushleft}


\begin{flushleft}
flow: Linearly polarized (LP) modes in step-index optical fibers; Mode
\end{flushleft}


\begin{flushleft}
cutoff, single mode operation; Mode field diameter in single mode
\end{flushleft}


\begin{flushleft}
fibers, LP modes of infinitely extended parabolic medium, Intermodal
\end{flushleft}


\begin{flushleft}
dispersion in multimode fibers; Optimum profile fibers; Dispersion and
\end{flushleft}


\begin{flushleft}
chirping of pulses in single mode fibers, Dispersion compensation and
\end{flushleft}


\begin{flushleft}
dispersion tailoring; Birefringence in optical fibers, Polarization mode
\end{flushleft}


\begin{flushleft}
dispersion; Specialty fibers: Birefringent fibers, Photonic crystal fibers;
\end{flushleft}


\begin{flushleft}
Erbium doped fiber amplifiers and lasers; Fiber optic components:
\end{flushleft}


\begin{flushleft}
fiber Bragg gratings, directional couplers; Fiber fabrication and
\end{flushleft}


\begin{flushleft}
characterization techniques; OTDR, connectors and splices.
\end{flushleft}





\begin{flushleft}
Solar photovoltaics: Inorganic, Organic and Polymeric solar cells:
\end{flushleft}


\begin{flushleft}
Principles, Technology and Applications. Role of solar concentrators.
\end{flushleft}





\begin{flushleft}
JOP792 Laboratory-II (Fiber Optics and Opt. Comm. Lab)
\end{flushleft}


\begin{flushleft}
3 Credits (0-0-6)
\end{flushleft}





\begin{flushleft}
PYL772 Plasmonic sensors
\end{flushleft}


\begin{flushleft}
3 Credits (3-0-0)
\end{flushleft}





\begin{flushleft}
Experiments on characterisation of optical fibers, sources,
\end{flushleft}


\begin{flushleft}
detectors and modulators, in the Physics Department and
\end{flushleft}


\begin{flushleft}
experiments on electronics and communication in the Electrical
\end{flushleft}


\begin{flushleft}
Engineering Department.
\end{flushleft}





\begin{flushleft}
Optical fiber, optical fiber sensors, characteristics and components of
\end{flushleft}


\begin{flushleft}
optical fiber sensors, Spectroscopic techniques, Modulation schemes;
\end{flushleft}


\begin{flushleft}
Physics of plasmons, Surface plasmons at semi-infinite metal-dielectric
\end{flushleft}


\begin{flushleft}
interface, Excitation of Surface plasmons, surface plasmon resonance
\end{flushleft}


\begin{flushleft}
(SPR) condition, Interrogation techniques; Theory of SPR based
\end{flushleft}


\begin{flushleft}
optical fiber sensors, N-layer model, excitation by meridional rays:
\end{flushleft}


\begin{flushleft}
on axis excitation, performance parameters: sensitivity, detection
\end{flushleft}


\begin{flushleft}
accuracy and figure of merit; SPR based sensing applications,
\end{flushleft}





\begin{flushleft}
PYL792 Optical Electronics
\end{flushleft}


\begin{flushleft}
3 Credits (3-0-0)
\end{flushleft}


\begin{flushleft}
Light propagation through anisotropic media, Electro optic effect and
\end{flushleft}


\begin{flushleft}
electro optic modulators and switches, Liquid crystal devices and
\end{flushleft}


\begin{flushleft}
spatial light modulators, Acousto optic effect, acousto optic tunable
\end{flushleft}





282





\begin{flushleft}
\newpage
Physics
\end{flushleft}





\begin{flushleft}
filter, acousto optic deflector, scanner and spectrum analyser, Basics
\end{flushleft}


\begin{flushleft}
of nonlinear optical effects, Second harmonic generation, phase
\end{flushleft}


\begin{flushleft}
matching, quasi phase matching, Sum and difference frequency
\end{flushleft}


\begin{flushleft}
generation, parametric amplification and parametric oscillation, Third
\end{flushleft}


\begin{flushleft}
order nonlinear optical effects, Self phase modulation and soliton
\end{flushleft}


\begin{flushleft}
formation, Cross phase modulation and four wave mixing, Stimulated
\end{flushleft}


\begin{flushleft}
Raman and Brillouin scattering. Nonlinear effects in optical fibers
\end{flushleft}





\begin{flushleft}
simultaneous equations, eigenvalues and eigenvectors of a real
\end{flushleft}


\begin{flushleft}
symmetry matrix, least square curve fittings, numerical integration,
\end{flushleft}


\begin{flushleft}
integral equations, ordinary differential equation with boundary
\end{flushleft}


\begin{flushleft}
conditions, Monte Carlo methods and random numbers.
\end{flushleft}





\begin{flushleft}
JOD801 Major Project Part-I
\end{flushleft}


\begin{flushleft}
6 Credits (0-0-12)
\end{flushleft}


\begin{flushleft}
Analysis/Design/Simulation/Experimental study on topics in the
\end{flushleft}


\begin{flushleft}
board area of Optoelectronics and Optical Communication, offered
\end{flushleft}


\begin{flushleft}
by the faculty.
\end{flushleft}





\begin{flushleft}
JOL793 Selected Topics-I
\end{flushleft}


\begin{flushleft}
3 Credits (3-0-0)
\end{flushleft}


\begin{flushleft}
PYL793 Photonic Devices
\end{flushleft}


\begin{flushleft}
3 Credits (3-0-0)
\end{flushleft}


\begin{flushleft}
Overlaps with : PYL312
\end{flushleft}





\begin{flushleft}
PYD801 Major Project Part-I
\end{flushleft}


\begin{flushleft}
6 Credits (0-0-12)
\end{flushleft}





\begin{flushleft}
Review of Semiconductor Physics for Photonics: The Density of States
\end{flushleft}


\begin{flushleft}
$\rho$(k) and $\rho$(E); Density of States in a Quantum Well Structure; Carrier
\end{flushleft}


\begin{flushleft}
Concentration \& Fermi Level; Quasi Fermi Levels. Semiconductor
\end{flushleft}


\begin{flushleft}
Optoelectronic Materials; Heterostructures, Strained-Layers, Bandgap
\end{flushleft}


\begin{flushleft}
Engineering; p-n junctions; Schottky Junctions \& Ohmic Contact.
\end{flushleft}


\begin{flushleft}
Interaction of Photons with Electrons and Holes in a Semiconductor;
\end{flushleft}


\begin{flushleft}
Rates of Emission and Absorption; Amplification by Stimulated
\end{flushleft}


\begin{flushleft}
Emission; The Semiconductor Optical Amplifier. Quantum Confined
\end{flushleft}


\begin{flushleft}
Stark Effect and Franz-Keldysh Effect. Electro-absorption Modulator:
\end{flushleft}


\begin{flushleft}
Principle of Operation and Device Configuration. Light Emitting Diode:
\end{flushleft}


\begin{flushleft}
Device Structure and Output Characteristics, Modulation Bandwidth,
\end{flushleft}


\begin{flushleft}
Materials for LED, and Applications. White light LEDs.
\end{flushleft}


\begin{flushleft}
Laser Diodes: Device Structure and Output Characteristics, Single
\end{flushleft}


\begin{flushleft}
Frequency Lasers; DFB, DBR Lasers, VCSEL, Quantum Well
\end{flushleft}


\begin{flushleft}
and Quantum Cascade Laser, Micro-cavity lasers. Modulation of
\end{flushleft}


\begin{flushleft}
Laser Diodes, Practical Laser Diodes \& Handling. Photodetectors:
\end{flushleft}


\begin{flushleft}
General Characteristics of Photodetectors, Impulse Response,
\end{flushleft}


\begin{flushleft}
Photoconductors, PIN, APD, Array Detectors, CCD, Solar Cell. Photonic
\end{flushleft}


\begin{flushleft}
Integrated Circuits.
\end{flushleft}





\begin{flushleft}
JOL794 Selected Topics-II
\end{flushleft}


\begin{flushleft}
3 Credits (3-0-0)
\end{flushleft}





\begin{flushleft}
Study on topics in the board area of Solid State Materials, offered
\end{flushleft}


\begin{flushleft}
by the faculty.
\end{flushleft}





\begin{flushleft}
JOD802 Major Project Part-II
\end{flushleft}


\begin{flushleft}
12 Credits (0-0-24)
\end{flushleft}


\begin{flushleft}
Detailed investigations on the study of contemporary topics in
\end{flushleft}


\begin{flushleft}
the board area of Optoelectronics and Optical Communication.
\end{flushleft}


\begin{flushleft}
Normally this is a follow-up of the study carried out under Part-1
\end{flushleft}


\begin{flushleft}
of the Major Project.
\end{flushleft}





\begin{flushleft}
PYD802 Major Project Part-II
\end{flushleft}


\begin{flushleft}
12 Credits (0-0-24)
\end{flushleft}


\begin{flushleft}
Detailed investigations on the study of contemporary topics in the
\end{flushleft}


\begin{flushleft}
board area of Solid State Materials. Normally this is a follow-up of the
\end{flushleft}


\begin{flushleft}
study carried out under Part-1 of the Major Project.
\end{flushleft}





\begin{flushleft}
PYD801 Major Project Part-I
\end{flushleft}


\begin{flushleft}
6 Credits (0-0-12)
\end{flushleft}


\begin{flushleft}
Study on topics in the board area of Applied Optics, offered by
\end{flushleft}


\begin{flushleft}
the faculty.
\end{flushleft}





\begin{flushleft}
PYD802 Major Project Part-II
\end{flushleft}


\begin{flushleft}
12 Credits (0-0-24)
\end{flushleft}





\begin{flushleft}
JOS795 Independent Study
\end{flushleft}


\begin{flushleft}
3 Credits (0-3-0)
\end{flushleft}


\begin{flushleft}
Detailed study on a contemporary topic in the area of Optoelectronics/
\end{flushleft}


\begin{flushleft}
Optical Communication, as suggested by the Course Coordinator.
\end{flushleft}





\begin{flushleft}
PYL795 Optics and Lasers
\end{flushleft}


\begin{flushleft}
3 Credits (3-0-0)
\end{flushleft}


\begin{flushleft}
Overlaps with : PYL115, PYL311, PYL560 and PYL655
\end{flushleft}


\begin{flushleft}
Review of basic optics: Reflection and refraction of plane waves;
\end{flushleft}


\begin{flushleft}
Polarization and polarizing devices; Diffraction: diffraction due to single
\end{flushleft}


\begin{flushleft}
slit and circular aperture, grating, Gaussian beam; Interference: two
\end{flushleft}


\begin{flushleft}
beam and multiple beam interference, Fabry-Perot interferometer,
\end{flushleft}


\begin{flushleft}
Michelson interferometer; Fourier optics and its applications, spatial
\end{flushleft}


\begin{flushleft}
frequency filters.
\end{flushleft}





\begin{flushleft}
Detailed investigations on the study of contemporary topics in the
\end{flushleft}


\begin{flushleft}
board area of Applied Optics. Normally this is a follow-up of the study
\end{flushleft}


\begin{flushleft}
carried out under Part-1 of the Major Project.
\end{flushleft}





\begin{flushleft}
PYS855 Independent Study
\end{flushleft}


\begin{flushleft}
3 Credits (0-3-0)
\end{flushleft}


\begin{flushleft}
PYL858 Advanced Holographic techniques
\end{flushleft}


\begin{flushleft}
3 Credits (3-0-0)
\end{flushleft}





\begin{flushleft}
Interaction of light with matter, light amplification and oscillaton, Laser
\end{flushleft}


\begin{flushleft}
rate equations, three level and four level systems, Line broadening
\end{flushleft}


\begin{flushleft}
mechanisms, Laser power around threshold, Optical resonators and
\end{flushleft}


\begin{flushleft}
resonator stability, Modes of a spherical mirror resonator, mode
\end{flushleft}


\begin{flushleft}
selection, Q-switching, mode locking in lasers, properties of laser
\end{flushleft}


\begin{flushleft}
radiation, laser systems and some applications of lasers.
\end{flushleft}





\begin{flushleft}
Basic concepts in holography, Holographic displays and stereograms,
\end{flushleft}


\begin{flushleft}
Image holograms, White light, Rainbow holograms, Color holograms,
\end{flushleft}


\begin{flushleft}
Volume holograms, Diffraction efficiencies, Fourier Transform
\end{flushleft}


\begin{flushleft}
holograms, Pattern recognition, Correlators. Computer generated
\end{flushleft}


\begin{flushleft}
holography, Digital holography and its applications: Holgraphic
\end{flushleft}


\begin{flushleft}
interferometry, Holographic contouring, NDT of engineering
\end{flushleft}


\begin{flushleft}
objects, Optical testing, HOEs, Particle sizing, holographic Particle
\end{flushleft}


\begin{flushleft}
Image Velocimetry, Microscopy, Interferoemtry, Imaging through
\end{flushleft}


\begin{flushleft}
aberrated media, phase amplification by holography, Multifunction
\end{flushleft}


\begin{flushleft}
elements, diffusers, interconnects, couplers, scanners, Optical data
\end{flushleft}


\begin{flushleft}
storage, optical data processing, holographic solar concentrators,
\end{flushleft}


\begin{flushleft}
Associative memory.
\end{flushleft}





\begin{flushleft}
JOV796 Selected Topics in Photonics
\end{flushleft}


\begin{flushleft}
1 Credit (1-0-0)
\end{flushleft}





\begin{flushleft}
PYL879 Selected Topics in Applied Optics
\end{flushleft}


\begin{flushleft}
3 Credits (3-0-0)
\end{flushleft}





\begin{flushleft}
PYL800 Numerical and Computational Methods in
\end{flushleft}


\begin{flushleft}
Research
\end{flushleft}


\begin{flushleft}
3 Credits (3-0-0)
\end{flushleft}


\begin{flushleft}
Solution of polynomial and transcendental equations, ordinary
\end{flushleft}


\begin{flushleft}
differential equations with initial conditions, matrix algebra and
\end{flushleft}





\begin{flushleft}
PYV881 Selected Topics-I
\end{flushleft}


\begin{flushleft}
1 Credit (1-0-0)
\end{flushleft}


\begin{flushleft}
PYV882 Selected Topics-II
\end{flushleft}


\begin{flushleft}
1 Credit (1-0-0)
\end{flushleft}





283





\begin{flushleft}
\newpage
Physics
\end{flushleft}





\begin{flushleft}
PYD883 Minor Project
\end{flushleft}


\begin{flushleft}
3 Credits (0-0-6)
\end{flushleft}





\begin{flushleft}
PYL892 Guided Wave Photonic Sensors
\end{flushleft}





\begin{flushleft}
PYL891 Fiber Optic Components and Devices
\end{flushleft}


\begin{flushleft}
3 Credits (3-0-0)
\end{flushleft}


\begin{flushleft}
Pre-requisites: PYL413 or PYL650 or PYL791
\end{flushleft}





\begin{flushleft}
Pre-requisites: PYL413 or PYL650 or PYL790 or PYL791
\end{flushleft}





\begin{flushleft}
3 credits (3-0-0)
\end{flushleft}





\begin{flushleft}
Review of optical fiber properties: step and graded index fibers,
\end{flushleft}


\begin{flushleft}
multimode, single mode, birefringent, photonic crystal and holey fiber:
\end{flushleft}


\begin{flushleft}
Directional couplers: Analysis, fabrication and characterization: Fused
\end{flushleft}


\begin{flushleft}
and polished fiber couplers application in power dividers, wavelength
\end{flushleft}


\begin{flushleft}
division multiplexing, interleavers and loop mirrors: Fiber half-block
\end{flushleft}


\begin{flushleft}
devices and application in polarizers, and wavelength filters. Fiber
\end{flushleft}


\begin{flushleft}
grating: Short and Long period gratings, Analysis, fabrication and
\end{flushleft}


\begin{flushleft}
characterization: application in add-drop multiplexing, gain flattening,
\end{flushleft}


\begin{flushleft}
dispersion compensation and wavelength locking and sensing.
\end{flushleft}


\begin{flushleft}
Polarization effects in Optical fibers, Fiber polarization components:
\end{flushleft}


\begin{flushleft}
Fiber optic wave-plates, polarization controllers and associated microoptic components like isolators and circulators; Optical fiber sensors:
\end{flushleft}


\begin{flushleft}
Intensity, phase, polarization and wavelength-shift based sensors,
\end{flushleft}


\begin{flushleft}
applications in various disciplines.
\end{flushleft}





\begin{flushleft}
Review of propagation characteristics of single and multimode
\end{flushleft}


\begin{flushleft}
optical Fibers and Integrated optical Waveguides. Surface plasmon
\end{flushleft}


\begin{flushleft}
modes supported by a single metal/dielectric interface and dielectric/
\end{flushleft}


\begin{flushleft}
metal/dielectric waveguides. Fiber Optic Sensors: Intensity, phase,
\end{flushleft}


\begin{flushleft}
polarization and wavelength modulation schemes. Intensity based
\end{flushleft}


\begin{flushleft}
sensors: using microbends and tapers in multimode fibers, MachZehnder interferometer sensors, Fiber Optic gyroscope, Fiber optic
\end{flushleft}


\begin{flushleft}
current sensor, Photonic crystal based sensors. Sensors based on Bragg
\end{flushleft}


\begin{flushleft}
and Long period gratings in Fiber and integrated optical waveguides,
\end{flushleft}


\begin{flushleft}
Sensors based on modal interference: Applications in temperature,
\end{flushleft}


\begin{flushleft}
strain and refractive index sensing. Distributed Sensors based on
\end{flushleft}


\begin{flushleft}
Raman and Brillouin Scattering. Surface Plasmon Resonance (SPR)
\end{flushleft}


\begin{flushleft}
bio-sensors based on Krechman and Otto configurations, coupling
\end{flushleft}


\begin{flushleft}
with optical fiber modes, Grating coupled, Localised SPR, Plasmonic
\end{flushleft}


\begin{flushleft}
nanoparticles, interferometry. Signal processing, Noise factors in
\end{flushleft}


\begin{flushleft}
sensors and sensor networking.
\end{flushleft}





284





\begin{flushleft}
\newpage
Department of Textile Technology
\end{flushleft}


\begin{flushleft}
TXL110 Polymer Chemistry
\end{flushleft}


\begin{flushleft}
3 Credits (3-0-0)
\end{flushleft}


\begin{flushleft}
Pre-requisites: CML100
\end{flushleft}


\begin{flushleft}
The course will deal with chain and step growth polymerization
\end{flushleft}


\begin{flushleft}
methods, polymer's macromolecular architecture, molecular weight of
\end{flushleft}


\begin{flushleft}
polymers, copolymerization, cross-linked polymers, general structure
\end{flushleft}


\begin{flushleft}
and characteristics of polymers, properties of fiber forming polymers
\end{flushleft}


\begin{flushleft}
and their applications.
\end{flushleft}





\begin{flushleft}
TXL111 Textile Fibres
\end{flushleft}


\begin{flushleft}
3 Credits (2-0-2)
\end{flushleft}


\begin{flushleft}
Pre-requisites: PYL100/MTL100/CML100
\end{flushleft}


\begin{flushleft}
Classification of fibres. Basic structure of a fibre. General properties
\end{flushleft}


\begin{flushleft}
of a fibre such as moisture absorption, tenacity, elongation, initial
\end{flushleft}


\begin{flushleft}
modulus, yield point, toughness, elastic recovery. Relationship between
\end{flushleft}


\begin{flushleft}
polymer structure and fiber properties. Detailed chemical and physical
\end{flushleft}


\begin{flushleft}
structure of natural fibres: cotton, wool and silk, their basic properties.
\end{flushleft}


\begin{flushleft}
Introduction to important bast and leaf fibres. Basic introduction to
\end{flushleft}


\begin{flushleft}
Fibre spinning. Introduction Manmade and synthetic fibres: Viscose,
\end{flushleft}


\begin{flushleft}
Acetate, Acrylic, Nylon, polyester. High Performance Fibres.
\end{flushleft}


\begin{flushleft}
Laboratory exercises would include experiments on fibre identification
\end{flushleft}


\begin{flushleft}
through physical appearance, microscopic (optical, SEM), and burning
\end{flushleft}


\begin{flushleft}
behaviour. Chemical identification through solvent treatment and
\end{flushleft}


\begin{flushleft}
elemental analysis.
\end{flushleft}





\begin{flushleft}
TXL211 Structure and Physical Properties of Fibres
\end{flushleft}


\begin{flushleft}
3 Credits (3-0-0)
\end{flushleft}


\begin{flushleft}
Pre-requisites: TXL110/TXL111/TXN100
\end{flushleft}


\begin{flushleft}
Molecular architecture. Configuration. Conformation. Amorphous and
\end{flushleft}


\begin{flushleft}
crystalline phases. Glass transition. Crystallization. Melting. Structures
\end{flushleft}


\begin{flushleft}
in natural and synthetic fibres. Characterization techniques. Fibre
\end{flushleft}


\begin{flushleft}
properties. Moisture absorption properties. Mechanical properties.
\end{flushleft}


\begin{flushleft}
Fibre friction. Optical properties. Thermal properties.
\end{flushleft}





\begin{flushleft}
TXL212 Manufactured Fibre Technology
\end{flushleft}


\begin{flushleft}
3 Credits (3-0-0)
\end{flushleft}


\begin{flushleft}
Pre-requisites: TXL110/TXL111/TXN100
\end{flushleft}





\begin{flushleft}
fibre mass. Influence of process parameters on opening and cleaning.
\end{flushleft}


\begin{flushleft}
Analysis of opening and cleaning processes. Principles and methods of
\end{flushleft}


\begin{flushleft}
fibre mixing and blending. Principles of carding. Machine elements and
\end{flushleft}


\begin{flushleft}
operations in card. Sliver formation, packing and fibre configurations
\end{flushleft}


\begin{flushleft}
in sliver. Objectives, principles and methods of roller drafting. Purpose
\end{flushleft}


\begin{flushleft}
and principle of condensation of fibres. Causes of mass variation of
\end{flushleft}


\begin{flushleft}
fibrous assembly and control. Automation and recent developments in
\end{flushleft}


\begin{flushleft}
blowroom, card and draw frames. Fibre opening, carding and drawing
\end{flushleft}


\begin{flushleft}
for wool, jute and other fibres. Modification in process parameters for
\end{flushleft}


\begin{flushleft}
processing blended fibres in blowroom, card and drawframe.
\end{flushleft}





\begin{flushleft}
TXP221 Yarn Manufacture Laboratory-I
\end{flushleft}


\begin{flushleft}
1 Credit (0-0-2)
\end{flushleft}


\begin{flushleft}
Pre-requisites: TXL110/TXL111/TXN100
\end{flushleft}


\begin{flushleft}
Experiments related to the lecture course entitled {``}Yarn Manufacture
\end{flushleft}


\begin{flushleft}
I (TXL221)''.
\end{flushleft}





\begin{flushleft}
TXL222 Yarn Manufacture-II
\end{flushleft}


\begin{flushleft}
3 Credits (3-0-0)
\end{flushleft}


\begin{flushleft}
Pre-requisites: TXL110/TXL111/TXN100
\end{flushleft}


\begin{flushleft}
Fibre fractionation and combing. Preparation of fibre assembly for
\end{flushleft}


\begin{flushleft}
combing. Principle of operations in a rectilinear comber. Combing
\end{flushleft}


\begin{flushleft}
machine elements. Theory of fibre fractionation. Roving formation:
\end{flushleft}


\begin{flushleft}
Elements of roving frame, drafting, twisting and winding in speed
\end{flushleft}


\begin{flushleft}
frame, principle and mechanism of builder motion in speed frame.
\end{flushleft}


\begin{flushleft}
Yarn formation: Elements of ring frame, drafting, twisting and winding
\end{flushleft}


\begin{flushleft}
in ring frame, design aspects of spindles, rings and travellers, builder
\end{flushleft}


\begin{flushleft}
motion in ring frame. Spinning geometry. Twist and yarn strength. Yarn
\end{flushleft}


\begin{flushleft}
doubling : Purpose of doubling and plying of yarns, ring doubling, twofor-one and three for one twisting. New spinning methods: Principles
\end{flushleft}


\begin{flushleft}
of yarn formation in rotors, friction, airjet, vortex and electrostatic
\end{flushleft}


\begin{flushleft}
spinning. Yarn structure and property comparison.
\end{flushleft}





\begin{flushleft}
TXP222 Yarn Manufacture Laboratory-II
\end{flushleft}


\begin{flushleft}
1 Credit (0-0-2)
\end{flushleft}


\begin{flushleft}
Pre-requisites: TXL110/TXL111/TXN100
\end{flushleft}


\begin{flushleft}
Experiments related to the lecture course entitled {``}Yarn Manufacture
\end{flushleft}


\begin{flushleft}
II (TXL222)''.
\end{flushleft}





\begin{flushleft}
Polymer rheology in shear as well as extension. Polymer entanglements.
\end{flushleft}


\begin{flushleft}
Flow instabilities in polymer fluids. Principles of solidification. Heat and
\end{flushleft}


\begin{flushleft}
mass transfer. Melt spinning. Force and momentum balance in spinline.
\end{flushleft}


\begin{flushleft}
Stress induced crystallization. Experimental observations from melt
\end{flushleft}


\begin{flushleft}
spinning of polyamides and polyesters. Solution spinning. Dry and wet
\end{flushleft}


\begin{flushleft}
spinning. Transport phenomena. Kinetic and thermodynamic effects in
\end{flushleft}


\begin{flushleft}
solution spinning. Solution spinning of viscose and acrylic fibres. Dry
\end{flushleft}


\begin{flushleft}
jet wet spinning. Post spinning processes. Drawing and heat setting.
\end{flushleft}


\begin{flushleft}
Stress-strain-structure relationship. Effect of post spinning operations
\end{flushleft}


\begin{flushleft}
on fibre structure and properties. Spin finish applications. Introduction
\end{flushleft}


\begin{flushleft}
to electrospinning.
\end{flushleft}





\begin{flushleft}
TXP212 Manufactured Fibre Technology Lab
\end{flushleft}


\begin{flushleft}
1 Credit (0-0-2)
\end{flushleft}


\begin{flushleft}
Pre-requisites: TXL110/TXL111/TXN100
\end{flushleft}


\begin{flushleft}
The laboratory experiments are planned to provide knowledge on fibre
\end{flushleft}


\begin{flushleft}
formation of selected synthetic polymers and the characterization of
\end{flushleft}


\begin{flushleft}
fibres/tapes produced. Melt-spinning, extrusion, wet spinning and
\end{flushleft}


\begin{flushleft}
dry-jet wet spinning techniques is used to produce fibres or tapes.
\end{flushleft}


\begin{flushleft}
The evaluation of structure through thermo-mechanical properties,
\end{flushleft}


\begin{flushleft}
polymer solution rheology and microscopic analysis of materials is
\end{flushleft}


\begin{flushleft}
carried out using established methods.
\end{flushleft}





\begin{flushleft}
TXL231 Fabric Manufacture-I
\end{flushleft}


\begin{flushleft}
3 Credits (3-0-0)
\end{flushleft}


\begin{flushleft}
Pre-requisites: TXL110/TXL111/TXN100
\end{flushleft}


\begin{flushleft}
Introduction to various fabric forming principles: weaving, knitting,
\end{flushleft}


\begin{flushleft}
nonwoven and braiding. Stages of woven fabric manufacturing:
\end{flushleft}


\begin{flushleft}
winding, warping, drawing and tying in and weaving. Winding:
\end{flushleft}


\begin{flushleft}
principles, precision and random winding, digicone winding, yarn
\end{flushleft}


\begin{flushleft}
tensioning and clearing. Warping: direct and sectional warping. Sizing:
\end{flushleft}


\begin{flushleft}
size materials, sizing machines, process and quality control, modern
\end{flushleft}


\begin{flushleft}
trends. Drawing and tying in. Basic fabric designs: plain, matt, rib,
\end{flushleft}


\begin{flushleft}
twill and satin, drafting and lifting plans. Primary motions of shuttle
\end{flushleft}


\begin{flushleft}
looms: cam shedding, cam designs, dobby and jacquared systems,
\end{flushleft}


\begin{flushleft}
picking systems, loom timing, beat up, sley eccentricity. Secondary
\end{flushleft}


\begin{flushleft}
and auxiliary motions: take up, let off, warp and weft stop and warp
\end{flushleft}


\begin{flushleft}
protecting motions.
\end{flushleft}





\begin{flushleft}
TXP231 Fabric Manufacture Laboratory-I
\end{flushleft}


\begin{flushleft}
1 Credit (0-0-2)
\end{flushleft}


\begin{flushleft}
Pre-requisites: TXL110/TXL111/TXN100
\end{flushleft}


\begin{flushleft}
Experiments related to the theoretical paper TXL231.
\end{flushleft}





\begin{flushleft}
TXL221 Yarn Manufacture-I
\end{flushleft}


\begin{flushleft}
3 Credits (3-0-0)
\end{flushleft}


\begin{flushleft}
Pre-requisites: TXL110/TXL111/TXN100
\end{flushleft}





\begin{flushleft}
TXL232 Fabric Manufacture-II
\end{flushleft}


\begin{flushleft}
3 Credits (3-0-0)
\end{flushleft}


\begin{flushleft}
Pre-requisites: TXL110/TXL111/TXN100
\end{flushleft}





\begin{flushleft}
Impurities in natural fibres. Separation of trash and lint. Pre-baling
\end{flushleft}


\begin{flushleft}
operations for staple fibres. Purpose of opening, cleaning, mixing and
\end{flushleft}


\begin{flushleft}
blending of fibres. Blow room machinery and operating elements.
\end{flushleft}


\begin{flushleft}
Principles of fibre opening and cleaning in blow room. Transportation of
\end{flushleft}





\begin{flushleft}
Shuttleless looms: Principles of weft insertion in projectile, rapier,
\end{flushleft}


\begin{flushleft}
air-jet and water-jet looms, comparison of various weft insertion
\end{flushleft}


\begin{flushleft}
systems, principles of two phase, multiphase, circular and narrow
\end{flushleft}


\begin{flushleft}
fabric weaving. Leno weaving, Triaxial weaving. Knitting: Basic weft
\end{flushleft}





285





\begin{flushleft}
\newpage
Textile
\end{flushleft}





\begin{flushleft}
and warp knitted constructions, cams and needles, different weft and
\end{flushleft}


\begin{flushleft}
warp knitted structures and their properties, weft and warp knitting
\end{flushleft}


\begin{flushleft}
machines. Nonwovens: Definitions and classifications, production
\end{flushleft}


\begin{flushleft}
technology, selection criteria and important properties of fibres
\end{flushleft}


\begin{flushleft}
used, different types of webs and bonding techniques, production
\end{flushleft}


\begin{flushleft}
and properties of needle punched, adhesive bonded, thermally
\end{flushleft}


\begin{flushleft}
bonded, hydroentangled, spun bonded and meltblown fabrics. Braided
\end{flushleft}


\begin{flushleft}
structures: Types of braiding processes, classification of braids, braid
\end{flushleft}


\begin{flushleft}
geometry, structure-property relationship, over braiding.
\end{flushleft}





\begin{flushleft}
TXR301 Professional Practices
\end{flushleft}


\begin{flushleft}
2 Credits (0-1-2)
\end{flushleft}


\begin{flushleft}
Pre-requisites: TXL211/TXL221/TXL222/TXL231/TXL232 and
\end{flushleft}


\begin{flushleft}
EC65
\end{flushleft}





\begin{flushleft}
TXP232 Fabric Manufacture Laboratory-II
\end{flushleft}


\begin{flushleft}
1 Credit (0-0-2)
\end{flushleft}


\begin{flushleft}
Pre-requisites: TXL110/TXL111/TXN100
\end{flushleft}





\begin{flushleft}
TXL321 Multi and Long Fibre Spinning
\end{flushleft}


\begin{flushleft}
3 Credits (3-0-0)
\end{flushleft}


\begin{flushleft}
Pre-requisites: TXL221/TXL222 and EC50
\end{flushleft}





\begin{flushleft}
Experiments related to the theoretical paper TXL232.
\end{flushleft}





\begin{flushleft}
TXL241 Technology of Textile Preparation \& Finishing
\end{flushleft}


\begin{flushleft}
3 Credits (3-0-0)
\end{flushleft}


\begin{flushleft}
Pre-requisites: TXL110/TXL111/TXN100
\end{flushleft}


\begin{flushleft}
Natural and added impurities in textiles. Singeing, desizing, scouring,
\end{flushleft}


\begin{flushleft}
bleaching, mercerisation and optical whitening of cotton. Combined
\end{flushleft}


\begin{flushleft}
preparatory processes Carbonisation, scouring and bleaching of wool,
\end{flushleft}


\begin{flushleft}
degumming of silk. Preparation of synthetic fibres and blends, heat
\end{flushleft}


\begin{flushleft}
setting. Machinery for preparation of textiles. Surfactants and their
\end{flushleft}


\begin{flushleft}
application. Introduction to chemical and mechanical finishes. Chemical
\end{flushleft}


\begin{flushleft}
finishes for hand modification. Biopolishing, easy care, oil, water and
\end{flushleft}


\begin{flushleft}
soil repellent finishes. Fire retardancy, antimicrobial finishes. Finishes
\end{flushleft}


\begin{flushleft}
for wool. Mechanical finishes like shrink proofing and calendering;
\end{flushleft}


\begin{flushleft}
Raising, sueding and emerising. Low liquor application techniques
\end{flushleft}


\begin{flushleft}
and machinery; Stenters and dryers.
\end{flushleft}





\begin{flushleft}
TXP241 Technology of Textile Preparation \& Finishing Lab
\end{flushleft}


\begin{flushleft}
1.5 Credits (0-0-3)
\end{flushleft}


\begin{flushleft}
Pre-requisites: TXL110/TXL111/TXN100
\end{flushleft}


\begin{flushleft}
Natural and added impurities in textiles. Singeing, desizing, scouring,
\end{flushleft}


\begin{flushleft}
bleaching, mercerization and optical whitening of cotton. Combined
\end{flushleft}


\begin{flushleft}
preparatory processes Carbonization, scouring and bleaching of
\end{flushleft}


\begin{flushleft}
wool, degumming of silk. Chemical finishes for hand modification.
\end{flushleft}


\begin{flushleft}
Bio-polishing, Resin finishing, Water and Oil repellent finishes. Fire
\end{flushleft}


\begin{flushleft}
retardant finish, Antimicrobial finish, Weight reduction of cotton.
\end{flushleft}





\begin{flushleft}
TXS301 Independent Study
\end{flushleft}


\begin{flushleft}
3 Credits (0-3-0)
\end{flushleft}


\begin{flushleft}
Pre-requisites: EC65
\end{flushleft}





\begin{flushleft}
Blending of fibres during staple fibre spinning , Characteristics of
\end{flushleft}


\begin{flushleft}
manmade fibres and their spinnability. Blending at draw frame.
\end{flushleft}


\begin{flushleft}
Fundamentals of strictly similar yarns. Processing of manmade fibres
\end{flushleft}


\begin{flushleft}
and blends on staple fibre spinning system. Properties of blended
\end{flushleft}


\begin{flushleft}
yarns. Spinning of dyed fibres. M\'{e}lange yarns. Worsted /semi-worsted/
\end{flushleft}


\begin{flushleft}
Woolen spinning. Jute and Flax Spinning. Tow to top Conversion. Bulk
\end{flushleft}


\begin{flushleft}
yarn. Spun silk yarn.
\end{flushleft}





\begin{flushleft}
TXL331 Woven Textile Design
\end{flushleft}


\begin{flushleft}
3 Credits (3-0-0)
\end{flushleft}


\begin{flushleft}
Pre-requisites: TXL231/TXL232 and EC50
\end{flushleft}


\begin{flushleft}
Elements of woven design. Construction of elementary weaves; plain,
\end{flushleft}


\begin{flushleft}
twill, satin weaves and their derivatives. Rib and cord structures.
\end{flushleft}


\begin{flushleft}
Construction of standard woven fabrics; poplin, sheeting, denim,
\end{flushleft}


\begin{flushleft}
drill and jean, gabardine, granite, diamond and diaper weaves,
\end{flushleft}


\begin{flushleft}
Honey comb, Huckaback and Mockleno weaves. Colour effect on
\end{flushleft}


\begin{flushleft}
woven design. Dobby design, stripes and checks. Construction of
\end{flushleft}


\begin{flushleft}
jacquard design. Figuring with extra threads. Damasks and Brocades.
\end{flushleft}


\begin{flushleft}
Double cloths. Multilayer fabrics. Tapestry structures. Gauze and
\end{flushleft}


\begin{flushleft}
Leno structures. Whip cord and Bedford cord. Pique and Wadded
\end{flushleft}


\begin{flushleft}
structures. Terry pile structures. Velvet and velveteen. Axminster
\end{flushleft}


\begin{flushleft}
carpet structures. Indian traditional designs. Introduction to CAD for
\end{flushleft}


\begin{flushleft}
woven designs.
\end{flushleft}





\begin{flushleft}
TXL341 Colour Science
\end{flushleft}


\begin{flushleft}
2 Credits (2-0-0)
\end{flushleft}


\begin{flushleft}
Pre-requisites: TXL241/TXL242 and EC50
\end{flushleft}


\begin{flushleft}
The course will deal with aspects of colour science that are important
\end{flushleft}


\begin{flushleft}
to the colour technologist in the day-today manufacture and control
\end{flushleft}


\begin{flushleft}
of coloured products in textile applications.
\end{flushleft}





\begin{flushleft}
TXL242 Technology of Textile Coloration
\end{flushleft}


\begin{flushleft}
3 Credits (3-0-0)
\end{flushleft}


\begin{flushleft}
Pre-requisites: TXL110/TXL111/TXN100
\end{flushleft}


\begin{flushleft}
The principles of dyeing and printing of textile materials. Basic
\end{flushleft}


\begin{flushleft}
characteristics of dyes, chemical structure of dyes, and classification
\end{flushleft}


\begin{flushleft}
of dyes. Dyeing equipment and the specific dyes and procedures used
\end{flushleft}


\begin{flushleft}
to dye textiles. Evaluation of Fastness. Methods of printing namely,
\end{flushleft}


\begin{flushleft}
roller, screen, transfer, ink jet and the preparation of printing paste.
\end{flushleft}


\begin{flushleft}
Direct, discharge and resist printing styles. Physical chemistry of fibre/
\end{flushleft}


\begin{flushleft}
fabric dyeing. Physicochemical theories of the application of dyestuffs
\end{flushleft}


\begin{flushleft}
to textile and related materials, including the thermodynamics and
\end{flushleft}


\begin{flushleft}
kinetic principles involved.
\end{flushleft}





\begin{flushleft}
TXP242 Technology of Textile Coloration Lab
\end{flushleft}


\begin{flushleft}
1.5 Credits (0-0-3)
\end{flushleft}


\begin{flushleft}
Pre-requisites: TXL110/TXL111/TXN100
\end{flushleft}


\begin{flushleft}
The principles of dyeing and printing of textile materials. Dyeing
\end{flushleft}


\begin{flushleft}
equipment and the specific dyes and procedures used to dye textiles.
\end{flushleft}


\begin{flushleft}
Evaluation of Fastness. Methods of printing namely, screen, transfer,
\end{flushleft}


\begin{flushleft}
ink jet and the preparation of printing paste. Direct, discharge and
\end{flushleft}


\begin{flushleft}
resist printing styles.
\end{flushleft}





\begin{flushleft}
TXD301 Mini Project
\end{flushleft}


\begin{flushleft}
3 Credits (0-0-6)
\end{flushleft}


\begin{flushleft}
Pre-requisites: TXL211/TXL221/TXL222/TXL231/TXL232 and
\end{flushleft}


\begin{flushleft}
EC65
\end{flushleft}





\begin{flushleft}
TXL361 Evaluation of Textile Materials
\end{flushleft}


\begin{flushleft}
3 Credits (3-0-0)
\end{flushleft}


\begin{flushleft}
Pre-requisites: TXL211/TXL221/TXL222/TXL231/TXL232 and
\end{flushleft}


\begin{flushleft}
EC50
\end{flushleft}


\begin{flushleft}
Introduction to textile testing; Sampling and basic statistics: Selection
\end{flushleft}


\begin{flushleft}
of samples for testing; Random and biased samples; Different
\end{flushleft}


\begin{flushleft}
types of sampling of textile materials; The estimation of population
\end{flushleft}


\begin{flushleft}
characteristics from samples and the use of confidence intervals;
\end{flushleft}


\begin{flushleft}
Determination of number of tests to be carried out to give chosen
\end{flushleft}


\begin{flushleft}
degree of accuracy; Test of significance of means and variance; Related
\end{flushleft}


\begin{flushleft}
numerical; Quality control charts and their interpretation; Standard
\end{flushleft}


\begin{flushleft}
tests, analysis of data and test reports, Correlation and coefficient of
\end{flushleft}


\begin{flushleft}
determination; Analysis of variance (ANOVA).
\end{flushleft}


\begin{flushleft}
Testing methods: Measurement of length, fineness and crimp of fibres;
\end{flushleft}


\begin{flushleft}
Determination of maturity, foreign matter, and moisture content of
\end{flushleft}


\begin{flushleft}
cotton; Principles of AFIS, HVI etc.; Measurement of twist, linear
\end{flushleft}


\begin{flushleft}
density and hairiness of yarn; Evenness testing of silvers, rovings and
\end{flushleft}


\begin{flushleft}
yarns; Analysis of periodic variations in mass per unit length; Uster
\end{flushleft}


\begin{flushleft}
classimat; Spectrogram and V-L curve analysis; Tensile testing of fibres,
\end{flushleft}


\begin{flushleft}
yarns and fabrics; Automation in tensile testers; Tearing, bursting
\end{flushleft}


\begin{flushleft}
and abrasion resistance tests for fabrics; Pilling resistance of fabrics;
\end{flushleft}


\begin{flushleft}
Bending, shear and compressional properties of fabrics, fabric drape
\end{flushleft}


\begin{flushleft}
and handle (KESF, FAST etc); Crease and wrinkle behavior; Fastness
\end{flushleft}


\begin{flushleft}
characteristics of textiles; Matching of shade; Air, water and water-
\end{flushleft}





286





\begin{flushleft}
\newpage
Textile
\end{flushleft}





\begin{flushleft}
vapour transmission through fabrics; Thermal resistance of fabrics;
\end{flushleft}


\begin{flushleft}
Testing of interlaced and textured yarns; Special tests for carpets
\end{flushleft}


\begin{flushleft}
and nonwoven fabrics. Testing of special yarns (textured yarns, core
\end{flushleft}


\begin{flushleft}
yarn, ropes, braids etc). Testing of special fabrics (different types of
\end{flushleft}


\begin{flushleft}
nonwovens, carpets, different types of technical textiles like bullet
\end{flushleft}


\begin{flushleft}
proof fabrics, UV protective fabrics, EMS fabrics etc.).
\end{flushleft}





\begin{flushleft}
TXP361 Evaluation of Textiles Lab
\end{flushleft}


\begin{flushleft}
1 Credit (0-0-2)
\end{flushleft}


\begin{flushleft}
Pre-requisites: TXL211/TXL221/TXL222/TXL231/TXL232, EC50
\end{flushleft}


\begin{flushleft}
Introduction to textile testing; Experiments related to the lecture
\end{flushleft}


\begin{flushleft}
course entitled {``}Evaluation of Textile Material''.
\end{flushleft}





\begin{flushleft}
TXL371 Theory of Textile Structures
\end{flushleft}


\begin{flushleft}
4 Credits (3-1-0)
\end{flushleft}


\begin{flushleft}
Pre-requisites: TXL221/TXL222, TXL231/TXL232, EC50
\end{flushleft}


\begin{flushleft}
Basic characteristics of yarn structure. Koechlin's theory of relations
\end{flushleft}


\begin{flushleft}
among yarn count, twist, packing density, and diameter. Helical model
\end{flushleft}


\begin{flushleft}
of fibres in yarns. Radial migration of fibres in yarns. Tensile behavior
\end{flushleft}


\begin{flushleft}
of yarns. Theory of yarn mass variation. Theory of plied yarn. Basic
\end{flushleft}


\begin{flushleft}
characteristics of fabric structure. Flexible and rigid thread models of
\end{flushleft}


\begin{flushleft}
woven fabric geometry. Tensile, bending, and shear deformation of
\end{flushleft}


\begin{flushleft}
woven fabric. Geometry of knitted and nonwoven fabrics.
\end{flushleft}





\begin{flushleft}
TXL372 Speciality Yarns and Fabrics
\end{flushleft}


\begin{flushleft}
2 Credits (2-0-0)
\end{flushleft}


\begin{flushleft}
Pre-requisites: TXL221/TXL222 and TXL231/TXL232 and EC50
\end{flushleft}


\begin{flushleft}
Design, manufacture, characterization and applications of specialty
\end{flushleft}


\begin{flushleft}
yarns. Hybrid yarns. High bulk yarns. Electro-conductive yarns.
\end{flushleft}


\begin{flushleft}
Technical sewing threads. Coated yarns. Reflective yarns. Elastomeric
\end{flushleft}


\begin{flushleft}
yarns. Yarn quality requirement. Yarn preparation \& production
\end{flushleft}


\begin{flushleft}
technology. Structural design, properties-Performance and applications
\end{flushleft}


\begin{flushleft}
of specialty fabrics. Denim. Pile fabrics. Narrow fabrics. 3D fabrics.
\end{flushleft}


\begin{flushleft}
Spacer fabrics. Profiled fabrics. Contour fabrics. Polar fabrics. Spiral
\end{flushleft}


\begin{flushleft}
fabrics. Multi-functional fabrics.
\end{flushleft}





\begin{flushleft}
TXL381 Costing and its Application in Textiles
\end{flushleft}


\begin{flushleft}
4 Credits (3-1-0)
\end{flushleft}


\begin{flushleft}
Pre-requisites: TXL211/TXL221/TXL222/TXL231/TXL232 and
\end{flushleft}


\begin{flushleft}
EC50
\end{flushleft}


\begin{flushleft}
Importance of costing. Material costing in textile industry. Methods
\end{flushleft}


\begin{flushleft}
of inventory costing. Economic order quantity, price discount, safety
\end{flushleft}


\begin{flushleft}
stock, lead time. Allocation of labour cost-shift premium, overtime,
\end{flushleft}


\begin{flushleft}
idle time, rush orders in garment industry. Allocation of overheads
\end{flushleft}


\begin{flushleft}
in composite mills. Job order costing in garment industry. Economic
\end{flushleft}


\begin{flushleft}
batch quantity. Process costing in mill. Unit cost of yarns, fabric and
\end{flushleft}


\begin{flushleft}
processing. Joint and by-product costing. Absorption costing. Variable
\end{flushleft}


\begin{flushleft}
costing for decision making. Profit planning in textile industry, variation
\end{flushleft}


\begin{flushleft}
of price, costs etc., breakeven capacity. Standard costs of fibres, yarns,
\end{flushleft}


\begin{flushleft}
labour etc. HOK, OHS, UKG etc. Cost variance analysis-iteration of
\end{flushleft}


\begin{flushleft}
actual costs of fibre, labour and overhead with respect to standard
\end{flushleft}


\begin{flushleft}
costs. Work allocation to spinner. Balancing of machine for optimizing
\end{flushleft}


\begin{flushleft}
product mix in a spinning mill. Financial information-balance sheet,
\end{flushleft}


\begin{flushleft}
profit/loss account, balance sheet. Ratio analysis.
\end{flushleft}





\begin{flushleft}
TXD401 Major Project Part I
\end{flushleft}


\begin{flushleft}
4 Credits (0-0-8)
\end{flushleft}


\begin{flushleft}
Pre-requisites: TXL361/TXP361/TXL371/TXL372 and EC100
\end{flushleft}


\begin{flushleft}
Formation of project team (up to two students and up to two faculty
\end{flushleft}


\begin{flushleft}
guides); formulation of work plan completing targeted work for the
\end{flushleft}


\begin{flushleft}
semester and presentation of complete work of progress for award
\end{flushleft}


\begin{flushleft}
of grade.
\end{flushleft}





\begin{flushleft}
TXD402 Major Project Part II
\end{flushleft}


\begin{flushleft}
8 Credits (0-0-16)
\end{flushleft}


\begin{flushleft}
Pre-requisites: EC100 and Minimum B Grade in TXD401
\end{flushleft}


\begin{flushleft}
Continuation of planned tasks started in Major Project Part I, TXD411,
\end{flushleft}


\begin{flushleft}
to completion, thesis writing and presentation of complete work of
\end{flushleft}


\begin{flushleft}
progress for award of grade.
\end{flushleft}





\begin{flushleft}
TXL700 Modelling and Simulation in Fibrous Assemblies
\end{flushleft}


\begin{flushleft}
3 Credits (2-0-2)
\end{flushleft}


\begin{flushleft}
Pre-requisites: TXL211/TXL221/TXL222/TXL231/TXL232 and
\end{flushleft}


\begin{flushleft}
EC 75
\end{flushleft}


\begin{flushleft}
Introduction to Textile Modelling and Simulation, types of model.
\end{flushleft}


\begin{flushleft}
Curve Fitting Techniques: Prediction of mechanical properties of
\end{flushleft}


\begin{flushleft}
fibrous assemblies.
\end{flushleft}


\begin{flushleft}
Artificial Neural Network (ANN): Mathematical models of artificial
\end{flushleft}


\begin{flushleft}
neurons, ANN architecture, Learning rules, Back propagation
\end{flushleft}


\begin{flushleft}
algorithm, Applications of ANN. Fuzzy Logic: Crisp and fuzzy sets,
\end{flushleft}


\begin{flushleft}
Operations of fuzzy sets, Fuzzy rule generation, Defuzzification,
\end{flushleft}


\begin{flushleft}
Applications of fuzzy logic. Genetic Algorithm (G.A.): Basics of G.A.,
\end{flushleft}


\begin{flushleft}
G. A. in fabric engineering.
\end{flushleft}


\begin{flushleft}
Stochastic and Stereological Methods: Random fibrous assemblies,
\end{flushleft}


\begin{flushleft}
anisotropy characteristics, two and three-dimensional fibrous
\end{flushleft}


\begin{flushleft}
assemblies. Statistical Mechanics: Monte Carlo simulation of random
\end{flushleft}


\begin{flushleft}
fibrous assemblies,
\end{flushleft}


\begin{flushleft}
Multiscale Modelling: Geometrical modelling of textile structures,
\end{flushleft}


\begin{flushleft}
modelling of properties of fibrous assemblies
\end{flushleft}


\begin{flushleft}
Computational Fluid Dynamics: Newtonian and Non-Newtonian Fluids
\end{flushleft}


\begin{flushleft}
and their applications in extrusion processes, Computer simulation
\end{flushleft}


\begin{flushleft}
of fluid flows through porous materials, heat and mass transfer in
\end{flushleft}


\begin{flushleft}
fibrous assemblies.
\end{flushleft}





\begin{flushleft}
TXV701 Process Cont. and Econ. in Manmade Fibre Prod.
\end{flushleft}


\begin{flushleft}
1 Credit (1-0-0)
\end{flushleft}


\begin{flushleft}
Pre-requisites: TXL211/TXL221/TXL222/TXL231/TXL232 and
\end{flushleft}


\begin{flushleft}
EC 75
\end{flushleft}


\begin{flushleft}
Introduction to manmade fibres. Consumption pattern in India
\end{flushleft}


\begin{flushleft}
and World. Factors affecting their growth. Economics of manmade
\end{flushleft}


\begin{flushleft}
fibre production. Modern polyester manufacturing plant technology.
\end{flushleft}


\begin{flushleft}
Capacities, raw materials and economics. Process and parameters
\end{flushleft}


\begin{flushleft}
at polymerization. Melt spinning and draw line. Control of modulus,
\end{flushleft}


\begin{flushleft}
tenacity, crimp properties, Dye affinity during production. Typical
\end{flushleft}


\begin{flushleft}
properties of polyester staple fibre. Partially oriented yarn and fully
\end{flushleft}


\begin{flushleft}
drawn yarn. Commodity and specialty polyester fibres. Recycled
\end{flushleft}


\begin{flushleft}
polyester staple fibres. Bio-degradable polyester PLA. Applications,
\end{flushleft}


\begin{flushleft}
properties and selection of fibres as per end uses.
\end{flushleft}





\begin{flushleft}
TXV702 Management of Textile Business
\end{flushleft}


\begin{flushleft}
1 Credit (1-0-0)
\end{flushleft}


\begin{flushleft}
Pre-requisites: TXL211/TXL221/TXL222/TXL231/TXL232 and
\end{flushleft}


\begin{flushleft}
EC 75
\end{flushleft}


\begin{flushleft}
The textile industry of India : Past \& its evolution to the present day.
\end{flushleft}


\begin{flushleft}
The structure of the Indian textile industry. Cotton textile sector, Jute
\end{flushleft}


\begin{flushleft}
textile sector. Silk textile sector. Manmade textile sector. Wool textile
\end{flushleft}


\begin{flushleft}
sector. Statistics of Indian textile business (domestic \& export) and
\end{flushleft}


\begin{flushleft}
world textile trade. Textile policy 2000. Govt. of India. World trade
\end{flushleft}


\begin{flushleft}
practices. Norms, barriers etc. Various pertinent issues prevailing
\end{flushleft}


\begin{flushleft}
impacting textile industry and trade. Corporate social responsibility.
\end{flushleft}


\begin{flushleft}
Other compliances. ISO accreditation, etc. Retailing in textiles vis-a-vis
\end{flushleft}


\begin{flushleft}
consumer trend and behaviour. The challenging future of the Indian
\end{flushleft}


\begin{flushleft}
textile industry and trade.
\end{flushleft}





\begin{flushleft}
TXV703 Special Module in Textile Technology
\end{flushleft}


\begin{flushleft}
1 Credit (1-0-0)
\end{flushleft}


\begin{flushleft}
Pre-requisites: TXL211/TXL221/TXL222/TXL231/TXL232 and
\end{flushleft}


\begin{flushleft}
EC 75
\end{flushleft}


\begin{flushleft}
The course aims at introducing special topics in textile technology.
\end{flushleft}


\begin{flushleft}
The course topics and content are likely to change with each offering
\end{flushleft}


\begin{flushleft}
depending upon the current requirement and expertise available with
\end{flushleft}


\begin{flushleft}
the department including that of the visiting professionals.
\end{flushleft}





\begin{flushleft}
TXV704 Special Module in Yarn Manufacture
\end{flushleft}


\begin{flushleft}
1 Credit (1-0-0)
\end{flushleft}


\begin{flushleft}
Pre-requisites: TXL211/TXL221/TXL222/TXL231/TXL232 and
\end{flushleft}


\begin{flushleft}
EC 75
\end{flushleft}





287





\begin{flushleft}
\newpage
Textile
\end{flushleft}





\begin{flushleft}
The course aims at introducing new or highly specialized technological
\end{flushleft}


\begin{flushleft}
aspects in yarn manufacture. The course topics and content are likely
\end{flushleft}


\begin{flushleft}
to change with each offering depending upon the current requirement
\end{flushleft}


\begin{flushleft}
and expertise available with the department including that of the
\end{flushleft}


\begin{flushleft}
visiting professionals.
\end{flushleft}





\begin{flushleft}
TXV705 Special Module in Fabric Manufacture
\end{flushleft}


\begin{flushleft}
1 Credit (1-0-0)
\end{flushleft}


\begin{flushleft}
Pre-requisites: TXL211/TXL221/TXL222/TXL231/TXL232 and
\end{flushleft}


\begin{flushleft}
EC 75
\end{flushleft}


\begin{flushleft}
The course aims at introducing new or highly specialized technological
\end{flushleft}


\begin{flushleft}
aspects in fabric manufacture. The course topics and content are likely
\end{flushleft}


\begin{flushleft}
to change with each offering depending upon the current requirement
\end{flushleft}


\begin{flushleft}
and expertise available with the department including that of the
\end{flushleft}


\begin{flushleft}
visiting professionals.
\end{flushleft}





\begin{flushleft}
TXV706 Special Module in Fibre Science
\end{flushleft}


\begin{flushleft}
1 Credit (1-0-0)
\end{flushleft}


\begin{flushleft}
Pre-requisites: TXL211/TXL221/TXL222/TXL231/TXL232 and
\end{flushleft}


\begin{flushleft}
EC 75
\end{flushleft}


\begin{flushleft}
The course aims at introducing new or highly specialized technological
\end{flushleft}


\begin{flushleft}
aspects in fibre science. The course topics and content are likely to
\end{flushleft}


\begin{flushleft}
change with each offering depending upon the current requirement
\end{flushleft}


\begin{flushleft}
and expertise available with the department including that of the
\end{flushleft}


\begin{flushleft}
visiting professionals.
\end{flushleft}





\begin{flushleft}
TXV707 Special Module in Textile Chemical Processing
\end{flushleft}


\begin{flushleft}
1 Credit (1-0-0)
\end{flushleft}


\begin{flushleft}
Pre-requisites: TXL211/TXL221/TXL222/TXL231/TXL232 and
\end{flushleft}


\begin{flushleft}
EC 75
\end{flushleft}


\begin{flushleft}
The course aims at introducing new or highly specialized technological
\end{flushleft}


\begin{flushleft}
aspects in textile chemical processing. The course topics and content
\end{flushleft}


\begin{flushleft}
are likely to change with each offering depending upon the current
\end{flushleft}


\begin{flushleft}
requirement and expertise available with the department including
\end{flushleft}


\begin{flushleft}
that of the visiting professionals.
\end{flushleft}





\begin{flushleft}
TXL710 High Performance and Specialty Fibres
\end{flushleft}


\begin{flushleft}
3 Credits (3-0-0)
\end{flushleft}


\begin{flushleft}
Pre-requisites: TXL212 and EC75
\end{flushleft}


\begin{flushleft}
Definition, classification and structural requirements of high
\end{flushleft}


\begin{flushleft}
performance and specialty fibres, Polymerization, spinning and
\end{flushleft}


\begin{flushleft}
properties of aramids, aromatic polyesters, rigid rod and ladder
\end{flushleft}


\begin{flushleft}
polymers such as PBZT, PBO, PBI, PIPD, Manufacture of carbon
\end{flushleft}


\begin{flushleft}
fibres from polyacrylonitrile, viscose and pitch precursors, Concept
\end{flushleft}


\begin{flushleft}
of gel spinning and spinning of UHMPE fibres, Elastomeric polymers
\end{flushleft}


\begin{flushleft}
and fibres, Lyocell fibre production, Conducting fibres, Thermally
\end{flushleft}


\begin{flushleft}
and chemically resistant polymers and fibres, Methods of synthesis,
\end{flushleft}


\begin{flushleft}
production and properties of: glass and ceramic fibres. Specialty fibres:
\end{flushleft}


\begin{flushleft}
profile fibres, optical fibres, bicomponent fibres and hybrid fibres,
\end{flushleft}


\begin{flushleft}
Superabsorbent polymers and fibres.
\end{flushleft}





\begin{flushleft}
TXL711 Polymer and Fibre Chemistry
\end{flushleft}


\begin{flushleft}
3 Credits (3-0-0)
\end{flushleft}


\begin{flushleft}
The course will deal with chain and step growth polymerization
\end{flushleft}


\begin{flushleft}
methods, polymer's macromolecular architecture, molecular weight of
\end{flushleft}


\begin{flushleft}
polymers, copolymerization, cross-linked polymers, general structure
\end{flushleft}


\begin{flushleft}
and characteristics of polymers, spectroscopic analysis of polymers,
\end{flushleft}


\begin{flushleft}
properties of fiber forming polymers and their applications.
\end{flushleft}





\begin{flushleft}
TXP711 Polymer and Fibre Chemistry Laboratory
\end{flushleft}


\begin{flushleft}
1 Credit (0-0-2)
\end{flushleft}


\begin{flushleft}
Identification of fibres by chemical and burning tests, polymerization of
\end{flushleft}


\begin{flushleft}
vinyl monomers such as styrene, acrylamide using bulk polymerization,
\end{flushleft}


\begin{flushleft}
solution polymerization, emulsion polymerization, radiation induced
\end{flushleft}


\begin{flushleft}
polymerization. Condensation polymerization and interfacial
\end{flushleft}


\begin{flushleft}
polymerization of nylon-6, Molecular weight measurement. Intrinsic
\end{flushleft}


\begin{flushleft}
viscosity and end group analysis, preparation of phenol-formaldehyde
\end{flushleft}


\begin{flushleft}
resin. Analysis of chemical structure by FTIR, UV spectroscopy.
\end{flushleft}





\begin{flushleft}
TXL712 Polymer and Fibre Physics
\end{flushleft}


\begin{flushleft}
3 Credits (3-0-0)
\end{flushleft}


\begin{flushleft}
Molecular architecture, configuration, conformation of ideal and real
\end{flushleft}


\begin{flushleft}
chains, Random Walk models of polymer conformations, Gaussian
\end{flushleft}


\begin{flushleft}
chain, Self-avoiding walks and excluded-volume interaction, the
\end{flushleft}


\begin{flushleft}
amorphous phase and its chemical-physical aspects, the glass
\end{flushleft}


\begin{flushleft}
transition phenomenon, the WLF-equation, crystalline state and its
\end{flushleft}


\begin{flushleft}
chemical-physical aspect, cross-linked polymers and rubber elasticity,
\end{flushleft}


\begin{flushleft}
behaviour of polymers in solutions and mixtures, viscoelasticity and
\end{flushleft}


\begin{flushleft}
rheology of polymers, mechanical properties, physical properties of
\end{flushleft}


\begin{flushleft}
fibres: moisture absorption properties, mechanical properties, optical
\end{flushleft}


\begin{flushleft}
properties, thermal properties.
\end{flushleft}





\begin{flushleft}
TXP712 Polymer and Fibre Physics Laboratory
\end{flushleft}


\begin{flushleft}
1 Credit (0-0-2)
\end{flushleft}


\begin{flushleft}
Laboratory Experiments on Characterization of fibres by Infrared
\end{flushleft}


\begin{flushleft}
spectroscopy, Density measurements; Thermal analysis:
\end{flushleft}


\begin{flushleft}
Thermogravimetric Analysis (TGA), Differential Scanning calorimetry
\end{flushleft}


\begin{flushleft}
(DSC) and Thermo-Mechanical Analysis (TMA); Dynamic Mechanical
\end{flushleft}


\begin{flushleft}
Analysis (DMA); Sonic modulus ;X-ray diffraction studies; Birefringence
\end{flushleft}


\begin{flushleft}
measurement; Optical microscopy studies; Scanning Electron
\end{flushleft}


\begin{flushleft}
Microscopy (SEM) of fibres: Creep and Stress Relaxation study,
\end{flushleft}


\begin{flushleft}
Mechanical property testing such as tensile and flexural rigidity.
\end{flushleft}





\begin{flushleft}
TXL713 Technology of Melt Spun Fibres
\end{flushleft}


\begin{flushleft}
4 Credits (3-1-0)
\end{flushleft}


\begin{flushleft}
Importance of transport phenomena in fibre manufacturing;
\end{flushleft}


\begin{flushleft}
Fundamentals of momentum transfer, heat transfer, mass transfer,
\end{flushleft}


\begin{flushleft}
building differential equations using shell balance and generalized
\end{flushleft}


\begin{flushleft}
equations; Polymer rheology- shear flow, elongational flow; Melt
\end{flushleft}


\begin{flushleft}
spinning lines for filament and staple fibre; Role of spin finish;
\end{flushleft}


\begin{flushleft}
Necessary conditions for fibre formation, elasticity versus plasticity
\end{flushleft}


\begin{flushleft}
of melts; Melt instabilities; Thermodynamic limitations; Force balance
\end{flushleft}


\begin{flushleft}
and heat balance in melt spinning; Low speed melt spinning; Necking
\end{flushleft}


\begin{flushleft}
and stress induced crystallization in high speed melt spinning; Effect
\end{flushleft}


\begin{flushleft}
of process parameters on fibre spinning and structure of nylon 6, PET
\end{flushleft}


\begin{flushleft}
and PP; Drawing Process and its necessity; Neck or flow deformational
\end{flushleft}


\begin{flushleft}
drawing; Drawing machines; Effect of parameters on structure
\end{flushleft}


\begin{flushleft}
development in nylon 6, PET, PP; Types of heat setting, Effect of setting
\end{flushleft}


\begin{flushleft}
parameters on structure and properties; Concept of bulking/texturing.
\end{flushleft}





\begin{flushleft}
TXL714 Advanced Materials Characterization Techniques
\end{flushleft}


\begin{flushleft}
1 Credit (1-0-0)
\end{flushleft}


\begin{flushleft}
Relevance of advanced characterization techniques in material
\end{flushleft}


\begin{flushleft}
development; scattering techniques (SAXS/WAXS); advanced surface
\end{flushleft}


\begin{flushleft}
characterization techniques (X-ray photoelectrosn spectroscopy
\end{flushleft}


\begin{flushleft}
(XPS), Auger electron spectroscopy (AES), secondary ion mass
\end{flushleft}


\begin{flushleft}
spectroscopy (SIMS)); microscopy techniques: basics of electronmaterials interaction; SEM combined with FIB techniques; TEM and
\end{flushleft}


\begin{flushleft}
cryo-TEM; chemical analysis utilizing microscopy techniques; AFM;
\end{flushleft}


\begin{flushleft}
confocal laser microscopy.
\end{flushleft}





\begin{flushleft}
TXL715 Technology of Solution Spun Fibres
\end{flushleft}


\begin{flushleft}
3 Credits (3-0-0)
\end{flushleft}


\begin{flushleft}
Pre-requisites: TXL711/TXL713
\end{flushleft}


\begin{flushleft}
PAN properties; Solution rheology and its dependence on parameters.
\end{flushleft}


\begin{flushleft}
Effect of parameterson entanglement density, fibre spinning and
\end{flushleft}


\begin{flushleft}
subsequent drawing; Various solvent systems; Dope preparation; Wet
\end{flushleft}


\begin{flushleft}
and dry spinning processes; Effect of process parameters such as dope
\end{flushleft}


\begin{flushleft}
concentration, bath concentration, temperature and jet stretch ratio
\end{flushleft}


\begin{flushleft}
on coagulation rate, fibre breakage and fibre structure; Modeling of
\end{flushleft}


\begin{flushleft}
coagulation process; properties and structure of dry and wet spun
\end{flushleft}


\begin{flushleft}
fibres; Dry jet wet spinning. Solution spinning of PAN.
\end{flushleft}


\begin{flushleft}
Bicomponent and bulk acrylic fibres. Acrylic fibre line, crimping
\end{flushleft}


\begin{flushleft}
and annealing, tow to top conversion systems; Viscose rayon
\end{flushleft}


\begin{flushleft}
process, Spinning with and without zinc sulfate; Polynosics and high
\end{flushleft}


\begin{flushleft}
performance cellulosic fibre; Non viscose processes, Lyocell spinning
\end{flushleft}


\begin{flushleft}
process, structure and properties; Gel spinning of PE, Gel spinning
\end{flushleft}


\begin{flushleft}
of PAN and PVA. Introduction to high performance fibres and their
\end{flushleft}


\begin{flushleft}
spinning systems such as rigid rod polymer, liquid crystalline polymers,
\end{flushleft}


\begin{flushleft}
polylactic acid and spandex fibre manufacturing.
\end{flushleft}





288





\begin{flushleft}
\newpage
Textile
\end{flushleft}





\begin{flushleft}
TXP716 Fibre Production and Post Spinning Operation
\end{flushleft}


\begin{flushleft}
Laboratory
\end{flushleft}


\begin{flushleft}
2 Credits (0-0-4)
\end{flushleft}


\begin{flushleft}
Experiments related to fibres production processes. Effect of moisture
\end{flushleft}


\begin{flushleft}
and temperature on MFI of PET and PP. Melt spinning of PET, PP \&
\end{flushleft}


\begin{flushleft}
nylon-6 filament yams on laboratory spinning machines. Single and two
\end{flushleft}


\begin{flushleft}
stage drawing of the as-spun yams or industrial POY. Demonstration
\end{flushleft}


\begin{flushleft}
of high speed spinning machine. Wet and dry heat setting of PET and
\end{flushleft}


\begin{flushleft}
nylon drawn yarns. Effect of temperature and tension on heat setting.
\end{flushleft}


\begin{flushleft}
Determination of structure and mechanical properties of as spun, POY,
\end{flushleft}


\begin{flushleft}
drawn and heat set yams using DSC, X-ray, FTIR, density, sonic modulus.
\end{flushleft}


\begin{flushleft}
Effect of shear rate, temperature on polymer solution viscosity using
\end{flushleft}


\begin{flushleft}
Brookfield Rheometer and ball-fall method. Wet spinning or dry jet wet
\end{flushleft}


\begin{flushleft}
spinning of PAN copolymers. False twist and air jet texturing processes.
\end{flushleft}


\begin{flushleft}
Determination of structure of textured yam under microscope.
\end{flushleft}





\begin{flushleft}
TXL719 Functional and Smart Textiles
\end{flushleft}


\begin{flushleft}
3 Credits (3-0-0)
\end{flushleft}


\begin{flushleft}
Pre-requisites: TXL212/TXL221/ TXL231 and EC75
\end{flushleft}


\begin{flushleft}
Definition and Classification of Functional and Smart textiles ;
\end{flushleft}


\begin{flushleft}
Introduction to Composites : Theory, Types, Properties ; High
\end{flushleft}


\begin{flushleft}
Performance fibers, thermoplastic and thermosetting Resins;
\end{flushleft}


\begin{flushleft}
Composite Manufacturing and Applications; Coated and laminated
\end{flushleft}


\begin{flushleft}
Textiles: materials, formulations, techniques and applications ;
\end{flushleft}


\begin{flushleft}
Protective Textiles- Materials, design, principles and evaluation for
\end{flushleft}


\begin{flushleft}
protection against fire, harmful radiation, chemicals and pesticides;
\end{flushleft}


\begin{flushleft}
Sportswear: design, testing and materials -- fibers , yarns, fabrics for
\end{flushleft}


\begin{flushleft}
temperature control and moisture management; Medical textiles:
\end{flushleft}


\begin{flushleft}
Classification, types and products, Health and Hygiene Textilesprotection against microbes, Wound management- dressings, suture
\end{flushleft}


\begin{flushleft}
and bandages, Implants and drug delivery systems ; Smart and
\end{flushleft}


\begin{flushleft}
Intelligent Textiles : Passive and Active functionality, stimuli sensitive
\end{flushleft}


\begin{flushleft}
textiles, Electronic Textiles : wearable computers, flexible electronics.
\end{flushleft}





\begin{flushleft}
TXL721 Theory of Yarn Structure
\end{flushleft}


\begin{flushleft}
3 Credits (3-0-0)
\end{flushleft}


\begin{flushleft}
General description of yarn structure, Fibre packing arrangement in
\end{flushleft}


\begin{flushleft}
yarns, Fibre directional arrangement in yarns, Geometry of pores in
\end{flushleft}


\begin{flushleft}
yarns, Relationship among yarn count, twist, and diameter, Helical
\end{flushleft}


\begin{flushleft}
model of fibers in yarns, Yarn retraction, Limits of twisting, Radial
\end{flushleft}


\begin{flushleft}
migration of fibers in yarns, Model of ideal fibre migration, Model of
\end{flushleft}


\begin{flushleft}
equidistant migration, Tensile mechanics of yarns, Yarn tensile behavior
\end{flushleft}


\begin{flushleft}
in light of helical model, Relationship between tensile behaviors of
\end{flushleft}


\begin{flushleft}
fiber and yarn, Yarn strength as a function of gauge length, Bending
\end{flushleft}


\begin{flushleft}
mechanics of yarns, Mass unevenness of yarns, Martindale's model of
\end{flushleft}


\begin{flushleft}
mass irregularity, Model of hierarchical structure of fibre aggregates,
\end{flushleft}


\begin{flushleft}
Hairiness of staple fiber yarns, Single- and double-exponential models
\end{flushleft}


\begin{flushleft}
of yarn hairiness, Structure and mechanics of plied yarns.
\end{flushleft}





\begin{flushleft}
Draw-texturing- the need and fundamental approaches; Friction
\end{flushleft}


\begin{flushleft}
texturing- the need and development, mechanics of friction texturing,
\end{flushleft}


\begin{flushleft}
latest development in twisting devices, optimization of quality
\end{flushleft}


\begin{flushleft}
parameters. Noise control in texturing.
\end{flushleft}


\begin{flushleft}
Air jet texturing- Principle, mechanisms, development of jets
\end{flushleft}


\begin{flushleft}
and machinery, process optimization and characterization, air jet
\end{flushleft}


\begin{flushleft}
texturing of spun yarns. Air interlacement-Principle and mechanism,
\end{flushleft}


\begin{flushleft}
jet development and characterization. Bulked continuous filament
\end{flushleft}


\begin{flushleft}
yarns- Need, principle, technology development. Hi-bulk yarns- Acrylic
\end{flushleft}


\begin{flushleft}
Hi-bulk yarn production, mechanism and machines involved, other
\end{flushleft}


\begin{flushleft}
such products. Solvent and chemical texturing- Need, texturing of
\end{flushleft}


\begin{flushleft}
synthetic and natural fibres.
\end{flushleft}





\begin{flushleft}
TXL725 Mechanics of Spinning Machines
\end{flushleft}


\begin{flushleft}
3 Credits (3-0-0)
\end{flushleft}


\begin{flushleft}
Pre-requisites: TXL221/TXL222 and EC75
\end{flushleft}


\begin{flushleft}
Drive systems, belt drives, belt tensions, power transmission, variable,
\end{flushleft}


\begin{flushleft}
PIV and reversing drives. Polygonal effect in chain drives. Gear types,
\end{flushleft}


\begin{flushleft}
design aspects, interference and periodic faults, thrust loads and
\end{flushleft}


\begin{flushleft}
elimination, gear selection, planetary gear trains in spinning machines.
\end{flushleft}


\begin{flushleft}
Design of cone pulleys, design of transmission shafts and drafting
\end{flushleft}


\begin{flushleft}
rollers-materials, design against torsional \& lateral rigidity. Clutches
\end{flushleft}


\begin{flushleft}
and brakes (design, torque transmission capacity, applications in
\end{flushleft}


\begin{flushleft}
textile machines, bush bearings-theory of lubrication, Rolling contact
\end{flushleft}


\begin{flushleft}
bearings in textile machines. Machine balancing (static, couple,
\end{flushleft}


\begin{flushleft}
dynamic unbalance, balancing of cylinder-plane transposition, practical
\end{flushleft}


\begin{flushleft}
aspects of balancing. Cams in roving and ring spinning machines.
\end{flushleft}





\begin{flushleft}
TXP725 Mechanics of Textile Machines Laboratory
\end{flushleft}


\begin{flushleft}
1 Credit (0-0-2)
\end{flushleft}


\begin{flushleft}
Students will do experimental analysis of various machine elements
\end{flushleft}


\begin{flushleft}
on textile machines.
\end{flushleft}





\begin{flushleft}
TXL731 Theory of Fabric Structure
\end{flushleft}


\begin{flushleft}
3 Credits (3-0-0)
\end{flushleft}


\begin{flushleft}
Engineering approach to fabric formation. Fibre, yarn and fabric
\end{flushleft}


\begin{flushleft}
structure- property relationships. Crimp interchange in woven fabric.
\end{flushleft}


\begin{flushleft}
Elastica model for fabric parameters and crimp balance. Concept of
\end{flushleft}


\begin{flushleft}
fabric relaxation and set. Practical application of geometrical and
\end{flushleft}


\begin{flushleft}
elastica models, Uniaxial and biaxial tensile deformation of woven
\end{flushleft}


\begin{flushleft}
fabric. Bending deformation of woven fabric, bending behaviour of
\end{flushleft}


\begin{flushleft}
set and unset fabrics and bending in bias direction. Bending, Shear
\end{flushleft}


\begin{flushleft}
and drape properties of woven fabric. Buckling and compressional
\end{flushleft}


\begin{flushleft}
behaviour of woven fabrics. Mathematical models and their application
\end{flushleft}


\begin{flushleft}
in the study of tensile, bending, shear, compressional and buckling
\end{flushleft}


\begin{flushleft}
deformation of woven fabrics. Structure and properties of knitted
\end{flushleft}


\begin{flushleft}
fabrics, Structure-property relationship of nonwoven fabrics,
\end{flushleft}


\begin{flushleft}
Mechanical behavior of braided structures.
\end{flushleft}





\begin{flushleft}
TXL732 Advanced Fabric Manufacturing Systems
\end{flushleft}


\begin{flushleft}
3 Credits (3-0-0)
\end{flushleft}





\begin{flushleft}
TXL722 Mechanics of Spinning Processes
\end{flushleft}


\begin{flushleft}
3 Credits (3-0-0)
\end{flushleft}


\begin{flushleft}
Pre-requisites: TXL221/TXL222 and EC75
\end{flushleft}


\begin{flushleft}
Principles of bale management. Forces acting on fibres during opening
\end{flushleft}


\begin{flushleft}
and cleaning, analysis of fibre compactness and blending in blowroom.
\end{flushleft}


\begin{flushleft}
Carding process, cylinder load and transfer efficiency, design of
\end{flushleft}


\begin{flushleft}
high production card, fibre shedding and card wire geometry, fibre
\end{flushleft}


\begin{flushleft}
configuration in card and drawn sliver. Fibre movement in drafting
\end{flushleft}


\begin{flushleft}
field, drafting wave, drafting force, roller slip, roller eccentricity
\end{flushleft}


\begin{flushleft}
and vibration, autolevelling. Fibre fractionation in comber, combing
\end{flushleft}


\begin{flushleft}
performance. Analysis of forces on yarn and traveller, spinning tension
\end{flushleft}


\begin{flushleft}
in ring and rotor spinning, spinning geometry, twist flow in ring and
\end{flushleft}


\begin{flushleft}
rotor spinning, end breaks. Mechanism of drafting and yarn formation
\end{flushleft}


\begin{flushleft}
in high speed spinning systems.
\end{flushleft}





\begin{flushleft}
TXL724 Textured Yarn Technology
\end{flushleft}


\begin{flushleft}
3 Credits (3-0-0)
\end{flushleft}


\begin{flushleft}
Pre-requisites: TXL221/TXL222 and EC75
\end{flushleft}


\begin{flushleft}
Principles of texturing and modern classification; False twist texturing
\end{flushleft}


\begin{flushleft}
process- mechanisms and machinery, optimization of texturing
\end{flushleft}


\begin{flushleft}
parameters, barre', structure-property correlation of textured yarns;
\end{flushleft}





\begin{flushleft}
Fabric manufacturing systems, Yarn quality and weavability, Yarn
\end{flushleft}


\begin{flushleft}
Preparation for High speed weaving, Preparation of high performance
\end{flushleft}


\begin{flushleft}
fibres/tows for weaving, Sizing of filament yarn, Shuttle less weaving
\end{flushleft}


\begin{flushleft}
systems: Advancements in each system with respect to productivity,
\end{flushleft}


\begin{flushleft}
yarn characteristics and fabric quality, energy requirement, design
\end{flushleft}


\begin{flushleft}
flexibility, applications and limitations, Specialty weaving: 3D weaving,
\end{flushleft}


\begin{flushleft}
Multilayer weaving, Spacer weaving, Profiled weaving, Polar and Spiral
\end{flushleft}


\begin{flushleft}
fabric, Circular Weaving, Honeycomb weaving, Denim manufacturing,
\end{flushleft}


\begin{flushleft}
Multiaxial weaving, Multiphase weaving, Terry weaving, Leno Weaving,
\end{flushleft}


\begin{flushleft}
Filament Weaving, Properties and applications of fabrics produced
\end{flushleft}


\begin{flushleft}
in these systems. Weft and warp knitted structures for technical
\end{flushleft}


\begin{flushleft}
applications, Braiding; biaxial and triaxial braids, 3D braiding,
\end{flushleft}


\begin{flushleft}
Structure, properties and applications of braided fabrics, Developments
\end{flushleft}


\begin{flushleft}
in nonwoven technologies, Stitch bonding methods, Nonwoven
\end{flushleft}


\begin{flushleft}
composite fabrics, Electrospinning, 3D nonwovens.
\end{flushleft}





\begin{flushleft}
TXL734 Nonwoven Process and Products
\end{flushleft}


\begin{flushleft}
3 Credits (3-0-0)
\end{flushleft}


\begin{flushleft}
Pre-requisites: TXL211/TXL221/TXL222/TXL231/TXL232 and
\end{flushleft}


\begin{flushleft}
EC75
\end{flushleft}





289





\begin{flushleft}
\newpage
Textile
\end{flushleft}





\begin{flushleft}
Definitions of nonwoven and their scopes and limitations. Staple fibre
\end{flushleft}


\begin{flushleft}
preparation processes. Staple fibre web formation processes: carding,
\end{flushleft}


\begin{flushleft}
air-laying, and wet-laying. Staple fibre web stacking processes:
\end{flushleft}


\begin{flushleft}
parallel-laying, cross-laying, and perpendicular-laying, Mechanical
\end{flushleft}


\begin{flushleft}
bonding processes: needle-punching and hydroentanglement.
\end{flushleft}


\begin{flushleft}
Thermal bonding processes: calendar, through-air, impingement,
\end{flushleft}


\begin{flushleft}
infra-red, and ultrasonic bonding. Chemical bonding process.
\end{flushleft}


\begin{flushleft}
Spunmelt processes: spunbonding and meltblowing, Medical
\end{flushleft}


\begin{flushleft}
nonwovens, Hygiene nonwovens, Nonwoven wipes, Nonwoven filters,
\end{flushleft}


\begin{flushleft}
Geononwovens, Automotive nonwovens, Case studies.
\end{flushleft}





\begin{flushleft}
TXL740 Science \& App. of Nanotechnology in Textiles
\end{flushleft}


\begin{flushleft}
3 Credits (3-0-0)
\end{flushleft}


\begin{flushleft}
Pre-requisites: EC75
\end{flushleft}


\begin{flushleft}
Introduction to Nanoscience and Nanotechnology; Size and surface
\end{flushleft}


\begin{flushleft}
dependence of their physical and chemical properties such as
\end{flushleft}


\begin{flushleft}
mechanical, thermodynamical, electronic, catalysis etc; Synthesis of
\end{flushleft}


\begin{flushleft}
Nanomaterials used in Textiles such as carbon nanotube, fullerenes,
\end{flushleft}


\begin{flushleft}
metal and metal oxide nanoparticles i.e. nano silver, nano silica,
\end{flushleft}


\begin{flushleft}
nano titania, nano zinc oxide, nano magnesium oxide etc.; Surface
\end{flushleft}


\begin{flushleft}
functionalization and Dispersion of nanomaterials; Nanotoxicity,
\end{flushleft}


\begin{flushleft}
Characterization techniques i.e. XRD, AFM, SEM/TEM, DLS etc.;
\end{flushleft}


\begin{flushleft}
Nanomaterial applications in textiles and polymers; Nanocomposites:
\end{flushleft}


\begin{flushleft}
definition types, synthesis routes; nanocomposite fibres and coatings
\end{flushleft}


\begin{flushleft}
e.g. gas barrier, antimicrobial, conducting etc.; Nanofibres: preparation,
\end{flushleft}


\begin{flushleft}
properties and applications i.e. filtration, tissue engineering etc.;
\end{flushleft}


\begin{flushleft}
Nanofinishing: self-cleaning, antimicrobial, UV protective etc.;
\end{flushleft}


\begin{flushleft}
Nanocoating on textile substrates: Plasma Polymerisation, Layer-bylayer Self Assembly, Sol-Gel coating etc.
\end{flushleft}





\begin{flushleft}
retardant, water repellent. Principle of repellency, oil, water and soil,
\end{flushleft}


\begin{flushleft}
self-cleaning textiles. Wellness finishes for aroma, health and hygiene.
\end{flushleft}


\begin{flushleft}
New technologies - microencapsulation, plasma, nanotechnology.
\end{flushleft}


\begin{flushleft}
Finishing of technical textiles. Membranes and laminates.
\end{flushleft}





\begin{flushleft}
TXP 748 Textile Preparation and Finishing Lab
\end{flushleft}


\begin{flushleft}
1 Credit (0-0-2)
\end{flushleft}


\begin{flushleft}
Pre-requisites: TXL747/TXL753
\end{flushleft}


\begin{flushleft}
Preparatory and finishing related project based experiments, Chemistry
\end{flushleft}


\begin{flushleft}
and principle of each treatment and analysis of results.
\end{flushleft}





\begin{flushleft}
TXL 749 Theory and Practice of Dyeing
\end{flushleft}


\begin{flushleft}
3 Credits (3-0-0)
\end{flushleft}


\begin{flushleft}
Pre-requisites: EC 75
\end{flushleft}


\begin{flushleft}
Advances in dyes, Speciality dyes: photochromic, thermochromic,
\end{flushleft}


\begin{flushleft}
electrochromic, mechanochromic; Fluorescent and near IR dyes; Dyes
\end{flushleft}


\begin{flushleft}
for camouflage; Banned dyes; Safe and eco-friendly dyes, natural dyes;
\end{flushleft}


\begin{flushleft}
Mechanisms of dyeing; Thermodynamics of dyeing; Kinetics of dyeing;
\end{flushleft}


\begin{flushleft}
Dye-fibre interactions; Role of fibre structure in dyeing; Advances in
\end{flushleft}


\begin{flushleft}
dyeing processes: low liquor, salt free, low energy intensive dyeing;
\end{flushleft}


\begin{flushleft}
Dyeing of blends; Mass coloration of man-made fibres; Dyeing of
\end{flushleft}


\begin{flushleft}
speciality fabrics: stretch fabrics, light weight, textured, garment
\end{flushleft}


\begin{flushleft}
dyeing, micro-denier fabrics, fibre dyeing; Effect of finishes on shade
\end{flushleft}


\begin{flushleft}
and fastness; Dyeing faults and case studies.
\end{flushleft}





\begin{flushleft}
TXP 749 Textile Coloration Lab
\end{flushleft}


\begin{flushleft}
1 Credit (0-0-2)
\end{flushleft}


\begin{flushleft}
Pre-requisites: B Tech. Textile/ BE Textile/ MSc Textile
\end{flushleft}


\begin{flushleft}
Project based experiments in dyeing and colouration, dyeing of fabric,
\end{flushleft}


\begin{flushleft}
visual and instrumental assessment of shade variation. Subjective vs
\end{flushleft}


\begin{flushleft}
objective evaluation, Shade sorting, whiteness index. Azo dye synthesis
\end{flushleft}


\begin{flushleft}
and characterization.
\end{flushleft}





\begin{flushleft}
TXL741 Env. Manag. in Textile and Allied Industries
\end{flushleft}


\begin{flushleft}
3 Credits (3-0-0)
\end{flushleft}


\begin{flushleft}
Pre-requisites: TXL212/TXL241/TXL242 and EC 75
\end{flushleft}


\begin{flushleft}
Importance of ecological balance and environmental protection.
\end{flushleft}


\begin{flushleft}
Definition of waste and pollutant. Pollutant Categories and types.
\end{flushleft}


\begin{flushleft}
International and Indian legislation and enforcing agencies in pollution
\end{flushleft}


\begin{flushleft}
control. Waste management approaches; Environmental Management
\end{flushleft}


\begin{flushleft}
Systems' ISO 14000. Environmental impact along the textile chain from
\end{flushleft}


\begin{flushleft}
fibre production to disposal. Toxicity of intermediates, dyes and other
\end{flushleft}


\begin{flushleft}
auxiliaries etc. Pollution load from different wet processing operations.
\end{flushleft}


\begin{flushleft}
Textile effluents and their characterization. Technology and principles
\end{flushleft}


\begin{flushleft}
of effluent treatment. Advanced colour removal technologies, Recovery
\end{flushleft}


\begin{flushleft}
and reuse of water and chemicals. Air and noise pollution and its
\end{flushleft}


\begin{flushleft}
control. Eco labeling schemes. Industrial hygiene and safe working
\end{flushleft}


\begin{flushleft}
practices. Analytical testing of eco and environmental parameters.
\end{flushleft}


\begin{flushleft}
Eco friendly textile processing: waste minimization. Standardization
\end{flushleft}


\begin{flushleft}
and optimization, process modification. Safe \& ecofriendly dyes and
\end{flushleft}


\begin{flushleft}
auxiliaries. Organic cotton, natural dyes, naturally coloured cotton,
\end{flushleft}


\begin{flushleft}
Solid (fibre \& polymer waste) recycling recovery of monomers, energy
\end{flushleft}


\begin{flushleft}
recovery and chemical modification of fibre waste.
\end{flushleft}





\begin{flushleft}
TXL 747 Colour Science
\end{flushleft}


\begin{flushleft}
3 Credits (3-0-0)
\end{flushleft}


\begin{flushleft}
Pre-requisites: EC 75
\end{flushleft}


\begin{flushleft}
Colour and chemical constitution, physics and chemistry of colour,
\end{flushleft}


\begin{flushleft}
measurement of colour Colorimetry and CIE system, Qualities of
\end{flushleft}


\begin{flushleft}
Colorants, Colour-order systems, Colour Sensors, Physiology of Colour
\end{flushleft}


\begin{flushleft}
Vision, Visual and instrumental evaluation of whiteness, shade sorting,
\end{flushleft}


\begin{flushleft}
colour uncertainty.
\end{flushleft}





\begin{flushleft}
TXL750 Science of Clothing Comfort
\end{flushleft}


\begin{flushleft}
3 Credits (3-0-0)
\end{flushleft}


\begin{flushleft}
Pre-requisites: TXL211/TXL221/TXL222/TXL231/TXL232 and
\end{flushleft}


\begin{flushleft}
EC75
\end{flushleft}


\begin{flushleft}
Clothing Comfort: Brief introduction to the various processes related
\end{flushleft}


\begin{flushleft}
to comfort, Application of science of clothing comfort. Psychology and
\end{flushleft}


\begin{flushleft}
comfort: basic concepts, Psychological research techniques, General
\end{flushleft}


\begin{flushleft}
aspects and measurement of aesthetic properties, changes in aesthetic
\end{flushleft}


\begin{flushleft}
behaviour. Neurophysiological Processes of Comfort: Neurophysiologic
\end{flushleft}


\begin{flushleft}
basis of sensory perceptions, Perceptions of sensations related to
\end{flushleft}


\begin{flushleft}
mechanical, thermal and moisture stimuli. Thermal transmission:
\end{flushleft}


\begin{flushleft}
Thermoregulatory mechanisms of human body, heat transfer theories,
\end{flushleft}


\begin{flushleft}
thermal conductivity of fibrous materials, steady state measurement
\end{flushleft}


\begin{flushleft}
techniques for heat transfer, transient heat transfer mechanism:
\end{flushleft}


\begin{flushleft}
warm-cool feeling. Moisture Transmission: transfer of liquid moisture
\end{flushleft}


\begin{flushleft}
and vapour transfer through fibrous materials. Dynamic Transmission
\end{flushleft}


\begin{flushleft}
of heat and moisture: Relationship of moisture and heat, multiphase
\end{flushleft}


\begin{flushleft}
flow through porous media, moisture exchange between fibre and
\end{flushleft}


\begin{flushleft}
air, temperature and moisture sensations: theories and objective
\end{flushleft}


\begin{flushleft}
measurement techniques, impact of microclimate. Tactile Aspects of
\end{flushleft}


\begin{flushleft}
Comfort: Fabric mechanical properties and tactile- pressure sensations
\end{flushleft}


\begin{flushleft}
like fabric prickliness, itchiness, stiffness, softness, smoothness,
\end{flushleft}


\begin{flushleft}
roughness and scratchiness, fabric hand value, clothing comfort
\end{flushleft}


\begin{flushleft}
aspects in relations with garment size and fit.
\end{flushleft}





\begin{flushleft}
TXL748 Advances in Finishing of Textiles
\end{flushleft}


\begin{flushleft}
3 Credits (3-0-0)
\end{flushleft}


\begin{flushleft}
Pre-requisites: EC 75/TXL747/TXL753
\end{flushleft}





\begin{flushleft}
TXL751 Apparel Engineering and Quality Control
\end{flushleft}


\begin{flushleft}
3 Credits (2-0-2)
\end{flushleft}


\begin{flushleft}
Pre-requisites: TXL211/TXL221/TXL222/TXL231/TXL232 and
\end{flushleft}


\begin{flushleft}
EC75
\end{flushleft}





\begin{flushleft}
Overview of textile processing industry- current and future trends.
\end{flushleft}


\begin{flushleft}
Merging of technologies for creative solutions. Advances in
\end{flushleft}


\begin{flushleft}
preparatory processes- bioscouring, combined processes, bleaching
\end{flushleft}


\begin{flushleft}
and mercerisation. Reducing water and energy consumption Efficient liquor extraction, Low wet pick up and drying technologies.
\end{flushleft}


\begin{flushleft}
Classification of finishes. Advance in mechanical finishes - calendering,
\end{flushleft}


\begin{flushleft}
raising, emerising, softening. Principles and chemistry of Chemical
\end{flushleft}


\begin{flushleft}
finishes - easy care, antimicrobial, anti UV, antistat, softening, Flame
\end{flushleft}





\begin{flushleft}
Introduction to clothing manufacture, Apparel Engineering
\end{flushleft}


\begin{flushleft}
Concept in Garment Industry, Need of Apparel engineering, Role
\end{flushleft}


\begin{flushleft}
and Methodology of Apparel Engineering, Industrial engineering
\end{flushleft}


\begin{flushleft}
concept in apparel engineering, Standardization and Production
\end{flushleft}


\begin{flushleft}
scheduling, Sewing Dynamics, Mechanics of sewing operation,
\end{flushleft}


\begin{flushleft}
Measurement and controls in sewing operation, Automation in
\end{flushleft}


\begin{flushleft}
sewing process, Modeling of sewing machine and operation, Fabric
\end{flushleft}


\begin{flushleft}
quality assessment for clothing industry, Evaluation and Application
\end{flushleft}





290





\begin{flushleft}
\newpage
Textile
\end{flushleft}





\begin{flushleft}
of low stress mechanical properties for making up process, Fabric
\end{flushleft}


\begin{flushleft}
mechanical properties and sewing operation interaction, Concept
\end{flushleft}


\begin{flushleft}
of Tailorability, Formability and Lindberg theory, Quality control in
\end{flushleft}


\begin{flushleft}
apparel manufacturing, Determination of sewability, Effect of sewing
\end{flushleft}


\begin{flushleft}
on fabric mechanical and aesthetic properties, Fabric defects and
\end{flushleft}


\begin{flushleft}
their impact on garment quality, Quality inspection and defects
\end{flushleft}


\begin{flushleft}
in apparels, Evaluation of sewing threads, Evaluation of clothing
\end{flushleft}


\begin{flushleft}
accessories, Material Functionality in clothing, Engineering of
\end{flushleft}


\begin{flushleft}
functional clothing.
\end{flushleft}





\begin{flushleft}
TXP 751: Characterization of Chemicals and Finished
\end{flushleft}


\begin{flushleft}
Textiles Lab
\end{flushleft}


\begin{flushleft}
1 Credit (0-0-2)
\end{flushleft}


\begin{flushleft}
Pre-requisites: B Tech. Textile/ BE Textile/ MSc Textile
\end{flushleft}


\begin{flushleft}
Evaluation, characterization and analysis of textile auxiliaries,
\end{flushleft}


\begin{flushleft}
chemicals, dyes, and water, Project based experiments for evaluation
\end{flushleft}


\begin{flushleft}
of the dyed and finished textiles.
\end{flushleft}





\begin{flushleft}
TXS751: Research Seminar
\end{flushleft}


\begin{flushleft}
1 Credit (0-0-2)
\end{flushleft}


\begin{flushleft}
Pre-requisites: TXT800
\end{flushleft}


\begin{flushleft}
Presentation and discussion based on work done during internship or
\end{flushleft}


\begin{flushleft}
selected topics on current and future technologies.
\end{flushleft}





\begin{flushleft}
TXL752 Design of Functional Clothing
\end{flushleft}


\begin{flushleft}
3 Credits (3-0-0)
\end{flushleft}


\begin{flushleft}
Pre-requisites: TXL211/TXL221/TXL222/TXL231/TXL232 and
\end{flushleft}


\begin{flushleft}
EC75
\end{flushleft}


\begin{flushleft}
Functional clothing - definition and classification. Techniques in design
\end{flushleft}


\begin{flushleft}
of functional clothing - 3D body scanning, human motion analysis,
\end{flushleft}


\begin{flushleft}
2D/3D CAD and 3D modelling. Design of patterns, garment assembling
\end{flushleft}


\begin{flushleft}
methods. Ergonomics in design of functional clothing. Principles and
\end{flushleft}


\begin{flushleft}
practice of Anthropometrics. Biomechanical considerations in design of
\end{flushleft}


\begin{flushleft}
clothing. Performance evaluation of performance clothing - subjective
\end{flushleft}


\begin{flushleft}
and objective methods, modeling and simulation. Human mechanics
\end{flushleft}


\begin{flushleft}
and operational performance. Modelling, optimization and decision
\end{flushleft}


\begin{flushleft}
making techniques in design of functional clothing. Certification
\end{flushleft}


\begin{flushleft}
and standardization. Case studies - swimwear, sportswear, pressure
\end{flushleft}


\begin{flushleft}
garments, space suit, military clothing with a view to study specific
\end{flushleft}


\begin{flushleft}
design and manufacturing considerations.
\end{flushleft}





\begin{flushleft}
TXR752 Professional Practices
\end{flushleft}


\begin{flushleft}
1 Credit (0-0-2)
\end{flushleft}


\begin{flushleft}
Pre-requisites: EC 75
\end{flushleft}


\begin{flushleft}
Interaction and discussion with experts from industry and academia
\end{flushleft}


\begin{flushleft}
in the field of textiles and allied industries for sharing best practices
\end{flushleft}


\begin{flushleft}
followed in the industry including case studies, Exposure to a
\end{flushleft}


\begin{flushleft}
variety of topics and issues related to professional ethics.
\end{flushleft}





\begin{flushleft}
TXL753 Advanced Textile Printing Technology
\end{flushleft}


\begin{flushleft}
2 Credits (2-0-0)
\end{flushleft}


\begin{flushleft}
Pre-requisites: EC 75
\end{flushleft}


\begin{flushleft}
Historical development in textile printing techniques and machines;
\end{flushleft}


\begin{flushleft}
limitations thereof; theoretical concepts of transfer printing and
\end{flushleft}


\begin{flushleft}
scope; transfer printing inks, transfer paper, machines and process
\end{flushleft}


\begin{flushleft}
conditions; concept of digital printing, technology and challenges
\end{flushleft}


\begin{flushleft}
thereof, machines and principles, continuous jet verses drop-ondemand, suitability of inks for different class of fibre/ fabrics, auxiliaries
\end{flushleft}


\begin{flushleft}
needed, issues related to standardization, pre- and post-printing
\end{flushleft}


\begin{flushleft}
operations, scale and economics of operation. Printing faults and
\end{flushleft}


\begin{flushleft}
related process control principles, novel printing methods, raised,
\end{flushleft}


\begin{flushleft}
plasma, fancy, 3-D effects.
\end{flushleft}





\begin{flushleft}
TXL 754 Sustainable Chemical Processing of Textiles
\end{flushleft}


\begin{flushleft}
2 Credits (2-0-0)
\end{flushleft}


\begin{flushleft}
Pre-requisites: EC 75
\end{flushleft}





\begin{flushleft}
Process technologies using new enzymes, ozone, and foam technology,
\end{flushleft}


\begin{flushleft}
Low-salt reactive dyes, Combined dyeing and finishing, Industrial
\end{flushleft}


\begin{flushleft}
Hazardous Waste Management, in-plant management, reduction,
\end{flushleft}


\begin{flushleft}
recycling and disposal of waste, Laws related to environmental
\end{flushleft}


\begin{flushleft}
protection specially with reference to textile industry, Compliance,
\end{flushleft}


\begin{flushleft}
certification, social accountability and ethical practices.
\end{flushleft}





\begin{flushleft}
TXL 755 Textile Wet Processing Machines: Automation
\end{flushleft}


\begin{flushleft}
and Control
\end{flushleft}


\begin{flushleft}
3 Credits (3-0-0)
\end{flushleft}


\begin{flushleft}
Pre-requisites: EC 75
\end{flushleft}


\begin{flushleft}
Basic concepts of fluid flow, heat and mass transfer with specific
\end{flushleft}


\begin{flushleft}
emphasis on textile processes, Feedback control principles and
\end{flushleft}


\begin{flushleft}
systems, Sensors and transducers used in chemical processing
\end{flushleft}


\begin{flushleft}
machines; Machinery for processing of textiles in fibre, yarn and fabric
\end{flushleft}


\begin{flushleft}
form, batch and continuous machines. Machines for pre-treatment,
\end{flushleft}


\begin{flushleft}
dyeing, printing and finishing, developments in machinery for
\end{flushleft}


\begin{flushleft}
improving the effectiveness of treatment and reduction in chemical,
\end{flushleft}


\begin{flushleft}
energy and water consumption, mechanical finishing machines,
\end{flushleft}


\begin{flushleft}
garment processing.
\end{flushleft}





\begin{flushleft}
TXL756 Textile Auxiliaries
\end{flushleft}


\begin{flushleft}
3 Credits (3-0-0)
\end{flushleft}


\begin{flushleft}
Pre-requisites: EC 75
\end{flushleft}


\begin{flushleft}
Auxiliaries in textile chemical processing; Surfactants, emulsifiers,
\end{flushleft}


\begin{flushleft}
wetting agents, dispersing agents, foaming agents. Buffers,
\end{flushleft}


\begin{flushleft}
Electrolytes, Sequestering agents, enzymes, Sizing agents, thickeners,
\end{flushleft}


\begin{flushleft}
Binders, Fluorescent brightening agents, Oxidising and reducing
\end{flushleft}


\begin{flushleft}
agents, discharging agents, stain removing agents. Environmental
\end{flushleft}


\begin{flushleft}
assessment.
\end{flushleft}





\begin{flushleft}
TXP761 Evaluation of Textile Materials
\end{flushleft}


\begin{flushleft}
2 Credits (0-0-4)
\end{flushleft}


\begin{flushleft}
Evaluation of clothing comfort, flammability, bursting strength,
\end{flushleft}


\begin{flushleft}
bandage pressure, UPF, impact resistance, pore size and
\end{flushleft}


\begin{flushleft}
filtration efficiency.
\end{flushleft}





\begin{flushleft}
TXL766 Design and Manufacturing of Textile Structural
\end{flushleft}


\begin{flushleft}
Composites
\end{flushleft}


\begin{flushleft}
3 Credits (3-0-0)
\end{flushleft}


\begin{flushleft}
Pre-requisites: TXL211/TXL221/TXL222/TXL231/TXL232 and
\end{flushleft}


\begin{flushleft}
EC75
\end{flushleft}


\begin{flushleft}
Definition of composites, textile composites and textile structural
\end{flushleft}


\begin{flushleft}
composites, Textile materials for composites, Matrix and
\end{flushleft}


\begin{flushleft}
Reinforcements, Classification of Textile Reinforced Structures based
\end{flushleft}


\begin{flushleft}
on axis and dimension; non-axial, mono-axial, biaxial, triaxial and
\end{flushleft}


\begin{flushleft}
multiaxial structures, UD, 2D,3D structures, Structural anisotropy,
\end{flushleft}


\begin{flushleft}
parallel arrangement and series arrangement of components, Chopped
\end{flushleft}


\begin{flushleft}
strand and Milled fibres, Hybrid fabrics, Non-crimp fabrics, Laminates,
\end{flushleft}


\begin{flushleft}
Stitched structure, Embroidery structures, Composite Rope, Design,
\end{flushleft}


\begin{flushleft}
manufacture and applications of reinforcements, Manufacture
\end{flushleft}


\begin{flushleft}
and characterization of extra-light 3D hollow textile structures
\end{flushleft}


\begin{flushleft}
for composites, Methods of composite processing, Manufacturing
\end{flushleft}


\begin{flushleft}
techniques of complex structural Composites, Characterization of
\end{flushleft}


\begin{flushleft}
structural Composites, Theory of composites, Composite concepts
\end{flushleft}


\begin{flushleft}
and theory, Rule of mixture, the synergy effect, Logarthmic mixing
\end{flushleft}


\begin{flushleft}
rule, Geometry of reinforcement, Particular, granular, fibrillar, lamellar,
\end{flushleft}


\begin{flushleft}
Properties of components, properties of interface, mechanism of
\end{flushleft}


\begin{flushleft}
adhesion, Mechanics of composite, Failure theory, Damage analysis,
\end{flushleft}


\begin{flushleft}
Modeling and simulation of various reinforcement structures and
\end{flushleft}


\begin{flushleft}
their composites, Applications of Textile structural composites, Textile
\end{flushleft}


\begin{flushleft}
Reinforced Concretes, Fibre concrete bonding, textile structure
\end{flushleft}


\begin{flushleft}
reinforcement concrete architecture, Characterization and applications
\end{flushleft}


\begin{flushleft}
of reinforced concretes.
\end{flushleft}





\begin{flushleft}
TXL771 Electronics and Controls for Textile Industry
\end{flushleft}


\begin{flushleft}
4 Credits (3-0-2)
\end{flushleft}





\begin{flushleft}
Sustainability, Green Processing technologies, which require fewer
\end{flushleft}


\begin{flushleft}
chemicals, consume less energy and water and release cleaner
\end{flushleft}


\begin{flushleft}
effluent, Technologies using organic and natural fibers, Biocomposites,
\end{flushleft}





\begin{flushleft}
Overview of electronics and controls in modern textiles equipments
\end{flushleft}


\begin{flushleft}
and machines. Overview of basic analog electronics: Elements (R, L, C,
\end{flushleft}


\begin{flushleft}
V, I), circuit laws and theorems. Overview of basic digital electronics:
\end{flushleft}


\begin{flushleft}
Gates and ICs. Sensors and transducers (displacement, position,
\end{flushleft}





291





\begin{flushleft}
\newpage
Textile
\end{flushleft}





\begin{flushleft}
force, temperature, pressure, flow). Control elements, systems and
\end{flushleft}


\begin{flushleft}
examples. Data acquisition, analysis, control and automation by
\end{flushleft}


\begin{flushleft}
microprocessors and micro controllers. Motor and power drives. Power
\end{flushleft}


\begin{flushleft}
control devices. Some applications of data acquisitions and control
\end{flushleft}


\begin{flushleft}
systems in textiles and case studies.
\end{flushleft}





\begin{flushleft}
TXL772 Computational Methods for Textiles
\end{flushleft}


\begin{flushleft}
3 Credits (2-0-2)
\end{flushleft}


\begin{flushleft}
Numerical analysis, First-degree approximation methods, Linear
\end{flushleft}


\begin{flushleft}
algebraic equations, ordinary differential equations, interpolation,
\end{flushleft}


\begin{flushleft}
Fundamentals of Computer Programming, Programming Methodology:
\end{flushleft}


\begin{flushleft}
Structured Programming and concepts of Object-Oriented
\end{flushleft}


\begin{flushleft}
Programming. Programming in C++ - Statements and Expressions,
\end{flushleft}


\begin{flushleft}
Control statements. Structure, Functions: Function Overloading etc.
\end{flushleft}


\begin{flushleft}
C++ as Object-Oriented Programming Language- Classes and Objects,
\end{flushleft}


\begin{flushleft}
Data Abstraction, Inheritance - Multilevel and Multiple inheritance
\end{flushleft}


\begin{flushleft}
etc., Polymorphism - operator overloading and virtual functions, file
\end{flushleft}


\begin{flushleft}
handling. Application development using C++.
\end{flushleft}





\begin{flushleft}
TXL773 Medical Textiles
\end{flushleft}


\begin{flushleft}
3 Credits (3-0-0)
\end{flushleft}


\begin{flushleft}
Pre-requisites: TXL211/TXL221/TXL222/TXL231/TXL232 and
\end{flushleft}


\begin{flushleft}
EC75
\end{flushleft}


\begin{flushleft}
Natural and synthetic polymers and Textile-based techniques used for
\end{flushleft}


\begin{flushleft}
medical application, Fibrous extracellular matrix of human body and
\end{flushleft}


\begin{flushleft}
their characteristic features, Cell-Polymer interaction, Non-implantable
\end{flushleft}


\begin{flushleft}
materials (Wound-dressing, related hydrogel and composite products,
\end{flushleft}


\begin{flushleft}
Bandages, Gauges), Implantable biomedical devices (Vascular grafts,
\end{flushleft}


\begin{flushleft}
Sutures, Heart valves), Extra-corporeal materials (Scaffolds for Tissue
\end{flushleft}


\begin{flushleft}
engineering, Rapid prototyping, Cartilage, Liver, Blood Vessel, Kidney,
\end{flushleft}


\begin{flushleft}
Urinary bladder, Tendons, Ligaments, Cornea), Healthcare and hygiene
\end{flushleft}


\begin{flushleft}
products (Surgical Gowns, masks, wipes, Antibacterial Textiles, Super
\end{flushleft}


\begin{flushleft}
absorbent polymers, Dialysis, Soluble factor release), Safety, Legal
\end{flushleft}


\begin{flushleft}
and ethical issues involved in the medical textile materials.
\end{flushleft}





\begin{flushleft}
TXL774 Process Control in Yarn \& Fabric Manufacturing
\end{flushleft}


\begin{flushleft}
3 Credits (3-0-0)
\end{flushleft}


\begin{flushleft}
Pre-requisites: TXL211/TXL221/TXL222/TXL231/TXL232 and
\end{flushleft}


\begin{flushleft}
EC75
\end{flushleft}


\begin{flushleft}
Basics of automatic control, Statistical considerations in process
\end{flushleft}


\begin{flushleft}
control. Online and offline control measures in spinning. Control of
\end{flushleft}


\begin{flushleft}
yarn quality attributes. Spinning process performance. Post spinning
\end{flushleft}


\begin{flushleft}
problems. Control of winding, warping, sizing, weaving and knitting
\end{flushleft}


\begin{flushleft}
processes. Control of fabric defects and value loss. Yarn quality
\end{flushleft}


\begin{flushleft}
requirement and assessment for weaving.
\end{flushleft}





\begin{flushleft}
TXL775 Technical Textiles
\end{flushleft}


\begin{flushleft}
3 Credits (3-0-0)
\end{flushleft}


\begin{flushleft}
Pre-requisites: TXL211/TXL221/TXL222/TXL231/TXL232 and
\end{flushleft}


\begin{flushleft}
EC75
\end{flushleft}


\begin{flushleft}
Definition, classification, products, market overview and growth
\end{flushleft}


\begin{flushleft}
projections of technical textiles. Fibres, yarns and fabric structures in
\end{flushleft}


\begin{flushleft}
technical textiles and their relevant properties. Type and important
\end{flushleft}


\begin{flushleft}
characteristics of sewing threads. cords, ropes, braids and narrow
\end{flushleft}


\begin{flushleft}
fabrics. Textile and other filter media for dry and wet filtration.
\end{flushleft}


\begin{flushleft}
Fibre and fabric selection for filtration. Types and application of
\end{flushleft}


\begin{flushleft}
geosynthetics. Fibres and fabric selection criteria for geotextile
\end{flushleft}


\begin{flushleft}
applications. Application of textiles in automobiles. Requirement and
\end{flushleft}


\begin{flushleft}
design for pneumatic tyres, airbags and belts. Clothing requirements
\end{flushleft}


\begin{flushleft}
for thermal protection, ballistic protection. Materials used in bullet
\end{flushleft}


\begin{flushleft}
proof and cut resistant clothing. Material, method of production and
\end{flushleft}


\begin{flushleft}
areas of application of agrotextiles. Different types of fabrics used
\end{flushleft}


\begin{flushleft}
for packaging. Methods of production and properties of textiles used
\end{flushleft}


\begin{flushleft}
in these applications.
\end{flushleft}





\begin{flushleft}
TXL777 Product Design and Development
\end{flushleft}


\begin{flushleft}
3 Credits (3-0-0)
\end{flushleft}


\begin{flushleft}
Pre-requisites: TXL211/TXL221/TXL222/TXL231/TXL232 and
\end{flushleft}


\begin{flushleft}
EC75
\end{flushleft}


\begin{flushleft}
Introduction to product development, distinguishing features of
\end{flushleft}





\begin{flushleft}
textile products, and its classification. Generic product development
\end{flushleft}


\begin{flushleft}
process, identifying customer need and its analysis, development
\end{flushleft}


\begin{flushleft}
of specification, need to metric conversion. Concept generation
\end{flushleft}


\begin{flushleft}
methodology, concept selection. Material selection, performance
\end{flushleft}


\begin{flushleft}
characteristics of apparel, home textile and technical products,
\end{flushleft}


\begin{flushleft}
criterion for material selection. Role of fibre, yarn and fabric and
\end{flushleft}


\begin{flushleft}
finishing process on product performance. Industrial design,
\end{flushleft}


\begin{flushleft}
ergonomics and aesthetics, Product architecture, Anthropometric
\end{flushleft}


\begin{flushleft}
principles, fit. Principles of prototyping, 3 D computer modeling,
\end{flushleft}


\begin{flushleft}
free-form fabrication. Design options for improving properties and
\end{flushleft}


\begin{flushleft}
functional attributes of different products. Design logic for developing
\end{flushleft}


\begin{flushleft}
selected products, Analysis of products; Calculation of design
\end{flushleft}


\begin{flushleft}
parameters for a given end use, developing detail specification for
\end{flushleft}


\begin{flushleft}
each structural element.
\end{flushleft}





\begin{flushleft}
TXL781 Project Appraisal and Finance
\end{flushleft}


\begin{flushleft}
3 Credits (3-0-0)
\end{flushleft}


\begin{flushleft}
Pre-requisites: TXL211/TXL221/TXL/TXL231/TXL232 and EC75
\end{flushleft}


\begin{flushleft}
Introduction to Project Finance - Description of Project Finance
\end{flushleft}


\begin{flushleft}
Transaction, difference between corporate finance and project
\end{flushleft}


\begin{flushleft}
finance, Indian Financial system, Structuring the Project, Limited
\end{flushleft}


\begin{flushleft}
Resource Structures, Capital Investments : Importance \& Difficulties,
\end{flushleft}


\begin{flushleft}
CPV analysis, Financial statements, Financial statement analysis,
\end{flushleft}


\begin{flushleft}
Working capital management, Inventory management, Project
\end{flushleft}


\begin{flushleft}
cycle, Project Formulation, Project Appraisal, Financial appraisal,
\end{flushleft}


\begin{flushleft}
Economic Appraisal, Social Cost Benefit Analysis- Shadow Prices
\end{flushleft}


\begin{flushleft}
and Economic rate of return, Financing Projects, Sources of funding,
\end{flushleft}


\begin{flushleft}
Valuing Projects, NPV, IRR, MIRR, Real Options, Decision Trees and
\end{flushleft}


\begin{flushleft}
Monte Carlo Simulations, Financial Estimates \& projections, Technical
\end{flushleft}


\begin{flushleft}
Analysis, Market \& Demand Analysis, Investment Criteria, Cost of
\end{flushleft}


\begin{flushleft}
capital, Project Risk analysis, Sensitivity Analysis, Leverage analysis,
\end{flushleft}


\begin{flushleft}
Environment Appraisal of the project and Detailed Project Report,
\end{flushleft}


\begin{flushleft}
Case studies on Textile projects.
\end{flushleft}





\begin{flushleft}
TXL782 Prod. \& Operations Management in Textile
\end{flushleft}


\begin{flushleft}
Industry
\end{flushleft}


\begin{flushleft}
3 Credits (3-0-0)
\end{flushleft}


\begin{flushleft}
Pre-requisites: TXL211/TXL221/TXL222/TXL231/TXL232 and
\end{flushleft}


\begin{flushleft}
EC75
\end{flushleft}


\begin{flushleft}
Indian textile industry scenario. Textile Policy. Production
\end{flushleft}


\begin{flushleft}
andoperations management function. Operation strategy. Facility
\end{flushleft}


\begin{flushleft}
location and capacity planning. Production planning and control,
\end{flushleft}


\begin{flushleft}
aggregate planning, scheduling, PERT and CPM, product mix linear
\end{flushleft}


\begin{flushleft}
programming concepts. Inventory models, optimal order quantity,
\end{flushleft}


\begin{flushleft}
economic manufacturing batch size, classification of materials,
\end{flushleft}


\begin{flushleft}
materials requirement planning, Just in time concept. Supply chain
\end{flushleft}


\begin{flushleft}
Management. Maintenance management. Plant modernisation. Motion
\end{flushleft}


\begin{flushleft}
and time study. Job evaluation and incentive scheme. Productivity,
\end{flushleft}


\begin{flushleft}
partial and total productivity, machine, labour and energy productivity,
\end{flushleft}


\begin{flushleft}
efficiency and effectiveness, benchmarking, measure to increase
\end{flushleft}


\begin{flushleft}
productivity. Forecasting, methods of forecasting. Total quality
\end{flushleft}


\begin{flushleft}
management and Six Sigma. Product pricing. Financial and profit
\end{flushleft}


\begin{flushleft}
analysis, investment decisions. Management information system.
\end{flushleft}





\begin{flushleft}
TXL783 Design of Experiments and Statistical Techniques
\end{flushleft}


\begin{flushleft}
3 Credits (3-0-0)
\end{flushleft}


\begin{flushleft}
Pre-requisites: TXL211/TXL221/TXL222/TXL231/TXL232 and
\end{flushleft}


\begin{flushleft}
EC75
\end{flushleft}


\begin{flushleft}
Objectives, principles, terminologies, guidelines, and applications of
\end{flushleft}


\begin{flushleft}
design of experiments. Completely randomized design. Randomized
\end{flushleft}


\begin{flushleft}
block design. Latin square design. Two level and three level full
\end{flushleft}


\begin{flushleft}
factorial designs. Fractional factorial designs. Robust design. Mixture
\end{flushleft}


\begin{flushleft}
experiments. Central composite and Box-Behnken designs. Response
\end{flushleft}


\begin{flushleft}
surface methodology. Multi-response optimization. Analysis of
\end{flushleft}


\begin{flushleft}
variance. Statistical test of hypothesis. Analysis of multiple linear
\end{flushleft}


\begin{flushleft}
regression. Use of statistical software packages.
\end{flushleft}





\begin{flushleft}
TXL784 Supply Chain Management in Textile Industry
\end{flushleft}


\begin{flushleft}
3 Credits (3-0-0)
\end{flushleft}


\begin{flushleft}
Definition, objectives, stages and metrics of textile supply chain;
\end{flushleft}


\begin{flushleft}
Life cycle of textile products, demand and fashion forecasting,
\end{flushleft}





292





\begin{flushleft}
\newpage
Textile
\end{flushleft}





\begin{flushleft}
forecasting techniques, bull-whip effect, aggregate forecasting in
\end{flushleft}


\begin{flushleft}
apparel industry; Designing of textile supply chain network, make vs
\end{flushleft}


\begin{flushleft}
buy and location decisions of textile SCM, reverse logistics in textile
\end{flushleft}


\begin{flushleft}
SCM; Risk mitigation in global textile supply chain, coordination among
\end{flushleft}


\begin{flushleft}
fabric, apparel and accessories manufacturers, role of dominant
\end{flushleft}


\begin{flushleft}
power; Transportation and distribution strategies; Supplier selection
\end{flushleft}


\begin{flushleft}
in textile SCM, quantitative models; Lean, agile and leagile textile
\end{flushleft}


\begin{flushleft}
supply chains and their enablers, designing resilient textile supply
\end{flushleft}


\begin{flushleft}
chain; Push-pull supply chain, decoupling point in textile SCM; Green
\end{flushleft}


\begin{flushleft}
and low carbon textile supply chain; Case studies related to textile
\end{flushleft}


\begin{flushleft}
and apparel supply chains.
\end{flushleft}





\begin{flushleft}
TXT800: Industrial Summer Training
\end{flushleft}


\begin{flushleft}
Non-Credit Mandatory for TCP
\end{flushleft}


\begin{flushleft}
Pre-requisites: TXL747/TXL753/TXL749
\end{flushleft}


\begin{flushleft}
Non-Credit course. The students will be required to undergo
\end{flushleft}


\begin{flushleft}
summer internship in a textile industry and present the experience
\end{flushleft}


\begin{flushleft}
of internship.
\end{flushleft}





\begin{flushleft}
TXD801 Major Project Part-I (TXE)
\end{flushleft}


\begin{flushleft}
6 Credits (0-0-12)
\end{flushleft}


\begin{flushleft}
To learn about preparation of research plan and systematically carry
\end{flushleft}


\begin{flushleft}
out research project.
\end{flushleft}





\begin{flushleft}
TXD802 Major Project Part-I (TXF)
\end{flushleft}


\begin{flushleft}
6 Credits (0-0-12)
\end{flushleft}


\begin{flushleft}
To learn about preparation of research plan and systematically carry
\end{flushleft}


\begin{flushleft}
out research project.
\end{flushleft}





\begin{flushleft}
To learn about preparation of research plan and systematically carry
\end{flushleft}


\begin{flushleft}
out research project.
\end{flushleft}





\begin{flushleft}
TXD804 Major Project Part-II (TXF)
\end{flushleft}


\begin{flushleft}
12 Credits (0-0-24)
\end{flushleft}


\begin{flushleft}
To learn about preparation of research plan and systematically carry
\end{flushleft}


\begin{flushleft}
out research project.
\end{flushleft}





\begin{flushleft}
Project work related to the area.
\end{flushleft}





\begin{flushleft}
TXD806: Major Project Part II (TCP)
\end{flushleft}


\begin{flushleft}
12 Credits (0-0-24)
\end{flushleft}


\begin{flushleft}
Pre-requisites: TXL747/TXL748/TXL749/TXL753
\end{flushleft}


\begin{flushleft}
Project work related to the area.
\end{flushleft}





\begin{flushleft}
Student should undertake a research oriented activity including
\end{flushleft}


\begin{flushleft}
software development, machine design \& development, product \&
\end{flushleft}


\begin{flushleft}
process development, instrumentation and in-depth study of a subject
\end{flushleft}


\begin{flushleft}
which is outside the regular courses offered in the program. This study
\end{flushleft}


\begin{flushleft}
should be carried out under the guidance of a faculty member. The
\end{flushleft}


\begin{flushleft}
subject area chosen by the student should be sufficiently different
\end{flushleft}


\begin{flushleft}
from the area of major project being pursued by the student.
\end{flushleft}


\begin{flushleft}
The student should submit a detailed plan of work to the program
\end{flushleft}


\begin{flushleft}
coordinator before approval of registration for the course. The student
\end{flushleft}


\begin{flushleft}
registered for this course should give one mid-term presentation
\end{flushleft}


\begin{flushleft}
followed by a final presentation before a committee constituted by
\end{flushleft}


\begin{flushleft}
the program coordinator.
\end{flushleft}





\begin{flushleft}
TXS806 Independent Study (TTF)
\end{flushleft}


\begin{flushleft}
3 Credits (0-3-0)
\end{flushleft}


\begin{flushleft}
Student should undertake a research oriented activity including
\end{flushleft}


\begin{flushleft}
software development, machine design and development, product \&
\end{flushleft}


\begin{flushleft}
process development, instrumentation and in-depth study of a subject
\end{flushleft}


\begin{flushleft}
which is outside the regular courses offered in the program. This study
\end{flushleft}


\begin{flushleft}
should be carried out under the guidance of a faculty member. The
\end{flushleft}


\begin{flushleft}
subject area chosen by the student should be sufficiently different
\end{flushleft}


\begin{flushleft}
from the area of major project being pursued by the student.
\end{flushleft}


\begin{flushleft}
The student should submit a detailed plan of work to the program
\end{flushleft}


\begin{flushleft}
coordinator before approval of registration for the course. The student
\end{flushleft}


\begin{flushleft}
registered for this course should give one mid-term presentation
\end{flushleft}


\begin{flushleft}
followed by a final presentation before a committee constituted by
\end{flushleft}


\begin{flushleft}
the program coordinator.
\end{flushleft}





\begin{flushleft}
TXL807 Seminar (Textile Engineering)
\end{flushleft}


\begin{flushleft}
2 Credits (0-2-0)
\end{flushleft}





\begin{flushleft}
TXD803 Major Project Part-II (TXE)
\end{flushleft}


\begin{flushleft}
12 Credits (0-0-24)
\end{flushleft}





\begin{flushleft}
TXD805: Major Project Part I (TCP)
\end{flushleft}


\begin{flushleft}
6 Credits (0-0-12)
\end{flushleft}


\begin{flushleft}
Pre-requisites: TXL747/TXL748/TXL749/TXL753
\end{flushleft}





\begin{flushleft}
TXS805 Independent Study (Textile Engineering)
\end{flushleft}


\begin{flushleft}
3 Credits (0-3-0)
\end{flushleft}





\begin{flushleft}
A comprehensive literature review on a research topic of current
\end{flushleft}


\begin{flushleft}
interest or futuristic, pertaining to a textile process or product or
\end{flushleft}


\begin{flushleft}
technology. Student should perform a comprehensive literature review
\end{flushleft}


\begin{flushleft}
on a research topic of current interest or futuristic, pertaining to a
\end{flushleft}


\begin{flushleft}
textile process or product or technology. The student should give an
\end{flushleft}


\begin{flushleft}
outline of the review and get approval from the program coordinator
\end{flushleft}


\begin{flushleft}
for registration of this course. The student registered for this course
\end{flushleft}


\begin{flushleft}
should give one mid-term presentation followed by a final presentation
\end{flushleft}


\begin{flushleft}
before a committee constituted by the program coordinator.
\end{flushleft}





\begin{flushleft}
TXD809 Mini Project (Textile Engineering)
\end{flushleft}


\begin{flushleft}
4 Credits (0-0-8)
\end{flushleft}


\begin{flushleft}
This is an open ended course where the students are expected to
\end{flushleft}


\begin{flushleft}
design and develop a product or equipment or instrument relevant
\end{flushleft}


\begin{flushleft}
to the field of textile technology. In this process, the students are
\end{flushleft}


\begin{flushleft}
expected to demonstrate their ability to think on their own in design
\end{flushleft}


\begin{flushleft}
and development of hardware item. They are also expected to put
\end{flushleft}


\begin{flushleft}
down their thinking process in a report form with relevant literature
\end{flushleft}


\begin{flushleft}
background, methodology of design and development process
\end{flushleft}


\begin{flushleft}
and should have conducted some experiments with the developed
\end{flushleft}


\begin{flushleft}
hardware system. Finally, they need to present their work for the
\end{flushleft}


\begin{flushleft}
award of grade.
\end{flushleft}





293





\begin{flushleft}
\newpage
Centre for Applied Research in Electronics
\end{flushleft}


\begin{flushleft}
CRL601 Basics of Statistical Signal Analysis
\end{flushleft}


\begin{flushleft}
3 Credits (2-0-2)
\end{flushleft}


\begin{flushleft}
Fundamentals of signals and systems, LTI systems, convolution, Fourier
\end{flushleft}


\begin{flushleft}
transforms, Z- transform, sampling and Nyquist criteria, set \& probability
\end{flushleft}


\begin{flushleft}
theory, random variables, probability density / distribution functions,
\end{flushleft}


\begin{flushleft}
moments, characteristic and moment generating functions, transformation
\end{flushleft}


\begin{flushleft}
of a random variable, random process, stationarity, ergodicity.
\end{flushleft}


\begin{flushleft}
Lab experiments using MATLAB will be given to understand the
\end{flushleft}


\begin{flushleft}
practical aspects of these concepts.
\end{flushleft}





\begin{flushleft}
CRL611 Basics of RF and Microwaves
\end{flushleft}


\begin{flushleft}
3 Credits (2-0-2)
\end{flushleft}


\begin{flushleft}
Basic electromagnetics, plane waves and scattering, waveguide modes,
\end{flushleft}


\begin{flushleft}
Fourier series and transform, autocorrelation and power spectral
\end{flushleft}


\begin{flushleft}
density, holes and electrons in semiconductors, p-n junction.
\end{flushleft}





\begin{flushleft}
CRL621 Fundamentals of Semiconductor Devices
\end{flushleft}


\begin{flushleft}
3 Credits (3-0-0)
\end{flushleft}


\begin{flushleft}
Si Crystal structure, crystal planes and directions, band formation in
\end{flushleft}


\begin{flushleft}
semiconductors, direct and indirect gap semiconductors, E-k diagram,
\end{flushleft}


\begin{flushleft}
concept of {``}hole'' as charge particle, effective mass, carrier mobility,
\end{flushleft}


\begin{flushleft}
life time of carriers, recombination, doping of semiconductors, drift
\end{flushleft}


\begin{flushleft}
and diffusion currents in semiconductors, metal-semiconductor
\end{flushleft}


\begin{flushleft}
junctions, ohmic and non-ohmic contacts, Schottky diode, abrupt p-n
\end{flushleft}


\begin{flushleft}
junction, energy- band diagram, junction under zero-bias, forward bias
\end{flushleft}


\begin{flushleft}
and reverse bias; current calculations, break-down in p-n junction,
\end{flushleft}


\begin{flushleft}
diffused p-n junction; bipolar transistor: theory and operation;
\end{flushleft}


\begin{flushleft}
theory of MOS FET, ideal MOSFET, threshold voltage, sub-threshold
\end{flushleft}


\begin{flushleft}
conduction in MOSFET, C-V characteristics of MOS capacitor, shortchannel effects.
\end{flushleft}





\begin{flushleft}
CRL702 Architectures and Algorithms for DSP
\end{flushleft}


\begin{flushleft}
Systems
\end{flushleft}


\begin{flushleft}
4 Credits (2-0-4)
\end{flushleft}





\begin{flushleft}
Adaptive beamforming: Least mean squares algorithms; Recursive
\end{flushleft}


\begin{flushleft}
least squares; Generalized sidelobe canceler; Array geometries in
\end{flushleft}


\begin{flushleft}
higher dimensions: Rectangular arrays; Circular arrays; Spherical
\end{flushleft}


\begin{flushleft}
arrays; Cylindrical arrays.
\end{flushleft}





\begin{flushleft}
CRL706 Selected Topics in Radars and Sonars
\end{flushleft}


\begin{flushleft}
3 Credits (3-0-0)
\end{flushleft}


\begin{flushleft}
The Radar and Sonar Equations: Basic System Parameters; Radar and
\end{flushleft}


\begin{flushleft}
Sonar Applications; Signal Design for range and Doppler resolution:
\end{flushleft}


\begin{flushleft}
Ambiguity functions, waveforms for CTFM/FMCW, MTI Radar, Pulse
\end{flushleft}


\begin{flushleft}
Doppler Radar; Detection theory for target extraction from clutter/
\end{flushleft}


\begin{flushleft}
reverberation and noise (clutter/reverberation modeling); Synthetic
\end{flushleft}


\begin{flushleft}
Aperture Radar/Sonar; Target Tracking: active/passive, Monopulse
\end{flushleft}


\begin{flushleft}
Radar; Modern Techniques: thru-the-wall imaging, multi-static systems.
\end{flushleft}





\begin{flushleft}
CRL707 Human \& Machine Speech Communication
\end{flushleft}


\begin{flushleft}
3 Credits (3-0-0)
\end{flushleft}


\begin{flushleft}
Overview of human and machine speech communication: Applications;
\end{flushleft}


\begin{flushleft}
Speech signal measurement and representation. Speech science topics:
\end{flushleft}


\begin{flushleft}
Speech production and phonetics: Speech production mechanism;
\end{flushleft}


\begin{flushleft}
Articulatory and acoustic phonetics; Speech production model;
\end{flushleft}


\begin{flushleft}
International Phonetic Alphabet; Phonetic transcription; Hearing and
\end{flushleft}


\begin{flushleft}
perception. Speech signal analysis: Time domain analysis; Spectrum
\end{flushleft}


\begin{flushleft}
domain analysis; Spectrogram; Cepstrum domain analysis; Pitch
\end{flushleft}


\begin{flushleft}
estimation; Voicing analysis; Linear prediction analysis. Engineering
\end{flushleft}


\begin{flushleft}
applications: Speech coding; Speech quality assessment: Subjective
\end{flushleft}


\begin{flushleft}
and objective evaluation of quality; Automatic speech recognition:
\end{flushleft}


\begin{flushleft}
HMM; Language models; Keyword spotting; Text-to-speech synthesis:
\end{flushleft}


\begin{flushleft}
Concatenative and HMM speech synthesis; Prosody modification.
\end{flushleft}


\begin{flushleft}
The course will include audio demonstrations and require students to
\end{flushleft}


\begin{flushleft}
do practical exercises with recorded speech signals. An isolated word
\end{flushleft}


\begin{flushleft}
speech recognizer using open source resources shall be designed.
\end{flushleft}





\begin{flushleft}
CRL708 Sonar System Engineering
\end{flushleft}


\begin{flushleft}
3 Credits (3-0-0)
\end{flushleft}





\begin{flushleft}
Lectures:
\end{flushleft}


\begin{flushleft}
Introduction -- DSP Tasks and Applications, Real-time Signal Processing,
\end{flushleft}


\begin{flushleft}
Representation of DSP algorithms; Number Representations and
\end{flushleft}


\begin{flushleft}
Arithmetic Operations - Fixed point and floating point representations
\end{flushleft}


\begin{flushleft}
and arithmetic operations; Q notation; Digital Signal Processor
\end{flushleft}


\begin{flushleft}
Architectures -- CPU, Peripherals; Specific DSP processor architecture;
\end{flushleft}


\begin{flushleft}
DSP Instruction Set and Assembly Language Programming -- Instruction
\end{flushleft}


\begin{flushleft}
types; Parallel programming; Pipelining; Efficient programming; DSP
\end{flushleft}


\begin{flushleft}
Algorithms and their Efficient Implementation - a) Linear filtering; b)
\end{flushleft}


\begin{flushleft}
FFT and spectrum analysis; c) Scalar and vector quantization, source
\end{flushleft}


\begin{flushleft}
coding, linear prediction coding; d) Function generation; Software
\end{flushleft}


\begin{flushleft}
Design for Low Power Consumption.
\end{flushleft}





\begin{flushleft}
Introduction to Sonar applications, Units, Sonar Equations and their
\end{flushleft}


\begin{flushleft}
limitations, Propagation of sound, Transmission loss, Ambient Noise,
\end{flushleft}


\begin{flushleft}
Spatial Correlation, Directivity Index, Array Gain, Beam-patterns,
\end{flushleft}


\begin{flushleft}
Projector Source level, Reverberation, Scattering by targets, echo
\end{flushleft}


\begin{flushleft}
formation, Radiated Noise and Self Noise, Transmission and Reception
\end{flushleft}


\begin{flushleft}
modes, Dynamic Range Compression and Normalisation, Receiver
\end{flushleft}


\begin{flushleft}
Beamforming techniques, Sidelobe nulling, Detection Performance
\end{flushleft}


\begin{flushleft}
issues, Performance prediction, Sonar System Design examples.
\end{flushleft}





\begin{flushleft}
The DSP architecture and assembly language programming will
\end{flushleft}


\begin{flushleft}
be studied in lectures and laboratory with reference to a specific
\end{flushleft}


\begin{flushleft}
DSP processor.
\end{flushleft}





\begin{flushleft}
Introduction to High Resolution Underwater Imaging Applications,
\end{flushleft}


\begin{flushleft}
Sidescan Sonar principles, Sector Scan Sonar Principles: Principle
\end{flushleft}


\begin{flushleft}
of within-pulse scanning, role of grating lobe in sector coverage,
\end{flushleft}


\begin{flushleft}
Swept-frequency delay line scanning technique, Time-Delay-Integrate
\end{flushleft}


\begin{flushleft}
scanning technique, Modulation Scanning Technique: Multi-stage
\end{flushleft}


\begin{flushleft}
scanning, Spatial DFT-based imaging technique, True PhaseShift beamforming: Near-field focusing, Hilbert-transform based
\end{flushleft}


\begin{flushleft}
implementation, Synthetic Aperture Sonar: range migration issue,
\end{flushleft}


\begin{flushleft}
PRF limits, swath coverage, real beam pattern effects, tow-body
\end{flushleft}


\begin{flushleft}
precision issues, CTFM Sonar, Dual Demodulation CTFM Sonar PhaseDifference based SAS, Radial Projection method of imaging, Monopulse
\end{flushleft}


\begin{flushleft}
technique, Navigation: Doppler Log, JANUS system, Localization: LBL
\end{flushleft}


\begin{flushleft}
(Long baseline), SBL (Short baseline), SSBL/USBL (super/ultra short
\end{flushleft}


\begin{flushleft}
baseline), requirements of tracking and positioning systems, hyperbolic
\end{flushleft}


\begin{flushleft}
and spherical-based localization using pingers and transponders,
\end{flushleft}


\begin{flushleft}
Passive Inverse Synthetic Aperture for localizing radiated tonals from
\end{flushleft}


\begin{flushleft}
moving platforms, Underwater Acoustic Communication Modems and
\end{flushleft}


\begin{flushleft}
their applications.
\end{flushleft}





\begin{flushleft}
Laboratory:
\end{flushleft}


\begin{flushleft}
1. Basic DSP algorithms using MATLAB, 2. Familiarization with DSP
\end{flushleft}


\begin{flushleft}
kit, 3. Real-time filtering, 4. PN Sequence generation, 5. FFT, 6.
\end{flushleft}


\begin{flushleft}
Lab project.
\end{flushleft}





\begin{flushleft}
CRL704 Sensor Array Signal Processing
\end{flushleft}


\begin{flushleft}
3 Credits (3-0-0)
\end{flushleft}


\begin{flushleft}
Representation of space - time signals: Coordinate systems;
\end{flushleft}


\begin{flushleft}
propagating waves; wave number-frequency space; arrays and
\end{flushleft}


\begin{flushleft}
apertures; space-time random processes and their characterization;
\end{flushleft}


\begin{flushleft}
Signal modeling and optimal filters: AR, MA, ARMA models;
\end{flushleft}


\begin{flushleft}
Autocorrelation and power spectral density; linear MMSE estimator;
\end{flushleft}


\begin{flushleft}
optimum filters; Power spectrum estimation: Non-parametric and
\end{flushleft}


\begin{flushleft}
parametric methods; Arrays and spatial filters: Frequency-wavenumber
\end{flushleft}


\begin{flushleft}
response and beam patterns; ULA; Performance measures; Synthesis
\end{flushleft}


\begin{flushleft}
of linear arrays and apertures: Spectral weighting; array polynomials;
\end{flushleft}


\begin{flushleft}
pattern sampling in wavenumber space, minimum beamwidth for
\end{flushleft}


\begin{flushleft}
specified sidelobe levels, broadband arrays; Optimum beamforming:
\end{flushleft}


\begin{flushleft}
MVDR beamformers; MMSE beamformers; Eigenvector beamformers;
\end{flushleft}





\begin{flushleft}
CRL709 Underwater Electronic Systems
\end{flushleft}


\begin{flushleft}
3 Credits, (3-0-0)
\end{flushleft}





\begin{flushleft}
CRL711 CAD of RF and Microwave Circuits
\end{flushleft}


\begin{flushleft}
4 Credits (3-0-2)
\end{flushleft}


\begin{flushleft}
Review of basic microwave theory: Transmission lines-concepts
\end{flushleft}


\begin{flushleft}
of characteristics impedance, reflection coefficient, standing and
\end{flushleft}





294





\begin{flushleft}
\newpage
Applied Research in Electronics
\end{flushleft}





\begin{flushleft}
propagating waves, equivalent circuit. Smith chart, Network analysis:
\end{flushleft}


\begin{flushleft}
Z, ABCD, Y, T, S-parameters, Impedance matching technique,
\end{flushleft}


\begin{flushleft}
Implementation using simulators. Planar transmission lines. Filterslumped as well as distributed element realization, Implementation
\end{flushleft}


\begin{flushleft}
using simulators. Direction couplers and Power divider.
\end{flushleft}


\begin{flushleft}
Familiarization of photolithography process, mask making using
\end{flushleft}


\begin{flushleft}
intellicad and measurement using Automatic Network Analyzer in the
\end{flushleft}


\begin{flushleft}
laboratory classes. Design, optimization, fabrication and testing of
\end{flushleft}


\begin{flushleft}
Microstrip components and determining equivalent circuits.
\end{flushleft}





\begin{flushleft}
CRL712 RF and Microwave Active Circuits
\end{flushleft}


\begin{flushleft}
3 Credits (3-0-0)
\end{flushleft}


\begin{flushleft}
Microwave Amplifier theory and design. Theory and design of
\end{flushleft}


\begin{flushleft}
microwave phase shifters, switches and attenuator. Analysis of
\end{flushleft}


\begin{flushleft}
microwave mixers.
\end{flushleft}





\begin{flushleft}
CRL715 Radiating Systems for RF Communication
\end{flushleft}


\begin{flushleft}
3 Credits (3-0-0)
\end{flushleft}


\begin{flushleft}
Revision of Maxwell's equations,radiation, Poynting vector; antenna
\end{flushleft}


\begin{flushleft}
parameters like gain, radiation pattern, VSWR wire antennas -- dipole
\end{flushleft}


\begin{flushleft}
monopole; antenna arrays; aperture antennas and equivalence
\end{flushleft}


\begin{flushleft}
theorems; printed antennas, scattering.
\end{flushleft}





\begin{flushleft}
CRL722 RF and Microwave Solid State Devices
\end{flushleft}


\begin{flushleft}
3 Credits (3-0-0)
\end{flushleft}


\begin{flushleft}
Review of basics of semiconductor devices. Schottky diode, qualitative
\end{flushleft}


\begin{flushleft}
description, junction properties, I-V characteristics in forward and
\end{flushleft}


\begin{flushleft}
reverse biased diodes, high frequency application of Schottky diode,
\end{flushleft}


\begin{flushleft}
Schotty barrier gate FET. GaAs MESFET I-V characteristics, High
\end{flushleft}


\begin{flushleft}
Electron Mobility Transistor (HEMT), Hetro-structures, SOI technologies
\end{flushleft}


\begin{flushleft}
and MOSFETs, Fabrication technologies for GaAs MESFET, MBE, Ion
\end{flushleft}


\begin{flushleft}
Implantation. Pattern transfer at sub-micron level.
\end{flushleft}





\begin{flushleft}
CRL727 Introduction to Quantum Electron Devices
\end{flushleft}


\begin{flushleft}
3 Credits (3-0-0)
\end{flushleft}


\begin{flushleft}
The foundation of quantum electronics; Nanoscale resistors:
\end{flushleft}


\begin{flushleft}
quantum resistance, quantum conductance; Scattering at quantum
\end{flushleft}


\begin{flushleft}
levels: quantum contacts, quantum interference, Andrev scattering,
\end{flushleft}


\begin{flushleft}
spin-dependent scattering; Coulomb blockade, Resonant tunneling,
\end{flushleft}


\begin{flushleft}
Quantum capacitance, Single electron and tom transistors: coulomb
\end{flushleft}


\begin{flushleft}
blockade memory and logic devices, single electron invertors; Electron
\end{flushleft}


\begin{flushleft}
transport through single molecule: molecular transistors, memories
\end{flushleft}


\begin{flushleft}
and switches; Spinning of electron: spin valve and transistors, Subband quantum devices: quantum wells, wires an dots, sub band
\end{flushleft}


\begin{flushleft}
infrared and terahertz detectors; Quantum bit: quantum computers,
\end{flushleft}


\begin{flushleft}
different types of qubit, initialization, quantum manipulation, readout,
\end{flushleft}


\begin{flushleft}
charge qubit, phase and flux qubit, spin qubit.
\end{flushleft}





\begin{flushleft}
CRL729 Sensors and Transducers
\end{flushleft}


\begin{flushleft}
3 Credits (3-0-0)
\end{flushleft}


\begin{flushleft}
Introduction to sensors and transducers, basic parameters and
\end{flushleft}


\begin{flushleft}
principles and applications of various sensors and transducers in
\end{flushleft}


\begin{flushleft}
characterization of materials, devices, circuits and systems; Acoustic
\end{flushleft}


\begin{flushleft}
and Ultrasonic sensors and transducers; Magnetic and Electrical sensors
\end{flushleft}


\begin{flushleft}
and transducers; Thermal sensors and transducers; Radiation including
\end{flushleft}


\begin{flushleft}
Optical sensors and transducers; Smart Sensors for characterization of
\end{flushleft}


\begin{flushleft}
RF materials, devices, circuits and systems; Mechanical and Thermal
\end{flushleft}


\begin{flushleft}
Engineering issues for RF Modules/Instruments; Typical applications
\end{flushleft}


\begin{flushleft}
and use of transducers in systems/instruments.
\end{flushleft}





\begin{flushleft}
CRL731 Selected Topics in RFDT-I
\end{flushleft}


\begin{flushleft}
3 Credits (3-0-0)
\end{flushleft}


\begin{flushleft}
Advanced course on selected topics of relevance to the RFDT M.Tech. Program.
\end{flushleft}





\begin{flushleft}
CRL732 Selected Topics in RFDT-II
\end{flushleft}


\begin{flushleft}
3 Credits (3-0-0)
\end{flushleft}


\begin{flushleft}
Advanced course on selected topics of relevance to the RFDT M.Tech.
\end{flushleft}


\begin{flushleft}
Program.
\end{flushleft}





\begin{flushleft}
CRL724 RF and Microwave Measurements
\end{flushleft}


\begin{flushleft}
3 Credits (3-0-0)
\end{flushleft}


\begin{flushleft}
Theory of operation of network analyzer, and spectrum analyzer.
\end{flushleft}


\begin{flushleft}
VNA calibration, synthesized signal generation, noise measurement,
\end{flushleft}


\begin{flushleft}
measurement of antenna properties.
\end{flushleft}





\begin{flushleft}
CRL725 Technology of RF and Microwave Solid State
\end{flushleft}


\begin{flushleft}
Devices
\end{flushleft}


\begin{flushleft}
3 Credits (3-0-0)
\end{flushleft}


\begin{flushleft}
Review of semiconductor device processing technologies: process
\end{flushleft}


\begin{flushleft}
sequence development for a representative MOS technology, overview
\end{flushleft}


\begin{flushleft}
of oxidation, diffusion, mask making, pattern transfer, etching,
\end{flushleft}


\begin{flushleft}
metallization etc., process integration. Techniques of metallization:
\end{flushleft}


\begin{flushleft}
Introduction to vacuum systems. Sputtering (DC,RF and magnetron),
\end{flushleft}


\begin{flushleft}
e-beam evaporation for ohmic and Schottky. Contact formation,
\end{flushleft}


\begin{flushleft}
silicides for gate and interconnect. Fine line lithography process: optical
\end{flushleft}


\begin{flushleft}
lithography, x-ray and e-beam lithography, lift-off techniques. Wet and
\end{flushleft}


\begin{flushleft}
plasma assisted etching techniques, RIE, RIBE. Introduction to Ion
\end{flushleft}


\begin{flushleft}
Implantation, Molecular Beam Epitaxy. Chemical Vapour Deposition
\end{flushleft}


\begin{flushleft}
(epitaxial growth, polycrystalline, silicon, dielectric films, flow pressure
\end{flushleft}


\begin{flushleft}
and plasma chemical deposition), Atomic layer deposition. GaAs
\end{flushleft}


\begin{flushleft}
MESFET technology.
\end{flushleft}





\begin{flushleft}
CRL726 MEMS Design and Technology
\end{flushleft}


\begin{flushleft}
3 Credits (3-0-0)
\end{flushleft}


\begin{flushleft}
Introduction, origin and driving force for MEMS; extension of IC
\end{flushleft}


\begin{flushleft}
technologies for MEMS fabrication, major technologies for MEMSL:
\end{flushleft}


\begin{flushleft}
bulk and surface micromachining, LIGA process anisotropic etching
\end{flushleft}


\begin{flushleft}
of silicon, piezoresistive -piezoelectric effect, piezoresistive silicon
\end{flushleft}


\begin{flushleft}
based pressure sensor, capacitive pressure sensor, RF switch
\end{flushleft}


\begin{flushleft}
design, fabrication and characterization, actuation in MEMS, MEMS
\end{flushleft}


\begin{flushleft}
accelerometer design, fabrication, vibration sensor, energy harvesting
\end{flushleft}


\begin{flushleft}
devices, piezoelectric materials for MEMS, MEMS based RF and
\end{flushleft}


\begin{flushleft}
microwave circuits.
\end{flushleft}





\begin{flushleft}
CRL733 Selected Topics in RFDT-III
\end{flushleft}


\begin{flushleft}
3 Credits (3-0-0)
\end{flushleft}


\begin{flushleft}
Advanced course on selected topics of relevance to the RFDT M.Tech.
\end{flushleft}


\begin{flushleft}
Program.
\end{flushleft}





\begin{flushleft}
CRL734 Selected Topics in RFDT-IV
\end{flushleft}


\begin{flushleft}
3 Credits (3-0-0)
\end{flushleft}


\begin{flushleft}
Advanced course on selected topics of relevance to the RFDT M.Tech.
\end{flushleft}


\begin{flushleft}
Program.
\end{flushleft}





\begin{flushleft}
CRP718 RF and Microwave Measurement Lab
\end{flushleft}


\begin{flushleft}
4 Credits (1-0-6)
\end{flushleft}


\begin{flushleft}
Laboratory experiments based on network analyzer, spectrum
\end{flushleft}


\begin{flushleft}
analyzer, antenna pattern measurement, thermography, data
\end{flushleft}


\begin{flushleft}
acquisition and digitization.
\end{flushleft}





\begin{flushleft}
CRP723 Fabrication Techniques for RF and
\end{flushleft}


\begin{flushleft}
Microwave Devices
\end{flushleft}


\begin{flushleft}
3 Credits (1-0-4)
\end{flushleft}


\begin{flushleft}
Concept of process flow in IC fabrication, representative process flow for
\end{flushleft}


\begin{flushleft}
diode/MOSFET. High temperature processes;oxidation, diffusion, and
\end{flushleft}


\begin{flushleft}
annealing. Use of masks in IC fabrication, mask design and fabrication.,
\end{flushleft}


\begin{flushleft}
Photolithography processes. Chemical etching processes: dry and wet
\end{flushleft}


\begin{flushleft}
etching. Vacuum and vacuum systems. Thin films in IC processing,
\end{flushleft}


\begin{flushleft}
resistive evaporation, ebeam, RF and DC sputtering processes. Concept
\end{flushleft}


\begin{flushleft}
of test chip design and process parameter extraction. Practicals:
\end{flushleft}


\begin{flushleft}
Vacuum system, Thermal evaporation, DC/RF puttering, Mask
\end{flushleft}


\begin{flushleft}
making techniques: Coordinatograph/Photo-plotter first Reduction
\end{flushleft}


\begin{flushleft}
Camera, Step and Repeat process, Photolithography process,
\end{flushleft}


\begin{flushleft}
Etching techniques, Oxidation/diffusion processes, Diode fabrication,
\end{flushleft}


\begin{flushleft}
Band Pass filter fabrication, Measurement equipment calibration.
\end{flushleft}





295





\begin{flushleft}
\newpage
Applied Research in Electronics
\end{flushleft}





\begin{flushleft}
CRS735 Independent Study
\end{flushleft}


\begin{flushleft}
3 Credits (0-3-0)
\end{flushleft}


\begin{flushleft}
Advanced course on selected topics of relevance to the RFDT M.Tech.
\end{flushleft}


\begin{flushleft}
Program.
\end{flushleft}





\begin{flushleft}
CRV741 Acoustic Classification using Passive Sonar
\end{flushleft}


\begin{flushleft}
1 Credit (1-0-0)
\end{flushleft}


\begin{flushleft}
The challenges faced by a sonar designer, involved in developing
\end{flushleft}


\begin{flushleft}
underwater classification systems will be introduced and possible
\end{flushleft}


\begin{flushleft}
solutions will be discussed. The radiated noise characteristics from
\end{flushleft}


\begin{flushleft}
marine vessels and the unique characteristics of the acoustic signature
\end{flushleft}


\begin{flushleft}
with respect to the class of the marine platform will be presented.
\end{flushleft}


\begin{flushleft}
Recent research work has shown that classical homomorphic signal
\end{flushleft}


\begin{flushleft}
processing techniques and other channel inversion techniques can
\end{flushleft}


\begin{flushleft}
be used to significantly reduce the unwanted underwater channel
\end{flushleft}


\begin{flushleft}
distortions that otherwise affect the classifier performance drastically.
\end{flushleft}


\begin{flushleft}
The course shall provide insight into some of the methods that can
\end{flushleft}


\begin{flushleft}
improve sonar classification performance.
\end{flushleft}





\begin{flushleft}
CRV743 Special Module in Radio Frequency Design
\end{flushleft}


\begin{flushleft}
and Technology-II
\end{flushleft}


\begin{flushleft}
1 Credit (1-0-0)
\end{flushleft}


\begin{flushleft}
Advanced module on selected topics of relevance to the RFDT M.Tech.
\end{flushleft}


\begin{flushleft}
program.
\end{flushleft}





\begin{flushleft}
CRD802 Minor Project
\end{flushleft}


\begin{flushleft}
3 Credits (0-0-6)
\end{flushleft}


\begin{flushleft}
The project work shall be specific to each student.
\end{flushleft}





\begin{flushleft}
CRD811 Major Project-I
\end{flushleft}


\begin{flushleft}
6 Credits (0-0-12)
\end{flushleft}


\begin{flushleft}
The project work shall be specific to each student.
\end{flushleft}





\begin{flushleft}
CRD812 Major Project-II
\end{flushleft}


\begin{flushleft}
12 Credits (0-0-24)
\end{flushleft}


\begin{flushleft}
The project work shall be specific to each student.
\end{flushleft}





\begin{flushleft}
CRV742 Special Module in Radio Frequency Design
\end{flushleft}


\begin{flushleft}
and Technology-I
\end{flushleft}


\begin{flushleft}
1 Credit (1-0-0)
\end{flushleft}


\begin{flushleft}
Advanced module on selected topics of relevance to the RFDT M.Tech.
\end{flushleft}


\begin{flushleft}
program.
\end{flushleft}





\begin{flushleft}
CRD814 Major Project-III
\end{flushleft}


\begin{flushleft}
6 Credits (0-0-12)
\end{flushleft}


\begin{flushleft}
The project work shall be specific to each student.
\end{flushleft}





296





\begin{flushleft}
\newpage
Centre for Atmospheric Sciences
\end{flushleft}


\begin{flushleft}
ASL310 Fundamentals of Atmosphere and Ocean
\end{flushleft}


\begin{flushleft}
4 Credits (3-0-2)
\end{flushleft}


\begin{flushleft}
Composition of atmosphere and ocean, Thermodynamic state:
\end{flushleft}


\begin{flushleft}
distribution of temperature, density, pressure, water vapour,
\end{flushleft}


\begin{flushleft}
salinity, etc., Equations of state, Fundamental forces in the
\end{flushleft}


\begin{flushleft}
atmosphere and ocean: Pressure gradient, gravitational, Coriolis
\end{flushleft}


\begin{flushleft}
and frictional forces, Atmospheric chemistry: gas phase chemical
\end{flushleft}


\begin{flushleft}
reactions, tropospheric and stratospheric chemistry, Laws
\end{flushleft}


\begin{flushleft}
of motion in the rotating earth, geostrophic and hydrostatic
\end{flushleft}


\begin{flushleft}
balances, Thermodynamic laws and energy cycle : Radiation,
\end{flushleft}


\begin{flushleft}
conduction, convection and advection; adiabatic and diabatic
\end{flushleft}


\begin{flushleft}
cooling and warming, thermodynamic diagrams, General circulation
\end{flushleft}


\begin{flushleft}
in the atmosphere, Monsoons, Global ocean currents, unique
\end{flushleft}


\begin{flushleft}
characteristics of Indian Ocean circulation, Wave propagation:
\end{flushleft}


\begin{flushleft}
Gravity waves, Oceanic Tides, Surges and Tsunamis, AtmosphereOcean interaction: some examples of air-sea interaction.
\end{flushleft}





\begin{flushleft}
ASL320 Climate Change: Impacts, Adaptation and
\end{flushleft}


\begin{flushleft}
Mitigation
\end{flushleft}


\begin{flushleft}
4 Credits (3-0-2)
\end{flushleft}


\begin{flushleft}
Elements of physical climatology, climate variability; anthropogenic
\end{flushleft}


\begin{flushleft}
causes of climate change; concepts of radioactive forcing climate
\end{flushleft}


\begin{flushleft}
feedbacks and climate sensitivity; Observed climate record and paleo
\end{flushleft}


\begin{flushleft}
reconstruction, modeling aspects of the climate system; Carbon
\end{flushleft}


\begin{flushleft}
emission pathways, scenario development, climate simulations
\end{flushleft}


\begin{flushleft}
of the future; Socio-economic impacts, quantifying uncertainties,
\end{flushleft}


\begin{flushleft}
tipping points and irreversible changes; Observed and projected
\end{flushleft}


\begin{flushleft}
changes in weather, monsoons, teleconnections, extreme weather
\end{flushleft}


\begin{flushleft}
events, sea level in India; Climate hot spots, sector wise vulnerability
\end{flushleft}


\begin{flushleft}
and adaptation; Reducing greenhouse gas emissions, clean energy
\end{flushleft}


\begin{flushleft}
technologies, geoengineering options.
\end{flushleft}





\begin{flushleft}
ASD330 Mini Project
\end{flushleft}


\begin{flushleft}
6 Credits (0-0-12)
\end{flushleft}


\begin{flushleft}
ASL410 Numerical Simulation of Atmospheric and
\end{flushleft}


\begin{flushleft}
Oceanic Phenomena
\end{flushleft}


\begin{flushleft}
4 Credits (3-0-2)
\end{flushleft}


\begin{flushleft}
Density stratification in atmosphere and ocean, static stability,
\end{flushleft}


\begin{flushleft}
equations of motion of a rotating fluid, scale analysis, hydrostatic
\end{flushleft}


\begin{flushleft}
approximation, vorticity and divergence, a coordinate system for
\end{flushleft}


\begin{flushleft}
planetary scale motion, Saint-Venant (shallow-water) equations;
\end{flushleft}


\begin{flushleft}
meteorologically important waves, Rossby and vertically propagating
\end{flushleft}


\begin{flushleft}
waves; basic concepts of barotropic and baroclinic instability.
\end{flushleft}


\begin{flushleft}
Numerical methods: (a) Finite difference methods - advection
\end{flushleft}


\begin{flushleft}
equation, stability analysis, oscillation equations, (b) Galerkin Methods
\end{flushleft}


\begin{flushleft}
-- transform method, application of spectral and finite element methods
\end{flushleft}


\begin{flushleft}
to barotropic vorticity equation. Time integration schemes for the
\end{flushleft}


\begin{flushleft}
advection equation.
\end{flushleft}


\begin{flushleft}
Introduction to consequences of sound waves, surface gravity waves,
\end{flushleft}


\begin{flushleft}
internal gravity waves in weather prediction models. Boundary
\end{flushleft}


\begin{flushleft}
layers: Prandtl layer, Ekman layer, Monin-Obukhov similarity theory
\end{flushleft}


\begin{flushleft}
and surface layer, closure assumption, eddy diffusion and K-theory,
\end{flushleft}


\begin{flushleft}
one-dimensional models of boundary layer. Objective analysis and
\end{flushleft}


\begin{flushleft}
initialization: data preparation, need for initialization of numerical
\end{flushleft}


\begin{flushleft}
models; introductory dynamic and normal mode initialization,
\end{flushleft}


\begin{flushleft}
variational and 4-dimensional data assimilation
\end{flushleft}





\begin{flushleft}
ASL730 Introduction to Weather, Climate and Air
\end{flushleft}


\begin{flushleft}
Pollution (Not allowed for : Any program other than
\end{flushleft}


\begin{flushleft}
AST and ASZ)
\end{flushleft}


\begin{flushleft}
1 Credit (1-0-0)
\end{flushleft}


\begin{flushleft}
Overview of the discipline, history and landmarks, career options,
\end{flushleft}


\begin{flushleft}
weather vs climate, online resources; composition of the atmosphere,
\end{flushleft}


\begin{flushleft}
Greenhouse Effect, Ozone Hole, vertical structure of the atmosphere
\end{flushleft}


\begin{flushleft}
and oceans; energy in the atmosphere, mechanisms of radiative
\end{flushleft}


\begin{flushleft}
transfer; water in the atmosphere, origin and types of clouds and
\end{flushleft}


\begin{flushleft}
precipitation; atmospheric and oceanic motion, forces, major wind
\end{flushleft}





\begin{flushleft}
patterns and ocean currents, monsoons, local circulations, scales of
\end{flushleft}


\begin{flushleft}
motion; climate and climate change, IPCC; air pollution, pollutants,
\end{flushleft}


\begin{flushleft}
acid rain, plumes, effects of wind and stability, episodes; observation
\end{flushleft}


\begin{flushleft}
tools including AWS, radar, satellite; weather and climate models, NWP,
\end{flushleft}


\begin{flushleft}
chaos theory; field trip to IMD and Hindon AFB to see meteorological
\end{flushleft}


\begin{flushleft}
instruments in operation.
\end{flushleft}





\begin{flushleft}
ASP731 Data Analysis Methods for Atmospheric and
\end{flushleft}


\begin{flushleft}
Oceanic Sciences (Not allowed for : Any program other
\end{flushleft}


\begin{flushleft}
than AST and ASZ)
\end{flushleft}


\begin{flushleft}
2 Credits (0-0-4)
\end{flushleft}


\begin{flushleft}
Introduction to UNIX/LINUX, basic commands, file management;
\end{flushleft}


\begin{flushleft}
introduction to MATLAB, using Mathworks resources; MATLAB I/O
\end{flushleft}


\begin{flushleft}
with NetCDF, HDF and GRIB2; plotting 1, 2 and 3 dimensional
\end{flushleft}


\begin{flushleft}
weather/climate data and animations with MATLAB; Univariate \&
\end{flushleft}


\begin{flushleft}
bivariate statistics, mean/median/mode, variance/standard deviation,
\end{flushleft}


\begin{flushleft}
correlation, errors, regression; probability and distributions, how to
\end{flushleft}


\begin{flushleft}
frame and test a hypothesis, principles of statistical significance,
\end{flushleft}


\begin{flushleft}
using MATLAB functions to test hypotheses and estimate statistical
\end{flushleft}


\begin{flushleft}
significance; working with spatial weather/climate data, regridding
\end{flushleft}


\begin{flushleft}
meteorological station data, interpolation, map overlays; working with
\end{flushleft}


\begin{flushleft}
time-series, interpolation, estimating trend in weather/climate variables.
\end{flushleft}





\begin{flushleft}
ASL732 Mathematical and Computational Methods for
\end{flushleft}


\begin{flushleft}
Atmospheric and Oceanic Sciences (Not allowed for :
\end{flushleft}


\begin{flushleft}
Any program other than AST and ASZ)
\end{flushleft}


\begin{flushleft}
3 Credits (2-0-2)
\end{flushleft}


\begin{flushleft}
Elements of FORTRAN programming; Initial and boundary value
\end{flushleft}


\begin{flushleft}
problems; second order ordinary differential equations, variation
\end{flushleft}


\begin{flushleft}
of parameters, orthogonal functions; Partial differential equations
\end{flushleft}


\begin{flushleft}
and their classification, method of separation of variables; Euler and
\end{flushleft}


\begin{flushleft}
RungeKutta methods for ODE; Spatial and temporal finite differencing
\end{flushleft}


\begin{flushleft}
schemes of various orders, comparison with exact solutions, accuracy
\end{flushleft}


\begin{flushleft}
and numerical stability, limitation of finite difference methods;
\end{flushleft}


\begin{flushleft}
Numerical solution of linear advection equation, advection-diffusion
\end{flushleft}


\begin{flushleft}
equation, and shallow water equation.
\end{flushleft}





\begin{flushleft}
ASL733 Physics of the Atmosphere
\end{flushleft}


\begin{flushleft}
3 Credits (3-0-0)
\end{flushleft}


\begin{flushleft}
Structure of the atmosphere; Hydrostatic equilibrium, Geopotential,
\end{flushleft}


\begin{flushleft}
Hypsometric equation and scale height, Altimetry; Adiabatic processes,
\end{flushleft}


\begin{flushleft}
Lapse rates, Static stability, dynamic stability; Atmospheric Boundary
\end{flushleft}


\begin{flushleft}
Layer Structure and evolution, turbulence etc.
\end{flushleft}


\begin{flushleft}
Atmospheric Thermodynamics: Thermodynamic laws; Thermodynamics
\end{flushleft}


\begin{flushleft}
of water vapour and moist air: Moisture parameters, Saturated
\end{flushleft}


\begin{flushleft}
adiabatic and Pseudoadiabatic processes, Conditional and convective
\end{flushleft}


\begin{flushleft}
instability, Free and forced convection; Thermodynamic diagrams;
\end{flushleft}


\begin{flushleft}
Phase change and Clausius-Clapeyron equation; Clouds: Formation
\end{flushleft}


\begin{flushleft}
and classification, Precipitation; Atmospheric visibility: Dew, Frost
\end{flushleft}


\begin{flushleft}
and fog, smog etc.
\end{flushleft}


\begin{flushleft}
The fundamental physics of radiation: solar and terrestrial
\end{flushleft}


\begin{flushleft}
radiation, radiation laws; absorption, emission and scattering in
\end{flushleft}


\begin{flushleft}
the atmosphere, Schwarzchild's equation; Radiation in the earthatmosphere system: Geographical and seasonal distribution,
\end{flushleft}


\begin{flushleft}
Radiative heating and cooling of the atmosphere, Surface energy
\end{flushleft}


\begin{flushleft}
budget, The mean annual heat balance.
\end{flushleft}





\begin{flushleft}
ASL734 Dynamics of the Atmosphere
\end{flushleft}


\begin{flushleft}
3 Credits (3-0-0)
\end{flushleft}


\begin{flushleft}
Fundamental forces; basic laws of conservation; hydrodynamic
\end{flushleft}


\begin{flushleft}
equations in rotating frame of reference; dimensional analysis;
\end{flushleft}


\begin{flushleft}
geostrophic and hydrostatic approximations; Atmospheric stability;
\end{flushleft}


\begin{flushleft}
Isobaric coordinate system; Gradient wind approximation; thermal
\end{flushleft}


\begin{flushleft}
wind; vertical motion; barotropic and baroclinic atmospheres;
\end{flushleft}


\begin{flushleft}
Circulation and vorticity; vorticity equation; potential vorticity
\end{flushleft}


\begin{flushleft}
conservation. Boussinesq approximation; Reynolds averaging; mixing
\end{flushleft}


\begin{flushleft}
length hypothesis; Ekman layer; Acoustic, gravity, Poincare, Rossby
\end{flushleft}


\begin{flushleft}
and Kelvin waves. Atmospheric general circulation.
\end{flushleft}





297





\begin{flushleft}
\newpage
Atmospheric Sciences
\end{flushleft}





\begin{flushleft}
ASL735 Atmospheric Chemistry and Air Pollution
\end{flushleft}


\begin{flushleft}
3 Credits (3-0-0)
\end{flushleft}


\begin{flushleft}
Atmospheric Composition and air pollutants, Geochemical cycles:
\end{flushleft}


\begin{flushleft}
Evolution of the atmosphere and geochemical cycling of elements;
\end{flushleft}


\begin{flushleft}
Atmospheric photochemistry; Chemistry of the troposphere: Basic
\end{flushleft}


\begin{flushleft}
photochemical cycle, atmospheric chemistry dealing with various
\end{flushleft}


\begin{flushleft}
pollutant species and photochemical smog; Oxidising power of the
\end{flushleft}


\begin{flushleft}
troposphere and the Hydroxyl radical, global budgets of precursor
\end{flushleft}


\begin{flushleft}
species; Stratospheric Chemistry and Ozone: Overview, Chapman
\end{flushleft}


\begin{flushleft}
mechanism, reservoir species and catalytic cycles, Ozone hole
\end{flushleft}


\begin{flushleft}
and polar stratospheric clouds, Arctic Ozone loss, Ozone depletion
\end{flushleft}


\begin{flushleft}
potential; Aqueous phase atmospheric chemistry and acid rain;
\end{flushleft}


\begin{flushleft}
Atmospheric Aerosols: sources and characteristics, radiative effects
\end{flushleft}


\begin{flushleft}
and perturbation to climate; Atmospheric air pollutants: sources,
\end{flushleft}


\begin{flushleft}
impacts and standards; Air Pollution Meteorology: sources of air
\end{flushleft}


\begin{flushleft}
pollutants, classification and air quality standards, stability conditions,
\end{flushleft}


\begin{flushleft}
wind velocity profile, turbulence, mixing depth, characteristics of
\end{flushleft}


\begin{flushleft}
stack plumes; Dispersion of pollutants in the atmosphere: A Gaussian
\end{flushleft}


\begin{flushleft}
dispersion model, dispersion parameters and effective stack height.
\end{flushleft}





\begin{flushleft}
ASL736 Science of Climate Change
\end{flushleft}


\begin{flushleft}
3 Credits (3-0-0)
\end{flushleft}


\begin{flushleft}
Description of the climate system (General circulation, hydrological
\end{flushleft}


\begin{flushleft}
cycle, carbon cycle). Natural greenhouse effect and the effect of trace
\end{flushleft}


\begin{flushleft}
gases and aerosols. Forcings (natural \& anthropogenic), Fast and Slow
\end{flushleft}


\begin{flushleft}
Feedbacks, Equilibrium Climate Sensitivity, Transient Climate Response.
\end{flushleft}


\begin{flushleft}
Climates of the past (ice ages, proxy records, abrupt climate change,
\end{flushleft}


\begin{flushleft}
instrumental record of climate). Climate variability and time-scales;
\end{flushleft}


\begin{flushleft}
MJO, ENSO, PDO, Milankovic cycles. Modeling climate: Simple EBMs,
\end{flushleft}


\begin{flushleft}
Coupled Climate Models. Natural and Anthropogenic climate change.
\end{flushleft}


\begin{flushleft}
Future climate projections.
\end{flushleft}





\begin{flushleft}
ASL737 Physical and Dynamical Oceanography
\end{flushleft}


\begin{flushleft}
3 Credits (3-0-0)
\end{flushleft}


\begin{flushleft}
Properties of sea water; temperature and salinity distributions;
\end{flushleft}


\begin{flushleft}
stratification and stability of oceanic water column; equation of state
\end{flushleft}


\begin{flushleft}
of sea water; oceanic mixed layer processes; governing equations
\end{flushleft}


\begin{flushleft}
for oceanic motions; inertial and geostrophic currents; wind-driven
\end{flushleft}


\begin{flushleft}
circulation; thermohaline circulation; Barotropic and baroclinic
\end{flushleft}


\begin{flushleft}
transports; western boundary intensification; gyres and meso-scale
\end{flushleft}


\begin{flushleft}
eddies; gyre systems, major currents in world oceans; Indian ocean
\end{flushleft}


\begin{flushleft}
circulation; physics and dynamics of ocean wind waves, internal waves
\end{flushleft}


\begin{flushleft}
and tides; coastal ocean processes; upwelling and downwelling in
\end{flushleft}


\begin{flushleft}
coastal and equatorial oceans; Rossby and Kelvin waves, biological
\end{flushleft}


\begin{flushleft}
productivity of oceans; heat and salt budget of oceans; observational
\end{flushleft}


\begin{flushleft}
methods in oceans; storm surges, ENSO and IOD phenomenon.
\end{flushleft}





\begin{flushleft}
ASL738 Numerical Modeling of the Atmosphere and
\end{flushleft}


\begin{flushleft}
Ocean (Not allowed for : Any program other than AST
\end{flushleft}


\begin{flushleft}
and ASZ)
\end{flushleft}


\begin{flushleft}
3 Credits (2-0-2)
\end{flushleft}


\begin{flushleft}
Introduction to weather and climate models, Numerical Modeling Vs.
\end{flushleft}


\begin{flushleft}
Other Modeling Approaches, Examples of atmospheric and oceanic
\end{flushleft}


\begin{flushleft}
simulations, Model Hierarchy (Simple, Intermediate, Complex);
\end{flushleft}


\begin{flushleft}
Governing equations in Cartesian, Isobaric and sigma coordinate
\end{flushleft}


\begin{flushleft}
systems; Numerical discretization (finite difference, finite volume,
\end{flushleft}


\begin{flushleft}
spectral) and integration, stability, CFL criterion, unconditionally stable
\end{flushleft}


\begin{flushleft}
numerical scheme; model components, dynamical core, physical
\end{flushleft}


\begin{flushleft}
parameterization, tracers, coupling of components; global and regional
\end{flushleft}


\begin{flushleft}
models used in weather forecasting and climate simulations.
\end{flushleft}





\begin{flushleft}
ASL750 Boundary Layer Meteorology
\end{flushleft}


\begin{flushleft}
3 Credits (3-0-0)
\end{flushleft}


\begin{flushleft}
Introduction to the boundary layer, definition and qualitative
\end{flushleft}


\begin{flushleft}
description of temporal evolution and vertical structure; Fourier series
\end{flushleft}


\begin{flushleft}
and turbulence spectra, Reynold's averaging, interpreting variance/
\end{flushleft}


\begin{flushleft}
covariance as turbulent kinetic energy and fluxes, tensors and Einstein
\end{flushleft}


\begin{flushleft}
summation notation; Prognostic equations for mean variables in a
\end{flushleft}


\begin{flushleft}
turbulent flow, simplifications; Prognostic equations for turbulent
\end{flushleft}


\begin{flushleft}
fluxes and variances; TKE equation, static and dynamic instability,
\end{flushleft}


\begin{flushleft}
Reynold's number, Richardson number, Obukhov length, stability
\end{flushleft}





\begin{flushleft}
parameter relationships, closure problem in turbulent flow, first-order
\end{flushleft}


\begin{flushleft}
local closure; surface boundary conditions, surface momentum, energy
\end{flushleft}


\begin{flushleft}
and moisture budgets, fluxes at surface and entrainment zone, drag
\end{flushleft}


\begin{flushleft}
and Bowen ratio methods; surface layer Similarity Theory, Buckingham
\end{flushleft}


\begin{flushleft}
Pi method, applications to wind profiles; Stable and convective mixed
\end{flushleft}


\begin{flushleft}
layer phenomena including nocturnal jets, thermals, dust devils;
\end{flushleft}


\begin{flushleft}
boundary layer clouds, fair-weather cumulus, fog; geographically
\end{flushleft}


\begin{flushleft}
generated local circulations like slope and valley winds, sea/lake
\end{flushleft}


\begin{flushleft}
breeze, geographically modified flow, fetch, internal boundary layer.
\end{flushleft}





\begin{flushleft}
ASL751 Dispersion of Air Pollutants
\end{flushleft}


\begin{flushleft}
3 Credits (3-0-0)
\end{flushleft}


\begin{flushleft}
Air Pollution, Various types, sources and effects of pollutants in
\end{flushleft}


\begin{flushleft}
the atmospheric environment; Particulate matter and atmospheric
\end{flushleft}


\begin{flushleft}
visibility; Atmospheric diffusion theories and types of dispersion
\end{flushleft}


\begin{flushleft}
models; Lapse rates and various types of stability classification,
\end{flushleft}


\begin{flushleft}
Wind-profile ,Wind rose, Mixing Depth, General characteristics of the
\end{flushleft}


\begin{flushleft}
stack plumes; Dispersion of pollutants in the atmosphere and solution
\end{flushleft}


\begin{flushleft}
of advection diffusion equation with Gaussian distribution for point,
\end{flushleft}


\begin{flushleft}
line and area sources, plume rise, dispersion parameters and various
\end{flushleft}


\begin{flushleft}
methods of their evaluation; Atmospheric Removal processes and
\end{flushleft}


\begin{flushleft}
residence time; Effect of buildings and topography on dispersion;
\end{flushleft}


\begin{flushleft}
Similarity theory and profiles in the surface layer; Air Quality and
\end{flushleft}


\begin{flushleft}
Emission standards, their measurements and statistics; Introduction
\end{flushleft}


\begin{flushleft}
of air quality models for regulatory applications.
\end{flushleft}





\begin{flushleft}
ASL752 Mesoscale Meteorology
\end{flushleft}


\begin{flushleft}
3 Credits (3-0-0)
\end{flushleft}


\begin{flushleft}
Overview of mesoscale phenomena relevant to India including
\end{flushleft}


\begin{flushleft}
tornadoes, thunderstorms, cloud bursts, fog, extreme rain events,
\end{flushleft}


\begin{flushleft}
lightning, etc; Circulation systems related to orography, mountain
\end{flushleft}


\begin{flushleft}
drag, mountain waves, valley winds, valley air pollution; Adiabatic
\end{flushleft}


\begin{flushleft}
mesoscale perturbations in a straight atmospheric flow; Theory of
\end{flushleft}


\begin{flushleft}
linear gravity waves, orographic gravity-wave drag; Parameterization
\end{flushleft}


\begin{flushleft}
of mesoscale phenomena in general circulation models; Mesoscale
\end{flushleft}


\begin{flushleft}
models and their application in India.
\end{flushleft}





\begin{flushleft}
ASL753 Atmospheric Aerosols
\end{flushleft}


\begin{flushleft}
3 Credits (3-0-0)
\end{flushleft}


\begin{flushleft}
Introduction to atmospheric aerosols; Characterization of Aerosols;
\end{flushleft}


\begin{flushleft}
Physical and Optical properties of aerosols, size distribution, refractive
\end{flushleft}


\begin{flushleft}
indices of aerosols, absorption and scattering of radiation by aerosols;
\end{flushleft}


\begin{flushleft}
single scattering albedo, aerosol optical depth, aerosol phase
\end{flushleft}


\begin{flushleft}
function, hygroscopic growth; mixing state, vertical distribution in the
\end{flushleft}


\begin{flushleft}
atmosphere; Aerosol Chemical Composition; mixing state of aerosols;
\end{flushleft}


\begin{flushleft}
New particle formation; volatile chemical compounds and gas-toparticle conversion processes; Observations and Measurements of
\end{flushleft}


\begin{flushleft}
aerosols; Climatology of Tropospheric Aerosols; Stratospheric aerosols;
\end{flushleft}


\begin{flushleft}
Dynamics of single aerosol particle and aerosol population; Transport
\end{flushleft}


\begin{flushleft}
and transformation of aerosols; Removal of aerosols; Thermodynamics
\end{flushleft}


\begin{flushleft}
of aerosols; Role in Nucleation; Role in Cloud Physics; Interaction of
\end{flushleft}


\begin{flushleft}
aerosols with radiation; Direct, indirect, and semi-direct effects of
\end{flushleft}


\begin{flushleft}
aerosols and their influence on Climate; Aerosol effects on human
\end{flushleft}


\begin{flushleft}
health and air quality; Aerosols in chemistry transport models;
\end{flushleft}


\begin{flushleft}
Aerosols in climate models; Latest trends in aerosol research and
\end{flushleft}


\begin{flushleft}
future directions.
\end{flushleft}





\begin{flushleft}
ASL754 Cloud Physics
\end{flushleft}


\begin{flushleft}
3 Credits (3-0-0)
\end{flushleft}


\begin{flushleft}
Cloud types; cloud formation; cloud dynamics: entrainment,
\end{flushleft}


\begin{flushleft}
detrainment and downdraft initiation in cumuli, large scale
\end{flushleft}


\begin{flushleft}
convergence, mesoscale convective system; Kohler theory; CCN
\end{flushleft}


\begin{flushleft}
and IN; homogeneous and heterogeneous nucleation; fundamental
\end{flushleft}


\begin{flushleft}
equations governing cloud processes; warm cloud microphysics:
\end{flushleft}


\begin{flushleft}
diffusional growth, droplet population, collision-coalescence, radiative
\end{flushleft}


\begin{flushleft}
cooling; ice cloud microphysics: nucleation, ice multiplication, growth
\end{flushleft}


\begin{flushleft}
of ice particles by accretion and ice particle melting; hydrometeor;
\end{flushleft}


\begin{flushleft}
impact of microphysical processes on dynamics; cloud chemistry;
\end{flushleft}


\begin{flushleft}
aerosol-cloud interaction: direct, indirect and semi-direct effects;
\end{flushleft}


\begin{flushleft}
clouds in numerical models: parameterization of cloud microphysics;
\end{flushleft}


\begin{flushleft}
cloud-climate interaction.
\end{flushleft}





298





\begin{flushleft}
\newpage
Atmospheric Sciences
\end{flushleft}





\begin{flushleft}
ASL755 Remote Sensing of the Atmosphere and Ocean
\end{flushleft}


\begin{flushleft}
3 Credits (3-0-0)
\end{flushleft}


\begin{flushleft}
Basics of satellite remote sensing: satellite orbits, sensor characteristics,
\end{flushleft}


\begin{flushleft}
view angle, passive and active remote sensing; atmospheric radiative
\end{flushleft}


\begin{flushleft}
transfer application in retrievals of geophysical parameters; aerosol
\end{flushleft}


\begin{flushleft}
remote sensing using ground-based (passive radiometer and lidar) and
\end{flushleft}


\begin{flushleft}
satellite platforms, retrieval algorithm, vertical distribution, application
\end{flushleft}


\begin{flushleft}
of aerosol products in climate studies; cloud remote sensing, cloud
\end{flushleft}


\begin{flushleft}
detection using multi-spectral technique, issues in cloud-masking, CO2
\end{flushleft}


\begin{flushleft}
slice technique; trace gas retrievals; ocean colour remote sensing, SST
\end{flushleft}


\begin{flushleft}
retrieval, wind scatterometry, altimetry; microwave remote sensing:
\end{flushleft}


\begin{flushleft}
soil moisture retrieval, passive (brightness temperature) and active
\end{flushleft}


\begin{flushleft}
(radar) microwave remote sensing for precipitation, sounding, remote
\end{flushleft}


\begin{flushleft}
sensing of cryosphere; satellite meteorology for extreme weather
\end{flushleft}


\begin{flushleft}
events (e.g. cyclone, thunderstorms etc.); land-use/land-cover change;
\end{flushleft}


\begin{flushleft}
hydrological application using gravity anomaly from satellites.
\end{flushleft}





\begin{flushleft}
ASL756 Synoptic Meteorology
\end{flushleft}


\begin{flushleft}
3 Credits (3-0-0)
\end{flushleft}


\begin{flushleft}
Different scales of atmospheric motion; Different types of air
\end{flushleft}


\begin{flushleft}
masses and tropical weather systems; Western disturbances and
\end{flushleft}


\begin{flushleft}
monsoonal cyclonic systems, Meteorological charts and diagrams, map
\end{flushleft}


\begin{flushleft}
projections, plotting of synoptic maps; Analysis of sea level pressure
\end{flushleft}


\begin{flushleft}
patterns, pressure tendency, surface temperature and dew point,
\end{flushleft}


\begin{flushleft}
stream lines and wind patterns, temperature patterns and isotach;
\end{flushleft}


\begin{flushleft}
Analysis of the vertical structure of the atmosphere.
\end{flushleft}





\begin{flushleft}
ASL757 Tropical Weather and Climate
\end{flushleft}


\begin{flushleft}
3 Credits (3-0-0)
\end{flushleft}


\begin{flushleft}
Overview; Structure of the tropical atmosphere; Role of the Tropics
\end{flushleft}


\begin{flushleft}
in the Global Mass, Momentum, and Energy Balance; Tropical
\end{flushleft}


\begin{flushleft}
Circulation \& Mean Precipitation Distribution; ITCZ (Inter-tropical
\end{flushleft}


\begin{flushleft}
Convergence Zone); Tropical Waves and Tropical Variability (Intraseasonal: MJO (Madden-Julian Oscillation), CCEWs (Convectively
\end{flushleft}


\begin{flushleft}
Coupled Equatorial Waves), Inter-annual: ENSO (El Ni\~{n}o Southern
\end{flushleft}


\begin{flushleft}
Oscillation), QBO (Quasi-biennial oscillation), Decadal: PDO (Pacific
\end{flushleft}


\begin{flushleft}
Decadal Oscillation), AMO (Atlantic Multi-decadal Oscillation), NAO
\end{flushleft}


\begin{flushleft}
(North Atlantic Oscillation)); Monsoons (Mean and variability); Tropical
\end{flushleft}


\begin{flushleft}
Cyclones; Modeling of the Tropical Climate \& Weather.
\end{flushleft}





\begin{flushleft}
ASL758 General Circulation of the Atmosphere
\end{flushleft}


\begin{flushleft}
3 Credits (3-0-0)
\end{flushleft}


\begin{flushleft}
General Principles of Atmospheric Motion (Simplifications of Force
\end{flushleft}


\begin{flushleft}
Balances Important to Large-scale Motions, Large-scale Structures
\end{flushleft}


\begin{flushleft}
in the Atmosphere , Simplifications for Large-scale Vertical Structure,
\end{flushleft}


\begin{flushleft}
Scale Analysis of the Tropics), General Circulation of the Atmosphere
\end{flushleft}


\begin{flushleft}
(Historical Evolution of Global Circulation Conceptual Models,
\end{flushleft}


\begin{flushleft}
Axisymmetric Hadley Cell: Theories and Assumptions, A Road Map to
\end{flushleft}


\begin{flushleft}
the Tropics and Subtropics, Walker circulation, Comparing the Tropics
\end{flushleft}


\begin{flushleft}
and Midlatitudes, Stratospheric Circulations), Surface ocean circulation,
\end{flushleft}


\begin{flushleft}
Atmospheric response to Equatorial Heating, Monsoons (Defining the
\end{flushleft}


\begin{flushleft}
Monsoon, A Conceptual Model of Monsoon Evolution, Evolution of the
\end{flushleft}


\begin{flushleft}
South Asian Monsoon System, Other Monsoons Around the World,
\end{flushleft}


\begin{flushleft}
Australian-Maritime Continent Monsoon, West African Monsoon,
\end{flushleft}


\begin{flushleft}
Monsoons in the Americas), General Circulation Modeling (Basics of
\end{flushleft}


\begin{flushleft}
an atmospheric general circulation model, Representation of physical
\end{flushleft}


\begin{flushleft}
processes in GCMs, analysis of GCM simulations and comparison with
\end{flushleft}


\begin{flushleft}
observations, challenges for improving GCM simulations).
\end{flushleft}





\begin{flushleft}
ASL759 Land-Atmosphere Interactions
\end{flushleft}


\begin{flushleft}
3 Credits (3-0-0)
\end{flushleft}


\begin{flushleft}
Introduction: components of the Earth System, energy, hydrologic
\end{flushleft}


\begin{flushleft}
and biogeochemical cycles; Weather and climate processes including
\end{flushleft}


\begin{flushleft}
atmospheric boundary layer, convection, clouds and precipitation,
\end{flushleft}


\begin{flushleft}
surface energy and moisture fluxes, climate, climate variability;
\end{flushleft}


\begin{flushleft}
Canopy-air interactions: canopy processes, observations, big
\end{flushleft}


\begin{flushleft}
leaf models, canopy models; Terrestrial hydrology: watershed
\end{flushleft}


\begin{flushleft}
hydrology, river routing models; Soil: soil physics, soil moisture, soil
\end{flushleft}


\begin{flushleft}
biogeochemistry, soil models; Carbon cycle: photosynthesis, vegetation
\end{flushleft}


\begin{flushleft}
dynamics, global biogeography, carbon cycle models; Terrestrial
\end{flushleft}


\begin{flushleft}
forcings: landscape heterogeneity, landscape induced and modified
\end{flushleft}





\begin{flushleft}
flow, feedbacks, land models, coupled Earth System models; Landuse/land-cover change: Deforestation, agriculture, urbanization, forest
\end{flushleft}


\begin{flushleft}
fires, effects on weather and climate.
\end{flushleft}





\begin{flushleft}
ASL760 Renewable Energy Meteorology
\end{flushleft}


\begin{flushleft}
3 Credits (3-0-0)
\end{flushleft}


\begin{flushleft}
Introduction to the atmosphere: weather and climate processes; Solar
\end{flushleft}


\begin{flushleft}
radiation and surface energy balance: Solar constant, solar geometry,
\end{flushleft}


\begin{flushleft}
atmospheric radiative transfer, clouds and aerosols, surface energy
\end{flushleft}


\begin{flushleft}
budget, urban energy use, sensors and observations; Meteorological
\end{flushleft}


\begin{flushleft}
considerations for solar power: solar resource assessment, solar
\end{flushleft}


\begin{flushleft}
forecasting for different timescales, uncertainty estimation, types of
\end{flushleft}


\begin{flushleft}
solar systems; Wind in the atmospheric boundary layer: boundary
\end{flushleft}


\begin{flushleft}
layer structure and evolution, surface layer, stability, log and power
\end{flushleft}


\begin{flushleft}
laws, flow over complex terrain, low-level jets, offshore winds, sensors
\end{flushleft}


\begin{flushleft}
and observations; Meteorological considerations for wind power: wind
\end{flushleft}


\begin{flushleft}
resource assessment, wind forecasting for different timescales using
\end{flushleft}


\begin{flushleft}
statistical and numerical methods, uncertainty estimation, types of
\end{flushleft}


\begin{flushleft}
turbines, turbine wakes, wake interactions in wind farms, turbine and
\end{flushleft}


\begin{flushleft}
wake models, LES and mesoscale models of wind farms; Solar-wind
\end{flushleft}


\begin{flushleft}
coupling: resource variability, power demand, optimization.
\end{flushleft}





\begin{flushleft}
ASL761 Earth System Modeling
\end{flushleft}


\begin{flushleft}
3 Credits (3-0-0)
\end{flushleft}


\begin{flushleft}
Basics of Earth System Science (Earth system components, Physical
\end{flushleft}


\begin{flushleft}
phenomena in the Earth system, Globally averaged energy budget,
\end{flushleft}


\begin{flushleft}
Energy transports by atmosphere and ocean, concepts of radiative
\end{flushleft}


\begin{flushleft}
forcing, feedbacks and climate change), Physical Processes in the
\end{flushleft}


\begin{flushleft}
Earth System and governing principles (Equation of state, Continuity
\end{flushleft}


\begin{flushleft}
equation, Conservation of momentum, Temperature equation, Moisture
\end{flushleft}


\begin{flushleft}
equation and salinity equation, Moist processes, Wave processes in the
\end{flushleft}


\begin{flushleft}
atmosphere and ocean), Representation of Physical processes in Earth
\end{flushleft}


\begin{flushleft}
System Models (Treatment of sub-grid scale processes such as dry
\end{flushleft}


\begin{flushleft}
convection, moist convection, land surface, snow, ice and vegetation;
\end{flushleft}


\begin{flushleft}
Radiation, greenhouse gases, aerosols and other climate forcings),
\end{flushleft}


\begin{flushleft}
Biogeochemical and Biophysical Processes, coupling between physics
\end{flushleft}


\begin{flushleft}
packages, Dynamics in Earth System Models (Dynamical core, Grid
\end{flushleft}


\begin{flushleft}
scale processes, Numerical representation of the grid scale processes,
\end{flushleft}


\begin{flushleft}
Grids, Resolution, Accuracy, Efficiency, and Scalability), Earth system
\end{flushleft}


\begin{flushleft}
model simulations (Climate simulations and climate drift, Verification
\end{flushleft}


\begin{flushleft}
and Validation of simulations with observations, Emission Scenarios \&
\end{flushleft}


\begin{flushleft}
forcings, Global-average response to greenhouse warming scenarios,
\end{flushleft}


\begin{flushleft}
Transient climate change versus equilibrium response experiments,
\end{flushleft}


\begin{flushleft}
Trends \& natural variability, scale dependency of simulations, Multimodel simulations \& ensemble averages, Simulation examples from
\end{flushleft}


\begin{flushleft}
Coupled Model Inter-comparison Project).
\end{flushleft}





\begin{flushleft}
ASL762 Air-Sea Interaction
\end{flushleft}


\begin{flushleft}
3 Credits (3-0-0)
\end{flushleft}


\begin{flushleft}
State of matter near the air-sea interface, marine boundary layer,
\end{flushleft}


\begin{flushleft}
transfer properties between atmosphere and ocean, solar and
\end{flushleft}


\begin{flushleft}
terrestrial radiation, sea surface radiation budget, surface wind waves,
\end{flushleft}


\begin{flushleft}
air-sea interaction processes using examples of ENSO, hurricane,
\end{flushleft}


\begin{flushleft}
Indian monsoon, turbulent transfer near the interface, bubbles and
\end{flushleft}


\begin{flushleft}
spray, transport of trace gases across the interface; latent, sensible,
\end{flushleft}


\begin{flushleft}
and momentum fluxes in the surface boundary layer over the sea, bulk
\end{flushleft}


\begin{flushleft}
parameterizations, large-scale forcing by sea surface buoyancy fluxes,
\end{flushleft}


\begin{flushleft}
spatio-temporal variability of ocean surface fluxes with reference to
\end{flushleft}


\begin{flushleft}
Indian ocean.
\end{flushleft}





\begin{flushleft}
ASL763 Coastal Ocean and Estuarine Processes
\end{flushleft}


\begin{flushleft}
3 Credits (3-0-0)
\end{flushleft}


\begin{flushleft}
Wave generating and restoring forces, shallow water waves, coastally
\end{flushleft}


\begin{flushleft}
trapped long waves, influence of sea-bed friction, Wave spectra,
\end{flushleft}


\begin{flushleft}
Refraction and shoaling of waves, Seiches, waves-current interaction,
\end{flushleft}


\begin{flushleft}
wave transformation in shallow waters, Tsunamis, Breaking waves,
\end{flushleft}


\begin{flushleft}
Phenomenon of wave reflection, refraction, and diffraction, Surf zone
\end{flushleft}


\begin{flushleft}
hydrodynamics, shoreline setup, Swash and runup heights, wave
\end{flushleft}


\begin{flushleft}
generated alongshore currents, Rip currents, Storm surges, theory
\end{flushleft}


\begin{flushleft}
of tides, Tides in rivers and coastal lagoons, General characteristics
\end{flushleft}


\begin{flushleft}
of estuaries, Classification of estuaries, stratification, estuarine
\end{flushleft}





299





\begin{flushleft}
\newpage
Atmospheric Sciences
\end{flushleft}





\begin{flushleft}
circulation and mixing, Shear instability at an interface, Entrainment
\end{flushleft}


\begin{flushleft}
and sedimentation in estuaries, Dispersion processes: Advective and
\end{flushleft}


\begin{flushleft}
turbulent diffusion, River-estuary-near-shore systems, Sediment
\end{flushleft}


\begin{flushleft}
characteristics, Sediment transport mechanisms: bedform dynamics,
\end{flushleft}


\begin{flushleft}
suspended particles in wave flows and vortices, Morpho-dynamics:
\end{flushleft}


\begin{flushleft}
Beach profiles, Tide range influence on beach morphology, Lee side
\end{flushleft}


\begin{flushleft}
erosion, Beach realignment due to climate change, Interaction of an
\end{flushleft}


\begin{flushleft}
estuary with the near-shore bay.
\end{flushleft}





\begin{flushleft}
ASS800 Independent Study
\end{flushleft}


\begin{flushleft}
3 Credits (0-3-0)
\end{flushleft}


\begin{flushleft}
To be given by the interested faculty.
\end{flushleft}





\begin{flushleft}
ASP820 Advanced Data Analysis for Weather and
\end{flushleft}


\begin{flushleft}
Climate (Not allowed for : Any program other than
\end{flushleft}


\begin{flushleft}
AST and ASZ)
\end{flushleft}


\begin{flushleft}
3 Credits (1-0-4)
\end{flushleft}


\begin{flushleft}
Weather Forecast Evaluation: Jet stream analysis, standard diagnostics
\end{flushleft}


\begin{flushleft}
and skill scores, Extreme events analysis. Using correlation to explore
\end{flushleft}


\begin{flushleft}
the relationships between large-scale atmospheric conditions, and
\end{flushleft}


\begin{flushleft}
local weather. Analyzing trends in climate data, and determining if
\end{flushleft}


\begin{flushleft}
they are statistically significant (regression, Mann-Kendall test etc).
\end{flushleft}


\begin{flushleft}
Regression based approaches, simple linear \& multiple. Using indexes,
\end{flushleft}


\begin{flushleft}
Compositing patterns, Isolating patterns using EOF/PC analysis;
\end{flushleft}


\begin{flushleft}
Analysis of Time Series, Autocorrelation and Spectra.
\end{flushleft}





\begin{flushleft}
ASL821 Advanced Dynamic Meteorology
\end{flushleft}


\begin{flushleft}
3 Credits (3-0-0)
\end{flushleft}


\begin{flushleft}
Pre-requisites: ASL734
\end{flushleft}


\begin{flushleft}
Quasi-geostrophic motions in the atmosphere: circulation and
\end{flushleft}


\begin{flushleft}
vorticity; Ertel-Rossby invariants; Ertel's potential vorticity conservation
\end{flushleft}


\begin{flushleft}
theorem; Kelvin and Bjerknesbaroclinic circulation theorem; quasigeostrophic turbulence. Instabilities in the atmosphere: barotropic
\end{flushleft}


\begin{flushleft}
and baroclinic instability; symmetric instabilities. Quasi-geostrophic
\end{flushleft}


\begin{flushleft}
motions in equatorial region, heat-induced tropical circulations:
\end{flushleft}


\begin{flushleft}
monsoons, El Nino and Madden-Julian Oscillation. Waves in the
\end{flushleft}


\begin{flushleft}
atmosphere: Kelvin, Rossby and Poincar\'{e} waves, Lamb wave, internal
\end{flushleft}


\begin{flushleft}
gravity waves, vertically propagating waves, Rossby adjustment theory.
\end{flushleft}


\begin{flushleft}
Middle atmosphere dynamics: sudden atmospheric warming, QBO.
\end{flushleft}


\begin{flushleft}
General circulation of the atmosphere: analysis of surface pressure and
\end{flushleft}


\begin{flushleft}
associated wind circulation, upper-tropospheric (200hPa) circulation;
\end{flushleft}


\begin{flushleft}
scale analysis, formulation of the governing set of equations for a
\end{flushleft}


\begin{flushleft}
weather prediction model.
\end{flushleft}





\begin{flushleft}
ASL822 Climate Variability
\end{flushleft}


\begin{flushleft}
3 Credits (3-0-0)
\end{flushleft}


\begin{flushleft}
Major modes or patterns of climate variability on intraseasonal to
\end{flushleft}


\begin{flushleft}
interannual and decadal time scales. Well-known modes including
\end{flushleft}


\begin{flushleft}
Madden-Julian Oscillation, El Nino-Southern Oscillation, Pacific Decadal
\end{flushleft}


\begin{flushleft}
Oscillation, Atlantic Multidecadal Oscillation, Indian Ocean Dipole,
\end{flushleft}


\begin{flushleft}
Monsoon, North Atlantic Oscillation, and Annular Modes (Arctic and
\end{flushleft}


\begin{flushleft}
Antarctic Oscillation) and their impacts on extreme weather and
\end{flushleft}


\begin{flushleft}
climate. The course will review climate mode/pattern dynamics, their
\end{flushleft}


\begin{flushleft}
teleconnection mechanisms and impacts on weather/climate such as
\end{flushleft}


\begin{flushleft}
droughts etc. Temporal behavior --including how these modes have
\end{flushleft}


\begin{flushleft}
changed in the past, and how anthropogenic climate change may
\end{flushleft}


\begin{flushleft}
affect future mode behavior). Discussion of predictability of climate
\end{flushleft}


\begin{flushleft}
modes/patterns on seasonal to interannual time scales. Detection and
\end{flushleft}


\begin{flushleft}
attribution of climate change.
\end{flushleft}





\begin{flushleft}
ASL823 Geophysical Fluid Dynamics
\end{flushleft}


\begin{flushleft}
3 Credits (3-0-0)
\end{flushleft}


\begin{flushleft}
(i) Fundamental concepts in geophysical fluid dynamics: equations
\end{flushleft}


\begin{flushleft}
of motion on a rotating planet, vorticity and circulation, conservation
\end{flushleft}


\begin{flushleft}
of potential vorticity, thermal wind, Taylor-Proudman theorem;
\end{flushleft}


\begin{flushleft}
Ertel-Rossby invariants; Ertel's potential vorticity conservation
\end{flushleft}


\begin{flushleft}
theorem; consequences of geostrophic and hydrostatic approximation.
\end{flushleft}


\begin{flushleft}
(ii) Shallow-water theory: derivation of shallow-water equations;
\end{flushleft}


\begin{flushleft}
derivation of vorticity equation; linearized form of shallow-water
\end{flushleft}


\begin{flushleft}
equations; plane waves in a layer of constant depth; dispersion
\end{flushleft}


\begin{flushleft}
diagrams of Kelvin and Poincar\'{e} waves. (iii) Rossby wave theory:
\end{flushleft}





\begin{flushleft}
mechanism of Rossby wave generation; inertial boundary currents;
\end{flushleft}


\begin{flushleft}
derivation of potential vorticity on beta-plane; quasigeostrophic scaling;
\end{flushleft}


\begin{flushleft}
Rossby waves in a zonal current; method of multiple scales for linear
\end{flushleft}


\begin{flushleft}
potential vorticity equation; reflection and radiation of Rossby waves;
\end{flushleft}


\begin{flushleft}
generation of Rossby waves by an initial disturbance; Quasigeostrophic
\end{flushleft}


\begin{flushleft}
normal modes in a closed basin; resonant interaction; energy and
\end{flushleft}


\begin{flushleft}
enstrophy conservation; upscale energy transfer. (iv) Friction effects
\end{flushleft}


\begin{flushleft}
in geophysical flows: Turbulent Reynolds stresses; Ekman layers in
\end{flushleft}


\begin{flushleft}
a homogeneous, incompressible rotating fluid; Ekman layer on a
\end{flushleft}


\begin{flushleft}
sloping surface; quasigeostrophic potential vorticity with friction and
\end{flushleft}


\begin{flushleft}
topography. (v) Instability theory: linear stability; normal modes;
\end{flushleft}


\begin{flushleft}
growth rates; baroclinic instability; Eady model and Charney model;
\end{flushleft}


\begin{flushleft}
instability in a two-layer model.
\end{flushleft}





\begin{flushleft}
ASL824 Parameterization of Physical Processes
\end{flushleft}


\begin{flushleft}
3 Credits (3-0-0)
\end{flushleft}


\begin{flushleft}
Pre-requisites: Any one of ASL733, ASL734
\end{flushleft}


\begin{flushleft}
A simple model of atmosphere with Rayleigh friction and Newtonian
\end{flushleft}


\begin{flushleft}
cooling: Gill's analytical solutions for heat-induced tropical circulations
\end{flushleft}


\begin{flushleft}
(especially El Ni\~{n}o and monsoon circulation); horizontal diffusion in
\end{flushleft}


\begin{flushleft}
NWP models: prevention of accumulation of small scale noise, inverse
\end{flushleft}


\begin{flushleft}
cascade. Aerodynamic formulae for surface turbulent fluxes, vertical
\end{flushleft}


\begin{flushleft}
turbulent diffusion: one-dimensional PBL model. Parameterization
\end{flushleft}


\begin{flushleft}
of orographic drag. Dry and moist processes in the atmosphere: a
\end{flushleft}


\begin{flushleft}
simple model of convection, dry adiabatic adjustment, large-scale
\end{flushleft}


\begin{flushleft}
condensation, and parameterization of deep and shallow convection.
\end{flushleft}


\begin{flushleft}
Simple and complex radiative transfer in the atmosphere; absorption
\end{flushleft}


\begin{flushleft}
of radiation by ozone, carbon dioxide and water vapour; shortwave and
\end{flushleft}


\begin{flushleft}
longwave radiation computation; radiative heating in the atmosphere.
\end{flushleft}





\begin{flushleft}
ASP825 Mesoscale Modeling (Not allowed for : Any
\end{flushleft}


\begin{flushleft}
program other than AST and ASZ)
\end{flushleft}


\begin{flushleft}
3 Credits (0-0-6)
\end{flushleft}


\begin{flushleft}
Introduction to the Weather Research and Forecasting (WRF) model
\end{flushleft}


\begin{flushleft}
and parallel computing; Install WRF, NCL and associated libraries;
\end{flushleft}


\begin{flushleft}
Conduct test simulations for 2-d idealized cases such as flow over a
\end{flushleft}


\begin{flushleft}
hill, sea-breeze, etc., configure and conduct test simulations for a full
\end{flushleft}


\begin{flushleft}
3-d real case, conduct numerical experiments by changing initial \&
\end{flushleft}


\begin{flushleft}
boundary conditions and namelist parameters/flags; Understand WRF
\end{flushleft}


\begin{flushleft}
code structure and registry by adding new variables into different
\end{flushleft}


\begin{flushleft}
modules; Introduction to parameterizations in WRF, explore the science
\end{flushleft}


\begin{flushleft}
and the codes of a land surface scheme and a cumulus scheme, make
\end{flushleft}


\begin{flushleft}
simple modifications to the schemes, conduct numerical experiments
\end{flushleft}


\begin{flushleft}
with modified schemes.
\end{flushleft}





\begin{flushleft}
ASL826 Ocean Modeling (Not allowed for : Any program
\end{flushleft}


\begin{flushleft}
other than AST and ASZ)
\end{flushleft}


\begin{flushleft}
3 Credits (2-0-2)
\end{flushleft}


\begin{flushleft}
Introduction to ocean dynamics, governing equations of oceanic
\end{flushleft}


\begin{flushleft}
motions, numerical methods in ocean modelling, hydrostatic and nonhydrostatic phenomenon, barotropic and baroclinic processes, lateral
\end{flushleft}


\begin{flushleft}
and open boundary conditions, parameterization of sub-grid scale
\end{flushleft}


\begin{flushleft}
processes, large scale ocean circulation, modelling of shelf circulation,
\end{flushleft}


\begin{flushleft}
tides and storm surge modelling, regional and coastal ocean models,
\end{flushleft}


\begin{flushleft}
shallow water models, multi-level basin scale and global ocean models,
\end{flushleft}


\begin{flushleft}
ocean wave modelling, introduction to data assimilation techniques.
\end{flushleft}





\begin{flushleft}
ASL827 Advanced Dynamic Oceanography
\end{flushleft}


\begin{flushleft}
3 Credits (3-0-0)
\end{flushleft}


\begin{flushleft}
Pre-requisites: Either ASL734 or ASL737
\end{flushleft}


\begin{flushleft}
Conservation laws for moving fluids, Ekman and Sverdrup theories,
\end{flushleft}


\begin{flushleft}
coastal upwelling and fronts, Western boundary intensification,
\end{flushleft}


\begin{flushleft}
barotropic currents, baroclinic transport over topography, thermohaline
\end{flushleft}


\begin{flushleft}
circulation, Mesoscale eddies and variability. Indian ocean circulation,
\end{flushleft}


\begin{flushleft}
wave theory, ocean wave spectra, wave energy equation, breaking
\end{flushleft}


\begin{flushleft}
waves, reflection and dissipation, theory of tides, tidal currents, tidal
\end{flushleft}


\begin{flushleft}
processes in embayment and estuaries, wind and buoyancy driven
\end{flushleft}


\begin{flushleft}
currents, near-shore circulation, alongshore currents, wave-current
\end{flushleft}


\begin{flushleft}
interaction, sediment transport, coastal ocean response to extreme
\end{flushleft}


\begin{flushleft}
wind forcing, storm surges, Planetary and equatorial waves, coastally
\end{flushleft}


\begin{flushleft}
trapped Kelvin waves.
\end{flushleft}





300





\begin{flushleft}
\newpage
Atmospheric Sciences
\end{flushleft}





\begin{flushleft}
ASL851 Special Topics in Climate
\end{flushleft}


\begin{flushleft}
3 Credits (3-0-0)
\end{flushleft}





\begin{flushleft}
ASV864 Special Module in Atmosphere
\end{flushleft}


\begin{flushleft}
1 Credit (1-0-0)
\end{flushleft}





\begin{flushleft}
To be given by the interested faculty.
\end{flushleft}





\begin{flushleft}
To be given by the interested faculty.
\end{flushleft}





\begin{flushleft}
ASL852 Special Topics in Oceans
\end{flushleft}


\begin{flushleft}
3 Credits (3-0-0)
\end{flushleft}





\begin{flushleft}
ASV865 Special Module in Air Pollution Studies
\end{flushleft}


\begin{flushleft}
1 Credit (1-0-0)
\end{flushleft}


\begin{flushleft}
To be given by the interested faculty.
\end{flushleft}





\begin{flushleft}
To be given by the interested faculty.
\end{flushleft}





\begin{flushleft}
ASV866 Special Module in Atmosphere and Ocean
\end{flushleft}


\begin{flushleft}
1 Credit (1-0-0)
\end{flushleft}





\begin{flushleft}
ASL853 Special Topics in Atmosphere
\end{flushleft}


\begin{flushleft}
3 Credits (3-0-0)
\end{flushleft}





\begin{flushleft}
To be given by the interested faculty.
\end{flushleft}





\begin{flushleft}
To be given by the interested faculty.
\end{flushleft}





\begin{flushleft}
ASP867 Special Module in Weather Forecasting (Not
\end{flushleft}


\begin{flushleft}
allowed for : Any program other than AST and ASZ)
\end{flushleft}


\begin{flushleft}
1 Credit (0-0-2)
\end{flushleft}





\begin{flushleft}
ASL854 Special Topics in Air Pollution Studies
\end{flushleft}


\begin{flushleft}
3 Credits (3-0-0)
\end{flushleft}


\begin{flushleft}
To be given by the interested faculty.
\end{flushleft}





\begin{flushleft}
To be given by the interested faculty.
\end{flushleft}





\begin{flushleft}
ASP855 Special Topics in Atmosphere and Ocean (Not
\end{flushleft}


\begin{flushleft}
allowed for : Any program other than AST and ASZ)
\end{flushleft}


\begin{flushleft}
3 Credits (1-0-4)
\end{flushleft}





\begin{flushleft}
ASP868 Special Module in Atmospheric and Oceanic
\end{flushleft}


\begin{flushleft}
Observations (Not allowed for : Any program other
\end{flushleft}


\begin{flushleft}
than AST and ASZ)
\end{flushleft}


\begin{flushleft}
1 Credit (0-0-2)
\end{flushleft}





\begin{flushleft}
To be given by the interested faculty.
\end{flushleft}





\begin{flushleft}
To be given by the interested faculty.
\end{flushleft}





\begin{flushleft}
ASL856 Special Topics in Atmospheric and Oceanic
\end{flushleft}


\begin{flushleft}
Observations (Not allowed for : Any program other
\end{flushleft}


\begin{flushleft}
than AST and ASZ)
\end{flushleft}


\begin{flushleft}
3 Credits (2-0-2)
\end{flushleft}





\begin{flushleft}
ASC869 Atmospheric and Oceanic Science Colloquium
\end{flushleft}


\begin{flushleft}
(Not allowed for : Any program other than AST and ASZ)
\end{flushleft}


\begin{flushleft}
1 Credit (0-1-0)
\end{flushleft}





\begin{flushleft}
To be given by the interested faculty.
\end{flushleft}





\begin{flushleft}
ASD881 Project-I (Not allowed for : Any program
\end{flushleft}


\begin{flushleft}
other than AST and ASZ)
\end{flushleft}


\begin{flushleft}
6 Credits (0-0-12)
\end{flushleft}





\begin{flushleft}
ASV862 Special Module in Climate
\end{flushleft}


\begin{flushleft}
1 Credit (1-0-0)
\end{flushleft}


\begin{flushleft}
To be given by the interested faculty.
\end{flushleft}





\begin{flushleft}
ASV863 Special Module in Oceans
\end{flushleft}


\begin{flushleft}
1 Credit (1-0-0)
\end{flushleft}


\begin{flushleft}
To be given by the interested faculty.
\end{flushleft}





\begin{flushleft}
To be given by the interested faculty.
\end{flushleft}





\begin{flushleft}
To be given by the interested faculty.
\end{flushleft}





\begin{flushleft}
ASD882 Project-II (Not allowed for : Any program
\end{flushleft}


\begin{flushleft}
other than AST and ASZ)
\end{flushleft}


\begin{flushleft}
12 Credits (0-0-24)
\end{flushleft}


\begin{flushleft}
Pre-requisites: ASD881
\end{flushleft}





301





\begin{flushleft}
\newpage
Centre for Biomedical Engineering
\end{flushleft}


\begin{flushleft}
BML330 Safety Principles for Engineers
\end{flushleft}


\begin{flushleft}
3 Credit (3-0-0)
\end{flushleft}


\begin{flushleft}
Pre-requisites: EC 60
\end{flushleft}





\begin{flushleft}
BML720 Medical Imaging
\end{flushleft}


\begin{flushleft}
3 Credits (3-0-0)
\end{flushleft}





\begin{flushleft}
Value theory-Risk and Reliability-Decision theory. Injury and damage
\end{flushleft}


\begin{flushleft}
control. Epidemiology of accidents. Human tolerance to energy inputs.
\end{flushleft}


\begin{flushleft}
Biomedical/biomechanical aspects of long term exposure to hazardous
\end{flushleft}


\begin{flushleft}
environment. Socio-technical aspects of safety standards. Case studies
\end{flushleft}


\begin{flushleft}
of well known disasters.
\end{flushleft}





\begin{flushleft}
BML700 Introduction to Basic Medical Sciences for
\end{flushleft}


\begin{flushleft}
Engineers
\end{flushleft}


\begin{flushleft}
3 Credit (3-0-0)
\end{flushleft}


\begin{flushleft}
Anatomical and physiological study of different human systems.
\end{flushleft}


\begin{flushleft}
Cell and tissue organization and metabolism Cardiovascular System;
\end{flushleft}


\begin{flushleft}
hemodynamics, blood, conduction system in the heart. Soft and hard
\end{flushleft}


\begin{flushleft}
tissues and joints endocrine and nervous system and their role in
\end{flushleft}


\begin{flushleft}
homeostasis; Respiratory physiology; kidneys and the urinary system.
\end{flushleft}





\begin{flushleft}
BMV700 Biomechanical Design of Medical Devices
\end{flushleft}


\begin{flushleft}
1 Credit (1-0-0)
\end{flushleft}


\begin{flushleft}
Introduction. Mechanics of cells and Tissues. Basics of Finite Element
\end{flushleft}


\begin{flushleft}
Modelling of tissues and organs. Design and Fabrication of Microfluidic
\end{flushleft}


\begin{flushleft}
devices. Design of diagnostic devices. Design of Endocrinal devices.
\end{flushleft}


\begin{flushleft}
Design of orthopaedic devices. Design of cardiovascular devices.
\end{flushleft}


\begin{flushleft}
Student presentations on innovative designs.
\end{flushleft}





\begin{flushleft}
Overview of medical imaging modalities. Radiation physics. X-ray
\end{flushleft}


\begin{flushleft}
background, physics, principles, instrumentation, developments,
\end{flushleft}


\begin{flushleft}
applications. CT background, physics, principles, instrumentation,
\end{flushleft}


\begin{flushleft}
developments, applications. MRI background, physics, principles,
\end{flushleft}


\begin{flushleft}
instrumentation, developments, applications. Ultrasound background,
\end{flushleft}


\begin{flushleft}
physics, principles, instrumentation, developments, applications.
\end{flushleft}


\begin{flushleft}
Nuclear (SPECT, PET, Gamma) background, physics, principles,
\end{flushleft}


\begin{flushleft}
instrumentation, developments, applications. Optical Imaging
\end{flushleft}


\begin{flushleft}
background, physics, principles, instrumentation, developments,
\end{flushleft}


\begin{flushleft}
applications. Contrast enhanced imaging modalities - physics, latest
\end{flushleft}


\begin{flushleft}
developments, applications. Emerging imaging modalities (microwave,
\end{flushleft}


\begin{flushleft}
electrical impedence,etc.) -physics, latest developments, applications.
\end{flushleft}





\begin{flushleft}
BML 735 Biomedical Signal and Image Processing
\end{flushleft}


\begin{flushleft}
3 Credits (2-0-2)
\end{flushleft}


\begin{flushleft}
Introduction to Biomedical Signal and Image data obtained using
\end{flushleft}


\begin{flushleft}
various techniques (ECG, FTIR, NMR spectroscopy, MRI, CT, nuclear
\end{flushleft}


\begin{flushleft}
imaging, ultrasound and optical imaging). Noise and error propagation
\end{flushleft}


\begin{flushleft}
in Biomedical Signal and Image data. Basic statistics for biomedical
\end{flushleft}


\begin{flushleft}
signal and image data analysis. Biomedical signal processing in time
\end{flushleft}


\begin{flushleft}
domain. Fourier and Laplace transform. Biomedical signal processing
\end{flushleft}


\begin{flushleft}
in frequency domain. Biomedical image processing, including
\end{flushleft}


\begin{flushleft}
segmentation, registration and pattern recognition. Mathematical
\end{flushleft}


\begin{flushleft}
models used in biomedical signal and image data analysis.
\end{flushleft}





\begin{flushleft}
BML736 Application of Mathematics in Biomedical Engg.
\end{flushleft}


\begin{flushleft}
2 Credit (2-0-0)
\end{flushleft}





\begin{flushleft}
BMV701 Basic Electronics
\end{flushleft}


\begin{flushleft}
1 Credits (1-0-0)
\end{flushleft}


\begin{flushleft}
Fundamentals of Electronics, Electronics components and their
\end{flushleft}


\begin{flushleft}
principle of operation with respect to medical equipments. Operational
\end{flushleft}


\begin{flushleft}
Amplifiers and their use in biomedical instrumentation. Electronic
\end{flushleft}


\begin{flushleft}
interfacing of analog circuits with computer. LabVIEW based virtual
\end{flushleft}


\begin{flushleft}
instrumentation. Electronic signal conditioning.
\end{flushleft}





\begin{flushleft}
Mathematical functions commenly used in biomdeical engineering;
\end{flushleft}


\begin{flushleft}
biomedical data, data analysis (Basic Biostatistics), data fitting;
\end{flushleft}


\begin{flushleft}
refreshing Engineering Mathematics; Applications of Mathematcs in
\end{flushleft}


\begin{flushleft}
various areas of Biomedical Engineering (Biochemistry, Biomedical
\end{flushleft}


\begin{flushleft}
Signal and Imaging, Biosensors, Biomechanics, etc.), Mathematical
\end{flushleft}


\begin{flushleft}
modeling and simulations.
\end{flushleft}





\begin{flushleft}
BMV702 Basic Mathematics for Biologists	
\end{flushleft}


\begin{flushleft}
1 Credits (1-0-0)
\end{flushleft}





\begin{flushleft}
BML740 Biomedical Instrumentation
\end{flushleft}


\begin{flushleft}
3 Credits (3-0-0)
\end{flushleft}





\begin{flushleft}
Introduction to calculus, function, sets, Derivatives, integrals,
\end{flushleft}


\begin{flushleft}
exponentials and logarithm, complex numbers, sequence, series.
\end{flushleft}


\begin{flushleft}
Linear Algebra: Matrix, vector, basic operations on matrix, system
\end{flushleft}


\begin{flushleft}
of equations, linear and non linear equations. Differential Equations.
\end{flushleft}


\begin{flushleft}
Exposure to other topics like complex analysis, Fourier series.
\end{flushleft}





\begin{flushleft}
BMV703 Basic biology \& Physiology
\end{flushleft}


\begin{flushleft}
1 Credits (1-0-0)
\end{flushleft}


\begin{flushleft}
Basics of biology: Biomacromolecules, cells, biological membranes,
\end{flushleft}


\begin{flushleft}
Enzymes and enzyme catalysis, cellular metabolic pathways, cell cycle,
\end{flushleft}


\begin{flushleft}
cell division, molecular biology of cell, gene expression, DNA and its
\end{flushleft}


\begin{flushleft}
role in hereditary, cell signalling and communication. Microbiology bacteria, fungi and prokaryotes, their cytological features and genetics.
\end{flushleft}


\begin{flushleft}
Human anatomy, organs and systems, hormones, nervous and sensory
\end{flushleft}


\begin{flushleft}
systems, nutrition, Innate and adaptive immunity.
\end{flushleft}





\begin{flushleft}
BMV704 Mechanics of Biomaterials	
\end{flushleft}


\begin{flushleft}
1 Credits (1-0-0)
\end{flushleft}


\begin{flushleft}
Forces and moments. Free body diagrams. Elastic, Plastic, Viscoelastic, visco-plastic. Bending, torsional and shear strength. Stressstrain relationships of linear and non-linear solids. Fatigue loading
\end{flushleft}


\begin{flushleft}
with and without corrosion. Fluid mechanics - Shear thinning and
\end{flushleft}


\begin{flushleft}
shear thickening fluids
\end{flushleft}





\begin{flushleft}
BML710 Industrial Biomaterial Technology
\end{flushleft}


\begin{flushleft}
3 Credits (3-0-0)
\end{flushleft}


\begin{flushleft}
Good Manufacturing practice regulations, biomedical materials, quality
\end{flushleft}


\begin{flushleft}
assurance and quality control, Labeling, Device failure, synthetic and
\end{flushleft}


\begin{flushleft}
biopoloymers, Bioerodible materials, Host reactions to biomaterials,
\end{flushleft}


\begin{flushleft}
Sterilization of Medical devices, Advances in Sterilization Technology
\end{flushleft}


\begin{flushleft}
of clean room, Polymeric materials for drug delivery systems, active
\end{flushleft}


\begin{flushleft}
and passive targeting, intelligent materials.
\end{flushleft}





\begin{flushleft}
Generalized medical instrumentation systems and transducersDigital Systems for biomedical instrumentation -Introduction to
\end{flushleft}


\begin{flushleft}
Microprocessor, Microcontroller, Digital Signal Processors, Introduction
\end{flushleft}


\begin{flushleft}
to Control systems- open loop and closed loop system, Transfer
\end{flushleft}


\begin{flushleft}
Function, P, PI, and PID Controllers. Genesis and recording of
\end{flushleft}


\begin{flushleft}
biosignals, ECG Instrumentation, Biomechanical measurementskinematics, kinetics, anthropometry, Ear and Ophthalmological
\end{flushleft}


\begin{flushleft}
measurement - Ear Hearing loss, Sound conduction system, Basic and
\end{flushleft}


\begin{flushleft}
Pure tone audiometer, Evoked response audiometer, Vision - Tonometer,
\end{flushleft}


\begin{flushleft}
Ophthalmoscope, Perimeter, Blood related physical measurements sound, flow, volume, and pressure. Clinical Laboratory Instrumentation
\end{flushleft}


\begin{flushleft}
- Introduction to electrochemistry, medical diagnosis with chemical
\end{flushleft}


\begin{flushleft}
test, Spectrophotometer, Colorimeter, Auto analyzers, clinical flame
\end{flushleft}


\begin{flushleft}
photometer, selective ion based electrodes, Pathology Instrumentation
\end{flushleft}


\begin{flushleft}
- Flow Cytometer, Haemocytometer, Haemoglobinometer, Anaesthesia
\end{flushleft}


\begin{flushleft}
Instrumentation - Need of anaesthesia, anaesthesia delivery
\end{flushleft}


\begin{flushleft}
system, breathing circuits. Ventilators, Diathermy, Introduction to
\end{flushleft}


\begin{flushleft}
instrumentation in X-Ray, MRI, CT scan, Introduction to MEMS. Electrical
\end{flushleft}


\begin{flushleft}
safety in medical environment - shock hazards, leakage current, safety
\end{flushleft}


\begin{flushleft}
codes, electrical safety analyzer, testing of biomedical equipments.
\end{flushleft}





\begin{flushleft}
BML741 Medical Device Design
\end{flushleft}


\begin{flushleft}
4 Credits (2-0-4)
\end{flushleft}


\begin{flushleft}
Introduction to medical device design course and its significance in
\end{flushleft}


\begin{flushleft}
the current scenario; basic human physiology, communicable and noncommunicable diesese; different approaches to medical device design;
\end{flushleft}


\begin{flushleft}
considerations in medical device design; case studies of medical
\end{flushleft}


\begin{flushleft}
device design; identification of need, immersion, disease burden,
\end{flushleft}


\begin{flushleft}
disease state fundamentals, and the need for validation; development
\end{flushleft}


\begin{flushleft}
of concepts, ideation \& brainstorming, evaluation of concepts, risk/
\end{flushleft}


\begin{flushleft}
benefit analysis; usability analysis \& methods of prototyping; user
\end{flushleft}


\begin{flushleft}
feedback, stakeholder analysis \& characterization; IP and regulatory
\end{flushleft}


\begin{flushleft}
requirements; conclusions.
\end{flushleft}





302





\begin{flushleft}
\newpage
Biomedical Engineering
\end{flushleft}





\begin{flushleft}
BMD742 Minor Biodesign Project
\end{flushleft}


\begin{flushleft}
4 Credits (0-0-8)
\end{flushleft}


\begin{flushleft}
The course will cover activities pertaining to design-build-test-modify
\end{flushleft}


\begin{flushleft}
iterations in order to build functional prototypes of medical devices.
\end{flushleft}





\begin{flushleft}
BML743 Special Topics In Biodesign
\end{flushleft}


\begin{flushleft}
3 Credits (3-0-0)
\end{flushleft}


\begin{flushleft}
The course contents will be flexible covering state of the art design,
\end{flushleft}


\begin{flushleft}
research and innovation issues pertaining to biodesign.
\end{flushleft}





\begin{flushleft}
BMP743 Basic Biomedical Laboratory
\end{flushleft}


\begin{flushleft}
2 Credits (0-0-4)
\end{flushleft}


\begin{flushleft}
Students shall be introduced with practical training on basic
\end{flushleft}


\begin{flushleft}
electronic design and interfacing, and be given laboratory exercises
\end{flushleft}


\begin{flushleft}
on bioinstrumentation. Students will also be exposed to the role of
\end{flushleft}


\begin{flushleft}
medical imaging and signal processing in biomedical engineering. A
\end{flushleft}


\begin{flushleft}
few experiments training in materials synthesis, characterization and
\end{flushleft}


\begin{flushleft}
modification of various biomaterials will be given. Students will also
\end{flushleft}


\begin{flushleft}
get trained on sterile techniques of cell culture, cytotoxcity assays
\end{flushleft}


\begin{flushleft}
and cell staining techniques.
\end{flushleft}





\begin{flushleft}
BML750 Point of Care Medical Diagnostic Devices
\end{flushleft}


\begin{flushleft}
3 Credits (3-0-0)
\end{flushleft}


\begin{flushleft}
Brief introductions to analytical chemistry and biochemistry; sensors
\end{flushleft}


\begin{flushleft}
and biosensors (immobilization, transducers, electronic components,
\end{flushleft}


\begin{flushleft}
op-amps and general circuits; data processing and presentation LabVIEW based virtual instrumentation, etc.); Medical diagnostic
\end{flushleft}


\begin{flushleft}
techniques (biochemical, pathological, hematological analysis, DNA/
\end{flushleft}


\begin{flushleft}
RNA based analysis, etc.; Necessity for rapid and in-situ medical
\end{flushleft}


\begin{flushleft}
analysis; Point of care technology (POCT); Minaturization of medical
\end{flushleft}


\begin{flushleft}
diagnostic devices -- Microfabrication (materials, processes, techniques
\end{flushleft}


\begin{flushleft}
for detection); Microfluidics (concept, procedure, applications and
\end{flushleft}


\begin{flushleft}
challenges); Integrated microfluidic devices: Lab-on-a-chip, systemon-a-chip, micro-total analysis system ($\mu$TAS); Present research
\end{flushleft}


\begin{flushleft}
scenario and future prospects; Case studies on POCT devices;
\end{flushleft}


\begin{flushleft}
Laboratory visit and demonstration of microfabrication processes and
\end{flushleft}


\begin{flushleft}
Lab-on-a-chip devices.
\end{flushleft}





\begin{flushleft}
BML760 Biomedical Ethics, Safety and Regulatory
\end{flushleft}


\begin{flushleft}
Affairs
\end{flushleft}


\begin{flushleft}
2 Credits (2-0-0)
\end{flushleft}


\begin{flushleft}
Introduction to medical ethics and bioethics, environmental ethics.
\end{flushleft}


\begin{flushleft}
Use of animals in pre-clinical trials and ethical approval. Ethics
\end{flushleft}


\begin{flushleft}
issues in biomedical sciences (inhalable, injectable, implantable
\end{flushleft}


\begin{flushleft}
systems). Principles of biosafety. Biosafety cabinets, Laboratory
\end{flushleft}


\begin{flushleft}
biosafety levels. vertebrates and invertebrates safety levels, Biosafety
\end{flushleft}


\begin{flushleft}
of infectious agents: bacteria, fungus, parasite, prions, viruses,
\end{flushleft}


\begin{flushleft}
Biosafety of infectious agents: bacteria, fungus, parasite, prions,
\end{flushleft}


\begin{flushleft}
viruses. Laboratory security and emergency response, guidelines
\end{flushleft}


\begin{flushleft}
to work with infectious agents and toxins. Regulatory frameworks:
\end{flushleft}


\begin{flushleft}
FDA, BIS, ISO certification, CDSCO; health \& family welfare laws
\end{flushleft}


\begin{flushleft}
and regulations on design, development, testing and production of
\end{flushleft}


\begin{flushleft}
biomedical products, including biologics, drugs, biotechnology-derived
\end{flushleft}


\begin{flushleft}
therapeutics, vaccines and medical devices. Clinical trials and current
\end{flushleft}


\begin{flushleft}
good manufacturing Practices. Basic introduction to IPR. Post-market
\end{flushleft}


\begin{flushleft}
issues and requirements.
\end{flushleft}





\begin{flushleft}
BML770 Fundamentals of Biomechanics
\end{flushleft}


\begin{flushleft}
3 Credits (3-0-0)
\end{flushleft}


\begin{flushleft}
Overview and significance or biomechanics to lead a better life,
\end{flushleft}


\begin{flushleft}
challenges and opportunities, inventions/research. Orthopaedic
\end{flushleft}


\begin{flushleft}
components - bones, tendons, ligaments and cartilages -primary
\end{flushleft}


\begin{flushleft}
functions, material constituents (osteoclasts, osetoblasts, collagen,
\end{flushleft}


\begin{flushleft}
collagen fibrils), mechanical strength, building vs recuperation rate,
\end{flushleft}


\begin{flushleft}
force analysis. Cardiovascular components-arteries, veins, primary
\end{flushleft}


\begin{flushleft}
functions \& flowrate, material constituents, mechanical strength,
\end{flushleft}


\begin{flushleft}
inflammation, life span vs. recuperation rate, force analysis.
\end{flushleft}


\begin{flushleft}
Biomaterials - Metals/alloys, polymers, ceramics, shape-memory alloys,
\end{flushleft}


\begin{flushleft}
composites and functionally graded materials. Basic Principles - ForceMotion, Force-Time, Inertia, Range of Motion, Segmental Interaction,
\end{flushleft}


\begin{flushleft}
Balance, Cordination continuum, Projection \& Spin. Force analysis
\end{flushleft}





\begin{flushleft}
of Joints at various kinetic states - Spine (running, climbing, stairs,
\end{flushleft}


\begin{flushleft}
running downhill etc), Knee (squatting, jumping, climbing stairs,
\end{flushleft}


\begin{flushleft}
kickingsoccer), shoulder (abduction, adduction, bowling, smashing racquet sports), Elbow (tennis, golf) and Hip (during fall, running).
\end{flushleft}


\begin{flushleft}
Demonstrations - Characterization, fractures \& ruptures, non-invasive
\end{flushleft}


\begin{flushleft}
analysis (MRI, CT scan).
\end{flushleft}





\begin{flushleft}
BML771 Orthopaedic Device Design
\end{flushleft}


\begin{flushleft}
2 Credits (2-0-0)
\end{flushleft}


\begin{flushleft}
Pre-requisites: AML732/AML835/AML851/MEL739
\end{flushleft}


\begin{flushleft}
Introduction: a. Bones, tissues and muscles, b. Properties; Static and
\end{flushleft}


\begin{flushleft}
dynamic loads; Kinematics and Kinetics; Bone healing and remodelling;
\end{flushleft}


\begin{flushleft}
Strength, Wear and Corrosion; Design of Orthopaedic prostheses;
\end{flushleft}


\begin{flushleft}
Methods to avoid reoccurance of fractures; Bone modelling; Guest
\end{flushleft}


\begin{flushleft}
lectures; Demonstrations.
\end{flushleft}





\begin{flushleft}
BML772 Biofabrication
\end{flushleft}


\begin{flushleft}
3 Credits (2-0-2)
\end{flushleft}


\begin{flushleft}
Pre-requisites: 50 Credits
\end{flushleft}


\begin{flushleft}
Introduction; bioprinting tissues, bones and cartilages; self-assembly,
\end{flushleft}


\begin{flushleft}
directed assembly, enzymatic assembly; laser-assisted bio-printing;
\end{flushleft}


\begin{flushleft}
fabrication of scaffolds (hydrogel method and fibre based); artificial
\end{flushleft}


\begin{flushleft}
bacteria (active/passive drug delivery, microswimmer); component
\end{flushleft}


\begin{flushleft}
fabrication (stereolithography, laser machining etc); mass production
\end{flushleft}


\begin{flushleft}
(stamping, micro-injection molding etc). Experiments: CAD
\end{flushleft}


\begin{flushleft}
(solidworks) and data import (from MRI/CT) - hands-on; Fused
\end{flushleft}


\begin{flushleft}
deposition molding (3D printing) - hands on; Fused deposition
\end{flushleft}


\begin{flushleft}
molding (3D printing) - hands on; Tissue \& Organ printing (3D organ
\end{flushleft}


\begin{flushleft}
printer) - demonstration only; Scaffold generation - Hydrogel (Wet
\end{flushleft}


\begin{flushleft}
Chemistry) and Fibres (electrospinning) - hands on; Laser machining
\end{flushleft}


\begin{flushleft}
- hands on; Mask generation (E-beam lithography and focussed ion
\end{flushleft}


\begin{flushleft}
beam) -demonstration only; Characterization (Imaging, Profilometry,
\end{flushleft}


\begin{flushleft}
optical scanner) - hands on; Stamping - hands on; Micro-injection
\end{flushleft}


\begin{flushleft}
molding - demonstration only.
\end{flushleft}





\begin{flushleft}
BML790 Modern Medicine: An Engineering Perspective
\end{flushleft}


\begin{flushleft}
3 Credit (2-1-0)
\end{flushleft}


\begin{flushleft}
The course will cover an overview of patho-physiology of some of
\end{flushleft}


\begin{flushleft}
the common non-communicable human diseases. Details of Cerebral
\end{flushleft}


\begin{flushleft}
Ischemia/Stroke, Diabetes and cardiac abnormalities will be discussed.
\end{flushleft}


\begin{flushleft}
With respect to each diseases the corresponding diagnostic techniques,
\end{flushleft}


\begin{flushleft}
tools, and physical principles of these instruments will be discussed.
\end{flushleft}


\begin{flushleft}
Students will be encouraged with lateral thinking and brain storming
\end{flushleft}


\begin{flushleft}
future engineering research potentials in improvement of current
\end{flushleft}


\begin{flushleft}
diagnostic and treatment modalities.
\end{flushleft}





\begin{flushleft}
BML800 Research Techniques in Biomedical Engineering
\end{flushleft}


\begin{flushleft}
3 Credits (3-0-0)
\end{flushleft}


\begin{flushleft}
Simulation and analysis of physiological systems by up to date computer
\end{flushleft}


\begin{flushleft}
techniques and development of physical models; Biomechanical
\end{flushleft}


\begin{flushleft}
analysis and network representation; State of art bioinstrumentation
\end{flushleft}


\begin{flushleft}
techniques; monitoring physiological parameters electrical, mechanical
\end{flushleft}


\begin{flushleft}
and chemical parameters of human body, Microminiaturization
\end{flushleft}


\begin{flushleft}
of electronics including MEMS; BIOMEMS technology; Biomedical
\end{flushleft}


\begin{flushleft}
signal processing and imaging modalities; Research planning and
\end{flushleft}


\begin{flushleft}
interpretation of biomedical data; Telemedicine; Robotics in Medicine.
\end{flushleft}





\begin{flushleft}
BMD801 Major Project-1	
\end{flushleft}


\begin{flushleft}
9 Credits (0-0-18)
\end{flushleft}


\begin{flushleft}
The curriculum shall comprise of practical training on chosen research
\end{flushleft}


\begin{flushleft}
topic, optimization of experimental conditions, so as to take up
\end{flushleft}


\begin{flushleft}
independent research in major project.	
\end{flushleft}





\begin{flushleft}
BMD802 Major Project-2	
\end{flushleft}


\begin{flushleft}
12 Credits (0-0-24)
\end{flushleft}


\begin{flushleft}
Students are expected to carry out research in biomedical engineering
\end{flushleft}


\begin{flushleft}
disciplines and preferably be able to publish or communicate their
\end{flushleft}


\begin{flushleft}
work at the end of project. A total of 18 credits including 12 from this
\end{flushleft}


\begin{flushleft}
curriculum shall enable them to submit M.Tech. Dissertation, which
\end{flushleft}


\begin{flushleft}
shall also be regarded as a publication.
\end{flushleft}





303





\begin{flushleft}
\newpage
Biomedical Engineering
\end{flushleft}





\begin{flushleft}
BML810 Tissue Engineering
\end{flushleft}


\begin{flushleft}
3 Credits (3-0-0)
\end{flushleft}


\begin{flushleft}
The course will cover importance and scope of tissue engineering,
\end{flushleft}


\begin{flushleft}
Introduction to biomaterials and scaffolds, Criteria of modifying
\end{flushleft}


\begin{flushleft}
biomaterials as tissue engineering scaffolds, Properties and types of
\end{flushleft}


\begin{flushleft}
scaffolds, Different methods employed in the synthesis of scaffolds,
\end{flushleft}


\begin{flushleft}
animal cell biology, stem cells, organization of cells into tissues, tissue
\end{flushleft}


\begin{flushleft}
microenvironment, tissue injury and wound healing. Basic immunology,
\end{flushleft}


\begin{flushleft}
response of body to foreign materials. Animal cell culture on scaffolds,
\end{flushleft}


\begin{flushleft}
consequences, optimization strategies and important considerations
\end{flushleft}


\begin{flushleft}
for Skin, Liver, Bone, Cartilage, Nerve and Vascular tissue engineering.
\end{flushleft}





\begin{flushleft}
BML815 Selected Topics in Biomedical Engineering
\end{flushleft}


\begin{flushleft}
2 Credits (2-0-0)
\end{flushleft}


\begin{flushleft}
Select current and emerging topics in biomedical engineering will be
\end{flushleft}


\begin{flushleft}
covered; details will be decided by the instructor.
\end{flushleft}





\begin{flushleft}
BML820 Biomaterials
\end{flushleft}


\begin{flushleft}
3 Credits (3-0-0)
\end{flushleft}


\begin{flushleft}
Introduction to the use of implants. Structure and properties
\end{flushleft}


\begin{flushleft}
of materials used as implants : polymers, ceramics, metal and
\end{flushleft}


\begin{flushleft}
composites; biological response to implants, wound healing process,
\end{flushleft}


\begin{flushleft}
cellular response to foreign materials, criteria for selecting implants
\end{flushleft}


\begin{flushleft}
both for soft tissue and hard tissue, polymers used as vascular
\end{flushleft}


\begin{flushleft}
prosthesis, contact lens and reconstructive surgery materials.
\end{flushleft}





\begin{flushleft}
BML830 Biosensor Technology
\end{flushleft}


\begin{flushleft}
4 Credits (3-0-2)
\end{flushleft}


\begin{flushleft}
Measurements and instrumentation principles. Fundamentals of
\end{flushleft}


\begin{flushleft}
transducers and sensors, their sensitivity, specificity, linearity and
\end{flushleft}


\begin{flushleft}
transduction system analysis. Introduction to biosensors; transduction
\end{flushleft}


\begin{flushleft}
principles used in biosensors viz. electrical, optical, microchip sensors
\end{flushleft}


\begin{flushleft}
and Surface acoustic wave devices and transducers and related
\end{flushleft}


\begin{flushleft}
technology. Biotechnological components of the sensor based on
\end{flushleft}


\begin{flushleft}
enzymes, antigen-antibody reaction, biochemical detection of analytes,
\end{flushleft}


\begin{flushleft}
organelles, whole cell, receptors, DNA probe, pesticide detection,
\end{flushleft}


\begin{flushleft}
sensors for pollutant gases. Kinetics, stability and reusability of
\end{flushleft}


\begin{flushleft}
sensors. Selected examples and future developments.
\end{flushleft}





\begin{flushleft}
BMV840 Emerging Biomedical Technology \& Health Care
\end{flushleft}


\begin{flushleft}
1 Credit (1-0-0)
\end{flushleft}


\begin{flushleft}
Importance of health related data collection and analysis,
\end{flushleft}


\begin{flushleft}
Epidemiological survey; brief them about various communicable \& noncommunicable diseases, path-physiological processes, environmental
\end{flushleft}





\begin{flushleft}
health and Life style diseases. Define the process of evolution of
\end{flushleft}


\begin{flushleft}
emerging technologies to solve the current health problems through
\end{flushleft}


\begin{flushleft}
an integrated approach of synergizing the discipline of medicine,
\end{flushleft}


\begin{flushleft}
engineering and management systems. Importance/ methodology
\end{flushleft}


\begin{flushleft}
of conducting clinical trials-human \& animals.
\end{flushleft}





\begin{flushleft}
BML850 Cancer: Diagnosis and Therapy
\end{flushleft}


\begin{flushleft}
3 Credit (3-0-0)
\end{flushleft}


\begin{flushleft}
Cancer and its classes; Hallmarks of cancer: Evasion of Apoptosis,
\end{flushleft}


\begin{flushleft}
Limitless replicative potential, Sustained Angiogenesis, Inflammation;
\end{flushleft}


\begin{flushleft}
Causes of Cancer: Carcinogens, oncogenes, mutations, viruses,
\end{flushleft}


\begin{flushleft}
disregulation of cell cycle and the checkpoints; Tumor architecture,
\end{flushleft}


\begin{flushleft}
Importance of Hypoxia and angiogenesis in cancer; Tumor metabolism,
\end{flushleft}


\begin{flushleft}
Metastatic potential of cancer; Cancer Stem Cells and Biomarkers
\end{flushleft}


\begin{flushleft}
of Cancer; Diagnosis of cancer: Biopsy, Imaging, Endoscopy, Blood
\end{flushleft}


\begin{flushleft}
work; Therapy: Chemotherapy (small molecule, nanoparticle based),
\end{flushleft}


\begin{flushleft}
radiation, hyperthermia, immunotherapy, photodynamic, transplants
\end{flushleft}


\begin{flushleft}
and transfusions, targeted therapy, RNAi, non-invasive technologies;
\end{flushleft}


\begin{flushleft}
Resistance in Cancer; Scientific advances for understanding the origin,
\end{flushleft}


\begin{flushleft}
diagnosis and treatment of Cancer; Future prospects for cancer cure
\end{flushleft}


\begin{flushleft}
and diagnosis.
\end{flushleft}





\begin{flushleft}
BML860 Nanomedicine
\end{flushleft}


\begin{flushleft}
3 Credit (3-0-0)
\end{flushleft}


\begin{flushleft}
Introduction to some basic nanoscience: quantum confinement and
\end{flushleft}


\begin{flushleft}
its effect; surface plasmon etc. Nanomaterial synthesis including
\end{flushleft}


\begin{flushleft}
bottoms-up and top-down approaches. The significance of nano
\end{flushleft}


\begin{flushleft}
size, multiplexing and multilayering. Properties of nanoparticles and
\end{flushleft}


\begin{flushleft}
its dependence on shape, size, charge and aspect ratio. Interface
\end{flushleft}


\begin{flushleft}
of nanoparticles with biological systems (cells, viruses, bacteria, in
\end{flushleft}


\begin{flushleft}
vivo etc.) Techniques used for nanoparticle characterization before
\end{flushleft}


\begin{flushleft}
and after biological interface. Functional nanomaterials for biological
\end{flushleft}


\begin{flushleft}
and medical applications: Design criteria and synthetic protocols;
\end{flushleft}


\begin{flushleft}
Nanomaterials in tissue engineering, drug delivery, biosensors,
\end{flushleft}


\begin{flushleft}
hyperthermia, photodynamic therapy, etc. Modulating the specific
\end{flushleft}


\begin{flushleft}
biological response by nanostructures. Nanotoxicology.
\end{flushleft}





\begin{flushleft}
BMV870 Vascular Bioengineering
\end{flushleft}


\begin{flushleft}
1 Credit (1-0-0)
\end{flushleft}


\begin{flushleft}
Embryology and formation of vascular networking in fetus and adult
\end{flushleft}


\begin{flushleft}
human body, autonomic nervous system influences, peculiarities
\end{flushleft}


\begin{flushleft}
of micro and macro vasculatures, the physiological fluid dynamic
\end{flushleft}


\begin{flushleft}
principles involved, the molecular level changes occurring in normal
\end{flushleft}


\begin{flushleft}
and abnormal conditions like atherosclerosis, cancers, utero-placental
\end{flushleft}


\begin{flushleft}
system and various imaging modalities.
\end{flushleft}





304





\begin{flushleft}
\newpage
Centre for Energy Studies
\end{flushleft}


\begin{flushleft}
ESL300 Self-Organizing Dynamical Systems
\end{flushleft}


\begin{flushleft}
3 Credits (3-0-0)
\end{flushleft}


\begin{flushleft}
Pre-requisites: EC60 (for UG students)
\end{flushleft}





\begin{flushleft}
(iii) Photovoltaic energy systems (iv) Fuel cells (v) Plasma diodes (vi)
\end{flushleft}


\begin{flushleft}
Magneto hydrodynamic Power generators and (vii) Advanced energy
\end{flushleft}


\begin{flushleft}
conversion systems.
\end{flushleft}





\begin{flushleft}
Dynamical systems dissipative and area preserving, Patterns in
\end{flushleft}


\begin{flushleft}
Hamiltonian dynamics invariants and symmetry, KAM theorem / coherent
\end{flushleft}


\begin{flushleft}
structures, complexity and pattern formation, Belousov - Zhabutinsky
\end{flushleft}


\begin{flushleft}
reaction, Landau-Ginzburg / mean-field models, scaling fractals,
\end{flushleft}


\begin{flushleft}
Cellular automata, Wavelet transforms, Phase transitions and order
\end{flushleft}


\begin{flushleft}
parameter, Criticality the border of order and chaos, Entropy and
\end{flushleft}


\begin{flushleft}
direction of time, Negentropic systems, Self-organized criticality,
\end{flushleft}


\begin{flushleft}
lattice models, Examples: Electrical circuits, Management systems,
\end{flushleft}


\begin{flushleft}
Astrophysical systems, Plasma and magnetic surface systems,
\end{flushleft}


\begin{flushleft}
Biological systems, Non-linear systems.
\end{flushleft}





\begin{flushleft}
ESL330 Energy, Ecology \& Environment
\end{flushleft}


\begin{flushleft}
4 Credits (3-1-0)
\end{flushleft}


\begin{flushleft}
Overlaps with: Some overlap with ESL710
\end{flushleft}


\begin{flushleft}
Pre-requisites: EC60 (for UG students)
\end{flushleft}





\begin{flushleft}
ESL710 Energy, Ecology and Environment
\end{flushleft}


\begin{flushleft}
3 Credits (3-0-0)
\end{flushleft}


\begin{flushleft}
Interrelationship between energy and environment, Sun as a source
\end{flushleft}


\begin{flushleft}
of energy, nature of its radiation, Biological processes, photosynthesis,
\end{flushleft}


\begin{flushleft}
Autecology and Synecology, Population, Community Ecosystem
\end{flushleft}


\begin{flushleft}
(wetland, terrestrial, marine) Food chains, Ecosystem theories. Sources
\end{flushleft}


\begin{flushleft}
of energy, Classification of energy sources, Environmental issues
\end{flushleft}


\begin{flushleft}
related to harnessing to fossil fuels (coal, oil, natural gas), geothermal,
\end{flushleft}


\begin{flushleft}
tidal, nuclear energy, solar, wind, hydropower, biomass, Energy flow
\end{flushleft}


\begin{flushleft}
and nutrient cycling in ecosystems, Environmental degradation,
\end{flushleft}


\begin{flushleft}
primary and secondary pollutants. Thermal/ radioactive pollution, air
\end{flushleft}


\begin{flushleft}
and water pollution, Micro climatic effects of pollution, Pollution from
\end{flushleft}


\begin{flushleft}
stationary and mobile sources, Biological effects of radiation, heat
\end{flushleft}


\begin{flushleft}
and radioactivity disposal, Acid rain, Global warming and greenhouse
\end{flushleft}


\begin{flushleft}
gases, Ozone layer depletion.
\end{flushleft}





\begin{flushleft}
Concepts of ecosystems and environment, Characteristics and types of
\end{flushleft}


\begin{flushleft}
ecosystems, Autecology and synecology, Energy flow in ecosystems,
\end{flushleft}


\begin{flushleft}
Feedback loops, Trophic webs, Eco-technology and Eco-development,
\end{flushleft}


\begin{flushleft}
Energy-environment interaction, Impact of energy sources (coal, oil,
\end{flushleft}


\begin{flushleft}
natural gas, solar, wind, biomass, hydro, geothermal, tidal, wave,
\end{flushleft}


\begin{flushleft}
ocean thermal and nuclear) on environment, local regional and global
\end{flushleft}


\begin{flushleft}
implications, Approaches to mitigate environmental emissions from
\end{flushleft}


\begin{flushleft}
energy sector, Global initiatives Kyoto Protocol, Clean development
\end{flushleft}


\begin{flushleft}
mechanism, Case studies.
\end{flushleft}





\begin{flushleft}
ESL711 Fuel Technology
\end{flushleft}


\begin{flushleft}
3 Credits (3-0-0)
\end{flushleft}





\begin{flushleft}
ESL340 Non-Conventional Source of Energy
\end{flushleft}


\begin{flushleft}
4 Credits (3-0-2)
\end{flushleft}


\begin{flushleft}
Overlaps with: Some overlap with ESL740
\end{flushleft}


\begin{flushleft}
Pre-requisites: EC60 (for UG students)
\end{flushleft}





\begin{flushleft}
ESL714 Power Plant Engineering
\end{flushleft}


\begin{flushleft}
3 Credits (3-0-0)
\end{flushleft}





\begin{flushleft}
Global \& National energy scenarios, Forms \& characteristics of
\end{flushleft}


\begin{flushleft}
renewable energy sources, Solar radiation, Flat plate collectors, Solar
\end{flushleft}


\begin{flushleft}
concentrators, Thermal Applications of solar energy, Photovoltaics
\end{flushleft}


\begin{flushleft}
technology and applications, Energy storage, Energy from biomass,
\end{flushleft}


\begin{flushleft}
Thermochemical, Biochemical conversion to fuels, biogas and its
\end{flushleft}


\begin{flushleft}
applications, Wind characteristics, Resource assessment, Horizontal \&
\end{flushleft}


\begin{flushleft}
vertical axis wind turbines, Electricity generation and water pumping,
\end{flushleft}


\begin{flushleft}
Micro/Mini hydropower systems, Water pumping and conversion to
\end{flushleft}


\begin{flushleft}
electricity, Hydraulic ram pump, Ocean Thermal Energy Conversion
\end{flushleft}


\begin{flushleft}
(OTEC), Geothermal, Tidal and Wave energies, Material aspects of
\end{flushleft}


\begin{flushleft}
Renewable energy technologies and systems
\end{flushleft}





\begin{flushleft}
ESL350 Energy Conservation and Management
\end{flushleft}


\begin{flushleft}
3 Credits (3-0-0)
\end{flushleft}


\begin{flushleft}
Overlaps with: ESL720
\end{flushleft}


\begin{flushleft}
Pre-requisites: EC60 (for UG students)
\end{flushleft}





\begin{flushleft}
Solid, liquid and gaseous fuels, Coal as a source of energy and
\end{flushleft}


\begin{flushleft}
chemicals in India, Coal preparation, Carbonization, Gasification and
\end{flushleft}


\begin{flushleft}
liquefaction of coal and lignite, Principle of combustion, Petroleum
\end{flushleft}


\begin{flushleft}
and its derived products, Testing of liquid fuels, Petroleum refining
\end{flushleft}


\begin{flushleft}
processes, Inter-conversion of fuels, Natural gases and its derivatives,
\end{flushleft}


\begin{flushleft}
sources, potential, Gas hydrates, Combustion appliances for solid, liquid
\end{flushleft}


\begin{flushleft}
and gaseous fuels, Introduction to nuclear fuel, RDF, Bio-fuels, etc.
\end{flushleft}





\begin{flushleft}
Types of thermal power stations, Steam power stations based on fossil
\end{flushleft}


\begin{flushleft}
fuels, Economy and thermal scheme of the steam power stations,
\end{flushleft}


\begin{flushleft}
Thermal power plant equipment boilers, super heaters, super critical
\end{flushleft}


\begin{flushleft}
steam generator, economizers, feed water heater, condensers,
\end{flushleft}


\begin{flushleft}
combustion chamber and gas loop, turbines, cooling towers, etc.
\end{flushleft}


\begin{flushleft}
Gas turbine power stations, Combined cycle power plants, Internal
\end{flushleft}


\begin{flushleft}
combustion engine plant for peak load, standby and start up, Elements
\end{flushleft}


\begin{flushleft}
of hydropower and wind turbine, Elements of nuclear power plants,
\end{flushleft}


\begin{flushleft}
Nuclear reactors and fuels. Recent advances in power plants (IGCC,
\end{flushleft}


\begin{flushleft}
super critical power plants, etc.). Case studies, Introduction to solar
\end{flushleft}


\begin{flushleft}
power generation, Sterling engine, Decentralized power technologies.
\end{flushleft}





\begin{flushleft}
ESL718 Power Generation, Transmission and
\end{flushleft}


\begin{flushleft}
Distribution
\end{flushleft}


\begin{flushleft}
3 Credits (3-0-0)
\end{flushleft}


\begin{flushleft}
Generation: Synchronous generator operation, Power angle
\end{flushleft}


\begin{flushleft}
characteristics and the infinite bus concept, dynamic analysis and
\end{flushleft}


\begin{flushleft}
modeling of synchronous machines, Excitations systems, Prime-mover
\end{flushleft}


\begin{flushleft}
governing systems, Automatic generation control.
\end{flushleft}





\begin{flushleft}
Thermodynamic basis of energy conservation, Irreversible processes,
\end{flushleft}


\begin{flushleft}
Reversibility and Availability, Exergy and available energy, Energy
\end{flushleft}


\begin{flushleft}
conservation in HVAC systems and thermal power plants, Solar
\end{flushleft}


\begin{flushleft}
systems, Second law efficiency and LAW, Heat pumps and Heat
\end{flushleft}


\begin{flushleft}
pipes for space conditioning, Heat recovery and Heat exchangers,
\end{flushleft}


\begin{flushleft}
Furnaces and cooling towers, Energy conservation in buildings,
\end{flushleft}


\begin{flushleft}
U-Value of walls / roof, Ventilation systems - Fan and ducts Lighting
\end{flushleft}


\begin{flushleft}
Systems -- Different light sources and luminous efficacy, Insulation
\end{flushleft}


\begin{flushleft}
use -- Materials properties, Optimum thickness, Thermo economic
\end{flushleft}


\begin{flushleft}
analysis, Energy conservation in electrical devices and systems,
\end{flushleft}


\begin{flushleft}
Economic evaluation of energy conservation measures, Electric motors
\end{flushleft}


\begin{flushleft}
and transformers, Inverters and UPS, Voltages stabilizers, Energy audit
\end{flushleft}


\begin{flushleft}
and Instrumentation.
\end{flushleft}





\begin{flushleft}
ESL360 Direct Energy Conversion Methods
\end{flushleft}


\begin{flushleft}
4 Credits (3-1-0)
\end{flushleft}


\begin{flushleft}
Overlaps with: Some overlap with ESL730
\end{flushleft}


\begin{flushleft}
Energy classification, Sources and utilization, Principle of energy
\end{flushleft}


\begin{flushleft}
conversion, Indirect / direct energy conversion, Basic principles of
\end{flushleft}


\begin{flushleft}
design and operations of (i) Thermoelectric (ii) Thermionic convertors
\end{flushleft}





\begin{flushleft}
Auxiliaries: Power system stabilizer, Artificial intelligent controls,
\end{flushleft}


\begin{flushleft}
Power quality.
\end{flushleft}


\begin{flushleft}
AC Transmission: Overhead and cables, Transmission line equations,
\end{flushleft}


\begin{flushleft}
Regulation and transmission line losses, Reactive power compensation,
\end{flushleft}


\begin{flushleft}
Flexible AC transmission.
\end{flushleft}


\begin{flushleft}
HVDC transmission: HVDC converters, advantages and economic
\end{flushleft}


\begin{flushleft}
considerations, converter control characteristics, analysis of HVDC link
\end{flushleft}


\begin{flushleft}
performance, Multi-terminal DC system, HVDC and FACTS.
\end{flushleft}


\begin{flushleft}
Distribution: Distribution systems, conductor size, Kelvin's law,
\end{flushleft}


\begin{flushleft}
performance calculations and analysis, Distribution inside and
\end{flushleft}


\begin{flushleft}
commercial buildings entrance terminology, Substation and feeder
\end{flushleft}


\begin{flushleft}
circuit design considerations, distribution automation, Futuristic
\end{flushleft}


\begin{flushleft}
power generation.
\end{flushleft}





\begin{flushleft}
ESL720 Energy Conservation
\end{flushleft}


\begin{flushleft}
3 Credits (3-0-0)
\end{flushleft}


\begin{flushleft}
Introduction, Thermodynamics of energy conservation, Energy and
\end{flushleft}


\begin{flushleft}
exergy concepts, Irreversibility and second law analysis and efficiency
\end{flushleft}


\begin{flushleft}
of thermal systems such as mixing, throttling, drying and solar thermal
\end{flushleft}





305





\begin{flushleft}
\newpage
Energy Studies
\end{flushleft}





\begin{flushleft}
systems, Thermal power plant cycles. Refrigeration and air conditioning
\end{flushleft}


\begin{flushleft}
cycles, thermal insulation in energy conservation, energy conservation
\end{flushleft}


\begin{flushleft}
through controls, electric energy conservation in building heating and
\end{flushleft}


\begin{flushleft}
lighting, energy efficient motors, Tariffs and power factor improvement
\end{flushleft}


\begin{flushleft}
in electrical systems, Energy conservation in domestic appliances,
\end{flushleft}


\begin{flushleft}
transport, energy auditing, energy savings in boilers and furnaces,
\end{flushleft}


\begin{flushleft}
energy conservation Act, Energy conservation in small scale domestic
\end{flushleft}


\begin{flushleft}
appliances and agriculture.
\end{flushleft}





\begin{flushleft}
ESL722 Integrated Energy Systems
\end{flushleft}


\begin{flushleft}
3 Credits (3-0-0)
\end{flushleft}


\begin{flushleft}
Pattern of fuel consumption: agricultural, domestic, industrial and
\end{flushleft}


\begin{flushleft}
community needs, Projection of energy demands, Substitution of
\end{flushleft}


\begin{flushleft}
conventional sources by alternative sources and more efficient modern
\end{flushleft}


\begin{flushleft}
technologies, Potential, availability as well as capacity of solar, wind,
\end{flushleft}


\begin{flushleft}
biogas, natural gas, forest produce, tidal, geothermal, mini-hydro and
\end{flushleft}


\begin{flushleft}
other modern applications, Hybrid and integrated energy systems,
\end{flushleft}


\begin{flushleft}
Total energy concept and waste heat utilization, Energy modeling to
\end{flushleft}


\begin{flushleft}
optimize different systems.
\end{flushleft}





\begin{flushleft}
ESL726 Waste Heat Recovery
\end{flushleft}


\begin{flushleft}
3 Credits (3-0-0)
\end{flushleft}


\begin{flushleft}
Pre-requisites: EC 75 (for UG Students in Minor Area)
\end{flushleft}


\begin{flushleft}
Introduction to Waste heat recovery, Classifications, Principles,
\end{flushleft}


\begin{flushleft}
Utilizations, Strategy of using waste heat recovery, Basic Heat
\end{flushleft}


\begin{flushleft}
Exchanger Design Concepts, Heat Exchanger equipment classifications,
\end{flushleft}


\begin{flushleft}
Steam generation equipment, Power plant heat recovery systems,
\end{flushleft}


\begin{flushleft}
Commercial waste heat recovery systems with detailed study of
\end{flushleft}


\begin{flushleft}
Recuperators, Radiation/Convective Hybrid Recuperator, Ceramic
\end{flushleft}


\begin{flushleft}
Regenerator, Introduction to efficient building design.
\end{flushleft}





\begin{flushleft}
ESL730 Direct Energy Conversion
\end{flushleft}


\begin{flushleft}
3 Credits (3-0-0)
\end{flushleft}


\begin{flushleft}
Basic science of energy conversion, Indirect verses direct conversion,
\end{flushleft}


\begin{flushleft}
Physics of semiconductor junctions for photovoltaic and photoelectrochemical conversion of solar energy, Fabrication and evaluation
\end{flushleft}


\begin{flushleft}
of various solar cells in photovoltaic power generation systems,
\end{flushleft}


\begin{flushleft}
Technology and physics of thermo-electric generations, Thermalelectric materials and optimization studies, Basic concepts and design
\end{flushleft}


\begin{flushleft}
considerations of MHD generators, Cycle analysis of MHD systems,
\end{flushleft}


\begin{flushleft}
Thermonic power conversion and plasma diodes, Thermodynamics
\end{flushleft}


\begin{flushleft}
and performance of fuel cells and their applications.
\end{flushleft}





\begin{flushleft}
conversion processes, hydrolysis and hydrogenation, Solvent extraction
\end{flushleft}


\begin{flushleft}
of hydrocarbons, Fuel combustion into electricity, case studies.
\end{flushleft}





\begin{flushleft}
ESL734 Nuclear Energy
\end{flushleft}


\begin{flushleft}
3 Credits (3-0-0)
\end{flushleft}


\begin{flushleft}
Introduction: Scope of nuclear energy (fission and fusion energy),
\end{flushleft}


\begin{flushleft}
typical reactions
\end{flushleft}


\begin{flushleft}
Basics Concepts: Binding Energy of a nuclear reaction, mass energy
\end{flushleft}


\begin{flushleft}
equivalence and conservation laws, nuclear stability and radioactive
\end{flushleft}


\begin{flushleft}
decay, radioactivity calculations.
\end{flushleft}


\begin{flushleft}
Interaction of Neutrons with Matter: Compound nucleus formation,
\end{flushleft}


\begin{flushleft}
elastic and inelastic scattering, cross sections, energy loss in scattering
\end{flushleft}


\begin{flushleft}
collisions, polyenergetic neutrons, critical energy of fission, fission
\end{flushleft}


\begin{flushleft}
cross sections, fission products, fission neutrons, energy released in
\end{flushleft}


\begin{flushleft}
fission, r-ray interaction with matter and energy deposition, fission
\end{flushleft}


\begin{flushleft}
fragments.
\end{flushleft}


\begin{flushleft}
The Fission Reactor: The fission chain reaction, reactor fuels,
\end{flushleft}


\begin{flushleft}
conversion and breeding, the nuclear power resources, nuclear power
\end{flushleft}


\begin{flushleft}
plant \& its components, power reactors and current status.
\end{flushleft}


\begin{flushleft}
Reactor Theory: Neutron flux, Fick's law, continuity equation, diffusion
\end{flushleft}


\begin{flushleft}
equation, boundary conditions, solutions of the DE, group diffusion
\end{flushleft}


\begin{flushleft}
method, Neutron moderation (two group calculation), one group
\end{flushleft}


\begin{flushleft}
reactor equation and the slab reactor Health Hazards: radiation
\end{flushleft}


\begin{flushleft}
protection \& shielding.
\end{flushleft}


\begin{flushleft}
Nuclear Fusion: Fusion reactions, reaction cross-sections, reaction
\end{flushleft}


\begin{flushleft}
rates, fusion power density, radiation losses, ideal fusion ignition,
\end{flushleft}


\begin{flushleft}
Ideal plasma confinement \& Lawson criterion.
\end{flushleft}


\begin{flushleft}
Plasma Concepts: Saha equation, Coulomb scattering, radiation from
\end{flushleft}


\begin{flushleft}
plasma, transport phenomena.
\end{flushleft}


\begin{flushleft}
Plasma Confinement Schemes: Magnetic and inertial confinement,
\end{flushleft}


\begin{flushleft}
current status.
\end{flushleft}





\begin{flushleft}
ESL737 Plasma Based Materials Processing
\end{flushleft}


\begin{flushleft}
3 Credits (3-0-0)
\end{flushleft}


\begin{flushleft}
Introduction: Plasma based processing of materials
\end{flushleft}


\begin{flushleft}
Plasma Concepts: Plasma fluid equations, single particle motions,
\end{flushleft}


\begin{flushleft}
unmagnetized plasma dynamics, diffusion and resistivity, the DC
\end{flushleft}


\begin{flushleft}
sheath and probe diagnostics.
\end{flushleft}


\begin{flushleft}
Basics of Plasma Chemistry: Chemical reactions and equilibrium,
\end{flushleft}


\begin{flushleft}
chemical kinetics, particle and energy balance in discharges.
\end{flushleft}





\begin{flushleft}
ESL731 Biomass - A Renewable Resource
\end{flushleft}


\begin{flushleft}
3 Credits (3-0-0)
\end{flushleft}


\begin{flushleft}
Pre-requisites: EC 75 (for UG Students in Minor Area)
\end{flushleft}


\begin{flushleft}
Biogas-animal dung and agroresidues and other cellulose wastes,
\end{flushleft}


\begin{flushleft}
ethanol from wheat or corn, sugar cane, sweetsorghum, beet roots,
\end{flushleft}


\begin{flushleft}
grapes, starchetc, pyrolysis of biomass. Direct combustion of biomass,
\end{flushleft}


\begin{flushleft}
Improved stoves routes. Second Generation Biofuels:- Biodiesel from
\end{flushleft}


\begin{flushleft}
oil seeds, Gasification of agroresidues, sawdust etc, Micro-power
\end{flushleft}


\begin{flushleft}
generation through biomass gasifiers, waste incineration fluidized
\end{flushleft}


\begin{flushleft}
bed combustion of biomass. Third Generation Biofuels:- Algae based
\end{flushleft}


\begin{flushleft}
Biodiesel, Ethanol, Hydrogen, alcohols from agroresidues, chemical
\end{flushleft}


\begin{flushleft}
composition of lignocellulosicbiomass, fuels and chemicals from each
\end{flushleft}


\begin{flushleft}
component biomass (Hemicellulose, Cellulose, and Lignin), Chemical,
\end{flushleft}


\begin{flushleft}
Thermochemical and Biochemical processes, Availability of biomass,
\end{flushleft}


\begin{flushleft}
petrocrops, aquatic biomass and its potential, concept of bioeconomy
\end{flushleft}


\begin{flushleft}
and biorefineries.
\end{flushleft}





\begin{flushleft}
Low Pressure Plasma Discharges: DC discharges, RF discharges Capacitively and inductively coupled, microwave, ECR and helicon
\end{flushleft}


\begin{flushleft}
discharges.
\end{flushleft}


\begin{flushleft}
Low Pressure Materials Processing Applications: Etching for VLSI,
\end{flushleft}


\begin{flushleft}
film deposition, surface modification and other applications (plasma
\end{flushleft}


\begin{flushleft}
nitriding, plasma ion implantation, biomedical and tribological
\end{flushleft}


\begin{flushleft}
applications).
\end{flushleft}


\begin{flushleft}
High Pressure Plasmas: High pressure non-equilibrium plasmas,
\end{flushleft}


\begin{flushleft}
thermal plasmas -- the plasma arc, the plasma as a heat source, the
\end{flushleft}


\begin{flushleft}
plasma as chemical catalyst.
\end{flushleft}


\begin{flushleft}
Applications of High Pressure Plasmas: Air pollution control, plasma
\end{flushleft}


\begin{flushleft}
pyrolysis and waste removal, plasma based metallurgy -- ore
\end{flushleft}


\begin{flushleft}
enrichment, applications in ceramics, plasma assisted recycling.
\end{flushleft}





\begin{flushleft}
ESL732 Bioconversion and Processing of Waste
\end{flushleft}


\begin{flushleft}
3 Credits (3-0-0)
\end{flushleft}





\begin{flushleft}
ESL740 Non-conventional Sources of Energy
\end{flushleft}


\begin{flushleft}
3 Credits (3-0-0)
\end{flushleft}





\begin{flushleft}
Biomass and solid wastes, Broad classification, Production of
\end{flushleft}


\begin{flushleft}
biomass, photosynthesis, Separation of components of solid wastes
\end{flushleft}


\begin{flushleft}
and processing techniques, Agro and forestry residues utilisation
\end{flushleft}


\begin{flushleft}
through conversion routes: biological, chemical and thermo chemical,
\end{flushleft}


\begin{flushleft}
Bioconversion into biogas, mechanism, Composting technique,
\end{flushleft}


\begin{flushleft}
Bioconversion of substrates into alcohols, Bioconversion into hydrogen,
\end{flushleft}


\begin{flushleft}
Thermo chemical conversion of biomass, conversion to solid, liquid
\end{flushleft}


\begin{flushleft}
and gaseous fuels, pyrolysis, gasification, combustion, Chemical
\end{flushleft}





\begin{flushleft}
Types of non-conventional sources, Solar energy principles and
\end{flushleft}


\begin{flushleft}
applications, efficiency of solar thermal and PV systems, Biomass:
\end{flushleft}


\begin{flushleft}
generation, characterization, Biogas: aerobic and anaerobic bioconversion processes, microbial reactions purification, properties of
\end{flushleft}


\begin{flushleft}
biogas, Storage and enrichment, Tidal and wind energy potential
\end{flushleft}


\begin{flushleft}
and conversion efficiency, Mini/micro hydro power: classification of
\end{flushleft}


\begin{flushleft}
hydropower schemes, classification of water turbine, Turbine theory,
\end{flushleft}


\begin{flushleft}
Essential components of hydroelectric system, system efficiency,
\end{flushleft}





306





\begin{flushleft}
\newpage
Energy Studies
\end{flushleft}





\begin{flushleft}
Fusion: Basic concepts, fusion reaction physics, Thermonuclear
\end{flushleft}


\begin{flushleft}
fusion reaction criteria, Confinement schemes, Inertial and magnetic
\end{flushleft}


\begin{flushleft}
confinement fusion, Current status, Geothermal: Geothermal regions,
\end{flushleft}


\begin{flushleft}
geothermal sources, dry rock and hot aquifer analysis, Geothermal
\end{flushleft}


\begin{flushleft}
energy conversion technologies, OTEC, Wave Energy.
\end{flushleft}





\begin{flushleft}
ESL742 Economics and Financing of Renewable Energy
\end{flushleft}


\begin{flushleft}
Systems
\end{flushleft}


\begin{flushleft}
3 Credits (3-0-0)
\end{flushleft}


\begin{flushleft}
Pre-requisites: EC 75 (for UG students in Minor Area)
\end{flushleft}


\begin{flushleft}
Overview of renewable energy technologies. Relevance of economic
\end{flushleft}


\begin{flushleft}
and financial viability evaluation of renewable energy technologies,
\end{flushleft}


\begin{flushleft}
Basics of engineering economics, Financial feasibility evaluation of
\end{flushleft}


\begin{flushleft}
renewable energy technologies, Social cost -- benefit analysis of
\end{flushleft}


\begin{flushleft}
renewable energy technologies. Technology dissemination models,
\end{flushleft}


\begin{flushleft}
Volume and learning effects on costs of renewable energy systems,
\end{flushleft}


\begin{flushleft}
Dynamics of fuel substitution by renewable energy systems and
\end{flushleft}


\begin{flushleft}
quantification of benefits. Fiscal, financial and other incentives for
\end{flushleft}


\begin{flushleft}
promotion of renewable energy systems and their effect on financial
\end{flushleft}


\begin{flushleft}
and economic viability. Financing of renewable energy systems, Carbon
\end{flushleft}


\begin{flushleft}
finance potential of renewable energy technologies and associated
\end{flushleft}


\begin{flushleft}
provisions. Software for financial evaluation of renewable energy
\end{flushleft}


\begin{flushleft}
systems. Case studies on financial and economic feasibility evaluation
\end{flushleft}


\begin{flushleft}
of renewable energy devices and systems.
\end{flushleft}





\begin{flushleft}
ESL746 Hydrogen Energy
\end{flushleft}


\begin{flushleft}
3 Credits (3-0-0)
\end{flushleft}


\begin{flushleft}
Introduction of Hydrogen Energy Systems
\end{flushleft}


\begin{flushleft}
Hydrogen pathways introduction -- current uses, General introduction
\end{flushleft}


\begin{flushleft}
to infrastructure requirement for hydrogen production, storage,
\end{flushleft}


\begin{flushleft}
dispensing and utilization, and Hydrogen production power plants.
\end{flushleft}


\begin{flushleft}
Hydrogen Production Processes
\end{flushleft}


\begin{flushleft}
Thermal-Steam Reformation -- Thermo chemical Water Splitting --
\end{flushleft}


\begin{flushleft}
Gasification -- Pyrolysis, Nuclear thermo catalytic and partial oxidation
\end{flushleft}


\begin{flushleft}
methods. Electrochemical -- Electrolysis -- Photo electro chemical.
\end{flushleft}


\begin{flushleft}
Biological -- Photo Biological -- Anaerobic Digestion -- Fermentative
\end{flushleft}


\begin{flushleft}
Micro-organisms.
\end{flushleft}


\begin{flushleft}
Hydrogen Storage
\end{flushleft}


\begin{flushleft}
Physical and chemical properties -- General storage methods,
\end{flushleft}


\begin{flushleft}
compressed storage -- Composite cylinders -- Glass micro sphere
\end{flushleft}


\begin{flushleft}
storage - Zeolites, Metal hydride storage, chemical hydride storage
\end{flushleft}


\begin{flushleft}
and cryogenic storage.
\end{flushleft}


\begin{flushleft}
Hydrogen Utilization
\end{flushleft}


\begin{flushleft}
Overview of Hydrogen utilization: I.C. Engines, gas turbines, hydrogen
\end{flushleft}


\begin{flushleft}
burners, power plant, refineries, domestic and marine applications,
\end{flushleft}


\begin{flushleft}
Hydrogen fuel quality, performance, COV, emission and combustion
\end{flushleft}


\begin{flushleft}
characteristics of Spark Ignition engines for hydrogen, back firing,
\end{flushleft}


\begin{flushleft}
knocking, volumetric efficiency, hydrogen manifold and direct injection,
\end{flushleft}


\begin{flushleft}
fumigation, NOx controlling techniques, dual fuel engine, durability
\end{flushleft}


\begin{flushleft}
studies, field trials, emissions and climate change.
\end{flushleft}


\begin{flushleft}
Hydrogen Safety
\end{flushleft}


\begin{flushleft}
Safety barrier diagram, risk analysis, safety in handling and refueling
\end{flushleft}


\begin{flushleft}
station, safety in vehicular and stationary applications, fire detecting
\end{flushleft}


\begin{flushleft}
system, safety management, and simulation of crash tests.
\end{flushleft}





\begin{flushleft}
ESL748 Economics of Energy Conservation
\end{flushleft}


\begin{flushleft}
3 Credits (3-0-0)
\end{flushleft}


\begin{flushleft}
Pre-requisites: EC 75 (for UG Students in Minor Area)
\end{flushleft}





\begin{flushleft}
Approaches for considering uncertainty in appraisal/evaluation;
\end{flushleft}


\begin{flushleft}
Existing and potential incentives for promoting energy conservation
\end{flushleft}


\begin{flushleft}
measures, regulations and policy measures; Carbon mitigation
\end{flushleft}


\begin{flushleft}
benefits; Development of techno-economic models; Software for
\end{flushleft}


\begin{flushleft}
economic assessment of energy conservation projects; Financing of
\end{flushleft}


\begin{flushleft}
energy conservation projects; Case studies.
\end{flushleft}





\begin{flushleft}
ESL750 Economics and Planning of Energy Systems
\end{flushleft}


\begin{flushleft}
3 Credits (3-0-0)
\end{flushleft}


\begin{flushleft}
Relevance of financial and economic feasibility evaluation of energy
\end{flushleft}


\begin{flushleft}
technologies and systems, Basics of engineering economics, Financial
\end{flushleft}


\begin{flushleft}
evaluation of energy technologies, Social cost benefit analysis, Case
\end{flushleft}


\begin{flushleft}
studies on techno-economics of energy conservation and renewable
\end{flushleft}


\begin{flushleft}
energy technologies.
\end{flushleft}


\begin{flushleft}
Energy demand analysis and forecasting, Energy supply assessment
\end{flushleft}


\begin{flushleft}
and evaluation, Energy demand -- supply balancing, Energy models.
\end{flushleft}


\begin{flushleft}
Energy -- economy interaction, Energy investment planning and project
\end{flushleft}


\begin{flushleft}
formulation. Energy pricing. Policy and planning implications of energy
\end{flushleft}


\begin{flushleft}
-- environment interaction, Clean development mechanism. Financing
\end{flushleft}


\begin{flushleft}
of energy systems. Energy policy related acts and regulations. Software
\end{flushleft}


\begin{flushleft}
for energy planning.
\end{flushleft}





\begin{flushleft}
ESL755 Solar Photovoltaic Devices and Systems
\end{flushleft}


\begin{flushleft}
3 Credits (3-0-0)
\end{flushleft}


\begin{flushleft}
Photovoltaic materials bulk and thin film forms. The role of
\end{flushleft}


\begin{flushleft}
microstructure (single crystal, multi crystalline, polycrystalline,
\end{flushleft}


\begin{flushleft}
amorphous and nano-crystalline) in electrical and optical properties
\end{flushleft}


\begin{flushleft}
of the materials. Different cell design and the technology route for
\end{flushleft}


\begin{flushleft}
making solar cells. Different methods of characterization of materials
\end{flushleft}


\begin{flushleft}
and devices. Applications of Photovoltaic for power generation from
\end{flushleft}


\begin{flushleft}
few watts to Megawatts.
\end{flushleft}





\begin{flushleft}
ESL760 Heat Transfer
\end{flushleft}


\begin{flushleft}
3 Credits (3-0-0)
\end{flushleft}


\begin{flushleft}
General heat conduction equation with heat generation, Analysis
\end{flushleft}


\begin{flushleft}
of extended surfaces, transient (and periodic) heat conduction,
\end{flushleft}


\begin{flushleft}
Two dimensional heat conduction problems and solutions, Theory
\end{flushleft}


\begin{flushleft}
of convective heat transfer, Boundary layer theory, Heat transfer in
\end{flushleft}


\begin{flushleft}
duct flows laminar and turbulent, Boiling, condensation and heat
\end{flushleft}


\begin{flushleft}
exchangers, Laws of thermal radiation, Radiation heat transfer
\end{flushleft}


\begin{flushleft}
between black and grey bodies, Numerical solutions of radiation
\end{flushleft}


\begin{flushleft}
network analysis, Thermal circuit analysis and correlations for various
\end{flushleft}


\begin{flushleft}
heat transfer coefficients, Overall heat transfer.
\end{flushleft}





\begin{flushleft}
ESL768 Wind Energy and Hydro Power Systems
\end{flushleft}


\begin{flushleft}
3 Credits (3-0-0)
\end{flushleft}


\begin{flushleft}
Introduction, General theories of wind machines, Basic laws and
\end{flushleft}


\begin{flushleft}
concepts of aerodynamics, Micro-siting, Description and performance
\end{flushleft}


\begin{flushleft}
of the horizontal--axis wind machines, Blade design, Description and
\end{flushleft}


\begin{flushleft}
performance of the vertical--axis wind machines, The generation of
\end{flushleft}


\begin{flushleft}
electricity by wind machines, case studies, Overview of micro mini and
\end{flushleft}


\begin{flushleft}
small hydro, Site selection and civil works, Penstocks and turbines,
\end{flushleft}


\begin{flushleft}
Speed and voltage regulation, Investment issues, load management
\end{flushleft}


\begin{flushleft}
and tariff collection, Distribution and marketing issues, case studies,
\end{flushleft}


\begin{flushleft}
Wind and hydro based stand-alone / hybrid power systems, Control
\end{flushleft}


\begin{flushleft}
of hybrid power systems, Wind diesel hybrid systems.
\end{flushleft}





\begin{flushleft}
ESL770 Solar Energy Utilization
\end{flushleft}


\begin{flushleft}
3 Credits (3-0-0)
\end{flushleft}





\begin{flushleft}
Overview of measures and approaches towards improved energy
\end{flushleft}


\begin{flushleft}
efficiency and energy conservation in various sectors of the economy;
\end{flushleft}


\begin{flushleft}
Need for studying economics of energy conservation; Identification
\end{flushleft}


\begin{flushleft}
and quantification of costs and benefits associated with energy
\end{flushleft}


\begin{flushleft}
conservation projects; Time value of money, discount rate and basic
\end{flushleft}


\begin{flushleft}
formulae of engineering economics; Measures of financial/economic
\end{flushleft}


\begin{flushleft}
performance for appraisal/evaluation of energy conservation projects;
\end{flushleft}





\begin{flushleft}
Solar radiation and modeling, solar collectors and types: flat
\end{flushleft}


\begin{flushleft}
plate, concentrating solar collectors, advanced collectors and
\end{flushleft}


\begin{flushleft}
solar concentrators, Selective coatings, Solar water heating, Solar
\end{flushleft}


\begin{flushleft}
cooking, Solar drying, Solar distillation and solar refrigeration, Active
\end{flushleft}


\begin{flushleft}
and passive heating and cooling of buildings, Solar thermal power
\end{flushleft}


\begin{flushleft}
generation, Solar cells, Home lighting systems, Solar lanterns, Solar PV
\end{flushleft}


\begin{flushleft}
pumps, Solar energy storage options, Industrial process heat systems,
\end{flushleft}


\begin{flushleft}
Solar thermal power generation and sterling engine, Solar economics.
\end{flushleft}





307





\begin{flushleft}
\newpage
Energy Studies
\end{flushleft}





\begin{flushleft}
ESL776 Industrial Energy and Environmental Analysis
\end{flushleft}


\begin{flushleft}
3 Credits (3-0-0)
\end{flushleft}


\begin{flushleft}
Pre-requisites: EC 75 (for UG Students in Minor Area)
\end{flushleft}


\begin{flushleft}
Energy and the environment, The greenhouse effect, Global
\end{flushleft}


\begin{flushleft}
energy and environmental management, Energy management
\end{flushleft}


\begin{flushleft}
and conservation, Energy in manufacture, Energy technologies,
\end{flushleft}


\begin{flushleft}
Instrumentation measurement and control, Energy management
\end{flushleft}


\begin{flushleft}
information systems, Hazardous waste management, Contamination
\end{flushleft}


\begin{flushleft}
of ground water, Treatment \& disposal, Pollution from combustion and
\end{flushleft}


\begin{flushleft}
atmospheric pollution control methods.
\end{flushleft}





\begin{flushleft}
ESL784 Cogeneration and Energy Efficiency
\end{flushleft}


\begin{flushleft}
3 Credits (3-0-0)
\end{flushleft}


\begin{flushleft}
Pre-requisites: EC 75 (for UG Students in Minor Area)
\end{flushleft}


\begin{flushleft}
The cogeneration concept, Main design parameters for cogeneration,
\end{flushleft}


\begin{flushleft}
Cogeneration alternatives, Bottoming and Topping cycles, Steam
\end{flushleft}


\begin{flushleft}
turbine plants, Gas turbine plants, Diesel and gas engine plants,
\end{flushleft}


\begin{flushleft}
Thermodynamic evaluation, Combined cycle applications, Sterling
\end{flushleft}


\begin{flushleft}
engine, Industry/Utility cogeneration, Trigeneration, Techno economic
\end{flushleft}


\begin{flushleft}
and environmental aspects, Cogeneration in sugar, textile, paper and
\end{flushleft}


\begin{flushleft}
steel industry, Case studies.
\end{flushleft}





\begin{flushleft}
ESL785 Energy Analysis
\end{flushleft}


\begin{flushleft}
3 Credits (3-0-0)
\end{flushleft}


\begin{flushleft}
Pre-requisites: EC 75 (for UG Students in Minor Area)
\end{flushleft}


\begin{flushleft}
Energy theory of value: Principles and systems of energy flows,
\end{flushleft}


\begin{flushleft}
Methods of energy analysis, Energy intensity method, Process analysis
\end{flushleft}


\begin{flushleft}
input-output method based energy accounting, Energy cost of goods
\end{flushleft}


\begin{flushleft}
and services energy to produce fuels: Coal, Oil, Natural Gas, Energy
\end{flushleft}


\begin{flushleft}
to produce electricity, Energy cost of various modes of passenger \&
\end{flushleft}


\begin{flushleft}
freight transportation, Industrial energy analysis: Aluminium, Steel,
\end{flushleft}


\begin{flushleft}
Cement, Fertilizers, Energetics of materials recycling, Energetics of
\end{flushleft}


\begin{flushleft}
renewable energy utilization (case studies), General energy equation,
\end{flushleft}


\begin{flushleft}
Energy loss, Reversibility \& irreversibility, Pictorial representation of
\end{flushleft}


\begin{flushleft}
energy, Energy analysis of simple processes, Expansion, Compression,
\end{flushleft}


\begin{flushleft}
Mixing and separation, Heat transfer, Combustion, Energy analysis of
\end{flushleft}


\begin{flushleft}
thermal and chemical plants, Thermo economic applications of energy
\end{flushleft}


\begin{flushleft}
analysis and national energy balance.
\end{flushleft}





\begin{flushleft}
ESL786 Exergy Analysis
\end{flushleft}


\begin{flushleft}
3 Credits (3-0-0)
\end{flushleft}


\begin{flushleft}
Pre-requisites: EC 75 (for UG Students in Minor Area)
\end{flushleft}


\begin{flushleft}
Thermodynamic basis of available energy, exergy and entropy, Exergy
\end{flushleft}


\begin{flushleft}
balance equations for closed and open flow systems under steady
\end{flushleft}


\begin{flushleft}
state and unsteady state conditions, Exergetic efficiency definition
\end{flushleft}


\begin{flushleft}
for various devices, components including heat exchangers, mixing
\end{flushleft}


\begin{flushleft}
chamber and drying process, Exergy analysis of thermal energy
\end{flushleft}


\begin{flushleft}
systems including thermal power plants, refrigeration and heat pump/
\end{flushleft}


\begin{flushleft}
air-conditioning plants, Exergy analysis of solar energy systems,
\end{flushleft}


\begin{flushleft}
solar cooker/ dryer/ collector/ concentrator/ solar still/ solar pond/
\end{flushleft}


\begin{flushleft}
thermal storage systems and solar thermal power generation, solar
\end{flushleft}


\begin{flushleft}
photo voltaic system, Economics based on exergy analysis of thermal
\end{flushleft}


\begin{flushleft}
energy systems.
\end{flushleft}





\begin{flushleft}
ESL796 Operation and Control of Electrical Energy
\end{flushleft}


\begin{flushleft}
Systems
\end{flushleft}


\begin{flushleft}
3 Credits (3-0-0)
\end{flushleft}


\begin{flushleft}
Real Time Monitoring of Power Systems : State Estimation, Topological
\end{flushleft}


\begin{flushleft}
observability Analysis, Security Analysis of Power Systems, Economic
\end{flushleft}


\begin{flushleft}
Dispatch \& Unit Commitment.
\end{flushleft}


\begin{flushleft}
Control of Power \& Frequency : Turbine-Governor Control Loops,
\end{flushleft}


\begin{flushleft}
Single Area and Multi-Area Systems Control, Effect of high penetration
\end{flushleft}


\begin{flushleft}
of Wind \& Other Renewable/Distributed Generation on P-F Control.
\end{flushleft}


\begin{flushleft}
Control of Voltage \& Reactive Power : Generator Excitation Systems, \&
\end{flushleft}


\begin{flushleft}
Automatic Voltage Regulators, Transformer Tap Changes Controls, Voltage
\end{flushleft}


\begin{flushleft}
Control in Distribution Networks using New Power Electronic Devices.
\end{flushleft}


\begin{flushleft}
Introduction to Market operations in Electric Power Systems:
\end{flushleft}


\begin{flushleft}
Restructured Power Systems, Short Term Load Forecasting, Power
\end{flushleft}





\begin{flushleft}
Trading through Bilateral, Multilateral Contracts and Power Exchanges,
\end{flushleft}


\begin{flushleft}
Role of Distributed Generators in market Operations.
\end{flushleft}





\begin{flushleft}
JSD799 Minor Project (JES)
\end{flushleft}


\begin{flushleft}
3 Credits (3-0-0)
\end{flushleft}


\begin{flushleft}
JSD801 Major Project Part -- 1 (JES)
\end{flushleft}


\begin{flushleft}
6 Credits (0-0-12)
\end{flushleft}


\begin{flushleft}
JSD802 Major Project Part -- 2 (JES)
\end{flushleft}


\begin{flushleft}
12 Credits (0-0-24)
\end{flushleft}


\begin{flushleft}
JSS801 Independent Study (JES)
\end{flushleft}


\begin{flushleft}
3 Credits (0-3-0)
\end{flushleft}


\begin{flushleft}
ESL810 MHD Power Generation
\end{flushleft}


\begin{flushleft}
3 Credits (3-0-0)
\end{flushleft}


\begin{flushleft}
Principle of MHD power generation, Properties of working fluids,
\end{flushleft}


\begin{flushleft}
MHD equation and types of MHD duct, Losses in MHD generators,
\end{flushleft}


\begin{flushleft}
Diagnostics of parameters, MHD cycles, MHD components (air heater,
\end{flushleft}


\begin{flushleft}
combustion chamber, heat exchanger, diffuser, insulating materials
\end{flushleft}


\begin{flushleft}
and electrode walls, magnetic field etc.), Economics and applications
\end{flushleft}


\begin{flushleft}
of MHD, Liquid metal MHD generators.
\end{flushleft}





\begin{flushleft}
ESL840 Solar Architecture
\end{flushleft}


\begin{flushleft}
3 Credits (3-0-0)
\end{flushleft}


\begin{flushleft}
Thermal comfort, sun motion, Building orientation and design, passive
\end{flushleft}


\begin{flushleft}
heating and cooling concepts, thumb rules, heat transfer in buildings:
\end{flushleft}


\begin{flushleft}
Thermal modeling of passive concepts, Evaporative cooling, Energy
\end{flushleft}


\begin{flushleft}
efficient windows and day lighting, Earth air tunnel and heat exchanger,
\end{flushleft}


\begin{flushleft}
Zero energy building concept and rating systems, Energy conservation
\end{flushleft}


\begin{flushleft}
building codes, Software for Building Simulation, Automation and
\end{flushleft}


\begin{flushleft}
Energy Management of Buildings.
\end{flushleft}





\begin{flushleft}
ESL850 Solar Refrigeration and Air Conditioning
\end{flushleft}


\begin{flushleft}
3 Credits (3-0-0)
\end{flushleft}


\begin{flushleft}
Potential and scope of solar cooling, Types of solar cooling systems,
\end{flushleft}


\begin{flushleft}
Solar collectors and storage systems for solar refrigeration and
\end{flushleft}


\begin{flushleft}
air-conditioning, Solar operation of vapour absorption and vapour
\end{flushleft}


\begin{flushleft}
compression refrigeration cycles and their thermodynamic assessment,
\end{flushleft}


\begin{flushleft}
Rankine cycle, sterling cycle based solar cooling systems, Jet ejector
\end{flushleft}


\begin{flushleft}
solar cooling systems, Fuel assisted solar cooling systems, Solar
\end{flushleft}


\begin{flushleft}
desiccant cooling systems, Open cycle absorption / desorption solar
\end{flushleft}


\begin{flushleft}
cooling alternatives, Advanced solar cooling systems, Thermal
\end{flushleft}


\begin{flushleft}
modeling and computer simulation for continuous and intermittent
\end{flushleft}


\begin{flushleft}
solar refrigeration and air-conditioning systems, Refrigerant storage
\end{flushleft}


\begin{flushleft}
for solar absorption cooling systems, Solar thermoelectric refrigeration
\end{flushleft}


\begin{flushleft}
and air-conditioning, Solar thermo acoustic cooling and hybrid airconditioning, Solar economics of cooling systems.
\end{flushleft}





\begin{flushleft}
ESL860 Electrical Power Systems Analysis
\end{flushleft}


\begin{flushleft}
3 Credits (3-0-0)
\end{flushleft}


\begin{flushleft}
Network modeling and short circuit analysis: Primitive network, Y
\end{flushleft}


\begin{flushleft}
bus and Z bus matrices formulation, Power invariant transformations,
\end{flushleft}


\begin{flushleft}
Mutually coupled branches Z bus, Fault calculations using Z bus,
\end{flushleft}


\begin{flushleft}
Power flow solutions: AC load flow formulations, Gauss-siedel
\end{flushleft}


\begin{flushleft}
method, Newton Raphson method, Decoupled power flow method,
\end{flushleft}


\begin{flushleft}
Security analysis: Z bus methods in contingency analysis, Adding and
\end{flushleft}


\begin{flushleft}
removing multiple lines, Interconnected systems, Single contingency
\end{flushleft}


\begin{flushleft}
and multiple contingencies, Analysis by DC model, System reduction
\end{flushleft}


\begin{flushleft}
for contingency studies, State Estimation: Lone power flow state
\end{flushleft}


\begin{flushleft}
estimator, Method of least squares, Statistics error and estimates,
\end{flushleft}


\begin{flushleft}
Test for bad data, Monitoring the power system, Determination
\end{flushleft}


\begin{flushleft}
of variance, Improving state estimates by adding measurements,
\end{flushleft}


\begin{flushleft}
Hierarchical state estimation, Dynamic state estimation, Power system
\end{flushleft}


\begin{flushleft}
stability: transient and dynamic stability, Swing equation, Electric
\end{flushleft}


\begin{flushleft}
power relations, Concepts in transient stability, Method for stability
\end{flushleft}


\begin{flushleft}
assessment, Improving system stability.
\end{flushleft}





308





\begin{flushleft}
\newpage
Energy Studies
\end{flushleft}





\begin{flushleft}
ESL870 Fusion Energy
\end{flushleft}


\begin{flushleft}
3 Credits (3-0-0)
\end{flushleft}


\begin{flushleft}
Fission and fusion, Need for plasma, Lawson criterion, Confinement
\end{flushleft}


\begin{flushleft}
problem, Laser driven fusion, Magnetic confinement, Plasma concept,
\end{flushleft}


\begin{flushleft}
Single particle motions in complex magnetic field geometries, Equilibrium
\end{flushleft}


\begin{flushleft}
and stability, Cross field transport, Important heating schemes,
\end{flushleft}


\begin{flushleft}
Tokamak and magnetic mirror, Reactor concepts, Current status.
\end{flushleft}





\begin{flushleft}
ESL871 Advanced Fusion Energy
\end{flushleft}


\begin{flushleft}
3 Credits (3-0-0)
\end{flushleft}


\begin{flushleft}
Tokamak confinement Physics, Particle motions in a tokamak, Toroidal
\end{flushleft}


\begin{flushleft}
equilibrium, Toroidal stability, High-beta Tokamak, Experimental
\end{flushleft}


\begin{flushleft}
observations, Fusion Technology, Commercial Tokamak Fusion-power
\end{flushleft}


\begin{flushleft}
plant, Tandem-mirror fusion power plant, Other Fusion reactors
\end{flushleft}


\begin{flushleft}
concepts, Inertial confinement fusion reactors, Reactor cavity, Hybrid
\end{flushleft}


\begin{flushleft}
fusion/fission systems, Process heat and synthetic fuel production.
\end{flushleft}





\begin{flushleft}
ESL875 Alternative Fuels for Transportation
\end{flushleft}


\begin{flushleft}
3 Credits (3-0-0)
\end{flushleft}


\begin{flushleft}
Pre-requisites: EC 100 (for UG Students in Minor Area)
\end{flushleft}


\begin{flushleft}
An introduction to hydrocarbon fuels-their availability and effect
\end{flushleft}


\begin{flushleft}
on environment, Gasoline and diesel self-ignition characteristics of
\end{flushleft}


\begin{flushleft}
the fuel, octane number , cetane number , Alternative fuels -- liquid
\end{flushleft}


\begin{flushleft}
and gaseous fuels, physico-chemical characteristics, Alternative
\end{flushleft}


\begin{flushleft}
liquid fuels, Alcohol fuels -- ethanol and methanol, fuel composition,
\end{flushleft}





\begin{flushleft}
Fuel induction techniques, Fumigation, Emission of oxygenates,
\end{flushleft}


\begin{flushleft}
Applications to engines and automotive conversions, Biodiesel
\end{flushleft}


\begin{flushleft}
formulation techniques, Trans esterification, Application in diesel
\end{flushleft}


\begin{flushleft}
engines, DME (Dimethyl ether), properties fuel injection consideration
\end{flushleft}


\begin{flushleft}
general introduction to LPG and LNG, Compressed natural gas
\end{flushleft}


\begin{flushleft}
components, mixtures and kits, Fuel supply system and emission
\end{flushleft}


\begin{flushleft}
studies and control, Hydrogen combustion characteristics, Flashback
\end{flushleft}


\begin{flushleft}
control techniques, Safety aspects and system development, NOx
\end{flushleft}


\begin{flushleft}
emission control, Biogas, Producer gas and their characteristics
\end{flushleft}


\begin{flushleft}
system development for engine application.
\end{flushleft}





\begin{flushleft}
ESL880 Solar Thermal Power Generation
\end{flushleft}


\begin{flushleft}
3 Credits (3-0-0)
\end{flushleft}


\begin{flushleft}
Relevance of solar thermal power generation; Design and performance
\end{flushleft}


\begin{flushleft}
characteristics of different solar concentrator types suitable for thermal
\end{flushleft}


\begin{flushleft}
power generation; Tracking of solar concentrators; performance
\end{flushleft}


\begin{flushleft}
characterization of solar concentrators, Storage option for solar thermal
\end{flushleft}


\begin{flushleft}
power plants; Modes of power generation in solar thermal power
\end{flushleft}


\begin{flushleft}
plants; Sizing solar thermal power plants; Operation and maintenance
\end{flushleft}


\begin{flushleft}
issues; Emerging trends in solar thermal power generation; Economics
\end{flushleft}


\begin{flushleft}
of solar thermal power generation; Case studies.
\end{flushleft}





\begin{flushleft}
ESP713 Energy Laboratories
\end{flushleft}


\begin{flushleft}
3 Credits (0-0-6)
\end{flushleft}


\begin{flushleft}
Pre-requisites: EC 75 (for UG Students in Minor Area)
\end{flushleft}





309





\begin{flushleft}
\newpage
Industrial Tribology, Machine Dynamics and Maintenance
\end{flushleft}


\begin{flushleft}
Engineering Centre
\end{flushleft}


\begin{flushleft}
ITL702 Diagnostic Maintenance and Condition Monitoring
\end{flushleft}


\begin{flushleft}
4 Credits (3-0-2)
\end{flushleft}


\begin{flushleft}
Maintenance strategies and introduction to Condition Based
\end{flushleft}


\begin{flushleft}
Maintenance (CBM), Application and economic benefits, Signature
\end{flushleft}


\begin{flushleft}
analysis - online and off-line techniques, Various Condition Monitoring
\end{flushleft}


\begin{flushleft}
(CM) techniques - Vibration monitoring and analysis, Shock Pulse
\end{flushleft}


\begin{flushleft}
Method, Noise monitoring, Envelope detection technique, Cepstrum
\end{flushleft}


\begin{flushleft}
analysis, Oil analysis including wear debris and contaminant
\end{flushleft}


\begin{flushleft}
monitoring, Performance monitoring, Acoustic emission and other
\end{flushleft}


\begin{flushleft}
techniques, Non-destructive testing techniques, Temperature
\end{flushleft}


\begin{flushleft}
monitoring including Thermography, Application and choice of the
\end{flushleft}


\begin{flushleft}
method, Practical applications of diagnostic maintenance, Condition
\end{flushleft}


\begin{flushleft}
monitoring of mechanical and electrical machines, Case studies.
\end{flushleft}





\begin{flushleft}
ITL703 Fundamentals of Tribology
\end{flushleft}


\begin{flushleft}
4 Credits (3-0-2)
\end{flushleft}


\begin{flushleft}
Introduction to tribology and its historical background. Factors
\end{flushleft}


\begin{flushleft}
influencing Tribological phenomena. Engineering surfaces -- Surface
\end{flushleft}


\begin{flushleft}
characterization, Computation of surface parameters. Surface
\end{flushleft}


\begin{flushleft}
measurement techniques. Apparent and real area of contact.
\end{flushleft}


\begin{flushleft}
Contact of engineering surfaces- Hertzian and non-hertzian contact.
\end{flushleft}


\begin{flushleft}
Contact pressure and deformation in non-conformal contacts. Genesis
\end{flushleft}


\begin{flushleft}
of friction, friction in contacting rough surfaces, sliding and rolling
\end{flushleft}


\begin{flushleft}
friction, Various laws and theory of friction. Stick-slip friction behaviour,
\end{flushleft}


\begin{flushleft}
frictional heating and temperature rise. Friction measurement
\end{flushleft}


\begin{flushleft}
techniques. Wear and wear types. Mechanisms of wear - Adhesive,
\end{flushleft}


\begin{flushleft}
abrasive, corrosive, erosion, fatigue, fretting, etc., Wear of metals and
\end{flushleft}


\begin{flushleft}
non-metals. Wear models - asperity contact, constant and variable
\end{flushleft}


\begin{flushleft}
wear rate, geometrical influence in wear models, wear damage. Wear
\end{flushleft}


\begin{flushleft}
in various mechanical components,
\end{flushleft}


\begin{flushleft}
wear controlling techniques. Introduction to lubrication. Lubrication
\end{flushleft}


\begin{flushleft}
regimes. Introduction to micro and nano-tribology.
\end{flushleft}





\begin{flushleft}
ITL705 Materials for Tribological Applications
\end{flushleft}


\begin{flushleft}
3 Credits (3-0-0)
\end{flushleft}


\begin{flushleft}
Introduction to tribological processes and tribological relevant
\end{flushleft}


\begin{flushleft}
properties of materials. An overview of engineering materials having
\end{flushleft}


\begin{flushleft}
potential for tribological application.
\end{flushleft}


\begin{flushleft}
Characterization and evaluation of Ferrous materials for tribological
\end{flushleft}


\begin{flushleft}
requirements/applications, Selection of ferrous materials for rolling
\end{flushleft}


\begin{flushleft}
element bearings, gears, crank shafts, piston rings, cylinder liners, etc.
\end{flushleft}


\begin{flushleft}
Non-ferrous materials and their applications such as sliding bearings,
\end{flushleft}


\begin{flushleft}
piston rings, cylinder liners, etc., materials for dry friction materials.
\end{flushleft}


\begin{flushleft}
Composite materials (PM, CMC and MMC) for tribological applications.
\end{flushleft}


\begin{flushleft}
Surface treatment techniques with applications such as carburizing,
\end{flushleft}


\begin{flushleft}
nitriding, induction hardening, hard facing, laser surface treatments, etc.
\end{flushleft}


\begin{flushleft}
Surface coating techniques such as electrochemical depositions,
\end{flushleft}


\begin{flushleft}
anodizing, thermal spraying, Chemical Vapour Deposition (CVD),
\end{flushleft}


\begin{flushleft}
Physical Vapour Deposition (PVD), etc. and their applications.
\end{flushleft}


\begin{flushleft}
Lubricants- Introduction, requirements, types, Evaluation and testing
\end{flushleft}


\begin{flushleft}
of lubricants.
\end{flushleft}





\begin{flushleft}
ITL709 Maintenance Planning and Control
\end{flushleft}


\begin{flushleft}
3 Credits (3-0-0)
\end{flushleft}


\begin{flushleft}
Objectives of planned maintenance, Maintenance philosophies,
\end{flushleft}


\begin{flushleft}
Preventive and Predictive maintenance, Emerging trends in
\end{flushleft}


\begin{flushleft}
maintenance-Proactive Maintenance, Reliability Centred Maintenance
\end{flushleft}


\begin{flushleft}
(RCM), Total Productive Maintenance (TPM), etc, Implementation of
\end{flushleft}


\begin{flushleft}
Maintenance strategy, Maintenance organization, Basis of planned
\end{flushleft}


\begin{flushleft}
maintenance system, Maintenance planning and scheduling,
\end{flushleft}


\begin{flushleft}
Maintenance control system and documentation. Spares and inventory
\end{flushleft}


\begin{flushleft}
planning, Manpower planning, maintenance auditing. Human factors
\end{flushleft}


\begin{flushleft}
in maintenance and training, maintenance costing, Maintenance
\end{flushleft}


\begin{flushleft}
performance. Repair decisions- Repair, replacement and overhaul,
\end{flushleft}





\begin{flushleft}
Computer applications in maintenance, Expert systems applications,
\end{flushleft}


\begin{flushleft}
maintenance effectiveness, Case studies.
\end{flushleft}





\begin{flushleft}
ITL710 Design of Tribological Elements
\end{flushleft}


\begin{flushleft}
3 Credits (3-0-0)
\end{flushleft}


\begin{flushleft}
Introduction-Tribological consideration in design, Conceptual design,
\end{flushleft}


\begin{flushleft}
Classification of tribological components, Mechanisms of tribological
\end{flushleft}


\begin{flushleft}
failures in machines, Zero wear concept, Computational techniques
\end{flushleft}


\begin{flushleft}
in design.
\end{flushleft}


\begin{flushleft}
Design of Dry Frictional Elements-Dry friction concepts, Brakes and
\end{flushleft}


\begin{flushleft}
Clutches, Friction belts and Dry rubbing bearing.
\end{flushleft}


\begin{flushleft}
Design of Fluid Frictional Elements- Fluid friction concepts, Design
\end{flushleft}


\begin{flushleft}
of hydrodynamically loaded journal bearings, externally pressurized
\end{flushleft}


\begin{flushleft}
bearings, Oscillating journal bearings, Externally pressurized bearings,
\end{flushleft}


\begin{flushleft}
Design of oil groove, Design of elliptical, multi-lobe and titled pad bearings.
\end{flushleft}


\begin{flushleft}
Rolling elements bearings, Performance analysis of bearings, gears, seals,
\end{flushleft}


\begin{flushleft}
piston rings, machine tool slide ways, cams and follower and wire rope.
\end{flushleft}





\begin{flushleft}
ITL711 Reliability, Availability and Maintainability
\end{flushleft}


\begin{flushleft}
(RAM) Engineering
\end{flushleft}


\begin{flushleft}
3 Credits (3-0-0)
\end{flushleft}


\begin{flushleft}
System concepts in RAM Engineering, Fundamentals of reliability,
\end{flushleft}


\begin{flushleft}
Failure distributions, Statistical analysis of failure data, Weibull analysis,
\end{flushleft}


\begin{flushleft}
Monte Carlo simulation, System reliability assessment. Reliability of
\end{flushleft}


\begin{flushleft}
repairable and non-repairable systems. Point, mission and steady state
\end{flushleft}


\begin{flushleft}
availability. Availability assessment. Maintainability and its assessment.
\end{flushleft}


\begin{flushleft}
Design for reliability and maintainability', Practical applications of RAM
\end{flushleft}


\begin{flushleft}
Engineering to systems, products and processes.
\end{flushleft}





\begin{flushleft}
ITL714 Failure Analysis and Repair
\end{flushleft}


\begin{flushleft}
4 Credits (3-0-2)
\end{flushleft}


\begin{flushleft}
Introduction, need for failure analysis, Classification of failures,
\end{flushleft}


\begin{flushleft}
Fundamental causes of failures, influence of type of loading (e.g.
\end{flushleft}


\begin{flushleft}
static, fatigue, shock, etc.) on nature of failures, Role of stress;
\end{flushleft}


\begin{flushleft}
processing and fabrication defects, Effect of residual stresses
\end{flushleft}


\begin{flushleft}
induced during fabrication processes, Influence of temperature and
\end{flushleft}


\begin{flushleft}
environment on failure, Crack and subsurface crack like defects and
\end{flushleft}


\begin{flushleft}
their significance in failure.
\end{flushleft}


\begin{flushleft}
Micro mechanisms of failures; Ductile and brittle fracture, Fracture
\end{flushleft}


\begin{flushleft}
initiation and propagation, Fatigue failures, Wear related failures,
\end{flushleft}


\begin{flushleft}
High temperature failures, low temperature failures, etc., Studies and
\end{flushleft}


\begin{flushleft}
analysis of failed surfaces.
\end{flushleft}


\begin{flushleft}
Identification of failures, Techniques of failure analysis, Microscopic
\end{flushleft}


\begin{flushleft}
methods, Fracture mechanics techniques, Prediction of failures,
\end{flushleft}


\begin{flushleft}
Residual life assessment and life extension, Typical case studies in
\end{flushleft}


\begin{flushleft}
failure analysis, Logical fault finding and its application, Inspection
\end{flushleft}


\begin{flushleft}
and safety measures, Repair techniques and economic considerations,
\end{flushleft}


\begin{flushleft}
Failure analysis for design improvement and proactive maintenance,
\end{flushleft}


\begin{flushleft}
Design for repairbility, Case Studies.
\end{flushleft}





\begin{flushleft}
ITL717 Corrosion and its Control
\end{flushleft}


\begin{flushleft}
3 Credits (3-0-0)
\end{flushleft}


\begin{flushleft}
Importance of corrosion control in industrial practices, Thermodynamics
\end{flushleft}


\begin{flushleft}
of corrosion, Broad forms of corrosion -- uniform, uneven, pitting,
\end{flushleft}


\begin{flushleft}
cracking, etc. influencing factors on corrosion. Surface film, Polarisation
\end{flushleft}


\begin{flushleft}
and effect, Theory of passivity, kinetics of corrosion, Various types
\end{flushleft}


\begin{flushleft}
of corrosion along with case studies -- Galvanic, Thermogalvanic,
\end{flushleft}


\begin{flushleft}
High temperature corrosion, Intergranular, Pitting, Selective attack
\end{flushleft}


\begin{flushleft}
(leaching), fretting corrosion -- erosion, cavitation, Stress corrosion
\end{flushleft}


\begin{flushleft}
cracking, hydrogen embrittlement, etc., Various techniques for
\end{flushleft}


\begin{flushleft}
corrosion evaluation and monitoring, Corrosion Control-Design
\end{flushleft}


\begin{flushleft}
improvement, Selection of material, fabrication process for corrosion
\end{flushleft}


\begin{flushleft}
control, Role of residual stress, Changes in operating conditions, Use
\end{flushleft}


\begin{flushleft}
of inhibitors, Anodic and Cathodic protection, Corrosion resistant
\end{flushleft}


\begin{flushleft}
coatings, Case studies.
\end{flushleft}





310





\begin{flushleft}
\newpage
Industrial Tribology, Machine Dynamics and Maintenance Engineering Centre
\end{flushleft}





\begin{flushleft}
ITL730 Lubricants
\end{flushleft}


\begin{flushleft}
3 Credits (2-0-2)
\end{flushleft}


\begin{flushleft}
Overview of friction ( F), wear (W) and lubrication, Primary role
\end{flushleft}


\begin{flushleft}
of lubricants in mitigation of F \& W \& heat transfer medium,
\end{flushleft}


\begin{flushleft}
Composition and properties of lubricant; Types of lubricants such as
\end{flushleft}


\begin{flushleft}
mineral oil based, synthetic lubricants, solid lubricants, and greases;
\end{flushleft}


\begin{flushleft}
Characteristics properties of lubes \& greases; their evaluation methods,
\end{flushleft}


\begin{flushleft}
Classification systems such as API, SAE, AGMA, NLGI, ISO; Additives
\end{flushleft}


\begin{flushleft}
such as Viscosity-index improver (VII); Anti-oxidant (AO); Anti-friction
\end{flushleft}


\begin{flushleft}
(AF) Antiwear (AW) Extreme-pressure (EP); Corrosion inhibitors (CI),
\end{flushleft}


\begin{flushleft}
detergents, dispersants; Selection criteria for lubricants for various
\end{flushleft}


\begin{flushleft}
tribological situations and applications; Used lubes-environment
\end{flushleft}


\begin{flushleft}
\& health hazards and disposibility and recycling, evaluation of oil
\end{flushleft}


\begin{flushleft}
degradation by various techniques.
\end{flushleft}





\begin{flushleft}
ITL740 Risk Analysis and Safety
\end{flushleft}


\begin{flushleft}
3 Credits (2-1-0)
\end{flushleft}


\begin{flushleft}
Introduction, Typical hazards, Accident indices, Fire and explosion
\end{flushleft}


\begin{flushleft}
hazards, Dow's fire and explosion index, Hazards identification
\end{flushleft}


\begin{flushleft}
procedures for plants and machinery; Preliminary hazard analysis
\end{flushleft}


\begin{flushleft}
(PHA), Fault Hazard Analysis (FHA), Hazard and operability (HAZOP),
\end{flushleft}


\begin{flushleft}
What if, Check lists, Failure mode and effects analysis (FMEA), Failure
\end{flushleft}


\begin{flushleft}
mode, effects and criticality analysis (FMECA), HAZAN: Hazard
\end{flushleft}


\begin{flushleft}
analysis; FTA (Fault tree analysis), ETA (Event tree analysis), and CCA
\end{flushleft}


\begin{flushleft}
(Cause consequence analysis), Transportation of hazardous materials,
\end{flushleft}


\begin{flushleft}
Safety audit, Health and safety aspects of lubricants, Human factors in
\end{flushleft}


\begin{flushleft}
safety, Risk evaluation and acceptance criteria, Disaster management,
\end{flushleft}


\begin{flushleft}
Safety codes and Case studies.
\end{flushleft}





\begin{flushleft}
ITL752 Bulk Materials Handling
\end{flushleft}


\begin{flushleft}
3 Credits (2-0-2)
\end{flushleft}


\begin{flushleft}
Nature of bulk materials, Flow of gas-solids in pipelines, Mechanical
\end{flushleft}


\begin{flushleft}
Handling equipments like screw conveyors, belt conveyors and bucket
\end{flushleft}


\begin{flushleft}
elevators, Pneumatic conveying systems- Components, Design and
\end{flushleft}


\begin{flushleft}
Selection, Troubleshooting and Maintenance of pneumatic conveying
\end{flushleft}


\begin{flushleft}
systems, Performance evaluation of alternative systems, Bend erosioninfluencing factors, materials selection and potential solutions, Case
\end{flushleft}


\begin{flushleft}
studies, and Design exercises.
\end{flushleft}





\begin{flushleft}
ITL760 Noise Monitoring and Control
\end{flushleft}


\begin{flushleft}
3 Credits (2-0-2)
\end{flushleft}


\begin{flushleft}
Introduction to noise, Properties of noise, Loudness and weighting
\end{flushleft}


\begin{flushleft}
networks, Noise measurement parameters and standards, Impulse
\end{flushleft}


\begin{flushleft}
noise, Frequency analysis - octave, one third octave and FFT analysis,
\end{flushleft}


\begin{flushleft}
Instrumentation for noise measurement and analysis, Sound power,
\end{flushleft}


\begin{flushleft}
Sound intensity measurement technique with applications, Noise
\end{flushleft}


\begin{flushleft}
source location, Noise diagnostics, Noise monitoring of machines with
\end{flushleft}





\begin{flushleft}
examples, Estimation of machinery noise, Cepstrum analysis, Noise
\end{flushleft}


\begin{flushleft}
control methods, Maintenance and noise reduction, Road vehicle and
\end{flushleft}


\begin{flushleft}
aircraft noise sources and control, Case studies.
\end{flushleft}





\begin{flushleft}
ITL810 Bearing Lubrication
\end{flushleft}


\begin{flushleft}
3 Credits (3-0-0)
\end{flushleft}


\begin{flushleft}
Introduction-Historical background, Bearing concepts and typical
\end{flushleft}


\begin{flushleft}
applications. Viscous flow concepts-Conservation of laws and its
\end{flushleft}


\begin{flushleft}
derivations: continuity, momentum (N-S equations) and energy,
\end{flushleft}


\begin{flushleft}
Solutions of Navier-Strokes equations. Order of magnitude analysis,
\end{flushleft}


\begin{flushleft}
General Reynolds equation-2D and 3D (Cartesian and Cylindrical),
\end{flushleft}


\begin{flushleft}
Various mechanisms of pressure development in an oil film,
\end{flushleft}


\begin{flushleft}
Performance parameters.
\end{flushleft}


\begin{flushleft}
Boundary Layer Concepts-Laminar and turbulent flow in bearings,
\end{flushleft}


\begin{flushleft}
mathematical modeling of flow in high-speed bearings. Elastic
\end{flushleft}


\begin{flushleft}
Deformation of bearing surfaces-Contact of smooth and rough
\end{flushleft}


\begin{flushleft}
solid surfaces, elasticity equation, Stress distribution and local
\end{flushleft}


\begin{flushleft}
deformation in mating surfaces due to loadings, methods to avoid
\end{flushleft}


\begin{flushleft}
singularity effects, Estimation of elastic deformation by numerical
\end{flushleft}


\begin{flushleft}
methods-Finite Difference Method (FDM), Governing equation for
\end{flushleft}


\begin{flushleft}
evaluation of film thickness in Elastohydrodynamic Lubrication (EHL)
\end{flushleft}


\begin{flushleft}
and its solution, Boundary conditions. Development of computer
\end{flushleft}


\begin{flushleft}
programs for mathematical modeling of flow in bearings, Numerical
\end{flushleft}


\begin{flushleft}
simulation of elastic deformation in bearing surfaces by FDM.
\end{flushleft}





\begin{flushleft}
JID800 Minor Project
\end{flushleft}


\begin{flushleft}
3 Credits (0-3-0)
\end{flushleft}


\begin{flushleft}
The students will select a research topic for the minor project. It
\end{flushleft}


\begin{flushleft}
is expected that such topics would involve understanding of basic
\end{flushleft}


\begin{flushleft}
processes and applications.
\end{flushleft}





\begin{flushleft}
JIS800 Independent Study
\end{flushleft}


\begin{flushleft}
3 Credits (0-3-0)
\end{flushleft}


\begin{flushleft}
This is meant only for such students who are selected for DAAD
\end{flushleft}


\begin{flushleft}
fellowship.
\end{flushleft}





\begin{flushleft}
JID801 Major Project-Part-I
\end{flushleft}


\begin{flushleft}
6 Credits (0-0-12)
\end{flushleft}


\begin{flushleft}
The students will select a research topic for the major project. It
\end{flushleft}


\begin{flushleft}
is expected that such topics would involve understanding of basic
\end{flushleft}


\begin{flushleft}
processes and extensive experimentation.
\end{flushleft}





\begin{flushleft}
JID802 Major Project-Part-II
\end{flushleft}


\begin{flushleft}
12 Credits (0-0-24)
\end{flushleft}


\begin{flushleft}
The research topic selected in Part-I shall continue in Part-II also.
\end{flushleft}





311





\begin{flushleft}
\newpage
Instrument Design and Development Centre
\end{flushleft}


\begin{flushleft}
DSL601 Electronic Components and Circuits (for
\end{flushleft}


\begin{flushleft}
students other than Electrical/Electronics/Electronics
\end{flushleft}


\begin{flushleft}
and Communication)
\end{flushleft}


\begin{flushleft}
3 Credits (3-0-0)
\end{flushleft}


\begin{flushleft}
Review of Electronic Components: Passive Components, Active
\end{flushleft}


\begin{flushleft}
Components including components used in Industrial Environment.
\end{flushleft}





\begin{flushleft}
familiarity with various subsystems of instrumentation set up. The
\end{flushleft}


\begin{flushleft}
subsystems may consist of a detector-transducer, signal conditioner,
\end{flushleft}


\begin{flushleft}
a level power amplifier, display, actuator/final control element. The
\end{flushleft}


\begin{flushleft}
study will generally focus attention on one of the subsystems. In
\end{flushleft}


\begin{flushleft}
electronics conditioning.
\end{flushleft}


\begin{flushleft}
Specific Case Study Experiments as below:
\end{flushleft}





\begin{flushleft}
Electronic Circuits: Choppers, Clampers, analog circuits, precision
\end{flushleft}


\begin{flushleft}
and instrumentation amplifiers, signal conditioning circuits, industrial
\end{flushleft}


\begin{flushleft}
electronic circuits. Nonlinear devices and circuits, computing circuits
\end{flushleft}


\begin{flushleft}
and waveform generators.
\end{flushleft}





\begin{flushleft}
$\bullet$	 Experiments in Control involving speed, position, temperature
\end{flushleft}


\begin{flushleft}
controls using MATLAB
\end{flushleft}





\begin{flushleft}
Analog-Digital circuits: A/D and D/A converters, classification and
\end{flushleft}


\begin{flushleft}
characteristic parameters of DAC's and ADC's. Testing criteria.
\end{flushleft}


\begin{flushleft}
Multiplying DAC's.
\end{flushleft}





\begin{flushleft}
$\bullet$	 Data conversion, ADC \& DAC, synchronous detectors, multipliers,
\end{flushleft}


\begin{flushleft}
dividers, instrumentation amplifiers
\end{flushleft}





\begin{flushleft}
Digital Electronics: Logic gates, Combinational logic design, Sequential
\end{flushleft}


\begin{flushleft}
logic design, Counters;
\end{flushleft}


\begin{flushleft}
Memory Devices, SRAM, DRAM, ROM, EPROM, Flash Memories and
\end{flushleft}


\begin{flushleft}
Programmable Gate Arrays.
\end{flushleft}


\begin{flushleft}
Microprocessors: 8 bit and 16 bit microprocessor, basic structure and
\end{flushleft}


\begin{flushleft}
programming.
\end{flushleft}


\begin{flushleft}
Application of microprocessors in instruments. Introduction to microcontrollers and embedded systems.
\end{flushleft}





\begin{flushleft}
DSL603 Material and Mechanical Design (for
\end{flushleft}


\begin{flushleft}
students from Electrical/Electronics/Electronics and
\end{flushleft}


\begin{flushleft}
communication)
\end{flushleft}


\begin{flushleft}
3 Credits (3-0-0)
\end{flushleft}


\begin{flushleft}
Basics of Design: Stresses, strain, hardness, toughness, visco-elasticity,
\end{flushleft}


\begin{flushleft}
torision, bending, deflection of beams, combined stresses, energy
\end{flushleft}


\begin{flushleft}
methods. Material: metals and their alloys, heat treatment, polymers,
\end{flushleft}


\begin{flushleft}
composites, ceramics etc. Design of machine elements: Failure theories
\end{flushleft}


\begin{flushleft}
for static and alternating loadings. Design of shafts, fasteners, springs,
\end{flushleft}


\begin{flushleft}
curved beams, thick and thin vessels, gears etc; Lubrication, journal
\end{flushleft}


\begin{flushleft}
bearings and rolling contact bearings, limits, fits and tolerances.
\end{flushleft}


\begin{flushleft}
Deflection of thin plates. Design of mechanical elements for strain gage
\end{flushleft}


\begin{flushleft}
and other instrumentation applications. Introduction to vibrations and
\end{flushleft}


\begin{flushleft}
its isolation. Mechanical Fabrication techniques used in instruments.
\end{flushleft}


\begin{flushleft}
Basic mechanical fabrication processes. Design and drawing sessions.
\end{flushleft}





\begin{flushleft}
DSP703 Instrument Technology Laboratory 1
\end{flushleft}


\begin{flushleft}
3 Credits (0-0-6)
\end{flushleft}


\begin{flushleft}
The laboratory essentially supports the courses taught in the first
\end{flushleft}


\begin{flushleft}
semester courses. It consists of experiments on:Study of packaging and characterization of transducers used for
\end{flushleft}


\begin{flushleft}
measurement of different physical variables like displacement,
\end{flushleft}


\begin{flushleft}
temperature, pressure, strain, flow etc.; Study of practical signal
\end{flushleft}


\begin{flushleft}
conditioning techniques and electronic measurement methods;
\end{flushleft}


\begin{flushleft}
Study of Electronic subsystems used in instruments Experiments
\end{flushleft}


\begin{flushleft}
on Cardinal points measurements using Nodal slide method,
\end{flushleft}


\begin{flushleft}
Measurement of wedge angle of optical flat and right angle of a
\end{flushleft}


\begin{flushleft}
prism by Autocollimation, Measurement the long radius of curvature
\end{flushleft}


\begin{flushleft}
of concave mirror using Foucault Knife edge test and Ronchi test,
\end{flushleft}


\begin{flushleft}
Newton and Fizeau Interferometer for Testing of optical surface,
\end{flushleft}


\begin{flushleft}
Quantitative testing of optical elements using polarisation based
\end{flushleft}


\begin{flushleft}
Twyman-Green interferometer, Measurement of small radius of
\end{flushleft}


\begin{flushleft}
curvature of lens using {``}Optical Spherometer'', Moire interferometry
\end{flushleft}


\begin{flushleft}
for displacement measurement.
\end{flushleft}





\begin{flushleft}
DSP704 Instrument Technology Laboratory 2
\end{flushleft}


\begin{flushleft}
3 Credits (0-0-6)
\end{flushleft}


\begin{flushleft}
The laboratory supports the subjects taught in the second semester
\end{flushleft}


\begin{flushleft}
courses. The laboratory consists experiments on:
\end{flushleft}


\begin{flushleft}
Study of various techniques used for analog and digital conditioning
\end{flushleft}


\begin{flushleft}
of signals from various transducers/ detectors; Study on modulation/
\end{flushleft}


\begin{flushleft}
demodulation techniques, noise generation and measurement, Study
\end{flushleft}


\begin{flushleft}
of testing and calibration methods of instruments.
\end{flushleft}


\begin{flushleft}
The structure of experiments has been designed to impart design level
\end{flushleft}





\begin{flushleft}
$\bullet$	 Experiments in Heat Conduction/ Convection. And Heat Sink
\end{flushleft}


\begin{flushleft}
Characterisation
\end{flushleft}





\begin{flushleft}
$\bullet$	 Microprocessor/Microcontroller based system design with emphasis
\end{flushleft}


\begin{flushleft}
on real world interfacing
\end{flushleft}


\begin{flushleft}
$\bullet$	 Experiments on precision measurement methods and metrology.
\end{flushleft}





\begin{flushleft}
DSP705 Advanced Instrument Technology Lab
\end{flushleft}


\begin{flushleft}
3 Credits (0-0-6)
\end{flushleft}


\begin{flushleft}
Experiments on design, simulation and verification of instrumentation
\end{flushleft}


\begin{flushleft}
sub-systems addressing the following objectives:
\end{flushleft}


\begin{flushleft}
the performance of practical transducer systems and their processing
\end{flushleft}


\begin{flushleft}
circuits dealing with other devices and circuital noise validation of
\end{flushleft}


\begin{flushleft}
algorithms for information extraction from sensor signatures dynamic
\end{flushleft}


\begin{flushleft}
range, threshold and sensitivity characterization and response time
\end{flushleft}


\begin{flushleft}
evaluation in practical environments
\end{flushleft}


\begin{flushleft}
Experiments based on Digital Signal Processing hardware and software
\end{flushleft}


\begin{flushleft}
to: Study of DSP architecture; Interfacing with peripheral components;
\end{flushleft}


\begin{flushleft}
Implementation of DSP algorithm; Experiments based on Talbot effect,
\end{flushleft}


\begin{flushleft}
Digital Speckle Pattern Interferometry and Shack Hartmann Sensor.
\end{flushleft}





\begin{flushleft}
DSL710 Framework of Design
\end{flushleft}


\begin{flushleft}
2 Credits (2-0-0)
\end{flushleft}


\begin{flushleft}
Definition of design as an industrial and social activity. Understanding
\end{flushleft}


\begin{flushleft}
of {`}design' as a noun and as a verb. Design as a case of ill structured,
\end{flushleft}


\begin{flushleft}
ill defined, ill constrained problem solving. Comparative study of
\end{flushleft}


\begin{flushleft}
production processes in art, engineering and design. Design as cycle of
\end{flushleft}


\begin{flushleft}
analysis, synthesis, and validation of ideas. Design as the meeting point
\end{flushleft}


\begin{flushleft}
of the user needs, technology affordance and business goals. History
\end{flushleft}


\begin{flushleft}
of art and design. Influence of society and culture on design. Study of
\end{flushleft}


\begin{flushleft}
successful and failed products. Study of evolution of designed products.
\end{flushleft}


\begin{flushleft}
Consideration of advertising, marketing, consumer satisfaction,
\end{flushleft}


\begin{flushleft}
prevalent expertise, economic viability, production ecosystem, future
\end{flushleft}


\begin{flushleft}
prediction, legal and statutory concerns, IPR issues in design success.
\end{flushleft}





\begin{flushleft}
DSL711 Sensors and Transducers
\end{flushleft}


\begin{flushleft}
3 Credits (3-0-0)
\end{flushleft}


\begin{flushleft}
Transducer Fundamentals: Transducer terminology Design and
\end{flushleft}


\begin{flushleft}
performance characteristics, --- criteria for transducer selection, Case
\end{flushleft}


\begin{flushleft}
Studies -- Transducers principles of representative cases with emphasis
\end{flushleft}


\begin{flushleft}
on special {``}Electronic Conditioning requirements'' of different type
\end{flushleft}


\begin{flushleft}
of sensors-- Resistive transducer; Inductive transducers; capacitive
\end{flushleft}


\begin{flushleft}
transducers; piezoelectric transducer; semiconductor and other
\end{flushleft}


\begin{flushleft}
sensing structures. Displacement transducers; tachometers and
\end{flushleft}


\begin{flushleft}
velocity transducers; accelerometers and gyros; strain gauges; force
\end{flushleft}


\begin{flushleft}
and torque transducers; flow meters and level sensors; pressure
\end{flushleft}


\begin{flushleft}
transducers; sound and ultrasonic transducer. Phototubes and
\end{flushleft}


\begin{flushleft}
photodiodes; photovoltaic and photoconductive cells, photoemission,
\end{flushleft}


\begin{flushleft}
photo electromagnetic, detectors pressure actuated photoelectric
\end{flushleft}


\begin{flushleft}
detectors, design and operation of optical detectors, detector
\end{flushleft}


\begin{flushleft}
characteristics.
\end{flushleft}


\begin{flushleft}
Brief Introduction -- Smart Intelligent Sensors, MEMS, Nano.
\end{flushleft}


\begin{flushleft}
Transducer Performance: Static and dynamic performance parameters
\end{flushleft}


\begin{flushleft}
Standards: Electrical tests, measurement unit, measurement standards
\end{flushleft}


\begin{flushleft}
of of voltage, current, frequency, impedance etc .
\end{flushleft}


\begin{flushleft}
Errors and noise: types of errors, Effect of noise and errors on
\end{flushleft}


\begin{flushleft}
resolution and threshold. Dynamic range.
\end{flushleft}





312





\begin{flushleft}
\newpage
Instrument Design and Development Centre
\end{flushleft}





\begin{flushleft}
Testing: Calibration, dynamic tests, environmental test, life test.
\end{flushleft}


\begin{flushleft}
Case Studies in Application of transducers: displacement, velocity,
\end{flushleft}


\begin{flushleft}
acceleration, force, stress, strain, pressure and temperature
\end{flushleft}


\begin{flushleft}
measurement. Angular and linear encoders, Radar, laser and sonar
\end{flushleft}


\begin{flushleft}
distance measurement, Tachometers, Viscometer, densitometer.
\end{flushleft}





\begin{flushleft}
DSP711 Computer Aided Product Detailing
\end{flushleft}


\begin{flushleft}
3 Credits (1-0-4)
\end{flushleft}


\begin{flushleft}
DSL712 Electronic Techniques for Signal Conditioning
\end{flushleft}


\begin{flushleft}
and Interfacing
\end{flushleft}


\begin{flushleft}
3 Credits (3-0-0)
\end{flushleft}


\begin{flushleft}
Review of Network theory, transmission lines and Circuit parameters (Z
\end{flushleft}


\begin{flushleft}
Y Hybrid, etc) and introduction to HF Design and S parameters Analog
\end{flushleft}


\begin{flushleft}
signal conditioning, Ultra- precision conditioning, Gain; attenuation;
\end{flushleft}


\begin{flushleft}
input and output impedances; single ended and differential signals;
\end{flushleft}


\begin{flushleft}
CMRR; system-module interfacing consideration; measurement and
\end{flushleft}


\begin{flushleft}
characterization of electronic system modules.
\end{flushleft}





\begin{flushleft}
products in the futuristic context. 2. Research planning strategies,
\end{flushleft}


\begin{flushleft}
finding real challenges, Methods of Exploring design situations,
\end{flushleft}


\begin{flushleft}
developing questionnaires for interviewing users. 3. Searching for
\end{flushleft}


\begin{flushleft}
visual inconsistencies, 4. Creativity methods, Brainstorming, Synectics,
\end{flushleft}


\begin{flushleft}
5. Issue Tree, Mind Mapping, 6. Story Boarding, Image boarding 7. SixThinking Hats, Harvey Cards, Lotus Blossom, COCD, 8. Lateral Thinking,
\end{flushleft}


\begin{flushleft}
Wishful Thinking. 9. Specification writing, 10. Methods of exploring
\end{flushleft}


\begin{flushleft}
problem structure, Product-environment and Product-component
\end{flushleft}


\begin{flushleft}
interaction, 11. Alexenders method of determining components, 12.
\end{flushleft}


\begin{flushleft}
Interaction Matrix and net, 13. Analysis of interconnected decision
\end{flushleft}


\begin{flushleft}
areas, System innovation, 14. Functional Innovation by boundary
\end{flushleft}


\begin{flushleft}
shifting, through boundary searching and experimentation.
\end{flushleft}





\begin{flushleft}
DSL722 Precision Measurement Systems
\end{flushleft}


\begin{flushleft}
3 Credits (3-0-0)
\end{flushleft}





\begin{flushleft}
Analog and digital data transmission; modulation \& demodulation;
\end{flushleft}


\begin{flushleft}
Data transmission; channel noise and noise immunity factors. Data
\end{flushleft}


\begin{flushleft}
busses; GPIB and other standards in parallel data transmission. Optoelectronic interfacing techniques.
\end{flushleft}





\begin{flushleft}
Fundamentals of precision measurements: accuracy, precision,
\end{flushleft}


\begin{flushleft}
resolution, repeatability, reproducibility, consistency, drift analysis,
\end{flushleft}


\begin{flushleft}
dynamic range, Measurements and error estimation, systematic and
\end{flushleft}


\begin{flushleft}
random errors, Instrument transfer function, least square method
\end{flushleft}


\begin{flushleft}
and its applications, filtering, polynomial fitting, data analysis and
\end{flushleft}


\begin{flushleft}
statistical inference, correlation, Surface roughness, waviness and
\end{flushleft}


\begin{flushleft}
shape measurements, Study of some measurement systems such
\end{flushleft}


\begin{flushleft}
as mechanical and optical profilers, circularity, cylindricity and
\end{flushleft}


\begin{flushleft}
conicity measurement systems, Co-ordinate measuring machine,
\end{flushleft}


\begin{flushleft}
profile projector, long trace slope measuring profilometer, ShackHartmann sensor for slope measurement, Different Interferometers
\end{flushleft}


\begin{flushleft}
for optical metrology, absolute testing techniques, Moire techniques
\end{flushleft}


\begin{flushleft}
for measurements in industrial applications.
\end{flushleft}





\begin{flushleft}
Analog and digital representation of data; comparisons and relative
\end{flushleft}


\begin{flushleft}
merits; multiplexing and demultiplexing of analog and digital data,
\end{flushleft}


\begin{flushleft}
ADC/DAC. Microcontroller and DSP applications.
\end{flushleft}





\begin{flushleft}
DSP722 Applied Ergonomics
\end{flushleft}


\begin{flushleft}
2 Credits (1-0-2)
\end{flushleft}





\begin{flushleft}
Analog and digital System Co-housing: EMI effects and EMC measures;
\end{flushleft}


\begin{flushleft}
circuit card placement; shielding and grounding techniques; ground
\end{flushleft}


\begin{flushleft}
loop management; isolation and interference filtering. EMI hardening
\end{flushleft}


\begin{flushleft}
and EMC interfacing.
\end{flushleft}





\begin{flushleft}
Application of CPU's in signal and data handling; response
\end{flushleft}


\begin{flushleft}
linearization and drift compensation; data logger, computer aided
\end{flushleft}


\begin{flushleft}
measurement and control.
\end{flushleft}





\begin{flushleft}
DSP712 Exhibitions and Environmental Design
\end{flushleft}


\begin{flushleft}
3 Credits (2-0-2)
\end{flushleft}


\begin{flushleft}
course contents History of exhibition design. Human movements and
\end{flushleft}


\begin{flushleft}
exhibition plans. Concept of physical and psychological space. Design of
\end{flushleft}


\begin{flushleft}
physical environment for human comfort and function. Study of fixtures
\end{flushleft}


\begin{flushleft}
and fittings. Design and use of modular and fixed elements in display
\end{flushleft}


\begin{flushleft}
design. Lights and illumination systems. Types of lights. Study of indoor
\end{flushleft}


\begin{flushleft}
and outdoor lighting requirements. Innovative materials and processes
\end{flushleft}


\begin{flushleft}
in exhibitions. Outdoor and indoor land-scaping . Types of pavilions.
\end{flushleft}


\begin{flushleft}
Space requirement calculation in design. Design of murals, artifacts,
\end{flushleft}


\begin{flushleft}
exhibits and models. Use of background, negative space, foreground,
\end{flushleft}


\begin{flushleft}
proportion and scale in exhibition composition. Exercises and projects.
\end{flushleft}





\begin{flushleft}
DSL714 Instrument Design and Simulations
\end{flushleft}


\begin{flushleft}
3 Credits (2-0-2)
\end{flushleft}


\begin{flushleft}
Review of circuit analysis and design. Review of signals and systems
\end{flushleft}


\begin{flushleft}
in time and frequency domain: Fourier and Laplace Transforms,
\end{flushleft}


\begin{flushleft}
response plots.
\end{flushleft}


\begin{flushleft}
Dynamic properties of instrument systems: Review of instrument
\end{flushleft}


\begin{flushleft}
control systems, on-off, proportional and PID controllers. Stability
\end{flushleft}


\begin{flushleft}
considerations, gain and phase margin.
\end{flushleft}


\begin{flushleft}
Use of pulse and harmonic test signals for performance evaluation.
\end{flushleft}


\begin{flushleft}
Linear modelling of instrument systems. Models for basic instrument
\end{flushleft}


\begin{flushleft}
building blocks. Simulation studies of circuit blocks. Simulation studies
\end{flushleft}


\begin{flushleft}
of circuits, instrument modules, transducers and control schemes using
\end{flushleft}


\begin{flushleft}
PSPICE and MATLAB expert simulation software.
\end{flushleft}





\begin{flushleft}
DSS720 Independent Study
\end{flushleft}


\begin{flushleft}
3 Credits (0-3-0)
\end{flushleft}





\begin{flushleft}
Definition, origin, scope and goals of ergonomics as a field of study.
\end{flushleft}


\begin{flushleft}
Examples of applications of ergonomics in design. Types of data from
\end{flushleft}


\begin{flushleft}
human at physical, physiological, cognitive and affective levels. Data
\end{flushleft}


\begin{flushleft}
gathering and analysis techniques. Use of descriptive and inferential
\end{flushleft}


\begin{flushleft}
statistics in ergonomic data. Applications of mean, median, mode and
\end{flushleft}


\begin{flushleft}
percentile in anthropometry. Use of anthropometry in workstation
\end{flushleft}


\begin{flushleft}
design. Human physiological potentials and limitations in terms of load
\end{flushleft}


\begin{flushleft}
carrying capacity. Concept of comfort, fatigue and stress. Design for
\end{flushleft}


\begin{flushleft}
the cognitive user. Concept of mental workload. Cognitive perspective
\end{flushleft}


\begin{flushleft}
in control panel design and graphical user interface design.
\end{flushleft}





\begin{flushleft}
DSL731 Optical Components and Basic Instruments
\end{flushleft}


\begin{flushleft}
3 Credits (3-0-0)
\end{flushleft}


\begin{flushleft}
Generation of light: Thermal, non-thermal and semiconductor light
\end{flushleft}


\begin{flushleft}
sources. Measurement of light and instrumentation, Properties and
\end{flushleft}


\begin{flushleft}
propagation of light; The Ray Optics, Wave Optics, and Electromagnetic
\end{flushleft}


\begin{flushleft}
Optics; Basics of interference, diffraction and polarization of light.
\end{flushleft}


\begin{flushleft}
Optical Components: Reflecting components, plane, Spherical,
\end{flushleft}


\begin{flushleft}
paraboloidal, total internal reflection. Refracting components;
\end{flushleft}


\begin{flushleft}
Converging, diverging and combination of lenses, Design analysis and
\end{flushleft}


\begin{flushleft}
image formation by lenses, Wavefront aberrations; Monochromatic
\end{flushleft}


\begin{flushleft}
(Seidel), and chromatic aberrations. Eyepices: Huygens, Ramsden,
\end{flushleft}


\begin{flushleft}
and special eyepieces; Prisms, Polarizing prisms: Glan Taylor Polarizer,
\end{flushleft}


\begin{flushleft}
Glan- Thomson prism polarizer, Rochon Prism Polarizer, Senarmont
\end{flushleft}


\begin{flushleft}
prism polarizer, Wollaston Prism, Phase plates ($\lambda$/2, $\lambda$/4), Soleil --
\end{flushleft}


\begin{flushleft}
Babinet compensator, Diffracting components; diffraction by single/
\end{flushleft}


\begin{flushleft}
multiple/openings, types of gratings and fabrication techniques,
\end{flushleft}


\begin{flushleft}
diffractive optical elements. Polarizing components; Polarization
\end{flushleft}


\begin{flushleft}
by reflection, and double refraction, birefringence crystals, and
\end{flushleft}


\begin{flushleft}
polarization based optical devices, Rotatory Polarization, Polarization
\end{flushleft}


\begin{flushleft}
rotators; Optical instruments: Microscopes, Telescopes, cystoscope;
\end{flushleft}


\begin{flushleft}
Refracting, reflecting, interferometric telescopes. Interferometers;
\end{flushleft}


\begin{flushleft}
two- beam, multiple-beam, and shearing interferometers; Detectors:
\end{flushleft}


\begin{flushleft}
Photodetectors, CCD and CMOS detectors, IR-detectors.
\end{flushleft}





\begin{flushleft}
DSP731 Communication and presentation skills
\end{flushleft}


\begin{flushleft}
3 Credits (1-0-4)
\end{flushleft}





\begin{flushleft}
DSP721 Design and Innovation Methods
\end{flushleft}


\begin{flushleft}
3 Credits (1-0-4)
\end{flushleft}


\begin{flushleft}
1. Understanding of factors that directly or indirectly influence the
\end{flushleft}


\begin{flushleft}
product definition and its context. Assessing relevance of available
\end{flushleft}





\begin{flushleft}
Concept of sketching for designers, sketching through geometrical
\end{flushleft}


\begin{flushleft}
shapes, Sketching in isometric grids, sculpting conceptual objects
\end{flushleft}


\begin{flushleft}
while sketching through cuboids. The use of shade and shadows,
\end{flushleft}





313





\begin{flushleft}
\newpage
Instrument Design and Development Centre
\end{flushleft}





\begin{flushleft}
Rendering, physical product modeling through frugal materials and
\end{flushleft}


\begin{flushleft}
by the use of MDF, HIPS, Vacuum forming, modeling in FRP, product
\end{flushleft}


\begin{flushleft}
photography, video recording for presentations.
\end{flushleft}





\begin{flushleft}
DSL732 Adv Mat Processes \& Die Design
\end{flushleft}


\begin{flushleft}
3 Credits (2-0-2)
\end{flushleft}


\begin{flushleft}
Understanding properties and selection of natural and manmade
\end{flushleft}


\begin{flushleft}
materials including metals, plastics, ceramics, composites and
\end{flushleft}


\begin{flushleft}
natural materials.
\end{flushleft}


\begin{flushleft}
Understanding various manufacturing and prototyping methods
\end{flushleft}


\begin{flushleft}
including digital manufacturing/ prototyping.
\end{flushleft}


\begin{flushleft}
Hands on product realization exercises involving selection of materials
\end{flushleft}


\begin{flushleft}
and manufacturing processes.
\end{flushleft}


\begin{flushleft}
Die and mould manufacturing methods including surface treatment
\end{flushleft}


\begin{flushleft}
and finishing processes.
\end{flushleft}


\begin{flushleft}
Prototyping projects involving CNC, 3-D Printing, Vacuum forming, etc.
\end{flushleft}





\begin{flushleft}
DSL733 Optical Materials and Optical Techniques in
\end{flushleft}


\begin{flushleft}
Instrumentation
\end{flushleft}


\begin{flushleft}
3 Credits (2-0-2)
\end{flushleft}


\begin{flushleft}
Optical materials: Optical and mechanical characteristics of optical
\end{flushleft}


\begin{flushleft}
glass, metal optics, plastic optics and optical crystals, Manufacturing of
\end{flushleft}


\begin{flushleft}
optics on optical glass and plastics, Injection molding of plastic optics,
\end{flushleft}


\begin{flushleft}
Single point diamond turning and CNC milling and micromachining,
\end{flushleft}


\begin{flushleft}
replication techniques
\end{flushleft}


\begin{flushleft}
Photo-Lithography and its optical system, Illumination and projection
\end{flushleft}


\begin{flushleft}
systems, Astronomical and remote sensing systems, detectors:
\end{flushleft}


\begin{flushleft}
Thermal detectors, photon detectors and Imaging detectors.
\end{flushleft}





\begin{flushleft}
DSL734 Laser Based Instrumentation
\end{flushleft}


\begin{flushleft}
3 Credits (3-0-0)
\end{flushleft}


\begin{flushleft}
Radiation and matter interaction and fundamental of LASER action.
\end{flushleft}


\begin{flushleft}
The LASER and it's properties. Laser systems - gas and semiconductor
\end{flushleft}


\begin{flushleft}
LASERS. LASER beam optics and propagation of LASER beams.
\end{flushleft}


\begin{flushleft}
Fundamentals of holography - basic theory of holography, recording
\end{flushleft}


\begin{flushleft}
medium and type of holograms for display purposes. Applications
\end{flushleft}


\begin{flushleft}
of holography in metrology - Holographic Interferometry, double
\end{flushleft}


\begin{flushleft}
exposure, time averaged and real time holographic interferometry.
\end{flushleft}


\begin{flushleft}
Laser speckles and laser speckles techniques. Digital speckle pattern
\end{flushleft}


\begin{flushleft}
interferometry (DSPI) and digital holographic interferometry (DHI)
\end{flushleft}


\begin{flushleft}
in measurements of displacement, refractive index, temperature,
\end{flushleft}


\begin{flushleft}
shape, vibration and material properties. Two wavelength and
\end{flushleft}


\begin{flushleft}
phase-shifting interferometry. Laser based temperature measurement
\end{flushleft}


\begin{flushleft}
techniques. Collimation testing and laser based alignment systems.
\end{flushleft}


\begin{flushleft}
Laser based techniques for low frequency and high frequency
\end{flushleft}


\begin{flushleft}
vibration measurements. Talbot interferometry and its applications in
\end{flushleft}


\begin{flushleft}
scientific and industrial measurements. Shearing interferometry and
\end{flushleft}


\begin{flushleft}
its applications in scientific and industrial measurements. Sensing of
\end{flushleft}


\begin{flushleft}
high currents on high voltage lines using magneto optic effect.
\end{flushleft}





\begin{flushleft}
DSL737 Display Devices and Technology
\end{flushleft}


\begin{flushleft}
3 Credits (3-0-0)
\end{flushleft}





\begin{flushleft}
Display electronics and digital light processing technologies. Threedimensional (3-D) imaging and display technologies: Micro-displays,
\end{flushleft}


\begin{flushleft}
STEREOSCOPIC 3D displays. HOLOGRAPHIC 3-D displays. Laser
\end{flushleft}


\begin{flushleft}
based 3D-TV.
\end{flushleft}





\begin{flushleft}
DSL740 Instrument Organization and Ergonomics
\end{flushleft}


\begin{flushleft}
3 Credits (2-0-2)
\end{flushleft}


\begin{flushleft}
Functions of instrument systems, classification of tasks as manmachine systems, need analysis, product specifications, solutions
\end{flushleft}


\begin{flushleft}
search, product planning, systems break-up. Strengths and
\end{flushleft}


\begin{flushleft}
weaknesses of the machines.
\end{flushleft}


\begin{flushleft}
Understanding the potentials and weaknesses of the human
\end{flushleft}


\begin{flushleft}
beings, application of force, load lifting, load carrying, stride
\end{flushleft}


\begin{flushleft}
patterns. Functions of controls and displays, handles, levers, knobs,
\end{flushleft}


\begin{flushleft}
switches, dials, LCD screens. Hand-held devices, workstations,
\end{flushleft}


\begin{flushleft}
large control systems.
\end{flushleft}


\begin{flushleft}
Identification of constraints emerging from scientific, technical,
\end{flushleft}


\begin{flushleft}
production, environmental and maintenance considerations. Aesthetics
\end{flushleft}


\begin{flushleft}
of color, form and graphics. Value engineering. Design of manuals,
\end{flushleft}


\begin{flushleft}
job-aids and training aids.
\end{flushleft}


\begin{flushleft}
Case studies, Exercises \& Projects.
\end{flushleft}





\begin{flushleft}
DSP741 Product Interface \& Design
\end{flushleft}


\begin{flushleft}
2 Credits (1-0-2)
\end{flushleft}


\begin{flushleft}
A product as a living organism, its interface externally with the
\end{flushleft}


\begin{flushleft}
environment and internally with its components. 2. Interface for
\end{flushleft}


\begin{flushleft}
modulating user involvement. 3. Product Semantics, communication
\end{flushleft}


\begin{flushleft}
of feelings, 4. Communication of structure and purpose. 5.
\end{flushleft}


\begin{flushleft}
Communication through form, color, graphics and text. 6. Typography
\end{flushleft}


\begin{flushleft}
choice and readability, Printing and Transfer Techniques. 7. Product
\end{flushleft}


\begin{flushleft}
graphics. 8. Functioning of controls and display elements, knobs, push
\end{flushleft}


\begin{flushleft}
buttons, handles and electronic displays. 9. Investigation and study
\end{flushleft}


\begin{flushleft}
of visual, functional and ergonomic requirements of controls and
\end{flushleft}


\begin{flushleft}
displays, legibility of display elements. 10. Study of different textures,
\end{flushleft}


\begin{flushleft}
patterns and materials. 11. Area, volume and Proportion. 12. Order
\end{flushleft}


\begin{flushleft}
and system. 13. Human factors and safety in interface design. 14.
\end{flushleft}


\begin{flushleft}
Individually planned design projects involving research analysis and
\end{flushleft}


\begin{flushleft}
design of product interface.
\end{flushleft}





\begin{flushleft}
DSL751 Form and Aesthetics
\end{flushleft}


\begin{flushleft}
3 Credits (2-0-2)
\end{flushleft}


\begin{flushleft}
Elements of design, Nature inspired design, Gestalt, Product semantics,
\end{flushleft}


\begin{flushleft}
Color theory and color trends, Varied approaches for form, Product
\end{flushleft}


\begin{flushleft}
styling, case studies and design discourse on form, exercises on from
\end{flushleft}


\begin{flushleft}
development of a product (existing or conceptual).
\end{flushleft}





\begin{flushleft}
DSR761 Social Immersion :
\end{flushleft}


\begin{flushleft}
1 Credit (0-0-2)
\end{flushleft}


\begin{flushleft}
Documentation of societal issues through photography, recordings,
\end{flushleft}


\begin{flushleft}
sketches. Identification of locale of working, Identification of a
\end{flushleft}


\begin{flushleft}
potential societal issue. Presenting the societal issues as a report/
\end{flushleft}


\begin{flushleft}
presentation/ video.
\end{flushleft}





\begin{flushleft}
Human vision. Basics of luminescence, fluorescence, and
\end{flushleft}


\begin{flushleft}
phosphorescence. Display materials and their characterization.
\end{flushleft}


\begin{flushleft}
Emissive displays: Review of cathode ray tube (CRT) displays. Plasma
\end{flushleft}


\begin{flushleft}
display devices and technologies, field-emissive, electro-chromic and
\end{flushleft}


\begin{flushleft}
photo-chromic displays. Inorganic, organic and polymeric LED based
\end{flushleft}


\begin{flushleft}
display devices: Device physics, materials, fabrication processes,
\end{flushleft}


\begin{flushleft}
structures, and drive circuits. Electro-optical characterization of LEDs.
\end{flushleft}


\begin{flushleft}
Transparent thin film (TFTs) displays, electronics, and manufacturing
\end{flushleft}


\begin{flushleft}
technologies and applications.
\end{flushleft}





\begin{flushleft}
DSR762 Vehicle Design
\end{flushleft}


\begin{flushleft}
3 Credits (2-0-2)
\end{flushleft}





\begin{flushleft}
Non-emissive displays: basics of liquid-crystal materials, their properties
\end{flushleft}


\begin{flushleft}
and characterization. Liquid-crystal display devices and technologies.
\end{flushleft}


\begin{flushleft}
Transmissive, reflective, active and passive matrix, thin-film transistor
\end{flushleft}


\begin{flushleft}
(TFT), transreflective, and back lighting technologies for LCDs.
\end{flushleft}





\begin{flushleft}
DSR772 Transportation Design
\end{flushleft}


\begin{flushleft}
3 Credits (2-0-2)
\end{flushleft}





\begin{flushleft}
Electronic-ink, electronic paper, and flexible and transparent display
\end{flushleft}


\begin{flushleft}
technologies and their applications. Laser based projection displays
\end{flushleft}





\begin{flushleft}
History of transportation, technology trends in transportation and
\end{flushleft}


\begin{flushleft}
futuristic predictions. Types of personal vehicles, mass transportation
\end{flushleft}


\begin{flushleft}
vehicles, their benefits and challenges in design. Trends and styling
\end{flushleft}


\begin{flushleft}
of two wheelers and four wheelers. Material and finish considerations
\end{flushleft}


\begin{flushleft}
in styling. Use of mood boards and cultural trends in transportation
\end{flushleft}


\begin{flushleft}
design. Vehicle design for rural India. Design projects and exercises.
\end{flushleft}





\begin{flushleft}
Understanding different segments of design practice in transportation.
\end{flushleft}


\begin{flushleft}
Different role of designers in Automobile Industry. Design of human
\end{flushleft}


\begin{flushleft}
powered vehicles, Two wheelers design, Design of four wheelers,
\end{flushleft}


\begin{flushleft}
future of transportation, Styling, Professional practice: CAS and Clay.
\end{flushleft}





314





\begin{flushleft}
\newpage
Instrument Design and Development Centre
\end{flushleft}





\begin{flushleft}
DSL782 Design for Usability
\end{flushleft}


\begin{flushleft}
3 Credits (2-0-2)
\end{flushleft}





\begin{flushleft}
DSR822 Design for Sustainability
\end{flushleft}


\begin{flushleft}
3 Credits (2-0-2)
\end{flushleft}





\begin{flushleft}
History of product increase in complexity and usability since
\end{flushleft}


\begin{flushleft}
WW-2. Story of transition in human society form Man-Machine
\end{flushleft}


\begin{flushleft}
Interaction to Human-Computer Interaction. Relationship between
\end{flushleft}


\begin{flushleft}
product complexity and mental workload. Subjective and objective
\end{flushleft}


\begin{flushleft}
measurements of product complexity and mental workload. User
\end{flushleft}


\begin{flushleft}
centered design process for usable product design. Introduction to
\end{flushleft}


\begin{flushleft}
concept of {`}mental models' and exercises in understanding users'
\end{flushleft}


\begin{flushleft}
mental models. Creation of Personas and scenarios. Conduct of task
\end{flushleft}


\begin{flushleft}
analysis. Operational definitions of usability. Measurement of ease
\end{flushleft}


\begin{flushleft}
of use, efficiency and effectiveness of digital products. Design of
\end{flushleft}


\begin{flushleft}
interactive products from usability perspective. Development of user
\end{flushleft}


\begin{flushleft}
screeners, testing protocols and conduct of usability tests. Creation
\end{flushleft}


\begin{flushleft}
of paper prototypes, wireframes, information architecture. Conduct of
\end{flushleft}


\begin{flushleft}
low fidelity tests, card sorting, reverse card sorting, affordance tests,
\end{flushleft}


\begin{flushleft}
high fidelity testing and brand testing.
\end{flushleft}





\begin{flushleft}
Concept of sustainability, Tipple bottom line, world vision for
\end{flushleft}


\begin{flushleft}
sustainability, Emerging trends in the area of sustainability, Metrics
\end{flushleft}


\begin{flushleft}
for measurement of sustainability, Product lifecycle management and
\end{flushleft}


\begin{flushleft}
sustainability, Ecodesign.
\end{flushleft}





\begin{flushleft}
DSR832 Design for User Experience
\end{flushleft}


\begin{flushleft}
3 Credits (3-0-0)
\end{flushleft}





\begin{flushleft}
Aim of the project to help the student independently solve a design
\end{flushleft}


\begin{flushleft}
problem against an pre-identified design brief.
\end{flushleft}





\begin{flushleft}
Importance of user experience approach in design. Methods to
\end{flushleft}


\begin{flushleft}
understand users' experiences. Modeling of user behaviors. Cognitive,
\end{flushleft}


\begin{flushleft}
affective and cultural perspectives in experiences. Consideration for
\end{flushleft}


\begin{flushleft}
human experiences in interaction design. Methods of direct, indirect,
\end{flushleft}


\begin{flushleft}
subjective and objective measurements of human experience. Issues
\end{flushleft}


\begin{flushleft}
of reliability and validity in experience measurement. Qualitative
\end{flushleft}


\begin{flushleft}
interview techniques for gathering user motivations and emotions.
\end{flushleft}


\begin{flushleft}
Analysis of qualitative experiential data from users. Gender and
\end{flushleft}


\begin{flushleft}
cultural biases in experience measurements. Management of
\end{flushleft}


\begin{flushleft}
psychological space in user experience testing setups. Development
\end{flushleft}


\begin{flushleft}
of user experience strategy, creation of user interfaces and testing
\end{flushleft}


\begin{flushleft}
of digital products from experiential perspective.
\end{flushleft}





\begin{flushleft}
DSD801 Major Project Part-I
\end{flushleft}


\begin{flushleft}
6 Credits (0-0-12)
\end{flushleft}





\begin{flushleft}
DSL841 Design Management and Professional Practice
\end{flushleft}


\begin{flushleft}
3 Credits (3-0-0)
\end{flushleft}





\begin{flushleft}
DSD792 Design Project-I
\end{flushleft}


\begin{flushleft}
3 Credits (0-0-6)
\end{flushleft}





\begin{flushleft}
DSR801 Summer Internship:
\end{flushleft}


\begin{flushleft}
2 Credits (0-0-4)
\end{flushleft}


\begin{flushleft}
Identifying of the project area, setting the objectives, milieu and
\end{flushleft}


\begin{flushleft}
deliverables of the Internship, report writing/Presentation.
\end{flushleft}





\begin{flushleft}
DSD802 Major Project Part-I
\end{flushleft}


\begin{flushleft}
12 Credits (0-0-24)
\end{flushleft}


\begin{flushleft}
DSL810 Special Topics in Design-I
\end{flushleft}


\begin{flushleft}
3 Credits (3-0-0)
\end{flushleft}





\begin{flushleft}
Considerations in professional design startups including setting up
\end{flushleft}


\begin{flushleft}
a design office, getting finances, finding clients, running the office,
\end{flushleft}


\begin{flushleft}
business correspondence, brief and briefing, feasibility reports, letters
\end{flushleft}


\begin{flushleft}
of contract. Estimates of design fee as lump sum, hourly basis,
\end{flushleft}


\begin{flushleft}
consulting, commissioning and royalties. Study of govt. regulations,
\end{flushleft}


\begin{flushleft}
consumer protection acts, ISI standards, design registrations,
\end{flushleft}


\begin{flushleft}
patents and copyrights. Professional ethics in design practice.
\end{flushleft}


\begin{flushleft}
Creativity theory. Integrated product development. Assessing risks
\end{flushleft}


\begin{flushleft}
and opportunities. Cost cutting in design.
\end{flushleft}





\begin{flushleft}
Special topics in design.
\end{flushleft}





\begin{flushleft}
DSR852 Strategic Design Management
\end{flushleft}


\begin{flushleft}
3 Credits (2-0-2)
\end{flushleft}





\begin{flushleft}
DSL811 Selected Topics in Instrumentation-I
\end{flushleft}


\begin{flushleft}
3 Credits (3-0-0)
\end{flushleft}





\begin{flushleft}
Branding and brand development, Repositioning in market, disruptive
\end{flushleft}


\begin{flushleft}
innovation for market capitalization, Retail design, design of services,
\end{flushleft}


\begin{flushleft}
designing for new businesses.
\end{flushleft}





\begin{flushleft}
Advanced course on Selected Topics in Instrumentation to the M.Tech.
\end{flushleft}


\begin{flushleft}
Instrument Technology Programme.
\end{flushleft}





\begin{flushleft}
DSC812 Term Paper and Seminar
\end{flushleft}


\begin{flushleft}
3 Credits (0-3-0)
\end{flushleft}


\begin{flushleft}
DSR812 Media Studies
\end{flushleft}


\begin{flushleft}
3 Credits (2-0-2)
\end{flushleft}


\begin{flushleft}
To enable designers to use different media optimally. Principles and
\end{flushleft}


\begin{flushleft}
processes of photography, videography, print and animation. Study
\end{flushleft}


\begin{flushleft}
of design constraints and affordances in differ media. Exercises in
\end{flushleft}


\begin{flushleft}
photography, videography, print, animation Cinematography etc.
\end{flushleft}


\begin{flushleft}
Design of corporate identity programs.
\end{flushleft}





\begin{flushleft}
DSL814 Selected Topics in Instrumentation-II
\end{flushleft}


\begin{flushleft}
3 Credits (3-0-0)
\end{flushleft}


\begin{flushleft}
Advanced course on Selected Topics in Instrumentation to the M.Tech.
\end{flushleft}


\begin{flushleft}
Instrument Technology Programme.
\end{flushleft}





\begin{flushleft}
DSL815 Special Topics in Instrumentation
\end{flushleft}


\begin{flushleft}
1 Credit (1-0-0)
\end{flushleft}


\begin{flushleft}
DSL820 Special Topics in Design-II
\end{flushleft}


\begin{flushleft}
3 Credits (3-0-0)
\end{flushleft}


\begin{flushleft}
Special topics in design.
\end{flushleft}





\begin{flushleft}
DSV820 Special Modules in Design
\end{flushleft}


\begin{flushleft}
1 Credit (1-0-0)
\end{flushleft}





\begin{flushleft}
DSR862 Design in Indian Context
\end{flushleft}


\begin{flushleft}
3 Credits (3-0-0)
\end{flushleft}


\begin{flushleft}
Introduction to culture form product design perspective. Models and
\end{flushleft}


\begin{flushleft}
definitions of culture. Product design culture of India. Culture as an aid
\end{flushleft}


\begin{flushleft}
in consumer product choice. Cross cultural biases in product decisions.
\end{flushleft}


\begin{flushleft}
Cross cultural design teams. Considerations in designing for a user from
\end{flushleft}


\begin{flushleft}
another culture. Exercises in product as a cultural thought. Culture in
\end{flushleft}


\begin{flushleft}
evolutionary perspective and design of new material cultures through
\end{flushleft}


\begin{flushleft}
products and lifestyles design.
\end{flushleft}





\begin{flushleft}
DSD891 Design Project-II
\end{flushleft}


\begin{flushleft}
6 Credits (0-0-12)
\end{flushleft}


\begin{flushleft}
The student will be able to practice the design process to solve a
\end{flushleft}


\begin{flushleft}
professionally challenging design problem. The student should exhibit
\end{flushleft}


\begin{flushleft}
the sensitivity to the multidimensionality of the problems in the design
\end{flushleft}


\begin{flushleft}
domain. They should be able to prove their design outcome as viable
\end{flushleft}


\begin{flushleft}
and practice solution for the given problem. The students are expected
\end{flushleft}


\begin{flushleft}
to exhibit their work to the professional community.
\end{flushleft}





\begin{flushleft}
DSD892 Industry/ Research Design Project
\end{flushleft}


\begin{flushleft}
9 Credits (0-0-18)
\end{flushleft}


\begin{flushleft}
To develop the ability to look at design problems from a research
\end{flushleft}


\begin{flushleft}
perspective. The student is expected to contribute to the professional
\end{flushleft}


\begin{flushleft}
design field thorough new design knowledge generation. The project
\end{flushleft}


\begin{flushleft}
is aimed to polish the designer's research skills. The designer is
\end{flushleft}


\begin{flushleft}
expected to deliver cutting edge research and be able to articulate
\end{flushleft}


\begin{flushleft}
it professionally.
\end{flushleft}





315





\begin{flushleft}
\newpage
Centre for Polymer Science and Engineering
\end{flushleft}


\begin{flushleft}
PTV700 Special Lectures in Polymers
\end{flushleft}


\begin{flushleft}
1 Credit (1-0-0)
\end{flushleft}





\begin{flushleft}
PTP710 Polymer Engineering Lab
\end{flushleft}


\begin{flushleft}
1 Credit (0-0-2)
\end{flushleft}





\begin{flushleft}
There will only be special lectures followed by a final assignment or
\end{flushleft}


\begin{flushleft}
quiz.
\end{flushleft}





\begin{flushleft}
The course comprises of eight regular expreriments on various
\end{flushleft}


\begin{flushleft}
processing equipments and two experientys dealing with rheology
\end{flushleft}


\begin{flushleft}
of polymer melts.
\end{flushleft}





\begin{flushleft}
PTL701 Polymer Chemistry
\end{flushleft}


\begin{flushleft}
3 Credits (3-0-0)
\end{flushleft}


\begin{flushleft}
Introduction to polymers, nomenclature, addition, condensation, chain
\end{flushleft}


\begin{flushleft}
growth and step growth polymerization, kinetics of polymerization,
\end{flushleft}


\begin{flushleft}
material classes, polymerization techniques: bulk, suspension
\end{flushleft}


\begin{flushleft}
and emulsion polymerization; cationic, anionic and free radical
\end{flushleft}


\begin{flushleft}
polymerization; copolymerization, reactivity ratios; atom transfer
\end{flushleft}


\begin{flushleft}
radical polymerization.
\end{flushleft}





\begin{flushleft}
PTL702 Polymer Processing
\end{flushleft}


\begin{flushleft}
3 Credits (3-0-0)
\end{flushleft}


\begin{flushleft}
Course covers the classification of polymer processing operations,
\end{flushleft}


\begin{flushleft}
extrusion, molding based processes, compounding and mixing,
\end{flushleft}


\begin{flushleft}
thermoforming and other processing methods.
\end{flushleft}





\begin{flushleft}
PTL703 Polymer Physics
\end{flushleft}


\begin{flushleft}
3 Credits (3-0-0)
\end{flushleft}


\begin{flushleft}
The course content will include polymer molecules, their conformations,
\end{flushleft}


\begin{flushleft}
crystalline and two phase structures and their effects on various
\end{flushleft}


\begin{flushleft}
thermo-physical properties such as melting, glass transition and
\end{flushleft}


\begin{flushleft}
crystallization kinetics.
\end{flushleft}





\begin{flushleft}
PTL704 Polymer Technology
\end{flushleft}


\begin{flushleft}
3 Credits (3-0-0)
\end{flushleft}


\begin{flushleft}
Polymers of commercial importance; additives for plastics; stabilizers,
\end{flushleft}


\begin{flushleft}
fillers, plasticizers and extenders, lubricants and flow promoters,
\end{flushleft}


\begin{flushleft}
flame retardants, blowing agents, colourants, cross-lnking agents and
\end{flushleft}


\begin{flushleft}
biodegradation additives; manufacture, properties and applications
\end{flushleft}


\begin{flushleft}
of major thermoplastic and thermosetting polymers: polyethylene,
\end{flushleft}


\begin{flushleft}
polypropylene, poly(vinylene chloride), polystyrene and other
\end{flushleft}


\begin{flushleft}
styrenics, pheonol-formaldehyde, urea-melamine formaldehyde and
\end{flushleft}


\begin{flushleft}
unsaturated polyester resings.
\end{flushleft}





\begin{flushleft}
PTL705 Polymer Characterization
\end{flushleft}


\begin{flushleft}
3 Credits (3-0-0)
\end{flushleft}


\begin{flushleft}
Molecular weight and molecular dimensions by end-group
\end{flushleft}


\begin{flushleft}
analysis, osmometry, light scattering, viscometry, gel permeation
\end{flushleft}


\begin{flushleft}
chromatography, MALDI-TOF, Infra-red, NMR, UV-visible and
\end{flushleft}


\begin{flushleft}
Raman spectroscopic techniques. Thermal properties by differential
\end{flushleft}


\begin{flushleft}
scanning calorimetry, differential thermal analysis, thermogravimetry;
\end{flushleft}


\begin{flushleft}
Microscopy: optical and electron microscopy, X-ray scattering from
\end{flushleft}


\begin{flushleft}
polymers, small angle light scattering; crystallinity by density
\end{flushleft}


\begin{flushleft}
measurements. .
\end{flushleft}





\begin{flushleft}
PTL707 Polymer Engineering and Rheology
\end{flushleft}


\begin{flushleft}
3 Credits (3-0-0)
\end{flushleft}


\begin{flushleft}
Course covers Newtonian and non-Newtonian flow, simple shear flow
\end{flushleft}


\begin{flushleft}
and its significance, normal stresses, simple elongational flow and its
\end{flushleft}


\begin{flushleft}
significance, viscoelasticity, Rheometers, molecular, theoratical and
\end{flushleft}


\begin{flushleft}
related models.
\end{flushleft}





\begin{flushleft}
PTL711 Engineering Plastics and Speciality Polymers
\end{flushleft}


\begin{flushleft}
3 Credits (3-0-0)
\end{flushleft}


\begin{flushleft}
Introduction to engineering polymers, applications, processing,
\end{flushleft}


\begin{flushleft}
thermoplastic engineering plastics, polycarbonates, polyimides,
\end{flushleft}


\begin{flushleft}
polyphenylene oxide, liquid crystalline polymers, poly(ether ketone),
\end{flushleft}


\begin{flushleft}
thermosets, speciality polymers, hydrogels, conducting polymers,
\end{flushleft}


\begin{flushleft}
fluoropolymers.
\end{flushleft}





\begin{flushleft}
PTL712 Polymer Blends and Composites
\end{flushleft}


\begin{flushleft}
3 Credits (3-0-0)
\end{flushleft}


\begin{flushleft}
The course will cover definition and classification of blends and
\end{flushleft}


\begin{flushleft}
composites, miscibility, phase behaviour, nature of interface, nature
\end{flushleft}


\begin{flushleft}
of polymer matrices, reinforcements, basic theoretical models to
\end{flushleft}


\begin{flushleft}
predict mechanical properties and the role of fibre length, distribution,
\end{flushleft}


\begin{flushleft}
dispersion etc. on the performance properties of polymer based blends
\end{flushleft}


\begin{flushleft}
and composites.
\end{flushleft}





\begin{flushleft}
PTL713 Polymer Testing and Properties
\end{flushleft}


\begin{flushleft}
3 Credits (3-0-0)
\end{flushleft}


\begin{flushleft}
Properties of polymers and their measurements by standard test
\end{flushleft}


\begin{flushleft}
methods; tensile, flexural and impact properties; hardness, abrasion
\end{flushleft}


\begin{flushleft}
resistance and long term fracture properties; softening point, heat
\end{flushleft}


\begin{flushleft}
distortion temperature, thermal expansion coefficient and thermal
\end{flushleft}


\begin{flushleft}
conductivity; electrical insulation and conductivity; sorption, diffusion
\end{flushleft}


\begin{flushleft}
and permeation of gases/liquids through polymer membranes;
\end{flushleft}


\begin{flushleft}
standards used are BIS, BS, ASTM, ISO and DIM; correlation of test
\end{flushleft}


\begin{flushleft}
with actual performance; statistical quality control in various tests.
\end{flushleft}





\begin{flushleft}
PTL714 Biodegradable Polymeric Materials
\end{flushleft}


\begin{flushleft}
3 Credits (3-0-0)
\end{flushleft}


\begin{flushleft}
Concept of biodegradation; mechanism of biodegradaton; kinetics
\end{flushleft}


\begin{flushleft}
of biodegradation; methods to evaluate biodegradation; bioplastics,
\end{flushleft}


\begin{flushleft}
biodegradable polymers and their synthesis; biodegradable polymer
\end{flushleft}


\begin{flushleft}
blends and composites; technology and processing of biodegradable
\end{flushleft}


\begin{flushleft}
polymers; applications of biodegradable polymers.
\end{flushleft}





\begin{flushleft}
PTL716 Rubber Technology
\end{flushleft}


\begin{flushleft}
3 Credits (3-0-0)
\end{flushleft}


\begin{flushleft}
Rubber and elastomers, compounding and vulcanization, mastication,
\end{flushleft}


\begin{flushleft}
fillers-reinforcing and non-black (loading type). Other compounding
\end{flushleft}


\begin{flushleft}
ingredients; peptizers, vulcanizing agents, accelerators, accelerator
\end{flushleft}


\begin{flushleft}
activator, softeners, anti aging additives, miscellaneous additives,
\end{flushleft}


\begin{flushleft}
colourant, flame retarders, blowing agents, deodorants, abrasive
\end{flushleft}


\begin{flushleft}
retarders etc. Processing and vulcanization tests, vulcanization theory
\end{flushleft}


\begin{flushleft}
and technology, natural and synthetic rubbers, stryene butadiene
\end{flushleft}


\begin{flushleft}
rubbers, polybutadiene and polyisoprene rubbers, ethylene-propylene
\end{flushleft}


\begin{flushleft}
rubber, butyl and halobutyl rubber, nitrile and silicone rubber,
\end{flushleft}


\begin{flushleft}
thermoplastic elastomers, acrylate and fluoro elastomers. .
\end{flushleft}





\begin{flushleft}
PTL718 Polymer Reaction Engineering
\end{flushleft}


\begin{flushleft}
3 Credits (3-0-0)
\end{flushleft}





\begin{flushleft}
PTP709 Polymer Science Laboratory
\end{flushleft}


\begin{flushleft}
2 Credits (0-0-4)
\end{flushleft}


\begin{flushleft}
Experiments: identification of polymers; purification of monomers;
\end{flushleft}


\begin{flushleft}
suspension polymerization of styrene; emulsion polymerization of
\end{flushleft}


\begin{flushleft}
vinyl acetate and butyl acrylate; bulk and solution polymerization
\end{flushleft}


\begin{flushleft}
of methyl methacrylate; preparation and testing of epoxy resins;
\end{flushleft}


\begin{flushleft}
unsaturated polyester resin technology; preparation of nylon 6
\end{flushleft}


\begin{flushleft}
and nylon 10 by interfacial polymerization; copolymerization and
\end{flushleft}


\begin{flushleft}
determination of reactivity ratios; epoxide equivalent; molecular weight
\end{flushleft}


\begin{flushleft}
determination by viscometry and end-group analysis; atom transfer
\end{flushleft}


\begin{flushleft}
radical polymerization of styrene; thermal characterization by DSC
\end{flushleft}


\begin{flushleft}
and TGA; GPC; FTIR and NMR.
\end{flushleft}





\begin{flushleft}
Course covers reaction kinectics in condensation and all types of
\end{flushleft}


\begin{flushleft}
addition polymerisation reactions, prediction of molecular weight for
\end{flushleft}


\begin{flushleft}
polymerisation in different types of reactors, batch and continuous
\end{flushleft}


\begin{flushleft}
processes, the effect of mixing on kinetics and MWD, reactor design.
\end{flushleft}





\begin{flushleft}
PTL720 Polymer Product and Mould Design
\end{flushleft}


\begin{flushleft}
3 Credits (2-0-2)
\end{flushleft}





\begin{flushleft}
Course covers the types of moulds and dies, product and mould
\end{flushleft}


\begin{flushleft}
design, details of construction and manufacturing methods of tools,
\end{flushleft}


\begin{flushleft}
dies and moulds.
\end{flushleft}





316





\begin{flushleft}
\newpage
Polymer Science and Engineering
\end{flushleft}





\begin{flushleft}
PTL722 Polymer Degradation and Stabilization
\end{flushleft}


\begin{flushleft}
3 Credits (3-0-0)
\end{flushleft}





\begin{flushleft}
JPD799 Minor Project
\end{flushleft}


\begin{flushleft}
3 Credits (0-0-6)
\end{flushleft}





\begin{flushleft}
Introduction to degradation, thermal and oxidative degradation;
\end{flushleft}


\begin{flushleft}
radiative, mechanical and chemical degradation; biological
\end{flushleft}


\begin{flushleft}
degradation; degradation pathways for common polymers; methods
\end{flushleft}


\begin{flushleft}
to monitor degradation; mechanical degradation, waste management.
\end{flushleft}





\begin{flushleft}
A project in any area of polymers as decided by the supervisor.
\end{flushleft}





\begin{flushleft}
PTL724 Polymeric Coatings
\end{flushleft}


\begin{flushleft}
3 Credits (3-0-0)
\end{flushleft}


\begin{flushleft}
Introduction and mechanism of adhesion of polymeric coatings on
\end{flushleft}


\begin{flushleft}
various substrates. Solvent based polymeric coatings. Water based
\end{flushleft}


\begin{flushleft}
polymeric coatings. UV and EB curable coatings. 100\% convertible
\end{flushleft}


\begin{flushleft}
coatings. Selection criteria of coatings for various substrates.
\end{flushleft}


\begin{flushleft}
Health, safety hazard and environmental aspects of coatings during
\end{flushleft}


\begin{flushleft}
manufacturing and applications.
\end{flushleft}





\begin{flushleft}
PTL726 Polymeric Nanomaterials and Nanocomposites
\end{flushleft}


\begin{flushleft}
3 Credits (3-0-0)
\end{flushleft}


\begin{flushleft}
The course content include the basic concepts and elements related
\end{flushleft}


\begin{flushleft}
to the understanding of nano structured polymer materials and
\end{flushleft}


\begin{flushleft}
nanocomposites.
\end{flushleft}





\begin{flushleft}
JPS800 Independent Study
\end{flushleft}


\begin{flushleft}
3 Credits (0-3-0)
\end{flushleft}


\begin{flushleft}
The course contents are as defined for the program elective courses
\end{flushleft}


\begin{flushleft}
offered by the Centre.
\end{flushleft}





\begin{flushleft}
JPD801 Major Project Part-I
\end{flushleft}


\begin{flushleft}
6 Credits (0-0-12)
\end{flushleft}


\begin{flushleft}
A project in any area of polymer science and technology.
\end{flushleft}





\begin{flushleft}
JPD802 Major Project Part-II
\end{flushleft}


\begin{flushleft}
12 Credits (0-0-24)
\end{flushleft}


\begin{flushleft}
A project in any area of polymer science and technology.
\end{flushleft}





317





\begin{flushleft}
\newpage
Centre for Rural Development and Technology
\end{flushleft}


\begin{flushleft}
RDL700 Biomass Production
\end{flushleft}


\begin{flushleft}
3 Credits (3-0-0)
\end{flushleft}


\begin{flushleft}
Introduction to biomass and biomass classification.	
\end{flushleft}


\begin{flushleft}
Phytobiomass : Primary production-photosynthesis, measurement of
\end{flushleft}


\begin{flushleft}
productivity and statistical analysis of data. Plant's nutrient cycles.
\end{flushleft}


\begin{flushleft}
Plant improvements-Tissue culture and other vegetative methods,
\end{flushleft}


\begin{flushleft}
seed technology and nursery raising. Biofertilizers., bioinoculants
\end{flushleft}


\begin{flushleft}
and biopesticides-Organic manures., nitrogen fixers, phosphorus
\end{flushleft}


\begin{flushleft}
solubilizers and organic matter decomposers, allelopathy, interactions
\end{flushleft}


\begin{flushleft}
among micro and macroflora and biological equilibrium. Plantations
\end{flushleft}


\begin{flushleft}
and cropping pattern agroforestry models, plantations crops,
\end{flushleft}


\begin{flushleft}
tuber crops, petro crops, forage crops and grasses. Soil and water
\end{flushleft}


\begin{flushleft}
conservation in farm, grassland and forest management. Aquatic
\end{flushleft}


\begin{flushleft}
Phytobiomass-Floating plants, submerged plants and potential aquatic
\end{flushleft}


\begin{flushleft}
algal biomass. Animal biomass : Cow, buffalo, goats, sheep and pigs.
\end{flushleft}


\begin{flushleft}
Fisheries and bee keeping.
\end{flushleft}





\begin{flushleft}
RDL701 Rural Industrialisation: Policies, Programmes
\end{flushleft}


\begin{flushleft}
and Cases
\end{flushleft}


\begin{flushleft}
3 Credits (3-0-0)
\end{flushleft}





\begin{flushleft}
Operation and Uses, Biogas Technology, Biogas production using
\end{flushleft}


\begin{flushleft}
various substrates including MSW and industrial wastes, Types of
\end{flushleft}


\begin{flushleft}
digesters and feed materials, Biogas power generation, biogas
\end{flushleft}


\begin{flushleft}
enrichment and bottling technology, Entrepreneurship avenues in
\end{flushleft}


\begin{flushleft}
Biogas sector, Biogas to Natural gas -- BBG technology, Bio diesel:
\end{flushleft}


\begin{flushleft}
potential and scope in India, Bio-diesel production technology:
\end{flushleft}


\begin{flushleft}
Uses and Advantages, Ethanol as Alternate fuel, Ethanol production
\end{flushleft}


\begin{flushleft}
Technologies, Uses and Advantages Problems, Cook Stoves,
\end{flushleft}


\begin{flushleft}
Multifuel and biomass cook stoves, improved chulhas, Micro Hydel
\end{flushleft}


\begin{flushleft}
: Site characterization, Hydro potential estimation, Micro TurbinesClassification, design, sizing analyses, Hydel power utilities, Techno
\end{flushleft}


\begin{flushleft}
economic feasibility and social issues, Animal power: Application
\end{flushleft}


\begin{flushleft}
and power generation, Solar Energy Technologies, Solar Pond, Solar
\end{flushleft}


\begin{flushleft}
Photovoltaic, Concept of Hybrid energy system: Value proposition
\end{flushleft}


\begin{flushleft}
and benefits, Creating renewable energy based livelihood and micro
\end{flushleft}


\begin{flushleft}
enterprises Integrated Rural Energy Planning : Objectives, Technical
\end{flushleft}


\begin{flushleft}
options Benefits, Financial Incentives.
\end{flushleft}





\begin{flushleft}
RDL724 Technologies for Water and Waste Management
\end{flushleft}


\begin{flushleft}
3 Credits (2-0-2)
\end{flushleft}





\begin{flushleft}
Background: Rural Industrialisation, India's rural poverty and possible
\end{flushleft}


\begin{flushleft}
solutions, Rural Industrialisation during planned era. Farm \& Non-Farm
\end{flushleft}


\begin{flushleft}
Sector Synergy: Lessons from Asian experience, Rural Industrialisation
\end{flushleft}


\begin{flushleft}
in China: Township \& village enterprises. Rural transformation
\end{flushleft}


\begin{flushleft}
through decentralized technologies,Sustainable Livelihoods:
\end{flushleft}


\begin{flushleft}
Participatory Management Approach, Appropriate strategy for Rural
\end{flushleft}


\begin{flushleft}
Industrialization, Policies for Rural Industrialisation Entrepreneurship,
\end{flushleft}


\begin{flushleft}
Development for Rural Youth, Women and appropriate Technology in
\end{flushleft}


\begin{flushleft}
Rural Industrialization, Industrialization of rural areas around urban
\end{flushleft}


\begin{flushleft}
centres, Industrialization in tribal area, Role of Govt. \& Financial
\end{flushleft}


\begin{flushleft}
Institutions in Rural Industrialization, Role and Impact of District
\end{flushleft}


\begin{flushleft}
Industries Centres in Rural Industrialization,Gramodaya Scheme and
\end{flushleft}


\begin{flushleft}
rural industrialization, Development of Handloom Industry, Growth
\end{flushleft}


\begin{flushleft}
of production \& employment in KVIs in India, Rural Industrialisation
\end{flushleft}


\begin{flushleft}
through Artisanal industry, Rural Industrialisation: Case Studies,
\end{flushleft}


\begin{flushleft}
Industrialisation of a drought-prone district: Grass root level planning,
\end{flushleft}


\begin{flushleft}
PURA Model of Rural Development, Some successful case Studies.
\end{flushleft}





\begin{flushleft}
Water and wastes: General considerations, Role of water in life, Water
\end{flushleft}


\begin{flushleft}
crisis \& causes, Concept of waste, Solid wastes \& industrial effluents,
\end{flushleft}


\begin{flushleft}
Hazardous and toxic wastes, Natural cycles for zero waste systems,
\end{flushleft}


\begin{flushleft}
Eco sanitation, Water resources and management, Rainwater, runoff
\end{flushleft}


\begin{flushleft}
and ground water, Rainwater harvesting, Water storage and lifting
\end{flushleft}


\begin{flushleft}
devices, Processes for degrading waste, Role of microbes, earthworms,
\end{flushleft}


\begin{flushleft}
Anaerobic Digestion, Aerobic processes, hermo-chemical pathways,
\end{flushleft}


\begin{flushleft}
Water and waste in the domestic sector, Drinking water and nonpotable uses, Domestic wastewater quality and recycling options,
\end{flushleft}


\begin{flushleft}
Domestic solid waste management, Micro enterprises for waste \&
\end{flushleft}


\begin{flushleft}
water treatment, Agricultural systems, Irrigation requirements and
\end{flushleft}


\begin{flushleft}
water audit, water conservation measures, Composting and its
\end{flushleft}


\begin{flushleft}
application, Water management in industries, Water requirement
\end{flushleft}


\begin{flushleft}
industries, Environmental regulations, wastewater treatment and
\end{flushleft}


\begin{flushleft}
recycling in rural industries.
\end{flushleft}





\begin{flushleft}
RDL705 Rural Resources and Livelihoods
\end{flushleft}


\begin{flushleft}
3 Credits (2-0-2)
\end{flushleft}





\begin{flushleft}
Herbal, Medicinal and Aromatic plants of India: Overview and Uses
\end{flushleft}


\begin{flushleft}
Ayurveda, Siddha, Homeopathy, Unani \& Tribal systems of Medicine,
\end{flushleft}


\begin{flushleft}
Role of Traditional Medicine in Primary Health Care, Identification
\end{flushleft}


\begin{flushleft}
of Medicinal and Aromatic plants,Classification of Medicinal plants,
\end{flushleft}


\begin{flushleft}
Pharmacology and Phytochemistry, Medical Bio-prospecting and
\end{flushleft}


\begin{flushleft}
Chemo prospecting, Biomarkers, Active principle and Phytomedicine,
\end{flushleft}


\begin{flushleft}
Cultivation, Harvesting and Storage of Medicinal and Aromatic plants:
\end{flushleft}


\begin{flushleft}
Organic farming of Medicinal and Aromatic Plants, Good Agriculture
\end{flushleft}


\begin{flushleft}
Practice, Post Harvest Processing of Medicinal and aromatic Plants,
\end{flushleft}


\begin{flushleft}
Cleaning and Washing, Drying, Grinding, Processing of Medicinal
\end{flushleft}


\begin{flushleft}
and Aromatic plants -- Extraction, Purification of Active Principle/
\end{flushleft}


\begin{flushleft}
Phytomedicine -- Distillation, Herbal food formulation, Herbal cosmetics
\end{flushleft}


\begin{flushleft}
and cosmochemicals, Nutraceuticals, Mosquito control Products,
\end{flushleft}


\begin{flushleft}
Aromatheraphy, Herbal Veterinary medicine, Natural Dyes and Colours,
\end{flushleft}


\begin{flushleft}
Quality Control and Analysis.
\end{flushleft}





\begin{flushleft}
Overview of different Livelihood Systems, Socio-economic, cultural
\end{flushleft}


\begin{flushleft}
and historic perspectives, Linkages between sustainable livelihood
\end{flushleft}


\begin{flushleft}
systems \& development, Issues of subsistence and survival, challenges
\end{flushleft}


\begin{flushleft}
and threats, livelihood, Impact of globalization on livelihood systems,
\end{flushleft}


\begin{flushleft}
Formal and informal sector livelihood sources, Issue of Women,
\end{flushleft}


\begin{flushleft}
Agro-based/Post Harvest Technology based Livelihoods, Problems
\end{flushleft}


\begin{flushleft}
and challenges for agro-based small enterprises, Natural Resources
\end{flushleft}


\begin{flushleft}
based livelihoods, nature dependency, Scope, challenge for survival
\end{flushleft}


\begin{flushleft}
\& enhancement of Natural Resource based livelihoods, Craft-based
\end{flushleft}


\begin{flushleft}
livelihoods, Challenge/Problems of traditional crafts, Need and strategy
\end{flushleft}


\begin{flushleft}
for preserving/revival of crafts and craftsmen, New product \& service
\end{flushleft}


\begin{flushleft}
based livelihoods, SWOT for survival \& growth, Sustainable Resource
\end{flushleft}


\begin{flushleft}
use and Livelihoods, Role of Continuing Education, Skill Development
\end{flushleft}


\begin{flushleft}
\& EDP, Enterprise Management, SHGs, Cooperatives, Microenterprises
\end{flushleft}


\begin{flushleft}
Identifying entrepreneurial opportunities \& mkt potential, Up scaling
\end{flushleft}


\begin{flushleft}
Microenterprises to SMEs -- Issues \& Perspectives, Importance and
\end{flushleft}


\begin{flushleft}
Scope of Training for Development, Goals for designing training
\end{flushleft}


\begin{flushleft}
programmes for development Self Development, Organisational
\end{flushleft}


\begin{flushleft}
development, Team Building, Skill Training, Technology Transfer etc.,
\end{flushleft}


\begin{flushleft}
Types and Methods of training \& learning, learning paradigms, Training
\end{flushleft}


\begin{flushleft}
:Strategy and Designs, Need Assessment Training: Planning, Methods
\end{flushleft}


\begin{flushleft}
and interaction styles, Evaluation: Types, process, components,
\end{flushleft}


\begin{flushleft}
methods, techniques, Framework \& indicators for evaluating Training
\end{flushleft}


\begin{flushleft}
Programmes Post Training factors, Tourism \& Livelihoods, NTEP
\end{flushleft}


\begin{flushleft}
based livelihood.
\end{flushleft}


\begin{flushleft}
Field projects related to Natural resource based livelihoods.
\end{flushleft}





\begin{flushleft}
RDL726 Herbal, Medicinal and Aromatic Products
\end{flushleft}


\begin{flushleft}
3 Credits (2-0-2)
\end{flushleft}





\begin{flushleft}
Practical and Project related to Herbal, Medicinal and Aromatic
\end{flushleft}


\begin{flushleft}
products.
\end{flushleft}





\begin{flushleft}
RDL730 Technology Alternatives for Rural
\end{flushleft}


\begin{flushleft}
Development
\end{flushleft}


\begin{flushleft}
3 Credits (3-0-0)
\end{flushleft}


\begin{flushleft}
Concept of technologies appropriate for Rural India. Social, economic
\end{flushleft}


\begin{flushleft}
and environmental considerations. Appropriate technology for energy,
\end{flushleft}


\begin{flushleft}
agriculture, housing, textiles, water-supply and sanitation, health care,
\end{flushleft}


\begin{flushleft}
transport and small-scale industries. An integrated approach to the
\end{flushleft}


\begin{flushleft}
use of alternate technologies. Issues of technology transfer.
\end{flushleft}





\begin{flushleft}
RDL740 Technology for Utilization of Wastelands and
\end{flushleft}


\begin{flushleft}
Weeds
\end{flushleft}


\begin{flushleft}
3 Credits (3-0-0)
\end{flushleft}





\begin{flushleft}
RDL722 Rural Energy Systems
\end{flushleft}


\begin{flushleft}
3 Credits (2-0-2)
\end{flushleft}


\begin{flushleft}
Biomass based energy systems, Pyrolysis : Concept, Types, Technology
\end{flushleft}


\begin{flushleft}
\& Waste Management, Gasification: Concept, Technology, Design,
\end{flushleft}





\begin{flushleft}
Land as a parameter in rural development. Wastelands and importance
\end{flushleft}


\begin{flushleft}
of using them. Biomass growth on various types of lands. Introduction
\end{flushleft}


\begin{flushleft}
to plant taxonomy, under-utilized terrestrial plants and aquatic weeds,
\end{flushleft}





318





\begin{flushleft}
\newpage
Rural Development and Technology
\end{flushleft}





\begin{flushleft}
flora of tropics, arid lands and hilly areas. Constituents of biomass,
\end{flushleft}


\begin{flushleft}
biochemical and chemical conversion processes.
\end{flushleft}


\begin{flushleft}
Applications of biomass as unconventional plant-based source for
\end{flushleft}


\begin{flushleft}
food, cattle feed, chemicals, fibres, construction materials and energy.
\end{flushleft}


\begin{flushleft}
An integrated technological approach to biomass and wasteland
\end{flushleft}


\begin{flushleft}
utilization. Possible ecological effects.
\end{flushleft}





\begin{flushleft}
RDD750 Minor Project: Intensive Study on Topics of
\end{flushleft}


\begin{flushleft}
Specific Interest
\end{flushleft}


\begin{flushleft}
3 Credits (0-0-6)
\end{flushleft}


\begin{flushleft}
Project work related to any topics of interest within the specified
\end{flushleft}


\begin{flushleft}
time frame.
\end{flushleft}





\begin{flushleft}
RDP750 Biomass Laboratory
\end{flushleft}


\begin{flushleft}
3 Credits (0-0-6)
\end{flushleft}


\begin{flushleft}
Soil and Water analysis for Biomass Production : Soil Sampling from a
\end{flushleft}


\begin{flushleft}
plot/field and soil analysis for its texture, pH. EC. C.N.P and K. Water
\end{flushleft}


\begin{flushleft}
analysis : TDS, Alkalinity, Total Hardness, EC and pH.
\end{flushleft}


\begin{flushleft}
Soil Microflora and Root Association : Isolation and culturing of nitrogen
\end{flushleft}


\begin{flushleft}
fixers (Rhizobium. Azotobacter, Azospirillum and blue green algae).
\end{flushleft}


\begin{flushleft}
ecto and endomycorrhizal fungi. Measurement of total microbial
\end{flushleft}


\begin{flushleft}
biomass in soil and respiration rate of microbes. Bacterial and fungal
\end{flushleft}


\begin{flushleft}
root infection.
\end{flushleft}


\begin{flushleft}
Biomass Production and Recycling : Micropropagation and other
\end{flushleft}


\begin{flushleft}
vegetative techniques for biomass production. Seed treatment. Seed
\end{flushleft}


\begin{flushleft}
germination and nursery raising. Vermiculturing and Vermicomposting,
\end{flushleft}


\begin{flushleft}
mushroom culturing and spawn production, silkworm rearing.
\end{flushleft}


\begin{flushleft}
Bioinoculants for rapid composting.
\end{flushleft}


\begin{flushleft}
Compost Analysis : C.N.P.K. cellulose, hemicellulose, lignin, humus
\end{flushleft}


\begin{flushleft}
and its fractions. Physico-chemicial properties of biomass.
\end{flushleft}





\begin{flushleft}
RDL760 Food Quality and Safety
\end{flushleft}


\begin{flushleft}
3 Credits (3-0-0)
\end{flushleft}





\begin{flushleft}
Concept of Holistic Health, Holistic Food, Food Quality \& Safety. Food
\end{flushleft}


\begin{flushleft}
quality parameters and standards,(Natural and chemical preservatives
\end{flushleft}


\begin{flushleft}
\& colors, toxins, pesticides, pathogens etc. Nutrients (macro and
\end{flushleft}


\begin{flushleft}
micro), shelf life, seasonal food and diversity 'satvik' characteristics),
\end{flushleft}


\begin{flushleft}
Food processing industries-Current Status and Policy guidelines, Multi
\end{flushleft}


\begin{flushleft}
residue analysis and mycotoxin contamination in food, Processing
\end{flushleft}


\begin{flushleft}
techniques for enhancing bioavailability of micronutrients, Minimizing
\end{flushleft}


\begin{flushleft}
pesticide residue and mycotoxins in food products, Organic food:
\end{flushleft}


\begin{flushleft}
quality control and export potential, APEDA and IFOAM Certification
\end{flushleft}


\begin{flushleft}
`BIS, MRL's under India conditions etc. Policy and regulatory safe
\end{flushleft}


\begin{flushleft}
guards, Food fortification and Nutraceuticals, Traditional as well
\end{flushleft}


\begin{flushleft}
as modern system, Botanicalpesticides for stored grain protection,
\end{flushleft}


\begin{flushleft}
Major storage pests and their life cycle, traditional system and
\end{flushleft}


\begin{flushleft}
their limitations, Traditional systems and their limitations (storage
\end{flushleft}


\begin{flushleft}
structures, pest control measures etc.) Innovations, Village cluster
\end{flushleft}


\begin{flushleft}
Grain storage model for Rural Entrepreneurship, Enhancing shelf life
\end{flushleft}


\begin{flushleft}
of G-K products, grain flour, raw milk, fruits and vegetables, bamboo
\end{flushleft}


\begin{flushleft}
shoot mushroom etc., Equipments \& machinery for food processing
\end{flushleft}


\begin{flushleft}
and preservation small scale food outlets (vendors), SHG, WTP and
\end{flushleft}


\begin{flushleft}
quality control (case study).
\end{flushleft}





\begin{flushleft}
RDL801 Successful Grassroot Organisations
\end{flushleft}


\begin{flushleft}
3 Credits (2-0-2)
\end{flushleft}





\begin{flushleft}
T h e D o m a i n a n d C h a l l e n g e s o f A g r i c u l t u ra l a n d R u ra l
\end{flushleft}


\begin{flushleft}
Development,Participatory Approaches to agricultural and Rural
\end{flushleft}


\begin{flushleft}
Development,Participatory approach in the irrigation sector,Learning
\end{flushleft}


\begin{flushleft}
process and assisted-Self-Reliance, Initiation \& Leadership, CASE
\end{flushleft}


\begin{flushleft}
STUDIES-Set I, The AMUL Dairy Cooperatives, The Grameen Bank
\end{flushleft}


\begin{flushleft}
Story: Rural Credit in Bangladesh, CASE STUDIES - Set II SEWA: Women
\end{flushleft}


\begin{flushleft}
in movement, The Bangladesh Rural Advancement Committee, CASE
\end{flushleft}


\begin{flushleft}
STUDIES - Set III Participatory Watershed Development in Rajasthan,
\end{flushleft}


\begin{flushleft}
The selt-Help Rural Water Supply Program in Malawi, CAMPFIRE
\end{flushleft}


\begin{flushleft}
Program: Community- Based Wildlife Management, Management,
\end{flushleft}


\begin{flushleft}
Planning and Implementation, Technology and Training, Information
\end{flushleft}


\begin{flushleft}
as a Management Tool, Utilization of External Resources, Dealing with
\end{flushleft}


\begin{flushleft}
Government and Politics, Understanding Social Capita from experience
\end{flushleft}


\begin{flushleft}
of participation, Mapping \&Measuring social Capita-Assessment
\end{flushleft}


\begin{flushleft}
of collective action, Understanding civil society as a continuum,
\end{flushleft}


\begin{flushleft}
Measuring empowerment at community \& local level-Analytical
\end{flushleft}


\begin{flushleft}
Issues, Strategies for strengthening organizations at the local level
\end{flushleft}


\begin{flushleft}
success \& Sustainability : Strategic Goals for Planning \& Management.
\end{flushleft}





\begin{flushleft}
RDL803 Informatics and Rural Development
\end{flushleft}


\begin{flushleft}
2 Credits (2-0-2)
\end{flushleft}





\begin{flushleft}
Introduction to ICT and Elements of ICT, Trends in Computing
\end{flushleft}


\begin{flushleft}
\& Telecommunication Technologies, User Devices, Transmission
\end{flushleft}


\begin{flushleft}
Technologies, Wireless Technologies, Emerging Trends and
\end{flushleft}


\begin{flushleft}
Convergence, ICT a Tool for Socio-Economic Development,
\end{flushleft}


\begin{flushleft}
Information Revolution and Information Society, Social Informatics
\end{flushleft}


\begin{flushleft}
: ICT \& RD, Impact of ICT for Development \& Critique, ICT a Tool
\end{flushleft}


\begin{flushleft}
for Rural Empowerment, Techniques for Access to Technology, ICT
\end{flushleft}


\begin{flushleft}
in Agriculture, ICT for Rural Market, ICT in Dairy Mgmt, ICT \& GIS,
\end{flushleft}


\begin{flushleft}
e-Government and Rural Development, What is e-Government and
\end{flushleft}


\begin{flushleft}
e-Governance, Trends in e-Government, Application of e-Gov for
\end{flushleft}


\begin{flushleft}
Service Delivery, Access to Information, Grievance Radressal, Some
\end{flushleft}


\begin{flushleft}
Cases of e-Governance for RD, Implementation of ICT in Rural context,
\end{flushleft}


\begin{flushleft}
Software project management approach, Models of implementation,
\end{flushleft}


\begin{flushleft}
Rural Needs Assessment, People first approach to rural informatics,
\end{flushleft}


\begin{flushleft}
Citizen participation in Design of rural informatics, Role of Community
\end{flushleft}


\begin{flushleft}
based organisations, Challenges to rural informatice, digital divide,
\end{flushleft}


\begin{flushleft}
Gender and other marginalised groups in information society, Issues in
\end{flushleft}


\begin{flushleft}
use of iCT for RD, Critical success factors for e-Gov in Rural Context,
\end{flushleft}


\begin{flushleft}
Global scenarios and national policies, International organization,
\end{flushleft}


\begin{flushleft}
regulatory interventions in computing and telecom industry, Cyberlaws
\end{flushleft}


\begin{flushleft}
\& IT act of India.
\end{flushleft}





\begin{flushleft}
RDL807 Women, Technology and Development
\end{flushleft}


\begin{flushleft}
2 Credits (2-0-2)
\end{flushleft}





\begin{flushleft}
Role of women in development, Gender bias and indicators, Strategies
\end{flushleft}


\begin{flushleft}
for women empowerment, Technology and Women uplift, Women and
\end{flushleft}


\begin{flushleft}
energy, Women and water management, Women and health care,
\end{flushleft}


\begin{flushleft}
Women and holistic health, Women and Vector control, Women in
\end{flushleft}


\begin{flushleft}
farm sector, Women in non-farm sector, Women in the service sector,
\end{flushleft}


\begin{flushleft}
Women and information technology.
\end{flushleft}


\begin{flushleft}
Field projects related to Women, Technology and Development.
\end{flushleft}





319





\begin{flushleft}
\newpage
National Resource Centre for Value Education in Engineering
\end{flushleft}


\begin{flushleft}
VEL700 Human Values and Technology
\end{flushleft}


\begin{flushleft}
3 Credits (2-1-0)
\end{flushleft}





\begin{flushleft}
VEV734 Special Module on Leadership-II
\end{flushleft}


\begin{flushleft}
1 Credit (0.5-0-1)
\end{flushleft}





\begin{flushleft}
Present state of society-achievements and maladies. Notions of
\end{flushleft}


\begin{flushleft}
progress, development and human welfare. Distinction between
\end{flushleft}


\begin{flushleft}
{`}pleasure' and {`}happiness', {`}good' and {`}pleasant', {`}needs' and {`}wants'.
\end{flushleft}


\begin{flushleft}
Are there any universal human values? Complementarity of values and
\end{flushleft}


\begin{flushleft}
knowledge. Typical modern technologies- their impact on mankind.
\end{flushleft}


\begin{flushleft}
Fundamental characteristics of modern technology-their relationship
\end{flushleft}


\begin{flushleft}
to values. Sustainability of modern technology. Values for harmonious
\end{flushleft}


\begin{flushleft}
and sustainable development. Rationales behind universal human
\end{flushleft}


\begin{flushleft}
values. Values and humanistic psychology. Practical difficulties in living
\end{flushleft}


\begin{flushleft}
upto these values typical dilemmas. Need for inner transformation.
\end{flushleft}


\begin{flushleft}
Various approaches towards purification of mind. Concept of holistic
\end{flushleft}


\begin{flushleft}
development and holistic technology. Integrating scientific knowledge
\end{flushleft}


\begin{flushleft}
and human values, understanding engineering ethics.
\end{flushleft}





\begin{flushleft}
(Same as VEV733).
\end{flushleft}





\begin{flushleft}
VEL710 Traditional Knowledge Systems and Values
\end{flushleft}


\begin{flushleft}
3 Credits (3-0-0)
\end{flushleft}


\begin{flushleft}
The values inherent in The Traditional Knowledge Systems (TKS) viz.,
\end{flushleft}


\begin{flushleft}
respect for all life and non-life, respect for diversity; awareness of
\end{flushleft}


\begin{flushleft}
social and ecological impact of activities; self-sufficiency; sustainability,
\end{flushleft}


\begin{flushleft}
socially appropriate, use of local natural and knowledge resources
\end{flushleft}


\begin{flushleft}
viz., decentralized, aesthetically pleasing, wealth distributive etc. It
\end{flushleft}


\begin{flushleft}
would be emphasized that these values are inherently present in the
\end{flushleft}


\begin{flushleft}
framework of traditional knowledge systems and are not add-ons.
\end{flushleft}


\begin{flushleft}
Traditional Technologies which are developed as part of the TKS
\end{flushleft}


\begin{flushleft}
framework are invented and tested in the field, where all environmental
\end{flushleft}


\begin{flushleft}
and social interaction, in particular its effect on other life-forms known
\end{flushleft}


\begin{flushleft}
and unknown are allowed to play their part. This non-fragmented
\end{flushleft}


\begin{flushleft}
approach makes such knowledge holistic and avoids the errors and
\end{flushleft}


\begin{flushleft}
pitfalls when technologies are applied on the basis of incomplete or
\end{flushleft}


\begin{flushleft}
inadequate theories.
\end{flushleft}





\begin{flushleft}
VEV735 Special Module on Sustainability-I
\end{flushleft}


\begin{flushleft}
1 Credit (0.5-0-1)
\end{flushleft}


\begin{flushleft}
This module will consist of courses which address one or more
\end{flushleft}


\begin{flushleft}
aspects of sustainability vis-a-vis the societal value system. The three
\end{flushleft}


\begin{flushleft}
core components of sustainability, viz, sustainable use of resources,
\end{flushleft}


\begin{flushleft}
environmental protection and equity in the society need to be
\end{flushleft}


\begin{flushleft}
understood in-depth with respect to the values of excessive materialism
\end{flushleft}


\begin{flushleft}
and individualism, competitiveness and unlimited economic growth on
\end{flushleft}


\begin{flushleft}
one hand and the values of compassion, fraternity and cooperation
\end{flushleft}


\begin{flushleft}
on the other. The practical sessions will be used to carry out group
\end{flushleft}


\begin{flushleft}
exercises of planning and analysis of real life case studies.
\end{flushleft}





\begin{flushleft}
VEV736 Special Module on Sustainability-II
\end{flushleft}


\begin{flushleft}
1 Credit (0.5-0-1)
\end{flushleft}


\begin{flushleft}
(Same as VEV735).
\end{flushleft}





\begin{flushleft}
VEV737 Special Module on Civilization-I
\end{flushleft}


\begin{flushleft}
1 Credit (0.5-0-1)
\end{flushleft}


\begin{flushleft}
This module will address one or more aspects of development of
\end{flushleft}


\begin{flushleft}
civilizations and promotion of societal peace which have strong linkages
\end{flushleft}


\begin{flushleft}
with the value system of the society. This could include value systems
\end{flushleft}


\begin{flushleft}
reflected in constitutions of different countries, the way a society deals
\end{flushleft}


\begin{flushleft}
with human rights and the like.
\end{flushleft}





\begin{flushleft}
VEV738 Special Module on Civilization-II
\end{flushleft}


\begin{flushleft}
1 Credit (0.5-0-1)
\end{flushleft}


\begin{flushleft}
(Same as VEV737).
\end{flushleft}





\begin{flushleft}
VEV731 Special Module on Inner Development-I
\end{flushleft}


\begin{flushleft}
1 Credit (0.5-0-1)
\end{flushleft}


\begin{flushleft}
This module will primarily consist of courses which address one or more
\end{flushleft}


\begin{flushleft}
aspects of inner development such as comprehensive mindfulness,
\end{flushleft}


\begin{flushleft}
in-depth intellectual understanding of oneself and one's aspirations,
\end{flushleft}


\begin{flushleft}
selfless service etc. These courses are expected to provide a practical
\end{flushleft}


\begin{flushleft}
experience to the students in how small positive changes can be
\end{flushleft}


\begin{flushleft}
brought about in one's inner self through a systematic practice of
\end{flushleft}


\begin{flushleft}
looking within.
\end{flushleft}





\begin{flushleft}
VEV732 Special Module on Inner Development-II
\end{flushleft}


\begin{flushleft}
1 Credit (0.5-0-1)
\end{flushleft}





\begin{flushleft}
VEV739 Special Module on Professional Ethics-I
\end{flushleft}


\begin{flushleft}
1 Credit (0.5-0-1)
\end{flushleft}


\begin{flushleft}
This module will bring out the need for professional ethics as recognised
\end{flushleft}


\begin{flushleft}
by several professional bodies in the world through discussion of
\end{flushleft}


\begin{flushleft}
practical case studies and the underlying tenets of the code of conduct
\end{flushleft}


\begin{flushleft}
of professional bodies. The course will initiate discussion on reasons
\end{flushleft}


\begin{flushleft}
behind deviation from these tenets and the relevance of these tenets
\end{flushleft}


\begin{flushleft}
of professional ethics in the contemporary world.
\end{flushleft}





\begin{flushleft}
VEV740 Special Module on Professional Ethics-II
\end{flushleft}


\begin{flushleft}
1 Credit (0.5-0-1)
\end{flushleft}





\begin{flushleft}
(Same as VEV731).
\end{flushleft}





\begin{flushleft}
(Same as VEV740).
\end{flushleft}





\begin{flushleft}
VEV733 Special Module on Leadership-I
\end{flushleft}


\begin{flushleft}
1 Credit (0.5-0-1)
\end{flushleft}


\begin{flushleft}
This module will address the strong linkages between the personal
\end{flushleft}


\begin{flushleft}
values of an individual and the desirable qualities of a leader. Going
\end{flushleft}


\begin{flushleft}
beyond the theories, it will emphasize on the practical aspect of
\end{flushleft}


\begin{flushleft}
looking within as well as connecting to the outside world and hence
\end{flushleft}


\begin{flushleft}
developing the qualities of a leader.
\end{flushleft}





\begin{flushleft}
VED750 Minor Project
\end{flushleft}


\begin{flushleft}
3 Credits (0-0-6)
\end{flushleft}


\begin{flushleft}
To carry out detailed studies (under the guidance of a faculty member)
\end{flushleft}


\begin{flushleft}
on issues like Science, Technology and Human Values, Engineering
\end{flushleft}


\begin{flushleft}
Ethics, Sustainable Development, Scientific basis of human values etc.
\end{flushleft}





320





\begin{flushleft}
\newpage
Amar Nath and Shashi Khosla School of Information Technology
\end{flushleft}


\begin{flushleft}
SIL765 Networks \& System Security
\end{flushleft}


\begin{flushleft}
4 Credits (3-0-2)
\end{flushleft}


\begin{flushleft}
The goal of this course is to introduce challenges in securing computer
\end{flushleft}


\begin{flushleft}
systems and networks. We will discuss various types of vulnerabilities
\end{flushleft}


\begin{flushleft}
in existing software interfaces, such as buffer overflows, unsafe libc
\end{flushleft}


\begin{flushleft}
functions, filesystem design issues, etc. We will also discuss modernday defenses against attacks exploiting these vulnerabilities. In
\end{flushleft}


\begin{flushleft}
network security, we will discuss security problems in network protocols
\end{flushleft}


\begin{flushleft}
and routing, such as sniffing, denial of service, viruses, worms, etc.
\end{flushleft}


\begin{flushleft}
and defenses against them. The course will involve reading research
\end{flushleft}


\begin{flushleft}
papers on relevant topics, programming assignments, and projects.
\end{flushleft}





\begin{flushleft}
SIL769 Internet Traffic -Measurement, Modeling \&
\end{flushleft}


\begin{flushleft}
Analysis
\end{flushleft}


\begin{flushleft}
4 Credits (3-0-2)
\end{flushleft}


\begin{flushleft}
Internet architecture: overview of TCP/IP protocol stack.
\end{flushleft}


\begin{flushleft}
Mathematics for studying the Internet: Review of basic probability
\end{flushleft}


\begin{flushleft}
and statistics, analytic modeling approaches. Practical issues in
\end{flushleft}


\begin{flushleft}
Internet Measurements: Challenges, tools and techniques for
\end{flushleft}


\begin{flushleft}
measuring performance. Internet Traffic Characterization: Poisson
\end{flushleft}


\begin{flushleft}
models for Internet traffic, self-similarity in network traffic.
\end{flushleft}


\begin{flushleft}
Web Performance: workload characterization, caching, content
\end{flushleft}


\begin{flushleft}
distribution networks. Multimedia Systems: Video-on-Demand,
\end{flushleft}


\begin{flushleft}
IP-TV, Peer-to-Peer file sharing, Peer-to-Peer Streaming. Social
\end{flushleft}


\begin{flushleft}
Networks. Network Security.
\end{flushleft}





\begin{flushleft}
SIL801 Special Topics in Multimedia System
\end{flushleft}


\begin{flushleft}
3 Credits (3-0-0)
\end{flushleft}


\begin{flushleft}
Content of this course, depending upon the teacher, will be focused
\end{flushleft}


\begin{flushleft}
on some aspect(s) of multimedia systems like content based retrieval,
\end{flushleft}


\begin{flushleft}
multimedia communication, compression techniques, speech and
\end{flushleft}


\begin{flushleft}
audio technology, etc.
\end{flushleft}





\begin{flushleft}
SIV861 Information and Comm Technologies for
\end{flushleft}


\begin{flushleft}
Development
\end{flushleft}


\begin{flushleft}
1 Credit (1-0-0)
\end{flushleft}


\begin{flushleft}
Notion of appropriate technology; case studies of ICTD projects
\end{flushleft}


\begin{flushleft}
such as KioskNet, WiLDNet (Wireless Long Distance Networks), AIR
\end{flushleft}


\begin{flushleft}
(Advanced Interactive Radio), Spoken Web, GRINS (Gramin Radio
\end{flushleft}


\begin{flushleft}
Inter Networking System), Digital Green; design principles to be kept
\end{flushleft}


\begin{flushleft}
in mind; evaluation methodologies.
\end{flushleft}





\begin{flushleft}
SIV864 Special Module on Media Processing \&
\end{flushleft}


\begin{flushleft}
Communication
\end{flushleft}


\begin{flushleft}
1 Credit (1-0-0)
\end{flushleft}


\begin{flushleft}
Communication today has rich multimedia contents. Under the varying
\end{flushleft}


\begin{flushleft}
bandwidth attention is required for appropriate processing of the
\end{flushleft}


\begin{flushleft}
media contents satisfying desired quality of service. This course will
\end{flushleft}


\begin{flushleft}
focus on bringing the two broad areas of multimedia processing and
\end{flushleft}


\begin{flushleft}
communication together. In media processing fundamental concepts
\end{flushleft}


\begin{flushleft}
of media processing and compression will be introduced with exposure
\end{flushleft}


\begin{flushleft}
to current techniques and standards. In communication protocols and
\end{flushleft}


\begin{flushleft}
algorithms for both wired and wireless networks will be discussed in
\end{flushleft}


\begin{flushleft}
relation to multimedia communication.
\end{flushleft}





\begin{flushleft}
SIV871 Special Module in Computational Neuroscience
\end{flushleft}


\begin{flushleft}
1 Credit (1-0-0)
\end{flushleft}


\begin{flushleft}
Special module that focuses on research problems of importance in
\end{flushleft}


\begin{flushleft}
this area of Neuroscience from a computational perspective. Specific
\end{flushleft}


\begin{flushleft}
coverage will vary with each offering, and may include project work
\end{flushleft}


\begin{flushleft}
and design / case studies. Topics for each offering of the course will
\end{flushleft}


\begin{flushleft}
be separately listed.
\end{flushleft}





\begin{flushleft}
SID880 Minor Project in Information Technology
\end{flushleft}


\begin{flushleft}
3 Credits (0-0-6)
\end{flushleft}


\begin{flushleft}
SIV889 Special Module in Human Computer Interface
\end{flushleft}


\begin{flushleft}
1 Credit (1-0-0)
\end{flushleft}





\begin{flushleft}
SIL802 Special Topics in Web Based Computing
\end{flushleft}


\begin{flushleft}
3 Credits (3-0-0)
\end{flushleft}


\begin{flushleft}
Content of this course, depending upon the teacher, will be focused
\end{flushleft}


\begin{flushleft}
on some aspect(s) of web based computing like sematic web, web
\end{flushleft}


\begin{flushleft}
based distributed computing, search methods, etc.
\end{flushleft}





\begin{flushleft}
SIV813 Applications of Computer in Medicines
\end{flushleft}


\begin{flushleft}
1 Credit (1-0-0)
\end{flushleft}


\begin{flushleft}
This course will consist of 14 lecture-hours that focus on information
\end{flushleft}


\begin{flushleft}
and communication technologies (ICT) that are being developed
\end{flushleft}


\begin{flushleft}
and used in medical education and clinical practice today. Various
\end{flushleft}


\begin{flushleft}
technologies ranging from computer aided instruction (CAI),
\end{flushleft}


\begin{flushleft}
simulations, and networked applications at one end to electronic
\end{flushleft}


\begin{flushleft}
medical records (EMR), telemedicine, and robotic surgery at the other
\end{flushleft}


\begin{flushleft}
end will be described. The process of research, development, and
\end{flushleft}


\begin{flushleft}
evaluation in the designing and making of these applications and tools
\end{flushleft}


\begin{flushleft}
will be detailed. Writing assignments, creative thinking, and interactive
\end{flushleft}


\begin{flushleft}
discussions will form an integral part of this course.
\end{flushleft}





\begin{flushleft}
Special module that focuses on research problems of importance
\end{flushleft}


\begin{flushleft}
in this area from a computational and design perspective. Specific
\end{flushleft}


\begin{flushleft}
coverage will vary with each offering, and may include project work
\end{flushleft}


\begin{flushleft}
and design / case studies. Topics for each offering of the course will
\end{flushleft}


\begin{flushleft}
be separately listed.
\end{flushleft}





\begin{flushleft}
SID890 Major Project (SIY)
\end{flushleft}


\begin{flushleft}
40 Credits (0-0-80)
\end{flushleft}


\begin{flushleft}
SIV895 Special Module on Intelligent Information
\end{flushleft}


\begin{flushleft}
Processing
\end{flushleft}


\begin{flushleft}
1 Credit (1-0-0)
\end{flushleft}


\begin{flushleft}
This course will focus on presenting conclave of methods which
\end{flushleft}


\begin{flushleft}
are being practiced for intelligent computing -- learning techniques,
\end{flushleft}


\begin{flushleft}
classification methods, embedding intelligence, neural networks, soft
\end{flushleft}


\begin{flushleft}
computing and evolutionally methods. Emphasis will also be given on
\end{flushleft}


\begin{flushleft}
the variety of multidisciplinary applications of such techniques.
\end{flushleft}





\begin{flushleft}
Bharti School of Telecommunication Technology
\end{flushleft}


\begin{flushleft}
and Management
\end{flushleft}


\begin{flushleft}
BSD895 MS Research Project
\end{flushleft}


\begin{flushleft}
36 Credits (0-0-72)
\end{flushleft}





321





\begin{flushleft}
\newpage
Kusuma School of Biological Sciences
\end{flushleft}


\begin{flushleft}
SBL100 Introductory Biology for Engineers
\end{flushleft}


\begin{flushleft}
4 Credits (3-0-2)
\end{flushleft}


\begin{flushleft}
Darwinian evolution \& molecular perspective; Introduction to phylogeny
\end{flushleft}


\begin{flushleft}
- Classification systems in biology and relationships; Cellular assemblies
\end{flushleft}


\begin{flushleft}
-- From single cell to multi-cellular organisms: Geometry, Structure and
\end{flushleft}


\begin{flushleft}
Energetics; Comparing natural vs. humanmade machines; Infection,
\end{flushleft}


\begin{flushleft}
disease and evolution -- synergy and antagonism; Immunology -- An
\end{flushleft}


\begin{flushleft}
example of permutations and combinations in biology; Cancer biology --
\end{flushleft}


\begin{flushleft}
Control and regulation; Stem cells -- Degeneracy in biological systems;
\end{flushleft}


\begin{flushleft}
Engineering designs inspired by biology -- Micro- to Macro- scales.
\end{flushleft}


\begin{flushleft}
Laboratory: Biosafety; Buffers in biology - Measuring microlitres,
\end{flushleft}


\begin{flushleft}
Preparation of standard biological buffers, buffering capacity and pKa
\end{flushleft}


\begin{flushleft}
of buffers, response of cells and plant tissues in different buffering
\end{flushleft}


\begin{flushleft}
conditions; Observing cell surface and intracellular contents using
\end{flushleft}


\begin{flushleft}
light and fluorescence microscopy, measuring cellular motion using
\end{flushleft}


\begin{flushleft}
real-time video microscopy; Measuring and visualizing intracellular
\end{flushleft}


\begin{flushleft}
molecular components - Proteins and Genomic DNA
\end{flushleft}





\begin{flushleft}
SBP200 Introduction to Practical Modern Biology
\end{flushleft}


\begin{flushleft}
2 Credits (0-0-4)
\end{flushleft}


\begin{flushleft}
Pre-requisites: SBL100
\end{flushleft}


\begin{flushleft}
Biosafety lab practices -- use of lab coats, gloves, safety goggles,
\end{flushleft}


\begin{flushleft}
eye wash, shower, chemical and biological waste disposal; Buffers in
\end{flushleft}


\begin{flushleft}
biology-- Preparation of standard biological buffers, buffering capacity
\end{flushleft}


\begin{flushleft}
and pKa of buffers, biomolecules such as enzymes, whole cells and
\end{flushleft}


\begin{flushleft}
plant tissues in different buffering conditions; Observing cell surface
\end{flushleft}


\begin{flushleft}
and intracellular contents using light and fluorescence microscopy,
\end{flushleft}


\begin{flushleft}
{``}autofluorescence'' of cells, real-time video microscopy of motile cells,
\end{flushleft}


\begin{flushleft}
cell growth and division; Plant genomic DNA isolation; Protoplast
\end{flushleft}


\begin{flushleft}
isolation and viability; Computer Modeling-From Genome Sequence
\end{flushleft}


\begin{flushleft}
to Protein Sequence and structure to screening for a {``}Hit'' Molecule.
\end{flushleft}





\begin{flushleft}
SBL201 High-Dimensional Biology
\end{flushleft}


\begin{flushleft}
3 Credits (3-0-0)
\end{flushleft}


\begin{flushleft}
Pre-requisites: SBL100
\end{flushleft}


\begin{flushleft}
Introduction to Genomics, proteomics, Metabolomics \& Cellomics;
\end{flushleft}


\begin{flushleft}
Size vis-\`{a}-vis packaging and replication challenges, Biomolecular
\end{flushleft}


\begin{flushleft}
architecture and assemblies leading to function, Immortal cells
\end{flushleft}


\begin{flushleft}
and aging, Minimalist Genomes \& Designer Genomes; Molecular
\end{flushleft}


\begin{flushleft}
Engines; Proteins as nanobiomachines; network circuits for genome
\end{flushleft}


\begin{flushleft}
organization and proteinprotein interactions, date hubs, party hubs,
\end{flushleft}


\begin{flushleft}
structure-function axioms, Biochemical cycles and feedback loops,
\end{flushleft}


\begin{flushleft}
Omics Applications, forensics, drug targets.
\end{flushleft}





\begin{flushleft}
SBD301 Mini Project
\end{flushleft}


\begin{flushleft}
3 Credits (0-0-6)
\end{flushleft}


\begin{flushleft}
Pre-requisites: SBL100 and EC80
\end{flushleft}


\begin{flushleft}
Systems Biology, Plant Molecular Biology, Bioprospecting, Tissue
\end{flushleft}


\begin{flushleft}
culture and Developmental Biology, Virology, Structural Biology,
\end{flushleft}


\begin{flushleft}
Cell Biophysics, Cellular Signalling, Protein folding and misfolding,
\end{flushleft}


\begin{flushleft}
Computational Biology.
\end{flushleft}





\begin{flushleft}
SBL701 Biometry
\end{flushleft}


\begin{flushleft}
3 Credits (3-0-0)
\end{flushleft}


\begin{flushleft}
Pre-requisites: EC 90
\end{flushleft}


\begin{flushleft}
Probability and Set theory: Application to biological data, Random
\end{flushleft}


\begin{flushleft}
variables: Individuals vs. populations in biological systems,
\end{flushleft}


\begin{flushleft}
Classification of data: {``}Discreteness or Continuity'' in biological
\end{flushleft}


\begin{flushleft}
evolution, Distributions, Descriptive statistics, Inferential statistics,
\end{flushleft}


\begin{flushleft}
Analysis of variance (ANOVA), ANOVA-advanced concepts, Power
\end{flushleft}


\begin{flushleft}
analysis of variance, Regression and Correlation, Count/Frequency
\end{flushleft}


\begin{flushleft}
data. MATLAB based assignment activities will be designed for data
\end{flushleft}


\begin{flushleft}
simulation and analysis corresponding to the covered lecture material.
\end{flushleft}





\begin{flushleft}
SBL702 Systems Biology
\end{flushleft}


\begin{flushleft}
3 Credits (3-0-0)
\end{flushleft}


\begin{flushleft}
Pre-requisites: EC 90
\end{flushleft}


\begin{flushleft}
Overview and history of systems biology; Basic elements of molecular
\end{flushleft}





\begin{flushleft}
biology -- DNA and protein, the genetic code, transfer RNA and
\end{flushleft}


\begin{flushleft}
protein sequences and control of gene expression; Signal transduction
\end{flushleft}


\begin{flushleft}
-- signaling pathways and cascades, information processing and
\end{flushleft}


\begin{flushleft}
transmission, pathway dynamics; Trees and sequences -- graphs,
\end{flushleft}


\begin{flushleft}
connectivity, trees, flows in networks; Elements of process control --
\end{flushleft}


\begin{flushleft}
feedback, feed forward and cascade control, dynamics of closed loops,
\end{flushleft}


\begin{flushleft}
analogies with control of gene expression; Examples of transcription
\end{flushleft}


\begin{flushleft}
networks, determination of simple motifs that are repeated in genetics;
\end{flushleft}


\begin{flushleft}
guidelines for analyzing genetics circuits, layouts and representations,
\end{flushleft}


\begin{flushleft}
circuit dynamics; modeling, simulation and prediction of cellular events,
\end{flushleft}


\begin{flushleft}
micro-macro relations; Experimental methods in systems biology,
\end{flushleft}


\begin{flushleft}
creation of directed information, existing databases; platforms and
\end{flushleft}


\begin{flushleft}
applications; Case studies from literature -- circadian clock, metabolic
\end{flushleft}


\begin{flushleft}
networks, gene circuit design; New frontiers.
\end{flushleft}





\begin{flushleft}
SBL703 Advanced Cell Biology
\end{flushleft}


\begin{flushleft}
3 Credits (3-0-0)
\end{flushleft}


\begin{flushleft}
Pre-requisites: EC 90
\end{flushleft}


\begin{flushleft}
Chemistry of biological structure, function and information flow,
\end{flushleft}


\begin{flushleft}
Cellular compartmentalization and molecular organization of
\end{flushleft}


\begin{flushleft}
organelles, Properties and growth of HeLa, Jurkat, SF9 etc.; De-novo
\end{flushleft}


\begin{flushleft}
synthesis of organelles versus templated replication, Microtubule,
\end{flushleft}


\begin{flushleft}
microfilament and intermediate filaments; Transport of biomolecules;
\end{flushleft}


\begin{flushleft}
Nuclear structure, chromatin packing and transport; Microtubule,
\end{flushleft}


\begin{flushleft}
action and filament based motile systems, cell-cell recognition and
\end{flushleft}


\begin{flushleft}
adhesion; Fluorescence, phase contrast, confocal and AFM; Molecular
\end{flushleft}


\begin{flushleft}
basis of cancer, oncogenes and tumor suppressor genes; cell growth
\end{flushleft}


\begin{flushleft}
and differentiation.
\end{flushleft}





\begin{flushleft}
SBL704 Human Virology
\end{flushleft}


\begin{flushleft}
3 Credits (3-0-0)
\end{flushleft}


\begin{flushleft}
Pre-requisites: EC 90
\end{flushleft}


\begin{flushleft}
Introduction, overview and history of medical Virology; Virus
\end{flushleft}


\begin{flushleft}
structure, classification and replication -- symmetries, replication,
\end{flushleft}


\begin{flushleft}
maturation and release; Principles of viral pathogenensis- entry, cell
\end{flushleft}


\begin{flushleft}
tropism. Cellular pathogensis, clearance and persistence; Respiratory
\end{flushleft}


\begin{flushleft}
viruses -- Influenza, paramyxoviruses, adenonviruses, SARS, RSV;
\end{flushleft}


\begin{flushleft}
Viral gastroenteritis -- causative agents, epidemiology; Hepatitus
\end{flushleft}


\begin{flushleft}
viruses -- food borne and blood borne; Herpes viruses -- infections
\end{flushleft}


\begin{flushleft}
in immunocompetent and immunocompromised individuals, latency;
\end{flushleft}


\begin{flushleft}
Enteroviruses -- Polio, ECHO, coxsackie viruses; Congenital viral
\end{flushleft}


\begin{flushleft}
infections -- effects on foetus, prevention; Retroviruses -- HIV, AIDS;
\end{flushleft}


\begin{flushleft}
Arboviruses and Viral zoonoses -- arthropod vectors, vertebrate hosts,
\end{flushleft}


\begin{flushleft}
transmission cycles, rabies and viral haemorrhagic fevers; Tumour
\end{flushleft}


\begin{flushleft}
viruses -- oncogenic mechanisms of viruses; Strategies for control
\end{flushleft}


\begin{flushleft}
of viral infection -- active and passive immunoprophylaxis, antiviral
\end{flushleft}


\begin{flushleft}
agents; Safety precautions -- lab acquired infections, hazard groups
\end{flushleft}


\begin{flushleft}
and containment levels; Case studies from literature, evolving and
\end{flushleft}


\begin{flushleft}
emerging areas of interest.
\end{flushleft}





\begin{flushleft}
SBL705 Biology of Proteins
\end{flushleft}


\begin{flushleft}
3 Credits (3-0-0)
\end{flushleft}


\begin{flushleft}
Pre-requisites: EC 90
\end{flushleft}


\begin{flushleft}
Over-view of protein preparation, modification, maturation; proteinprotein interactions in cells, Heat shock proteins and their structure
\end{flushleft}


\begin{flushleft}
and functions in cells, protein mimicry, assisted protein maturation
\end{flushleft}


\begin{flushleft}
processes in cells, Protein trafficking and dislocation, protein secretion
\end{flushleft}


\begin{flushleft}
from cell, kinetics and thermodynamics of protein folding and unfolding
\end{flushleft}


\begin{flushleft}
reactions, biomarker discovery, ribosome profiling.
\end{flushleft}





\begin{flushleft}
SBL706 Biologics
\end{flushleft}


\begin{flushleft}
3 Credits (3-0-0)
\end{flushleft}


\begin{flushleft}
Pre-requisites: EC 90 and BEL 110 or CYL 110 or CYL 120 or
\end{flushleft}


\begin{flushleft}
Equivalent
\end{flushleft}


\begin{flushleft}
Definition and classification of biologics, Biologics, Biopharmaceuticals
\end{flushleft}


\begin{flushleft}
Vs. conventional drugs, Biosimilars, Role of rDNA technologies,
\end{flushleft}


\begin{flushleft}
transgenics (animal and plant), obligonucleotides, peptide, PNAs
\end{flushleft}


\begin{flushleft}
mediated therapeutics, drug delivery systems (lipids, cell penetrating
\end{flushleft}


\begin{flushleft}
peptides), vaccine, monoclonal antibodies produced by and in the
\end{flushleft}


\begin{flushleft}
living organisms, nanobiopharmaceutics, overview of the technologies
\end{flushleft}





322





\begin{flushleft}
\newpage
Biological Sciences
\end{flushleft}





\begin{flushleft}
employed for identification, characterization and production of
\end{flushleft}


\begin{flushleft}
biologics, Bioprospecting for novel drug discovery and development,
\end{flushleft}


\begin{flushleft}
Gene prospecting, plant bioprospecting, marine bioprospecting
\end{flushleft}


\begin{flushleft}
Phytomedicines, plant secondary metabolites, herbal drugs, edible
\end{flushleft}


\begin{flushleft}
vaccines, Bioresource based alternative medicine systems - AYUSH,
\end{flushleft}


\begin{flushleft}
Southeast Asian medicine system, PIC, MAT and ABS, assessing the
\end{flushleft}


\begin{flushleft}
role of biomimetics, system biology, synthetic biology in biologic
\end{flushleft}


\begin{flushleft}
production, GMPs, legislations, Safety Regulations associated with
\end{flushleft}


\begin{flushleft}
biologics in biopharmaceuticals.
\end{flushleft}





\begin{flushleft}
SBL707 Bacterial Pathogenesis
\end{flushleft}


\begin{flushleft}
3 Credits (3-0-0)
\end{flushleft}


\begin{flushleft}
Pre-requisites: EC 90 and BEL110 or CYL110 or CYL120 or
\end{flushleft}


\begin{flushleft}
Equivalent
\end{flushleft}


\begin{flushleft}
Common features of bacterial pathogens, structural features, capsules
\end{flushleft}


\begin{flushleft}
and cell walls, Pathogenicity islands, types of toxins produced, effect
\end{flushleft}


\begin{flushleft}
of toxins on host cells, secretion systems, production and function of
\end{flushleft}


\begin{flushleft}
adhesions, attachment to host cells, mechanisms of cellular invasion,
\end{flushleft}


\begin{flushleft}
extracellular and intracellular invasion, intracellular survival and
\end{flushleft}


\begin{flushleft}
multiplication, virulence factors, mechanisms of antibiotic resistance,
\end{flushleft}


\begin{flushleft}
interaction with the host immune system- innate and adaptive,
\end{flushleft}


\begin{flushleft}
evasion strategies, Immunocompromised individuals and opportunistic
\end{flushleft}


\begin{flushleft}
pathogens, specific examples such as Listeria, Mycobacterium, Shigella,
\end{flushleft}


\begin{flushleft}
Yersinia etc., strategies for prevention and cure, drug designing and
\end{flushleft}


\begin{flushleft}
scope for future studies, emerging infectious bacterial pathogens.
\end{flushleft}





\begin{flushleft}
SBL708 Epigenetics in Human Health and Disease
\end{flushleft}


\begin{flushleft}
3 Credits (3-0-0)
\end{flushleft}


\begin{flushleft}
Pre-requisites: EC 90 and BEL 110 or CYL 110 or CYL 120 or
\end{flushleft}


\begin{flushleft}
Equivalent
\end{flushleft}


\begin{flushleft}
Introduction -- overview of epigenetics in human health and disease;
\end{flushleft}


\begin{flushleft}
Epigenetic mechanisms -- basic mechanisms: DNA methylation
\end{flushleft}


\begin{flushleft}
and genome imprinting --role of DNA methylation; Epigenetics in
\end{flushleft}


\begin{flushleft}
cancer Biology -- global and region specific changes and effects
\end{flushleft}


\begin{flushleft}
on transcription; DNA methylation and repeat instability diseases;
\end{flushleft}


\begin{flushleft}
Epigenetic reprogramming and role of DNA methylation in mammalian
\end{flushleft}


\begin{flushleft}
development --role in embryogenesis; Epigenetics in pluriprotency and
\end{flushleft}


\begin{flushleft}
differentiation of embryonic stems cells; MicroRNA in carcinogenesis --
\end{flushleft}


\begin{flushleft}
mechanisms and potential therapeutic options; Epigenetic regulation
\end{flushleft}


\begin{flushleft}
of viruses by the host --role in pathogenesis; methods in epigeneticsmethylation patterns and histone modifications; Case studies from
\end{flushleft}


\begin{flushleft}
literature, evolving and emerging areas of interest.
\end{flushleft}





\begin{flushleft}
SBL709 Marine Bioprospecting
\end{flushleft}


\begin{flushleft}
3 Credits (3-0-0)
\end{flushleft}


\begin{flushleft}
Pre-requisites: EC90 and BEL110 or CYL110 or CYL120 or
\end{flushleft}


\begin{flushleft}
Equivalent
\end{flushleft}


\begin{flushleft}
Significance, Overview of Marine Bioresources, Marine Biomedical
\end{flushleft}


\begin{flushleft}
Research and Development; Drug discovery continuum in Marine
\end{flushleft}


\begin{flushleft}
Biotechnology, Omics, Biosensors, Biomaterials, Bionanotechnology,
\end{flushleft}


\begin{flushleft}
Bioactive compounds, Nutraceuticals, Pharmaceuticals, Cosmeceuticals,
\end{flushleft}


\begin{flushleft}
Novel Technologies in Marine Research, Sustainable development, Case
\end{flushleft}


\begin{flushleft}
studies, Emerging issues and challenges; IPRs, Marine Biodiversity
\end{flushleft}


\begin{flushleft}
and Traditional Knowledge (medicine).
\end{flushleft}





\begin{flushleft}
SBL710 Chemical Biology
\end{flushleft}


\begin{flushleft}
3 Credits (3-0-0)
\end{flushleft}


\begin{flushleft}
Pre-requisites: EC90 and BEL110 or CYL110 or CYL120 or
\end{flushleft}


\begin{flushleft}
Equivalent
\end{flushleft}


\begin{flushleft}
Chemical modifications of proteins, protein and nucleic acid
\end{flushleft}


\begin{flushleft}
immobilization; The Organic Chemistry of Biological Pathways; cross
\end{flushleft}


\begin{flushleft}
linking in biomolecules; Physical Chemistry of proteins; fluorescent
\end{flushleft}


\begin{flushleft}
labeling of proteins and nucleic acids, sequencing of proteins and
\end{flushleft}


\begin{flushleft}
amino acids, radio labeling of proteins and nucleic acids, chemistry
\end{flushleft}


\begin{flushleft}
of glycosylation, phosphorylation, sulphonylation, methylation, of
\end{flushleft}


\begin{flushleft}
proteins and nucleic acids, non-ribosomal peptide synthesis, nano
\end{flushleft}


\begin{flushleft}
particles mediated monitoring of protein conformational transition,
\end{flushleft}


\begin{flushleft}
folding and unfolding processes; surface properties of proteins and
\end{flushleft}


\begin{flushleft}
subsequent implications in cellular processes, solubility of proteins,
\end{flushleft}





\begin{flushleft}
physical basis for biomolecular structure formation, environmental
\end{flushleft}


\begin{flushleft}
effects on structure-function of biomolecules, chemistry of enzymatic
\end{flushleft}


\begin{flushleft}
digestion of nucleotides and proteins, role of metal ions in the
\end{flushleft}


\begin{flushleft}
cellular function, metallo-enzymes and their biosynthesis, Hydrogen/
\end{flushleft}


\begin{flushleft}
Deuterium exchange reaction and its application in monitoring
\end{flushleft}


\begin{flushleft}
biological processes, basic concept of chemical synthesis of life.
\end{flushleft}





\begin{flushleft}
SBL711 Cell Signalling
\end{flushleft}


\begin{flushleft}
3 Credits (3-0-0)
\end{flushleft}


\begin{flushleft}
Pre-requisites: SBL100 and SBL201 (or equivalent) and EC90
\end{flushleft}


\begin{flushleft}
Signaling as a basis of cellular communications, conversion of
\end{flushleft}


\begin{flushleft}
information into cellular response, first messenger, intracellular
\end{flushleft}


\begin{flushleft}
and extracellular receptors, second messenger, signaling proteins,
\end{flushleft}


\begin{flushleft}
signal amplification, cascade formation, adaptors, domains, scaffold,
\end{flushleft}


\begin{flushleft}
recruitment of signaling proteins, pseudosubstrates, convergence,
\end{flushleft}


\begin{flushleft}
divergence, cross talk, molecular switches, critical nodes, multisite
\end{flushleft}


\begin{flushleft}
protein phosphorylation, G-protein coupled signal transduction, nuclear
\end{flushleft}


\begin{flushleft}
receptors, growth factors and tyrosine kinases, mitogen activated
\end{flushleft}


\begin{flushleft}
protein kinases, insulin signal transduction, phosphatases, emerging
\end{flushleft}


\begin{flushleft}
technologies like antisense, omics, RNAi, high content screening, target
\end{flushleft}


\begin{flushleft}
hopping, combination of mutations, systems approach to understand
\end{flushleft}


\begin{flushleft}
signaling complexity.
\end{flushleft}





\begin{flushleft}
SBL712 Dynamics of Infection Biology
\end{flushleft}


\begin{flushleft}
3 Credits (3-0-0)
\end{flushleft}


\begin{flushleft}
Pre-requisites: SBL100 and either SBL201 or BEL204 or BEL311
\end{flushleft}


\begin{flushleft}
and EC90
\end{flushleft}


\begin{flushleft}
Features of bacterial/viral/other pathogens, molecular evolution and
\end{flushleft}


\begin{flushleft}
dissemination, factors influencing dissemination, host entry, receptors
\end{flushleft}


\begin{flushleft}
and pathways, host genetics, persistence and latency, co-infection
\end{flushleft}


\begin{flushleft}
dynamics, host-pathogen interactions, innate and adaptive immunity,
\end{flushleft}


\begin{flushleft}
Th1-Th2 balance, intracellular survival and dissemination, molecular
\end{flushleft}


\begin{flushleft}
mimicry, apoptosis and necrosis, intervention strategies and application
\end{flushleft}


\begin{flushleft}
of bioinformatics in infection biology.
\end{flushleft}





\begin{flushleft}
SBL713 Introduction to Structural Biology
\end{flushleft}


\begin{flushleft}
3 Credits (3-0-0)
\end{flushleft}


\begin{flushleft}
Pre-requisites: SBL100 and either BEL204 or BEL311 and EC90
\end{flushleft}


\begin{flushleft}
Introduction to protein structure; secondary, tertiary and quaternary
\end{flushleft}


\begin{flushleft}
structures; expression and purification of recombinant proteins for
\end{flushleft}


\begin{flushleft}
structure determination; basics of X-ray crystallography, space groups,
\end{flushleft}


\begin{flushleft}
diffraction basics, phasing techniques, validation and Ramachandran
\end{flushleft}


\begin{flushleft}
plot; cryoelectron microscopy, freezing and imaging techniques,
\end{flushleft}


\begin{flushleft}
model building; small angle X-ray scattering (SAXS), application to
\end{flushleft}


\begin{flushleft}
protein samples; NMR, chemical shifts, common NMR experiments,
\end{flushleft}


\begin{flushleft}
assignment, validation; advantages and disadvantages of each
\end{flushleft}


\begin{flushleft}
technique, types of applications.
\end{flushleft}





\begin{flushleft}
SBL714 Plant Biotechnology and Human Health
\end{flushleft}


\begin{flushleft}
3 Credits (3-0-0)
\end{flushleft}


\begin{flushleft}
Pre-requisites: EC90
\end{flushleft}


\begin{flushleft}
Overview of medicinal plants and their geographical distribution,
\end{flushleft}


\begin{flushleft}
economics of medicinal plants, KNapSACK family database,metabolic
\end{flushleft}


\begin{flushleft}
diversity, genomic and transcriptomic profiling, phenomics,
\end{flushleft}


\begin{flushleft}
antivenoms, plant toxins, bioactive peptides, genetic engineering
\end{flushleft}


\begin{flushleft}
and molecular biology technologies such as DNA barcoding, DNA chip
\end{flushleft}


\begin{flushleft}
technology, cDNA, AFLP, microarray, siRNA, antisense, bioanalytics,
\end{flushleft}


\begin{flushleft}
plant models systems, Nutragenomics, smart and functional foods,
\end{flushleft}


\begin{flushleft}
Plants based human diseases communicable and noncommunicable
\end{flushleft}


\begin{flushleft}
diseases, synthetic biology approaches
\end{flushleft}





\begin{flushleft}
SBL750 Quantitative Biology
\end{flushleft}


\begin{flushleft}
3 Credits (3-0-0)
\end{flushleft}


\begin{flushleft}
Pre-requisites: SBL100 and SBL201 (or equivalent) and EC90
\end{flushleft}


\begin{flushleft}
Overview of quantitative biology; Biomolecules - a study of how
\end{flushleft}


\begin{flushleft}
information is code in molecules - DNA, RNA and proteins, information
\end{flushleft}


\begin{flushleft}
representation; Molecular sequences - the alignment problem,
\end{flushleft}


\begin{flushleft}
PAM and BLOSUM matrices, applications - global, local and overlap
\end{flushleft}


\begin{flushleft}
alignment; Gene prediction - computational gene finding, ab-initio
\end{flushleft}


\begin{flushleft}
methods, comparative methods; Molecular evolution - molecular
\end{flushleft}





323





\begin{flushleft}
\newpage
Biological Sciences
\end{flushleft}





\begin{flushleft}
clock, explicit models and evolutionary rate estimation; Population
\end{flushleft}


\begin{flushleft}
genetics - polymorphism, genetic diversity and Neutral theory; Testing
\end{flushleft}


\begin{flushleft}
evolutionary hypothesis; Genetic circuits - motifs search, satiotemporal logic, methods of analyses; Protein structure prediction,
\end{flushleft}


\begin{flushleft}
protein-protein interaction networks, drug target identification,
\end{flushleft}


\begin{flushleft}
Biological network dynamics; Biological pattern formation; Self
\end{flushleft}


\begin{flushleft}
organization in biology.
\end{flushleft}





\begin{flushleft}
SBV750 Bioinspiration and Biomimetics
\end{flushleft}


\begin{flushleft}
1 Credit (1-0-0)
\end{flushleft}


\begin{flushleft}
Pre-requisites: EC 90
\end{flushleft}


\begin{flushleft}
Introduction to Bioinspiration and biomimetics, Bioinspiration pools
\end{flushleft}


\begin{flushleft}
marine and terresterial plants and animals, Biomimetic/Bioenabled
\end{flushleft}


\begin{flushleft}
materials, biomineralisation, Biomimetic ahesives and attachment
\end{flushleft}


\begin{flushleft}
devices in nature, prosthetics function and design, bioinspired robotics,
\end{flushleft}


\begin{flushleft}
biomimetic pattern formation, colour and camougflage, photocells,
\end{flushleft}


\begin{flushleft}
role in agriculture and human health, future prospects in the industry.
\end{flushleft}





\begin{flushleft}
SBL751 Chemical and Molecular Foundations of Cell
\end{flushleft}


\begin{flushleft}
3 Credits (3-0-0)
\end{flushleft}


\begin{flushleft}
Pre-requisites: SBL100 and SBL201 (or equivalent) and EC90
\end{flushleft}


\begin{flushleft}
Protein conformation, dynamics and function, Enzyme activity,
\end{flushleft}


\begin{flushleft}
Biomolecular interactions in cell, biomolecular assemblies in the
\end{flushleft}


\begin{flushleft}
cell, Generation and storage of metabolic energy, Biosynthesis of
\end{flushleft}


\begin{flushleft}
macromolecular precursors like, amino acids, lipids, hormones,
\end{flushleft}


\begin{flushleft}
nucleotides, Characterisation and identification of cells, Genes,
\end{flushleft}


\begin{flushleft}
genomics and chromosomes, Genetic material, DNA replication,
\end{flushleft}


\begin{flushleft}
Repair, Translation, Mutagenesis, mutations and mutants, Plasmid and
\end{flushleft}


\begin{flushleft}
transposable element, Recombinant DNA and genetic engineering,
\end{flushleft}


\begin{flushleft}
Protein targeting into membranes and organelles, Vesicular traffic,
\end{flushleft}


\begin{flushleft}
secretion, and endocytosis, Cellular organization of movement,
\end{flushleft}


\begin{flushleft}
microtubules, Eukaryotic cell cycle, functions and mode of action of
\end{flushleft}


\begin{flushleft}
nucleus, Nerve cells, Immune response, Evolution of cells, prebiotic
\end{flushleft}


\begin{flushleft}
synthesis, RNA catalysis, evolution of gene structure, Epigenetics,
\end{flushleft}


\begin{flushleft}
Non-coding RNA, Hologenome.
\end{flushleft}





\begin{flushleft}
SBC795 Graduate Student Research Seminar-I
\end{flushleft}


\begin{flushleft}
0.5 Credit (0-0-1)
\end{flushleft}


\begin{flushleft}
Pre-requisites: EC 90
\end{flushleft}


\begin{flushleft}
The course is aimed at giving the student a forum to periodically
\end{flushleft}


\begin{flushleft}
present their research, to critique the research of colleagues and learn
\end{flushleft}


\begin{flushleft}
about the best research in their fields. Discussions will be held on
\end{flushleft}


\begin{flushleft}
scientific methodology and inculcated with a value system for pursuing
\end{flushleft}


\begin{flushleft}
a career in science. Activities will be carried out in workshop mode.
\end{flushleft}





\begin{flushleft}
SBC796 Graduate Student Research Seminar-II
\end{flushleft}


\begin{flushleft}
0.5 Credits (0-0-1)
\end{flushleft}


\begin{flushleft}
Pre-requisites: EC 90
\end{flushleft}


\begin{flushleft}
Special topics in research will be assigned by Coordinator; results
\end{flushleft}


\begin{flushleft}
of the research of each student registered for the course will be
\end{flushleft}


\begin{flushleft}
discussed; Discussions on scientific material from recently published
\end{flushleft}


\begin{flushleft}
papers in areas related to their research; The {``}Laboratory'' activities
\end{flushleft}


\begin{flushleft}
will include delivery of seminars on their research and participation
\end{flushleft}


\begin{flushleft}
in the seminars and critique.
\end{flushleft}





\begin{flushleft}
SBS800 Independent Study
\end{flushleft}


\begin{flushleft}
3 Credits (0-3-0)
\end{flushleft}


\begin{flushleft}
Pre-requisites: EC 120
\end{flushleft}


\begin{flushleft}
The course is aimed at providing the student an opportunity to pursue
\end{flushleft}


\begin{flushleft}
a special research topic. A research topic assigned and mutually
\end{flushleft}


\begin{flushleft}
agreed upon by the faculty and student. Registration will require the
\end{flushleft}


\begin{flushleft}
submission of a proposal through the research committee on the topic
\end{flushleft}


\begin{flushleft}
clearly delineating the objectives to be achieved.
\end{flushleft}





\begin{flushleft}
SBL801 Signal Transduction and Drug Target
\end{flushleft}


\begin{flushleft}
Identification
\end{flushleft}


\begin{flushleft}
3 Credits (3-0-0)
\end{flushleft}


\begin{flushleft}
Pre-requisites: EC120
\end{flushleft}


\begin{flushleft}
Eukaryotic cellular communications, importance of signal transduction,
\end{flushleft}





\begin{flushleft}
principles of signaling, recurring themes of signal transduction,
\end{flushleft}


\begin{flushleft}
reception, transduction, response, signal amplification, coordination of
\end{flushleft}


\begin{flushleft}
signaling, cascade formation, structure to function, anchors, adaptors,
\end{flushleft}


\begin{flushleft}
scaffold, recruitment of signaling proteins, topology and functional
\end{flushleft}


\begin{flushleft}
domains, dual specificity, modules, convergence, divergence, cross
\end{flushleft}


\begin{flushleft}
talk, receptors, G-protein coupled signal transduction, growth
\end{flushleft}


\begin{flushleft}
factors and tyrosine kinases, mitogen activated protein kinases,
\end{flushleft}


\begin{flushleft}
insulin signal transduction, critical nodes, protein phosphorylation,
\end{flushleft}


\begin{flushleft}
drug target identification, mechanism of drug action against signal
\end{flushleft}


\begin{flushleft}
transduction, antagonists of cell surface receptors and nuclear and
\end{flushleft}


\begin{flushleft}
receptors, ion channel blockers, transport inhibitors, targeting protein
\end{flushleft}


\begin{flushleft}
kinases and phosphatases, inhibitors of kinases and phosphatases,
\end{flushleft}


\begin{flushleft}
pseudosubstrates, examples of clinical drugs against protein kinases/
\end{flushleft}


\begin{flushleft}
phosphatases, new and emerging technologies to identify drug target
\end{flushleft}


\begin{flushleft}
like antisense, omics, RNAi, high content screening, target hopping,
\end{flushleft}


\begin{flushleft}
combination of mutations, systems approach, complexity in signaling,
\end{flushleft}


\begin{flushleft}
techniques in signal transduction.
\end{flushleft}





\begin{flushleft}
SBL802 Macromolecular Structure and Data Processing
\end{flushleft}


\begin{flushleft}
3 Credits (3-0-0)
\end{flushleft}


\begin{flushleft}
Pre-requisites: EC120
\end{flushleft}


\begin{flushleft}
Treatment of macromolecules to generate suitable crystals, hanging
\end{flushleft}


\begin{flushleft}
drop and sitting drop techniques, seeding, cryopotecting and freezing
\end{flushleft}


\begin{flushleft}
crystals, acquisition of diffraction data, synchrotron radiation,
\end{flushleft}


\begin{flushleft}
indexing and scaling data, space group identification, symmetry
\end{flushleft}


\begin{flushleft}
elements, Fourier transformation and structure factors, the phase
\end{flushleft}


\begin{flushleft}
problem, heavy atom methods, molecular replacement, anomalous
\end{flushleft}


\begin{flushleft}
X-ray scattering, calculation of electron density, model building and
\end{flushleft}


\begin{flushleft}
phase refinement, co-crystallography, small angle X-ray scattering,
\end{flushleft}


\begin{flushleft}
preparing samples for transmission electron microscopy, negative
\end{flushleft}


\begin{flushleft}
staining, cryo-techniques for freezing grids, manual vs. automated
\end{flushleft}


\begin{flushleft}
data collection, cryotomography, software packages for data collection
\end{flushleft}


\begin{flushleft}
and processing, generating a model, refinement and validation, time
\end{flushleft}


\begin{flushleft}
resolved cryoEM.
\end{flushleft}





\begin{flushleft}
SBP810 Advanced Bioscience Techniques
\end{flushleft}


\begin{flushleft}
2 Credits (0-0-4)
\end{flushleft}


\begin{flushleft}
Pre-requisites: EC90
\end{flushleft}


\begin{flushleft}
Particle sizing, biological and biomolecular visualization tools, advanced
\end{flushleft}


\begin{flushleft}
and analytical spectrometry, cell and molecular separation techniques,
\end{flushleft}


\begin{flushleft}
DNA and protein interaction techniques, membrane interaction
\end{flushleft}


\begin{flushleft}
and signalling, bioreactors, tissue culture, transgene technology,
\end{flushleft}


\begin{flushleft}
electrophysiology methods.
\end{flushleft}





\begin{flushleft}
SBV881 Advances in Chemical Biology
\end{flushleft}


\begin{flushleft}
1 Credit (1-0-0)
\end{flushleft}


\begin{flushleft}
Pre-requisites: EC 120
\end{flushleft}


\begin{flushleft}
Structural aspects of proteins and nucleic acids, Mechanism of action
\end{flushleft}


\begin{flushleft}
of biological molecules, Chemical approaches to solve biological
\end{flushleft}


\begin{flushleft}
problems, Designing chemical tools for addressing problems in biology,
\end{flushleft}


\begin{flushleft}
Bioconjugate chemistry, Recent developments in these areas.
\end{flushleft}





\begin{flushleft}
SBV882 Biological Membranes
\end{flushleft}


\begin{flushleft}
1 Credit (1-0-0)
\end{flushleft}


\begin{flushleft}
Pre-requisites: EC 120
\end{flushleft}


\begin{flushleft}
Introduction to the hydrophobic effect, Phospholipid model systems,
\end{flushleft}


\begin{flushleft}
Cellular membrane asymmetry, Membrane dynamics, Membrane
\end{flushleft}


\begin{flushleft}
trafficking, Membrane fusion, Membrane proteins (Form and function),
\end{flushleft}


\begin{flushleft}
Small molecule permeability, Pores channels and transporters, Lipid
\end{flushleft}


\begin{flushleft}
systems for drug delivery.
\end{flushleft}





\begin{flushleft}
SBV883 Chaperone and Protein Conformational
\end{flushleft}


\begin{flushleft}
Disorders
\end{flushleft}


\begin{flushleft}
1 Credit (1-0-0)
\end{flushleft}


\begin{flushleft}
Pre-requisites: EC 120
\end{flushleft}


\begin{flushleft}
Molecular mechanism of protein misfolding, fate of aggregated proteins
\end{flushleft}


\begin{flushleft}
in the cell, various protein misfolding disorders in humans, mechanism
\end{flushleft}


\begin{flushleft}
of action of molecular chaperones in various cells, chaperone assisted
\end{flushleft}


\begin{flushleft}
suppression of protein misfolding.
\end{flushleft}





324





\begin{flushleft}
\newpage
Biological Sciences
\end{flushleft}





\begin{flushleft}
SBV884 Elements of Neuroscience
\end{flushleft}


\begin{flushleft}
1 Credit (1-0-0)
\end{flushleft}


\begin{flushleft}
Pre-requisites: EC 120
\end{flushleft}





\begin{flushleft}
SBV889 Diagnostic Virology
\end{flushleft}


\begin{flushleft}
1 Credit (1-0-0)
\end{flushleft}


\begin{flushleft}
Pre-requisites: EC 120
\end{flushleft}





\begin{flushleft}
Introduction to cell biology of neurons; presynaptic and post synaptic
\end{flushleft}


\begin{flushleft}
mechanisms; signal transduction cascades; neural integration;
\end{flushleft}


\begin{flushleft}
Hodgkin-Huxley experiments; Na and K pumps; physiological
\end{flushleft}


\begin{flushleft}
significance of pump modulation; Na and K channels; type and
\end{flushleft}


\begin{flushleft}
function of different Ca activated K channels; structure function and
\end{flushleft}


\begin{flushleft}
inactivation; tools for studying Ca signalling; caging and releasing Ca
\end{flushleft}


\begin{flushleft}
in the neurons; role of nitric oxide; Long term potentiation.
\end{flushleft}





\begin{flushleft}
Introduction to diagnostic virology -- direct and indirect methods,
\end{flushleft}


\begin{flushleft}
specimens and window period; Microscopy -- light microscopy, electron
\end{flushleft}


\begin{flushleft}
microscopy, and fluorescence microscopy in virus identifications;
\end{flushleft}


\begin{flushleft}
Methods of virus isolation -- cell culture, embryonated egg inoculation
\end{flushleft}


\begin{flushleft}
and animal inoculation; Viral antigen detection -- methods, assay
\end{flushleft}


\begin{flushleft}
characteristics, rapid antigen identification techniques; Detection of
\end{flushleft}


\begin{flushleft}
viral antibodies -- methods, role of quantitative measurements, classspecific immunoglobulin detection; Viral nucleic acids -- amplification,
\end{flushleft}


\begin{flushleft}
detection and quantitation methods; Molecular epidemiology of
\end{flushleft}


\begin{flushleft}
viral infections -- high throughput methods; Identifying antiviral
\end{flushleft}


\begin{flushleft}
resistance -- genotypic and phenotypic approaches; Quality control
\end{flushleft}


\begin{flushleft}
in diagnostic virology -- internal and external quality control,
\end{flushleft}


\begin{flushleft}
international standards, and Shewhart control charts.
\end{flushleft}





\begin{flushleft}
SBV885 Protein Aggregations and Diseases
\end{flushleft}


\begin{flushleft}
1 Credit (1-0-0)
\end{flushleft}


\begin{flushleft}
Pre-requisites: EC 120
\end{flushleft}


\begin{flushleft}
Introduction to protein aggregation (amorphous and amyloid), types
\end{flushleft}


\begin{flushleft}
of aggregates, difference between aggregation and precipitation;
\end{flushleft}


\begin{flushleft}
External and internal factors for protein aggregation, pH, temperature
\end{flushleft}


\begin{flushleft}
and protein concentration effects; hydrophobicity, discordant helices;
\end{flushleft}


\begin{flushleft}
Structural and conformational prerequisites of amyloidogenesis,
\end{flushleft}


\begin{flushleft}
predominance of beta-sheet, alpha-helices or random coils of native
\end{flushleft}


\begin{flushleft}
protein; generic nature of protein folding and misfolding, Cytotoxic
\end{flushleft}


\begin{flushleft}
intermediates in the fibrillation pathway, Oxidative stress and protein
\end{flushleft}


\begin{flushleft}
deposition disease, Protein aggregation, ion channel formation, and
\end{flushleft}


\begin{flushleft}
membrane damage, Recent trends in prevention of amyloidosis; drugs,
\end{flushleft}


\begin{flushleft}
antibodies, combination therapy.
\end{flushleft}





\begin{flushleft}
SBV886 Signaling Pathway Analysis
\end{flushleft}


\begin{flushleft}
1 Credit (1-0-0)
\end{flushleft}


\begin{flushleft}
Pre-requisites: EC 120
\end{flushleft}


\begin{flushleft}
Introduction to modelling of biological systems -- history, types of
\end{flushleft}


\begin{flushleft}
models, macroscopic phenomena, modelling of cellular systems;
\end{flushleft}


\begin{flushleft}
hierarchy in information transmission and utilization, interaction
\end{flushleft}


\begin{flushleft}
between different levels of information leading complex behaviour;
\end{flushleft}


\begin{flushleft}
robustness of cellular systems and its significance; molecules that
\end{flushleft}


\begin{flushleft}
transmit signals, role of signaling in regulation of cellular functions,
\end{flushleft}


\begin{flushleft}
gene regulation; signal transduction -- evolution and history; first
\end{flushleft}


\begin{flushleft}
messengers and receptors, GTP-binding proteins; Calcium Signaling
\end{flushleft}


\begin{flushleft}
-- free, bound and trapped calcium, mechanisms regulating calcium
\end{flushleft}


\begin{flushleft}
concentration, calcium changes in single cells; protein phosphorylation
\end{flushleft}


\begin{flushleft}
as a switch, protein kinase A, protein kinase C, structure of signaling
\end{flushleft}


\begin{flushleft}
pathways, extracting motifs from pathways, relating motifs to
\end{flushleft}


\begin{flushleft}
observations; dynamics and periodicity in signaling pathways.
\end{flushleft}





\begin{flushleft}
SBV887 Current Topics in Computational Biology
\end{flushleft}


\begin{flushleft}
1 Credit (1-0-0)
\end{flushleft}


\begin{flushleft}
Pre-requisites: EC 120
\end{flushleft}


\begin{flushleft}
Bring about awareness of the challenges in Genomics, Proteomics,
\end{flushleft}


\begin{flushleft}
Metabolomics and Structural Biology.
\end{flushleft}





\begin{flushleft}
SBV888 Current Trends in Computer Aided Drug
\end{flushleft}


\begin{flushleft}
Discovery
\end{flushleft}


\begin{flushleft}
1 Credit (1-0-0)
\end{flushleft}


\begin{flushleft}
Pre-requisites: EC 120
\end{flushleft}


\begin{flushleft}
Teach students various methods for target identification, and
\end{flushleft}


\begin{flushleft}
applications QSAR and molecular modelling in drug discovery.
\end{flushleft}





\begin{flushleft}
SBV890 Kinetoplastid Parasites and Novel Targets
\end{flushleft}


\begin{flushleft}
1 Credit(1-0-0)
\end{flushleft}


\begin{flushleft}
Pre-requisites: EC 120
\end{flushleft}


\begin{flushleft}
Kinetoplastid diseases, transmission, clinical features, immune
\end{flushleft}


\begin{flushleft}
evasion, treatment, antimicrobial chemotherapy, drug resistance,
\end{flushleft}


\begin{flushleft}
cross -- resistance, Leishmania, promastigotes and amastigotes,
\end{flushleft}


\begin{flushleft}
procyclic and metacyclic, macrophage, interaction with sand
\end{flushleft}


\begin{flushleft}
fly, cytokine response, transmission, syndromes associated with
\end{flushleft}


\begin{flushleft}
leishmaniasis, microtubules in kinetoplastida, dynamics and
\end{flushleft}


\begin{flushleft}
posttranslational modifications, drug interactions, resistance
\end{flushleft}


\begin{flushleft}
against tubulin binding agents, arsenite resistance in Leishmania,
\end{flushleft}


\begin{flushleft}
transporters in kinetoplastid protozoa and drug targets, leishmanial
\end{flushleft}


\begin{flushleft}
glucose transporters, function of histone deacytylases in
\end{flushleft}


\begin{flushleft}
kinetoplastid protozoa, DNA -- topoisomerases in Leishmania, a
\end{flushleft}


\begin{flushleft}
possible therapeutic target, exoproteome of leishmania, importane
\end{flushleft}


\begin{flushleft}
and its application in Leishmania.
\end{flushleft}





\begin{flushleft}
SBV891 -- Virus Host Interactions
\end{flushleft}


\begin{flushleft}
1 Credit (1-0-0)
\end{flushleft}


\begin{flushleft}
Pre-requisites: EC 120
\end{flushleft}


\begin{flushleft}
Introduction to the virus life cycle; host cell surface molecules
\end{flushleft}


\begin{flushleft}
utilized as virus receptors, mechanism of cellular membrane
\end{flushleft}


\begin{flushleft}
penetration for enveloped and non-enveloped animal viruses, cellular
\end{flushleft}


\begin{flushleft}
entry of bacteriophages and plant viruses; icosahedral and helical
\end{flushleft}


\begin{flushleft}
capsids, disassembly and transport of genome to the replication
\end{flushleft}


\begin{flushleft}
site, process of replication, modification of cellular organelles and
\end{flushleft}


\begin{flushleft}
hijacking of host cell resources ; site and manner of progeny virus
\end{flushleft}


\begin{flushleft}
assembly ; lytic and lysogenic viruses; virus egress and involvement
\end{flushleft}


\begin{flushleft}
of the host secretory pathway; host defence mechanisms, virus
\end{flushleft}


\begin{flushleft}
strategies to evade host immune system, antiviral therapies and
\end{flushleft}


\begin{flushleft}
drug discovery.
\end{flushleft}





\begin{flushleft}
SBD895 MS Research Project
\end{flushleft}


\begin{flushleft}
36 Credits (0-0-72)
\end{flushleft}


\begin{flushleft}
The research problem will be assigned by the supervisor. It is expected
\end{flushleft}


\begin{flushleft}
that the student will undertake the problem early in the program.
\end{flushleft}





325





\begin{flushleft}
\newpage
Interdisciplinary M.Tech. Programmes
\end{flushleft}


\begin{flushleft}
M.Tech. Programme in Optoelectronics and Optical
\end{flushleft}


\begin{flushleft}
Communications
\end{flushleft}





\begin{flushleft}
JOS795 Independent Study
\end{flushleft}


\begin{flushleft}
3 Credits (0-3-0)
\end{flushleft}





\begin{flushleft}
JOP791 Laboratory-I (Fiber Optics and Opt. Comm. Lab)
\end{flushleft}


\begin{flushleft}
3 Credits (0-0-6)
\end{flushleft}





\begin{flushleft}
JOV796 Selected Topics in Photonics
\end{flushleft}


\begin{flushleft}
1 Credit (1-0-0)
\end{flushleft}





\begin{flushleft}
JOP792 Laboratory-II (Fiber Optics and Opt. Comm.
\end{flushleft}


\begin{flushleft}
Lab)
\end{flushleft}


\begin{flushleft}
3 Credits (0-0-6)
\end{flushleft}





\begin{flushleft}
JOD801 Major Project Part-I
\end{flushleft}


\begin{flushleft}
6 Credits (0-0-12)
\end{flushleft}





\begin{flushleft}
JOL793 Selected Topics-I
\end{flushleft}


\begin{flushleft}
3 Credits (3-0-0)
\end{flushleft}





\begin{flushleft}
JOD802 Major Project Part-II
\end{flushleft}


\begin{flushleft}
12 Credits (0-0-24)
\end{flushleft}





\begin{flushleft}
JOL794 Selected Topics-II
\end{flushleft}


\begin{flushleft}
3 Credits (3-0-0)
\end{flushleft}





326





\begin{flushleft}
\newpage
Abbreviations
\end{flushleft}


\begin{flushleft}
BAP	
\end{flushleft}





\begin{flushleft}
Board of Academic Programmes
\end{flushleft}





\begin{flushleft}
B.Tech.	
\end{flushleft}





\begin{flushleft}
Bachelor of Technology
\end{flushleft}





\begin{flushleft}
CGPA	
\end{flushleft}





\begin{flushleft}
Cumulative Grade Point Average
\end{flushleft}





\begin{flushleft}
CRC	
\end{flushleft}





\begin{flushleft}
Centre Research Committee
\end{flushleft}





\begin{flushleft}
DGPA	
\end{flushleft}





\begin{flushleft}
Degree Grade Point Average
\end{flushleft}





\begin{flushleft}
D.I.I.T.	
\end{flushleft}





\begin{flushleft}
Diploma of I.I.T. Delhi
\end{flushleft}





\begin{flushleft}
DRC	
\end{flushleft}





\begin{flushleft}
Department Research Committee
\end{flushleft}





\begin{flushleft}
EC	
\end{flushleft}





\begin{flushleft}
Earned Credits
\end{flushleft}





\begin{flushleft}
IRD	
\end{flushleft}





\begin{flushleft}
Industrial Research and Development
\end{flushleft}





\begin{flushleft}
M.B.A.	
\end{flushleft}





\begin{flushleft}
Master of Business Administration
\end{flushleft}





\begin{flushleft}
M.Des.	
\end{flushleft}





\begin{flushleft}
Master of Design
\end{flushleft}





\begin{flushleft}
M.Sc.	
\end{flushleft}





\begin{flushleft}
Master of Science
\end{flushleft}





\begin{flushleft}
M.S.(R)	
\end{flushleft}





\begin{flushleft}
Master of Science (Research)
\end{flushleft}





\begin{flushleft}
M.Tech.	
\end{flushleft}





\begin{flushleft}
Master of Technology
\end{flushleft}





\begin{flushleft}
PGS\&R	
\end{flushleft}





\begin{flushleft}
Postgraduate Studies and Research
\end{flushleft}





\begin{flushleft}
Ph.D.	
\end{flushleft}





\begin{flushleft}
Doctor of Philosophy
\end{flushleft}





\begin{flushleft}
SGPA	
\end{flushleft}





\begin{flushleft}
Semester Grade Point Average
\end{flushleft}





\begin{flushleft}
SRC	
\end{flushleft}





\begin{flushleft}
Student Research Committee (for M.S.(R) and Ph.D. student)
\end{flushleft}





\begin{flushleft}
SRC	
\end{flushleft}





\begin{flushleft}
Student Research Committee (in respect to School)
\end{flushleft}





\begin{flushleft}
UGS	
\end{flushleft}





\begin{flushleft}
Undergraduate Studies
\end{flushleft}





327





\newpage
328





\begin{flushleft}
C
\end{flushleft}





\begin{flushleft}
C
\end{flushleft}





\begin{flushleft}
C
\end{flushleft}





\begin{flushleft}
A
\end{flushleft}





\begin{flushleft}
A
\end{flushleft}





8-850





\begin{flushleft}
D
\end{flushleft}





\begin{flushleft}
D
\end{flushleft}





\begin{flushleft}
D
\end{flushleft}





9-950





\begin{flushleft}
B
\end{flushleft}





\begin{flushleft}
B
\end{flushleft}





\begin{flushleft}
E
\end{flushleft}





\begin{flushleft}
E
\end{flushleft}





\begin{flushleft}
E
\end{flushleft}





10-1050





\begin{flushleft}
F
\end{flushleft}





\begin{flushleft}
F
\end{flushleft}





\begin{flushleft}
H
\end{flushleft}





\begin{flushleft}
F
\end{flushleft}





\begin{flushleft}
H
\end{flushleft}





11-1150





\begin{flushleft}
J
\end{flushleft}





\begin{flushleft}
H
\end{flushleft}





\begin{flushleft}
K
\end{flushleft}





\begin{flushleft}
J
\end{flushleft}





\begin{flushleft}
J
\end{flushleft}





12-1250





\begin{flushleft}
TF3
\end{flushleft}





3





\begin{flushleft}
TF4
\end{flushleft}





4





\begin{flushleft}
TA3
\end{flushleft}


\begin{flushleft}
TE4
\end{flushleft}





3


4





2





\begin{flushleft}
TF1
\end{flushleft}





\begin{flushleft}
TE3
\end{flushleft}





3


1





\begin{flushleft}
TA2
\end{flushleft}





2





1





4





3





2





1


\begin{flushleft}
L / TG1
\end{flushleft}





\begin{flushleft}
TE2
\end{flushleft}





2


4





\begin{flushleft}
TA1
\end{flushleft}





1





4





\begin{flushleft}
TA4
\end{flushleft}





\begin{flushleft}
TF2
\end{flushleft}





2


3





\begin{flushleft}
TE1
\end{flushleft}





1-150





1





\begin{flushleft}
Cycle No.
\end{flushleft}





\begin{flushleft}
PB2
\end{flushleft}





\begin{flushleft}
PB1
\end{flushleft}





\begin{flushleft}
PB4
\end{flushleft}





\begin{flushleft}
PB3
\end{flushleft}





\begin{flushleft}
TD3
\end{flushleft}





\begin{flushleft}
TD2
\end{flushleft}





\begin{flushleft}
TG2
\end{flushleft}





\begin{flushleft}
TD1
\end{flushleft}





\begin{flushleft}
TD4
\end{flushleft}





2-250





\begin{flushleft}
PC4
\end{flushleft}





\begin{flushleft}
PD1
\end{flushleft}





\begin{flushleft}
PD4
\end{flushleft}





\begin{flushleft}
PC3
\end{flushleft}





4-450





\begin{flushleft}
K
\end{flushleft}





\begin{flushleft}
M
\end{flushleft}





5-550





\begin{flushleft}
L
\end{flushleft}





6-650





\begin{flushleft}
PE4
\end{flushleft}





\begin{flushleft}
PF1
\end{flushleft}





\begin{flushleft}
PF4
\end{flushleft}





\begin{flushleft}
PE3
\end{flushleft}





\begin{flushleft}
PD3
\end{flushleft}





\begin{flushleft}
PC2
\end{flushleft}





\begin{flushleft}
PD2
\end{flushleft}





\begin{flushleft}
PC1
\end{flushleft}





\begin{flushleft}
K
\end{flushleft}





\begin{flushleft}
M
\end{flushleft}





\begin{flushleft}
L
\end{flushleft}





\begin{flushleft}
Seminar / Meeting / Project Activity
\end{flushleft}





\begin{flushleft}
PF3
\end{flushleft}





\begin{flushleft}
PE2
\end{flushleft}





\begin{flushleft}
PF2
\end{flushleft}





\begin{flushleft}
PE1
\end{flushleft}





3-350





\begin{flushleft}
1.	 Five-cycle lab / tutorial schedule would have lab / tutorial classes also on wednesday between 1-5 p.m.
\end{flushleft}


\begin{flushleft}
2.	 TG1 and TG2 are slots for courses that would like all groups to have tutorials together or in parallel at the same time.
\end{flushleft}





\begin{flushleft}
Note:
\end{flushleft}





\begin{flushleft}
Friday
\end{flushleft}





\begin{flushleft}
Thursday
\end{flushleft}





\begin{flushleft}
Wednesday
\end{flushleft}





\begin{flushleft}
Tuesday
\end{flushleft}





\begin{flushleft}
Monday
\end{flushleft}





\begin{flushleft}
Day
\end{flushleft}





\begin{flushleft}
Slot Timings (General -- 4 Cycles)
\end{flushleft}





\newpage



\end{document}
